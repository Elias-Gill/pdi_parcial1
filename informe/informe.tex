\documentclass[sigconf]{acmart}

% Metadata del documento
\title{Título del Artículo}
\subtitle{Opcional: Subtítulo del Artículo}
\author{Tu Nombre}
\affiliation{
    \institution{Nombre de tu Institución}
    \city{Ciudad}
    \country{País}
}
\email{tuemail@dominio.com}

% Para múltiples autores:
% \author{Otro Autor}
% \affiliation{
%     \institution{Otra Institución}
%     \city{Otra Ciudad}
%     \country{Otro País}
% }
% \email{otroemail@dominio.com}

\begin{abstract}
Aquí puedes escribir el resumen de tu artículo. Describe brevemente los objetivos, métodos y conclusiones principales.
\end{abstract}

\begin{document}

% Genera el título automáticamente
\maketitle

\section{Introducción}
Aquí empieza la introducción de tu artículo. Puedes agregar subsecciones o más secciones según sea necesario.
Aquí citamos un artículo~
\cite{ejemplo}.

\section{Trabajo Relacionado}
Incluye aquí los trabajos previos o literatura relevante.

\section{Conclusión}
Finaliza tu documento con conclusiones y trabajos futuros.

% Bibliografía
\bibliographystyle{ACM-Reference-Format}
\bibliography{bibliography}

\end{document}
