\documentclass[sigconf]{acmart}

% Para poner letras de colores
\usepackage{xcolor}
\newcommand{\rojo}[1]{\textcolor{red}{#1}}

% Metadata del documento
\title{Título del Artículo}
\subtitle{Opcional: Subtítulo del Artículo}

% Autores
\author{Elias Sebastian Gill Quintana}
\affiliation{
    \institution{Facultad Politecnica - UNA}
    \city{San Lorenzo}
    \country{Paraguay}
}
\author{Abigail Mercedes Nuñes Mendez}
\affiliation{
    \institution{Facultad Politecnica - UNA}
    \city{San Lorenzo}
    \country{Paraguay}
}
\author{Maria Jose Mendoza Recalde}
\affiliation{
    \institution{Facultad Politecnica - UNA}
    \city{San Lorenzo}
    \country{Paraguay}
}

\begin{abstract}
Falta completar \rojo{Completar}
\end{abstract}

\keywords{
    Falta completar \rojo{Completar}
}

\begin{document}

% Generar Titulo
\maketitle

\section{Introducción}
Aquí empieza la introducción de tu artículo. Puedes agregar subsecciones o más secciones según sea necesario.
Aquí citamos un artículo~
\cite{ejemplo}.

\section{Trabajo Relacionado}
Incluye aquí los trabajos previos o literatura relevante.

\section{Conclusión}
Finaliza tu documento con conclusiones y trabajos futuros.

% Bibliografía
\bibliographystyle{ACM-Reference-Format}
\bibliography{bibliography}

\end{document}
