\documentclass[sigchi]{acmart}

% para histogramas
\usepackage{pgf}                  % Para cargar archivos .pgf
\usepackage{pgfplots}             % Para gráficos avanzados
\pgfplotsset{compat=1.18}         % Versión estable
\usepackage{caption}
\captionsetup[figure]{font=small} % small, footnotesize, o scriptsize
\usepackage{graphicx}
\usepackage{float}
\usepackage{subcaption}

% Tablas
\usepackage{array}    % Para ajustar columnas
\setlength{\intextsep}{1.5em}      % Elimina espacio arriba/abajo de floats
\setlength{\textfloatsep}{1.5em}   % Elimina espacio entre floats y texto
\captionsetup[table]{skip=2pt}   % Espacio mínimo entre tabla y caption

% removes footnote with conference information in first column
\renewcommand\footnotetextcopyrightpermission[1]{}
\pagestyle{plain} % removes running headers
\setcopyright{none}

% Elimina la línea "ACM Reference Format" y el titulo de conferencia
\settopmatter{printacmref=false, printfolios=false}
\acmConference[Facultad Politécnica]{2025}{San Lorenzo, Paraguay}

% Para poner letras de colores
\usepackage{xcolor}
\newcommand{\rojo}[1]{\textcolor{red}{#1}}

% Metadata del documento
\title{Comparación de Algoritmos de Mejora de Contraste de Imágenes}
\subtitle{CLAHE, HE, DQHEPL y BHEPL-D}

% Autores
\author{Elias Sebastian Gill Quintana}
\affiliation{
    \institution{Facultad Politecnica - UNA}
    \city{San Lorenzo}
    \country{Paraguay}
}
\author{Abigail Mercedes Nuñes Mendez}
\affiliation{
    \institution{Facultad Politecnica - UNA}
    \city{San Lorenzo}
    \country{Paraguay}
}
\author{Maria Jose Mendoza Recalde}
\affiliation{
    \institution{Facultad Politecnica - UNA}
    \city{San Lorenzo}
    \country{Paraguay}
}

\begin{abstract}
Falta completar \rojo{Completar}
\end{abstract}

\keywords{
    Falta completar \rojo{Completar}
}

\begin{document}

% Generar Titulo
\maketitle

\section{Introducción}
\subsection{Contexto y Motivación}

El procesamiento digital de imágenes juega un papel crucial en diversos campos tecnológicos, donde la mejora de contraste permite:

\begin{itemize}
	\item Visualizar detalles críticos en imágenes médicas
	\item Automatizar inspecciones industriales
	\item Mejorar imágenes en condiciones de baja iluminación
\end{itemize}

Este trabajo evalúa métodos clásicos de mejora de contraste, proporcionando métricas
comparativas para aplicaciones que requieren bajo costo computacional y resultados
interpretables.

\subsection{Objetivo}
\label{subsec:objetivo}

Este informe tiene como objetivo principal evaluar cuantitativamente el desempeño de cuatro algoritmos de mejora de contraste en imágenes en escala de grises:

\begin{itemize}
	\item \textbf{CLAHE} (Contrast Limited Adaptive Histogram Equalization)
	\item \textbf{HE} (Histogram Equalization tradicional)
	\item \textbf{DQHEPL} (Dualistic Quantum-inspired Histogram Equalization with Plateau Limit)
	\item \textbf{BHEPL-D} (Brightness Preserving Histogram Equalization with Plateau Limit - Dynamic version)
\end{itemize}

La evaluación se realizará mediante métricas objetivas que consideran:

\begin{itemize}
	\item Preservación del brillo medio (AMBE)
	\item Relación señal-ruido pico (PSNR)
	\item Conservación de texturas (Entropía)
	\item Nivel de realce visual (Contraste)
	\item Distribución de intensidades (Uniformidad)
\end{itemize}

El análisis comparativo permitirá identificar:
\begin{enumerate}
	\item El algoritmo más efectivo para imágenes con bajo contraste inicial
	\item Compromisos entre realce visual y preservación de brillo
	\item Sensibilidad a diferentes distribuciones histogramáticas
\end{enumerate}

Los resultados proporcionarán criterios técnicos para seleccionar el método óptimo según
aplicaciones específicas en procesamiento digital de imágenes.

\subsection{Estructura del Informe}
\rojo{continuar}
- Resumen de las secciones del informe.

\section{Metodología}
\subsection{Descripción de los Algoritmos}
\rojo{continuar}
\subsubsection{CLAHE (Contrast Limited Adaptive Histogram Equalization)}
- Explicación del algoritmo.
- Detalles sobre los parámetros utilizados (clipLimit, tileGridSize).

\subsubsection{HE (Histogram Equalization)}
- Explicación del algoritmo.
- Comparativa con CLAHE.

\subsubsection{DQHEPL (Dynamic Quantization-based Histogram Equalization with Pulse-Level Enhancement)}
- Explicación del algoritmo.
- Características particulares.

\subsubsection{BHEPL-D (Bilateral Histogram Equalization with Pulse-Level Enhancement and Denoising)}
- Explicación del algoritmo.
- Detalles sobre cómo mejora la técnica HE tradicional.

\subsection{Preparación de las Imágenes}
\rojo{continuar}
- Detalles sobre el conjunto de datos utilizado.
- Preprocesamiento de las imágenes (si aplica).

\subsection{Métricas de Evaluación}

\subsubsection{AMBE (Average Mean Brightness Error)}
La métrica AMBE mide la diferencia absoluta entre el brillo promedio de la imagen original y el de la imagen procesada. Su fórmula es:

\begin{equation}
\text{AMBE} = \left| \mu_X - \mu_Y \right|
\end{equation}

donde:
\begin{itemize}
    \item $\mu_X$: Brillo promedio de la imagen original.
    \item $\mu_Y$: Brillo promedio de la imagen procesada.
\end{itemize}

Un valor de AMBE cercano a cero indica una buena preservación del brillo global de la imagen original.

\subsubsection{PSNR (Peak Signal-to-Noise Ratio)}
El PSNR mide la calidad de reconstrucción entre dos imágenes. Se basa en el error cuadrático medio (MSE) entre la imagen original y la imagen procesada:

\begin{equation}
\text{MSE} = \frac{1}{MN} \sum_{i=1}^{M} \sum_{j=1}^{N} \left[ X(i,j) - Y(i,j) \right]^2
\end{equation}

\begin{equation}
\text{PSNR} = 10 \cdot \log_{10} \left( \frac{L^2}{\text{MSE}} \right)
\end{equation}

donde:
\begin{itemize}
    \item $X(i,j)$: Intensidad del píxel en la posición $(i,j)$ de la imagen original.
    \item $Y(i,j)$: Intensidad del píxel correspondiente en la imagen procesada.
    \item $M, N$: Dimensiones de la imagen.
    \item $L$: Valor máximo posible de intensidad (por ejemplo, 255 para imágenes de 8 bits).
\end{itemize}

Un PSNR mayor indica mejor calidad de la imagen procesada.

\subsubsection{Entropía}
La entropía mide la cantidad de información o imprevisibilidad en la distribución de niveles de gris de una imagen:

\begin{equation}
\text{Entropía} = - \sum_{k=0}^{L-1} p_k \cdot \log_2(p_k)
\end{equation}

donde:
\begin{itemize}
    \item $p_k$: Probabilidad de ocurrencia del nivel de gris $k$.
    \item $L$: Número total de niveles de gris (por ejemplo, 256).
\end{itemize}

Una mayor entropía implica una imagen más rica en detalles e información visual.

\subsubsection{Contraste}
Una forma común de medir el contraste es utilizando la desviación estándar de las intensidades:

\begin{equation}
\text{Contraste} = \sigma = \sqrt{\frac{1}{MN} \sum_{i=1}^{M} \sum_{j=1}^{N} \left[ I(i,j) - \mu \right]^2}
\end{equation}

donde:
\begin{itemize}
    \item $I(i,j)$: Intensidad del píxel en la posición $(i,j)$.
    \item $\mu$: Promedio de intensidades en la imagen.
\end{itemize}

El contraste también puede definirse mediante medidas basadas en diferencias entre píxeles vecinos o métricas estructurales más avanzadas, pero la desviación estándar es una de las más utilizadas por su simplicidad y efectividad.


\section{Resultados}
\subsection{Evaluación Visual}
- Comparación visual de las imágenes originales y las mejoradas con cada algoritmo.
- Uso de imágenes de ejemplo para ilustrar las diferencias.

\newpage
\subsection{Evaluación de Histogramas}

\subsubsection{Análisis imagen 2015\_00002}

Lorem ipsum dolor sit amet, officia excepteur ex fugiat reprehenderit enim labore culpa sint ad
nisi Lorem pariatur mollit ex esse exercitation amet. Nisi anim cupidatat excepteur officia.
Reprehenderit nostrud nostrud ipsum Lorem est aliquip amet voluptate voluptate dolor minim
nulla est proident. Nostrud officia pariatur ut officia. Sit irure elit esse ea nulla sunt ex
occaecat reprehenderit commodo officia dolor Lorem duis laboris cupidatat officia voluptate.
Culpa proident adipisicing id nulla nisi laboris ex in Lorem sunt duis officia eiusmod. Aliqua
reprehenderit commodo ex non excepteur duis sunt velit enim. Voluptate laboris sint cupidatat
ullamco ut ea consectetur et est culpa et culpa duis.

\begin{table}[H]
	\centering
	\caption{Métricas de calidad para la imagen 2015\_00002.png}
	\label{tab:metricas_2015_00002}

	% Ajuste de espacios verticales
	\vspace{0.5em} % Reduce espacio superior

	\resizebox{\linewidth}{!}{%
		\begin{tabular}{@{}lrrrrr@{}} % @{} elimina espacios horizontales
			\toprule
			\textbf{Método} & \textbf{AMBE} & \textbf{PSNR} & \textbf{Entropía} & \textbf{Contraste} & \textbf{Uniformidad} \\
			\midrule
			CLAHE           & 41.03         & 14.27         & 7.91              & 64.99              & 0.629                \\
			HE              & 58.78         & 11.45         & 7.15              & 73.61              & 0.635                \\
			DQHEPL          & 6.20          & 24.14         & 7.25              & 54.04              & 0.583                \\
			BHEPL-D         & 8.38          & 23.82         & 7.26              & 55.01              & 0.586                \\
			\bottomrule
		\end{tabular}%
	}

	\vspace{0.5em} % Reduce espacio inferior
	\footnotesize%
	\textit{Nota: AMBE (Absolute Mean Brightness Error), PSNR (Peak Signal-to-Noise Ratio).}
\end{table}

Lorem ipsum dolor sit amet, officia excepteur ex fugiat reprehenderit enim labore culpa sint ad
nisi Lorem pariatur mollit ex esse exercitation amet. Nisi anim cupidatat excepteur officia.
Reprehenderit nostrud nostrud ipsum Lorem est aliquip amet voluptate voluptate dolor minim
nulla est proident. Nostrud officia pariatur ut officia. Sit irure elit esse ea nulla sunt ex
occaecat reprehenderit commodo officia dolor Lorem duis laboris cupidatat officia voluptate.
Culpa proident adipisicing id nulla nisi laboris ex in Lorem sunt duis officia eiusmod. Aliqua
reprehenderit commodo ex non excepteur duis sunt velit enim. Voluptate laboris sint cupidatat
ullamco ut ea consectetur et est culpa et culpa duis.

\begin{figure}[H]
	\centering

	% Primera fila (2 histogramas)
	\begin{subfigure}[b]{0.45\textwidth}
		\resizebox{\linewidth}{!}{%% Creator: Matplotlib, PGF backend
%%
%% To include the figure in your LaTeX document, write
%%   \input{<filename>.pgf}
%%
%% Make sure the required packages are loaded in your preamble
%%   \usepackage{pgf}
%%
%% Also ensure that all the required font packages are loaded; for instance,
%% the lmodern package is sometimes necessary when using math font.
%%   \usepackage{lmodern}
%%
%% Figures using additional raster images can only be included by \input if
%% they are in the same directory as the main LaTeX file. For loading figures
%% from other directories you can use the `import` package
%%   \usepackage{import}
%%
%% and then include the figures with
%%   \import{<path to file>}{<filename>.pgf}
%%
%% Matplotlib used the following preamble
%%   \def\mathdefault#1{#1}
%%   \everymath=\expandafter{\the\everymath\displaystyle}
%%   \IfFileExists{scrextend.sty}{
%%     \usepackage[fontsize=10.000000pt]{scrextend}
%%   }{
%%     \renewcommand{\normalsize}{\fontsize{10.000000}{12.000000}\selectfont}
%%     \normalsize
%%   }
%%   
%%   \ifdefined\pdftexversion\else  % non-pdftex case.
%%     \usepackage{fontspec}
%%     \setmainfont{DejaVuSerif.ttf}[Path=\detokenize{/home/elias/.virtualenvs/src-pway/lib/python3.13/site-packages/matplotlib/mpl-data/fonts/ttf/}]
%%     \setsansfont{DejaVuSans.ttf}[Path=\detokenize{/home/elias/.virtualenvs/src-pway/lib/python3.13/site-packages/matplotlib/mpl-data/fonts/ttf/}]
%%     \setmonofont{DejaVuSansMono.ttf}[Path=\detokenize{/home/elias/.virtualenvs/src-pway/lib/python3.13/site-packages/matplotlib/mpl-data/fonts/ttf/}]
%%   \fi
%%   \makeatletter\@ifpackageloaded{underscore}{}{\usepackage[strings]{underscore}}\makeatother
%%
\begingroup%
\makeatletter%
\begin{pgfpicture}%
\pgfpathrectangle{\pgfpointorigin}{\pgfqpoint{10.000000in}{5.000000in}}%
\pgfusepath{use as bounding box, clip}%
\begin{pgfscope}%
\pgfsetbuttcap%
\pgfsetmiterjoin%
\definecolor{currentfill}{rgb}{1.000000,1.000000,1.000000}%
\pgfsetfillcolor{currentfill}%
\pgfsetlinewidth{0.000000pt}%
\definecolor{currentstroke}{rgb}{1.000000,1.000000,1.000000}%
\pgfsetstrokecolor{currentstroke}%
\pgfsetdash{}{0pt}%
\pgfpathmoveto{\pgfqpoint{0.000000in}{0.000000in}}%
\pgfpathlineto{\pgfqpoint{10.000000in}{0.000000in}}%
\pgfpathlineto{\pgfqpoint{10.000000in}{5.000000in}}%
\pgfpathlineto{\pgfqpoint{0.000000in}{5.000000in}}%
\pgfpathlineto{\pgfqpoint{0.000000in}{0.000000in}}%
\pgfpathclose%
\pgfusepath{fill}%
\end{pgfscope}%
\begin{pgfscope}%
\pgfsetbuttcap%
\pgfsetmiterjoin%
\definecolor{currentfill}{rgb}{1.000000,1.000000,1.000000}%
\pgfsetfillcolor{currentfill}%
\pgfsetlinewidth{0.000000pt}%
\definecolor{currentstroke}{rgb}{0.000000,0.000000,0.000000}%
\pgfsetstrokecolor{currentstroke}%
\pgfsetstrokeopacity{0.000000}%
\pgfsetdash{}{0pt}%
\pgfpathmoveto{\pgfqpoint{1.250000in}{0.550000in}}%
\pgfpathlineto{\pgfqpoint{9.000000in}{0.550000in}}%
\pgfpathlineto{\pgfqpoint{9.000000in}{4.400000in}}%
\pgfpathlineto{\pgfqpoint{1.250000in}{4.400000in}}%
\pgfpathlineto{\pgfqpoint{1.250000in}{0.550000in}}%
\pgfpathclose%
\pgfusepath{fill}%
\end{pgfscope}%
\begin{pgfscope}%
\pgfpathrectangle{\pgfqpoint{1.250000in}{0.550000in}}{\pgfqpoint{7.750000in}{3.850000in}}%
\pgfusepath{clip}%
\pgfsetbuttcap%
\pgfsetmiterjoin%
\definecolor{currentfill}{rgb}{0.501961,0.501961,0.501961}%
\pgfsetfillcolor{currentfill}%
\pgfsetfillopacity{0.700000}%
\pgfsetlinewidth{0.000000pt}%
\definecolor{currentstroke}{rgb}{0.000000,0.000000,0.000000}%
\pgfsetstrokecolor{currentstroke}%
\pgfsetstrokeopacity{0.700000}%
\pgfsetdash{}{0pt}%
\pgfpathmoveto{\pgfqpoint{1.602273in}{0.550000in}}%
\pgfpathlineto{\pgfqpoint{1.629794in}{0.550000in}}%
\pgfpathlineto{\pgfqpoint{1.629794in}{2.398200in}}%
\pgfpathlineto{\pgfqpoint{1.602273in}{2.398200in}}%
\pgfpathlineto{\pgfqpoint{1.602273in}{0.550000in}}%
\pgfpathclose%
\pgfusepath{fill}%
\end{pgfscope}%
\begin{pgfscope}%
\pgfpathrectangle{\pgfqpoint{1.250000in}{0.550000in}}{\pgfqpoint{7.750000in}{3.850000in}}%
\pgfusepath{clip}%
\pgfsetbuttcap%
\pgfsetmiterjoin%
\definecolor{currentfill}{rgb}{0.501961,0.501961,0.501961}%
\pgfsetfillcolor{currentfill}%
\pgfsetfillopacity{0.700000}%
\pgfsetlinewidth{0.000000pt}%
\definecolor{currentstroke}{rgb}{0.000000,0.000000,0.000000}%
\pgfsetstrokecolor{currentstroke}%
\pgfsetstrokeopacity{0.700000}%
\pgfsetdash{}{0pt}%
\pgfpathmoveto{\pgfqpoint{1.629794in}{0.550000in}}%
\pgfpathlineto{\pgfqpoint{1.657315in}{0.550000in}}%
\pgfpathlineto{\pgfqpoint{1.657315in}{4.216667in}}%
\pgfpathlineto{\pgfqpoint{1.629794in}{4.216667in}}%
\pgfpathlineto{\pgfqpoint{1.629794in}{0.550000in}}%
\pgfpathclose%
\pgfusepath{fill}%
\end{pgfscope}%
\begin{pgfscope}%
\pgfpathrectangle{\pgfqpoint{1.250000in}{0.550000in}}{\pgfqpoint{7.750000in}{3.850000in}}%
\pgfusepath{clip}%
\pgfsetbuttcap%
\pgfsetmiterjoin%
\definecolor{currentfill}{rgb}{0.501961,0.501961,0.501961}%
\pgfsetfillcolor{currentfill}%
\pgfsetfillopacity{0.700000}%
\pgfsetlinewidth{0.000000pt}%
\definecolor{currentstroke}{rgb}{0.000000,0.000000,0.000000}%
\pgfsetstrokecolor{currentstroke}%
\pgfsetstrokeopacity{0.700000}%
\pgfsetdash{}{0pt}%
\pgfpathmoveto{\pgfqpoint{1.657315in}{0.550000in}}%
\pgfpathlineto{\pgfqpoint{1.684837in}{0.550000in}}%
\pgfpathlineto{\pgfqpoint{1.684837in}{3.586192in}}%
\pgfpathlineto{\pgfqpoint{1.657315in}{3.586192in}}%
\pgfpathlineto{\pgfqpoint{1.657315in}{0.550000in}}%
\pgfpathclose%
\pgfusepath{fill}%
\end{pgfscope}%
\begin{pgfscope}%
\pgfpathrectangle{\pgfqpoint{1.250000in}{0.550000in}}{\pgfqpoint{7.750000in}{3.850000in}}%
\pgfusepath{clip}%
\pgfsetbuttcap%
\pgfsetmiterjoin%
\definecolor{currentfill}{rgb}{0.501961,0.501961,0.501961}%
\pgfsetfillcolor{currentfill}%
\pgfsetfillopacity{0.700000}%
\pgfsetlinewidth{0.000000pt}%
\definecolor{currentstroke}{rgb}{0.000000,0.000000,0.000000}%
\pgfsetstrokecolor{currentstroke}%
\pgfsetstrokeopacity{0.700000}%
\pgfsetdash{}{0pt}%
\pgfpathmoveto{\pgfqpoint{1.684837in}{0.550000in}}%
\pgfpathlineto{\pgfqpoint{1.712358in}{0.550000in}}%
\pgfpathlineto{\pgfqpoint{1.712358in}{2.872296in}}%
\pgfpathlineto{\pgfqpoint{1.684837in}{2.872296in}}%
\pgfpathlineto{\pgfqpoint{1.684837in}{0.550000in}}%
\pgfpathclose%
\pgfusepath{fill}%
\end{pgfscope}%
\begin{pgfscope}%
\pgfpathrectangle{\pgfqpoint{1.250000in}{0.550000in}}{\pgfqpoint{7.750000in}{3.850000in}}%
\pgfusepath{clip}%
\pgfsetbuttcap%
\pgfsetmiterjoin%
\definecolor{currentfill}{rgb}{0.501961,0.501961,0.501961}%
\pgfsetfillcolor{currentfill}%
\pgfsetfillopacity{0.700000}%
\pgfsetlinewidth{0.000000pt}%
\definecolor{currentstroke}{rgb}{0.000000,0.000000,0.000000}%
\pgfsetstrokecolor{currentstroke}%
\pgfsetstrokeopacity{0.700000}%
\pgfsetdash{}{0pt}%
\pgfpathmoveto{\pgfqpoint{1.712358in}{0.550000in}}%
\pgfpathlineto{\pgfqpoint{1.739879in}{0.550000in}}%
\pgfpathlineto{\pgfqpoint{1.739879in}{2.257238in}}%
\pgfpathlineto{\pgfqpoint{1.712358in}{2.257238in}}%
\pgfpathlineto{\pgfqpoint{1.712358in}{0.550000in}}%
\pgfpathclose%
\pgfusepath{fill}%
\end{pgfscope}%
\begin{pgfscope}%
\pgfpathrectangle{\pgfqpoint{1.250000in}{0.550000in}}{\pgfqpoint{7.750000in}{3.850000in}}%
\pgfusepath{clip}%
\pgfsetbuttcap%
\pgfsetmiterjoin%
\definecolor{currentfill}{rgb}{0.501961,0.501961,0.501961}%
\pgfsetfillcolor{currentfill}%
\pgfsetfillopacity{0.700000}%
\pgfsetlinewidth{0.000000pt}%
\definecolor{currentstroke}{rgb}{0.000000,0.000000,0.000000}%
\pgfsetstrokecolor{currentstroke}%
\pgfsetstrokeopacity{0.700000}%
\pgfsetdash{}{0pt}%
\pgfpathmoveto{\pgfqpoint{1.739879in}{0.550000in}}%
\pgfpathlineto{\pgfqpoint{1.767401in}{0.550000in}}%
\pgfpathlineto{\pgfqpoint{1.767401in}{1.879229in}}%
\pgfpathlineto{\pgfqpoint{1.739879in}{1.879229in}}%
\pgfpathlineto{\pgfqpoint{1.739879in}{0.550000in}}%
\pgfpathclose%
\pgfusepath{fill}%
\end{pgfscope}%
\begin{pgfscope}%
\pgfpathrectangle{\pgfqpoint{1.250000in}{0.550000in}}{\pgfqpoint{7.750000in}{3.850000in}}%
\pgfusepath{clip}%
\pgfsetbuttcap%
\pgfsetmiterjoin%
\definecolor{currentfill}{rgb}{0.501961,0.501961,0.501961}%
\pgfsetfillcolor{currentfill}%
\pgfsetfillopacity{0.700000}%
\pgfsetlinewidth{0.000000pt}%
\definecolor{currentstroke}{rgb}{0.000000,0.000000,0.000000}%
\pgfsetstrokecolor{currentstroke}%
\pgfsetstrokeopacity{0.700000}%
\pgfsetdash{}{0pt}%
\pgfpathmoveto{\pgfqpoint{1.767401in}{0.550000in}}%
\pgfpathlineto{\pgfqpoint{1.794922in}{0.550000in}}%
\pgfpathlineto{\pgfqpoint{1.794922in}{1.607766in}}%
\pgfpathlineto{\pgfqpoint{1.767401in}{1.607766in}}%
\pgfpathlineto{\pgfqpoint{1.767401in}{0.550000in}}%
\pgfpathclose%
\pgfusepath{fill}%
\end{pgfscope}%
\begin{pgfscope}%
\pgfpathrectangle{\pgfqpoint{1.250000in}{0.550000in}}{\pgfqpoint{7.750000in}{3.850000in}}%
\pgfusepath{clip}%
\pgfsetbuttcap%
\pgfsetmiterjoin%
\definecolor{currentfill}{rgb}{0.501961,0.501961,0.501961}%
\pgfsetfillcolor{currentfill}%
\pgfsetfillopacity{0.700000}%
\pgfsetlinewidth{0.000000pt}%
\definecolor{currentstroke}{rgb}{0.000000,0.000000,0.000000}%
\pgfsetstrokecolor{currentstroke}%
\pgfsetstrokeopacity{0.700000}%
\pgfsetdash{}{0pt}%
\pgfpathmoveto{\pgfqpoint{1.794922in}{0.550000in}}%
\pgfpathlineto{\pgfqpoint{1.822443in}{0.550000in}}%
\pgfpathlineto{\pgfqpoint{1.822443in}{1.512232in}}%
\pgfpathlineto{\pgfqpoint{1.794922in}{1.512232in}}%
\pgfpathlineto{\pgfqpoint{1.794922in}{0.550000in}}%
\pgfpathclose%
\pgfusepath{fill}%
\end{pgfscope}%
\begin{pgfscope}%
\pgfpathrectangle{\pgfqpoint{1.250000in}{0.550000in}}{\pgfqpoint{7.750000in}{3.850000in}}%
\pgfusepath{clip}%
\pgfsetbuttcap%
\pgfsetmiterjoin%
\definecolor{currentfill}{rgb}{0.501961,0.501961,0.501961}%
\pgfsetfillcolor{currentfill}%
\pgfsetfillopacity{0.700000}%
\pgfsetlinewidth{0.000000pt}%
\definecolor{currentstroke}{rgb}{0.000000,0.000000,0.000000}%
\pgfsetstrokecolor{currentstroke}%
\pgfsetstrokeopacity{0.700000}%
\pgfsetdash{}{0pt}%
\pgfpathmoveto{\pgfqpoint{1.822443in}{0.550000in}}%
\pgfpathlineto{\pgfqpoint{1.849964in}{0.550000in}}%
\pgfpathlineto{\pgfqpoint{1.849964in}{1.347317in}}%
\pgfpathlineto{\pgfqpoint{1.822443in}{1.347317in}}%
\pgfpathlineto{\pgfqpoint{1.822443in}{0.550000in}}%
\pgfpathclose%
\pgfusepath{fill}%
\end{pgfscope}%
\begin{pgfscope}%
\pgfpathrectangle{\pgfqpoint{1.250000in}{0.550000in}}{\pgfqpoint{7.750000in}{3.850000in}}%
\pgfusepath{clip}%
\pgfsetbuttcap%
\pgfsetmiterjoin%
\definecolor{currentfill}{rgb}{0.501961,0.501961,0.501961}%
\pgfsetfillcolor{currentfill}%
\pgfsetfillopacity{0.700000}%
\pgfsetlinewidth{0.000000pt}%
\definecolor{currentstroke}{rgb}{0.000000,0.000000,0.000000}%
\pgfsetstrokecolor{currentstroke}%
\pgfsetstrokeopacity{0.700000}%
\pgfsetdash{}{0pt}%
\pgfpathmoveto{\pgfqpoint{1.849964in}{0.550000in}}%
\pgfpathlineto{\pgfqpoint{1.877486in}{0.550000in}}%
\pgfpathlineto{\pgfqpoint{1.877486in}{1.249855in}}%
\pgfpathlineto{\pgfqpoint{1.849964in}{1.249855in}}%
\pgfpathlineto{\pgfqpoint{1.849964in}{0.550000in}}%
\pgfpathclose%
\pgfusepath{fill}%
\end{pgfscope}%
\begin{pgfscope}%
\pgfpathrectangle{\pgfqpoint{1.250000in}{0.550000in}}{\pgfqpoint{7.750000in}{3.850000in}}%
\pgfusepath{clip}%
\pgfsetbuttcap%
\pgfsetmiterjoin%
\definecolor{currentfill}{rgb}{0.501961,0.501961,0.501961}%
\pgfsetfillcolor{currentfill}%
\pgfsetfillopacity{0.700000}%
\pgfsetlinewidth{0.000000pt}%
\definecolor{currentstroke}{rgb}{0.000000,0.000000,0.000000}%
\pgfsetstrokecolor{currentstroke}%
\pgfsetstrokeopacity{0.700000}%
\pgfsetdash{}{0pt}%
\pgfpathmoveto{\pgfqpoint{1.877486in}{0.550000in}}%
\pgfpathlineto{\pgfqpoint{1.905007in}{0.550000in}}%
\pgfpathlineto{\pgfqpoint{1.905007in}{1.151291in}}%
\pgfpathlineto{\pgfqpoint{1.877486in}{1.151291in}}%
\pgfpathlineto{\pgfqpoint{1.877486in}{0.550000in}}%
\pgfpathclose%
\pgfusepath{fill}%
\end{pgfscope}%
\begin{pgfscope}%
\pgfpathrectangle{\pgfqpoint{1.250000in}{0.550000in}}{\pgfqpoint{7.750000in}{3.850000in}}%
\pgfusepath{clip}%
\pgfsetbuttcap%
\pgfsetmiterjoin%
\definecolor{currentfill}{rgb}{0.501961,0.501961,0.501961}%
\pgfsetfillcolor{currentfill}%
\pgfsetfillopacity{0.700000}%
\pgfsetlinewidth{0.000000pt}%
\definecolor{currentstroke}{rgb}{0.000000,0.000000,0.000000}%
\pgfsetstrokecolor{currentstroke}%
\pgfsetstrokeopacity{0.700000}%
\pgfsetdash{}{0pt}%
\pgfpathmoveto{\pgfqpoint{1.905007in}{0.550000in}}%
\pgfpathlineto{\pgfqpoint{1.932528in}{0.550000in}}%
\pgfpathlineto{\pgfqpoint{1.932528in}{1.082187in}}%
\pgfpathlineto{\pgfqpoint{1.905007in}{1.082187in}}%
\pgfpathlineto{\pgfqpoint{1.905007in}{0.550000in}}%
\pgfpathclose%
\pgfusepath{fill}%
\end{pgfscope}%
\begin{pgfscope}%
\pgfpathrectangle{\pgfqpoint{1.250000in}{0.550000in}}{\pgfqpoint{7.750000in}{3.850000in}}%
\pgfusepath{clip}%
\pgfsetbuttcap%
\pgfsetmiterjoin%
\definecolor{currentfill}{rgb}{0.501961,0.501961,0.501961}%
\pgfsetfillcolor{currentfill}%
\pgfsetfillopacity{0.700000}%
\pgfsetlinewidth{0.000000pt}%
\definecolor{currentstroke}{rgb}{0.000000,0.000000,0.000000}%
\pgfsetstrokecolor{currentstroke}%
\pgfsetstrokeopacity{0.700000}%
\pgfsetdash{}{0pt}%
\pgfpathmoveto{\pgfqpoint{1.932528in}{0.550000in}}%
\pgfpathlineto{\pgfqpoint{1.960050in}{0.550000in}}%
\pgfpathlineto{\pgfqpoint{1.960050in}{1.053554in}}%
\pgfpathlineto{\pgfqpoint{1.932528in}{1.053554in}}%
\pgfpathlineto{\pgfqpoint{1.932528in}{0.550000in}}%
\pgfpathclose%
\pgfusepath{fill}%
\end{pgfscope}%
\begin{pgfscope}%
\pgfpathrectangle{\pgfqpoint{1.250000in}{0.550000in}}{\pgfqpoint{7.750000in}{3.850000in}}%
\pgfusepath{clip}%
\pgfsetbuttcap%
\pgfsetmiterjoin%
\definecolor{currentfill}{rgb}{0.501961,0.501961,0.501961}%
\pgfsetfillcolor{currentfill}%
\pgfsetfillopacity{0.700000}%
\pgfsetlinewidth{0.000000pt}%
\definecolor{currentstroke}{rgb}{0.000000,0.000000,0.000000}%
\pgfsetstrokecolor{currentstroke}%
\pgfsetstrokeopacity{0.700000}%
\pgfsetdash{}{0pt}%
\pgfpathmoveto{\pgfqpoint{1.960050in}{0.550000in}}%
\pgfpathlineto{\pgfqpoint{1.987571in}{0.550000in}}%
\pgfpathlineto{\pgfqpoint{1.987571in}{1.011431in}}%
\pgfpathlineto{\pgfqpoint{1.960050in}{1.011431in}}%
\pgfpathlineto{\pgfqpoint{1.960050in}{0.550000in}}%
\pgfpathclose%
\pgfusepath{fill}%
\end{pgfscope}%
\begin{pgfscope}%
\pgfpathrectangle{\pgfqpoint{1.250000in}{0.550000in}}{\pgfqpoint{7.750000in}{3.850000in}}%
\pgfusepath{clip}%
\pgfsetbuttcap%
\pgfsetmiterjoin%
\definecolor{currentfill}{rgb}{0.501961,0.501961,0.501961}%
\pgfsetfillcolor{currentfill}%
\pgfsetfillopacity{0.700000}%
\pgfsetlinewidth{0.000000pt}%
\definecolor{currentstroke}{rgb}{0.000000,0.000000,0.000000}%
\pgfsetstrokecolor{currentstroke}%
\pgfsetstrokeopacity{0.700000}%
\pgfsetdash{}{0pt}%
\pgfpathmoveto{\pgfqpoint{1.987571in}{0.550000in}}%
\pgfpathlineto{\pgfqpoint{2.015092in}{0.550000in}}%
\pgfpathlineto{\pgfqpoint{2.015092in}{0.968757in}}%
\pgfpathlineto{\pgfqpoint{1.987571in}{0.968757in}}%
\pgfpathlineto{\pgfqpoint{1.987571in}{0.550000in}}%
\pgfpathclose%
\pgfusepath{fill}%
\end{pgfscope}%
\begin{pgfscope}%
\pgfpathrectangle{\pgfqpoint{1.250000in}{0.550000in}}{\pgfqpoint{7.750000in}{3.850000in}}%
\pgfusepath{clip}%
\pgfsetbuttcap%
\pgfsetmiterjoin%
\definecolor{currentfill}{rgb}{0.501961,0.501961,0.501961}%
\pgfsetfillcolor{currentfill}%
\pgfsetfillopacity{0.700000}%
\pgfsetlinewidth{0.000000pt}%
\definecolor{currentstroke}{rgb}{0.000000,0.000000,0.000000}%
\pgfsetstrokecolor{currentstroke}%
\pgfsetstrokeopacity{0.700000}%
\pgfsetdash{}{0pt}%
\pgfpathmoveto{\pgfqpoint{2.015092in}{0.550000in}}%
\pgfpathlineto{\pgfqpoint{2.042614in}{0.550000in}}%
\pgfpathlineto{\pgfqpoint{2.042614in}{0.912592in}}%
\pgfpathlineto{\pgfqpoint{2.015092in}{0.912592in}}%
\pgfpathlineto{\pgfqpoint{2.015092in}{0.550000in}}%
\pgfpathclose%
\pgfusepath{fill}%
\end{pgfscope}%
\begin{pgfscope}%
\pgfpathrectangle{\pgfqpoint{1.250000in}{0.550000in}}{\pgfqpoint{7.750000in}{3.850000in}}%
\pgfusepath{clip}%
\pgfsetbuttcap%
\pgfsetmiterjoin%
\definecolor{currentfill}{rgb}{0.501961,0.501961,0.501961}%
\pgfsetfillcolor{currentfill}%
\pgfsetfillopacity{0.700000}%
\pgfsetlinewidth{0.000000pt}%
\definecolor{currentstroke}{rgb}{0.000000,0.000000,0.000000}%
\pgfsetstrokecolor{currentstroke}%
\pgfsetstrokeopacity{0.700000}%
\pgfsetdash{}{0pt}%
\pgfpathmoveto{\pgfqpoint{2.042614in}{0.550000in}}%
\pgfpathlineto{\pgfqpoint{2.070135in}{0.550000in}}%
\pgfpathlineto{\pgfqpoint{2.070135in}{0.906260in}}%
\pgfpathlineto{\pgfqpoint{2.042614in}{0.906260in}}%
\pgfpathlineto{\pgfqpoint{2.042614in}{0.550000in}}%
\pgfpathclose%
\pgfusepath{fill}%
\end{pgfscope}%
\begin{pgfscope}%
\pgfpathrectangle{\pgfqpoint{1.250000in}{0.550000in}}{\pgfqpoint{7.750000in}{3.850000in}}%
\pgfusepath{clip}%
\pgfsetbuttcap%
\pgfsetmiterjoin%
\definecolor{currentfill}{rgb}{0.501961,0.501961,0.501961}%
\pgfsetfillcolor{currentfill}%
\pgfsetfillopacity{0.700000}%
\pgfsetlinewidth{0.000000pt}%
\definecolor{currentstroke}{rgb}{0.000000,0.000000,0.000000}%
\pgfsetstrokecolor{currentstroke}%
\pgfsetstrokeopacity{0.700000}%
\pgfsetdash{}{0pt}%
\pgfpathmoveto{\pgfqpoint{2.070135in}{0.550000in}}%
\pgfpathlineto{\pgfqpoint{2.097656in}{0.550000in}}%
\pgfpathlineto{\pgfqpoint{2.097656in}{0.871019in}}%
\pgfpathlineto{\pgfqpoint{2.070135in}{0.871019in}}%
\pgfpathlineto{\pgfqpoint{2.070135in}{0.550000in}}%
\pgfpathclose%
\pgfusepath{fill}%
\end{pgfscope}%
\begin{pgfscope}%
\pgfpathrectangle{\pgfqpoint{1.250000in}{0.550000in}}{\pgfqpoint{7.750000in}{3.850000in}}%
\pgfusepath{clip}%
\pgfsetbuttcap%
\pgfsetmiterjoin%
\definecolor{currentfill}{rgb}{0.501961,0.501961,0.501961}%
\pgfsetfillcolor{currentfill}%
\pgfsetfillopacity{0.700000}%
\pgfsetlinewidth{0.000000pt}%
\definecolor{currentstroke}{rgb}{0.000000,0.000000,0.000000}%
\pgfsetstrokecolor{currentstroke}%
\pgfsetstrokeopacity{0.700000}%
\pgfsetdash{}{0pt}%
\pgfpathmoveto{\pgfqpoint{2.097656in}{0.550000in}}%
\pgfpathlineto{\pgfqpoint{2.125178in}{0.550000in}}%
\pgfpathlineto{\pgfqpoint{2.125178in}{0.875975in}}%
\pgfpathlineto{\pgfqpoint{2.097656in}{0.875975in}}%
\pgfpathlineto{\pgfqpoint{2.097656in}{0.550000in}}%
\pgfpathclose%
\pgfusepath{fill}%
\end{pgfscope}%
\begin{pgfscope}%
\pgfpathrectangle{\pgfqpoint{1.250000in}{0.550000in}}{\pgfqpoint{7.750000in}{3.850000in}}%
\pgfusepath{clip}%
\pgfsetbuttcap%
\pgfsetmiterjoin%
\definecolor{currentfill}{rgb}{0.501961,0.501961,0.501961}%
\pgfsetfillcolor{currentfill}%
\pgfsetfillopacity{0.700000}%
\pgfsetlinewidth{0.000000pt}%
\definecolor{currentstroke}{rgb}{0.000000,0.000000,0.000000}%
\pgfsetstrokecolor{currentstroke}%
\pgfsetstrokeopacity{0.700000}%
\pgfsetdash{}{0pt}%
\pgfpathmoveto{\pgfqpoint{2.125178in}{0.550000in}}%
\pgfpathlineto{\pgfqpoint{2.152699in}{0.550000in}}%
\pgfpathlineto{\pgfqpoint{2.152699in}{0.870193in}}%
\pgfpathlineto{\pgfqpoint{2.125178in}{0.870193in}}%
\pgfpathlineto{\pgfqpoint{2.125178in}{0.550000in}}%
\pgfpathclose%
\pgfusepath{fill}%
\end{pgfscope}%
\begin{pgfscope}%
\pgfpathrectangle{\pgfqpoint{1.250000in}{0.550000in}}{\pgfqpoint{7.750000in}{3.850000in}}%
\pgfusepath{clip}%
\pgfsetbuttcap%
\pgfsetmiterjoin%
\definecolor{currentfill}{rgb}{0.501961,0.501961,0.501961}%
\pgfsetfillcolor{currentfill}%
\pgfsetfillopacity{0.700000}%
\pgfsetlinewidth{0.000000pt}%
\definecolor{currentstroke}{rgb}{0.000000,0.000000,0.000000}%
\pgfsetstrokecolor{currentstroke}%
\pgfsetstrokeopacity{0.700000}%
\pgfsetdash{}{0pt}%
\pgfpathmoveto{\pgfqpoint{2.152699in}{0.550000in}}%
\pgfpathlineto{\pgfqpoint{2.180220in}{0.550000in}}%
\pgfpathlineto{\pgfqpoint{2.180220in}{0.874598in}}%
\pgfpathlineto{\pgfqpoint{2.152699in}{0.874598in}}%
\pgfpathlineto{\pgfqpoint{2.152699in}{0.550000in}}%
\pgfpathclose%
\pgfusepath{fill}%
\end{pgfscope}%
\begin{pgfscope}%
\pgfpathrectangle{\pgfqpoint{1.250000in}{0.550000in}}{\pgfqpoint{7.750000in}{3.850000in}}%
\pgfusepath{clip}%
\pgfsetbuttcap%
\pgfsetmiterjoin%
\definecolor{currentfill}{rgb}{0.501961,0.501961,0.501961}%
\pgfsetfillcolor{currentfill}%
\pgfsetfillopacity{0.700000}%
\pgfsetlinewidth{0.000000pt}%
\definecolor{currentstroke}{rgb}{0.000000,0.000000,0.000000}%
\pgfsetstrokecolor{currentstroke}%
\pgfsetstrokeopacity{0.700000}%
\pgfsetdash{}{0pt}%
\pgfpathmoveto{\pgfqpoint{2.180220in}{0.550000in}}%
\pgfpathlineto{\pgfqpoint{2.207741in}{0.550000in}}%
\pgfpathlineto{\pgfqpoint{2.207741in}{0.885060in}}%
\pgfpathlineto{\pgfqpoint{2.180220in}{0.885060in}}%
\pgfpathlineto{\pgfqpoint{2.180220in}{0.550000in}}%
\pgfpathclose%
\pgfusepath{fill}%
\end{pgfscope}%
\begin{pgfscope}%
\pgfpathrectangle{\pgfqpoint{1.250000in}{0.550000in}}{\pgfqpoint{7.750000in}{3.850000in}}%
\pgfusepath{clip}%
\pgfsetbuttcap%
\pgfsetmiterjoin%
\definecolor{currentfill}{rgb}{0.501961,0.501961,0.501961}%
\pgfsetfillcolor{currentfill}%
\pgfsetfillopacity{0.700000}%
\pgfsetlinewidth{0.000000pt}%
\definecolor{currentstroke}{rgb}{0.000000,0.000000,0.000000}%
\pgfsetstrokecolor{currentstroke}%
\pgfsetstrokeopacity{0.700000}%
\pgfsetdash{}{0pt}%
\pgfpathmoveto{\pgfqpoint{2.207741in}{0.550000in}}%
\pgfpathlineto{\pgfqpoint{2.235263in}{0.550000in}}%
\pgfpathlineto{\pgfqpoint{2.235263in}{0.871845in}}%
\pgfpathlineto{\pgfqpoint{2.207741in}{0.871845in}}%
\pgfpathlineto{\pgfqpoint{2.207741in}{0.550000in}}%
\pgfpathclose%
\pgfusepath{fill}%
\end{pgfscope}%
\begin{pgfscope}%
\pgfpathrectangle{\pgfqpoint{1.250000in}{0.550000in}}{\pgfqpoint{7.750000in}{3.850000in}}%
\pgfusepath{clip}%
\pgfsetbuttcap%
\pgfsetmiterjoin%
\definecolor{currentfill}{rgb}{0.501961,0.501961,0.501961}%
\pgfsetfillcolor{currentfill}%
\pgfsetfillopacity{0.700000}%
\pgfsetlinewidth{0.000000pt}%
\definecolor{currentstroke}{rgb}{0.000000,0.000000,0.000000}%
\pgfsetstrokecolor{currentstroke}%
\pgfsetstrokeopacity{0.700000}%
\pgfsetdash{}{0pt}%
\pgfpathmoveto{\pgfqpoint{2.235263in}{0.550000in}}%
\pgfpathlineto{\pgfqpoint{2.262784in}{0.550000in}}%
\pgfpathlineto{\pgfqpoint{2.262784in}{0.822839in}}%
\pgfpathlineto{\pgfqpoint{2.235263in}{0.822839in}}%
\pgfpathlineto{\pgfqpoint{2.235263in}{0.550000in}}%
\pgfpathclose%
\pgfusepath{fill}%
\end{pgfscope}%
\begin{pgfscope}%
\pgfpathrectangle{\pgfqpoint{1.250000in}{0.550000in}}{\pgfqpoint{7.750000in}{3.850000in}}%
\pgfusepath{clip}%
\pgfsetbuttcap%
\pgfsetmiterjoin%
\definecolor{currentfill}{rgb}{0.501961,0.501961,0.501961}%
\pgfsetfillcolor{currentfill}%
\pgfsetfillopacity{0.700000}%
\pgfsetlinewidth{0.000000pt}%
\definecolor{currentstroke}{rgb}{0.000000,0.000000,0.000000}%
\pgfsetstrokecolor{currentstroke}%
\pgfsetstrokeopacity{0.700000}%
\pgfsetdash{}{0pt}%
\pgfpathmoveto{\pgfqpoint{2.262784in}{0.550000in}}%
\pgfpathlineto{\pgfqpoint{2.290305in}{0.550000in}}%
\pgfpathlineto{\pgfqpoint{2.290305in}{0.820912in}}%
\pgfpathlineto{\pgfqpoint{2.262784in}{0.820912in}}%
\pgfpathlineto{\pgfqpoint{2.262784in}{0.550000in}}%
\pgfpathclose%
\pgfusepath{fill}%
\end{pgfscope}%
\begin{pgfscope}%
\pgfpathrectangle{\pgfqpoint{1.250000in}{0.550000in}}{\pgfqpoint{7.750000in}{3.850000in}}%
\pgfusepath{clip}%
\pgfsetbuttcap%
\pgfsetmiterjoin%
\definecolor{currentfill}{rgb}{0.501961,0.501961,0.501961}%
\pgfsetfillcolor{currentfill}%
\pgfsetfillopacity{0.700000}%
\pgfsetlinewidth{0.000000pt}%
\definecolor{currentstroke}{rgb}{0.000000,0.000000,0.000000}%
\pgfsetstrokecolor{currentstroke}%
\pgfsetstrokeopacity{0.700000}%
\pgfsetdash{}{0pt}%
\pgfpathmoveto{\pgfqpoint{2.290305in}{0.550000in}}%
\pgfpathlineto{\pgfqpoint{2.317827in}{0.550000in}}%
\pgfpathlineto{\pgfqpoint{2.317827in}{0.812652in}}%
\pgfpathlineto{\pgfqpoint{2.290305in}{0.812652in}}%
\pgfpathlineto{\pgfqpoint{2.290305in}{0.550000in}}%
\pgfpathclose%
\pgfusepath{fill}%
\end{pgfscope}%
\begin{pgfscope}%
\pgfpathrectangle{\pgfqpoint{1.250000in}{0.550000in}}{\pgfqpoint{7.750000in}{3.850000in}}%
\pgfusepath{clip}%
\pgfsetbuttcap%
\pgfsetmiterjoin%
\definecolor{currentfill}{rgb}{0.501961,0.501961,0.501961}%
\pgfsetfillcolor{currentfill}%
\pgfsetfillopacity{0.700000}%
\pgfsetlinewidth{0.000000pt}%
\definecolor{currentstroke}{rgb}{0.000000,0.000000,0.000000}%
\pgfsetstrokecolor{currentstroke}%
\pgfsetstrokeopacity{0.700000}%
\pgfsetdash{}{0pt}%
\pgfpathmoveto{\pgfqpoint{2.317827in}{0.550000in}}%
\pgfpathlineto{\pgfqpoint{2.345348in}{0.550000in}}%
\pgfpathlineto{\pgfqpoint{2.345348in}{0.811000in}}%
\pgfpathlineto{\pgfqpoint{2.317827in}{0.811000in}}%
\pgfpathlineto{\pgfqpoint{2.317827in}{0.550000in}}%
\pgfpathclose%
\pgfusepath{fill}%
\end{pgfscope}%
\begin{pgfscope}%
\pgfpathrectangle{\pgfqpoint{1.250000in}{0.550000in}}{\pgfqpoint{7.750000in}{3.850000in}}%
\pgfusepath{clip}%
\pgfsetbuttcap%
\pgfsetmiterjoin%
\definecolor{currentfill}{rgb}{0.501961,0.501961,0.501961}%
\pgfsetfillcolor{currentfill}%
\pgfsetfillopacity{0.700000}%
\pgfsetlinewidth{0.000000pt}%
\definecolor{currentstroke}{rgb}{0.000000,0.000000,0.000000}%
\pgfsetstrokecolor{currentstroke}%
\pgfsetstrokeopacity{0.700000}%
\pgfsetdash{}{0pt}%
\pgfpathmoveto{\pgfqpoint{2.345348in}{0.550000in}}%
\pgfpathlineto{\pgfqpoint{2.372869in}{0.550000in}}%
\pgfpathlineto{\pgfqpoint{2.372869in}{0.829722in}}%
\pgfpathlineto{\pgfqpoint{2.345348in}{0.829722in}}%
\pgfpathlineto{\pgfqpoint{2.345348in}{0.550000in}}%
\pgfpathclose%
\pgfusepath{fill}%
\end{pgfscope}%
\begin{pgfscope}%
\pgfpathrectangle{\pgfqpoint{1.250000in}{0.550000in}}{\pgfqpoint{7.750000in}{3.850000in}}%
\pgfusepath{clip}%
\pgfsetbuttcap%
\pgfsetmiterjoin%
\definecolor{currentfill}{rgb}{0.501961,0.501961,0.501961}%
\pgfsetfillcolor{currentfill}%
\pgfsetfillopacity{0.700000}%
\pgfsetlinewidth{0.000000pt}%
\definecolor{currentstroke}{rgb}{0.000000,0.000000,0.000000}%
\pgfsetstrokecolor{currentstroke}%
\pgfsetstrokeopacity{0.700000}%
\pgfsetdash{}{0pt}%
\pgfpathmoveto{\pgfqpoint{2.372869in}{0.550000in}}%
\pgfpathlineto{\pgfqpoint{2.400391in}{0.550000in}}%
\pgfpathlineto{\pgfqpoint{2.400391in}{0.808247in}}%
\pgfpathlineto{\pgfqpoint{2.372869in}{0.808247in}}%
\pgfpathlineto{\pgfqpoint{2.372869in}{0.550000in}}%
\pgfpathclose%
\pgfusepath{fill}%
\end{pgfscope}%
\begin{pgfscope}%
\pgfpathrectangle{\pgfqpoint{1.250000in}{0.550000in}}{\pgfqpoint{7.750000in}{3.850000in}}%
\pgfusepath{clip}%
\pgfsetbuttcap%
\pgfsetmiterjoin%
\definecolor{currentfill}{rgb}{0.501961,0.501961,0.501961}%
\pgfsetfillcolor{currentfill}%
\pgfsetfillopacity{0.700000}%
\pgfsetlinewidth{0.000000pt}%
\definecolor{currentstroke}{rgb}{0.000000,0.000000,0.000000}%
\pgfsetstrokecolor{currentstroke}%
\pgfsetstrokeopacity{0.700000}%
\pgfsetdash{}{0pt}%
\pgfpathmoveto{\pgfqpoint{2.400391in}{0.550000in}}%
\pgfpathlineto{\pgfqpoint{2.427912in}{0.550000in}}%
\pgfpathlineto{\pgfqpoint{2.427912in}{0.814304in}}%
\pgfpathlineto{\pgfqpoint{2.400391in}{0.814304in}}%
\pgfpathlineto{\pgfqpoint{2.400391in}{0.550000in}}%
\pgfpathclose%
\pgfusepath{fill}%
\end{pgfscope}%
\begin{pgfscope}%
\pgfpathrectangle{\pgfqpoint{1.250000in}{0.550000in}}{\pgfqpoint{7.750000in}{3.850000in}}%
\pgfusepath{clip}%
\pgfsetbuttcap%
\pgfsetmiterjoin%
\definecolor{currentfill}{rgb}{0.501961,0.501961,0.501961}%
\pgfsetfillcolor{currentfill}%
\pgfsetfillopacity{0.700000}%
\pgfsetlinewidth{0.000000pt}%
\definecolor{currentstroke}{rgb}{0.000000,0.000000,0.000000}%
\pgfsetstrokecolor{currentstroke}%
\pgfsetstrokeopacity{0.700000}%
\pgfsetdash{}{0pt}%
\pgfpathmoveto{\pgfqpoint{2.427912in}{0.550000in}}%
\pgfpathlineto{\pgfqpoint{2.455433in}{0.550000in}}%
\pgfpathlineto{\pgfqpoint{2.455433in}{0.811000in}}%
\pgfpathlineto{\pgfqpoint{2.427912in}{0.811000in}}%
\pgfpathlineto{\pgfqpoint{2.427912in}{0.550000in}}%
\pgfpathclose%
\pgfusepath{fill}%
\end{pgfscope}%
\begin{pgfscope}%
\pgfpathrectangle{\pgfqpoint{1.250000in}{0.550000in}}{\pgfqpoint{7.750000in}{3.850000in}}%
\pgfusepath{clip}%
\pgfsetbuttcap%
\pgfsetmiterjoin%
\definecolor{currentfill}{rgb}{0.501961,0.501961,0.501961}%
\pgfsetfillcolor{currentfill}%
\pgfsetfillopacity{0.700000}%
\pgfsetlinewidth{0.000000pt}%
\definecolor{currentstroke}{rgb}{0.000000,0.000000,0.000000}%
\pgfsetstrokecolor{currentstroke}%
\pgfsetstrokeopacity{0.700000}%
\pgfsetdash{}{0pt}%
\pgfpathmoveto{\pgfqpoint{2.455433in}{0.550000in}}%
\pgfpathlineto{\pgfqpoint{2.482955in}{0.550000in}}%
\pgfpathlineto{\pgfqpoint{2.482955in}{0.792829in}}%
\pgfpathlineto{\pgfqpoint{2.455433in}{0.792829in}}%
\pgfpathlineto{\pgfqpoint{2.455433in}{0.550000in}}%
\pgfpathclose%
\pgfusepath{fill}%
\end{pgfscope}%
\begin{pgfscope}%
\pgfpathrectangle{\pgfqpoint{1.250000in}{0.550000in}}{\pgfqpoint{7.750000in}{3.850000in}}%
\pgfusepath{clip}%
\pgfsetbuttcap%
\pgfsetmiterjoin%
\definecolor{currentfill}{rgb}{0.501961,0.501961,0.501961}%
\pgfsetfillcolor{currentfill}%
\pgfsetfillopacity{0.700000}%
\pgfsetlinewidth{0.000000pt}%
\definecolor{currentstroke}{rgb}{0.000000,0.000000,0.000000}%
\pgfsetstrokecolor{currentstroke}%
\pgfsetstrokeopacity{0.700000}%
\pgfsetdash{}{0pt}%
\pgfpathmoveto{\pgfqpoint{2.482955in}{0.550000in}}%
\pgfpathlineto{\pgfqpoint{2.510476in}{0.550000in}}%
\pgfpathlineto{\pgfqpoint{2.510476in}{0.764747in}}%
\pgfpathlineto{\pgfqpoint{2.482955in}{0.764747in}}%
\pgfpathlineto{\pgfqpoint{2.482955in}{0.550000in}}%
\pgfpathclose%
\pgfusepath{fill}%
\end{pgfscope}%
\begin{pgfscope}%
\pgfpathrectangle{\pgfqpoint{1.250000in}{0.550000in}}{\pgfqpoint{7.750000in}{3.850000in}}%
\pgfusepath{clip}%
\pgfsetbuttcap%
\pgfsetmiterjoin%
\definecolor{currentfill}{rgb}{0.501961,0.501961,0.501961}%
\pgfsetfillcolor{currentfill}%
\pgfsetfillopacity{0.700000}%
\pgfsetlinewidth{0.000000pt}%
\definecolor{currentstroke}{rgb}{0.000000,0.000000,0.000000}%
\pgfsetstrokecolor{currentstroke}%
\pgfsetstrokeopacity{0.700000}%
\pgfsetdash{}{0pt}%
\pgfpathmoveto{\pgfqpoint{2.510476in}{0.550000in}}%
\pgfpathlineto{\pgfqpoint{2.537997in}{0.550000in}}%
\pgfpathlineto{\pgfqpoint{2.537997in}{0.770529in}}%
\pgfpathlineto{\pgfqpoint{2.510476in}{0.770529in}}%
\pgfpathlineto{\pgfqpoint{2.510476in}{0.550000in}}%
\pgfpathclose%
\pgfusepath{fill}%
\end{pgfscope}%
\begin{pgfscope}%
\pgfpathrectangle{\pgfqpoint{1.250000in}{0.550000in}}{\pgfqpoint{7.750000in}{3.850000in}}%
\pgfusepath{clip}%
\pgfsetbuttcap%
\pgfsetmiterjoin%
\definecolor{currentfill}{rgb}{0.501961,0.501961,0.501961}%
\pgfsetfillcolor{currentfill}%
\pgfsetfillopacity{0.700000}%
\pgfsetlinewidth{0.000000pt}%
\definecolor{currentstroke}{rgb}{0.000000,0.000000,0.000000}%
\pgfsetstrokecolor{currentstroke}%
\pgfsetstrokeopacity{0.700000}%
\pgfsetdash{}{0pt}%
\pgfpathmoveto{\pgfqpoint{2.537997in}{0.550000in}}%
\pgfpathlineto{\pgfqpoint{2.565518in}{0.550000in}}%
\pgfpathlineto{\pgfqpoint{2.565518in}{0.755386in}}%
\pgfpathlineto{\pgfqpoint{2.537997in}{0.755386in}}%
\pgfpathlineto{\pgfqpoint{2.537997in}{0.550000in}}%
\pgfpathclose%
\pgfusepath{fill}%
\end{pgfscope}%
\begin{pgfscope}%
\pgfpathrectangle{\pgfqpoint{1.250000in}{0.550000in}}{\pgfqpoint{7.750000in}{3.850000in}}%
\pgfusepath{clip}%
\pgfsetbuttcap%
\pgfsetmiterjoin%
\definecolor{currentfill}{rgb}{0.501961,0.501961,0.501961}%
\pgfsetfillcolor{currentfill}%
\pgfsetfillopacity{0.700000}%
\pgfsetlinewidth{0.000000pt}%
\definecolor{currentstroke}{rgb}{0.000000,0.000000,0.000000}%
\pgfsetstrokecolor{currentstroke}%
\pgfsetstrokeopacity{0.700000}%
\pgfsetdash{}{0pt}%
\pgfpathmoveto{\pgfqpoint{2.565518in}{0.550000in}}%
\pgfpathlineto{\pgfqpoint{2.593040in}{0.550000in}}%
\pgfpathlineto{\pgfqpoint{2.593040in}{0.779339in}}%
\pgfpathlineto{\pgfqpoint{2.565518in}{0.779339in}}%
\pgfpathlineto{\pgfqpoint{2.565518in}{0.550000in}}%
\pgfpathclose%
\pgfusepath{fill}%
\end{pgfscope}%
\begin{pgfscope}%
\pgfpathrectangle{\pgfqpoint{1.250000in}{0.550000in}}{\pgfqpoint{7.750000in}{3.850000in}}%
\pgfusepath{clip}%
\pgfsetbuttcap%
\pgfsetmiterjoin%
\definecolor{currentfill}{rgb}{0.501961,0.501961,0.501961}%
\pgfsetfillcolor{currentfill}%
\pgfsetfillopacity{0.700000}%
\pgfsetlinewidth{0.000000pt}%
\definecolor{currentstroke}{rgb}{0.000000,0.000000,0.000000}%
\pgfsetstrokecolor{currentstroke}%
\pgfsetstrokeopacity{0.700000}%
\pgfsetdash{}{0pt}%
\pgfpathmoveto{\pgfqpoint{2.593040in}{0.550000in}}%
\pgfpathlineto{\pgfqpoint{2.620561in}{0.550000in}}%
\pgfpathlineto{\pgfqpoint{2.620561in}{0.763646in}}%
\pgfpathlineto{\pgfqpoint{2.593040in}{0.763646in}}%
\pgfpathlineto{\pgfqpoint{2.593040in}{0.550000in}}%
\pgfpathclose%
\pgfusepath{fill}%
\end{pgfscope}%
\begin{pgfscope}%
\pgfpathrectangle{\pgfqpoint{1.250000in}{0.550000in}}{\pgfqpoint{7.750000in}{3.850000in}}%
\pgfusepath{clip}%
\pgfsetbuttcap%
\pgfsetmiterjoin%
\definecolor{currentfill}{rgb}{0.501961,0.501961,0.501961}%
\pgfsetfillcolor{currentfill}%
\pgfsetfillopacity{0.700000}%
\pgfsetlinewidth{0.000000pt}%
\definecolor{currentstroke}{rgb}{0.000000,0.000000,0.000000}%
\pgfsetstrokecolor{currentstroke}%
\pgfsetstrokeopacity{0.700000}%
\pgfsetdash{}{0pt}%
\pgfpathmoveto{\pgfqpoint{2.620561in}{0.550000in}}%
\pgfpathlineto{\pgfqpoint{2.648082in}{0.550000in}}%
\pgfpathlineto{\pgfqpoint{2.648082in}{0.786222in}}%
\pgfpathlineto{\pgfqpoint{2.620561in}{0.786222in}}%
\pgfpathlineto{\pgfqpoint{2.620561in}{0.550000in}}%
\pgfpathclose%
\pgfusepath{fill}%
\end{pgfscope}%
\begin{pgfscope}%
\pgfpathrectangle{\pgfqpoint{1.250000in}{0.550000in}}{\pgfqpoint{7.750000in}{3.850000in}}%
\pgfusepath{clip}%
\pgfsetbuttcap%
\pgfsetmiterjoin%
\definecolor{currentfill}{rgb}{0.501961,0.501961,0.501961}%
\pgfsetfillcolor{currentfill}%
\pgfsetfillopacity{0.700000}%
\pgfsetlinewidth{0.000000pt}%
\definecolor{currentstroke}{rgb}{0.000000,0.000000,0.000000}%
\pgfsetstrokecolor{currentstroke}%
\pgfsetstrokeopacity{0.700000}%
\pgfsetdash{}{0pt}%
\pgfpathmoveto{\pgfqpoint{2.648082in}{0.550000in}}%
\pgfpathlineto{\pgfqpoint{2.675604in}{0.550000in}}%
\pgfpathlineto{\pgfqpoint{2.675604in}{0.762269in}}%
\pgfpathlineto{\pgfqpoint{2.648082in}{0.762269in}}%
\pgfpathlineto{\pgfqpoint{2.648082in}{0.550000in}}%
\pgfpathclose%
\pgfusepath{fill}%
\end{pgfscope}%
\begin{pgfscope}%
\pgfpathrectangle{\pgfqpoint{1.250000in}{0.550000in}}{\pgfqpoint{7.750000in}{3.850000in}}%
\pgfusepath{clip}%
\pgfsetbuttcap%
\pgfsetmiterjoin%
\definecolor{currentfill}{rgb}{0.501961,0.501961,0.501961}%
\pgfsetfillcolor{currentfill}%
\pgfsetfillopacity{0.700000}%
\pgfsetlinewidth{0.000000pt}%
\definecolor{currentstroke}{rgb}{0.000000,0.000000,0.000000}%
\pgfsetstrokecolor{currentstroke}%
\pgfsetstrokeopacity{0.700000}%
\pgfsetdash{}{0pt}%
\pgfpathmoveto{\pgfqpoint{2.675604in}{0.550000in}}%
\pgfpathlineto{\pgfqpoint{2.703125in}{0.550000in}}%
\pgfpathlineto{\pgfqpoint{2.703125in}{0.749329in}}%
\pgfpathlineto{\pgfqpoint{2.675604in}{0.749329in}}%
\pgfpathlineto{\pgfqpoint{2.675604in}{0.550000in}}%
\pgfpathclose%
\pgfusepath{fill}%
\end{pgfscope}%
\begin{pgfscope}%
\pgfpathrectangle{\pgfqpoint{1.250000in}{0.550000in}}{\pgfqpoint{7.750000in}{3.850000in}}%
\pgfusepath{clip}%
\pgfsetbuttcap%
\pgfsetmiterjoin%
\definecolor{currentfill}{rgb}{0.501961,0.501961,0.501961}%
\pgfsetfillcolor{currentfill}%
\pgfsetfillopacity{0.700000}%
\pgfsetlinewidth{0.000000pt}%
\definecolor{currentstroke}{rgb}{0.000000,0.000000,0.000000}%
\pgfsetstrokecolor{currentstroke}%
\pgfsetstrokeopacity{0.700000}%
\pgfsetdash{}{0pt}%
\pgfpathmoveto{\pgfqpoint{2.703125in}{0.550000in}}%
\pgfpathlineto{\pgfqpoint{2.730646in}{0.550000in}}%
\pgfpathlineto{\pgfqpoint{2.730646in}{0.746851in}}%
\pgfpathlineto{\pgfqpoint{2.703125in}{0.746851in}}%
\pgfpathlineto{\pgfqpoint{2.703125in}{0.550000in}}%
\pgfpathclose%
\pgfusepath{fill}%
\end{pgfscope}%
\begin{pgfscope}%
\pgfpathrectangle{\pgfqpoint{1.250000in}{0.550000in}}{\pgfqpoint{7.750000in}{3.850000in}}%
\pgfusepath{clip}%
\pgfsetbuttcap%
\pgfsetmiterjoin%
\definecolor{currentfill}{rgb}{0.501961,0.501961,0.501961}%
\pgfsetfillcolor{currentfill}%
\pgfsetfillopacity{0.700000}%
\pgfsetlinewidth{0.000000pt}%
\definecolor{currentstroke}{rgb}{0.000000,0.000000,0.000000}%
\pgfsetstrokecolor{currentstroke}%
\pgfsetstrokeopacity{0.700000}%
\pgfsetdash{}{0pt}%
\pgfpathmoveto{\pgfqpoint{2.730646in}{0.550000in}}%
\pgfpathlineto{\pgfqpoint{2.758168in}{0.550000in}}%
\pgfpathlineto{\pgfqpoint{2.758168in}{0.733911in}}%
\pgfpathlineto{\pgfqpoint{2.730646in}{0.733911in}}%
\pgfpathlineto{\pgfqpoint{2.730646in}{0.550000in}}%
\pgfpathclose%
\pgfusepath{fill}%
\end{pgfscope}%
\begin{pgfscope}%
\pgfpathrectangle{\pgfqpoint{1.250000in}{0.550000in}}{\pgfqpoint{7.750000in}{3.850000in}}%
\pgfusepath{clip}%
\pgfsetbuttcap%
\pgfsetmiterjoin%
\definecolor{currentfill}{rgb}{0.501961,0.501961,0.501961}%
\pgfsetfillcolor{currentfill}%
\pgfsetfillopacity{0.700000}%
\pgfsetlinewidth{0.000000pt}%
\definecolor{currentstroke}{rgb}{0.000000,0.000000,0.000000}%
\pgfsetstrokecolor{currentstroke}%
\pgfsetstrokeopacity{0.700000}%
\pgfsetdash{}{0pt}%
\pgfpathmoveto{\pgfqpoint{2.758168in}{0.550000in}}%
\pgfpathlineto{\pgfqpoint{2.785689in}{0.550000in}}%
\pgfpathlineto{\pgfqpoint{2.785689in}{0.735839in}}%
\pgfpathlineto{\pgfqpoint{2.758168in}{0.735839in}}%
\pgfpathlineto{\pgfqpoint{2.758168in}{0.550000in}}%
\pgfpathclose%
\pgfusepath{fill}%
\end{pgfscope}%
\begin{pgfscope}%
\pgfpathrectangle{\pgfqpoint{1.250000in}{0.550000in}}{\pgfqpoint{7.750000in}{3.850000in}}%
\pgfusepath{clip}%
\pgfsetbuttcap%
\pgfsetmiterjoin%
\definecolor{currentfill}{rgb}{0.501961,0.501961,0.501961}%
\pgfsetfillcolor{currentfill}%
\pgfsetfillopacity{0.700000}%
\pgfsetlinewidth{0.000000pt}%
\definecolor{currentstroke}{rgb}{0.000000,0.000000,0.000000}%
\pgfsetstrokecolor{currentstroke}%
\pgfsetstrokeopacity{0.700000}%
\pgfsetdash{}{0pt}%
\pgfpathmoveto{\pgfqpoint{2.785689in}{0.550000in}}%
\pgfpathlineto{\pgfqpoint{2.813210in}{0.550000in}}%
\pgfpathlineto{\pgfqpoint{2.813210in}{0.730883in}}%
\pgfpathlineto{\pgfqpoint{2.785689in}{0.730883in}}%
\pgfpathlineto{\pgfqpoint{2.785689in}{0.550000in}}%
\pgfpathclose%
\pgfusepath{fill}%
\end{pgfscope}%
\begin{pgfscope}%
\pgfpathrectangle{\pgfqpoint{1.250000in}{0.550000in}}{\pgfqpoint{7.750000in}{3.850000in}}%
\pgfusepath{clip}%
\pgfsetbuttcap%
\pgfsetmiterjoin%
\definecolor{currentfill}{rgb}{0.501961,0.501961,0.501961}%
\pgfsetfillcolor{currentfill}%
\pgfsetfillopacity{0.700000}%
\pgfsetlinewidth{0.000000pt}%
\definecolor{currentstroke}{rgb}{0.000000,0.000000,0.000000}%
\pgfsetstrokecolor{currentstroke}%
\pgfsetstrokeopacity{0.700000}%
\pgfsetdash{}{0pt}%
\pgfpathmoveto{\pgfqpoint{2.813210in}{0.550000in}}%
\pgfpathlineto{\pgfqpoint{2.840732in}{0.550000in}}%
\pgfpathlineto{\pgfqpoint{2.840732in}{0.725377in}}%
\pgfpathlineto{\pgfqpoint{2.813210in}{0.725377in}}%
\pgfpathlineto{\pgfqpoint{2.813210in}{0.550000in}}%
\pgfpathclose%
\pgfusepath{fill}%
\end{pgfscope}%
\begin{pgfscope}%
\pgfpathrectangle{\pgfqpoint{1.250000in}{0.550000in}}{\pgfqpoint{7.750000in}{3.850000in}}%
\pgfusepath{clip}%
\pgfsetbuttcap%
\pgfsetmiterjoin%
\definecolor{currentfill}{rgb}{0.501961,0.501961,0.501961}%
\pgfsetfillcolor{currentfill}%
\pgfsetfillopacity{0.700000}%
\pgfsetlinewidth{0.000000pt}%
\definecolor{currentstroke}{rgb}{0.000000,0.000000,0.000000}%
\pgfsetstrokecolor{currentstroke}%
\pgfsetstrokeopacity{0.700000}%
\pgfsetdash{}{0pt}%
\pgfpathmoveto{\pgfqpoint{2.840732in}{0.550000in}}%
\pgfpathlineto{\pgfqpoint{2.868253in}{0.550000in}}%
\pgfpathlineto{\pgfqpoint{2.868253in}{0.726753in}}%
\pgfpathlineto{\pgfqpoint{2.840732in}{0.726753in}}%
\pgfpathlineto{\pgfqpoint{2.840732in}{0.550000in}}%
\pgfpathclose%
\pgfusepath{fill}%
\end{pgfscope}%
\begin{pgfscope}%
\pgfpathrectangle{\pgfqpoint{1.250000in}{0.550000in}}{\pgfqpoint{7.750000in}{3.850000in}}%
\pgfusepath{clip}%
\pgfsetbuttcap%
\pgfsetmiterjoin%
\definecolor{currentfill}{rgb}{0.501961,0.501961,0.501961}%
\pgfsetfillcolor{currentfill}%
\pgfsetfillopacity{0.700000}%
\pgfsetlinewidth{0.000000pt}%
\definecolor{currentstroke}{rgb}{0.000000,0.000000,0.000000}%
\pgfsetstrokecolor{currentstroke}%
\pgfsetstrokeopacity{0.700000}%
\pgfsetdash{}{0pt}%
\pgfpathmoveto{\pgfqpoint{2.868253in}{0.550000in}}%
\pgfpathlineto{\pgfqpoint{2.895774in}{0.550000in}}%
\pgfpathlineto{\pgfqpoint{2.895774in}{0.734187in}}%
\pgfpathlineto{\pgfqpoint{2.868253in}{0.734187in}}%
\pgfpathlineto{\pgfqpoint{2.868253in}{0.550000in}}%
\pgfpathclose%
\pgfusepath{fill}%
\end{pgfscope}%
\begin{pgfscope}%
\pgfpathrectangle{\pgfqpoint{1.250000in}{0.550000in}}{\pgfqpoint{7.750000in}{3.850000in}}%
\pgfusepath{clip}%
\pgfsetbuttcap%
\pgfsetmiterjoin%
\definecolor{currentfill}{rgb}{0.501961,0.501961,0.501961}%
\pgfsetfillcolor{currentfill}%
\pgfsetfillopacity{0.700000}%
\pgfsetlinewidth{0.000000pt}%
\definecolor{currentstroke}{rgb}{0.000000,0.000000,0.000000}%
\pgfsetstrokecolor{currentstroke}%
\pgfsetstrokeopacity{0.700000}%
\pgfsetdash{}{0pt}%
\pgfpathmoveto{\pgfqpoint{2.895774in}{0.550000in}}%
\pgfpathlineto{\pgfqpoint{2.923295in}{0.550000in}}%
\pgfpathlineto{\pgfqpoint{2.923295in}{0.750430in}}%
\pgfpathlineto{\pgfqpoint{2.895774in}{0.750430in}}%
\pgfpathlineto{\pgfqpoint{2.895774in}{0.550000in}}%
\pgfpathclose%
\pgfusepath{fill}%
\end{pgfscope}%
\begin{pgfscope}%
\pgfpathrectangle{\pgfqpoint{1.250000in}{0.550000in}}{\pgfqpoint{7.750000in}{3.850000in}}%
\pgfusepath{clip}%
\pgfsetbuttcap%
\pgfsetmiterjoin%
\definecolor{currentfill}{rgb}{0.501961,0.501961,0.501961}%
\pgfsetfillcolor{currentfill}%
\pgfsetfillopacity{0.700000}%
\pgfsetlinewidth{0.000000pt}%
\definecolor{currentstroke}{rgb}{0.000000,0.000000,0.000000}%
\pgfsetstrokecolor{currentstroke}%
\pgfsetstrokeopacity{0.700000}%
\pgfsetdash{}{0pt}%
\pgfpathmoveto{\pgfqpoint{2.923295in}{0.550000in}}%
\pgfpathlineto{\pgfqpoint{2.950817in}{0.550000in}}%
\pgfpathlineto{\pgfqpoint{2.950817in}{0.741896in}}%
\pgfpathlineto{\pgfqpoint{2.923295in}{0.741896in}}%
\pgfpathlineto{\pgfqpoint{2.923295in}{0.550000in}}%
\pgfpathclose%
\pgfusepath{fill}%
\end{pgfscope}%
\begin{pgfscope}%
\pgfpathrectangle{\pgfqpoint{1.250000in}{0.550000in}}{\pgfqpoint{7.750000in}{3.850000in}}%
\pgfusepath{clip}%
\pgfsetbuttcap%
\pgfsetmiterjoin%
\definecolor{currentfill}{rgb}{0.501961,0.501961,0.501961}%
\pgfsetfillcolor{currentfill}%
\pgfsetfillopacity{0.700000}%
\pgfsetlinewidth{0.000000pt}%
\definecolor{currentstroke}{rgb}{0.000000,0.000000,0.000000}%
\pgfsetstrokecolor{currentstroke}%
\pgfsetstrokeopacity{0.700000}%
\pgfsetdash{}{0pt}%
\pgfpathmoveto{\pgfqpoint{2.950817in}{0.550000in}}%
\pgfpathlineto{\pgfqpoint{2.978338in}{0.550000in}}%
\pgfpathlineto{\pgfqpoint{2.978338in}{0.738592in}}%
\pgfpathlineto{\pgfqpoint{2.950817in}{0.738592in}}%
\pgfpathlineto{\pgfqpoint{2.950817in}{0.550000in}}%
\pgfpathclose%
\pgfusepath{fill}%
\end{pgfscope}%
\begin{pgfscope}%
\pgfpathrectangle{\pgfqpoint{1.250000in}{0.550000in}}{\pgfqpoint{7.750000in}{3.850000in}}%
\pgfusepath{clip}%
\pgfsetbuttcap%
\pgfsetmiterjoin%
\definecolor{currentfill}{rgb}{0.501961,0.501961,0.501961}%
\pgfsetfillcolor{currentfill}%
\pgfsetfillopacity{0.700000}%
\pgfsetlinewidth{0.000000pt}%
\definecolor{currentstroke}{rgb}{0.000000,0.000000,0.000000}%
\pgfsetstrokecolor{currentstroke}%
\pgfsetstrokeopacity{0.700000}%
\pgfsetdash{}{0pt}%
\pgfpathmoveto{\pgfqpoint{2.978338in}{0.550000in}}%
\pgfpathlineto{\pgfqpoint{3.005859in}{0.550000in}}%
\pgfpathlineto{\pgfqpoint{3.005859in}{0.724551in}}%
\pgfpathlineto{\pgfqpoint{2.978338in}{0.724551in}}%
\pgfpathlineto{\pgfqpoint{2.978338in}{0.550000in}}%
\pgfpathclose%
\pgfusepath{fill}%
\end{pgfscope}%
\begin{pgfscope}%
\pgfpathrectangle{\pgfqpoint{1.250000in}{0.550000in}}{\pgfqpoint{7.750000in}{3.850000in}}%
\pgfusepath{clip}%
\pgfsetbuttcap%
\pgfsetmiterjoin%
\definecolor{currentfill}{rgb}{0.501961,0.501961,0.501961}%
\pgfsetfillcolor{currentfill}%
\pgfsetfillopacity{0.700000}%
\pgfsetlinewidth{0.000000pt}%
\definecolor{currentstroke}{rgb}{0.000000,0.000000,0.000000}%
\pgfsetstrokecolor{currentstroke}%
\pgfsetstrokeopacity{0.700000}%
\pgfsetdash{}{0pt}%
\pgfpathmoveto{\pgfqpoint{3.005859in}{0.550000in}}%
\pgfpathlineto{\pgfqpoint{3.033381in}{0.550000in}}%
\pgfpathlineto{\pgfqpoint{3.033381in}{0.736940in}}%
\pgfpathlineto{\pgfqpoint{3.005859in}{0.736940in}}%
\pgfpathlineto{\pgfqpoint{3.005859in}{0.550000in}}%
\pgfpathclose%
\pgfusepath{fill}%
\end{pgfscope}%
\begin{pgfscope}%
\pgfpathrectangle{\pgfqpoint{1.250000in}{0.550000in}}{\pgfqpoint{7.750000in}{3.850000in}}%
\pgfusepath{clip}%
\pgfsetbuttcap%
\pgfsetmiterjoin%
\definecolor{currentfill}{rgb}{0.501961,0.501961,0.501961}%
\pgfsetfillcolor{currentfill}%
\pgfsetfillopacity{0.700000}%
\pgfsetlinewidth{0.000000pt}%
\definecolor{currentstroke}{rgb}{0.000000,0.000000,0.000000}%
\pgfsetstrokecolor{currentstroke}%
\pgfsetstrokeopacity{0.700000}%
\pgfsetdash{}{0pt}%
\pgfpathmoveto{\pgfqpoint{3.033381in}{0.550000in}}%
\pgfpathlineto{\pgfqpoint{3.060902in}{0.550000in}}%
\pgfpathlineto{\pgfqpoint{3.060902in}{0.730883in}}%
\pgfpathlineto{\pgfqpoint{3.033381in}{0.730883in}}%
\pgfpathlineto{\pgfqpoint{3.033381in}{0.550000in}}%
\pgfpathclose%
\pgfusepath{fill}%
\end{pgfscope}%
\begin{pgfscope}%
\pgfpathrectangle{\pgfqpoint{1.250000in}{0.550000in}}{\pgfqpoint{7.750000in}{3.850000in}}%
\pgfusepath{clip}%
\pgfsetbuttcap%
\pgfsetmiterjoin%
\definecolor{currentfill}{rgb}{0.501961,0.501961,0.501961}%
\pgfsetfillcolor{currentfill}%
\pgfsetfillopacity{0.700000}%
\pgfsetlinewidth{0.000000pt}%
\definecolor{currentstroke}{rgb}{0.000000,0.000000,0.000000}%
\pgfsetstrokecolor{currentstroke}%
\pgfsetstrokeopacity{0.700000}%
\pgfsetdash{}{0pt}%
\pgfpathmoveto{\pgfqpoint{3.060902in}{0.550000in}}%
\pgfpathlineto{\pgfqpoint{3.088423in}{0.550000in}}%
\pgfpathlineto{\pgfqpoint{3.088423in}{0.710785in}}%
\pgfpathlineto{\pgfqpoint{3.060902in}{0.710785in}}%
\pgfpathlineto{\pgfqpoint{3.060902in}{0.550000in}}%
\pgfpathclose%
\pgfusepath{fill}%
\end{pgfscope}%
\begin{pgfscope}%
\pgfpathrectangle{\pgfqpoint{1.250000in}{0.550000in}}{\pgfqpoint{7.750000in}{3.850000in}}%
\pgfusepath{clip}%
\pgfsetbuttcap%
\pgfsetmiterjoin%
\definecolor{currentfill}{rgb}{0.501961,0.501961,0.501961}%
\pgfsetfillcolor{currentfill}%
\pgfsetfillopacity{0.700000}%
\pgfsetlinewidth{0.000000pt}%
\definecolor{currentstroke}{rgb}{0.000000,0.000000,0.000000}%
\pgfsetstrokecolor{currentstroke}%
\pgfsetstrokeopacity{0.700000}%
\pgfsetdash{}{0pt}%
\pgfpathmoveto{\pgfqpoint{3.088423in}{0.550000in}}%
\pgfpathlineto{\pgfqpoint{3.115945in}{0.550000in}}%
\pgfpathlineto{\pgfqpoint{3.115945in}{0.706655in}}%
\pgfpathlineto{\pgfqpoint{3.088423in}{0.706655in}}%
\pgfpathlineto{\pgfqpoint{3.088423in}{0.550000in}}%
\pgfpathclose%
\pgfusepath{fill}%
\end{pgfscope}%
\begin{pgfscope}%
\pgfpathrectangle{\pgfqpoint{1.250000in}{0.550000in}}{\pgfqpoint{7.750000in}{3.850000in}}%
\pgfusepath{clip}%
\pgfsetbuttcap%
\pgfsetmiterjoin%
\definecolor{currentfill}{rgb}{0.501961,0.501961,0.501961}%
\pgfsetfillcolor{currentfill}%
\pgfsetfillopacity{0.700000}%
\pgfsetlinewidth{0.000000pt}%
\definecolor{currentstroke}{rgb}{0.000000,0.000000,0.000000}%
\pgfsetstrokecolor{currentstroke}%
\pgfsetstrokeopacity{0.700000}%
\pgfsetdash{}{0pt}%
\pgfpathmoveto{\pgfqpoint{3.115945in}{0.550000in}}%
\pgfpathlineto{\pgfqpoint{3.143466in}{0.550000in}}%
\pgfpathlineto{\pgfqpoint{3.143466in}{0.704728in}}%
\pgfpathlineto{\pgfqpoint{3.115945in}{0.704728in}}%
\pgfpathlineto{\pgfqpoint{3.115945in}{0.550000in}}%
\pgfpathclose%
\pgfusepath{fill}%
\end{pgfscope}%
\begin{pgfscope}%
\pgfpathrectangle{\pgfqpoint{1.250000in}{0.550000in}}{\pgfqpoint{7.750000in}{3.850000in}}%
\pgfusepath{clip}%
\pgfsetbuttcap%
\pgfsetmiterjoin%
\definecolor{currentfill}{rgb}{0.501961,0.501961,0.501961}%
\pgfsetfillcolor{currentfill}%
\pgfsetfillopacity{0.700000}%
\pgfsetlinewidth{0.000000pt}%
\definecolor{currentstroke}{rgb}{0.000000,0.000000,0.000000}%
\pgfsetstrokecolor{currentstroke}%
\pgfsetstrokeopacity{0.700000}%
\pgfsetdash{}{0pt}%
\pgfpathmoveto{\pgfqpoint{3.143466in}{0.550000in}}%
\pgfpathlineto{\pgfqpoint{3.170987in}{0.550000in}}%
\pgfpathlineto{\pgfqpoint{3.170987in}{0.722073in}}%
\pgfpathlineto{\pgfqpoint{3.143466in}{0.722073in}}%
\pgfpathlineto{\pgfqpoint{3.143466in}{0.550000in}}%
\pgfpathclose%
\pgfusepath{fill}%
\end{pgfscope}%
\begin{pgfscope}%
\pgfpathrectangle{\pgfqpoint{1.250000in}{0.550000in}}{\pgfqpoint{7.750000in}{3.850000in}}%
\pgfusepath{clip}%
\pgfsetbuttcap%
\pgfsetmiterjoin%
\definecolor{currentfill}{rgb}{0.501961,0.501961,0.501961}%
\pgfsetfillcolor{currentfill}%
\pgfsetfillopacity{0.700000}%
\pgfsetlinewidth{0.000000pt}%
\definecolor{currentstroke}{rgb}{0.000000,0.000000,0.000000}%
\pgfsetstrokecolor{currentstroke}%
\pgfsetstrokeopacity{0.700000}%
\pgfsetdash{}{0pt}%
\pgfpathmoveto{\pgfqpoint{3.170987in}{0.550000in}}%
\pgfpathlineto{\pgfqpoint{3.198509in}{0.550000in}}%
\pgfpathlineto{\pgfqpoint{3.198509in}{0.723725in}}%
\pgfpathlineto{\pgfqpoint{3.170987in}{0.723725in}}%
\pgfpathlineto{\pgfqpoint{3.170987in}{0.550000in}}%
\pgfpathclose%
\pgfusepath{fill}%
\end{pgfscope}%
\begin{pgfscope}%
\pgfpathrectangle{\pgfqpoint{1.250000in}{0.550000in}}{\pgfqpoint{7.750000in}{3.850000in}}%
\pgfusepath{clip}%
\pgfsetbuttcap%
\pgfsetmiterjoin%
\definecolor{currentfill}{rgb}{0.501961,0.501961,0.501961}%
\pgfsetfillcolor{currentfill}%
\pgfsetfillopacity{0.700000}%
\pgfsetlinewidth{0.000000pt}%
\definecolor{currentstroke}{rgb}{0.000000,0.000000,0.000000}%
\pgfsetstrokecolor{currentstroke}%
\pgfsetstrokeopacity{0.700000}%
\pgfsetdash{}{0pt}%
\pgfpathmoveto{\pgfqpoint{3.198509in}{0.550000in}}%
\pgfpathlineto{\pgfqpoint{3.226030in}{0.550000in}}%
\pgfpathlineto{\pgfqpoint{3.226030in}{0.724826in}}%
\pgfpathlineto{\pgfqpoint{3.198509in}{0.724826in}}%
\pgfpathlineto{\pgfqpoint{3.198509in}{0.550000in}}%
\pgfpathclose%
\pgfusepath{fill}%
\end{pgfscope}%
\begin{pgfscope}%
\pgfpathrectangle{\pgfqpoint{1.250000in}{0.550000in}}{\pgfqpoint{7.750000in}{3.850000in}}%
\pgfusepath{clip}%
\pgfsetbuttcap%
\pgfsetmiterjoin%
\definecolor{currentfill}{rgb}{0.501961,0.501961,0.501961}%
\pgfsetfillcolor{currentfill}%
\pgfsetfillopacity{0.700000}%
\pgfsetlinewidth{0.000000pt}%
\definecolor{currentstroke}{rgb}{0.000000,0.000000,0.000000}%
\pgfsetstrokecolor{currentstroke}%
\pgfsetstrokeopacity{0.700000}%
\pgfsetdash{}{0pt}%
\pgfpathmoveto{\pgfqpoint{3.226030in}{0.550000in}}%
\pgfpathlineto{\pgfqpoint{3.253551in}{0.550000in}}%
\pgfpathlineto{\pgfqpoint{3.253551in}{0.718218in}}%
\pgfpathlineto{\pgfqpoint{3.226030in}{0.718218in}}%
\pgfpathlineto{\pgfqpoint{3.226030in}{0.550000in}}%
\pgfpathclose%
\pgfusepath{fill}%
\end{pgfscope}%
\begin{pgfscope}%
\pgfpathrectangle{\pgfqpoint{1.250000in}{0.550000in}}{\pgfqpoint{7.750000in}{3.850000in}}%
\pgfusepath{clip}%
\pgfsetbuttcap%
\pgfsetmiterjoin%
\definecolor{currentfill}{rgb}{0.501961,0.501961,0.501961}%
\pgfsetfillcolor{currentfill}%
\pgfsetfillopacity{0.700000}%
\pgfsetlinewidth{0.000000pt}%
\definecolor{currentstroke}{rgb}{0.000000,0.000000,0.000000}%
\pgfsetstrokecolor{currentstroke}%
\pgfsetstrokeopacity{0.700000}%
\pgfsetdash{}{0pt}%
\pgfpathmoveto{\pgfqpoint{3.253551in}{0.550000in}}%
\pgfpathlineto{\pgfqpoint{3.281072in}{0.550000in}}%
\pgfpathlineto{\pgfqpoint{3.281072in}{0.714639in}}%
\pgfpathlineto{\pgfqpoint{3.253551in}{0.714639in}}%
\pgfpathlineto{\pgfqpoint{3.253551in}{0.550000in}}%
\pgfpathclose%
\pgfusepath{fill}%
\end{pgfscope}%
\begin{pgfscope}%
\pgfpathrectangle{\pgfqpoint{1.250000in}{0.550000in}}{\pgfqpoint{7.750000in}{3.850000in}}%
\pgfusepath{clip}%
\pgfsetbuttcap%
\pgfsetmiterjoin%
\definecolor{currentfill}{rgb}{0.501961,0.501961,0.501961}%
\pgfsetfillcolor{currentfill}%
\pgfsetfillopacity{0.700000}%
\pgfsetlinewidth{0.000000pt}%
\definecolor{currentstroke}{rgb}{0.000000,0.000000,0.000000}%
\pgfsetstrokecolor{currentstroke}%
\pgfsetstrokeopacity{0.700000}%
\pgfsetdash{}{0pt}%
\pgfpathmoveto{\pgfqpoint{3.281072in}{0.550000in}}%
\pgfpathlineto{\pgfqpoint{3.308594in}{0.550000in}}%
\pgfpathlineto{\pgfqpoint{3.308594in}{0.710234in}}%
\pgfpathlineto{\pgfqpoint{3.281072in}{0.710234in}}%
\pgfpathlineto{\pgfqpoint{3.281072in}{0.550000in}}%
\pgfpathclose%
\pgfusepath{fill}%
\end{pgfscope}%
\begin{pgfscope}%
\pgfpathrectangle{\pgfqpoint{1.250000in}{0.550000in}}{\pgfqpoint{7.750000in}{3.850000in}}%
\pgfusepath{clip}%
\pgfsetbuttcap%
\pgfsetmiterjoin%
\definecolor{currentfill}{rgb}{0.501961,0.501961,0.501961}%
\pgfsetfillcolor{currentfill}%
\pgfsetfillopacity{0.700000}%
\pgfsetlinewidth{0.000000pt}%
\definecolor{currentstroke}{rgb}{0.000000,0.000000,0.000000}%
\pgfsetstrokecolor{currentstroke}%
\pgfsetstrokeopacity{0.700000}%
\pgfsetdash{}{0pt}%
\pgfpathmoveto{\pgfqpoint{3.308594in}{0.550000in}}%
\pgfpathlineto{\pgfqpoint{3.336115in}{0.550000in}}%
\pgfpathlineto{\pgfqpoint{3.336115in}{0.709408in}}%
\pgfpathlineto{\pgfqpoint{3.308594in}{0.709408in}}%
\pgfpathlineto{\pgfqpoint{3.308594in}{0.550000in}}%
\pgfpathclose%
\pgfusepath{fill}%
\end{pgfscope}%
\begin{pgfscope}%
\pgfpathrectangle{\pgfqpoint{1.250000in}{0.550000in}}{\pgfqpoint{7.750000in}{3.850000in}}%
\pgfusepath{clip}%
\pgfsetbuttcap%
\pgfsetmiterjoin%
\definecolor{currentfill}{rgb}{0.501961,0.501961,0.501961}%
\pgfsetfillcolor{currentfill}%
\pgfsetfillopacity{0.700000}%
\pgfsetlinewidth{0.000000pt}%
\definecolor{currentstroke}{rgb}{0.000000,0.000000,0.000000}%
\pgfsetstrokecolor{currentstroke}%
\pgfsetstrokeopacity{0.700000}%
\pgfsetdash{}{0pt}%
\pgfpathmoveto{\pgfqpoint{3.336115in}{0.550000in}}%
\pgfpathlineto{\pgfqpoint{3.363636in}{0.550000in}}%
\pgfpathlineto{\pgfqpoint{3.363636in}{0.709408in}}%
\pgfpathlineto{\pgfqpoint{3.336115in}{0.709408in}}%
\pgfpathlineto{\pgfqpoint{3.336115in}{0.550000in}}%
\pgfpathclose%
\pgfusepath{fill}%
\end{pgfscope}%
\begin{pgfscope}%
\pgfpathrectangle{\pgfqpoint{1.250000in}{0.550000in}}{\pgfqpoint{7.750000in}{3.850000in}}%
\pgfusepath{clip}%
\pgfsetbuttcap%
\pgfsetmiterjoin%
\definecolor{currentfill}{rgb}{0.501961,0.501961,0.501961}%
\pgfsetfillcolor{currentfill}%
\pgfsetfillopacity{0.700000}%
\pgfsetlinewidth{0.000000pt}%
\definecolor{currentstroke}{rgb}{0.000000,0.000000,0.000000}%
\pgfsetstrokecolor{currentstroke}%
\pgfsetstrokeopacity{0.700000}%
\pgfsetdash{}{0pt}%
\pgfpathmoveto{\pgfqpoint{3.363636in}{0.550000in}}%
\pgfpathlineto{\pgfqpoint{3.391158in}{0.550000in}}%
\pgfpathlineto{\pgfqpoint{3.391158in}{0.692063in}}%
\pgfpathlineto{\pgfqpoint{3.363636in}{0.692063in}}%
\pgfpathlineto{\pgfqpoint{3.363636in}{0.550000in}}%
\pgfpathclose%
\pgfusepath{fill}%
\end{pgfscope}%
\begin{pgfscope}%
\pgfpathrectangle{\pgfqpoint{1.250000in}{0.550000in}}{\pgfqpoint{7.750000in}{3.850000in}}%
\pgfusepath{clip}%
\pgfsetbuttcap%
\pgfsetmiterjoin%
\definecolor{currentfill}{rgb}{0.501961,0.501961,0.501961}%
\pgfsetfillcolor{currentfill}%
\pgfsetfillopacity{0.700000}%
\pgfsetlinewidth{0.000000pt}%
\definecolor{currentstroke}{rgb}{0.000000,0.000000,0.000000}%
\pgfsetstrokecolor{currentstroke}%
\pgfsetstrokeopacity{0.700000}%
\pgfsetdash{}{0pt}%
\pgfpathmoveto{\pgfqpoint{3.391158in}{0.550000in}}%
\pgfpathlineto{\pgfqpoint{3.418679in}{0.550000in}}%
\pgfpathlineto{\pgfqpoint{3.418679in}{0.692063in}}%
\pgfpathlineto{\pgfqpoint{3.391158in}{0.692063in}}%
\pgfpathlineto{\pgfqpoint{3.391158in}{0.550000in}}%
\pgfpathclose%
\pgfusepath{fill}%
\end{pgfscope}%
\begin{pgfscope}%
\pgfpathrectangle{\pgfqpoint{1.250000in}{0.550000in}}{\pgfqpoint{7.750000in}{3.850000in}}%
\pgfusepath{clip}%
\pgfsetbuttcap%
\pgfsetmiterjoin%
\definecolor{currentfill}{rgb}{0.501961,0.501961,0.501961}%
\pgfsetfillcolor{currentfill}%
\pgfsetfillopacity{0.700000}%
\pgfsetlinewidth{0.000000pt}%
\definecolor{currentstroke}{rgb}{0.000000,0.000000,0.000000}%
\pgfsetstrokecolor{currentstroke}%
\pgfsetstrokeopacity{0.700000}%
\pgfsetdash{}{0pt}%
\pgfpathmoveto{\pgfqpoint{3.418679in}{0.550000in}}%
\pgfpathlineto{\pgfqpoint{3.446200in}{0.550000in}}%
\pgfpathlineto{\pgfqpoint{3.446200in}{0.684905in}}%
\pgfpathlineto{\pgfqpoint{3.418679in}{0.684905in}}%
\pgfpathlineto{\pgfqpoint{3.418679in}{0.550000in}}%
\pgfpathclose%
\pgfusepath{fill}%
\end{pgfscope}%
\begin{pgfscope}%
\pgfpathrectangle{\pgfqpoint{1.250000in}{0.550000in}}{\pgfqpoint{7.750000in}{3.850000in}}%
\pgfusepath{clip}%
\pgfsetbuttcap%
\pgfsetmiterjoin%
\definecolor{currentfill}{rgb}{0.501961,0.501961,0.501961}%
\pgfsetfillcolor{currentfill}%
\pgfsetfillopacity{0.700000}%
\pgfsetlinewidth{0.000000pt}%
\definecolor{currentstroke}{rgb}{0.000000,0.000000,0.000000}%
\pgfsetstrokecolor{currentstroke}%
\pgfsetstrokeopacity{0.700000}%
\pgfsetdash{}{0pt}%
\pgfpathmoveto{\pgfqpoint{3.446200in}{0.550000in}}%
\pgfpathlineto{\pgfqpoint{3.473722in}{0.550000in}}%
\pgfpathlineto{\pgfqpoint{3.473722in}{0.688209in}}%
\pgfpathlineto{\pgfqpoint{3.446200in}{0.688209in}}%
\pgfpathlineto{\pgfqpoint{3.446200in}{0.550000in}}%
\pgfpathclose%
\pgfusepath{fill}%
\end{pgfscope}%
\begin{pgfscope}%
\pgfpathrectangle{\pgfqpoint{1.250000in}{0.550000in}}{\pgfqpoint{7.750000in}{3.850000in}}%
\pgfusepath{clip}%
\pgfsetbuttcap%
\pgfsetmiterjoin%
\definecolor{currentfill}{rgb}{0.501961,0.501961,0.501961}%
\pgfsetfillcolor{currentfill}%
\pgfsetfillopacity{0.700000}%
\pgfsetlinewidth{0.000000pt}%
\definecolor{currentstroke}{rgb}{0.000000,0.000000,0.000000}%
\pgfsetstrokecolor{currentstroke}%
\pgfsetstrokeopacity{0.700000}%
\pgfsetdash{}{0pt}%
\pgfpathmoveto{\pgfqpoint{3.473722in}{0.550000in}}%
\pgfpathlineto{\pgfqpoint{3.501243in}{0.550000in}}%
\pgfpathlineto{\pgfqpoint{3.501243in}{0.700048in}}%
\pgfpathlineto{\pgfqpoint{3.473722in}{0.700048in}}%
\pgfpathlineto{\pgfqpoint{3.473722in}{0.550000in}}%
\pgfpathclose%
\pgfusepath{fill}%
\end{pgfscope}%
\begin{pgfscope}%
\pgfpathrectangle{\pgfqpoint{1.250000in}{0.550000in}}{\pgfqpoint{7.750000in}{3.850000in}}%
\pgfusepath{clip}%
\pgfsetbuttcap%
\pgfsetmiterjoin%
\definecolor{currentfill}{rgb}{0.501961,0.501961,0.501961}%
\pgfsetfillcolor{currentfill}%
\pgfsetfillopacity{0.700000}%
\pgfsetlinewidth{0.000000pt}%
\definecolor{currentstroke}{rgb}{0.000000,0.000000,0.000000}%
\pgfsetstrokecolor{currentstroke}%
\pgfsetstrokeopacity{0.700000}%
\pgfsetdash{}{0pt}%
\pgfpathmoveto{\pgfqpoint{3.501243in}{0.550000in}}%
\pgfpathlineto{\pgfqpoint{3.528764in}{0.550000in}}%
\pgfpathlineto{\pgfqpoint{3.528764in}{0.694817in}}%
\pgfpathlineto{\pgfqpoint{3.501243in}{0.694817in}}%
\pgfpathlineto{\pgfqpoint{3.501243in}{0.550000in}}%
\pgfpathclose%
\pgfusepath{fill}%
\end{pgfscope}%
\begin{pgfscope}%
\pgfpathrectangle{\pgfqpoint{1.250000in}{0.550000in}}{\pgfqpoint{7.750000in}{3.850000in}}%
\pgfusepath{clip}%
\pgfsetbuttcap%
\pgfsetmiterjoin%
\definecolor{currentfill}{rgb}{0.501961,0.501961,0.501961}%
\pgfsetfillcolor{currentfill}%
\pgfsetfillopacity{0.700000}%
\pgfsetlinewidth{0.000000pt}%
\definecolor{currentstroke}{rgb}{0.000000,0.000000,0.000000}%
\pgfsetstrokecolor{currentstroke}%
\pgfsetstrokeopacity{0.700000}%
\pgfsetdash{}{0pt}%
\pgfpathmoveto{\pgfqpoint{3.528764in}{0.550000in}}%
\pgfpathlineto{\pgfqpoint{3.556286in}{0.550000in}}%
\pgfpathlineto{\pgfqpoint{3.556286in}{0.700048in}}%
\pgfpathlineto{\pgfqpoint{3.528764in}{0.700048in}}%
\pgfpathlineto{\pgfqpoint{3.528764in}{0.550000in}}%
\pgfpathclose%
\pgfusepath{fill}%
\end{pgfscope}%
\begin{pgfscope}%
\pgfpathrectangle{\pgfqpoint{1.250000in}{0.550000in}}{\pgfqpoint{7.750000in}{3.850000in}}%
\pgfusepath{clip}%
\pgfsetbuttcap%
\pgfsetmiterjoin%
\definecolor{currentfill}{rgb}{0.501961,0.501961,0.501961}%
\pgfsetfillcolor{currentfill}%
\pgfsetfillopacity{0.700000}%
\pgfsetlinewidth{0.000000pt}%
\definecolor{currentstroke}{rgb}{0.000000,0.000000,0.000000}%
\pgfsetstrokecolor{currentstroke}%
\pgfsetstrokeopacity{0.700000}%
\pgfsetdash{}{0pt}%
\pgfpathmoveto{\pgfqpoint{3.556286in}{0.550000in}}%
\pgfpathlineto{\pgfqpoint{3.583807in}{0.550000in}}%
\pgfpathlineto{\pgfqpoint{3.583807in}{0.697294in}}%
\pgfpathlineto{\pgfqpoint{3.556286in}{0.697294in}}%
\pgfpathlineto{\pgfqpoint{3.556286in}{0.550000in}}%
\pgfpathclose%
\pgfusepath{fill}%
\end{pgfscope}%
\begin{pgfscope}%
\pgfpathrectangle{\pgfqpoint{1.250000in}{0.550000in}}{\pgfqpoint{7.750000in}{3.850000in}}%
\pgfusepath{clip}%
\pgfsetbuttcap%
\pgfsetmiterjoin%
\definecolor{currentfill}{rgb}{0.501961,0.501961,0.501961}%
\pgfsetfillcolor{currentfill}%
\pgfsetfillopacity{0.700000}%
\pgfsetlinewidth{0.000000pt}%
\definecolor{currentstroke}{rgb}{0.000000,0.000000,0.000000}%
\pgfsetstrokecolor{currentstroke}%
\pgfsetstrokeopacity{0.700000}%
\pgfsetdash{}{0pt}%
\pgfpathmoveto{\pgfqpoint{3.583807in}{0.550000in}}%
\pgfpathlineto{\pgfqpoint{3.611328in}{0.550000in}}%
\pgfpathlineto{\pgfqpoint{3.611328in}{0.687108in}}%
\pgfpathlineto{\pgfqpoint{3.583807in}{0.687108in}}%
\pgfpathlineto{\pgfqpoint{3.583807in}{0.550000in}}%
\pgfpathclose%
\pgfusepath{fill}%
\end{pgfscope}%
\begin{pgfscope}%
\pgfpathrectangle{\pgfqpoint{1.250000in}{0.550000in}}{\pgfqpoint{7.750000in}{3.850000in}}%
\pgfusepath{clip}%
\pgfsetbuttcap%
\pgfsetmiterjoin%
\definecolor{currentfill}{rgb}{0.501961,0.501961,0.501961}%
\pgfsetfillcolor{currentfill}%
\pgfsetfillopacity{0.700000}%
\pgfsetlinewidth{0.000000pt}%
\definecolor{currentstroke}{rgb}{0.000000,0.000000,0.000000}%
\pgfsetstrokecolor{currentstroke}%
\pgfsetstrokeopacity{0.700000}%
\pgfsetdash{}{0pt}%
\pgfpathmoveto{\pgfqpoint{3.611328in}{0.550000in}}%
\pgfpathlineto{\pgfqpoint{3.638849in}{0.550000in}}%
\pgfpathlineto{\pgfqpoint{3.638849in}{0.684355in}}%
\pgfpathlineto{\pgfqpoint{3.611328in}{0.684355in}}%
\pgfpathlineto{\pgfqpoint{3.611328in}{0.550000in}}%
\pgfpathclose%
\pgfusepath{fill}%
\end{pgfscope}%
\begin{pgfscope}%
\pgfpathrectangle{\pgfqpoint{1.250000in}{0.550000in}}{\pgfqpoint{7.750000in}{3.850000in}}%
\pgfusepath{clip}%
\pgfsetbuttcap%
\pgfsetmiterjoin%
\definecolor{currentfill}{rgb}{0.501961,0.501961,0.501961}%
\pgfsetfillcolor{currentfill}%
\pgfsetfillopacity{0.700000}%
\pgfsetlinewidth{0.000000pt}%
\definecolor{currentstroke}{rgb}{0.000000,0.000000,0.000000}%
\pgfsetstrokecolor{currentstroke}%
\pgfsetstrokeopacity{0.700000}%
\pgfsetdash{}{0pt}%
\pgfpathmoveto{\pgfqpoint{3.638849in}{0.550000in}}%
\pgfpathlineto{\pgfqpoint{3.666371in}{0.550000in}}%
\pgfpathlineto{\pgfqpoint{3.666371in}{0.683804in}}%
\pgfpathlineto{\pgfqpoint{3.638849in}{0.683804in}}%
\pgfpathlineto{\pgfqpoint{3.638849in}{0.550000in}}%
\pgfpathclose%
\pgfusepath{fill}%
\end{pgfscope}%
\begin{pgfscope}%
\pgfpathrectangle{\pgfqpoint{1.250000in}{0.550000in}}{\pgfqpoint{7.750000in}{3.850000in}}%
\pgfusepath{clip}%
\pgfsetbuttcap%
\pgfsetmiterjoin%
\definecolor{currentfill}{rgb}{0.501961,0.501961,0.501961}%
\pgfsetfillcolor{currentfill}%
\pgfsetfillopacity{0.700000}%
\pgfsetlinewidth{0.000000pt}%
\definecolor{currentstroke}{rgb}{0.000000,0.000000,0.000000}%
\pgfsetstrokecolor{currentstroke}%
\pgfsetstrokeopacity{0.700000}%
\pgfsetdash{}{0pt}%
\pgfpathmoveto{\pgfqpoint{3.666371in}{0.550000in}}%
\pgfpathlineto{\pgfqpoint{3.693892in}{0.550000in}}%
\pgfpathlineto{\pgfqpoint{3.693892in}{0.672241in}}%
\pgfpathlineto{\pgfqpoint{3.666371in}{0.672241in}}%
\pgfpathlineto{\pgfqpoint{3.666371in}{0.550000in}}%
\pgfpathclose%
\pgfusepath{fill}%
\end{pgfscope}%
\begin{pgfscope}%
\pgfpathrectangle{\pgfqpoint{1.250000in}{0.550000in}}{\pgfqpoint{7.750000in}{3.850000in}}%
\pgfusepath{clip}%
\pgfsetbuttcap%
\pgfsetmiterjoin%
\definecolor{currentfill}{rgb}{0.501961,0.501961,0.501961}%
\pgfsetfillcolor{currentfill}%
\pgfsetfillopacity{0.700000}%
\pgfsetlinewidth{0.000000pt}%
\definecolor{currentstroke}{rgb}{0.000000,0.000000,0.000000}%
\pgfsetstrokecolor{currentstroke}%
\pgfsetstrokeopacity{0.700000}%
\pgfsetdash{}{0pt}%
\pgfpathmoveto{\pgfqpoint{3.693892in}{0.550000in}}%
\pgfpathlineto{\pgfqpoint{3.721413in}{0.550000in}}%
\pgfpathlineto{\pgfqpoint{3.721413in}{0.679399in}}%
\pgfpathlineto{\pgfqpoint{3.693892in}{0.679399in}}%
\pgfpathlineto{\pgfqpoint{3.693892in}{0.550000in}}%
\pgfpathclose%
\pgfusepath{fill}%
\end{pgfscope}%
\begin{pgfscope}%
\pgfpathrectangle{\pgfqpoint{1.250000in}{0.550000in}}{\pgfqpoint{7.750000in}{3.850000in}}%
\pgfusepath{clip}%
\pgfsetbuttcap%
\pgfsetmiterjoin%
\definecolor{currentfill}{rgb}{0.501961,0.501961,0.501961}%
\pgfsetfillcolor{currentfill}%
\pgfsetfillopacity{0.700000}%
\pgfsetlinewidth{0.000000pt}%
\definecolor{currentstroke}{rgb}{0.000000,0.000000,0.000000}%
\pgfsetstrokecolor{currentstroke}%
\pgfsetstrokeopacity{0.700000}%
\pgfsetdash{}{0pt}%
\pgfpathmoveto{\pgfqpoint{3.721413in}{0.550000in}}%
\pgfpathlineto{\pgfqpoint{3.748935in}{0.550000in}}%
\pgfpathlineto{\pgfqpoint{3.748935in}{0.665082in}}%
\pgfpathlineto{\pgfqpoint{3.721413in}{0.665082in}}%
\pgfpathlineto{\pgfqpoint{3.721413in}{0.550000in}}%
\pgfpathclose%
\pgfusepath{fill}%
\end{pgfscope}%
\begin{pgfscope}%
\pgfpathrectangle{\pgfqpoint{1.250000in}{0.550000in}}{\pgfqpoint{7.750000in}{3.850000in}}%
\pgfusepath{clip}%
\pgfsetbuttcap%
\pgfsetmiterjoin%
\definecolor{currentfill}{rgb}{0.501961,0.501961,0.501961}%
\pgfsetfillcolor{currentfill}%
\pgfsetfillopacity{0.700000}%
\pgfsetlinewidth{0.000000pt}%
\definecolor{currentstroke}{rgb}{0.000000,0.000000,0.000000}%
\pgfsetstrokecolor{currentstroke}%
\pgfsetstrokeopacity{0.700000}%
\pgfsetdash{}{0pt}%
\pgfpathmoveto{\pgfqpoint{3.748935in}{0.550000in}}%
\pgfpathlineto{\pgfqpoint{3.776456in}{0.550000in}}%
\pgfpathlineto{\pgfqpoint{3.776456in}{0.667010in}}%
\pgfpathlineto{\pgfqpoint{3.748935in}{0.667010in}}%
\pgfpathlineto{\pgfqpoint{3.748935in}{0.550000in}}%
\pgfpathclose%
\pgfusepath{fill}%
\end{pgfscope}%
\begin{pgfscope}%
\pgfpathrectangle{\pgfqpoint{1.250000in}{0.550000in}}{\pgfqpoint{7.750000in}{3.850000in}}%
\pgfusepath{clip}%
\pgfsetbuttcap%
\pgfsetmiterjoin%
\definecolor{currentfill}{rgb}{0.501961,0.501961,0.501961}%
\pgfsetfillcolor{currentfill}%
\pgfsetfillopacity{0.700000}%
\pgfsetlinewidth{0.000000pt}%
\definecolor{currentstroke}{rgb}{0.000000,0.000000,0.000000}%
\pgfsetstrokecolor{currentstroke}%
\pgfsetstrokeopacity{0.700000}%
\pgfsetdash{}{0pt}%
\pgfpathmoveto{\pgfqpoint{3.776456in}{0.550000in}}%
\pgfpathlineto{\pgfqpoint{3.803977in}{0.550000in}}%
\pgfpathlineto{\pgfqpoint{3.803977in}{0.660402in}}%
\pgfpathlineto{\pgfqpoint{3.776456in}{0.660402in}}%
\pgfpathlineto{\pgfqpoint{3.776456in}{0.550000in}}%
\pgfpathclose%
\pgfusepath{fill}%
\end{pgfscope}%
\begin{pgfscope}%
\pgfpathrectangle{\pgfqpoint{1.250000in}{0.550000in}}{\pgfqpoint{7.750000in}{3.850000in}}%
\pgfusepath{clip}%
\pgfsetbuttcap%
\pgfsetmiterjoin%
\definecolor{currentfill}{rgb}{0.501961,0.501961,0.501961}%
\pgfsetfillcolor{currentfill}%
\pgfsetfillopacity{0.700000}%
\pgfsetlinewidth{0.000000pt}%
\definecolor{currentstroke}{rgb}{0.000000,0.000000,0.000000}%
\pgfsetstrokecolor{currentstroke}%
\pgfsetstrokeopacity{0.700000}%
\pgfsetdash{}{0pt}%
\pgfpathmoveto{\pgfqpoint{3.803977in}{0.550000in}}%
\pgfpathlineto{\pgfqpoint{3.831499in}{0.550000in}}%
\pgfpathlineto{\pgfqpoint{3.831499in}{0.651867in}}%
\pgfpathlineto{\pgfqpoint{3.803977in}{0.651867in}}%
\pgfpathlineto{\pgfqpoint{3.803977in}{0.550000in}}%
\pgfpathclose%
\pgfusepath{fill}%
\end{pgfscope}%
\begin{pgfscope}%
\pgfpathrectangle{\pgfqpoint{1.250000in}{0.550000in}}{\pgfqpoint{7.750000in}{3.850000in}}%
\pgfusepath{clip}%
\pgfsetbuttcap%
\pgfsetmiterjoin%
\definecolor{currentfill}{rgb}{0.501961,0.501961,0.501961}%
\pgfsetfillcolor{currentfill}%
\pgfsetfillopacity{0.700000}%
\pgfsetlinewidth{0.000000pt}%
\definecolor{currentstroke}{rgb}{0.000000,0.000000,0.000000}%
\pgfsetstrokecolor{currentstroke}%
\pgfsetstrokeopacity{0.700000}%
\pgfsetdash{}{0pt}%
\pgfpathmoveto{\pgfqpoint{3.831499in}{0.550000in}}%
\pgfpathlineto{\pgfqpoint{3.859020in}{0.550000in}}%
\pgfpathlineto{\pgfqpoint{3.859020in}{0.648288in}}%
\pgfpathlineto{\pgfqpoint{3.831499in}{0.648288in}}%
\pgfpathlineto{\pgfqpoint{3.831499in}{0.550000in}}%
\pgfpathclose%
\pgfusepath{fill}%
\end{pgfscope}%
\begin{pgfscope}%
\pgfpathrectangle{\pgfqpoint{1.250000in}{0.550000in}}{\pgfqpoint{7.750000in}{3.850000in}}%
\pgfusepath{clip}%
\pgfsetbuttcap%
\pgfsetmiterjoin%
\definecolor{currentfill}{rgb}{0.501961,0.501961,0.501961}%
\pgfsetfillcolor{currentfill}%
\pgfsetfillopacity{0.700000}%
\pgfsetlinewidth{0.000000pt}%
\definecolor{currentstroke}{rgb}{0.000000,0.000000,0.000000}%
\pgfsetstrokecolor{currentstroke}%
\pgfsetstrokeopacity{0.700000}%
\pgfsetdash{}{0pt}%
\pgfpathmoveto{\pgfqpoint{3.859020in}{0.550000in}}%
\pgfpathlineto{\pgfqpoint{3.886541in}{0.550000in}}%
\pgfpathlineto{\pgfqpoint{3.886541in}{0.647462in}}%
\pgfpathlineto{\pgfqpoint{3.859020in}{0.647462in}}%
\pgfpathlineto{\pgfqpoint{3.859020in}{0.550000in}}%
\pgfpathclose%
\pgfusepath{fill}%
\end{pgfscope}%
\begin{pgfscope}%
\pgfpathrectangle{\pgfqpoint{1.250000in}{0.550000in}}{\pgfqpoint{7.750000in}{3.850000in}}%
\pgfusepath{clip}%
\pgfsetbuttcap%
\pgfsetmiterjoin%
\definecolor{currentfill}{rgb}{0.501961,0.501961,0.501961}%
\pgfsetfillcolor{currentfill}%
\pgfsetfillopacity{0.700000}%
\pgfsetlinewidth{0.000000pt}%
\definecolor{currentstroke}{rgb}{0.000000,0.000000,0.000000}%
\pgfsetstrokecolor{currentstroke}%
\pgfsetstrokeopacity{0.700000}%
\pgfsetdash{}{0pt}%
\pgfpathmoveto{\pgfqpoint{3.886541in}{0.550000in}}%
\pgfpathlineto{\pgfqpoint{3.914062in}{0.550000in}}%
\pgfpathlineto{\pgfqpoint{3.914062in}{0.654620in}}%
\pgfpathlineto{\pgfqpoint{3.886541in}{0.654620in}}%
\pgfpathlineto{\pgfqpoint{3.886541in}{0.550000in}}%
\pgfpathclose%
\pgfusepath{fill}%
\end{pgfscope}%
\begin{pgfscope}%
\pgfpathrectangle{\pgfqpoint{1.250000in}{0.550000in}}{\pgfqpoint{7.750000in}{3.850000in}}%
\pgfusepath{clip}%
\pgfsetbuttcap%
\pgfsetmiterjoin%
\definecolor{currentfill}{rgb}{0.501961,0.501961,0.501961}%
\pgfsetfillcolor{currentfill}%
\pgfsetfillopacity{0.700000}%
\pgfsetlinewidth{0.000000pt}%
\definecolor{currentstroke}{rgb}{0.000000,0.000000,0.000000}%
\pgfsetstrokecolor{currentstroke}%
\pgfsetstrokeopacity{0.700000}%
\pgfsetdash{}{0pt}%
\pgfpathmoveto{\pgfqpoint{3.914062in}{0.550000in}}%
\pgfpathlineto{\pgfqpoint{3.941584in}{0.550000in}}%
\pgfpathlineto{\pgfqpoint{3.941584in}{0.651041in}}%
\pgfpathlineto{\pgfqpoint{3.914062in}{0.651041in}}%
\pgfpathlineto{\pgfqpoint{3.914062in}{0.550000in}}%
\pgfpathclose%
\pgfusepath{fill}%
\end{pgfscope}%
\begin{pgfscope}%
\pgfpathrectangle{\pgfqpoint{1.250000in}{0.550000in}}{\pgfqpoint{7.750000in}{3.850000in}}%
\pgfusepath{clip}%
\pgfsetbuttcap%
\pgfsetmiterjoin%
\definecolor{currentfill}{rgb}{0.501961,0.501961,0.501961}%
\pgfsetfillcolor{currentfill}%
\pgfsetfillopacity{0.700000}%
\pgfsetlinewidth{0.000000pt}%
\definecolor{currentstroke}{rgb}{0.000000,0.000000,0.000000}%
\pgfsetstrokecolor{currentstroke}%
\pgfsetstrokeopacity{0.700000}%
\pgfsetdash{}{0pt}%
\pgfpathmoveto{\pgfqpoint{3.941584in}{0.550000in}}%
\pgfpathlineto{\pgfqpoint{3.969105in}{0.550000in}}%
\pgfpathlineto{\pgfqpoint{3.969105in}{0.655722in}}%
\pgfpathlineto{\pgfqpoint{3.941584in}{0.655722in}}%
\pgfpathlineto{\pgfqpoint{3.941584in}{0.550000in}}%
\pgfpathclose%
\pgfusepath{fill}%
\end{pgfscope}%
\begin{pgfscope}%
\pgfpathrectangle{\pgfqpoint{1.250000in}{0.550000in}}{\pgfqpoint{7.750000in}{3.850000in}}%
\pgfusepath{clip}%
\pgfsetbuttcap%
\pgfsetmiterjoin%
\definecolor{currentfill}{rgb}{0.501961,0.501961,0.501961}%
\pgfsetfillcolor{currentfill}%
\pgfsetfillopacity{0.700000}%
\pgfsetlinewidth{0.000000pt}%
\definecolor{currentstroke}{rgb}{0.000000,0.000000,0.000000}%
\pgfsetstrokecolor{currentstroke}%
\pgfsetstrokeopacity{0.700000}%
\pgfsetdash{}{0pt}%
\pgfpathmoveto{\pgfqpoint{3.969105in}{0.550000in}}%
\pgfpathlineto{\pgfqpoint{3.996626in}{0.550000in}}%
\pgfpathlineto{\pgfqpoint{3.996626in}{0.654070in}}%
\pgfpathlineto{\pgfqpoint{3.969105in}{0.654070in}}%
\pgfpathlineto{\pgfqpoint{3.969105in}{0.550000in}}%
\pgfpathclose%
\pgfusepath{fill}%
\end{pgfscope}%
\begin{pgfscope}%
\pgfpathrectangle{\pgfqpoint{1.250000in}{0.550000in}}{\pgfqpoint{7.750000in}{3.850000in}}%
\pgfusepath{clip}%
\pgfsetbuttcap%
\pgfsetmiterjoin%
\definecolor{currentfill}{rgb}{0.501961,0.501961,0.501961}%
\pgfsetfillcolor{currentfill}%
\pgfsetfillopacity{0.700000}%
\pgfsetlinewidth{0.000000pt}%
\definecolor{currentstroke}{rgb}{0.000000,0.000000,0.000000}%
\pgfsetstrokecolor{currentstroke}%
\pgfsetstrokeopacity{0.700000}%
\pgfsetdash{}{0pt}%
\pgfpathmoveto{\pgfqpoint{3.996626in}{0.550000in}}%
\pgfpathlineto{\pgfqpoint{4.024148in}{0.550000in}}%
\pgfpathlineto{\pgfqpoint{4.024148in}{0.642782in}}%
\pgfpathlineto{\pgfqpoint{3.996626in}{0.642782in}}%
\pgfpathlineto{\pgfqpoint{3.996626in}{0.550000in}}%
\pgfpathclose%
\pgfusepath{fill}%
\end{pgfscope}%
\begin{pgfscope}%
\pgfpathrectangle{\pgfqpoint{1.250000in}{0.550000in}}{\pgfqpoint{7.750000in}{3.850000in}}%
\pgfusepath{clip}%
\pgfsetbuttcap%
\pgfsetmiterjoin%
\definecolor{currentfill}{rgb}{0.501961,0.501961,0.501961}%
\pgfsetfillcolor{currentfill}%
\pgfsetfillopacity{0.700000}%
\pgfsetlinewidth{0.000000pt}%
\definecolor{currentstroke}{rgb}{0.000000,0.000000,0.000000}%
\pgfsetstrokecolor{currentstroke}%
\pgfsetstrokeopacity{0.700000}%
\pgfsetdash{}{0pt}%
\pgfpathmoveto{\pgfqpoint{4.024148in}{0.550000in}}%
\pgfpathlineto{\pgfqpoint{4.051669in}{0.550000in}}%
\pgfpathlineto{\pgfqpoint{4.051669in}{0.647462in}}%
\pgfpathlineto{\pgfqpoint{4.024148in}{0.647462in}}%
\pgfpathlineto{\pgfqpoint{4.024148in}{0.550000in}}%
\pgfpathclose%
\pgfusepath{fill}%
\end{pgfscope}%
\begin{pgfscope}%
\pgfpathrectangle{\pgfqpoint{1.250000in}{0.550000in}}{\pgfqpoint{7.750000in}{3.850000in}}%
\pgfusepath{clip}%
\pgfsetbuttcap%
\pgfsetmiterjoin%
\definecolor{currentfill}{rgb}{0.501961,0.501961,0.501961}%
\pgfsetfillcolor{currentfill}%
\pgfsetfillopacity{0.700000}%
\pgfsetlinewidth{0.000000pt}%
\definecolor{currentstroke}{rgb}{0.000000,0.000000,0.000000}%
\pgfsetstrokecolor{currentstroke}%
\pgfsetstrokeopacity{0.700000}%
\pgfsetdash{}{0pt}%
\pgfpathmoveto{\pgfqpoint{4.051669in}{0.550000in}}%
\pgfpathlineto{\pgfqpoint{4.079190in}{0.550000in}}%
\pgfpathlineto{\pgfqpoint{4.079190in}{0.633696in}}%
\pgfpathlineto{\pgfqpoint{4.051669in}{0.633696in}}%
\pgfpathlineto{\pgfqpoint{4.051669in}{0.550000in}}%
\pgfpathclose%
\pgfusepath{fill}%
\end{pgfscope}%
\begin{pgfscope}%
\pgfpathrectangle{\pgfqpoint{1.250000in}{0.550000in}}{\pgfqpoint{7.750000in}{3.850000in}}%
\pgfusepath{clip}%
\pgfsetbuttcap%
\pgfsetmiterjoin%
\definecolor{currentfill}{rgb}{0.501961,0.501961,0.501961}%
\pgfsetfillcolor{currentfill}%
\pgfsetfillopacity{0.700000}%
\pgfsetlinewidth{0.000000pt}%
\definecolor{currentstroke}{rgb}{0.000000,0.000000,0.000000}%
\pgfsetstrokecolor{currentstroke}%
\pgfsetstrokeopacity{0.700000}%
\pgfsetdash{}{0pt}%
\pgfpathmoveto{\pgfqpoint{4.079190in}{0.550000in}}%
\pgfpathlineto{\pgfqpoint{4.106712in}{0.550000in}}%
\pgfpathlineto{\pgfqpoint{4.106712in}{0.654620in}}%
\pgfpathlineto{\pgfqpoint{4.079190in}{0.654620in}}%
\pgfpathlineto{\pgfqpoint{4.079190in}{0.550000in}}%
\pgfpathclose%
\pgfusepath{fill}%
\end{pgfscope}%
\begin{pgfscope}%
\pgfpathrectangle{\pgfqpoint{1.250000in}{0.550000in}}{\pgfqpoint{7.750000in}{3.850000in}}%
\pgfusepath{clip}%
\pgfsetbuttcap%
\pgfsetmiterjoin%
\definecolor{currentfill}{rgb}{0.501961,0.501961,0.501961}%
\pgfsetfillcolor{currentfill}%
\pgfsetfillopacity{0.700000}%
\pgfsetlinewidth{0.000000pt}%
\definecolor{currentstroke}{rgb}{0.000000,0.000000,0.000000}%
\pgfsetstrokecolor{currentstroke}%
\pgfsetstrokeopacity{0.700000}%
\pgfsetdash{}{0pt}%
\pgfpathmoveto{\pgfqpoint{4.106712in}{0.550000in}}%
\pgfpathlineto{\pgfqpoint{4.134233in}{0.550000in}}%
\pgfpathlineto{\pgfqpoint{4.134233in}{0.652142in}}%
\pgfpathlineto{\pgfqpoint{4.106712in}{0.652142in}}%
\pgfpathlineto{\pgfqpoint{4.106712in}{0.550000in}}%
\pgfpathclose%
\pgfusepath{fill}%
\end{pgfscope}%
\begin{pgfscope}%
\pgfpathrectangle{\pgfqpoint{1.250000in}{0.550000in}}{\pgfqpoint{7.750000in}{3.850000in}}%
\pgfusepath{clip}%
\pgfsetbuttcap%
\pgfsetmiterjoin%
\definecolor{currentfill}{rgb}{0.501961,0.501961,0.501961}%
\pgfsetfillcolor{currentfill}%
\pgfsetfillopacity{0.700000}%
\pgfsetlinewidth{0.000000pt}%
\definecolor{currentstroke}{rgb}{0.000000,0.000000,0.000000}%
\pgfsetstrokecolor{currentstroke}%
\pgfsetstrokeopacity{0.700000}%
\pgfsetdash{}{0pt}%
\pgfpathmoveto{\pgfqpoint{4.134233in}{0.550000in}}%
\pgfpathlineto{\pgfqpoint{4.161754in}{0.550000in}}%
\pgfpathlineto{\pgfqpoint{4.161754in}{0.646361in}}%
\pgfpathlineto{\pgfqpoint{4.134233in}{0.646361in}}%
\pgfpathlineto{\pgfqpoint{4.134233in}{0.550000in}}%
\pgfpathclose%
\pgfusepath{fill}%
\end{pgfscope}%
\begin{pgfscope}%
\pgfpathrectangle{\pgfqpoint{1.250000in}{0.550000in}}{\pgfqpoint{7.750000in}{3.850000in}}%
\pgfusepath{clip}%
\pgfsetbuttcap%
\pgfsetmiterjoin%
\definecolor{currentfill}{rgb}{0.501961,0.501961,0.501961}%
\pgfsetfillcolor{currentfill}%
\pgfsetfillopacity{0.700000}%
\pgfsetlinewidth{0.000000pt}%
\definecolor{currentstroke}{rgb}{0.000000,0.000000,0.000000}%
\pgfsetstrokecolor{currentstroke}%
\pgfsetstrokeopacity{0.700000}%
\pgfsetdash{}{0pt}%
\pgfpathmoveto{\pgfqpoint{4.161754in}{0.550000in}}%
\pgfpathlineto{\pgfqpoint{4.189276in}{0.550000in}}%
\pgfpathlineto{\pgfqpoint{4.189276in}{0.651592in}}%
\pgfpathlineto{\pgfqpoint{4.161754in}{0.651592in}}%
\pgfpathlineto{\pgfqpoint{4.161754in}{0.550000in}}%
\pgfpathclose%
\pgfusepath{fill}%
\end{pgfscope}%
\begin{pgfscope}%
\pgfpathrectangle{\pgfqpoint{1.250000in}{0.550000in}}{\pgfqpoint{7.750000in}{3.850000in}}%
\pgfusepath{clip}%
\pgfsetbuttcap%
\pgfsetmiterjoin%
\definecolor{currentfill}{rgb}{0.501961,0.501961,0.501961}%
\pgfsetfillcolor{currentfill}%
\pgfsetfillopacity{0.700000}%
\pgfsetlinewidth{0.000000pt}%
\definecolor{currentstroke}{rgb}{0.000000,0.000000,0.000000}%
\pgfsetstrokecolor{currentstroke}%
\pgfsetstrokeopacity{0.700000}%
\pgfsetdash{}{0pt}%
\pgfpathmoveto{\pgfqpoint{4.189276in}{0.550000in}}%
\pgfpathlineto{\pgfqpoint{4.216797in}{0.550000in}}%
\pgfpathlineto{\pgfqpoint{4.216797in}{0.653794in}}%
\pgfpathlineto{\pgfqpoint{4.189276in}{0.653794in}}%
\pgfpathlineto{\pgfqpoint{4.189276in}{0.550000in}}%
\pgfpathclose%
\pgfusepath{fill}%
\end{pgfscope}%
\begin{pgfscope}%
\pgfpathrectangle{\pgfqpoint{1.250000in}{0.550000in}}{\pgfqpoint{7.750000in}{3.850000in}}%
\pgfusepath{clip}%
\pgfsetbuttcap%
\pgfsetmiterjoin%
\definecolor{currentfill}{rgb}{0.501961,0.501961,0.501961}%
\pgfsetfillcolor{currentfill}%
\pgfsetfillopacity{0.700000}%
\pgfsetlinewidth{0.000000pt}%
\definecolor{currentstroke}{rgb}{0.000000,0.000000,0.000000}%
\pgfsetstrokecolor{currentstroke}%
\pgfsetstrokeopacity{0.700000}%
\pgfsetdash{}{0pt}%
\pgfpathmoveto{\pgfqpoint{4.216797in}{0.550000in}}%
\pgfpathlineto{\pgfqpoint{4.244318in}{0.550000in}}%
\pgfpathlineto{\pgfqpoint{4.244318in}{0.641130in}}%
\pgfpathlineto{\pgfqpoint{4.216797in}{0.641130in}}%
\pgfpathlineto{\pgfqpoint{4.216797in}{0.550000in}}%
\pgfpathclose%
\pgfusepath{fill}%
\end{pgfscope}%
\begin{pgfscope}%
\pgfpathrectangle{\pgfqpoint{1.250000in}{0.550000in}}{\pgfqpoint{7.750000in}{3.850000in}}%
\pgfusepath{clip}%
\pgfsetbuttcap%
\pgfsetmiterjoin%
\definecolor{currentfill}{rgb}{0.501961,0.501961,0.501961}%
\pgfsetfillcolor{currentfill}%
\pgfsetfillopacity{0.700000}%
\pgfsetlinewidth{0.000000pt}%
\definecolor{currentstroke}{rgb}{0.000000,0.000000,0.000000}%
\pgfsetstrokecolor{currentstroke}%
\pgfsetstrokeopacity{0.700000}%
\pgfsetdash{}{0pt}%
\pgfpathmoveto{\pgfqpoint{4.244318in}{0.550000in}}%
\pgfpathlineto{\pgfqpoint{4.271839in}{0.550000in}}%
\pgfpathlineto{\pgfqpoint{4.271839in}{0.650491in}}%
\pgfpathlineto{\pgfqpoint{4.244318in}{0.650491in}}%
\pgfpathlineto{\pgfqpoint{4.244318in}{0.550000in}}%
\pgfpathclose%
\pgfusepath{fill}%
\end{pgfscope}%
\begin{pgfscope}%
\pgfpathrectangle{\pgfqpoint{1.250000in}{0.550000in}}{\pgfqpoint{7.750000in}{3.850000in}}%
\pgfusepath{clip}%
\pgfsetbuttcap%
\pgfsetmiterjoin%
\definecolor{currentfill}{rgb}{0.501961,0.501961,0.501961}%
\pgfsetfillcolor{currentfill}%
\pgfsetfillopacity{0.700000}%
\pgfsetlinewidth{0.000000pt}%
\definecolor{currentstroke}{rgb}{0.000000,0.000000,0.000000}%
\pgfsetstrokecolor{currentstroke}%
\pgfsetstrokeopacity{0.700000}%
\pgfsetdash{}{0pt}%
\pgfpathmoveto{\pgfqpoint{4.271839in}{0.550000in}}%
\pgfpathlineto{\pgfqpoint{4.299361in}{0.550000in}}%
\pgfpathlineto{\pgfqpoint{4.299361in}{0.636174in}}%
\pgfpathlineto{\pgfqpoint{4.271839in}{0.636174in}}%
\pgfpathlineto{\pgfqpoint{4.271839in}{0.550000in}}%
\pgfpathclose%
\pgfusepath{fill}%
\end{pgfscope}%
\begin{pgfscope}%
\pgfpathrectangle{\pgfqpoint{1.250000in}{0.550000in}}{\pgfqpoint{7.750000in}{3.850000in}}%
\pgfusepath{clip}%
\pgfsetbuttcap%
\pgfsetmiterjoin%
\definecolor{currentfill}{rgb}{0.501961,0.501961,0.501961}%
\pgfsetfillcolor{currentfill}%
\pgfsetfillopacity{0.700000}%
\pgfsetlinewidth{0.000000pt}%
\definecolor{currentstroke}{rgb}{0.000000,0.000000,0.000000}%
\pgfsetstrokecolor{currentstroke}%
\pgfsetstrokeopacity{0.700000}%
\pgfsetdash{}{0pt}%
\pgfpathmoveto{\pgfqpoint{4.299361in}{0.550000in}}%
\pgfpathlineto{\pgfqpoint{4.326882in}{0.550000in}}%
\pgfpathlineto{\pgfqpoint{4.326882in}{0.643332in}}%
\pgfpathlineto{\pgfqpoint{4.299361in}{0.643332in}}%
\pgfpathlineto{\pgfqpoint{4.299361in}{0.550000in}}%
\pgfpathclose%
\pgfusepath{fill}%
\end{pgfscope}%
\begin{pgfscope}%
\pgfpathrectangle{\pgfqpoint{1.250000in}{0.550000in}}{\pgfqpoint{7.750000in}{3.850000in}}%
\pgfusepath{clip}%
\pgfsetbuttcap%
\pgfsetmiterjoin%
\definecolor{currentfill}{rgb}{0.501961,0.501961,0.501961}%
\pgfsetfillcolor{currentfill}%
\pgfsetfillopacity{0.700000}%
\pgfsetlinewidth{0.000000pt}%
\definecolor{currentstroke}{rgb}{0.000000,0.000000,0.000000}%
\pgfsetstrokecolor{currentstroke}%
\pgfsetstrokeopacity{0.700000}%
\pgfsetdash{}{0pt}%
\pgfpathmoveto{\pgfqpoint{4.326882in}{0.550000in}}%
\pgfpathlineto{\pgfqpoint{4.354403in}{0.550000in}}%
\pgfpathlineto{\pgfqpoint{4.354403in}{0.651317in}}%
\pgfpathlineto{\pgfqpoint{4.326882in}{0.651317in}}%
\pgfpathlineto{\pgfqpoint{4.326882in}{0.550000in}}%
\pgfpathclose%
\pgfusepath{fill}%
\end{pgfscope}%
\begin{pgfscope}%
\pgfpathrectangle{\pgfqpoint{1.250000in}{0.550000in}}{\pgfqpoint{7.750000in}{3.850000in}}%
\pgfusepath{clip}%
\pgfsetbuttcap%
\pgfsetmiterjoin%
\definecolor{currentfill}{rgb}{0.501961,0.501961,0.501961}%
\pgfsetfillcolor{currentfill}%
\pgfsetfillopacity{0.700000}%
\pgfsetlinewidth{0.000000pt}%
\definecolor{currentstroke}{rgb}{0.000000,0.000000,0.000000}%
\pgfsetstrokecolor{currentstroke}%
\pgfsetstrokeopacity{0.700000}%
\pgfsetdash{}{0pt}%
\pgfpathmoveto{\pgfqpoint{4.354403in}{0.550000in}}%
\pgfpathlineto{\pgfqpoint{4.381925in}{0.550000in}}%
\pgfpathlineto{\pgfqpoint{4.381925in}{0.645810in}}%
\pgfpathlineto{\pgfqpoint{4.354403in}{0.645810in}}%
\pgfpathlineto{\pgfqpoint{4.354403in}{0.550000in}}%
\pgfpathclose%
\pgfusepath{fill}%
\end{pgfscope}%
\begin{pgfscope}%
\pgfpathrectangle{\pgfqpoint{1.250000in}{0.550000in}}{\pgfqpoint{7.750000in}{3.850000in}}%
\pgfusepath{clip}%
\pgfsetbuttcap%
\pgfsetmiterjoin%
\definecolor{currentfill}{rgb}{0.501961,0.501961,0.501961}%
\pgfsetfillcolor{currentfill}%
\pgfsetfillopacity{0.700000}%
\pgfsetlinewidth{0.000000pt}%
\definecolor{currentstroke}{rgb}{0.000000,0.000000,0.000000}%
\pgfsetstrokecolor{currentstroke}%
\pgfsetstrokeopacity{0.700000}%
\pgfsetdash{}{0pt}%
\pgfpathmoveto{\pgfqpoint{4.381925in}{0.550000in}}%
\pgfpathlineto{\pgfqpoint{4.409446in}{0.550000in}}%
\pgfpathlineto{\pgfqpoint{4.409446in}{0.646636in}}%
\pgfpathlineto{\pgfqpoint{4.381925in}{0.646636in}}%
\pgfpathlineto{\pgfqpoint{4.381925in}{0.550000in}}%
\pgfpathclose%
\pgfusepath{fill}%
\end{pgfscope}%
\begin{pgfscope}%
\pgfpathrectangle{\pgfqpoint{1.250000in}{0.550000in}}{\pgfqpoint{7.750000in}{3.850000in}}%
\pgfusepath{clip}%
\pgfsetbuttcap%
\pgfsetmiterjoin%
\definecolor{currentfill}{rgb}{0.501961,0.501961,0.501961}%
\pgfsetfillcolor{currentfill}%
\pgfsetfillopacity{0.700000}%
\pgfsetlinewidth{0.000000pt}%
\definecolor{currentstroke}{rgb}{0.000000,0.000000,0.000000}%
\pgfsetstrokecolor{currentstroke}%
\pgfsetstrokeopacity{0.700000}%
\pgfsetdash{}{0pt}%
\pgfpathmoveto{\pgfqpoint{4.409446in}{0.550000in}}%
\pgfpathlineto{\pgfqpoint{4.436967in}{0.550000in}}%
\pgfpathlineto{\pgfqpoint{4.436967in}{0.654620in}}%
\pgfpathlineto{\pgfqpoint{4.409446in}{0.654620in}}%
\pgfpathlineto{\pgfqpoint{4.409446in}{0.550000in}}%
\pgfpathclose%
\pgfusepath{fill}%
\end{pgfscope}%
\begin{pgfscope}%
\pgfpathrectangle{\pgfqpoint{1.250000in}{0.550000in}}{\pgfqpoint{7.750000in}{3.850000in}}%
\pgfusepath{clip}%
\pgfsetbuttcap%
\pgfsetmiterjoin%
\definecolor{currentfill}{rgb}{0.501961,0.501961,0.501961}%
\pgfsetfillcolor{currentfill}%
\pgfsetfillopacity{0.700000}%
\pgfsetlinewidth{0.000000pt}%
\definecolor{currentstroke}{rgb}{0.000000,0.000000,0.000000}%
\pgfsetstrokecolor{currentstroke}%
\pgfsetstrokeopacity{0.700000}%
\pgfsetdash{}{0pt}%
\pgfpathmoveto{\pgfqpoint{4.436967in}{0.550000in}}%
\pgfpathlineto{\pgfqpoint{4.464489in}{0.550000in}}%
\pgfpathlineto{\pgfqpoint{4.464489in}{0.646085in}}%
\pgfpathlineto{\pgfqpoint{4.436967in}{0.646085in}}%
\pgfpathlineto{\pgfqpoint{4.436967in}{0.550000in}}%
\pgfpathclose%
\pgfusepath{fill}%
\end{pgfscope}%
\begin{pgfscope}%
\pgfpathrectangle{\pgfqpoint{1.250000in}{0.550000in}}{\pgfqpoint{7.750000in}{3.850000in}}%
\pgfusepath{clip}%
\pgfsetbuttcap%
\pgfsetmiterjoin%
\definecolor{currentfill}{rgb}{0.501961,0.501961,0.501961}%
\pgfsetfillcolor{currentfill}%
\pgfsetfillopacity{0.700000}%
\pgfsetlinewidth{0.000000pt}%
\definecolor{currentstroke}{rgb}{0.000000,0.000000,0.000000}%
\pgfsetstrokecolor{currentstroke}%
\pgfsetstrokeopacity{0.700000}%
\pgfsetdash{}{0pt}%
\pgfpathmoveto{\pgfqpoint{4.464489in}{0.550000in}}%
\pgfpathlineto{\pgfqpoint{4.492010in}{0.550000in}}%
\pgfpathlineto{\pgfqpoint{4.492010in}{0.642231in}}%
\pgfpathlineto{\pgfqpoint{4.464489in}{0.642231in}}%
\pgfpathlineto{\pgfqpoint{4.464489in}{0.550000in}}%
\pgfpathclose%
\pgfusepath{fill}%
\end{pgfscope}%
\begin{pgfscope}%
\pgfpathrectangle{\pgfqpoint{1.250000in}{0.550000in}}{\pgfqpoint{7.750000in}{3.850000in}}%
\pgfusepath{clip}%
\pgfsetbuttcap%
\pgfsetmiterjoin%
\definecolor{currentfill}{rgb}{0.501961,0.501961,0.501961}%
\pgfsetfillcolor{currentfill}%
\pgfsetfillopacity{0.700000}%
\pgfsetlinewidth{0.000000pt}%
\definecolor{currentstroke}{rgb}{0.000000,0.000000,0.000000}%
\pgfsetstrokecolor{currentstroke}%
\pgfsetstrokeopacity{0.700000}%
\pgfsetdash{}{0pt}%
\pgfpathmoveto{\pgfqpoint{4.492010in}{0.550000in}}%
\pgfpathlineto{\pgfqpoint{4.519531in}{0.550000in}}%
\pgfpathlineto{\pgfqpoint{4.519531in}{0.648563in}}%
\pgfpathlineto{\pgfqpoint{4.492010in}{0.648563in}}%
\pgfpathlineto{\pgfqpoint{4.492010in}{0.550000in}}%
\pgfpathclose%
\pgfusepath{fill}%
\end{pgfscope}%
\begin{pgfscope}%
\pgfpathrectangle{\pgfqpoint{1.250000in}{0.550000in}}{\pgfqpoint{7.750000in}{3.850000in}}%
\pgfusepath{clip}%
\pgfsetbuttcap%
\pgfsetmiterjoin%
\definecolor{currentfill}{rgb}{0.501961,0.501961,0.501961}%
\pgfsetfillcolor{currentfill}%
\pgfsetfillopacity{0.700000}%
\pgfsetlinewidth{0.000000pt}%
\definecolor{currentstroke}{rgb}{0.000000,0.000000,0.000000}%
\pgfsetstrokecolor{currentstroke}%
\pgfsetstrokeopacity{0.700000}%
\pgfsetdash{}{0pt}%
\pgfpathmoveto{\pgfqpoint{4.519531in}{0.550000in}}%
\pgfpathlineto{\pgfqpoint{4.547053in}{0.550000in}}%
\pgfpathlineto{\pgfqpoint{4.547053in}{0.659576in}}%
\pgfpathlineto{\pgfqpoint{4.519531in}{0.659576in}}%
\pgfpathlineto{\pgfqpoint{4.519531in}{0.550000in}}%
\pgfpathclose%
\pgfusepath{fill}%
\end{pgfscope}%
\begin{pgfscope}%
\pgfpathrectangle{\pgfqpoint{1.250000in}{0.550000in}}{\pgfqpoint{7.750000in}{3.850000in}}%
\pgfusepath{clip}%
\pgfsetbuttcap%
\pgfsetmiterjoin%
\definecolor{currentfill}{rgb}{0.501961,0.501961,0.501961}%
\pgfsetfillcolor{currentfill}%
\pgfsetfillopacity{0.700000}%
\pgfsetlinewidth{0.000000pt}%
\definecolor{currentstroke}{rgb}{0.000000,0.000000,0.000000}%
\pgfsetstrokecolor{currentstroke}%
\pgfsetstrokeopacity{0.700000}%
\pgfsetdash{}{0pt}%
\pgfpathmoveto{\pgfqpoint{4.547053in}{0.550000in}}%
\pgfpathlineto{\pgfqpoint{4.574574in}{0.550000in}}%
\pgfpathlineto{\pgfqpoint{4.574574in}{0.652968in}}%
\pgfpathlineto{\pgfqpoint{4.547053in}{0.652968in}}%
\pgfpathlineto{\pgfqpoint{4.547053in}{0.550000in}}%
\pgfpathclose%
\pgfusepath{fill}%
\end{pgfscope}%
\begin{pgfscope}%
\pgfpathrectangle{\pgfqpoint{1.250000in}{0.550000in}}{\pgfqpoint{7.750000in}{3.850000in}}%
\pgfusepath{clip}%
\pgfsetbuttcap%
\pgfsetmiterjoin%
\definecolor{currentfill}{rgb}{0.501961,0.501961,0.501961}%
\pgfsetfillcolor{currentfill}%
\pgfsetfillopacity{0.700000}%
\pgfsetlinewidth{0.000000pt}%
\definecolor{currentstroke}{rgb}{0.000000,0.000000,0.000000}%
\pgfsetstrokecolor{currentstroke}%
\pgfsetstrokeopacity{0.700000}%
\pgfsetdash{}{0pt}%
\pgfpathmoveto{\pgfqpoint{4.574574in}{0.550000in}}%
\pgfpathlineto{\pgfqpoint{4.602095in}{0.550000in}}%
\pgfpathlineto{\pgfqpoint{4.602095in}{0.652968in}}%
\pgfpathlineto{\pgfqpoint{4.574574in}{0.652968in}}%
\pgfpathlineto{\pgfqpoint{4.574574in}{0.550000in}}%
\pgfpathclose%
\pgfusepath{fill}%
\end{pgfscope}%
\begin{pgfscope}%
\pgfpathrectangle{\pgfqpoint{1.250000in}{0.550000in}}{\pgfqpoint{7.750000in}{3.850000in}}%
\pgfusepath{clip}%
\pgfsetbuttcap%
\pgfsetmiterjoin%
\definecolor{currentfill}{rgb}{0.501961,0.501961,0.501961}%
\pgfsetfillcolor{currentfill}%
\pgfsetfillopacity{0.700000}%
\pgfsetlinewidth{0.000000pt}%
\definecolor{currentstroke}{rgb}{0.000000,0.000000,0.000000}%
\pgfsetstrokecolor{currentstroke}%
\pgfsetstrokeopacity{0.700000}%
\pgfsetdash{}{0pt}%
\pgfpathmoveto{\pgfqpoint{4.602095in}{0.550000in}}%
\pgfpathlineto{\pgfqpoint{4.629616in}{0.550000in}}%
\pgfpathlineto{\pgfqpoint{4.629616in}{0.652693in}}%
\pgfpathlineto{\pgfqpoint{4.602095in}{0.652693in}}%
\pgfpathlineto{\pgfqpoint{4.602095in}{0.550000in}}%
\pgfpathclose%
\pgfusepath{fill}%
\end{pgfscope}%
\begin{pgfscope}%
\pgfpathrectangle{\pgfqpoint{1.250000in}{0.550000in}}{\pgfqpoint{7.750000in}{3.850000in}}%
\pgfusepath{clip}%
\pgfsetbuttcap%
\pgfsetmiterjoin%
\definecolor{currentfill}{rgb}{0.501961,0.501961,0.501961}%
\pgfsetfillcolor{currentfill}%
\pgfsetfillopacity{0.700000}%
\pgfsetlinewidth{0.000000pt}%
\definecolor{currentstroke}{rgb}{0.000000,0.000000,0.000000}%
\pgfsetstrokecolor{currentstroke}%
\pgfsetstrokeopacity{0.700000}%
\pgfsetdash{}{0pt}%
\pgfpathmoveto{\pgfqpoint{4.629616in}{0.550000in}}%
\pgfpathlineto{\pgfqpoint{4.657138in}{0.550000in}}%
\pgfpathlineto{\pgfqpoint{4.657138in}{0.651317in}}%
\pgfpathlineto{\pgfqpoint{4.629616in}{0.651317in}}%
\pgfpathlineto{\pgfqpoint{4.629616in}{0.550000in}}%
\pgfpathclose%
\pgfusepath{fill}%
\end{pgfscope}%
\begin{pgfscope}%
\pgfpathrectangle{\pgfqpoint{1.250000in}{0.550000in}}{\pgfqpoint{7.750000in}{3.850000in}}%
\pgfusepath{clip}%
\pgfsetbuttcap%
\pgfsetmiterjoin%
\definecolor{currentfill}{rgb}{0.501961,0.501961,0.501961}%
\pgfsetfillcolor{currentfill}%
\pgfsetfillopacity{0.700000}%
\pgfsetlinewidth{0.000000pt}%
\definecolor{currentstroke}{rgb}{0.000000,0.000000,0.000000}%
\pgfsetstrokecolor{currentstroke}%
\pgfsetstrokeopacity{0.700000}%
\pgfsetdash{}{0pt}%
\pgfpathmoveto{\pgfqpoint{4.657138in}{0.550000in}}%
\pgfpathlineto{\pgfqpoint{4.684659in}{0.550000in}}%
\pgfpathlineto{\pgfqpoint{4.684659in}{0.655997in}}%
\pgfpathlineto{\pgfqpoint{4.657138in}{0.655997in}}%
\pgfpathlineto{\pgfqpoint{4.657138in}{0.550000in}}%
\pgfpathclose%
\pgfusepath{fill}%
\end{pgfscope}%
\begin{pgfscope}%
\pgfpathrectangle{\pgfqpoint{1.250000in}{0.550000in}}{\pgfqpoint{7.750000in}{3.850000in}}%
\pgfusepath{clip}%
\pgfsetbuttcap%
\pgfsetmiterjoin%
\definecolor{currentfill}{rgb}{0.501961,0.501961,0.501961}%
\pgfsetfillcolor{currentfill}%
\pgfsetfillopacity{0.700000}%
\pgfsetlinewidth{0.000000pt}%
\definecolor{currentstroke}{rgb}{0.000000,0.000000,0.000000}%
\pgfsetstrokecolor{currentstroke}%
\pgfsetstrokeopacity{0.700000}%
\pgfsetdash{}{0pt}%
\pgfpathmoveto{\pgfqpoint{4.684659in}{0.550000in}}%
\pgfpathlineto{\pgfqpoint{4.712180in}{0.550000in}}%
\pgfpathlineto{\pgfqpoint{4.712180in}{0.648839in}}%
\pgfpathlineto{\pgfqpoint{4.684659in}{0.648839in}}%
\pgfpathlineto{\pgfqpoint{4.684659in}{0.550000in}}%
\pgfpathclose%
\pgfusepath{fill}%
\end{pgfscope}%
\begin{pgfscope}%
\pgfpathrectangle{\pgfqpoint{1.250000in}{0.550000in}}{\pgfqpoint{7.750000in}{3.850000in}}%
\pgfusepath{clip}%
\pgfsetbuttcap%
\pgfsetmiterjoin%
\definecolor{currentfill}{rgb}{0.501961,0.501961,0.501961}%
\pgfsetfillcolor{currentfill}%
\pgfsetfillopacity{0.700000}%
\pgfsetlinewidth{0.000000pt}%
\definecolor{currentstroke}{rgb}{0.000000,0.000000,0.000000}%
\pgfsetstrokecolor{currentstroke}%
\pgfsetstrokeopacity{0.700000}%
\pgfsetdash{}{0pt}%
\pgfpathmoveto{\pgfqpoint{4.712180in}{0.550000in}}%
\pgfpathlineto{\pgfqpoint{4.739702in}{0.550000in}}%
\pgfpathlineto{\pgfqpoint{4.739702in}{0.654620in}}%
\pgfpathlineto{\pgfqpoint{4.712180in}{0.654620in}}%
\pgfpathlineto{\pgfqpoint{4.712180in}{0.550000in}}%
\pgfpathclose%
\pgfusepath{fill}%
\end{pgfscope}%
\begin{pgfscope}%
\pgfpathrectangle{\pgfqpoint{1.250000in}{0.550000in}}{\pgfqpoint{7.750000in}{3.850000in}}%
\pgfusepath{clip}%
\pgfsetbuttcap%
\pgfsetmiterjoin%
\definecolor{currentfill}{rgb}{0.501961,0.501961,0.501961}%
\pgfsetfillcolor{currentfill}%
\pgfsetfillopacity{0.700000}%
\pgfsetlinewidth{0.000000pt}%
\definecolor{currentstroke}{rgb}{0.000000,0.000000,0.000000}%
\pgfsetstrokecolor{currentstroke}%
\pgfsetstrokeopacity{0.700000}%
\pgfsetdash{}{0pt}%
\pgfpathmoveto{\pgfqpoint{4.739702in}{0.550000in}}%
\pgfpathlineto{\pgfqpoint{4.767223in}{0.550000in}}%
\pgfpathlineto{\pgfqpoint{4.767223in}{0.647187in}}%
\pgfpathlineto{\pgfqpoint{4.739702in}{0.647187in}}%
\pgfpathlineto{\pgfqpoint{4.739702in}{0.550000in}}%
\pgfpathclose%
\pgfusepath{fill}%
\end{pgfscope}%
\begin{pgfscope}%
\pgfpathrectangle{\pgfqpoint{1.250000in}{0.550000in}}{\pgfqpoint{7.750000in}{3.850000in}}%
\pgfusepath{clip}%
\pgfsetbuttcap%
\pgfsetmiterjoin%
\definecolor{currentfill}{rgb}{0.501961,0.501961,0.501961}%
\pgfsetfillcolor{currentfill}%
\pgfsetfillopacity{0.700000}%
\pgfsetlinewidth{0.000000pt}%
\definecolor{currentstroke}{rgb}{0.000000,0.000000,0.000000}%
\pgfsetstrokecolor{currentstroke}%
\pgfsetstrokeopacity{0.700000}%
\pgfsetdash{}{0pt}%
\pgfpathmoveto{\pgfqpoint{4.767223in}{0.550000in}}%
\pgfpathlineto{\pgfqpoint{4.794744in}{0.550000in}}%
\pgfpathlineto{\pgfqpoint{4.794744in}{0.657373in}}%
\pgfpathlineto{\pgfqpoint{4.767223in}{0.657373in}}%
\pgfpathlineto{\pgfqpoint{4.767223in}{0.550000in}}%
\pgfpathclose%
\pgfusepath{fill}%
\end{pgfscope}%
\begin{pgfscope}%
\pgfpathrectangle{\pgfqpoint{1.250000in}{0.550000in}}{\pgfqpoint{7.750000in}{3.850000in}}%
\pgfusepath{clip}%
\pgfsetbuttcap%
\pgfsetmiterjoin%
\definecolor{currentfill}{rgb}{0.501961,0.501961,0.501961}%
\pgfsetfillcolor{currentfill}%
\pgfsetfillopacity{0.700000}%
\pgfsetlinewidth{0.000000pt}%
\definecolor{currentstroke}{rgb}{0.000000,0.000000,0.000000}%
\pgfsetstrokecolor{currentstroke}%
\pgfsetstrokeopacity{0.700000}%
\pgfsetdash{}{0pt}%
\pgfpathmoveto{\pgfqpoint{4.794744in}{0.550000in}}%
\pgfpathlineto{\pgfqpoint{4.822266in}{0.550000in}}%
\pgfpathlineto{\pgfqpoint{4.822266in}{0.655997in}}%
\pgfpathlineto{\pgfqpoint{4.794744in}{0.655997in}}%
\pgfpathlineto{\pgfqpoint{4.794744in}{0.550000in}}%
\pgfpathclose%
\pgfusepath{fill}%
\end{pgfscope}%
\begin{pgfscope}%
\pgfpathrectangle{\pgfqpoint{1.250000in}{0.550000in}}{\pgfqpoint{7.750000in}{3.850000in}}%
\pgfusepath{clip}%
\pgfsetbuttcap%
\pgfsetmiterjoin%
\definecolor{currentfill}{rgb}{0.501961,0.501961,0.501961}%
\pgfsetfillcolor{currentfill}%
\pgfsetfillopacity{0.700000}%
\pgfsetlinewidth{0.000000pt}%
\definecolor{currentstroke}{rgb}{0.000000,0.000000,0.000000}%
\pgfsetstrokecolor{currentstroke}%
\pgfsetstrokeopacity{0.700000}%
\pgfsetdash{}{0pt}%
\pgfpathmoveto{\pgfqpoint{4.822266in}{0.550000in}}%
\pgfpathlineto{\pgfqpoint{4.849787in}{0.550000in}}%
\pgfpathlineto{\pgfqpoint{4.849787in}{0.659301in}}%
\pgfpathlineto{\pgfqpoint{4.822266in}{0.659301in}}%
\pgfpathlineto{\pgfqpoint{4.822266in}{0.550000in}}%
\pgfpathclose%
\pgfusepath{fill}%
\end{pgfscope}%
\begin{pgfscope}%
\pgfpathrectangle{\pgfqpoint{1.250000in}{0.550000in}}{\pgfqpoint{7.750000in}{3.850000in}}%
\pgfusepath{clip}%
\pgfsetbuttcap%
\pgfsetmiterjoin%
\definecolor{currentfill}{rgb}{0.501961,0.501961,0.501961}%
\pgfsetfillcolor{currentfill}%
\pgfsetfillopacity{0.700000}%
\pgfsetlinewidth{0.000000pt}%
\definecolor{currentstroke}{rgb}{0.000000,0.000000,0.000000}%
\pgfsetstrokecolor{currentstroke}%
\pgfsetstrokeopacity{0.700000}%
\pgfsetdash{}{0pt}%
\pgfpathmoveto{\pgfqpoint{4.849787in}{0.550000in}}%
\pgfpathlineto{\pgfqpoint{4.877308in}{0.550000in}}%
\pgfpathlineto{\pgfqpoint{4.877308in}{0.658750in}}%
\pgfpathlineto{\pgfqpoint{4.849787in}{0.658750in}}%
\pgfpathlineto{\pgfqpoint{4.849787in}{0.550000in}}%
\pgfpathclose%
\pgfusepath{fill}%
\end{pgfscope}%
\begin{pgfscope}%
\pgfpathrectangle{\pgfqpoint{1.250000in}{0.550000in}}{\pgfqpoint{7.750000in}{3.850000in}}%
\pgfusepath{clip}%
\pgfsetbuttcap%
\pgfsetmiterjoin%
\definecolor{currentfill}{rgb}{0.501961,0.501961,0.501961}%
\pgfsetfillcolor{currentfill}%
\pgfsetfillopacity{0.700000}%
\pgfsetlinewidth{0.000000pt}%
\definecolor{currentstroke}{rgb}{0.000000,0.000000,0.000000}%
\pgfsetstrokecolor{currentstroke}%
\pgfsetstrokeopacity{0.700000}%
\pgfsetdash{}{0pt}%
\pgfpathmoveto{\pgfqpoint{4.877308in}{0.550000in}}%
\pgfpathlineto{\pgfqpoint{4.904830in}{0.550000in}}%
\pgfpathlineto{\pgfqpoint{4.904830in}{0.658750in}}%
\pgfpathlineto{\pgfqpoint{4.877308in}{0.658750in}}%
\pgfpathlineto{\pgfqpoint{4.877308in}{0.550000in}}%
\pgfpathclose%
\pgfusepath{fill}%
\end{pgfscope}%
\begin{pgfscope}%
\pgfpathrectangle{\pgfqpoint{1.250000in}{0.550000in}}{\pgfqpoint{7.750000in}{3.850000in}}%
\pgfusepath{clip}%
\pgfsetbuttcap%
\pgfsetmiterjoin%
\definecolor{currentfill}{rgb}{0.501961,0.501961,0.501961}%
\pgfsetfillcolor{currentfill}%
\pgfsetfillopacity{0.700000}%
\pgfsetlinewidth{0.000000pt}%
\definecolor{currentstroke}{rgb}{0.000000,0.000000,0.000000}%
\pgfsetstrokecolor{currentstroke}%
\pgfsetstrokeopacity{0.700000}%
\pgfsetdash{}{0pt}%
\pgfpathmoveto{\pgfqpoint{4.904830in}{0.550000in}}%
\pgfpathlineto{\pgfqpoint{4.932351in}{0.550000in}}%
\pgfpathlineto{\pgfqpoint{4.932351in}{0.651041in}}%
\pgfpathlineto{\pgfqpoint{4.904830in}{0.651041in}}%
\pgfpathlineto{\pgfqpoint{4.904830in}{0.550000in}}%
\pgfpathclose%
\pgfusepath{fill}%
\end{pgfscope}%
\begin{pgfscope}%
\pgfpathrectangle{\pgfqpoint{1.250000in}{0.550000in}}{\pgfqpoint{7.750000in}{3.850000in}}%
\pgfusepath{clip}%
\pgfsetbuttcap%
\pgfsetmiterjoin%
\definecolor{currentfill}{rgb}{0.501961,0.501961,0.501961}%
\pgfsetfillcolor{currentfill}%
\pgfsetfillopacity{0.700000}%
\pgfsetlinewidth{0.000000pt}%
\definecolor{currentstroke}{rgb}{0.000000,0.000000,0.000000}%
\pgfsetstrokecolor{currentstroke}%
\pgfsetstrokeopacity{0.700000}%
\pgfsetdash{}{0pt}%
\pgfpathmoveto{\pgfqpoint{4.932351in}{0.550000in}}%
\pgfpathlineto{\pgfqpoint{4.959872in}{0.550000in}}%
\pgfpathlineto{\pgfqpoint{4.959872in}{0.659576in}}%
\pgfpathlineto{\pgfqpoint{4.932351in}{0.659576in}}%
\pgfpathlineto{\pgfqpoint{4.932351in}{0.550000in}}%
\pgfpathclose%
\pgfusepath{fill}%
\end{pgfscope}%
\begin{pgfscope}%
\pgfpathrectangle{\pgfqpoint{1.250000in}{0.550000in}}{\pgfqpoint{7.750000in}{3.850000in}}%
\pgfusepath{clip}%
\pgfsetbuttcap%
\pgfsetmiterjoin%
\definecolor{currentfill}{rgb}{0.501961,0.501961,0.501961}%
\pgfsetfillcolor{currentfill}%
\pgfsetfillopacity{0.700000}%
\pgfsetlinewidth{0.000000pt}%
\definecolor{currentstroke}{rgb}{0.000000,0.000000,0.000000}%
\pgfsetstrokecolor{currentstroke}%
\pgfsetstrokeopacity{0.700000}%
\pgfsetdash{}{0pt}%
\pgfpathmoveto{\pgfqpoint{4.959872in}{0.550000in}}%
\pgfpathlineto{\pgfqpoint{4.987393in}{0.550000in}}%
\pgfpathlineto{\pgfqpoint{4.987393in}{0.652142in}}%
\pgfpathlineto{\pgfqpoint{4.959872in}{0.652142in}}%
\pgfpathlineto{\pgfqpoint{4.959872in}{0.550000in}}%
\pgfpathclose%
\pgfusepath{fill}%
\end{pgfscope}%
\begin{pgfscope}%
\pgfpathrectangle{\pgfqpoint{1.250000in}{0.550000in}}{\pgfqpoint{7.750000in}{3.850000in}}%
\pgfusepath{clip}%
\pgfsetbuttcap%
\pgfsetmiterjoin%
\definecolor{currentfill}{rgb}{0.501961,0.501961,0.501961}%
\pgfsetfillcolor{currentfill}%
\pgfsetfillopacity{0.700000}%
\pgfsetlinewidth{0.000000pt}%
\definecolor{currentstroke}{rgb}{0.000000,0.000000,0.000000}%
\pgfsetstrokecolor{currentstroke}%
\pgfsetstrokeopacity{0.700000}%
\pgfsetdash{}{0pt}%
\pgfpathmoveto{\pgfqpoint{4.987393in}{0.550000in}}%
\pgfpathlineto{\pgfqpoint{5.014915in}{0.550000in}}%
\pgfpathlineto{\pgfqpoint{5.014915in}{0.657098in}}%
\pgfpathlineto{\pgfqpoint{4.987393in}{0.657098in}}%
\pgfpathlineto{\pgfqpoint{4.987393in}{0.550000in}}%
\pgfpathclose%
\pgfusepath{fill}%
\end{pgfscope}%
\begin{pgfscope}%
\pgfpathrectangle{\pgfqpoint{1.250000in}{0.550000in}}{\pgfqpoint{7.750000in}{3.850000in}}%
\pgfusepath{clip}%
\pgfsetbuttcap%
\pgfsetmiterjoin%
\definecolor{currentfill}{rgb}{0.501961,0.501961,0.501961}%
\pgfsetfillcolor{currentfill}%
\pgfsetfillopacity{0.700000}%
\pgfsetlinewidth{0.000000pt}%
\definecolor{currentstroke}{rgb}{0.000000,0.000000,0.000000}%
\pgfsetstrokecolor{currentstroke}%
\pgfsetstrokeopacity{0.700000}%
\pgfsetdash{}{0pt}%
\pgfpathmoveto{\pgfqpoint{5.014915in}{0.550000in}}%
\pgfpathlineto{\pgfqpoint{5.042436in}{0.550000in}}%
\pgfpathlineto{\pgfqpoint{5.042436in}{0.646911in}}%
\pgfpathlineto{\pgfqpoint{5.014915in}{0.646911in}}%
\pgfpathlineto{\pgfqpoint{5.014915in}{0.550000in}}%
\pgfpathclose%
\pgfusepath{fill}%
\end{pgfscope}%
\begin{pgfscope}%
\pgfpathrectangle{\pgfqpoint{1.250000in}{0.550000in}}{\pgfqpoint{7.750000in}{3.850000in}}%
\pgfusepath{clip}%
\pgfsetbuttcap%
\pgfsetmiterjoin%
\definecolor{currentfill}{rgb}{0.501961,0.501961,0.501961}%
\pgfsetfillcolor{currentfill}%
\pgfsetfillopacity{0.700000}%
\pgfsetlinewidth{0.000000pt}%
\definecolor{currentstroke}{rgb}{0.000000,0.000000,0.000000}%
\pgfsetstrokecolor{currentstroke}%
\pgfsetstrokeopacity{0.700000}%
\pgfsetdash{}{0pt}%
\pgfpathmoveto{\pgfqpoint{5.042436in}{0.550000in}}%
\pgfpathlineto{\pgfqpoint{5.069957in}{0.550000in}}%
\pgfpathlineto{\pgfqpoint{5.069957in}{0.664256in}}%
\pgfpathlineto{\pgfqpoint{5.042436in}{0.664256in}}%
\pgfpathlineto{\pgfqpoint{5.042436in}{0.550000in}}%
\pgfpathclose%
\pgfusepath{fill}%
\end{pgfscope}%
\begin{pgfscope}%
\pgfpathrectangle{\pgfqpoint{1.250000in}{0.550000in}}{\pgfqpoint{7.750000in}{3.850000in}}%
\pgfusepath{clip}%
\pgfsetbuttcap%
\pgfsetmiterjoin%
\definecolor{currentfill}{rgb}{0.501961,0.501961,0.501961}%
\pgfsetfillcolor{currentfill}%
\pgfsetfillopacity{0.700000}%
\pgfsetlinewidth{0.000000pt}%
\definecolor{currentstroke}{rgb}{0.000000,0.000000,0.000000}%
\pgfsetstrokecolor{currentstroke}%
\pgfsetstrokeopacity{0.700000}%
\pgfsetdash{}{0pt}%
\pgfpathmoveto{\pgfqpoint{5.069957in}{0.550000in}}%
\pgfpathlineto{\pgfqpoint{5.097479in}{0.550000in}}%
\pgfpathlineto{\pgfqpoint{5.097479in}{0.657924in}}%
\pgfpathlineto{\pgfqpoint{5.069957in}{0.657924in}}%
\pgfpathlineto{\pgfqpoint{5.069957in}{0.550000in}}%
\pgfpathclose%
\pgfusepath{fill}%
\end{pgfscope}%
\begin{pgfscope}%
\pgfpathrectangle{\pgfqpoint{1.250000in}{0.550000in}}{\pgfqpoint{7.750000in}{3.850000in}}%
\pgfusepath{clip}%
\pgfsetbuttcap%
\pgfsetmiterjoin%
\definecolor{currentfill}{rgb}{0.501961,0.501961,0.501961}%
\pgfsetfillcolor{currentfill}%
\pgfsetfillopacity{0.700000}%
\pgfsetlinewidth{0.000000pt}%
\definecolor{currentstroke}{rgb}{0.000000,0.000000,0.000000}%
\pgfsetstrokecolor{currentstroke}%
\pgfsetstrokeopacity{0.700000}%
\pgfsetdash{}{0pt}%
\pgfpathmoveto{\pgfqpoint{5.097479in}{0.550000in}}%
\pgfpathlineto{\pgfqpoint{5.125000in}{0.550000in}}%
\pgfpathlineto{\pgfqpoint{5.125000in}{0.660127in}}%
\pgfpathlineto{\pgfqpoint{5.097479in}{0.660127in}}%
\pgfpathlineto{\pgfqpoint{5.097479in}{0.550000in}}%
\pgfpathclose%
\pgfusepath{fill}%
\end{pgfscope}%
\begin{pgfscope}%
\pgfpathrectangle{\pgfqpoint{1.250000in}{0.550000in}}{\pgfqpoint{7.750000in}{3.850000in}}%
\pgfusepath{clip}%
\pgfsetbuttcap%
\pgfsetmiterjoin%
\definecolor{currentfill}{rgb}{0.501961,0.501961,0.501961}%
\pgfsetfillcolor{currentfill}%
\pgfsetfillopacity{0.700000}%
\pgfsetlinewidth{0.000000pt}%
\definecolor{currentstroke}{rgb}{0.000000,0.000000,0.000000}%
\pgfsetstrokecolor{currentstroke}%
\pgfsetstrokeopacity{0.700000}%
\pgfsetdash{}{0pt}%
\pgfpathmoveto{\pgfqpoint{5.125000in}{0.550000in}}%
\pgfpathlineto{\pgfqpoint{5.152521in}{0.550000in}}%
\pgfpathlineto{\pgfqpoint{5.152521in}{0.662880in}}%
\pgfpathlineto{\pgfqpoint{5.125000in}{0.662880in}}%
\pgfpathlineto{\pgfqpoint{5.125000in}{0.550000in}}%
\pgfpathclose%
\pgfusepath{fill}%
\end{pgfscope}%
\begin{pgfscope}%
\pgfpathrectangle{\pgfqpoint{1.250000in}{0.550000in}}{\pgfqpoint{7.750000in}{3.850000in}}%
\pgfusepath{clip}%
\pgfsetbuttcap%
\pgfsetmiterjoin%
\definecolor{currentfill}{rgb}{0.501961,0.501961,0.501961}%
\pgfsetfillcolor{currentfill}%
\pgfsetfillopacity{0.700000}%
\pgfsetlinewidth{0.000000pt}%
\definecolor{currentstroke}{rgb}{0.000000,0.000000,0.000000}%
\pgfsetstrokecolor{currentstroke}%
\pgfsetstrokeopacity{0.700000}%
\pgfsetdash{}{0pt}%
\pgfpathmoveto{\pgfqpoint{5.152521in}{0.550000in}}%
\pgfpathlineto{\pgfqpoint{5.180043in}{0.550000in}}%
\pgfpathlineto{\pgfqpoint{5.180043in}{0.668937in}}%
\pgfpathlineto{\pgfqpoint{5.152521in}{0.668937in}}%
\pgfpathlineto{\pgfqpoint{5.152521in}{0.550000in}}%
\pgfpathclose%
\pgfusepath{fill}%
\end{pgfscope}%
\begin{pgfscope}%
\pgfpathrectangle{\pgfqpoint{1.250000in}{0.550000in}}{\pgfqpoint{7.750000in}{3.850000in}}%
\pgfusepath{clip}%
\pgfsetbuttcap%
\pgfsetmiterjoin%
\definecolor{currentfill}{rgb}{0.501961,0.501961,0.501961}%
\pgfsetfillcolor{currentfill}%
\pgfsetfillopacity{0.700000}%
\pgfsetlinewidth{0.000000pt}%
\definecolor{currentstroke}{rgb}{0.000000,0.000000,0.000000}%
\pgfsetstrokecolor{currentstroke}%
\pgfsetstrokeopacity{0.700000}%
\pgfsetdash{}{0pt}%
\pgfpathmoveto{\pgfqpoint{5.180043in}{0.550000in}}%
\pgfpathlineto{\pgfqpoint{5.207564in}{0.550000in}}%
\pgfpathlineto{\pgfqpoint{5.207564in}{0.678298in}}%
\pgfpathlineto{\pgfqpoint{5.180043in}{0.678298in}}%
\pgfpathlineto{\pgfqpoint{5.180043in}{0.550000in}}%
\pgfpathclose%
\pgfusepath{fill}%
\end{pgfscope}%
\begin{pgfscope}%
\pgfpathrectangle{\pgfqpoint{1.250000in}{0.550000in}}{\pgfqpoint{7.750000in}{3.850000in}}%
\pgfusepath{clip}%
\pgfsetbuttcap%
\pgfsetmiterjoin%
\definecolor{currentfill}{rgb}{0.501961,0.501961,0.501961}%
\pgfsetfillcolor{currentfill}%
\pgfsetfillopacity{0.700000}%
\pgfsetlinewidth{0.000000pt}%
\definecolor{currentstroke}{rgb}{0.000000,0.000000,0.000000}%
\pgfsetstrokecolor{currentstroke}%
\pgfsetstrokeopacity{0.700000}%
\pgfsetdash{}{0pt}%
\pgfpathmoveto{\pgfqpoint{5.207564in}{0.550000in}}%
\pgfpathlineto{\pgfqpoint{5.235085in}{0.550000in}}%
\pgfpathlineto{\pgfqpoint{5.235085in}{0.680500in}}%
\pgfpathlineto{\pgfqpoint{5.207564in}{0.680500in}}%
\pgfpathlineto{\pgfqpoint{5.207564in}{0.550000in}}%
\pgfpathclose%
\pgfusepath{fill}%
\end{pgfscope}%
\begin{pgfscope}%
\pgfpathrectangle{\pgfqpoint{1.250000in}{0.550000in}}{\pgfqpoint{7.750000in}{3.850000in}}%
\pgfusepath{clip}%
\pgfsetbuttcap%
\pgfsetmiterjoin%
\definecolor{currentfill}{rgb}{0.501961,0.501961,0.501961}%
\pgfsetfillcolor{currentfill}%
\pgfsetfillopacity{0.700000}%
\pgfsetlinewidth{0.000000pt}%
\definecolor{currentstroke}{rgb}{0.000000,0.000000,0.000000}%
\pgfsetstrokecolor{currentstroke}%
\pgfsetstrokeopacity{0.700000}%
\pgfsetdash{}{0pt}%
\pgfpathmoveto{\pgfqpoint{5.235085in}{0.550000in}}%
\pgfpathlineto{\pgfqpoint{5.262607in}{0.550000in}}%
\pgfpathlineto{\pgfqpoint{5.262607in}{0.677472in}}%
\pgfpathlineto{\pgfqpoint{5.235085in}{0.677472in}}%
\pgfpathlineto{\pgfqpoint{5.235085in}{0.550000in}}%
\pgfpathclose%
\pgfusepath{fill}%
\end{pgfscope}%
\begin{pgfscope}%
\pgfpathrectangle{\pgfqpoint{1.250000in}{0.550000in}}{\pgfqpoint{7.750000in}{3.850000in}}%
\pgfusepath{clip}%
\pgfsetbuttcap%
\pgfsetmiterjoin%
\definecolor{currentfill}{rgb}{0.501961,0.501961,0.501961}%
\pgfsetfillcolor{currentfill}%
\pgfsetfillopacity{0.700000}%
\pgfsetlinewidth{0.000000pt}%
\definecolor{currentstroke}{rgb}{0.000000,0.000000,0.000000}%
\pgfsetstrokecolor{currentstroke}%
\pgfsetstrokeopacity{0.700000}%
\pgfsetdash{}{0pt}%
\pgfpathmoveto{\pgfqpoint{5.262607in}{0.550000in}}%
\pgfpathlineto{\pgfqpoint{5.290128in}{0.550000in}}%
\pgfpathlineto{\pgfqpoint{5.290128in}{0.677196in}}%
\pgfpathlineto{\pgfqpoint{5.262607in}{0.677196in}}%
\pgfpathlineto{\pgfqpoint{5.262607in}{0.550000in}}%
\pgfpathclose%
\pgfusepath{fill}%
\end{pgfscope}%
\begin{pgfscope}%
\pgfpathrectangle{\pgfqpoint{1.250000in}{0.550000in}}{\pgfqpoint{7.750000in}{3.850000in}}%
\pgfusepath{clip}%
\pgfsetbuttcap%
\pgfsetmiterjoin%
\definecolor{currentfill}{rgb}{0.501961,0.501961,0.501961}%
\pgfsetfillcolor{currentfill}%
\pgfsetfillopacity{0.700000}%
\pgfsetlinewidth{0.000000pt}%
\definecolor{currentstroke}{rgb}{0.000000,0.000000,0.000000}%
\pgfsetstrokecolor{currentstroke}%
\pgfsetstrokeopacity{0.700000}%
\pgfsetdash{}{0pt}%
\pgfpathmoveto{\pgfqpoint{5.290128in}{0.550000in}}%
\pgfpathlineto{\pgfqpoint{5.317649in}{0.550000in}}%
\pgfpathlineto{\pgfqpoint{5.317649in}{0.683253in}}%
\pgfpathlineto{\pgfqpoint{5.290128in}{0.683253in}}%
\pgfpathlineto{\pgfqpoint{5.290128in}{0.550000in}}%
\pgfpathclose%
\pgfusepath{fill}%
\end{pgfscope}%
\begin{pgfscope}%
\pgfpathrectangle{\pgfqpoint{1.250000in}{0.550000in}}{\pgfqpoint{7.750000in}{3.850000in}}%
\pgfusepath{clip}%
\pgfsetbuttcap%
\pgfsetmiterjoin%
\definecolor{currentfill}{rgb}{0.501961,0.501961,0.501961}%
\pgfsetfillcolor{currentfill}%
\pgfsetfillopacity{0.700000}%
\pgfsetlinewidth{0.000000pt}%
\definecolor{currentstroke}{rgb}{0.000000,0.000000,0.000000}%
\pgfsetstrokecolor{currentstroke}%
\pgfsetstrokeopacity{0.700000}%
\pgfsetdash{}{0pt}%
\pgfpathmoveto{\pgfqpoint{5.317649in}{0.550000in}}%
\pgfpathlineto{\pgfqpoint{5.345170in}{0.550000in}}%
\pgfpathlineto{\pgfqpoint{5.345170in}{0.675269in}}%
\pgfpathlineto{\pgfqpoint{5.317649in}{0.675269in}}%
\pgfpathlineto{\pgfqpoint{5.317649in}{0.550000in}}%
\pgfpathclose%
\pgfusepath{fill}%
\end{pgfscope}%
\begin{pgfscope}%
\pgfpathrectangle{\pgfqpoint{1.250000in}{0.550000in}}{\pgfqpoint{7.750000in}{3.850000in}}%
\pgfusepath{clip}%
\pgfsetbuttcap%
\pgfsetmiterjoin%
\definecolor{currentfill}{rgb}{0.501961,0.501961,0.501961}%
\pgfsetfillcolor{currentfill}%
\pgfsetfillopacity{0.700000}%
\pgfsetlinewidth{0.000000pt}%
\definecolor{currentstroke}{rgb}{0.000000,0.000000,0.000000}%
\pgfsetstrokecolor{currentstroke}%
\pgfsetstrokeopacity{0.700000}%
\pgfsetdash{}{0pt}%
\pgfpathmoveto{\pgfqpoint{5.345170in}{0.550000in}}%
\pgfpathlineto{\pgfqpoint{5.372692in}{0.550000in}}%
\pgfpathlineto{\pgfqpoint{5.372692in}{0.679949in}}%
\pgfpathlineto{\pgfqpoint{5.345170in}{0.679949in}}%
\pgfpathlineto{\pgfqpoint{5.345170in}{0.550000in}}%
\pgfpathclose%
\pgfusepath{fill}%
\end{pgfscope}%
\begin{pgfscope}%
\pgfpathrectangle{\pgfqpoint{1.250000in}{0.550000in}}{\pgfqpoint{7.750000in}{3.850000in}}%
\pgfusepath{clip}%
\pgfsetbuttcap%
\pgfsetmiterjoin%
\definecolor{currentfill}{rgb}{0.501961,0.501961,0.501961}%
\pgfsetfillcolor{currentfill}%
\pgfsetfillopacity{0.700000}%
\pgfsetlinewidth{0.000000pt}%
\definecolor{currentstroke}{rgb}{0.000000,0.000000,0.000000}%
\pgfsetstrokecolor{currentstroke}%
\pgfsetstrokeopacity{0.700000}%
\pgfsetdash{}{0pt}%
\pgfpathmoveto{\pgfqpoint{5.372692in}{0.550000in}}%
\pgfpathlineto{\pgfqpoint{5.400213in}{0.550000in}}%
\pgfpathlineto{\pgfqpoint{5.400213in}{0.679949in}}%
\pgfpathlineto{\pgfqpoint{5.372692in}{0.679949in}}%
\pgfpathlineto{\pgfqpoint{5.372692in}{0.550000in}}%
\pgfpathclose%
\pgfusepath{fill}%
\end{pgfscope}%
\begin{pgfscope}%
\pgfpathrectangle{\pgfqpoint{1.250000in}{0.550000in}}{\pgfqpoint{7.750000in}{3.850000in}}%
\pgfusepath{clip}%
\pgfsetbuttcap%
\pgfsetmiterjoin%
\definecolor{currentfill}{rgb}{0.501961,0.501961,0.501961}%
\pgfsetfillcolor{currentfill}%
\pgfsetfillopacity{0.700000}%
\pgfsetlinewidth{0.000000pt}%
\definecolor{currentstroke}{rgb}{0.000000,0.000000,0.000000}%
\pgfsetstrokecolor{currentstroke}%
\pgfsetstrokeopacity{0.700000}%
\pgfsetdash{}{0pt}%
\pgfpathmoveto{\pgfqpoint{5.400213in}{0.550000in}}%
\pgfpathlineto{\pgfqpoint{5.427734in}{0.550000in}}%
\pgfpathlineto{\pgfqpoint{5.427734in}{0.684079in}}%
\pgfpathlineto{\pgfqpoint{5.400213in}{0.684079in}}%
\pgfpathlineto{\pgfqpoint{5.400213in}{0.550000in}}%
\pgfpathclose%
\pgfusepath{fill}%
\end{pgfscope}%
\begin{pgfscope}%
\pgfpathrectangle{\pgfqpoint{1.250000in}{0.550000in}}{\pgfqpoint{7.750000in}{3.850000in}}%
\pgfusepath{clip}%
\pgfsetbuttcap%
\pgfsetmiterjoin%
\definecolor{currentfill}{rgb}{0.501961,0.501961,0.501961}%
\pgfsetfillcolor{currentfill}%
\pgfsetfillopacity{0.700000}%
\pgfsetlinewidth{0.000000pt}%
\definecolor{currentstroke}{rgb}{0.000000,0.000000,0.000000}%
\pgfsetstrokecolor{currentstroke}%
\pgfsetstrokeopacity{0.700000}%
\pgfsetdash{}{0pt}%
\pgfpathmoveto{\pgfqpoint{5.427734in}{0.550000in}}%
\pgfpathlineto{\pgfqpoint{5.455256in}{0.550000in}}%
\pgfpathlineto{\pgfqpoint{5.455256in}{0.697570in}}%
\pgfpathlineto{\pgfqpoint{5.427734in}{0.697570in}}%
\pgfpathlineto{\pgfqpoint{5.427734in}{0.550000in}}%
\pgfpathclose%
\pgfusepath{fill}%
\end{pgfscope}%
\begin{pgfscope}%
\pgfpathrectangle{\pgfqpoint{1.250000in}{0.550000in}}{\pgfqpoint{7.750000in}{3.850000in}}%
\pgfusepath{clip}%
\pgfsetbuttcap%
\pgfsetmiterjoin%
\definecolor{currentfill}{rgb}{0.501961,0.501961,0.501961}%
\pgfsetfillcolor{currentfill}%
\pgfsetfillopacity{0.700000}%
\pgfsetlinewidth{0.000000pt}%
\definecolor{currentstroke}{rgb}{0.000000,0.000000,0.000000}%
\pgfsetstrokecolor{currentstroke}%
\pgfsetstrokeopacity{0.700000}%
\pgfsetdash{}{0pt}%
\pgfpathmoveto{\pgfqpoint{5.455256in}{0.550000in}}%
\pgfpathlineto{\pgfqpoint{5.482777in}{0.550000in}}%
\pgfpathlineto{\pgfqpoint{5.482777in}{0.686282in}}%
\pgfpathlineto{\pgfqpoint{5.455256in}{0.686282in}}%
\pgfpathlineto{\pgfqpoint{5.455256in}{0.550000in}}%
\pgfpathclose%
\pgfusepath{fill}%
\end{pgfscope}%
\begin{pgfscope}%
\pgfpathrectangle{\pgfqpoint{1.250000in}{0.550000in}}{\pgfqpoint{7.750000in}{3.850000in}}%
\pgfusepath{clip}%
\pgfsetbuttcap%
\pgfsetmiterjoin%
\definecolor{currentfill}{rgb}{0.501961,0.501961,0.501961}%
\pgfsetfillcolor{currentfill}%
\pgfsetfillopacity{0.700000}%
\pgfsetlinewidth{0.000000pt}%
\definecolor{currentstroke}{rgb}{0.000000,0.000000,0.000000}%
\pgfsetstrokecolor{currentstroke}%
\pgfsetstrokeopacity{0.700000}%
\pgfsetdash{}{0pt}%
\pgfpathmoveto{\pgfqpoint{5.482777in}{0.550000in}}%
\pgfpathlineto{\pgfqpoint{5.510298in}{0.550000in}}%
\pgfpathlineto{\pgfqpoint{5.510298in}{0.682427in}}%
\pgfpathlineto{\pgfqpoint{5.482777in}{0.682427in}}%
\pgfpathlineto{\pgfqpoint{5.482777in}{0.550000in}}%
\pgfpathclose%
\pgfusepath{fill}%
\end{pgfscope}%
\begin{pgfscope}%
\pgfpathrectangle{\pgfqpoint{1.250000in}{0.550000in}}{\pgfqpoint{7.750000in}{3.850000in}}%
\pgfusepath{clip}%
\pgfsetbuttcap%
\pgfsetmiterjoin%
\definecolor{currentfill}{rgb}{0.501961,0.501961,0.501961}%
\pgfsetfillcolor{currentfill}%
\pgfsetfillopacity{0.700000}%
\pgfsetlinewidth{0.000000pt}%
\definecolor{currentstroke}{rgb}{0.000000,0.000000,0.000000}%
\pgfsetstrokecolor{currentstroke}%
\pgfsetstrokeopacity{0.700000}%
\pgfsetdash{}{0pt}%
\pgfpathmoveto{\pgfqpoint{5.510298in}{0.550000in}}%
\pgfpathlineto{\pgfqpoint{5.537820in}{0.550000in}}%
\pgfpathlineto{\pgfqpoint{5.537820in}{0.683529in}}%
\pgfpathlineto{\pgfqpoint{5.510298in}{0.683529in}}%
\pgfpathlineto{\pgfqpoint{5.510298in}{0.550000in}}%
\pgfpathclose%
\pgfusepath{fill}%
\end{pgfscope}%
\begin{pgfscope}%
\pgfpathrectangle{\pgfqpoint{1.250000in}{0.550000in}}{\pgfqpoint{7.750000in}{3.850000in}}%
\pgfusepath{clip}%
\pgfsetbuttcap%
\pgfsetmiterjoin%
\definecolor{currentfill}{rgb}{0.501961,0.501961,0.501961}%
\pgfsetfillcolor{currentfill}%
\pgfsetfillopacity{0.700000}%
\pgfsetlinewidth{0.000000pt}%
\definecolor{currentstroke}{rgb}{0.000000,0.000000,0.000000}%
\pgfsetstrokecolor{currentstroke}%
\pgfsetstrokeopacity{0.700000}%
\pgfsetdash{}{0pt}%
\pgfpathmoveto{\pgfqpoint{5.537820in}{0.550000in}}%
\pgfpathlineto{\pgfqpoint{5.565341in}{0.550000in}}%
\pgfpathlineto{\pgfqpoint{5.565341in}{0.684079in}}%
\pgfpathlineto{\pgfqpoint{5.537820in}{0.684079in}}%
\pgfpathlineto{\pgfqpoint{5.537820in}{0.550000in}}%
\pgfpathclose%
\pgfusepath{fill}%
\end{pgfscope}%
\begin{pgfscope}%
\pgfpathrectangle{\pgfqpoint{1.250000in}{0.550000in}}{\pgfqpoint{7.750000in}{3.850000in}}%
\pgfusepath{clip}%
\pgfsetbuttcap%
\pgfsetmiterjoin%
\definecolor{currentfill}{rgb}{0.501961,0.501961,0.501961}%
\pgfsetfillcolor{currentfill}%
\pgfsetfillopacity{0.700000}%
\pgfsetlinewidth{0.000000pt}%
\definecolor{currentstroke}{rgb}{0.000000,0.000000,0.000000}%
\pgfsetstrokecolor{currentstroke}%
\pgfsetstrokeopacity{0.700000}%
\pgfsetdash{}{0pt}%
\pgfpathmoveto{\pgfqpoint{5.565341in}{0.550000in}}%
\pgfpathlineto{\pgfqpoint{5.592862in}{0.550000in}}%
\pgfpathlineto{\pgfqpoint{5.592862in}{0.678022in}}%
\pgfpathlineto{\pgfqpoint{5.565341in}{0.678022in}}%
\pgfpathlineto{\pgfqpoint{5.565341in}{0.550000in}}%
\pgfpathclose%
\pgfusepath{fill}%
\end{pgfscope}%
\begin{pgfscope}%
\pgfpathrectangle{\pgfqpoint{1.250000in}{0.550000in}}{\pgfqpoint{7.750000in}{3.850000in}}%
\pgfusepath{clip}%
\pgfsetbuttcap%
\pgfsetmiterjoin%
\definecolor{currentfill}{rgb}{0.501961,0.501961,0.501961}%
\pgfsetfillcolor{currentfill}%
\pgfsetfillopacity{0.700000}%
\pgfsetlinewidth{0.000000pt}%
\definecolor{currentstroke}{rgb}{0.000000,0.000000,0.000000}%
\pgfsetstrokecolor{currentstroke}%
\pgfsetstrokeopacity{0.700000}%
\pgfsetdash{}{0pt}%
\pgfpathmoveto{\pgfqpoint{5.592862in}{0.550000in}}%
\pgfpathlineto{\pgfqpoint{5.620384in}{0.550000in}}%
\pgfpathlineto{\pgfqpoint{5.620384in}{0.692063in}}%
\pgfpathlineto{\pgfqpoint{5.592862in}{0.692063in}}%
\pgfpathlineto{\pgfqpoint{5.592862in}{0.550000in}}%
\pgfpathclose%
\pgfusepath{fill}%
\end{pgfscope}%
\begin{pgfscope}%
\pgfpathrectangle{\pgfqpoint{1.250000in}{0.550000in}}{\pgfqpoint{7.750000in}{3.850000in}}%
\pgfusepath{clip}%
\pgfsetbuttcap%
\pgfsetmiterjoin%
\definecolor{currentfill}{rgb}{0.501961,0.501961,0.501961}%
\pgfsetfillcolor{currentfill}%
\pgfsetfillopacity{0.700000}%
\pgfsetlinewidth{0.000000pt}%
\definecolor{currentstroke}{rgb}{0.000000,0.000000,0.000000}%
\pgfsetstrokecolor{currentstroke}%
\pgfsetstrokeopacity{0.700000}%
\pgfsetdash{}{0pt}%
\pgfpathmoveto{\pgfqpoint{5.620384in}{0.550000in}}%
\pgfpathlineto{\pgfqpoint{5.647905in}{0.550000in}}%
\pgfpathlineto{\pgfqpoint{5.647905in}{0.681051in}}%
\pgfpathlineto{\pgfqpoint{5.620384in}{0.681051in}}%
\pgfpathlineto{\pgfqpoint{5.620384in}{0.550000in}}%
\pgfpathclose%
\pgfusepath{fill}%
\end{pgfscope}%
\begin{pgfscope}%
\pgfpathrectangle{\pgfqpoint{1.250000in}{0.550000in}}{\pgfqpoint{7.750000in}{3.850000in}}%
\pgfusepath{clip}%
\pgfsetbuttcap%
\pgfsetmiterjoin%
\definecolor{currentfill}{rgb}{0.501961,0.501961,0.501961}%
\pgfsetfillcolor{currentfill}%
\pgfsetfillopacity{0.700000}%
\pgfsetlinewidth{0.000000pt}%
\definecolor{currentstroke}{rgb}{0.000000,0.000000,0.000000}%
\pgfsetstrokecolor{currentstroke}%
\pgfsetstrokeopacity{0.700000}%
\pgfsetdash{}{0pt}%
\pgfpathmoveto{\pgfqpoint{5.647905in}{0.550000in}}%
\pgfpathlineto{\pgfqpoint{5.675426in}{0.550000in}}%
\pgfpathlineto{\pgfqpoint{5.675426in}{0.697570in}}%
\pgfpathlineto{\pgfqpoint{5.647905in}{0.697570in}}%
\pgfpathlineto{\pgfqpoint{5.647905in}{0.550000in}}%
\pgfpathclose%
\pgfusepath{fill}%
\end{pgfscope}%
\begin{pgfscope}%
\pgfpathrectangle{\pgfqpoint{1.250000in}{0.550000in}}{\pgfqpoint{7.750000in}{3.850000in}}%
\pgfusepath{clip}%
\pgfsetbuttcap%
\pgfsetmiterjoin%
\definecolor{currentfill}{rgb}{0.501961,0.501961,0.501961}%
\pgfsetfillcolor{currentfill}%
\pgfsetfillopacity{0.700000}%
\pgfsetlinewidth{0.000000pt}%
\definecolor{currentstroke}{rgb}{0.000000,0.000000,0.000000}%
\pgfsetstrokecolor{currentstroke}%
\pgfsetstrokeopacity{0.700000}%
\pgfsetdash{}{0pt}%
\pgfpathmoveto{\pgfqpoint{5.675426in}{0.550000in}}%
\pgfpathlineto{\pgfqpoint{5.702947in}{0.550000in}}%
\pgfpathlineto{\pgfqpoint{5.702947in}{0.702250in}}%
\pgfpathlineto{\pgfqpoint{5.675426in}{0.702250in}}%
\pgfpathlineto{\pgfqpoint{5.675426in}{0.550000in}}%
\pgfpathclose%
\pgfusepath{fill}%
\end{pgfscope}%
\begin{pgfscope}%
\pgfpathrectangle{\pgfqpoint{1.250000in}{0.550000in}}{\pgfqpoint{7.750000in}{3.850000in}}%
\pgfusepath{clip}%
\pgfsetbuttcap%
\pgfsetmiterjoin%
\definecolor{currentfill}{rgb}{0.501961,0.501961,0.501961}%
\pgfsetfillcolor{currentfill}%
\pgfsetfillopacity{0.700000}%
\pgfsetlinewidth{0.000000pt}%
\definecolor{currentstroke}{rgb}{0.000000,0.000000,0.000000}%
\pgfsetstrokecolor{currentstroke}%
\pgfsetstrokeopacity{0.700000}%
\pgfsetdash{}{0pt}%
\pgfpathmoveto{\pgfqpoint{5.702947in}{0.550000in}}%
\pgfpathlineto{\pgfqpoint{5.730469in}{0.550000in}}%
\pgfpathlineto{\pgfqpoint{5.730469in}{0.697845in}}%
\pgfpathlineto{\pgfqpoint{5.702947in}{0.697845in}}%
\pgfpathlineto{\pgfqpoint{5.702947in}{0.550000in}}%
\pgfpathclose%
\pgfusepath{fill}%
\end{pgfscope}%
\begin{pgfscope}%
\pgfpathrectangle{\pgfqpoint{1.250000in}{0.550000in}}{\pgfqpoint{7.750000in}{3.850000in}}%
\pgfusepath{clip}%
\pgfsetbuttcap%
\pgfsetmiterjoin%
\definecolor{currentfill}{rgb}{0.501961,0.501961,0.501961}%
\pgfsetfillcolor{currentfill}%
\pgfsetfillopacity{0.700000}%
\pgfsetlinewidth{0.000000pt}%
\definecolor{currentstroke}{rgb}{0.000000,0.000000,0.000000}%
\pgfsetstrokecolor{currentstroke}%
\pgfsetstrokeopacity{0.700000}%
\pgfsetdash{}{0pt}%
\pgfpathmoveto{\pgfqpoint{5.730469in}{0.550000in}}%
\pgfpathlineto{\pgfqpoint{5.757990in}{0.550000in}}%
\pgfpathlineto{\pgfqpoint{5.757990in}{0.703627in}}%
\pgfpathlineto{\pgfqpoint{5.730469in}{0.703627in}}%
\pgfpathlineto{\pgfqpoint{5.730469in}{0.550000in}}%
\pgfpathclose%
\pgfusepath{fill}%
\end{pgfscope}%
\begin{pgfscope}%
\pgfpathrectangle{\pgfqpoint{1.250000in}{0.550000in}}{\pgfqpoint{7.750000in}{3.850000in}}%
\pgfusepath{clip}%
\pgfsetbuttcap%
\pgfsetmiterjoin%
\definecolor{currentfill}{rgb}{0.501961,0.501961,0.501961}%
\pgfsetfillcolor{currentfill}%
\pgfsetfillopacity{0.700000}%
\pgfsetlinewidth{0.000000pt}%
\definecolor{currentstroke}{rgb}{0.000000,0.000000,0.000000}%
\pgfsetstrokecolor{currentstroke}%
\pgfsetstrokeopacity{0.700000}%
\pgfsetdash{}{0pt}%
\pgfpathmoveto{\pgfqpoint{5.757990in}{0.550000in}}%
\pgfpathlineto{\pgfqpoint{5.785511in}{0.550000in}}%
\pgfpathlineto{\pgfqpoint{5.785511in}{0.709133in}}%
\pgfpathlineto{\pgfqpoint{5.757990in}{0.709133in}}%
\pgfpathlineto{\pgfqpoint{5.757990in}{0.550000in}}%
\pgfpathclose%
\pgfusepath{fill}%
\end{pgfscope}%
\begin{pgfscope}%
\pgfpathrectangle{\pgfqpoint{1.250000in}{0.550000in}}{\pgfqpoint{7.750000in}{3.850000in}}%
\pgfusepath{clip}%
\pgfsetbuttcap%
\pgfsetmiterjoin%
\definecolor{currentfill}{rgb}{0.501961,0.501961,0.501961}%
\pgfsetfillcolor{currentfill}%
\pgfsetfillopacity{0.700000}%
\pgfsetlinewidth{0.000000pt}%
\definecolor{currentstroke}{rgb}{0.000000,0.000000,0.000000}%
\pgfsetstrokecolor{currentstroke}%
\pgfsetstrokeopacity{0.700000}%
\pgfsetdash{}{0pt}%
\pgfpathmoveto{\pgfqpoint{5.785511in}{0.550000in}}%
\pgfpathlineto{\pgfqpoint{5.813033in}{0.550000in}}%
\pgfpathlineto{\pgfqpoint{5.813033in}{0.703351in}}%
\pgfpathlineto{\pgfqpoint{5.785511in}{0.703351in}}%
\pgfpathlineto{\pgfqpoint{5.785511in}{0.550000in}}%
\pgfpathclose%
\pgfusepath{fill}%
\end{pgfscope}%
\begin{pgfscope}%
\pgfpathrectangle{\pgfqpoint{1.250000in}{0.550000in}}{\pgfqpoint{7.750000in}{3.850000in}}%
\pgfusepath{clip}%
\pgfsetbuttcap%
\pgfsetmiterjoin%
\definecolor{currentfill}{rgb}{0.501961,0.501961,0.501961}%
\pgfsetfillcolor{currentfill}%
\pgfsetfillopacity{0.700000}%
\pgfsetlinewidth{0.000000pt}%
\definecolor{currentstroke}{rgb}{0.000000,0.000000,0.000000}%
\pgfsetstrokecolor{currentstroke}%
\pgfsetstrokeopacity{0.700000}%
\pgfsetdash{}{0pt}%
\pgfpathmoveto{\pgfqpoint{5.813033in}{0.550000in}}%
\pgfpathlineto{\pgfqpoint{5.840554in}{0.550000in}}%
\pgfpathlineto{\pgfqpoint{5.840554in}{0.717117in}}%
\pgfpathlineto{\pgfqpoint{5.813033in}{0.717117in}}%
\pgfpathlineto{\pgfqpoint{5.813033in}{0.550000in}}%
\pgfpathclose%
\pgfusepath{fill}%
\end{pgfscope}%
\begin{pgfscope}%
\pgfpathrectangle{\pgfqpoint{1.250000in}{0.550000in}}{\pgfqpoint{7.750000in}{3.850000in}}%
\pgfusepath{clip}%
\pgfsetbuttcap%
\pgfsetmiterjoin%
\definecolor{currentfill}{rgb}{0.501961,0.501961,0.501961}%
\pgfsetfillcolor{currentfill}%
\pgfsetfillopacity{0.700000}%
\pgfsetlinewidth{0.000000pt}%
\definecolor{currentstroke}{rgb}{0.000000,0.000000,0.000000}%
\pgfsetstrokecolor{currentstroke}%
\pgfsetstrokeopacity{0.700000}%
\pgfsetdash{}{0pt}%
\pgfpathmoveto{\pgfqpoint{5.840554in}{0.550000in}}%
\pgfpathlineto{\pgfqpoint{5.868075in}{0.550000in}}%
\pgfpathlineto{\pgfqpoint{5.868075in}{0.711060in}}%
\pgfpathlineto{\pgfqpoint{5.840554in}{0.711060in}}%
\pgfpathlineto{\pgfqpoint{5.840554in}{0.550000in}}%
\pgfpathclose%
\pgfusepath{fill}%
\end{pgfscope}%
\begin{pgfscope}%
\pgfpathrectangle{\pgfqpoint{1.250000in}{0.550000in}}{\pgfqpoint{7.750000in}{3.850000in}}%
\pgfusepath{clip}%
\pgfsetbuttcap%
\pgfsetmiterjoin%
\definecolor{currentfill}{rgb}{0.501961,0.501961,0.501961}%
\pgfsetfillcolor{currentfill}%
\pgfsetfillopacity{0.700000}%
\pgfsetlinewidth{0.000000pt}%
\definecolor{currentstroke}{rgb}{0.000000,0.000000,0.000000}%
\pgfsetstrokecolor{currentstroke}%
\pgfsetstrokeopacity{0.700000}%
\pgfsetdash{}{0pt}%
\pgfpathmoveto{\pgfqpoint{5.868075in}{0.550000in}}%
\pgfpathlineto{\pgfqpoint{5.895597in}{0.550000in}}%
\pgfpathlineto{\pgfqpoint{5.895597in}{0.701975in}}%
\pgfpathlineto{\pgfqpoint{5.868075in}{0.701975in}}%
\pgfpathlineto{\pgfqpoint{5.868075in}{0.550000in}}%
\pgfpathclose%
\pgfusepath{fill}%
\end{pgfscope}%
\begin{pgfscope}%
\pgfpathrectangle{\pgfqpoint{1.250000in}{0.550000in}}{\pgfqpoint{7.750000in}{3.850000in}}%
\pgfusepath{clip}%
\pgfsetbuttcap%
\pgfsetmiterjoin%
\definecolor{currentfill}{rgb}{0.501961,0.501961,0.501961}%
\pgfsetfillcolor{currentfill}%
\pgfsetfillopacity{0.700000}%
\pgfsetlinewidth{0.000000pt}%
\definecolor{currentstroke}{rgb}{0.000000,0.000000,0.000000}%
\pgfsetstrokecolor{currentstroke}%
\pgfsetstrokeopacity{0.700000}%
\pgfsetdash{}{0pt}%
\pgfpathmoveto{\pgfqpoint{5.895597in}{0.550000in}}%
\pgfpathlineto{\pgfqpoint{5.923118in}{0.550000in}}%
\pgfpathlineto{\pgfqpoint{5.923118in}{0.704728in}}%
\pgfpathlineto{\pgfqpoint{5.895597in}{0.704728in}}%
\pgfpathlineto{\pgfqpoint{5.895597in}{0.550000in}}%
\pgfpathclose%
\pgfusepath{fill}%
\end{pgfscope}%
\begin{pgfscope}%
\pgfpathrectangle{\pgfqpoint{1.250000in}{0.550000in}}{\pgfqpoint{7.750000in}{3.850000in}}%
\pgfusepath{clip}%
\pgfsetbuttcap%
\pgfsetmiterjoin%
\definecolor{currentfill}{rgb}{0.501961,0.501961,0.501961}%
\pgfsetfillcolor{currentfill}%
\pgfsetfillopacity{0.700000}%
\pgfsetlinewidth{0.000000pt}%
\definecolor{currentstroke}{rgb}{0.000000,0.000000,0.000000}%
\pgfsetstrokecolor{currentstroke}%
\pgfsetstrokeopacity{0.700000}%
\pgfsetdash{}{0pt}%
\pgfpathmoveto{\pgfqpoint{5.923118in}{0.550000in}}%
\pgfpathlineto{\pgfqpoint{5.950639in}{0.550000in}}%
\pgfpathlineto{\pgfqpoint{5.950639in}{0.705829in}}%
\pgfpathlineto{\pgfqpoint{5.923118in}{0.705829in}}%
\pgfpathlineto{\pgfqpoint{5.923118in}{0.550000in}}%
\pgfpathclose%
\pgfusepath{fill}%
\end{pgfscope}%
\begin{pgfscope}%
\pgfpathrectangle{\pgfqpoint{1.250000in}{0.550000in}}{\pgfqpoint{7.750000in}{3.850000in}}%
\pgfusepath{clip}%
\pgfsetbuttcap%
\pgfsetmiterjoin%
\definecolor{currentfill}{rgb}{0.501961,0.501961,0.501961}%
\pgfsetfillcolor{currentfill}%
\pgfsetfillopacity{0.700000}%
\pgfsetlinewidth{0.000000pt}%
\definecolor{currentstroke}{rgb}{0.000000,0.000000,0.000000}%
\pgfsetstrokecolor{currentstroke}%
\pgfsetstrokeopacity{0.700000}%
\pgfsetdash{}{0pt}%
\pgfpathmoveto{\pgfqpoint{5.950639in}{0.550000in}}%
\pgfpathlineto{\pgfqpoint{5.978161in}{0.550000in}}%
\pgfpathlineto{\pgfqpoint{5.978161in}{0.720972in}}%
\pgfpathlineto{\pgfqpoint{5.950639in}{0.720972in}}%
\pgfpathlineto{\pgfqpoint{5.950639in}{0.550000in}}%
\pgfpathclose%
\pgfusepath{fill}%
\end{pgfscope}%
\begin{pgfscope}%
\pgfpathrectangle{\pgfqpoint{1.250000in}{0.550000in}}{\pgfqpoint{7.750000in}{3.850000in}}%
\pgfusepath{clip}%
\pgfsetbuttcap%
\pgfsetmiterjoin%
\definecolor{currentfill}{rgb}{0.501961,0.501961,0.501961}%
\pgfsetfillcolor{currentfill}%
\pgfsetfillopacity{0.700000}%
\pgfsetlinewidth{0.000000pt}%
\definecolor{currentstroke}{rgb}{0.000000,0.000000,0.000000}%
\pgfsetstrokecolor{currentstroke}%
\pgfsetstrokeopacity{0.700000}%
\pgfsetdash{}{0pt}%
\pgfpathmoveto{\pgfqpoint{5.978161in}{0.550000in}}%
\pgfpathlineto{\pgfqpoint{6.005682in}{0.550000in}}%
\pgfpathlineto{\pgfqpoint{6.005682in}{0.697294in}}%
\pgfpathlineto{\pgfqpoint{5.978161in}{0.697294in}}%
\pgfpathlineto{\pgfqpoint{5.978161in}{0.550000in}}%
\pgfpathclose%
\pgfusepath{fill}%
\end{pgfscope}%
\begin{pgfscope}%
\pgfpathrectangle{\pgfqpoint{1.250000in}{0.550000in}}{\pgfqpoint{7.750000in}{3.850000in}}%
\pgfusepath{clip}%
\pgfsetbuttcap%
\pgfsetmiterjoin%
\definecolor{currentfill}{rgb}{0.501961,0.501961,0.501961}%
\pgfsetfillcolor{currentfill}%
\pgfsetfillopacity{0.700000}%
\pgfsetlinewidth{0.000000pt}%
\definecolor{currentstroke}{rgb}{0.000000,0.000000,0.000000}%
\pgfsetstrokecolor{currentstroke}%
\pgfsetstrokeopacity{0.700000}%
\pgfsetdash{}{0pt}%
\pgfpathmoveto{\pgfqpoint{6.005682in}{0.550000in}}%
\pgfpathlineto{\pgfqpoint{6.033203in}{0.550000in}}%
\pgfpathlineto{\pgfqpoint{6.033203in}{0.688484in}}%
\pgfpathlineto{\pgfqpoint{6.005682in}{0.688484in}}%
\pgfpathlineto{\pgfqpoint{6.005682in}{0.550000in}}%
\pgfpathclose%
\pgfusepath{fill}%
\end{pgfscope}%
\begin{pgfscope}%
\pgfpathrectangle{\pgfqpoint{1.250000in}{0.550000in}}{\pgfqpoint{7.750000in}{3.850000in}}%
\pgfusepath{clip}%
\pgfsetbuttcap%
\pgfsetmiterjoin%
\definecolor{currentfill}{rgb}{0.501961,0.501961,0.501961}%
\pgfsetfillcolor{currentfill}%
\pgfsetfillopacity{0.700000}%
\pgfsetlinewidth{0.000000pt}%
\definecolor{currentstroke}{rgb}{0.000000,0.000000,0.000000}%
\pgfsetstrokecolor{currentstroke}%
\pgfsetstrokeopacity{0.700000}%
\pgfsetdash{}{0pt}%
\pgfpathmoveto{\pgfqpoint{6.033203in}{0.550000in}}%
\pgfpathlineto{\pgfqpoint{6.060724in}{0.550000in}}%
\pgfpathlineto{\pgfqpoint{6.060724in}{0.681877in}}%
\pgfpathlineto{\pgfqpoint{6.033203in}{0.681877in}}%
\pgfpathlineto{\pgfqpoint{6.033203in}{0.550000in}}%
\pgfpathclose%
\pgfusepath{fill}%
\end{pgfscope}%
\begin{pgfscope}%
\pgfpathrectangle{\pgfqpoint{1.250000in}{0.550000in}}{\pgfqpoint{7.750000in}{3.850000in}}%
\pgfusepath{clip}%
\pgfsetbuttcap%
\pgfsetmiterjoin%
\definecolor{currentfill}{rgb}{0.501961,0.501961,0.501961}%
\pgfsetfillcolor{currentfill}%
\pgfsetfillopacity{0.700000}%
\pgfsetlinewidth{0.000000pt}%
\definecolor{currentstroke}{rgb}{0.000000,0.000000,0.000000}%
\pgfsetstrokecolor{currentstroke}%
\pgfsetstrokeopacity{0.700000}%
\pgfsetdash{}{0pt}%
\pgfpathmoveto{\pgfqpoint{6.060724in}{0.550000in}}%
\pgfpathlineto{\pgfqpoint{6.088246in}{0.550000in}}%
\pgfpathlineto{\pgfqpoint{6.088246in}{0.675269in}}%
\pgfpathlineto{\pgfqpoint{6.060724in}{0.675269in}}%
\pgfpathlineto{\pgfqpoint{6.060724in}{0.550000in}}%
\pgfpathclose%
\pgfusepath{fill}%
\end{pgfscope}%
\begin{pgfscope}%
\pgfpathrectangle{\pgfqpoint{1.250000in}{0.550000in}}{\pgfqpoint{7.750000in}{3.850000in}}%
\pgfusepath{clip}%
\pgfsetbuttcap%
\pgfsetmiterjoin%
\definecolor{currentfill}{rgb}{0.501961,0.501961,0.501961}%
\pgfsetfillcolor{currentfill}%
\pgfsetfillopacity{0.700000}%
\pgfsetlinewidth{0.000000pt}%
\definecolor{currentstroke}{rgb}{0.000000,0.000000,0.000000}%
\pgfsetstrokecolor{currentstroke}%
\pgfsetstrokeopacity{0.700000}%
\pgfsetdash{}{0pt}%
\pgfpathmoveto{\pgfqpoint{6.088246in}{0.550000in}}%
\pgfpathlineto{\pgfqpoint{6.115767in}{0.550000in}}%
\pgfpathlineto{\pgfqpoint{6.115767in}{0.667285in}}%
\pgfpathlineto{\pgfqpoint{6.088246in}{0.667285in}}%
\pgfpathlineto{\pgfqpoint{6.088246in}{0.550000in}}%
\pgfpathclose%
\pgfusepath{fill}%
\end{pgfscope}%
\begin{pgfscope}%
\pgfpathrectangle{\pgfqpoint{1.250000in}{0.550000in}}{\pgfqpoint{7.750000in}{3.850000in}}%
\pgfusepath{clip}%
\pgfsetbuttcap%
\pgfsetmiterjoin%
\definecolor{currentfill}{rgb}{0.501961,0.501961,0.501961}%
\pgfsetfillcolor{currentfill}%
\pgfsetfillopacity{0.700000}%
\pgfsetlinewidth{0.000000pt}%
\definecolor{currentstroke}{rgb}{0.000000,0.000000,0.000000}%
\pgfsetstrokecolor{currentstroke}%
\pgfsetstrokeopacity{0.700000}%
\pgfsetdash{}{0pt}%
\pgfpathmoveto{\pgfqpoint{6.115767in}{0.550000in}}%
\pgfpathlineto{\pgfqpoint{6.143288in}{0.550000in}}%
\pgfpathlineto{\pgfqpoint{6.143288in}{0.662604in}}%
\pgfpathlineto{\pgfqpoint{6.115767in}{0.662604in}}%
\pgfpathlineto{\pgfqpoint{6.115767in}{0.550000in}}%
\pgfpathclose%
\pgfusepath{fill}%
\end{pgfscope}%
\begin{pgfscope}%
\pgfpathrectangle{\pgfqpoint{1.250000in}{0.550000in}}{\pgfqpoint{7.750000in}{3.850000in}}%
\pgfusepath{clip}%
\pgfsetbuttcap%
\pgfsetmiterjoin%
\definecolor{currentfill}{rgb}{0.501961,0.501961,0.501961}%
\pgfsetfillcolor{currentfill}%
\pgfsetfillopacity{0.700000}%
\pgfsetlinewidth{0.000000pt}%
\definecolor{currentstroke}{rgb}{0.000000,0.000000,0.000000}%
\pgfsetstrokecolor{currentstroke}%
\pgfsetstrokeopacity{0.700000}%
\pgfsetdash{}{0pt}%
\pgfpathmoveto{\pgfqpoint{6.143288in}{0.550000in}}%
\pgfpathlineto{\pgfqpoint{6.170810in}{0.550000in}}%
\pgfpathlineto{\pgfqpoint{6.170810in}{0.665082in}}%
\pgfpathlineto{\pgfqpoint{6.143288in}{0.665082in}}%
\pgfpathlineto{\pgfqpoint{6.143288in}{0.550000in}}%
\pgfpathclose%
\pgfusepath{fill}%
\end{pgfscope}%
\begin{pgfscope}%
\pgfpathrectangle{\pgfqpoint{1.250000in}{0.550000in}}{\pgfqpoint{7.750000in}{3.850000in}}%
\pgfusepath{clip}%
\pgfsetbuttcap%
\pgfsetmiterjoin%
\definecolor{currentfill}{rgb}{0.501961,0.501961,0.501961}%
\pgfsetfillcolor{currentfill}%
\pgfsetfillopacity{0.700000}%
\pgfsetlinewidth{0.000000pt}%
\definecolor{currentstroke}{rgb}{0.000000,0.000000,0.000000}%
\pgfsetstrokecolor{currentstroke}%
\pgfsetstrokeopacity{0.700000}%
\pgfsetdash{}{0pt}%
\pgfpathmoveto{\pgfqpoint{6.170810in}{0.550000in}}%
\pgfpathlineto{\pgfqpoint{6.198331in}{0.550000in}}%
\pgfpathlineto{\pgfqpoint{6.198331in}{0.672241in}}%
\pgfpathlineto{\pgfqpoint{6.170810in}{0.672241in}}%
\pgfpathlineto{\pgfqpoint{6.170810in}{0.550000in}}%
\pgfpathclose%
\pgfusepath{fill}%
\end{pgfscope}%
\begin{pgfscope}%
\pgfpathrectangle{\pgfqpoint{1.250000in}{0.550000in}}{\pgfqpoint{7.750000in}{3.850000in}}%
\pgfusepath{clip}%
\pgfsetbuttcap%
\pgfsetmiterjoin%
\definecolor{currentfill}{rgb}{0.501961,0.501961,0.501961}%
\pgfsetfillcolor{currentfill}%
\pgfsetfillopacity{0.700000}%
\pgfsetlinewidth{0.000000pt}%
\definecolor{currentstroke}{rgb}{0.000000,0.000000,0.000000}%
\pgfsetstrokecolor{currentstroke}%
\pgfsetstrokeopacity{0.700000}%
\pgfsetdash{}{0pt}%
\pgfpathmoveto{\pgfqpoint{6.198331in}{0.550000in}}%
\pgfpathlineto{\pgfqpoint{6.225852in}{0.550000in}}%
\pgfpathlineto{\pgfqpoint{6.225852in}{0.654345in}}%
\pgfpathlineto{\pgfqpoint{6.198331in}{0.654345in}}%
\pgfpathlineto{\pgfqpoint{6.198331in}{0.550000in}}%
\pgfpathclose%
\pgfusepath{fill}%
\end{pgfscope}%
\begin{pgfscope}%
\pgfpathrectangle{\pgfqpoint{1.250000in}{0.550000in}}{\pgfqpoint{7.750000in}{3.850000in}}%
\pgfusepath{clip}%
\pgfsetbuttcap%
\pgfsetmiterjoin%
\definecolor{currentfill}{rgb}{0.501961,0.501961,0.501961}%
\pgfsetfillcolor{currentfill}%
\pgfsetfillopacity{0.700000}%
\pgfsetlinewidth{0.000000pt}%
\definecolor{currentstroke}{rgb}{0.000000,0.000000,0.000000}%
\pgfsetstrokecolor{currentstroke}%
\pgfsetstrokeopacity{0.700000}%
\pgfsetdash{}{0pt}%
\pgfpathmoveto{\pgfqpoint{6.225852in}{0.550000in}}%
\pgfpathlineto{\pgfqpoint{6.253374in}{0.550000in}}%
\pgfpathlineto{\pgfqpoint{6.253374in}{0.648563in}}%
\pgfpathlineto{\pgfqpoint{6.225852in}{0.648563in}}%
\pgfpathlineto{\pgfqpoint{6.225852in}{0.550000in}}%
\pgfpathclose%
\pgfusepath{fill}%
\end{pgfscope}%
\begin{pgfscope}%
\pgfpathrectangle{\pgfqpoint{1.250000in}{0.550000in}}{\pgfqpoint{7.750000in}{3.850000in}}%
\pgfusepath{clip}%
\pgfsetbuttcap%
\pgfsetmiterjoin%
\definecolor{currentfill}{rgb}{0.501961,0.501961,0.501961}%
\pgfsetfillcolor{currentfill}%
\pgfsetfillopacity{0.700000}%
\pgfsetlinewidth{0.000000pt}%
\definecolor{currentstroke}{rgb}{0.000000,0.000000,0.000000}%
\pgfsetstrokecolor{currentstroke}%
\pgfsetstrokeopacity{0.700000}%
\pgfsetdash{}{0pt}%
\pgfpathmoveto{\pgfqpoint{6.253374in}{0.550000in}}%
\pgfpathlineto{\pgfqpoint{6.280895in}{0.550000in}}%
\pgfpathlineto{\pgfqpoint{6.280895in}{0.646085in}}%
\pgfpathlineto{\pgfqpoint{6.253374in}{0.646085in}}%
\pgfpathlineto{\pgfqpoint{6.253374in}{0.550000in}}%
\pgfpathclose%
\pgfusepath{fill}%
\end{pgfscope}%
\begin{pgfscope}%
\pgfpathrectangle{\pgfqpoint{1.250000in}{0.550000in}}{\pgfqpoint{7.750000in}{3.850000in}}%
\pgfusepath{clip}%
\pgfsetbuttcap%
\pgfsetmiterjoin%
\definecolor{currentfill}{rgb}{0.501961,0.501961,0.501961}%
\pgfsetfillcolor{currentfill}%
\pgfsetfillopacity{0.700000}%
\pgfsetlinewidth{0.000000pt}%
\definecolor{currentstroke}{rgb}{0.000000,0.000000,0.000000}%
\pgfsetstrokecolor{currentstroke}%
\pgfsetstrokeopacity{0.700000}%
\pgfsetdash{}{0pt}%
\pgfpathmoveto{\pgfqpoint{6.280895in}{0.550000in}}%
\pgfpathlineto{\pgfqpoint{6.308416in}{0.550000in}}%
\pgfpathlineto{\pgfqpoint{6.308416in}{0.643608in}}%
\pgfpathlineto{\pgfqpoint{6.280895in}{0.643608in}}%
\pgfpathlineto{\pgfqpoint{6.280895in}{0.550000in}}%
\pgfpathclose%
\pgfusepath{fill}%
\end{pgfscope}%
\begin{pgfscope}%
\pgfpathrectangle{\pgfqpoint{1.250000in}{0.550000in}}{\pgfqpoint{7.750000in}{3.850000in}}%
\pgfusepath{clip}%
\pgfsetbuttcap%
\pgfsetmiterjoin%
\definecolor{currentfill}{rgb}{0.501961,0.501961,0.501961}%
\pgfsetfillcolor{currentfill}%
\pgfsetfillopacity{0.700000}%
\pgfsetlinewidth{0.000000pt}%
\definecolor{currentstroke}{rgb}{0.000000,0.000000,0.000000}%
\pgfsetstrokecolor{currentstroke}%
\pgfsetstrokeopacity{0.700000}%
\pgfsetdash{}{0pt}%
\pgfpathmoveto{\pgfqpoint{6.308416in}{0.550000in}}%
\pgfpathlineto{\pgfqpoint{6.335937in}{0.550000in}}%
\pgfpathlineto{\pgfqpoint{6.335937in}{0.642506in}}%
\pgfpathlineto{\pgfqpoint{6.308416in}{0.642506in}}%
\pgfpathlineto{\pgfqpoint{6.308416in}{0.550000in}}%
\pgfpathclose%
\pgfusepath{fill}%
\end{pgfscope}%
\begin{pgfscope}%
\pgfpathrectangle{\pgfqpoint{1.250000in}{0.550000in}}{\pgfqpoint{7.750000in}{3.850000in}}%
\pgfusepath{clip}%
\pgfsetbuttcap%
\pgfsetmiterjoin%
\definecolor{currentfill}{rgb}{0.501961,0.501961,0.501961}%
\pgfsetfillcolor{currentfill}%
\pgfsetfillopacity{0.700000}%
\pgfsetlinewidth{0.000000pt}%
\definecolor{currentstroke}{rgb}{0.000000,0.000000,0.000000}%
\pgfsetstrokecolor{currentstroke}%
\pgfsetstrokeopacity{0.700000}%
\pgfsetdash{}{0pt}%
\pgfpathmoveto{\pgfqpoint{6.335938in}{0.550000in}}%
\pgfpathlineto{\pgfqpoint{6.363459in}{0.550000in}}%
\pgfpathlineto{\pgfqpoint{6.363459in}{0.635348in}}%
\pgfpathlineto{\pgfqpoint{6.335938in}{0.635348in}}%
\pgfpathlineto{\pgfqpoint{6.335938in}{0.550000in}}%
\pgfpathclose%
\pgfusepath{fill}%
\end{pgfscope}%
\begin{pgfscope}%
\pgfpathrectangle{\pgfqpoint{1.250000in}{0.550000in}}{\pgfqpoint{7.750000in}{3.850000in}}%
\pgfusepath{clip}%
\pgfsetbuttcap%
\pgfsetmiterjoin%
\definecolor{currentfill}{rgb}{0.501961,0.501961,0.501961}%
\pgfsetfillcolor{currentfill}%
\pgfsetfillopacity{0.700000}%
\pgfsetlinewidth{0.000000pt}%
\definecolor{currentstroke}{rgb}{0.000000,0.000000,0.000000}%
\pgfsetstrokecolor{currentstroke}%
\pgfsetstrokeopacity{0.700000}%
\pgfsetdash{}{0pt}%
\pgfpathmoveto{\pgfqpoint{6.363459in}{0.550000in}}%
\pgfpathlineto{\pgfqpoint{6.390980in}{0.550000in}}%
\pgfpathlineto{\pgfqpoint{6.390980in}{0.636449in}}%
\pgfpathlineto{\pgfqpoint{6.363459in}{0.636449in}}%
\pgfpathlineto{\pgfqpoint{6.363459in}{0.550000in}}%
\pgfpathclose%
\pgfusepath{fill}%
\end{pgfscope}%
\begin{pgfscope}%
\pgfpathrectangle{\pgfqpoint{1.250000in}{0.550000in}}{\pgfqpoint{7.750000in}{3.850000in}}%
\pgfusepath{clip}%
\pgfsetbuttcap%
\pgfsetmiterjoin%
\definecolor{currentfill}{rgb}{0.501961,0.501961,0.501961}%
\pgfsetfillcolor{currentfill}%
\pgfsetfillopacity{0.700000}%
\pgfsetlinewidth{0.000000pt}%
\definecolor{currentstroke}{rgb}{0.000000,0.000000,0.000000}%
\pgfsetstrokecolor{currentstroke}%
\pgfsetstrokeopacity{0.700000}%
\pgfsetdash{}{0pt}%
\pgfpathmoveto{\pgfqpoint{6.390980in}{0.550000in}}%
\pgfpathlineto{\pgfqpoint{6.418501in}{0.550000in}}%
\pgfpathlineto{\pgfqpoint{6.418501in}{0.619655in}}%
\pgfpathlineto{\pgfqpoint{6.390980in}{0.619655in}}%
\pgfpathlineto{\pgfqpoint{6.390980in}{0.550000in}}%
\pgfpathclose%
\pgfusepath{fill}%
\end{pgfscope}%
\begin{pgfscope}%
\pgfpathrectangle{\pgfqpoint{1.250000in}{0.550000in}}{\pgfqpoint{7.750000in}{3.850000in}}%
\pgfusepath{clip}%
\pgfsetbuttcap%
\pgfsetmiterjoin%
\definecolor{currentfill}{rgb}{0.501961,0.501961,0.501961}%
\pgfsetfillcolor{currentfill}%
\pgfsetfillopacity{0.700000}%
\pgfsetlinewidth{0.000000pt}%
\definecolor{currentstroke}{rgb}{0.000000,0.000000,0.000000}%
\pgfsetstrokecolor{currentstroke}%
\pgfsetstrokeopacity{0.700000}%
\pgfsetdash{}{0pt}%
\pgfpathmoveto{\pgfqpoint{6.418501in}{0.550000in}}%
\pgfpathlineto{\pgfqpoint{6.446023in}{0.550000in}}%
\pgfpathlineto{\pgfqpoint{6.446023in}{0.624886in}}%
\pgfpathlineto{\pgfqpoint{6.418501in}{0.624886in}}%
\pgfpathlineto{\pgfqpoint{6.418501in}{0.550000in}}%
\pgfpathclose%
\pgfusepath{fill}%
\end{pgfscope}%
\begin{pgfscope}%
\pgfpathrectangle{\pgfqpoint{1.250000in}{0.550000in}}{\pgfqpoint{7.750000in}{3.850000in}}%
\pgfusepath{clip}%
\pgfsetbuttcap%
\pgfsetmiterjoin%
\definecolor{currentfill}{rgb}{0.501961,0.501961,0.501961}%
\pgfsetfillcolor{currentfill}%
\pgfsetfillopacity{0.700000}%
\pgfsetlinewidth{0.000000pt}%
\definecolor{currentstroke}{rgb}{0.000000,0.000000,0.000000}%
\pgfsetstrokecolor{currentstroke}%
\pgfsetstrokeopacity{0.700000}%
\pgfsetdash{}{0pt}%
\pgfpathmoveto{\pgfqpoint{6.446023in}{0.550000in}}%
\pgfpathlineto{\pgfqpoint{6.473544in}{0.550000in}}%
\pgfpathlineto{\pgfqpoint{6.473544in}{0.616627in}}%
\pgfpathlineto{\pgfqpoint{6.446023in}{0.616627in}}%
\pgfpathlineto{\pgfqpoint{6.446023in}{0.550000in}}%
\pgfpathclose%
\pgfusepath{fill}%
\end{pgfscope}%
\begin{pgfscope}%
\pgfpathrectangle{\pgfqpoint{1.250000in}{0.550000in}}{\pgfqpoint{7.750000in}{3.850000in}}%
\pgfusepath{clip}%
\pgfsetbuttcap%
\pgfsetmiterjoin%
\definecolor{currentfill}{rgb}{0.501961,0.501961,0.501961}%
\pgfsetfillcolor{currentfill}%
\pgfsetfillopacity{0.700000}%
\pgfsetlinewidth{0.000000pt}%
\definecolor{currentstroke}{rgb}{0.000000,0.000000,0.000000}%
\pgfsetstrokecolor{currentstroke}%
\pgfsetstrokeopacity{0.700000}%
\pgfsetdash{}{0pt}%
\pgfpathmoveto{\pgfqpoint{6.473544in}{0.550000in}}%
\pgfpathlineto{\pgfqpoint{6.501065in}{0.550000in}}%
\pgfpathlineto{\pgfqpoint{6.501065in}{0.621582in}}%
\pgfpathlineto{\pgfqpoint{6.473544in}{0.621582in}}%
\pgfpathlineto{\pgfqpoint{6.473544in}{0.550000in}}%
\pgfpathclose%
\pgfusepath{fill}%
\end{pgfscope}%
\begin{pgfscope}%
\pgfpathrectangle{\pgfqpoint{1.250000in}{0.550000in}}{\pgfqpoint{7.750000in}{3.850000in}}%
\pgfusepath{clip}%
\pgfsetbuttcap%
\pgfsetmiterjoin%
\definecolor{currentfill}{rgb}{0.501961,0.501961,0.501961}%
\pgfsetfillcolor{currentfill}%
\pgfsetfillopacity{0.700000}%
\pgfsetlinewidth{0.000000pt}%
\definecolor{currentstroke}{rgb}{0.000000,0.000000,0.000000}%
\pgfsetstrokecolor{currentstroke}%
\pgfsetstrokeopacity{0.700000}%
\pgfsetdash{}{0pt}%
\pgfpathmoveto{\pgfqpoint{6.501065in}{0.550000in}}%
\pgfpathlineto{\pgfqpoint{6.528587in}{0.550000in}}%
\pgfpathlineto{\pgfqpoint{6.528587in}{0.608367in}}%
\pgfpathlineto{\pgfqpoint{6.501065in}{0.608367in}}%
\pgfpathlineto{\pgfqpoint{6.501065in}{0.550000in}}%
\pgfpathclose%
\pgfusepath{fill}%
\end{pgfscope}%
\begin{pgfscope}%
\pgfpathrectangle{\pgfqpoint{1.250000in}{0.550000in}}{\pgfqpoint{7.750000in}{3.850000in}}%
\pgfusepath{clip}%
\pgfsetbuttcap%
\pgfsetmiterjoin%
\definecolor{currentfill}{rgb}{0.501961,0.501961,0.501961}%
\pgfsetfillcolor{currentfill}%
\pgfsetfillopacity{0.700000}%
\pgfsetlinewidth{0.000000pt}%
\definecolor{currentstroke}{rgb}{0.000000,0.000000,0.000000}%
\pgfsetstrokecolor{currentstroke}%
\pgfsetstrokeopacity{0.700000}%
\pgfsetdash{}{0pt}%
\pgfpathmoveto{\pgfqpoint{6.528587in}{0.550000in}}%
\pgfpathlineto{\pgfqpoint{6.556108in}{0.550000in}}%
\pgfpathlineto{\pgfqpoint{6.556108in}{0.604788in}}%
\pgfpathlineto{\pgfqpoint{6.528587in}{0.604788in}}%
\pgfpathlineto{\pgfqpoint{6.528587in}{0.550000in}}%
\pgfpathclose%
\pgfusepath{fill}%
\end{pgfscope}%
\begin{pgfscope}%
\pgfpathrectangle{\pgfqpoint{1.250000in}{0.550000in}}{\pgfqpoint{7.750000in}{3.850000in}}%
\pgfusepath{clip}%
\pgfsetbuttcap%
\pgfsetmiterjoin%
\definecolor{currentfill}{rgb}{0.501961,0.501961,0.501961}%
\pgfsetfillcolor{currentfill}%
\pgfsetfillopacity{0.700000}%
\pgfsetlinewidth{0.000000pt}%
\definecolor{currentstroke}{rgb}{0.000000,0.000000,0.000000}%
\pgfsetstrokecolor{currentstroke}%
\pgfsetstrokeopacity{0.700000}%
\pgfsetdash{}{0pt}%
\pgfpathmoveto{\pgfqpoint{6.556108in}{0.550000in}}%
\pgfpathlineto{\pgfqpoint{6.583629in}{0.550000in}}%
\pgfpathlineto{\pgfqpoint{6.583629in}{0.611396in}}%
\pgfpathlineto{\pgfqpoint{6.556108in}{0.611396in}}%
\pgfpathlineto{\pgfqpoint{6.556108in}{0.550000in}}%
\pgfpathclose%
\pgfusepath{fill}%
\end{pgfscope}%
\begin{pgfscope}%
\pgfpathrectangle{\pgfqpoint{1.250000in}{0.550000in}}{\pgfqpoint{7.750000in}{3.850000in}}%
\pgfusepath{clip}%
\pgfsetbuttcap%
\pgfsetmiterjoin%
\definecolor{currentfill}{rgb}{0.501961,0.501961,0.501961}%
\pgfsetfillcolor{currentfill}%
\pgfsetfillopacity{0.700000}%
\pgfsetlinewidth{0.000000pt}%
\definecolor{currentstroke}{rgb}{0.000000,0.000000,0.000000}%
\pgfsetstrokecolor{currentstroke}%
\pgfsetstrokeopacity{0.700000}%
\pgfsetdash{}{0pt}%
\pgfpathmoveto{\pgfqpoint{6.583629in}{0.550000in}}%
\pgfpathlineto{\pgfqpoint{6.611151in}{0.550000in}}%
\pgfpathlineto{\pgfqpoint{6.611151in}{0.598180in}}%
\pgfpathlineto{\pgfqpoint{6.583629in}{0.598180in}}%
\pgfpathlineto{\pgfqpoint{6.583629in}{0.550000in}}%
\pgfpathclose%
\pgfusepath{fill}%
\end{pgfscope}%
\begin{pgfscope}%
\pgfpathrectangle{\pgfqpoint{1.250000in}{0.550000in}}{\pgfqpoint{7.750000in}{3.850000in}}%
\pgfusepath{clip}%
\pgfsetbuttcap%
\pgfsetmiterjoin%
\definecolor{currentfill}{rgb}{0.501961,0.501961,0.501961}%
\pgfsetfillcolor{currentfill}%
\pgfsetfillopacity{0.700000}%
\pgfsetlinewidth{0.000000pt}%
\definecolor{currentstroke}{rgb}{0.000000,0.000000,0.000000}%
\pgfsetstrokecolor{currentstroke}%
\pgfsetstrokeopacity{0.700000}%
\pgfsetdash{}{0pt}%
\pgfpathmoveto{\pgfqpoint{6.611151in}{0.550000in}}%
\pgfpathlineto{\pgfqpoint{6.638672in}{0.550000in}}%
\pgfpathlineto{\pgfqpoint{6.638672in}{0.611396in}}%
\pgfpathlineto{\pgfqpoint{6.611151in}{0.611396in}}%
\pgfpathlineto{\pgfqpoint{6.611151in}{0.550000in}}%
\pgfpathclose%
\pgfusepath{fill}%
\end{pgfscope}%
\begin{pgfscope}%
\pgfpathrectangle{\pgfqpoint{1.250000in}{0.550000in}}{\pgfqpoint{7.750000in}{3.850000in}}%
\pgfusepath{clip}%
\pgfsetbuttcap%
\pgfsetmiterjoin%
\definecolor{currentfill}{rgb}{0.501961,0.501961,0.501961}%
\pgfsetfillcolor{currentfill}%
\pgfsetfillopacity{0.700000}%
\pgfsetlinewidth{0.000000pt}%
\definecolor{currentstroke}{rgb}{0.000000,0.000000,0.000000}%
\pgfsetstrokecolor{currentstroke}%
\pgfsetstrokeopacity{0.700000}%
\pgfsetdash{}{0pt}%
\pgfpathmoveto{\pgfqpoint{6.638672in}{0.550000in}}%
\pgfpathlineto{\pgfqpoint{6.666193in}{0.550000in}}%
\pgfpathlineto{\pgfqpoint{6.666193in}{0.606715in}}%
\pgfpathlineto{\pgfqpoint{6.638672in}{0.606715in}}%
\pgfpathlineto{\pgfqpoint{6.638672in}{0.550000in}}%
\pgfpathclose%
\pgfusepath{fill}%
\end{pgfscope}%
\begin{pgfscope}%
\pgfpathrectangle{\pgfqpoint{1.250000in}{0.550000in}}{\pgfqpoint{7.750000in}{3.850000in}}%
\pgfusepath{clip}%
\pgfsetbuttcap%
\pgfsetmiterjoin%
\definecolor{currentfill}{rgb}{0.501961,0.501961,0.501961}%
\pgfsetfillcolor{currentfill}%
\pgfsetfillopacity{0.700000}%
\pgfsetlinewidth{0.000000pt}%
\definecolor{currentstroke}{rgb}{0.000000,0.000000,0.000000}%
\pgfsetstrokecolor{currentstroke}%
\pgfsetstrokeopacity{0.700000}%
\pgfsetdash{}{0pt}%
\pgfpathmoveto{\pgfqpoint{6.666193in}{0.550000in}}%
\pgfpathlineto{\pgfqpoint{6.693714in}{0.550000in}}%
\pgfpathlineto{\pgfqpoint{6.693714in}{0.609468in}}%
\pgfpathlineto{\pgfqpoint{6.666193in}{0.609468in}}%
\pgfpathlineto{\pgfqpoint{6.666193in}{0.550000in}}%
\pgfpathclose%
\pgfusepath{fill}%
\end{pgfscope}%
\begin{pgfscope}%
\pgfpathrectangle{\pgfqpoint{1.250000in}{0.550000in}}{\pgfqpoint{7.750000in}{3.850000in}}%
\pgfusepath{clip}%
\pgfsetbuttcap%
\pgfsetmiterjoin%
\definecolor{currentfill}{rgb}{0.501961,0.501961,0.501961}%
\pgfsetfillcolor{currentfill}%
\pgfsetfillopacity{0.700000}%
\pgfsetlinewidth{0.000000pt}%
\definecolor{currentstroke}{rgb}{0.000000,0.000000,0.000000}%
\pgfsetstrokecolor{currentstroke}%
\pgfsetstrokeopacity{0.700000}%
\pgfsetdash{}{0pt}%
\pgfpathmoveto{\pgfqpoint{6.693714in}{0.550000in}}%
\pgfpathlineto{\pgfqpoint{6.721236in}{0.550000in}}%
\pgfpathlineto{\pgfqpoint{6.721236in}{0.601760in}}%
\pgfpathlineto{\pgfqpoint{6.693714in}{0.601760in}}%
\pgfpathlineto{\pgfqpoint{6.693714in}{0.550000in}}%
\pgfpathclose%
\pgfusepath{fill}%
\end{pgfscope}%
\begin{pgfscope}%
\pgfpathrectangle{\pgfqpoint{1.250000in}{0.550000in}}{\pgfqpoint{7.750000in}{3.850000in}}%
\pgfusepath{clip}%
\pgfsetbuttcap%
\pgfsetmiterjoin%
\definecolor{currentfill}{rgb}{0.501961,0.501961,0.501961}%
\pgfsetfillcolor{currentfill}%
\pgfsetfillopacity{0.700000}%
\pgfsetlinewidth{0.000000pt}%
\definecolor{currentstroke}{rgb}{0.000000,0.000000,0.000000}%
\pgfsetstrokecolor{currentstroke}%
\pgfsetstrokeopacity{0.700000}%
\pgfsetdash{}{0pt}%
\pgfpathmoveto{\pgfqpoint{6.721236in}{0.550000in}}%
\pgfpathlineto{\pgfqpoint{6.748757in}{0.550000in}}%
\pgfpathlineto{\pgfqpoint{6.748757in}{0.609193in}}%
\pgfpathlineto{\pgfqpoint{6.721236in}{0.609193in}}%
\pgfpathlineto{\pgfqpoint{6.721236in}{0.550000in}}%
\pgfpathclose%
\pgfusepath{fill}%
\end{pgfscope}%
\begin{pgfscope}%
\pgfpathrectangle{\pgfqpoint{1.250000in}{0.550000in}}{\pgfqpoint{7.750000in}{3.850000in}}%
\pgfusepath{clip}%
\pgfsetbuttcap%
\pgfsetmiterjoin%
\definecolor{currentfill}{rgb}{0.501961,0.501961,0.501961}%
\pgfsetfillcolor{currentfill}%
\pgfsetfillopacity{0.700000}%
\pgfsetlinewidth{0.000000pt}%
\definecolor{currentstroke}{rgb}{0.000000,0.000000,0.000000}%
\pgfsetstrokecolor{currentstroke}%
\pgfsetstrokeopacity{0.700000}%
\pgfsetdash{}{0pt}%
\pgfpathmoveto{\pgfqpoint{6.748757in}{0.550000in}}%
\pgfpathlineto{\pgfqpoint{6.776278in}{0.550000in}}%
\pgfpathlineto{\pgfqpoint{6.776278in}{0.603962in}}%
\pgfpathlineto{\pgfqpoint{6.748757in}{0.603962in}}%
\pgfpathlineto{\pgfqpoint{6.748757in}{0.550000in}}%
\pgfpathclose%
\pgfusepath{fill}%
\end{pgfscope}%
\begin{pgfscope}%
\pgfpathrectangle{\pgfqpoint{1.250000in}{0.550000in}}{\pgfqpoint{7.750000in}{3.850000in}}%
\pgfusepath{clip}%
\pgfsetbuttcap%
\pgfsetmiterjoin%
\definecolor{currentfill}{rgb}{0.501961,0.501961,0.501961}%
\pgfsetfillcolor{currentfill}%
\pgfsetfillopacity{0.700000}%
\pgfsetlinewidth{0.000000pt}%
\definecolor{currentstroke}{rgb}{0.000000,0.000000,0.000000}%
\pgfsetstrokecolor{currentstroke}%
\pgfsetstrokeopacity{0.700000}%
\pgfsetdash{}{0pt}%
\pgfpathmoveto{\pgfqpoint{6.776278in}{0.550000in}}%
\pgfpathlineto{\pgfqpoint{6.803800in}{0.550000in}}%
\pgfpathlineto{\pgfqpoint{6.803800in}{0.599832in}}%
\pgfpathlineto{\pgfqpoint{6.776278in}{0.599832in}}%
\pgfpathlineto{\pgfqpoint{6.776278in}{0.550000in}}%
\pgfpathclose%
\pgfusepath{fill}%
\end{pgfscope}%
\begin{pgfscope}%
\pgfpathrectangle{\pgfqpoint{1.250000in}{0.550000in}}{\pgfqpoint{7.750000in}{3.850000in}}%
\pgfusepath{clip}%
\pgfsetbuttcap%
\pgfsetmiterjoin%
\definecolor{currentfill}{rgb}{0.501961,0.501961,0.501961}%
\pgfsetfillcolor{currentfill}%
\pgfsetfillopacity{0.700000}%
\pgfsetlinewidth{0.000000pt}%
\definecolor{currentstroke}{rgb}{0.000000,0.000000,0.000000}%
\pgfsetstrokecolor{currentstroke}%
\pgfsetstrokeopacity{0.700000}%
\pgfsetdash{}{0pt}%
\pgfpathmoveto{\pgfqpoint{6.803800in}{0.550000in}}%
\pgfpathlineto{\pgfqpoint{6.831321in}{0.550000in}}%
\pgfpathlineto{\pgfqpoint{6.831321in}{0.600108in}}%
\pgfpathlineto{\pgfqpoint{6.803800in}{0.600108in}}%
\pgfpathlineto{\pgfqpoint{6.803800in}{0.550000in}}%
\pgfpathclose%
\pgfusepath{fill}%
\end{pgfscope}%
\begin{pgfscope}%
\pgfpathrectangle{\pgfqpoint{1.250000in}{0.550000in}}{\pgfqpoint{7.750000in}{3.850000in}}%
\pgfusepath{clip}%
\pgfsetbuttcap%
\pgfsetmiterjoin%
\definecolor{currentfill}{rgb}{0.501961,0.501961,0.501961}%
\pgfsetfillcolor{currentfill}%
\pgfsetfillopacity{0.700000}%
\pgfsetlinewidth{0.000000pt}%
\definecolor{currentstroke}{rgb}{0.000000,0.000000,0.000000}%
\pgfsetstrokecolor{currentstroke}%
\pgfsetstrokeopacity{0.700000}%
\pgfsetdash{}{0pt}%
\pgfpathmoveto{\pgfqpoint{6.831321in}{0.550000in}}%
\pgfpathlineto{\pgfqpoint{6.858842in}{0.550000in}}%
\pgfpathlineto{\pgfqpoint{6.858842in}{0.599832in}}%
\pgfpathlineto{\pgfqpoint{6.831321in}{0.599832in}}%
\pgfpathlineto{\pgfqpoint{6.831321in}{0.550000in}}%
\pgfpathclose%
\pgfusepath{fill}%
\end{pgfscope}%
\begin{pgfscope}%
\pgfpathrectangle{\pgfqpoint{1.250000in}{0.550000in}}{\pgfqpoint{7.750000in}{3.850000in}}%
\pgfusepath{clip}%
\pgfsetbuttcap%
\pgfsetmiterjoin%
\definecolor{currentfill}{rgb}{0.501961,0.501961,0.501961}%
\pgfsetfillcolor{currentfill}%
\pgfsetfillopacity{0.700000}%
\pgfsetlinewidth{0.000000pt}%
\definecolor{currentstroke}{rgb}{0.000000,0.000000,0.000000}%
\pgfsetstrokecolor{currentstroke}%
\pgfsetstrokeopacity{0.700000}%
\pgfsetdash{}{0pt}%
\pgfpathmoveto{\pgfqpoint{6.858842in}{0.550000in}}%
\pgfpathlineto{\pgfqpoint{6.886364in}{0.550000in}}%
\pgfpathlineto{\pgfqpoint{6.886364in}{0.593225in}}%
\pgfpathlineto{\pgfqpoint{6.858842in}{0.593225in}}%
\pgfpathlineto{\pgfqpoint{6.858842in}{0.550000in}}%
\pgfpathclose%
\pgfusepath{fill}%
\end{pgfscope}%
\begin{pgfscope}%
\pgfpathrectangle{\pgfqpoint{1.250000in}{0.550000in}}{\pgfqpoint{7.750000in}{3.850000in}}%
\pgfusepath{clip}%
\pgfsetbuttcap%
\pgfsetmiterjoin%
\definecolor{currentfill}{rgb}{0.501961,0.501961,0.501961}%
\pgfsetfillcolor{currentfill}%
\pgfsetfillopacity{0.700000}%
\pgfsetlinewidth{0.000000pt}%
\definecolor{currentstroke}{rgb}{0.000000,0.000000,0.000000}%
\pgfsetstrokecolor{currentstroke}%
\pgfsetstrokeopacity{0.700000}%
\pgfsetdash{}{0pt}%
\pgfpathmoveto{\pgfqpoint{6.886364in}{0.550000in}}%
\pgfpathlineto{\pgfqpoint{6.913885in}{0.550000in}}%
\pgfpathlineto{\pgfqpoint{6.913885in}{0.600108in}}%
\pgfpathlineto{\pgfqpoint{6.886364in}{0.600108in}}%
\pgfpathlineto{\pgfqpoint{6.886364in}{0.550000in}}%
\pgfpathclose%
\pgfusepath{fill}%
\end{pgfscope}%
\begin{pgfscope}%
\pgfpathrectangle{\pgfqpoint{1.250000in}{0.550000in}}{\pgfqpoint{7.750000in}{3.850000in}}%
\pgfusepath{clip}%
\pgfsetbuttcap%
\pgfsetmiterjoin%
\definecolor{currentfill}{rgb}{0.501961,0.501961,0.501961}%
\pgfsetfillcolor{currentfill}%
\pgfsetfillopacity{0.700000}%
\pgfsetlinewidth{0.000000pt}%
\definecolor{currentstroke}{rgb}{0.000000,0.000000,0.000000}%
\pgfsetstrokecolor{currentstroke}%
\pgfsetstrokeopacity{0.700000}%
\pgfsetdash{}{0pt}%
\pgfpathmoveto{\pgfqpoint{6.913885in}{0.550000in}}%
\pgfpathlineto{\pgfqpoint{6.941406in}{0.550000in}}%
\pgfpathlineto{\pgfqpoint{6.941406in}{0.595978in}}%
\pgfpathlineto{\pgfqpoint{6.913885in}{0.595978in}}%
\pgfpathlineto{\pgfqpoint{6.913885in}{0.550000in}}%
\pgfpathclose%
\pgfusepath{fill}%
\end{pgfscope}%
\begin{pgfscope}%
\pgfpathrectangle{\pgfqpoint{1.250000in}{0.550000in}}{\pgfqpoint{7.750000in}{3.850000in}}%
\pgfusepath{clip}%
\pgfsetbuttcap%
\pgfsetmiterjoin%
\definecolor{currentfill}{rgb}{0.501961,0.501961,0.501961}%
\pgfsetfillcolor{currentfill}%
\pgfsetfillopacity{0.700000}%
\pgfsetlinewidth{0.000000pt}%
\definecolor{currentstroke}{rgb}{0.000000,0.000000,0.000000}%
\pgfsetstrokecolor{currentstroke}%
\pgfsetstrokeopacity{0.700000}%
\pgfsetdash{}{0pt}%
\pgfpathmoveto{\pgfqpoint{6.941406in}{0.550000in}}%
\pgfpathlineto{\pgfqpoint{6.968928in}{0.550000in}}%
\pgfpathlineto{\pgfqpoint{6.968928in}{0.591848in}}%
\pgfpathlineto{\pgfqpoint{6.941406in}{0.591848in}}%
\pgfpathlineto{\pgfqpoint{6.941406in}{0.550000in}}%
\pgfpathclose%
\pgfusepath{fill}%
\end{pgfscope}%
\begin{pgfscope}%
\pgfpathrectangle{\pgfqpoint{1.250000in}{0.550000in}}{\pgfqpoint{7.750000in}{3.850000in}}%
\pgfusepath{clip}%
\pgfsetbuttcap%
\pgfsetmiterjoin%
\definecolor{currentfill}{rgb}{0.501961,0.501961,0.501961}%
\pgfsetfillcolor{currentfill}%
\pgfsetfillopacity{0.700000}%
\pgfsetlinewidth{0.000000pt}%
\definecolor{currentstroke}{rgb}{0.000000,0.000000,0.000000}%
\pgfsetstrokecolor{currentstroke}%
\pgfsetstrokeopacity{0.700000}%
\pgfsetdash{}{0pt}%
\pgfpathmoveto{\pgfqpoint{6.968928in}{0.550000in}}%
\pgfpathlineto{\pgfqpoint{6.996449in}{0.550000in}}%
\pgfpathlineto{\pgfqpoint{6.996449in}{0.588269in}}%
\pgfpathlineto{\pgfqpoint{6.968928in}{0.588269in}}%
\pgfpathlineto{\pgfqpoint{6.968928in}{0.550000in}}%
\pgfpathclose%
\pgfusepath{fill}%
\end{pgfscope}%
\begin{pgfscope}%
\pgfpathrectangle{\pgfqpoint{1.250000in}{0.550000in}}{\pgfqpoint{7.750000in}{3.850000in}}%
\pgfusepath{clip}%
\pgfsetbuttcap%
\pgfsetmiterjoin%
\definecolor{currentfill}{rgb}{0.501961,0.501961,0.501961}%
\pgfsetfillcolor{currentfill}%
\pgfsetfillopacity{0.700000}%
\pgfsetlinewidth{0.000000pt}%
\definecolor{currentstroke}{rgb}{0.000000,0.000000,0.000000}%
\pgfsetstrokecolor{currentstroke}%
\pgfsetstrokeopacity{0.700000}%
\pgfsetdash{}{0pt}%
\pgfpathmoveto{\pgfqpoint{6.996449in}{0.550000in}}%
\pgfpathlineto{\pgfqpoint{7.023970in}{0.550000in}}%
\pgfpathlineto{\pgfqpoint{7.023970in}{0.600108in}}%
\pgfpathlineto{\pgfqpoint{6.996449in}{0.600108in}}%
\pgfpathlineto{\pgfqpoint{6.996449in}{0.550000in}}%
\pgfpathclose%
\pgfusepath{fill}%
\end{pgfscope}%
\begin{pgfscope}%
\pgfpathrectangle{\pgfqpoint{1.250000in}{0.550000in}}{\pgfqpoint{7.750000in}{3.850000in}}%
\pgfusepath{clip}%
\pgfsetbuttcap%
\pgfsetmiterjoin%
\definecolor{currentfill}{rgb}{0.501961,0.501961,0.501961}%
\pgfsetfillcolor{currentfill}%
\pgfsetfillopacity{0.700000}%
\pgfsetlinewidth{0.000000pt}%
\definecolor{currentstroke}{rgb}{0.000000,0.000000,0.000000}%
\pgfsetstrokecolor{currentstroke}%
\pgfsetstrokeopacity{0.700000}%
\pgfsetdash{}{0pt}%
\pgfpathmoveto{\pgfqpoint{7.023970in}{0.550000in}}%
\pgfpathlineto{\pgfqpoint{7.051491in}{0.550000in}}%
\pgfpathlineto{\pgfqpoint{7.051491in}{0.589646in}}%
\pgfpathlineto{\pgfqpoint{7.023970in}{0.589646in}}%
\pgfpathlineto{\pgfqpoint{7.023970in}{0.550000in}}%
\pgfpathclose%
\pgfusepath{fill}%
\end{pgfscope}%
\begin{pgfscope}%
\pgfpathrectangle{\pgfqpoint{1.250000in}{0.550000in}}{\pgfqpoint{7.750000in}{3.850000in}}%
\pgfusepath{clip}%
\pgfsetbuttcap%
\pgfsetmiterjoin%
\definecolor{currentfill}{rgb}{0.501961,0.501961,0.501961}%
\pgfsetfillcolor{currentfill}%
\pgfsetfillopacity{0.700000}%
\pgfsetlinewidth{0.000000pt}%
\definecolor{currentstroke}{rgb}{0.000000,0.000000,0.000000}%
\pgfsetstrokecolor{currentstroke}%
\pgfsetstrokeopacity{0.700000}%
\pgfsetdash{}{0pt}%
\pgfpathmoveto{\pgfqpoint{7.051491in}{0.550000in}}%
\pgfpathlineto{\pgfqpoint{7.079013in}{0.550000in}}%
\pgfpathlineto{\pgfqpoint{7.079013in}{0.597630in}}%
\pgfpathlineto{\pgfqpoint{7.051491in}{0.597630in}}%
\pgfpathlineto{\pgfqpoint{7.051491in}{0.550000in}}%
\pgfpathclose%
\pgfusepath{fill}%
\end{pgfscope}%
\begin{pgfscope}%
\pgfpathrectangle{\pgfqpoint{1.250000in}{0.550000in}}{\pgfqpoint{7.750000in}{3.850000in}}%
\pgfusepath{clip}%
\pgfsetbuttcap%
\pgfsetmiterjoin%
\definecolor{currentfill}{rgb}{0.501961,0.501961,0.501961}%
\pgfsetfillcolor{currentfill}%
\pgfsetfillopacity{0.700000}%
\pgfsetlinewidth{0.000000pt}%
\definecolor{currentstroke}{rgb}{0.000000,0.000000,0.000000}%
\pgfsetstrokecolor{currentstroke}%
\pgfsetstrokeopacity{0.700000}%
\pgfsetdash{}{0pt}%
\pgfpathmoveto{\pgfqpoint{7.079013in}{0.550000in}}%
\pgfpathlineto{\pgfqpoint{7.106534in}{0.550000in}}%
\pgfpathlineto{\pgfqpoint{7.106534in}{0.591573in}}%
\pgfpathlineto{\pgfqpoint{7.079013in}{0.591573in}}%
\pgfpathlineto{\pgfqpoint{7.079013in}{0.550000in}}%
\pgfpathclose%
\pgfusepath{fill}%
\end{pgfscope}%
\begin{pgfscope}%
\pgfpathrectangle{\pgfqpoint{1.250000in}{0.550000in}}{\pgfqpoint{7.750000in}{3.850000in}}%
\pgfusepath{clip}%
\pgfsetbuttcap%
\pgfsetmiterjoin%
\definecolor{currentfill}{rgb}{0.501961,0.501961,0.501961}%
\pgfsetfillcolor{currentfill}%
\pgfsetfillopacity{0.700000}%
\pgfsetlinewidth{0.000000pt}%
\definecolor{currentstroke}{rgb}{0.000000,0.000000,0.000000}%
\pgfsetstrokecolor{currentstroke}%
\pgfsetstrokeopacity{0.700000}%
\pgfsetdash{}{0pt}%
\pgfpathmoveto{\pgfqpoint{7.106534in}{0.550000in}}%
\pgfpathlineto{\pgfqpoint{7.134055in}{0.550000in}}%
\pgfpathlineto{\pgfqpoint{7.134055in}{0.592399in}}%
\pgfpathlineto{\pgfqpoint{7.106534in}{0.592399in}}%
\pgfpathlineto{\pgfqpoint{7.106534in}{0.550000in}}%
\pgfpathclose%
\pgfusepath{fill}%
\end{pgfscope}%
\begin{pgfscope}%
\pgfpathrectangle{\pgfqpoint{1.250000in}{0.550000in}}{\pgfqpoint{7.750000in}{3.850000in}}%
\pgfusepath{clip}%
\pgfsetbuttcap%
\pgfsetmiterjoin%
\definecolor{currentfill}{rgb}{0.501961,0.501961,0.501961}%
\pgfsetfillcolor{currentfill}%
\pgfsetfillopacity{0.700000}%
\pgfsetlinewidth{0.000000pt}%
\definecolor{currentstroke}{rgb}{0.000000,0.000000,0.000000}%
\pgfsetstrokecolor{currentstroke}%
\pgfsetstrokeopacity{0.700000}%
\pgfsetdash{}{0pt}%
\pgfpathmoveto{\pgfqpoint{7.134055in}{0.550000in}}%
\pgfpathlineto{\pgfqpoint{7.161577in}{0.550000in}}%
\pgfpathlineto{\pgfqpoint{7.161577in}{0.598731in}}%
\pgfpathlineto{\pgfqpoint{7.134055in}{0.598731in}}%
\pgfpathlineto{\pgfqpoint{7.134055in}{0.550000in}}%
\pgfpathclose%
\pgfusepath{fill}%
\end{pgfscope}%
\begin{pgfscope}%
\pgfpathrectangle{\pgfqpoint{1.250000in}{0.550000in}}{\pgfqpoint{7.750000in}{3.850000in}}%
\pgfusepath{clip}%
\pgfsetbuttcap%
\pgfsetmiterjoin%
\definecolor{currentfill}{rgb}{0.501961,0.501961,0.501961}%
\pgfsetfillcolor{currentfill}%
\pgfsetfillopacity{0.700000}%
\pgfsetlinewidth{0.000000pt}%
\definecolor{currentstroke}{rgb}{0.000000,0.000000,0.000000}%
\pgfsetstrokecolor{currentstroke}%
\pgfsetstrokeopacity{0.700000}%
\pgfsetdash{}{0pt}%
\pgfpathmoveto{\pgfqpoint{7.161577in}{0.550000in}}%
\pgfpathlineto{\pgfqpoint{7.189098in}{0.550000in}}%
\pgfpathlineto{\pgfqpoint{7.189098in}{0.584965in}}%
\pgfpathlineto{\pgfqpoint{7.161577in}{0.584965in}}%
\pgfpathlineto{\pgfqpoint{7.161577in}{0.550000in}}%
\pgfpathclose%
\pgfusepath{fill}%
\end{pgfscope}%
\begin{pgfscope}%
\pgfpathrectangle{\pgfqpoint{1.250000in}{0.550000in}}{\pgfqpoint{7.750000in}{3.850000in}}%
\pgfusepath{clip}%
\pgfsetbuttcap%
\pgfsetmiterjoin%
\definecolor{currentfill}{rgb}{0.501961,0.501961,0.501961}%
\pgfsetfillcolor{currentfill}%
\pgfsetfillopacity{0.700000}%
\pgfsetlinewidth{0.000000pt}%
\definecolor{currentstroke}{rgb}{0.000000,0.000000,0.000000}%
\pgfsetstrokecolor{currentstroke}%
\pgfsetstrokeopacity{0.700000}%
\pgfsetdash{}{0pt}%
\pgfpathmoveto{\pgfqpoint{7.189098in}{0.550000in}}%
\pgfpathlineto{\pgfqpoint{7.216619in}{0.550000in}}%
\pgfpathlineto{\pgfqpoint{7.216619in}{0.594326in}}%
\pgfpathlineto{\pgfqpoint{7.189098in}{0.594326in}}%
\pgfpathlineto{\pgfqpoint{7.189098in}{0.550000in}}%
\pgfpathclose%
\pgfusepath{fill}%
\end{pgfscope}%
\begin{pgfscope}%
\pgfpathrectangle{\pgfqpoint{1.250000in}{0.550000in}}{\pgfqpoint{7.750000in}{3.850000in}}%
\pgfusepath{clip}%
\pgfsetbuttcap%
\pgfsetmiterjoin%
\definecolor{currentfill}{rgb}{0.501961,0.501961,0.501961}%
\pgfsetfillcolor{currentfill}%
\pgfsetfillopacity{0.700000}%
\pgfsetlinewidth{0.000000pt}%
\definecolor{currentstroke}{rgb}{0.000000,0.000000,0.000000}%
\pgfsetstrokecolor{currentstroke}%
\pgfsetstrokeopacity{0.700000}%
\pgfsetdash{}{0pt}%
\pgfpathmoveto{\pgfqpoint{7.216619in}{0.550000in}}%
\pgfpathlineto{\pgfqpoint{7.244141in}{0.550000in}}%
\pgfpathlineto{\pgfqpoint{7.244141in}{0.590747in}}%
\pgfpathlineto{\pgfqpoint{7.216619in}{0.590747in}}%
\pgfpathlineto{\pgfqpoint{7.216619in}{0.550000in}}%
\pgfpathclose%
\pgfusepath{fill}%
\end{pgfscope}%
\begin{pgfscope}%
\pgfpathrectangle{\pgfqpoint{1.250000in}{0.550000in}}{\pgfqpoint{7.750000in}{3.850000in}}%
\pgfusepath{clip}%
\pgfsetbuttcap%
\pgfsetmiterjoin%
\definecolor{currentfill}{rgb}{0.501961,0.501961,0.501961}%
\pgfsetfillcolor{currentfill}%
\pgfsetfillopacity{0.700000}%
\pgfsetlinewidth{0.000000pt}%
\definecolor{currentstroke}{rgb}{0.000000,0.000000,0.000000}%
\pgfsetstrokecolor{currentstroke}%
\pgfsetstrokeopacity{0.700000}%
\pgfsetdash{}{0pt}%
\pgfpathmoveto{\pgfqpoint{7.244141in}{0.550000in}}%
\pgfpathlineto{\pgfqpoint{7.271662in}{0.550000in}}%
\pgfpathlineto{\pgfqpoint{7.271662in}{0.591573in}}%
\pgfpathlineto{\pgfqpoint{7.244141in}{0.591573in}}%
\pgfpathlineto{\pgfqpoint{7.244141in}{0.550000in}}%
\pgfpathclose%
\pgfusepath{fill}%
\end{pgfscope}%
\begin{pgfscope}%
\pgfpathrectangle{\pgfqpoint{1.250000in}{0.550000in}}{\pgfqpoint{7.750000in}{3.850000in}}%
\pgfusepath{clip}%
\pgfsetbuttcap%
\pgfsetmiterjoin%
\definecolor{currentfill}{rgb}{0.501961,0.501961,0.501961}%
\pgfsetfillcolor{currentfill}%
\pgfsetfillopacity{0.700000}%
\pgfsetlinewidth{0.000000pt}%
\definecolor{currentstroke}{rgb}{0.000000,0.000000,0.000000}%
\pgfsetstrokecolor{currentstroke}%
\pgfsetstrokeopacity{0.700000}%
\pgfsetdash{}{0pt}%
\pgfpathmoveto{\pgfqpoint{7.271662in}{0.550000in}}%
\pgfpathlineto{\pgfqpoint{7.299183in}{0.550000in}}%
\pgfpathlineto{\pgfqpoint{7.299183in}{0.593775in}}%
\pgfpathlineto{\pgfqpoint{7.271662in}{0.593775in}}%
\pgfpathlineto{\pgfqpoint{7.271662in}{0.550000in}}%
\pgfpathclose%
\pgfusepath{fill}%
\end{pgfscope}%
\begin{pgfscope}%
\pgfpathrectangle{\pgfqpoint{1.250000in}{0.550000in}}{\pgfqpoint{7.750000in}{3.850000in}}%
\pgfusepath{clip}%
\pgfsetbuttcap%
\pgfsetmiterjoin%
\definecolor{currentfill}{rgb}{0.501961,0.501961,0.501961}%
\pgfsetfillcolor{currentfill}%
\pgfsetfillopacity{0.700000}%
\pgfsetlinewidth{0.000000pt}%
\definecolor{currentstroke}{rgb}{0.000000,0.000000,0.000000}%
\pgfsetstrokecolor{currentstroke}%
\pgfsetstrokeopacity{0.700000}%
\pgfsetdash{}{0pt}%
\pgfpathmoveto{\pgfqpoint{7.299183in}{0.550000in}}%
\pgfpathlineto{\pgfqpoint{7.326705in}{0.550000in}}%
\pgfpathlineto{\pgfqpoint{7.326705in}{0.588820in}}%
\pgfpathlineto{\pgfqpoint{7.299183in}{0.588820in}}%
\pgfpathlineto{\pgfqpoint{7.299183in}{0.550000in}}%
\pgfpathclose%
\pgfusepath{fill}%
\end{pgfscope}%
\begin{pgfscope}%
\pgfpathrectangle{\pgfqpoint{1.250000in}{0.550000in}}{\pgfqpoint{7.750000in}{3.850000in}}%
\pgfusepath{clip}%
\pgfsetbuttcap%
\pgfsetmiterjoin%
\definecolor{currentfill}{rgb}{0.501961,0.501961,0.501961}%
\pgfsetfillcolor{currentfill}%
\pgfsetfillopacity{0.700000}%
\pgfsetlinewidth{0.000000pt}%
\definecolor{currentstroke}{rgb}{0.000000,0.000000,0.000000}%
\pgfsetstrokecolor{currentstroke}%
\pgfsetstrokeopacity{0.700000}%
\pgfsetdash{}{0pt}%
\pgfpathmoveto{\pgfqpoint{7.326705in}{0.550000in}}%
\pgfpathlineto{\pgfqpoint{7.354226in}{0.550000in}}%
\pgfpathlineto{\pgfqpoint{7.354226in}{0.587443in}}%
\pgfpathlineto{\pgfqpoint{7.326705in}{0.587443in}}%
\pgfpathlineto{\pgfqpoint{7.326705in}{0.550000in}}%
\pgfpathclose%
\pgfusepath{fill}%
\end{pgfscope}%
\begin{pgfscope}%
\pgfpathrectangle{\pgfqpoint{1.250000in}{0.550000in}}{\pgfqpoint{7.750000in}{3.850000in}}%
\pgfusepath{clip}%
\pgfsetbuttcap%
\pgfsetmiterjoin%
\definecolor{currentfill}{rgb}{0.501961,0.501961,0.501961}%
\pgfsetfillcolor{currentfill}%
\pgfsetfillopacity{0.700000}%
\pgfsetlinewidth{0.000000pt}%
\definecolor{currentstroke}{rgb}{0.000000,0.000000,0.000000}%
\pgfsetstrokecolor{currentstroke}%
\pgfsetstrokeopacity{0.700000}%
\pgfsetdash{}{0pt}%
\pgfpathmoveto{\pgfqpoint{7.354226in}{0.550000in}}%
\pgfpathlineto{\pgfqpoint{7.381747in}{0.550000in}}%
\pgfpathlineto{\pgfqpoint{7.381747in}{0.583589in}}%
\pgfpathlineto{\pgfqpoint{7.354226in}{0.583589in}}%
\pgfpathlineto{\pgfqpoint{7.354226in}{0.550000in}}%
\pgfpathclose%
\pgfusepath{fill}%
\end{pgfscope}%
\begin{pgfscope}%
\pgfpathrectangle{\pgfqpoint{1.250000in}{0.550000in}}{\pgfqpoint{7.750000in}{3.850000in}}%
\pgfusepath{clip}%
\pgfsetbuttcap%
\pgfsetmiterjoin%
\definecolor{currentfill}{rgb}{0.501961,0.501961,0.501961}%
\pgfsetfillcolor{currentfill}%
\pgfsetfillopacity{0.700000}%
\pgfsetlinewidth{0.000000pt}%
\definecolor{currentstroke}{rgb}{0.000000,0.000000,0.000000}%
\pgfsetstrokecolor{currentstroke}%
\pgfsetstrokeopacity{0.700000}%
\pgfsetdash{}{0pt}%
\pgfpathmoveto{\pgfqpoint{7.381747in}{0.550000in}}%
\pgfpathlineto{\pgfqpoint{7.409268in}{0.550000in}}%
\pgfpathlineto{\pgfqpoint{7.409268in}{0.588269in}}%
\pgfpathlineto{\pgfqpoint{7.381747in}{0.588269in}}%
\pgfpathlineto{\pgfqpoint{7.381747in}{0.550000in}}%
\pgfpathclose%
\pgfusepath{fill}%
\end{pgfscope}%
\begin{pgfscope}%
\pgfpathrectangle{\pgfqpoint{1.250000in}{0.550000in}}{\pgfqpoint{7.750000in}{3.850000in}}%
\pgfusepath{clip}%
\pgfsetbuttcap%
\pgfsetmiterjoin%
\definecolor{currentfill}{rgb}{0.501961,0.501961,0.501961}%
\pgfsetfillcolor{currentfill}%
\pgfsetfillopacity{0.700000}%
\pgfsetlinewidth{0.000000pt}%
\definecolor{currentstroke}{rgb}{0.000000,0.000000,0.000000}%
\pgfsetstrokecolor{currentstroke}%
\pgfsetstrokeopacity{0.700000}%
\pgfsetdash{}{0pt}%
\pgfpathmoveto{\pgfqpoint{7.409268in}{0.550000in}}%
\pgfpathlineto{\pgfqpoint{7.436790in}{0.550000in}}%
\pgfpathlineto{\pgfqpoint{7.436790in}{0.584965in}}%
\pgfpathlineto{\pgfqpoint{7.409268in}{0.584965in}}%
\pgfpathlineto{\pgfqpoint{7.409268in}{0.550000in}}%
\pgfpathclose%
\pgfusepath{fill}%
\end{pgfscope}%
\begin{pgfscope}%
\pgfpathrectangle{\pgfqpoint{1.250000in}{0.550000in}}{\pgfqpoint{7.750000in}{3.850000in}}%
\pgfusepath{clip}%
\pgfsetbuttcap%
\pgfsetmiterjoin%
\definecolor{currentfill}{rgb}{0.501961,0.501961,0.501961}%
\pgfsetfillcolor{currentfill}%
\pgfsetfillopacity{0.700000}%
\pgfsetlinewidth{0.000000pt}%
\definecolor{currentstroke}{rgb}{0.000000,0.000000,0.000000}%
\pgfsetstrokecolor{currentstroke}%
\pgfsetstrokeopacity{0.700000}%
\pgfsetdash{}{0pt}%
\pgfpathmoveto{\pgfqpoint{7.436790in}{0.550000in}}%
\pgfpathlineto{\pgfqpoint{7.464311in}{0.550000in}}%
\pgfpathlineto{\pgfqpoint{7.464311in}{0.587443in}}%
\pgfpathlineto{\pgfqpoint{7.436790in}{0.587443in}}%
\pgfpathlineto{\pgfqpoint{7.436790in}{0.550000in}}%
\pgfpathclose%
\pgfusepath{fill}%
\end{pgfscope}%
\begin{pgfscope}%
\pgfpathrectangle{\pgfqpoint{1.250000in}{0.550000in}}{\pgfqpoint{7.750000in}{3.850000in}}%
\pgfusepath{clip}%
\pgfsetbuttcap%
\pgfsetmiterjoin%
\definecolor{currentfill}{rgb}{0.501961,0.501961,0.501961}%
\pgfsetfillcolor{currentfill}%
\pgfsetfillopacity{0.700000}%
\pgfsetlinewidth{0.000000pt}%
\definecolor{currentstroke}{rgb}{0.000000,0.000000,0.000000}%
\pgfsetstrokecolor{currentstroke}%
\pgfsetstrokeopacity{0.700000}%
\pgfsetdash{}{0pt}%
\pgfpathmoveto{\pgfqpoint{7.464311in}{0.550000in}}%
\pgfpathlineto{\pgfqpoint{7.491832in}{0.550000in}}%
\pgfpathlineto{\pgfqpoint{7.491832in}{0.586892in}}%
\pgfpathlineto{\pgfqpoint{7.464311in}{0.586892in}}%
\pgfpathlineto{\pgfqpoint{7.464311in}{0.550000in}}%
\pgfpathclose%
\pgfusepath{fill}%
\end{pgfscope}%
\begin{pgfscope}%
\pgfpathrectangle{\pgfqpoint{1.250000in}{0.550000in}}{\pgfqpoint{7.750000in}{3.850000in}}%
\pgfusepath{clip}%
\pgfsetbuttcap%
\pgfsetmiterjoin%
\definecolor{currentfill}{rgb}{0.501961,0.501961,0.501961}%
\pgfsetfillcolor{currentfill}%
\pgfsetfillopacity{0.700000}%
\pgfsetlinewidth{0.000000pt}%
\definecolor{currentstroke}{rgb}{0.000000,0.000000,0.000000}%
\pgfsetstrokecolor{currentstroke}%
\pgfsetstrokeopacity{0.700000}%
\pgfsetdash{}{0pt}%
\pgfpathmoveto{\pgfqpoint{7.491832in}{0.550000in}}%
\pgfpathlineto{\pgfqpoint{7.519354in}{0.550000in}}%
\pgfpathlineto{\pgfqpoint{7.519354in}{0.584415in}}%
\pgfpathlineto{\pgfqpoint{7.491832in}{0.584415in}}%
\pgfpathlineto{\pgfqpoint{7.491832in}{0.550000in}}%
\pgfpathclose%
\pgfusepath{fill}%
\end{pgfscope}%
\begin{pgfscope}%
\pgfpathrectangle{\pgfqpoint{1.250000in}{0.550000in}}{\pgfqpoint{7.750000in}{3.850000in}}%
\pgfusepath{clip}%
\pgfsetbuttcap%
\pgfsetmiterjoin%
\definecolor{currentfill}{rgb}{0.501961,0.501961,0.501961}%
\pgfsetfillcolor{currentfill}%
\pgfsetfillopacity{0.700000}%
\pgfsetlinewidth{0.000000pt}%
\definecolor{currentstroke}{rgb}{0.000000,0.000000,0.000000}%
\pgfsetstrokecolor{currentstroke}%
\pgfsetstrokeopacity{0.700000}%
\pgfsetdash{}{0pt}%
\pgfpathmoveto{\pgfqpoint{7.519354in}{0.550000in}}%
\pgfpathlineto{\pgfqpoint{7.546875in}{0.550000in}}%
\pgfpathlineto{\pgfqpoint{7.546875in}{0.578082in}}%
\pgfpathlineto{\pgfqpoint{7.519354in}{0.578082in}}%
\pgfpathlineto{\pgfqpoint{7.519354in}{0.550000in}}%
\pgfpathclose%
\pgfusepath{fill}%
\end{pgfscope}%
\begin{pgfscope}%
\pgfpathrectangle{\pgfqpoint{1.250000in}{0.550000in}}{\pgfqpoint{7.750000in}{3.850000in}}%
\pgfusepath{clip}%
\pgfsetbuttcap%
\pgfsetmiterjoin%
\definecolor{currentfill}{rgb}{0.501961,0.501961,0.501961}%
\pgfsetfillcolor{currentfill}%
\pgfsetfillopacity{0.700000}%
\pgfsetlinewidth{0.000000pt}%
\definecolor{currentstroke}{rgb}{0.000000,0.000000,0.000000}%
\pgfsetstrokecolor{currentstroke}%
\pgfsetstrokeopacity{0.700000}%
\pgfsetdash{}{0pt}%
\pgfpathmoveto{\pgfqpoint{7.546875in}{0.550000in}}%
\pgfpathlineto{\pgfqpoint{7.574396in}{0.550000in}}%
\pgfpathlineto{\pgfqpoint{7.574396in}{0.581111in}}%
\pgfpathlineto{\pgfqpoint{7.546875in}{0.581111in}}%
\pgfpathlineto{\pgfqpoint{7.546875in}{0.550000in}}%
\pgfpathclose%
\pgfusepath{fill}%
\end{pgfscope}%
\begin{pgfscope}%
\pgfpathrectangle{\pgfqpoint{1.250000in}{0.550000in}}{\pgfqpoint{7.750000in}{3.850000in}}%
\pgfusepath{clip}%
\pgfsetbuttcap%
\pgfsetmiterjoin%
\definecolor{currentfill}{rgb}{0.501961,0.501961,0.501961}%
\pgfsetfillcolor{currentfill}%
\pgfsetfillopacity{0.700000}%
\pgfsetlinewidth{0.000000pt}%
\definecolor{currentstroke}{rgb}{0.000000,0.000000,0.000000}%
\pgfsetstrokecolor{currentstroke}%
\pgfsetstrokeopacity{0.700000}%
\pgfsetdash{}{0pt}%
\pgfpathmoveto{\pgfqpoint{7.574396in}{0.550000in}}%
\pgfpathlineto{\pgfqpoint{7.601918in}{0.550000in}}%
\pgfpathlineto{\pgfqpoint{7.601918in}{0.583313in}}%
\pgfpathlineto{\pgfqpoint{7.574396in}{0.583313in}}%
\pgfpathlineto{\pgfqpoint{7.574396in}{0.550000in}}%
\pgfpathclose%
\pgfusepath{fill}%
\end{pgfscope}%
\begin{pgfscope}%
\pgfpathrectangle{\pgfqpoint{1.250000in}{0.550000in}}{\pgfqpoint{7.750000in}{3.850000in}}%
\pgfusepath{clip}%
\pgfsetbuttcap%
\pgfsetmiterjoin%
\definecolor{currentfill}{rgb}{0.501961,0.501961,0.501961}%
\pgfsetfillcolor{currentfill}%
\pgfsetfillopacity{0.700000}%
\pgfsetlinewidth{0.000000pt}%
\definecolor{currentstroke}{rgb}{0.000000,0.000000,0.000000}%
\pgfsetstrokecolor{currentstroke}%
\pgfsetstrokeopacity{0.700000}%
\pgfsetdash{}{0pt}%
\pgfpathmoveto{\pgfqpoint{7.601918in}{0.550000in}}%
\pgfpathlineto{\pgfqpoint{7.629439in}{0.550000in}}%
\pgfpathlineto{\pgfqpoint{7.629439in}{0.583038in}}%
\pgfpathlineto{\pgfqpoint{7.601918in}{0.583038in}}%
\pgfpathlineto{\pgfqpoint{7.601918in}{0.550000in}}%
\pgfpathclose%
\pgfusepath{fill}%
\end{pgfscope}%
\begin{pgfscope}%
\pgfpathrectangle{\pgfqpoint{1.250000in}{0.550000in}}{\pgfqpoint{7.750000in}{3.850000in}}%
\pgfusepath{clip}%
\pgfsetbuttcap%
\pgfsetmiterjoin%
\definecolor{currentfill}{rgb}{0.501961,0.501961,0.501961}%
\pgfsetfillcolor{currentfill}%
\pgfsetfillopacity{0.700000}%
\pgfsetlinewidth{0.000000pt}%
\definecolor{currentstroke}{rgb}{0.000000,0.000000,0.000000}%
\pgfsetstrokecolor{currentstroke}%
\pgfsetstrokeopacity{0.700000}%
\pgfsetdash{}{0pt}%
\pgfpathmoveto{\pgfqpoint{7.629439in}{0.550000in}}%
\pgfpathlineto{\pgfqpoint{7.656960in}{0.550000in}}%
\pgfpathlineto{\pgfqpoint{7.656960in}{0.580560in}}%
\pgfpathlineto{\pgfqpoint{7.629439in}{0.580560in}}%
\pgfpathlineto{\pgfqpoint{7.629439in}{0.550000in}}%
\pgfpathclose%
\pgfusepath{fill}%
\end{pgfscope}%
\begin{pgfscope}%
\pgfpathrectangle{\pgfqpoint{1.250000in}{0.550000in}}{\pgfqpoint{7.750000in}{3.850000in}}%
\pgfusepath{clip}%
\pgfsetbuttcap%
\pgfsetmiterjoin%
\definecolor{currentfill}{rgb}{0.501961,0.501961,0.501961}%
\pgfsetfillcolor{currentfill}%
\pgfsetfillopacity{0.700000}%
\pgfsetlinewidth{0.000000pt}%
\definecolor{currentstroke}{rgb}{0.000000,0.000000,0.000000}%
\pgfsetstrokecolor{currentstroke}%
\pgfsetstrokeopacity{0.700000}%
\pgfsetdash{}{0pt}%
\pgfpathmoveto{\pgfqpoint{7.656960in}{0.550000in}}%
\pgfpathlineto{\pgfqpoint{7.684482in}{0.550000in}}%
\pgfpathlineto{\pgfqpoint{7.684482in}{0.582763in}}%
\pgfpathlineto{\pgfqpoint{7.656960in}{0.582763in}}%
\pgfpathlineto{\pgfqpoint{7.656960in}{0.550000in}}%
\pgfpathclose%
\pgfusepath{fill}%
\end{pgfscope}%
\begin{pgfscope}%
\pgfpathrectangle{\pgfqpoint{1.250000in}{0.550000in}}{\pgfqpoint{7.750000in}{3.850000in}}%
\pgfusepath{clip}%
\pgfsetbuttcap%
\pgfsetmiterjoin%
\definecolor{currentfill}{rgb}{0.501961,0.501961,0.501961}%
\pgfsetfillcolor{currentfill}%
\pgfsetfillopacity{0.700000}%
\pgfsetlinewidth{0.000000pt}%
\definecolor{currentstroke}{rgb}{0.000000,0.000000,0.000000}%
\pgfsetstrokecolor{currentstroke}%
\pgfsetstrokeopacity{0.700000}%
\pgfsetdash{}{0pt}%
\pgfpathmoveto{\pgfqpoint{7.684482in}{0.550000in}}%
\pgfpathlineto{\pgfqpoint{7.712003in}{0.550000in}}%
\pgfpathlineto{\pgfqpoint{7.712003in}{0.581661in}}%
\pgfpathlineto{\pgfqpoint{7.684482in}{0.581661in}}%
\pgfpathlineto{\pgfqpoint{7.684482in}{0.550000in}}%
\pgfpathclose%
\pgfusepath{fill}%
\end{pgfscope}%
\begin{pgfscope}%
\pgfpathrectangle{\pgfqpoint{1.250000in}{0.550000in}}{\pgfqpoint{7.750000in}{3.850000in}}%
\pgfusepath{clip}%
\pgfsetbuttcap%
\pgfsetmiterjoin%
\definecolor{currentfill}{rgb}{0.501961,0.501961,0.501961}%
\pgfsetfillcolor{currentfill}%
\pgfsetfillopacity{0.700000}%
\pgfsetlinewidth{0.000000pt}%
\definecolor{currentstroke}{rgb}{0.000000,0.000000,0.000000}%
\pgfsetstrokecolor{currentstroke}%
\pgfsetstrokeopacity{0.700000}%
\pgfsetdash{}{0pt}%
\pgfpathmoveto{\pgfqpoint{7.712003in}{0.550000in}}%
\pgfpathlineto{\pgfqpoint{7.739524in}{0.550000in}}%
\pgfpathlineto{\pgfqpoint{7.739524in}{0.586066in}}%
\pgfpathlineto{\pgfqpoint{7.712003in}{0.586066in}}%
\pgfpathlineto{\pgfqpoint{7.712003in}{0.550000in}}%
\pgfpathclose%
\pgfusepath{fill}%
\end{pgfscope}%
\begin{pgfscope}%
\pgfpathrectangle{\pgfqpoint{1.250000in}{0.550000in}}{\pgfqpoint{7.750000in}{3.850000in}}%
\pgfusepath{clip}%
\pgfsetbuttcap%
\pgfsetmiterjoin%
\definecolor{currentfill}{rgb}{0.501961,0.501961,0.501961}%
\pgfsetfillcolor{currentfill}%
\pgfsetfillopacity{0.700000}%
\pgfsetlinewidth{0.000000pt}%
\definecolor{currentstroke}{rgb}{0.000000,0.000000,0.000000}%
\pgfsetstrokecolor{currentstroke}%
\pgfsetstrokeopacity{0.700000}%
\pgfsetdash{}{0pt}%
\pgfpathmoveto{\pgfqpoint{7.739524in}{0.550000in}}%
\pgfpathlineto{\pgfqpoint{7.767045in}{0.550000in}}%
\pgfpathlineto{\pgfqpoint{7.767045in}{0.587443in}}%
\pgfpathlineto{\pgfqpoint{7.739524in}{0.587443in}}%
\pgfpathlineto{\pgfqpoint{7.739524in}{0.550000in}}%
\pgfpathclose%
\pgfusepath{fill}%
\end{pgfscope}%
\begin{pgfscope}%
\pgfpathrectangle{\pgfqpoint{1.250000in}{0.550000in}}{\pgfqpoint{7.750000in}{3.850000in}}%
\pgfusepath{clip}%
\pgfsetbuttcap%
\pgfsetmiterjoin%
\definecolor{currentfill}{rgb}{0.501961,0.501961,0.501961}%
\pgfsetfillcolor{currentfill}%
\pgfsetfillopacity{0.700000}%
\pgfsetlinewidth{0.000000pt}%
\definecolor{currentstroke}{rgb}{0.000000,0.000000,0.000000}%
\pgfsetstrokecolor{currentstroke}%
\pgfsetstrokeopacity{0.700000}%
\pgfsetdash{}{0pt}%
\pgfpathmoveto{\pgfqpoint{7.767045in}{0.550000in}}%
\pgfpathlineto{\pgfqpoint{7.794567in}{0.550000in}}%
\pgfpathlineto{\pgfqpoint{7.794567in}{0.582487in}}%
\pgfpathlineto{\pgfqpoint{7.767045in}{0.582487in}}%
\pgfpathlineto{\pgfqpoint{7.767045in}{0.550000in}}%
\pgfpathclose%
\pgfusepath{fill}%
\end{pgfscope}%
\begin{pgfscope}%
\pgfpathrectangle{\pgfqpoint{1.250000in}{0.550000in}}{\pgfqpoint{7.750000in}{3.850000in}}%
\pgfusepath{clip}%
\pgfsetbuttcap%
\pgfsetmiterjoin%
\definecolor{currentfill}{rgb}{0.501961,0.501961,0.501961}%
\pgfsetfillcolor{currentfill}%
\pgfsetfillopacity{0.700000}%
\pgfsetlinewidth{0.000000pt}%
\definecolor{currentstroke}{rgb}{0.000000,0.000000,0.000000}%
\pgfsetstrokecolor{currentstroke}%
\pgfsetstrokeopacity{0.700000}%
\pgfsetdash{}{0pt}%
\pgfpathmoveto{\pgfqpoint{7.794567in}{0.550000in}}%
\pgfpathlineto{\pgfqpoint{7.822088in}{0.550000in}}%
\pgfpathlineto{\pgfqpoint{7.822088in}{0.589095in}}%
\pgfpathlineto{\pgfqpoint{7.794567in}{0.589095in}}%
\pgfpathlineto{\pgfqpoint{7.794567in}{0.550000in}}%
\pgfpathclose%
\pgfusepath{fill}%
\end{pgfscope}%
\begin{pgfscope}%
\pgfpathrectangle{\pgfqpoint{1.250000in}{0.550000in}}{\pgfqpoint{7.750000in}{3.850000in}}%
\pgfusepath{clip}%
\pgfsetbuttcap%
\pgfsetmiterjoin%
\definecolor{currentfill}{rgb}{0.501961,0.501961,0.501961}%
\pgfsetfillcolor{currentfill}%
\pgfsetfillopacity{0.700000}%
\pgfsetlinewidth{0.000000pt}%
\definecolor{currentstroke}{rgb}{0.000000,0.000000,0.000000}%
\pgfsetstrokecolor{currentstroke}%
\pgfsetstrokeopacity{0.700000}%
\pgfsetdash{}{0pt}%
\pgfpathmoveto{\pgfqpoint{7.822088in}{0.550000in}}%
\pgfpathlineto{\pgfqpoint{7.849609in}{0.550000in}}%
\pgfpathlineto{\pgfqpoint{7.849609in}{0.580835in}}%
\pgfpathlineto{\pgfqpoint{7.822088in}{0.580835in}}%
\pgfpathlineto{\pgfqpoint{7.822088in}{0.550000in}}%
\pgfpathclose%
\pgfusepath{fill}%
\end{pgfscope}%
\begin{pgfscope}%
\pgfpathrectangle{\pgfqpoint{1.250000in}{0.550000in}}{\pgfqpoint{7.750000in}{3.850000in}}%
\pgfusepath{clip}%
\pgfsetbuttcap%
\pgfsetmiterjoin%
\definecolor{currentfill}{rgb}{0.501961,0.501961,0.501961}%
\pgfsetfillcolor{currentfill}%
\pgfsetfillopacity{0.700000}%
\pgfsetlinewidth{0.000000pt}%
\definecolor{currentstroke}{rgb}{0.000000,0.000000,0.000000}%
\pgfsetstrokecolor{currentstroke}%
\pgfsetstrokeopacity{0.700000}%
\pgfsetdash{}{0pt}%
\pgfpathmoveto{\pgfqpoint{7.849609in}{0.550000in}}%
\pgfpathlineto{\pgfqpoint{7.877131in}{0.550000in}}%
\pgfpathlineto{\pgfqpoint{7.877131in}{0.578082in}}%
\pgfpathlineto{\pgfqpoint{7.849609in}{0.578082in}}%
\pgfpathlineto{\pgfqpoint{7.849609in}{0.550000in}}%
\pgfpathclose%
\pgfusepath{fill}%
\end{pgfscope}%
\begin{pgfscope}%
\pgfpathrectangle{\pgfqpoint{1.250000in}{0.550000in}}{\pgfqpoint{7.750000in}{3.850000in}}%
\pgfusepath{clip}%
\pgfsetbuttcap%
\pgfsetmiterjoin%
\definecolor{currentfill}{rgb}{0.501961,0.501961,0.501961}%
\pgfsetfillcolor{currentfill}%
\pgfsetfillopacity{0.700000}%
\pgfsetlinewidth{0.000000pt}%
\definecolor{currentstroke}{rgb}{0.000000,0.000000,0.000000}%
\pgfsetstrokecolor{currentstroke}%
\pgfsetstrokeopacity{0.700000}%
\pgfsetdash{}{0pt}%
\pgfpathmoveto{\pgfqpoint{7.877131in}{0.550000in}}%
\pgfpathlineto{\pgfqpoint{7.904652in}{0.550000in}}%
\pgfpathlineto{\pgfqpoint{7.904652in}{0.588544in}}%
\pgfpathlineto{\pgfqpoint{7.877131in}{0.588544in}}%
\pgfpathlineto{\pgfqpoint{7.877131in}{0.550000in}}%
\pgfpathclose%
\pgfusepath{fill}%
\end{pgfscope}%
\begin{pgfscope}%
\pgfpathrectangle{\pgfqpoint{1.250000in}{0.550000in}}{\pgfqpoint{7.750000in}{3.850000in}}%
\pgfusepath{clip}%
\pgfsetbuttcap%
\pgfsetmiterjoin%
\definecolor{currentfill}{rgb}{0.501961,0.501961,0.501961}%
\pgfsetfillcolor{currentfill}%
\pgfsetfillopacity{0.700000}%
\pgfsetlinewidth{0.000000pt}%
\definecolor{currentstroke}{rgb}{0.000000,0.000000,0.000000}%
\pgfsetstrokecolor{currentstroke}%
\pgfsetstrokeopacity{0.700000}%
\pgfsetdash{}{0pt}%
\pgfpathmoveto{\pgfqpoint{7.904652in}{0.550000in}}%
\pgfpathlineto{\pgfqpoint{7.932173in}{0.550000in}}%
\pgfpathlineto{\pgfqpoint{7.932173in}{0.582212in}}%
\pgfpathlineto{\pgfqpoint{7.904652in}{0.582212in}}%
\pgfpathlineto{\pgfqpoint{7.904652in}{0.550000in}}%
\pgfpathclose%
\pgfusepath{fill}%
\end{pgfscope}%
\begin{pgfscope}%
\pgfpathrectangle{\pgfqpoint{1.250000in}{0.550000in}}{\pgfqpoint{7.750000in}{3.850000in}}%
\pgfusepath{clip}%
\pgfsetbuttcap%
\pgfsetmiterjoin%
\definecolor{currentfill}{rgb}{0.501961,0.501961,0.501961}%
\pgfsetfillcolor{currentfill}%
\pgfsetfillopacity{0.700000}%
\pgfsetlinewidth{0.000000pt}%
\definecolor{currentstroke}{rgb}{0.000000,0.000000,0.000000}%
\pgfsetstrokecolor{currentstroke}%
\pgfsetstrokeopacity{0.700000}%
\pgfsetdash{}{0pt}%
\pgfpathmoveto{\pgfqpoint{7.932173in}{0.550000in}}%
\pgfpathlineto{\pgfqpoint{7.959695in}{0.550000in}}%
\pgfpathlineto{\pgfqpoint{7.959695in}{0.583589in}}%
\pgfpathlineto{\pgfqpoint{7.932173in}{0.583589in}}%
\pgfpathlineto{\pgfqpoint{7.932173in}{0.550000in}}%
\pgfpathclose%
\pgfusepath{fill}%
\end{pgfscope}%
\begin{pgfscope}%
\pgfpathrectangle{\pgfqpoint{1.250000in}{0.550000in}}{\pgfqpoint{7.750000in}{3.850000in}}%
\pgfusepath{clip}%
\pgfsetbuttcap%
\pgfsetmiterjoin%
\definecolor{currentfill}{rgb}{0.501961,0.501961,0.501961}%
\pgfsetfillcolor{currentfill}%
\pgfsetfillopacity{0.700000}%
\pgfsetlinewidth{0.000000pt}%
\definecolor{currentstroke}{rgb}{0.000000,0.000000,0.000000}%
\pgfsetstrokecolor{currentstroke}%
\pgfsetstrokeopacity{0.700000}%
\pgfsetdash{}{0pt}%
\pgfpathmoveto{\pgfqpoint{7.959695in}{0.550000in}}%
\pgfpathlineto{\pgfqpoint{7.987216in}{0.550000in}}%
\pgfpathlineto{\pgfqpoint{7.987216in}{0.582487in}}%
\pgfpathlineto{\pgfqpoint{7.959695in}{0.582487in}}%
\pgfpathlineto{\pgfqpoint{7.959695in}{0.550000in}}%
\pgfpathclose%
\pgfusepath{fill}%
\end{pgfscope}%
\begin{pgfscope}%
\pgfpathrectangle{\pgfqpoint{1.250000in}{0.550000in}}{\pgfqpoint{7.750000in}{3.850000in}}%
\pgfusepath{clip}%
\pgfsetbuttcap%
\pgfsetmiterjoin%
\definecolor{currentfill}{rgb}{0.501961,0.501961,0.501961}%
\pgfsetfillcolor{currentfill}%
\pgfsetfillopacity{0.700000}%
\pgfsetlinewidth{0.000000pt}%
\definecolor{currentstroke}{rgb}{0.000000,0.000000,0.000000}%
\pgfsetstrokecolor{currentstroke}%
\pgfsetstrokeopacity{0.700000}%
\pgfsetdash{}{0pt}%
\pgfpathmoveto{\pgfqpoint{7.987216in}{0.550000in}}%
\pgfpathlineto{\pgfqpoint{8.014737in}{0.550000in}}%
\pgfpathlineto{\pgfqpoint{8.014737in}{0.589646in}}%
\pgfpathlineto{\pgfqpoint{7.987216in}{0.589646in}}%
\pgfpathlineto{\pgfqpoint{7.987216in}{0.550000in}}%
\pgfpathclose%
\pgfusepath{fill}%
\end{pgfscope}%
\begin{pgfscope}%
\pgfpathrectangle{\pgfqpoint{1.250000in}{0.550000in}}{\pgfqpoint{7.750000in}{3.850000in}}%
\pgfusepath{clip}%
\pgfsetbuttcap%
\pgfsetmiterjoin%
\definecolor{currentfill}{rgb}{0.501961,0.501961,0.501961}%
\pgfsetfillcolor{currentfill}%
\pgfsetfillopacity{0.700000}%
\pgfsetlinewidth{0.000000pt}%
\definecolor{currentstroke}{rgb}{0.000000,0.000000,0.000000}%
\pgfsetstrokecolor{currentstroke}%
\pgfsetstrokeopacity{0.700000}%
\pgfsetdash{}{0pt}%
\pgfpathmoveto{\pgfqpoint{8.014737in}{0.550000in}}%
\pgfpathlineto{\pgfqpoint{8.042259in}{0.550000in}}%
\pgfpathlineto{\pgfqpoint{8.042259in}{0.584139in}}%
\pgfpathlineto{\pgfqpoint{8.014737in}{0.584139in}}%
\pgfpathlineto{\pgfqpoint{8.014737in}{0.550000in}}%
\pgfpathclose%
\pgfusepath{fill}%
\end{pgfscope}%
\begin{pgfscope}%
\pgfpathrectangle{\pgfqpoint{1.250000in}{0.550000in}}{\pgfqpoint{7.750000in}{3.850000in}}%
\pgfusepath{clip}%
\pgfsetbuttcap%
\pgfsetmiterjoin%
\definecolor{currentfill}{rgb}{0.501961,0.501961,0.501961}%
\pgfsetfillcolor{currentfill}%
\pgfsetfillopacity{0.700000}%
\pgfsetlinewidth{0.000000pt}%
\definecolor{currentstroke}{rgb}{0.000000,0.000000,0.000000}%
\pgfsetstrokecolor{currentstroke}%
\pgfsetstrokeopacity{0.700000}%
\pgfsetdash{}{0pt}%
\pgfpathmoveto{\pgfqpoint{8.042259in}{0.550000in}}%
\pgfpathlineto{\pgfqpoint{8.069780in}{0.550000in}}%
\pgfpathlineto{\pgfqpoint{8.069780in}{0.581937in}}%
\pgfpathlineto{\pgfqpoint{8.042259in}{0.581937in}}%
\pgfpathlineto{\pgfqpoint{8.042259in}{0.550000in}}%
\pgfpathclose%
\pgfusepath{fill}%
\end{pgfscope}%
\begin{pgfscope}%
\pgfpathrectangle{\pgfqpoint{1.250000in}{0.550000in}}{\pgfqpoint{7.750000in}{3.850000in}}%
\pgfusepath{clip}%
\pgfsetbuttcap%
\pgfsetmiterjoin%
\definecolor{currentfill}{rgb}{0.501961,0.501961,0.501961}%
\pgfsetfillcolor{currentfill}%
\pgfsetfillopacity{0.700000}%
\pgfsetlinewidth{0.000000pt}%
\definecolor{currentstroke}{rgb}{0.000000,0.000000,0.000000}%
\pgfsetstrokecolor{currentstroke}%
\pgfsetstrokeopacity{0.700000}%
\pgfsetdash{}{0pt}%
\pgfpathmoveto{\pgfqpoint{8.069780in}{0.550000in}}%
\pgfpathlineto{\pgfqpoint{8.097301in}{0.550000in}}%
\pgfpathlineto{\pgfqpoint{8.097301in}{0.585241in}}%
\pgfpathlineto{\pgfqpoint{8.069780in}{0.585241in}}%
\pgfpathlineto{\pgfqpoint{8.069780in}{0.550000in}}%
\pgfpathclose%
\pgfusepath{fill}%
\end{pgfscope}%
\begin{pgfscope}%
\pgfpathrectangle{\pgfqpoint{1.250000in}{0.550000in}}{\pgfqpoint{7.750000in}{3.850000in}}%
\pgfusepath{clip}%
\pgfsetbuttcap%
\pgfsetmiterjoin%
\definecolor{currentfill}{rgb}{0.501961,0.501961,0.501961}%
\pgfsetfillcolor{currentfill}%
\pgfsetfillopacity{0.700000}%
\pgfsetlinewidth{0.000000pt}%
\definecolor{currentstroke}{rgb}{0.000000,0.000000,0.000000}%
\pgfsetstrokecolor{currentstroke}%
\pgfsetstrokeopacity{0.700000}%
\pgfsetdash{}{0pt}%
\pgfpathmoveto{\pgfqpoint{8.097301in}{0.550000in}}%
\pgfpathlineto{\pgfqpoint{8.124822in}{0.550000in}}%
\pgfpathlineto{\pgfqpoint{8.124822in}{0.580835in}}%
\pgfpathlineto{\pgfqpoint{8.097301in}{0.580835in}}%
\pgfpathlineto{\pgfqpoint{8.097301in}{0.550000in}}%
\pgfpathclose%
\pgfusepath{fill}%
\end{pgfscope}%
\begin{pgfscope}%
\pgfpathrectangle{\pgfqpoint{1.250000in}{0.550000in}}{\pgfqpoint{7.750000in}{3.850000in}}%
\pgfusepath{clip}%
\pgfsetbuttcap%
\pgfsetmiterjoin%
\definecolor{currentfill}{rgb}{0.501961,0.501961,0.501961}%
\pgfsetfillcolor{currentfill}%
\pgfsetfillopacity{0.700000}%
\pgfsetlinewidth{0.000000pt}%
\definecolor{currentstroke}{rgb}{0.000000,0.000000,0.000000}%
\pgfsetstrokecolor{currentstroke}%
\pgfsetstrokeopacity{0.700000}%
\pgfsetdash{}{0pt}%
\pgfpathmoveto{\pgfqpoint{8.124822in}{0.550000in}}%
\pgfpathlineto{\pgfqpoint{8.152344in}{0.550000in}}%
\pgfpathlineto{\pgfqpoint{8.152344in}{0.577532in}}%
\pgfpathlineto{\pgfqpoint{8.124822in}{0.577532in}}%
\pgfpathlineto{\pgfqpoint{8.124822in}{0.550000in}}%
\pgfpathclose%
\pgfusepath{fill}%
\end{pgfscope}%
\begin{pgfscope}%
\pgfpathrectangle{\pgfqpoint{1.250000in}{0.550000in}}{\pgfqpoint{7.750000in}{3.850000in}}%
\pgfusepath{clip}%
\pgfsetbuttcap%
\pgfsetmiterjoin%
\definecolor{currentfill}{rgb}{0.501961,0.501961,0.501961}%
\pgfsetfillcolor{currentfill}%
\pgfsetfillopacity{0.700000}%
\pgfsetlinewidth{0.000000pt}%
\definecolor{currentstroke}{rgb}{0.000000,0.000000,0.000000}%
\pgfsetstrokecolor{currentstroke}%
\pgfsetstrokeopacity{0.700000}%
\pgfsetdash{}{0pt}%
\pgfpathmoveto{\pgfqpoint{8.152344in}{0.550000in}}%
\pgfpathlineto{\pgfqpoint{8.179865in}{0.550000in}}%
\pgfpathlineto{\pgfqpoint{8.179865in}{0.583589in}}%
\pgfpathlineto{\pgfqpoint{8.152344in}{0.583589in}}%
\pgfpathlineto{\pgfqpoint{8.152344in}{0.550000in}}%
\pgfpathclose%
\pgfusepath{fill}%
\end{pgfscope}%
\begin{pgfscope}%
\pgfpathrectangle{\pgfqpoint{1.250000in}{0.550000in}}{\pgfqpoint{7.750000in}{3.850000in}}%
\pgfusepath{clip}%
\pgfsetbuttcap%
\pgfsetmiterjoin%
\definecolor{currentfill}{rgb}{0.501961,0.501961,0.501961}%
\pgfsetfillcolor{currentfill}%
\pgfsetfillopacity{0.700000}%
\pgfsetlinewidth{0.000000pt}%
\definecolor{currentstroke}{rgb}{0.000000,0.000000,0.000000}%
\pgfsetstrokecolor{currentstroke}%
\pgfsetstrokeopacity{0.700000}%
\pgfsetdash{}{0pt}%
\pgfpathmoveto{\pgfqpoint{8.179865in}{0.550000in}}%
\pgfpathlineto{\pgfqpoint{8.207386in}{0.550000in}}%
\pgfpathlineto{\pgfqpoint{8.207386in}{0.579734in}}%
\pgfpathlineto{\pgfqpoint{8.179865in}{0.579734in}}%
\pgfpathlineto{\pgfqpoint{8.179865in}{0.550000in}}%
\pgfpathclose%
\pgfusepath{fill}%
\end{pgfscope}%
\begin{pgfscope}%
\pgfpathrectangle{\pgfqpoint{1.250000in}{0.550000in}}{\pgfqpoint{7.750000in}{3.850000in}}%
\pgfusepath{clip}%
\pgfsetbuttcap%
\pgfsetmiterjoin%
\definecolor{currentfill}{rgb}{0.501961,0.501961,0.501961}%
\pgfsetfillcolor{currentfill}%
\pgfsetfillopacity{0.700000}%
\pgfsetlinewidth{0.000000pt}%
\definecolor{currentstroke}{rgb}{0.000000,0.000000,0.000000}%
\pgfsetstrokecolor{currentstroke}%
\pgfsetstrokeopacity{0.700000}%
\pgfsetdash{}{0pt}%
\pgfpathmoveto{\pgfqpoint{8.207386in}{0.550000in}}%
\pgfpathlineto{\pgfqpoint{8.234908in}{0.550000in}}%
\pgfpathlineto{\pgfqpoint{8.234908in}{0.583313in}}%
\pgfpathlineto{\pgfqpoint{8.207386in}{0.583313in}}%
\pgfpathlineto{\pgfqpoint{8.207386in}{0.550000in}}%
\pgfpathclose%
\pgfusepath{fill}%
\end{pgfscope}%
\begin{pgfscope}%
\pgfpathrectangle{\pgfqpoint{1.250000in}{0.550000in}}{\pgfqpoint{7.750000in}{3.850000in}}%
\pgfusepath{clip}%
\pgfsetbuttcap%
\pgfsetmiterjoin%
\definecolor{currentfill}{rgb}{0.501961,0.501961,0.501961}%
\pgfsetfillcolor{currentfill}%
\pgfsetfillopacity{0.700000}%
\pgfsetlinewidth{0.000000pt}%
\definecolor{currentstroke}{rgb}{0.000000,0.000000,0.000000}%
\pgfsetstrokecolor{currentstroke}%
\pgfsetstrokeopacity{0.700000}%
\pgfsetdash{}{0pt}%
\pgfpathmoveto{\pgfqpoint{8.234908in}{0.550000in}}%
\pgfpathlineto{\pgfqpoint{8.262429in}{0.550000in}}%
\pgfpathlineto{\pgfqpoint{8.262429in}{0.580285in}}%
\pgfpathlineto{\pgfqpoint{8.234908in}{0.580285in}}%
\pgfpathlineto{\pgfqpoint{8.234908in}{0.550000in}}%
\pgfpathclose%
\pgfusepath{fill}%
\end{pgfscope}%
\begin{pgfscope}%
\pgfpathrectangle{\pgfqpoint{1.250000in}{0.550000in}}{\pgfqpoint{7.750000in}{3.850000in}}%
\pgfusepath{clip}%
\pgfsetbuttcap%
\pgfsetmiterjoin%
\definecolor{currentfill}{rgb}{0.501961,0.501961,0.501961}%
\pgfsetfillcolor{currentfill}%
\pgfsetfillopacity{0.700000}%
\pgfsetlinewidth{0.000000pt}%
\definecolor{currentstroke}{rgb}{0.000000,0.000000,0.000000}%
\pgfsetstrokecolor{currentstroke}%
\pgfsetstrokeopacity{0.700000}%
\pgfsetdash{}{0pt}%
\pgfpathmoveto{\pgfqpoint{8.262429in}{0.550000in}}%
\pgfpathlineto{\pgfqpoint{8.289950in}{0.550000in}}%
\pgfpathlineto{\pgfqpoint{8.289950in}{0.578082in}}%
\pgfpathlineto{\pgfqpoint{8.262429in}{0.578082in}}%
\pgfpathlineto{\pgfqpoint{8.262429in}{0.550000in}}%
\pgfpathclose%
\pgfusepath{fill}%
\end{pgfscope}%
\begin{pgfscope}%
\pgfpathrectangle{\pgfqpoint{1.250000in}{0.550000in}}{\pgfqpoint{7.750000in}{3.850000in}}%
\pgfusepath{clip}%
\pgfsetbuttcap%
\pgfsetmiterjoin%
\definecolor{currentfill}{rgb}{0.501961,0.501961,0.501961}%
\pgfsetfillcolor{currentfill}%
\pgfsetfillopacity{0.700000}%
\pgfsetlinewidth{0.000000pt}%
\definecolor{currentstroke}{rgb}{0.000000,0.000000,0.000000}%
\pgfsetstrokecolor{currentstroke}%
\pgfsetstrokeopacity{0.700000}%
\pgfsetdash{}{0pt}%
\pgfpathmoveto{\pgfqpoint{8.289950in}{0.550000in}}%
\pgfpathlineto{\pgfqpoint{8.317472in}{0.550000in}}%
\pgfpathlineto{\pgfqpoint{8.317472in}{0.584965in}}%
\pgfpathlineto{\pgfqpoint{8.289950in}{0.584965in}}%
\pgfpathlineto{\pgfqpoint{8.289950in}{0.550000in}}%
\pgfpathclose%
\pgfusepath{fill}%
\end{pgfscope}%
\begin{pgfscope}%
\pgfpathrectangle{\pgfqpoint{1.250000in}{0.550000in}}{\pgfqpoint{7.750000in}{3.850000in}}%
\pgfusepath{clip}%
\pgfsetbuttcap%
\pgfsetmiterjoin%
\definecolor{currentfill}{rgb}{0.501961,0.501961,0.501961}%
\pgfsetfillcolor{currentfill}%
\pgfsetfillopacity{0.700000}%
\pgfsetlinewidth{0.000000pt}%
\definecolor{currentstroke}{rgb}{0.000000,0.000000,0.000000}%
\pgfsetstrokecolor{currentstroke}%
\pgfsetstrokeopacity{0.700000}%
\pgfsetdash{}{0pt}%
\pgfpathmoveto{\pgfqpoint{8.317472in}{0.550000in}}%
\pgfpathlineto{\pgfqpoint{8.344993in}{0.550000in}}%
\pgfpathlineto{\pgfqpoint{8.344993in}{0.577532in}}%
\pgfpathlineto{\pgfqpoint{8.317472in}{0.577532in}}%
\pgfpathlineto{\pgfqpoint{8.317472in}{0.550000in}}%
\pgfpathclose%
\pgfusepath{fill}%
\end{pgfscope}%
\begin{pgfscope}%
\pgfpathrectangle{\pgfqpoint{1.250000in}{0.550000in}}{\pgfqpoint{7.750000in}{3.850000in}}%
\pgfusepath{clip}%
\pgfsetbuttcap%
\pgfsetmiterjoin%
\definecolor{currentfill}{rgb}{0.501961,0.501961,0.501961}%
\pgfsetfillcolor{currentfill}%
\pgfsetfillopacity{0.700000}%
\pgfsetlinewidth{0.000000pt}%
\definecolor{currentstroke}{rgb}{0.000000,0.000000,0.000000}%
\pgfsetstrokecolor{currentstroke}%
\pgfsetstrokeopacity{0.700000}%
\pgfsetdash{}{0pt}%
\pgfpathmoveto{\pgfqpoint{8.344993in}{0.550000in}}%
\pgfpathlineto{\pgfqpoint{8.372514in}{0.550000in}}%
\pgfpathlineto{\pgfqpoint{8.372514in}{0.578082in}}%
\pgfpathlineto{\pgfqpoint{8.344993in}{0.578082in}}%
\pgfpathlineto{\pgfqpoint{8.344993in}{0.550000in}}%
\pgfpathclose%
\pgfusepath{fill}%
\end{pgfscope}%
\begin{pgfscope}%
\pgfpathrectangle{\pgfqpoint{1.250000in}{0.550000in}}{\pgfqpoint{7.750000in}{3.850000in}}%
\pgfusepath{clip}%
\pgfsetbuttcap%
\pgfsetmiterjoin%
\definecolor{currentfill}{rgb}{0.501961,0.501961,0.501961}%
\pgfsetfillcolor{currentfill}%
\pgfsetfillopacity{0.700000}%
\pgfsetlinewidth{0.000000pt}%
\definecolor{currentstroke}{rgb}{0.000000,0.000000,0.000000}%
\pgfsetstrokecolor{currentstroke}%
\pgfsetstrokeopacity{0.700000}%
\pgfsetdash{}{0pt}%
\pgfpathmoveto{\pgfqpoint{8.372514in}{0.550000in}}%
\pgfpathlineto{\pgfqpoint{8.400036in}{0.550000in}}%
\pgfpathlineto{\pgfqpoint{8.400036in}{0.581937in}}%
\pgfpathlineto{\pgfqpoint{8.372514in}{0.581937in}}%
\pgfpathlineto{\pgfqpoint{8.372514in}{0.550000in}}%
\pgfpathclose%
\pgfusepath{fill}%
\end{pgfscope}%
\begin{pgfscope}%
\pgfpathrectangle{\pgfqpoint{1.250000in}{0.550000in}}{\pgfqpoint{7.750000in}{3.850000in}}%
\pgfusepath{clip}%
\pgfsetbuttcap%
\pgfsetmiterjoin%
\definecolor{currentfill}{rgb}{0.501961,0.501961,0.501961}%
\pgfsetfillcolor{currentfill}%
\pgfsetfillopacity{0.700000}%
\pgfsetlinewidth{0.000000pt}%
\definecolor{currentstroke}{rgb}{0.000000,0.000000,0.000000}%
\pgfsetstrokecolor{currentstroke}%
\pgfsetstrokeopacity{0.700000}%
\pgfsetdash{}{0pt}%
\pgfpathmoveto{\pgfqpoint{8.400036in}{0.550000in}}%
\pgfpathlineto{\pgfqpoint{8.427557in}{0.550000in}}%
\pgfpathlineto{\pgfqpoint{8.427557in}{0.585791in}}%
\pgfpathlineto{\pgfqpoint{8.400036in}{0.585791in}}%
\pgfpathlineto{\pgfqpoint{8.400036in}{0.550000in}}%
\pgfpathclose%
\pgfusepath{fill}%
\end{pgfscope}%
\begin{pgfscope}%
\pgfpathrectangle{\pgfqpoint{1.250000in}{0.550000in}}{\pgfqpoint{7.750000in}{3.850000in}}%
\pgfusepath{clip}%
\pgfsetbuttcap%
\pgfsetmiterjoin%
\definecolor{currentfill}{rgb}{0.501961,0.501961,0.501961}%
\pgfsetfillcolor{currentfill}%
\pgfsetfillopacity{0.700000}%
\pgfsetlinewidth{0.000000pt}%
\definecolor{currentstroke}{rgb}{0.000000,0.000000,0.000000}%
\pgfsetstrokecolor{currentstroke}%
\pgfsetstrokeopacity{0.700000}%
\pgfsetdash{}{0pt}%
\pgfpathmoveto{\pgfqpoint{8.427557in}{0.550000in}}%
\pgfpathlineto{\pgfqpoint{8.455078in}{0.550000in}}%
\pgfpathlineto{\pgfqpoint{8.455078in}{0.577807in}}%
\pgfpathlineto{\pgfqpoint{8.427557in}{0.577807in}}%
\pgfpathlineto{\pgfqpoint{8.427557in}{0.550000in}}%
\pgfpathclose%
\pgfusepath{fill}%
\end{pgfscope}%
\begin{pgfscope}%
\pgfpathrectangle{\pgfqpoint{1.250000in}{0.550000in}}{\pgfqpoint{7.750000in}{3.850000in}}%
\pgfusepath{clip}%
\pgfsetbuttcap%
\pgfsetmiterjoin%
\definecolor{currentfill}{rgb}{0.501961,0.501961,0.501961}%
\pgfsetfillcolor{currentfill}%
\pgfsetfillopacity{0.700000}%
\pgfsetlinewidth{0.000000pt}%
\definecolor{currentstroke}{rgb}{0.000000,0.000000,0.000000}%
\pgfsetstrokecolor{currentstroke}%
\pgfsetstrokeopacity{0.700000}%
\pgfsetdash{}{0pt}%
\pgfpathmoveto{\pgfqpoint{8.455078in}{0.550000in}}%
\pgfpathlineto{\pgfqpoint{8.482599in}{0.550000in}}%
\pgfpathlineto{\pgfqpoint{8.482599in}{0.585791in}}%
\pgfpathlineto{\pgfqpoint{8.455078in}{0.585791in}}%
\pgfpathlineto{\pgfqpoint{8.455078in}{0.550000in}}%
\pgfpathclose%
\pgfusepath{fill}%
\end{pgfscope}%
\begin{pgfscope}%
\pgfpathrectangle{\pgfqpoint{1.250000in}{0.550000in}}{\pgfqpoint{7.750000in}{3.850000in}}%
\pgfusepath{clip}%
\pgfsetbuttcap%
\pgfsetmiterjoin%
\definecolor{currentfill}{rgb}{0.501961,0.501961,0.501961}%
\pgfsetfillcolor{currentfill}%
\pgfsetfillopacity{0.700000}%
\pgfsetlinewidth{0.000000pt}%
\definecolor{currentstroke}{rgb}{0.000000,0.000000,0.000000}%
\pgfsetstrokecolor{currentstroke}%
\pgfsetstrokeopacity{0.700000}%
\pgfsetdash{}{0pt}%
\pgfpathmoveto{\pgfqpoint{8.482599in}{0.550000in}}%
\pgfpathlineto{\pgfqpoint{8.510121in}{0.550000in}}%
\pgfpathlineto{\pgfqpoint{8.510121in}{0.582487in}}%
\pgfpathlineto{\pgfqpoint{8.482599in}{0.582487in}}%
\pgfpathlineto{\pgfqpoint{8.482599in}{0.550000in}}%
\pgfpathclose%
\pgfusepath{fill}%
\end{pgfscope}%
\begin{pgfscope}%
\pgfpathrectangle{\pgfqpoint{1.250000in}{0.550000in}}{\pgfqpoint{7.750000in}{3.850000in}}%
\pgfusepath{clip}%
\pgfsetbuttcap%
\pgfsetmiterjoin%
\definecolor{currentfill}{rgb}{0.501961,0.501961,0.501961}%
\pgfsetfillcolor{currentfill}%
\pgfsetfillopacity{0.700000}%
\pgfsetlinewidth{0.000000pt}%
\definecolor{currentstroke}{rgb}{0.000000,0.000000,0.000000}%
\pgfsetstrokecolor{currentstroke}%
\pgfsetstrokeopacity{0.700000}%
\pgfsetdash{}{0pt}%
\pgfpathmoveto{\pgfqpoint{8.510121in}{0.550000in}}%
\pgfpathlineto{\pgfqpoint{8.537642in}{0.550000in}}%
\pgfpathlineto{\pgfqpoint{8.537642in}{0.587994in}}%
\pgfpathlineto{\pgfqpoint{8.510121in}{0.587994in}}%
\pgfpathlineto{\pgfqpoint{8.510121in}{0.550000in}}%
\pgfpathclose%
\pgfusepath{fill}%
\end{pgfscope}%
\begin{pgfscope}%
\pgfpathrectangle{\pgfqpoint{1.250000in}{0.550000in}}{\pgfqpoint{7.750000in}{3.850000in}}%
\pgfusepath{clip}%
\pgfsetbuttcap%
\pgfsetmiterjoin%
\definecolor{currentfill}{rgb}{0.501961,0.501961,0.501961}%
\pgfsetfillcolor{currentfill}%
\pgfsetfillopacity{0.700000}%
\pgfsetlinewidth{0.000000pt}%
\definecolor{currentstroke}{rgb}{0.000000,0.000000,0.000000}%
\pgfsetstrokecolor{currentstroke}%
\pgfsetstrokeopacity{0.700000}%
\pgfsetdash{}{0pt}%
\pgfpathmoveto{\pgfqpoint{8.537642in}{0.550000in}}%
\pgfpathlineto{\pgfqpoint{8.565163in}{0.550000in}}%
\pgfpathlineto{\pgfqpoint{8.565163in}{0.584139in}}%
\pgfpathlineto{\pgfqpoint{8.537642in}{0.584139in}}%
\pgfpathlineto{\pgfqpoint{8.537642in}{0.550000in}}%
\pgfpathclose%
\pgfusepath{fill}%
\end{pgfscope}%
\begin{pgfscope}%
\pgfpathrectangle{\pgfqpoint{1.250000in}{0.550000in}}{\pgfqpoint{7.750000in}{3.850000in}}%
\pgfusepath{clip}%
\pgfsetbuttcap%
\pgfsetmiterjoin%
\definecolor{currentfill}{rgb}{0.501961,0.501961,0.501961}%
\pgfsetfillcolor{currentfill}%
\pgfsetfillopacity{0.700000}%
\pgfsetlinewidth{0.000000pt}%
\definecolor{currentstroke}{rgb}{0.000000,0.000000,0.000000}%
\pgfsetstrokecolor{currentstroke}%
\pgfsetstrokeopacity{0.700000}%
\pgfsetdash{}{0pt}%
\pgfpathmoveto{\pgfqpoint{8.565163in}{0.550000in}}%
\pgfpathlineto{\pgfqpoint{8.592685in}{0.550000in}}%
\pgfpathlineto{\pgfqpoint{8.592685in}{0.575604in}}%
\pgfpathlineto{\pgfqpoint{8.565163in}{0.575604in}}%
\pgfpathlineto{\pgfqpoint{8.565163in}{0.550000in}}%
\pgfpathclose%
\pgfusepath{fill}%
\end{pgfscope}%
\begin{pgfscope}%
\pgfpathrectangle{\pgfqpoint{1.250000in}{0.550000in}}{\pgfqpoint{7.750000in}{3.850000in}}%
\pgfusepath{clip}%
\pgfsetbuttcap%
\pgfsetmiterjoin%
\definecolor{currentfill}{rgb}{0.501961,0.501961,0.501961}%
\pgfsetfillcolor{currentfill}%
\pgfsetfillopacity{0.700000}%
\pgfsetlinewidth{0.000000pt}%
\definecolor{currentstroke}{rgb}{0.000000,0.000000,0.000000}%
\pgfsetstrokecolor{currentstroke}%
\pgfsetstrokeopacity{0.700000}%
\pgfsetdash{}{0pt}%
\pgfpathmoveto{\pgfqpoint{8.592685in}{0.550000in}}%
\pgfpathlineto{\pgfqpoint{8.620206in}{0.550000in}}%
\pgfpathlineto{\pgfqpoint{8.620206in}{0.563215in}}%
\pgfpathlineto{\pgfqpoint{8.592685in}{0.563215in}}%
\pgfpathlineto{\pgfqpoint{8.592685in}{0.550000in}}%
\pgfpathclose%
\pgfusepath{fill}%
\end{pgfscope}%
\begin{pgfscope}%
\pgfpathrectangle{\pgfqpoint{1.250000in}{0.550000in}}{\pgfqpoint{7.750000in}{3.850000in}}%
\pgfusepath{clip}%
\pgfsetbuttcap%
\pgfsetmiterjoin%
\definecolor{currentfill}{rgb}{0.501961,0.501961,0.501961}%
\pgfsetfillcolor{currentfill}%
\pgfsetfillopacity{0.700000}%
\pgfsetlinewidth{0.000000pt}%
\definecolor{currentstroke}{rgb}{0.000000,0.000000,0.000000}%
\pgfsetstrokecolor{currentstroke}%
\pgfsetstrokeopacity{0.700000}%
\pgfsetdash{}{0pt}%
\pgfpathmoveto{\pgfqpoint{8.620206in}{0.550000in}}%
\pgfpathlineto{\pgfqpoint{8.647727in}{0.550000in}}%
\pgfpathlineto{\pgfqpoint{8.647727in}{0.559361in}}%
\pgfpathlineto{\pgfqpoint{8.620206in}{0.559361in}}%
\pgfpathlineto{\pgfqpoint{8.620206in}{0.550000in}}%
\pgfpathclose%
\pgfusepath{fill}%
\end{pgfscope}%
\begin{pgfscope}%
\pgfsetbuttcap%
\pgfsetroundjoin%
\definecolor{currentfill}{rgb}{0.000000,0.000000,0.000000}%
\pgfsetfillcolor{currentfill}%
\pgfsetlinewidth{0.803000pt}%
\definecolor{currentstroke}{rgb}{0.000000,0.000000,0.000000}%
\pgfsetstrokecolor{currentstroke}%
\pgfsetdash{}{0pt}%
\pgfsys@defobject{currentmarker}{\pgfqpoint{0.000000in}{-0.048611in}}{\pgfqpoint{0.000000in}{0.000000in}}{%
\pgfpathmoveto{\pgfqpoint{0.000000in}{0.000000in}}%
\pgfpathlineto{\pgfqpoint{0.000000in}{-0.048611in}}%
\pgfusepath{stroke,fill}%
}%
\begin{pgfscope}%
\pgfsys@transformshift{1.602273in}{0.550000in}%
\pgfsys@useobject{currentmarker}{}%
\end{pgfscope}%
\end{pgfscope}%
\begin{pgfscope}%
\definecolor{textcolor}{rgb}{0.000000,0.000000,0.000000}%
\pgfsetstrokecolor{textcolor}%
\pgfsetfillcolor{textcolor}%
\pgftext[x=1.602273in,y=0.452778in,,top]{\color{textcolor}{\sffamily\fontsize{10.000000}{12.000000}\selectfont\catcode`\^=\active\def^{\ifmmode\sp\else\^{}\fi}\catcode`\%=\active\def%{\%}0}}%
\end{pgfscope}%
\begin{pgfscope}%
\pgfsetbuttcap%
\pgfsetroundjoin%
\definecolor{currentfill}{rgb}{0.000000,0.000000,0.000000}%
\pgfsetfillcolor{currentfill}%
\pgfsetlinewidth{0.803000pt}%
\definecolor{currentstroke}{rgb}{0.000000,0.000000,0.000000}%
\pgfsetstrokecolor{currentstroke}%
\pgfsetdash{}{0pt}%
\pgfsys@defobject{currentmarker}{\pgfqpoint{0.000000in}{-0.048611in}}{\pgfqpoint{0.000000in}{0.000000in}}{%
\pgfpathmoveto{\pgfqpoint{0.000000in}{0.000000in}}%
\pgfpathlineto{\pgfqpoint{0.000000in}{-0.048611in}}%
\pgfusepath{stroke,fill}%
}%
\begin{pgfscope}%
\pgfsys@transformshift{2.978338in}{0.550000in}%
\pgfsys@useobject{currentmarker}{}%
\end{pgfscope}%
\end{pgfscope}%
\begin{pgfscope}%
\definecolor{textcolor}{rgb}{0.000000,0.000000,0.000000}%
\pgfsetstrokecolor{textcolor}%
\pgfsetfillcolor{textcolor}%
\pgftext[x=2.978338in,y=0.452778in,,top]{\color{textcolor}{\sffamily\fontsize{10.000000}{12.000000}\selectfont\catcode`\^=\active\def^{\ifmmode\sp\else\^{}\fi}\catcode`\%=\active\def%{\%}50}}%
\end{pgfscope}%
\begin{pgfscope}%
\pgfsetbuttcap%
\pgfsetroundjoin%
\definecolor{currentfill}{rgb}{0.000000,0.000000,0.000000}%
\pgfsetfillcolor{currentfill}%
\pgfsetlinewidth{0.803000pt}%
\definecolor{currentstroke}{rgb}{0.000000,0.000000,0.000000}%
\pgfsetstrokecolor{currentstroke}%
\pgfsetdash{}{0pt}%
\pgfsys@defobject{currentmarker}{\pgfqpoint{0.000000in}{-0.048611in}}{\pgfqpoint{0.000000in}{0.000000in}}{%
\pgfpathmoveto{\pgfqpoint{0.000000in}{0.000000in}}%
\pgfpathlineto{\pgfqpoint{0.000000in}{-0.048611in}}%
\pgfusepath{stroke,fill}%
}%
\begin{pgfscope}%
\pgfsys@transformshift{4.354403in}{0.550000in}%
\pgfsys@useobject{currentmarker}{}%
\end{pgfscope}%
\end{pgfscope}%
\begin{pgfscope}%
\definecolor{textcolor}{rgb}{0.000000,0.000000,0.000000}%
\pgfsetstrokecolor{textcolor}%
\pgfsetfillcolor{textcolor}%
\pgftext[x=4.354403in,y=0.452778in,,top]{\color{textcolor}{\sffamily\fontsize{10.000000}{12.000000}\selectfont\catcode`\^=\active\def^{\ifmmode\sp\else\^{}\fi}\catcode`\%=\active\def%{\%}100}}%
\end{pgfscope}%
\begin{pgfscope}%
\pgfsetbuttcap%
\pgfsetroundjoin%
\definecolor{currentfill}{rgb}{0.000000,0.000000,0.000000}%
\pgfsetfillcolor{currentfill}%
\pgfsetlinewidth{0.803000pt}%
\definecolor{currentstroke}{rgb}{0.000000,0.000000,0.000000}%
\pgfsetstrokecolor{currentstroke}%
\pgfsetdash{}{0pt}%
\pgfsys@defobject{currentmarker}{\pgfqpoint{0.000000in}{-0.048611in}}{\pgfqpoint{0.000000in}{0.000000in}}{%
\pgfpathmoveto{\pgfqpoint{0.000000in}{0.000000in}}%
\pgfpathlineto{\pgfqpoint{0.000000in}{-0.048611in}}%
\pgfusepath{stroke,fill}%
}%
\begin{pgfscope}%
\pgfsys@transformshift{5.730469in}{0.550000in}%
\pgfsys@useobject{currentmarker}{}%
\end{pgfscope}%
\end{pgfscope}%
\begin{pgfscope}%
\definecolor{textcolor}{rgb}{0.000000,0.000000,0.000000}%
\pgfsetstrokecolor{textcolor}%
\pgfsetfillcolor{textcolor}%
\pgftext[x=5.730469in,y=0.452778in,,top]{\color{textcolor}{\sffamily\fontsize{10.000000}{12.000000}\selectfont\catcode`\^=\active\def^{\ifmmode\sp\else\^{}\fi}\catcode`\%=\active\def%{\%}150}}%
\end{pgfscope}%
\begin{pgfscope}%
\pgfsetbuttcap%
\pgfsetroundjoin%
\definecolor{currentfill}{rgb}{0.000000,0.000000,0.000000}%
\pgfsetfillcolor{currentfill}%
\pgfsetlinewidth{0.803000pt}%
\definecolor{currentstroke}{rgb}{0.000000,0.000000,0.000000}%
\pgfsetstrokecolor{currentstroke}%
\pgfsetdash{}{0pt}%
\pgfsys@defobject{currentmarker}{\pgfqpoint{0.000000in}{-0.048611in}}{\pgfqpoint{0.000000in}{0.000000in}}{%
\pgfpathmoveto{\pgfqpoint{0.000000in}{0.000000in}}%
\pgfpathlineto{\pgfqpoint{0.000000in}{-0.048611in}}%
\pgfusepath{stroke,fill}%
}%
\begin{pgfscope}%
\pgfsys@transformshift{7.106534in}{0.550000in}%
\pgfsys@useobject{currentmarker}{}%
\end{pgfscope}%
\end{pgfscope}%
\begin{pgfscope}%
\definecolor{textcolor}{rgb}{0.000000,0.000000,0.000000}%
\pgfsetstrokecolor{textcolor}%
\pgfsetfillcolor{textcolor}%
\pgftext[x=7.106534in,y=0.452778in,,top]{\color{textcolor}{\sffamily\fontsize{10.000000}{12.000000}\selectfont\catcode`\^=\active\def^{\ifmmode\sp\else\^{}\fi}\catcode`\%=\active\def%{\%}200}}%
\end{pgfscope}%
\begin{pgfscope}%
\pgfsetbuttcap%
\pgfsetroundjoin%
\definecolor{currentfill}{rgb}{0.000000,0.000000,0.000000}%
\pgfsetfillcolor{currentfill}%
\pgfsetlinewidth{0.803000pt}%
\definecolor{currentstroke}{rgb}{0.000000,0.000000,0.000000}%
\pgfsetstrokecolor{currentstroke}%
\pgfsetdash{}{0pt}%
\pgfsys@defobject{currentmarker}{\pgfqpoint{0.000000in}{-0.048611in}}{\pgfqpoint{0.000000in}{0.000000in}}{%
\pgfpathmoveto{\pgfqpoint{0.000000in}{0.000000in}}%
\pgfpathlineto{\pgfqpoint{0.000000in}{-0.048611in}}%
\pgfusepath{stroke,fill}%
}%
\begin{pgfscope}%
\pgfsys@transformshift{8.482599in}{0.550000in}%
\pgfsys@useobject{currentmarker}{}%
\end{pgfscope}%
\end{pgfscope}%
\begin{pgfscope}%
\definecolor{textcolor}{rgb}{0.000000,0.000000,0.000000}%
\pgfsetstrokecolor{textcolor}%
\pgfsetfillcolor{textcolor}%
\pgftext[x=8.482599in,y=0.452778in,,top]{\color{textcolor}{\sffamily\fontsize{10.000000}{12.000000}\selectfont\catcode`\^=\active\def^{\ifmmode\sp\else\^{}\fi}\catcode`\%=\active\def%{\%}250}}%
\end{pgfscope}%
\begin{pgfscope}%
\definecolor{textcolor}{rgb}{0.000000,0.000000,0.000000}%
\pgfsetstrokecolor{textcolor}%
\pgfsetfillcolor{textcolor}%
\pgftext[x=5.125000in,y=0.262809in,,top]{\color{textcolor}{\sffamily\fontsize{10.000000}{12.000000}\selectfont\catcode`\^=\active\def^{\ifmmode\sp\else\^{}\fi}\catcode`\%=\active\def%{\%}Intensidad}}%
\end{pgfscope}%
\begin{pgfscope}%
\pgfsetbuttcap%
\pgfsetroundjoin%
\definecolor{currentfill}{rgb}{0.000000,0.000000,0.000000}%
\pgfsetfillcolor{currentfill}%
\pgfsetlinewidth{0.803000pt}%
\definecolor{currentstroke}{rgb}{0.000000,0.000000,0.000000}%
\pgfsetstrokecolor{currentstroke}%
\pgfsetdash{}{0pt}%
\pgfsys@defobject{currentmarker}{\pgfqpoint{-0.048611in}{0.000000in}}{\pgfqpoint{-0.000000in}{0.000000in}}{%
\pgfpathmoveto{\pgfqpoint{-0.000000in}{0.000000in}}%
\pgfpathlineto{\pgfqpoint{-0.048611in}{0.000000in}}%
\pgfusepath{stroke,fill}%
}%
\begin{pgfscope}%
\pgfsys@transformshift{1.250000in}{0.550000in}%
\pgfsys@useobject{currentmarker}{}%
\end{pgfscope}%
\end{pgfscope}%
\begin{pgfscope}%
\definecolor{textcolor}{rgb}{0.000000,0.000000,0.000000}%
\pgfsetstrokecolor{textcolor}%
\pgfsetfillcolor{textcolor}%
\pgftext[x=1.064412in, y=0.497238in, left, base]{\color{textcolor}{\sffamily\fontsize{10.000000}{12.000000}\selectfont\catcode`\^=\active\def^{\ifmmode\sp\else\^{}\fi}\catcode`\%=\active\def%{\%}0}}%
\end{pgfscope}%
\begin{pgfscope}%
\pgfsetbuttcap%
\pgfsetroundjoin%
\definecolor{currentfill}{rgb}{0.000000,0.000000,0.000000}%
\pgfsetfillcolor{currentfill}%
\pgfsetlinewidth{0.803000pt}%
\definecolor{currentstroke}{rgb}{0.000000,0.000000,0.000000}%
\pgfsetstrokecolor{currentstroke}%
\pgfsetdash{}{0pt}%
\pgfsys@defobject{currentmarker}{\pgfqpoint{-0.048611in}{0.000000in}}{\pgfqpoint{-0.000000in}{0.000000in}}{%
\pgfpathmoveto{\pgfqpoint{-0.000000in}{0.000000in}}%
\pgfpathlineto{\pgfqpoint{-0.048611in}{0.000000in}}%
\pgfusepath{stroke,fill}%
}%
\begin{pgfscope}%
\pgfsys@transformshift{1.250000in}{1.100633in}%
\pgfsys@useobject{currentmarker}{}%
\end{pgfscope}%
\end{pgfscope}%
\begin{pgfscope}%
\definecolor{textcolor}{rgb}{0.000000,0.000000,0.000000}%
\pgfsetstrokecolor{textcolor}%
\pgfsetfillcolor{textcolor}%
\pgftext[x=0.799316in, y=1.047872in, left, base]{\color{textcolor}{\sffamily\fontsize{10.000000}{12.000000}\selectfont\catcode`\^=\active\def^{\ifmmode\sp\else\^{}\fi}\catcode`\%=\active\def%{\%}2000}}%
\end{pgfscope}%
\begin{pgfscope}%
\pgfsetbuttcap%
\pgfsetroundjoin%
\definecolor{currentfill}{rgb}{0.000000,0.000000,0.000000}%
\pgfsetfillcolor{currentfill}%
\pgfsetlinewidth{0.803000pt}%
\definecolor{currentstroke}{rgb}{0.000000,0.000000,0.000000}%
\pgfsetstrokecolor{currentstroke}%
\pgfsetdash{}{0pt}%
\pgfsys@defobject{currentmarker}{\pgfqpoint{-0.048611in}{0.000000in}}{\pgfqpoint{-0.000000in}{0.000000in}}{%
\pgfpathmoveto{\pgfqpoint{-0.000000in}{0.000000in}}%
\pgfpathlineto{\pgfqpoint{-0.048611in}{0.000000in}}%
\pgfusepath{stroke,fill}%
}%
\begin{pgfscope}%
\pgfsys@transformshift{1.250000in}{1.651266in}%
\pgfsys@useobject{currentmarker}{}%
\end{pgfscope}%
\end{pgfscope}%
\begin{pgfscope}%
\definecolor{textcolor}{rgb}{0.000000,0.000000,0.000000}%
\pgfsetstrokecolor{textcolor}%
\pgfsetfillcolor{textcolor}%
\pgftext[x=0.799316in, y=1.598505in, left, base]{\color{textcolor}{\sffamily\fontsize{10.000000}{12.000000}\selectfont\catcode`\^=\active\def^{\ifmmode\sp\else\^{}\fi}\catcode`\%=\active\def%{\%}4000}}%
\end{pgfscope}%
\begin{pgfscope}%
\pgfsetbuttcap%
\pgfsetroundjoin%
\definecolor{currentfill}{rgb}{0.000000,0.000000,0.000000}%
\pgfsetfillcolor{currentfill}%
\pgfsetlinewidth{0.803000pt}%
\definecolor{currentstroke}{rgb}{0.000000,0.000000,0.000000}%
\pgfsetstrokecolor{currentstroke}%
\pgfsetdash{}{0pt}%
\pgfsys@defobject{currentmarker}{\pgfqpoint{-0.048611in}{0.000000in}}{\pgfqpoint{-0.000000in}{0.000000in}}{%
\pgfpathmoveto{\pgfqpoint{-0.000000in}{0.000000in}}%
\pgfpathlineto{\pgfqpoint{-0.048611in}{0.000000in}}%
\pgfusepath{stroke,fill}%
}%
\begin{pgfscope}%
\pgfsys@transformshift{1.250000in}{2.201900in}%
\pgfsys@useobject{currentmarker}{}%
\end{pgfscope}%
\end{pgfscope}%
\begin{pgfscope}%
\definecolor{textcolor}{rgb}{0.000000,0.000000,0.000000}%
\pgfsetstrokecolor{textcolor}%
\pgfsetfillcolor{textcolor}%
\pgftext[x=0.799316in, y=2.149138in, left, base]{\color{textcolor}{\sffamily\fontsize{10.000000}{12.000000}\selectfont\catcode`\^=\active\def^{\ifmmode\sp\else\^{}\fi}\catcode`\%=\active\def%{\%}6000}}%
\end{pgfscope}%
\begin{pgfscope}%
\pgfsetbuttcap%
\pgfsetroundjoin%
\definecolor{currentfill}{rgb}{0.000000,0.000000,0.000000}%
\pgfsetfillcolor{currentfill}%
\pgfsetlinewidth{0.803000pt}%
\definecolor{currentstroke}{rgb}{0.000000,0.000000,0.000000}%
\pgfsetstrokecolor{currentstroke}%
\pgfsetdash{}{0pt}%
\pgfsys@defobject{currentmarker}{\pgfqpoint{-0.048611in}{0.000000in}}{\pgfqpoint{-0.000000in}{0.000000in}}{%
\pgfpathmoveto{\pgfqpoint{-0.000000in}{0.000000in}}%
\pgfpathlineto{\pgfqpoint{-0.048611in}{0.000000in}}%
\pgfusepath{stroke,fill}%
}%
\begin{pgfscope}%
\pgfsys@transformshift{1.250000in}{2.752533in}%
\pgfsys@useobject{currentmarker}{}%
\end{pgfscope}%
\end{pgfscope}%
\begin{pgfscope}%
\definecolor{textcolor}{rgb}{0.000000,0.000000,0.000000}%
\pgfsetstrokecolor{textcolor}%
\pgfsetfillcolor{textcolor}%
\pgftext[x=0.799316in, y=2.699771in, left, base]{\color{textcolor}{\sffamily\fontsize{10.000000}{12.000000}\selectfont\catcode`\^=\active\def^{\ifmmode\sp\else\^{}\fi}\catcode`\%=\active\def%{\%}8000}}%
\end{pgfscope}%
\begin{pgfscope}%
\pgfsetbuttcap%
\pgfsetroundjoin%
\definecolor{currentfill}{rgb}{0.000000,0.000000,0.000000}%
\pgfsetfillcolor{currentfill}%
\pgfsetlinewidth{0.803000pt}%
\definecolor{currentstroke}{rgb}{0.000000,0.000000,0.000000}%
\pgfsetstrokecolor{currentstroke}%
\pgfsetdash{}{0pt}%
\pgfsys@defobject{currentmarker}{\pgfqpoint{-0.048611in}{0.000000in}}{\pgfqpoint{-0.000000in}{0.000000in}}{%
\pgfpathmoveto{\pgfqpoint{-0.000000in}{0.000000in}}%
\pgfpathlineto{\pgfqpoint{-0.048611in}{0.000000in}}%
\pgfusepath{stroke,fill}%
}%
\begin{pgfscope}%
\pgfsys@transformshift{1.250000in}{3.303166in}%
\pgfsys@useobject{currentmarker}{}%
\end{pgfscope}%
\end{pgfscope}%
\begin{pgfscope}%
\definecolor{textcolor}{rgb}{0.000000,0.000000,0.000000}%
\pgfsetstrokecolor{textcolor}%
\pgfsetfillcolor{textcolor}%
\pgftext[x=0.710951in, y=3.250405in, left, base]{\color{textcolor}{\sffamily\fontsize{10.000000}{12.000000}\selectfont\catcode`\^=\active\def^{\ifmmode\sp\else\^{}\fi}\catcode`\%=\active\def%{\%}10000}}%
\end{pgfscope}%
\begin{pgfscope}%
\pgfsetbuttcap%
\pgfsetroundjoin%
\definecolor{currentfill}{rgb}{0.000000,0.000000,0.000000}%
\pgfsetfillcolor{currentfill}%
\pgfsetlinewidth{0.803000pt}%
\definecolor{currentstroke}{rgb}{0.000000,0.000000,0.000000}%
\pgfsetstrokecolor{currentstroke}%
\pgfsetdash{}{0pt}%
\pgfsys@defobject{currentmarker}{\pgfqpoint{-0.048611in}{0.000000in}}{\pgfqpoint{-0.000000in}{0.000000in}}{%
\pgfpathmoveto{\pgfqpoint{-0.000000in}{0.000000in}}%
\pgfpathlineto{\pgfqpoint{-0.048611in}{0.000000in}}%
\pgfusepath{stroke,fill}%
}%
\begin{pgfscope}%
\pgfsys@transformshift{1.250000in}{3.853799in}%
\pgfsys@useobject{currentmarker}{}%
\end{pgfscope}%
\end{pgfscope}%
\begin{pgfscope}%
\definecolor{textcolor}{rgb}{0.000000,0.000000,0.000000}%
\pgfsetstrokecolor{textcolor}%
\pgfsetfillcolor{textcolor}%
\pgftext[x=0.710951in, y=3.801038in, left, base]{\color{textcolor}{\sffamily\fontsize{10.000000}{12.000000}\selectfont\catcode`\^=\active\def^{\ifmmode\sp\else\^{}\fi}\catcode`\%=\active\def%{\%}12000}}%
\end{pgfscope}%
\begin{pgfscope}%
\definecolor{textcolor}{rgb}{0.000000,0.000000,0.000000}%
\pgfsetstrokecolor{textcolor}%
\pgfsetfillcolor{textcolor}%
\pgftext[x=0.655396in,y=2.475000in,,bottom,rotate=90.000000]{\color{textcolor}{\sffamily\fontsize{10.000000}{12.000000}\selectfont\catcode`\^=\active\def^{\ifmmode\sp\else\^{}\fi}\catcode`\%=\active\def%{\%}Frecuencia}}%
\end{pgfscope}%
\begin{pgfscope}%
\pgfsetrectcap%
\pgfsetmiterjoin%
\pgfsetlinewidth{0.803000pt}%
\definecolor{currentstroke}{rgb}{0.000000,0.000000,0.000000}%
\pgfsetstrokecolor{currentstroke}%
\pgfsetdash{}{0pt}%
\pgfpathmoveto{\pgfqpoint{1.250000in}{0.550000in}}%
\pgfpathlineto{\pgfqpoint{1.250000in}{4.400000in}}%
\pgfusepath{stroke}%
\end{pgfscope}%
\begin{pgfscope}%
\pgfsetrectcap%
\pgfsetmiterjoin%
\pgfsetlinewidth{0.803000pt}%
\definecolor{currentstroke}{rgb}{0.000000,0.000000,0.000000}%
\pgfsetstrokecolor{currentstroke}%
\pgfsetdash{}{0pt}%
\pgfpathmoveto{\pgfqpoint{9.000000in}{0.550000in}}%
\pgfpathlineto{\pgfqpoint{9.000000in}{4.400000in}}%
\pgfusepath{stroke}%
\end{pgfscope}%
\begin{pgfscope}%
\pgfsetrectcap%
\pgfsetmiterjoin%
\pgfsetlinewidth{0.803000pt}%
\definecolor{currentstroke}{rgb}{0.000000,0.000000,0.000000}%
\pgfsetstrokecolor{currentstroke}%
\pgfsetdash{}{0pt}%
\pgfpathmoveto{\pgfqpoint{1.250000in}{0.550000in}}%
\pgfpathlineto{\pgfqpoint{9.000000in}{0.550000in}}%
\pgfusepath{stroke}%
\end{pgfscope}%
\begin{pgfscope}%
\pgfsetrectcap%
\pgfsetmiterjoin%
\pgfsetlinewidth{0.803000pt}%
\definecolor{currentstroke}{rgb}{0.000000,0.000000,0.000000}%
\pgfsetstrokecolor{currentstroke}%
\pgfsetdash{}{0pt}%
\pgfpathmoveto{\pgfqpoint{1.250000in}{4.400000in}}%
\pgfpathlineto{\pgfqpoint{9.000000in}{4.400000in}}%
\pgfusepath{stroke}%
\end{pgfscope}%
\begin{pgfscope}%
\definecolor{textcolor}{rgb}{0.000000,0.000000,0.000000}%
\pgfsetstrokecolor{textcolor}%
\pgfsetfillcolor{textcolor}%
\pgftext[x=5.125000in,y=4.483333in,,base]{\color{textcolor}{\sffamily\fontsize{12.000000}{14.400000}\selectfont\catcode`\^=\active\def^{\ifmmode\sp\else\^{}\fi}\catcode`\%=\active\def%{\%}Histograma de Original}}%
\end{pgfscope}%
\end{pgfpicture}%
\makeatother%
\endgroup%
}
		\caption{Original}
		\label{fig:original_1}
	\end{subfigure}
	\hfill
	\begin{subfigure}[b]{0.45\textwidth}
		\resizebox{\linewidth}{!}{%% Creator: Matplotlib, PGF backend
%%
%% To include the figure in your LaTeX document, write
%%   \input{<filename>.pgf}
%%
%% Make sure the required packages are loaded in your preamble
%%   \usepackage{pgf}
%%
%% Also ensure that all the required font packages are loaded; for instance,
%% the lmodern package is sometimes necessary when using math font.
%%   \usepackage{lmodern}
%%
%% Figures using additional raster images can only be included by \input if
%% they are in the same directory as the main LaTeX file. For loading figures
%% from other directories you can use the `import` package
%%   \usepackage{import}
%%
%% and then include the figures with
%%   \import{<path to file>}{<filename>.pgf}
%%
%% Matplotlib used the following preamble
%%   \def\mathdefault#1{#1}
%%   \everymath=\expandafter{\the\everymath\displaystyle}
%%   \IfFileExists{scrextend.sty}{
%%     \usepackage[fontsize=10.000000pt]{scrextend}
%%   }{
%%     \renewcommand{\normalsize}{\fontsize{10.000000}{12.000000}\selectfont}
%%     \normalsize
%%   }
%%   
%%   \ifdefined\pdftexversion\else  % non-pdftex case.
%%     \usepackage{fontspec}
%%     \setmainfont{DejaVuSerif.ttf}[Path=\detokenize{/home/elias/.virtualenvs/src-pway/lib/python3.13/site-packages/matplotlib/mpl-data/fonts/ttf/}]
%%     \setsansfont{DejaVuSans.ttf}[Path=\detokenize{/home/elias/.virtualenvs/src-pway/lib/python3.13/site-packages/matplotlib/mpl-data/fonts/ttf/}]
%%     \setmonofont{DejaVuSansMono.ttf}[Path=\detokenize{/home/elias/.virtualenvs/src-pway/lib/python3.13/site-packages/matplotlib/mpl-data/fonts/ttf/}]
%%   \fi
%%   \makeatletter\@ifpackageloaded{underscore}{}{\usepackage[strings]{underscore}}\makeatother
%%
\begingroup%
\makeatletter%
\begin{pgfpicture}%
\pgfpathrectangle{\pgfpointorigin}{\pgfqpoint{10.000000in}{5.000000in}}%
\pgfusepath{use as bounding box, clip}%
\begin{pgfscope}%
\pgfsetbuttcap%
\pgfsetmiterjoin%
\definecolor{currentfill}{rgb}{1.000000,1.000000,1.000000}%
\pgfsetfillcolor{currentfill}%
\pgfsetlinewidth{0.000000pt}%
\definecolor{currentstroke}{rgb}{1.000000,1.000000,1.000000}%
\pgfsetstrokecolor{currentstroke}%
\pgfsetdash{}{0pt}%
\pgfpathmoveto{\pgfqpoint{0.000000in}{0.000000in}}%
\pgfpathlineto{\pgfqpoint{10.000000in}{0.000000in}}%
\pgfpathlineto{\pgfqpoint{10.000000in}{5.000000in}}%
\pgfpathlineto{\pgfqpoint{0.000000in}{5.000000in}}%
\pgfpathlineto{\pgfqpoint{0.000000in}{0.000000in}}%
\pgfpathclose%
\pgfusepath{fill}%
\end{pgfscope}%
\begin{pgfscope}%
\pgfsetbuttcap%
\pgfsetmiterjoin%
\definecolor{currentfill}{rgb}{1.000000,1.000000,1.000000}%
\pgfsetfillcolor{currentfill}%
\pgfsetlinewidth{0.000000pt}%
\definecolor{currentstroke}{rgb}{0.000000,0.000000,0.000000}%
\pgfsetstrokecolor{currentstroke}%
\pgfsetstrokeopacity{0.000000}%
\pgfsetdash{}{0pt}%
\pgfpathmoveto{\pgfqpoint{1.250000in}{0.550000in}}%
\pgfpathlineto{\pgfqpoint{9.000000in}{0.550000in}}%
\pgfpathlineto{\pgfqpoint{9.000000in}{4.400000in}}%
\pgfpathlineto{\pgfqpoint{1.250000in}{4.400000in}}%
\pgfpathlineto{\pgfqpoint{1.250000in}{0.550000in}}%
\pgfpathclose%
\pgfusepath{fill}%
\end{pgfscope}%
\begin{pgfscope}%
\pgfpathrectangle{\pgfqpoint{1.250000in}{0.550000in}}{\pgfqpoint{7.750000in}{3.850000in}}%
\pgfusepath{clip}%
\pgfsetbuttcap%
\pgfsetmiterjoin%
\definecolor{currentfill}{rgb}{0.501961,0.501961,0.501961}%
\pgfsetfillcolor{currentfill}%
\pgfsetfillopacity{0.700000}%
\pgfsetlinewidth{0.000000pt}%
\definecolor{currentstroke}{rgb}{0.000000,0.000000,0.000000}%
\pgfsetstrokecolor{currentstroke}%
\pgfsetstrokeopacity{0.700000}%
\pgfsetdash{}{0pt}%
\pgfpathmoveto{\pgfqpoint{1.602273in}{0.550000in}}%
\pgfpathlineto{\pgfqpoint{1.629794in}{0.550000in}}%
\pgfpathlineto{\pgfqpoint{1.629794in}{1.041835in}}%
\pgfpathlineto{\pgfqpoint{1.602273in}{1.041835in}}%
\pgfpathlineto{\pgfqpoint{1.602273in}{0.550000in}}%
\pgfpathclose%
\pgfusepath{fill}%
\end{pgfscope}%
\begin{pgfscope}%
\pgfpathrectangle{\pgfqpoint{1.250000in}{0.550000in}}{\pgfqpoint{7.750000in}{3.850000in}}%
\pgfusepath{clip}%
\pgfsetbuttcap%
\pgfsetmiterjoin%
\definecolor{currentfill}{rgb}{0.501961,0.501961,0.501961}%
\pgfsetfillcolor{currentfill}%
\pgfsetfillopacity{0.700000}%
\pgfsetlinewidth{0.000000pt}%
\definecolor{currentstroke}{rgb}{0.000000,0.000000,0.000000}%
\pgfsetstrokecolor{currentstroke}%
\pgfsetstrokeopacity{0.700000}%
\pgfsetdash{}{0pt}%
\pgfpathmoveto{\pgfqpoint{1.629794in}{0.550000in}}%
\pgfpathlineto{\pgfqpoint{1.657315in}{0.550000in}}%
\pgfpathlineto{\pgfqpoint{1.657315in}{2.069699in}}%
\pgfpathlineto{\pgfqpoint{1.629794in}{2.069699in}}%
\pgfpathlineto{\pgfqpoint{1.629794in}{0.550000in}}%
\pgfpathclose%
\pgfusepath{fill}%
\end{pgfscope}%
\begin{pgfscope}%
\pgfpathrectangle{\pgfqpoint{1.250000in}{0.550000in}}{\pgfqpoint{7.750000in}{3.850000in}}%
\pgfusepath{clip}%
\pgfsetbuttcap%
\pgfsetmiterjoin%
\definecolor{currentfill}{rgb}{0.501961,0.501961,0.501961}%
\pgfsetfillcolor{currentfill}%
\pgfsetfillopacity{0.700000}%
\pgfsetlinewidth{0.000000pt}%
\definecolor{currentstroke}{rgb}{0.000000,0.000000,0.000000}%
\pgfsetstrokecolor{currentstroke}%
\pgfsetstrokeopacity{0.700000}%
\pgfsetdash{}{0pt}%
\pgfpathmoveto{\pgfqpoint{1.657315in}{0.550000in}}%
\pgfpathlineto{\pgfqpoint{1.684837in}{0.550000in}}%
\pgfpathlineto{\pgfqpoint{1.684837in}{1.567885in}}%
\pgfpathlineto{\pgfqpoint{1.657315in}{1.567885in}}%
\pgfpathlineto{\pgfqpoint{1.657315in}{0.550000in}}%
\pgfpathclose%
\pgfusepath{fill}%
\end{pgfscope}%
\begin{pgfscope}%
\pgfpathrectangle{\pgfqpoint{1.250000in}{0.550000in}}{\pgfqpoint{7.750000in}{3.850000in}}%
\pgfusepath{clip}%
\pgfsetbuttcap%
\pgfsetmiterjoin%
\definecolor{currentfill}{rgb}{0.501961,0.501961,0.501961}%
\pgfsetfillcolor{currentfill}%
\pgfsetfillopacity{0.700000}%
\pgfsetlinewidth{0.000000pt}%
\definecolor{currentstroke}{rgb}{0.000000,0.000000,0.000000}%
\pgfsetstrokecolor{currentstroke}%
\pgfsetstrokeopacity{0.700000}%
\pgfsetdash{}{0pt}%
\pgfpathmoveto{\pgfqpoint{1.684837in}{0.550000in}}%
\pgfpathlineto{\pgfqpoint{1.712358in}{0.550000in}}%
\pgfpathlineto{\pgfqpoint{1.712358in}{1.999844in}}%
\pgfpathlineto{\pgfqpoint{1.684837in}{1.999844in}}%
\pgfpathlineto{\pgfqpoint{1.684837in}{0.550000in}}%
\pgfpathclose%
\pgfusepath{fill}%
\end{pgfscope}%
\begin{pgfscope}%
\pgfpathrectangle{\pgfqpoint{1.250000in}{0.550000in}}{\pgfqpoint{7.750000in}{3.850000in}}%
\pgfusepath{clip}%
\pgfsetbuttcap%
\pgfsetmiterjoin%
\definecolor{currentfill}{rgb}{0.501961,0.501961,0.501961}%
\pgfsetfillcolor{currentfill}%
\pgfsetfillopacity{0.700000}%
\pgfsetlinewidth{0.000000pt}%
\definecolor{currentstroke}{rgb}{0.000000,0.000000,0.000000}%
\pgfsetstrokecolor{currentstroke}%
\pgfsetstrokeopacity{0.700000}%
\pgfsetdash{}{0pt}%
\pgfpathmoveto{\pgfqpoint{1.712358in}{0.550000in}}%
\pgfpathlineto{\pgfqpoint{1.739879in}{0.550000in}}%
\pgfpathlineto{\pgfqpoint{1.739879in}{2.326309in}}%
\pgfpathlineto{\pgfqpoint{1.712358in}{2.326309in}}%
\pgfpathlineto{\pgfqpoint{1.712358in}{0.550000in}}%
\pgfpathclose%
\pgfusepath{fill}%
\end{pgfscope}%
\begin{pgfscope}%
\pgfpathrectangle{\pgfqpoint{1.250000in}{0.550000in}}{\pgfqpoint{7.750000in}{3.850000in}}%
\pgfusepath{clip}%
\pgfsetbuttcap%
\pgfsetmiterjoin%
\definecolor{currentfill}{rgb}{0.501961,0.501961,0.501961}%
\pgfsetfillcolor{currentfill}%
\pgfsetfillopacity{0.700000}%
\pgfsetlinewidth{0.000000pt}%
\definecolor{currentstroke}{rgb}{0.000000,0.000000,0.000000}%
\pgfsetstrokecolor{currentstroke}%
\pgfsetstrokeopacity{0.700000}%
\pgfsetdash{}{0pt}%
\pgfpathmoveto{\pgfqpoint{1.739879in}{0.550000in}}%
\pgfpathlineto{\pgfqpoint{1.767401in}{0.550000in}}%
\pgfpathlineto{\pgfqpoint{1.767401in}{2.290669in}}%
\pgfpathlineto{\pgfqpoint{1.739879in}{2.290669in}}%
\pgfpathlineto{\pgfqpoint{1.739879in}{0.550000in}}%
\pgfpathclose%
\pgfusepath{fill}%
\end{pgfscope}%
\begin{pgfscope}%
\pgfpathrectangle{\pgfqpoint{1.250000in}{0.550000in}}{\pgfqpoint{7.750000in}{3.850000in}}%
\pgfusepath{clip}%
\pgfsetbuttcap%
\pgfsetmiterjoin%
\definecolor{currentfill}{rgb}{0.501961,0.501961,0.501961}%
\pgfsetfillcolor{currentfill}%
\pgfsetfillopacity{0.700000}%
\pgfsetlinewidth{0.000000pt}%
\definecolor{currentstroke}{rgb}{0.000000,0.000000,0.000000}%
\pgfsetstrokecolor{currentstroke}%
\pgfsetstrokeopacity{0.700000}%
\pgfsetdash{}{0pt}%
\pgfpathmoveto{\pgfqpoint{1.767401in}{0.550000in}}%
\pgfpathlineto{\pgfqpoint{1.794922in}{0.550000in}}%
\pgfpathlineto{\pgfqpoint{1.794922in}{2.237921in}}%
\pgfpathlineto{\pgfqpoint{1.767401in}{2.237921in}}%
\pgfpathlineto{\pgfqpoint{1.767401in}{0.550000in}}%
\pgfpathclose%
\pgfusepath{fill}%
\end{pgfscope}%
\begin{pgfscope}%
\pgfpathrectangle{\pgfqpoint{1.250000in}{0.550000in}}{\pgfqpoint{7.750000in}{3.850000in}}%
\pgfusepath{clip}%
\pgfsetbuttcap%
\pgfsetmiterjoin%
\definecolor{currentfill}{rgb}{0.501961,0.501961,0.501961}%
\pgfsetfillcolor{currentfill}%
\pgfsetfillopacity{0.700000}%
\pgfsetlinewidth{0.000000pt}%
\definecolor{currentstroke}{rgb}{0.000000,0.000000,0.000000}%
\pgfsetstrokecolor{currentstroke}%
\pgfsetstrokeopacity{0.700000}%
\pgfsetdash{}{0pt}%
\pgfpathmoveto{\pgfqpoint{1.794922in}{0.550000in}}%
\pgfpathlineto{\pgfqpoint{1.822443in}{0.550000in}}%
\pgfpathlineto{\pgfqpoint{1.822443in}{2.269285in}}%
\pgfpathlineto{\pgfqpoint{1.794922in}{2.269285in}}%
\pgfpathlineto{\pgfqpoint{1.794922in}{0.550000in}}%
\pgfpathclose%
\pgfusepath{fill}%
\end{pgfscope}%
\begin{pgfscope}%
\pgfpathrectangle{\pgfqpoint{1.250000in}{0.550000in}}{\pgfqpoint{7.750000in}{3.850000in}}%
\pgfusepath{clip}%
\pgfsetbuttcap%
\pgfsetmiterjoin%
\definecolor{currentfill}{rgb}{0.501961,0.501961,0.501961}%
\pgfsetfillcolor{currentfill}%
\pgfsetfillopacity{0.700000}%
\pgfsetlinewidth{0.000000pt}%
\definecolor{currentstroke}{rgb}{0.000000,0.000000,0.000000}%
\pgfsetstrokecolor{currentstroke}%
\pgfsetstrokeopacity{0.700000}%
\pgfsetdash{}{0pt}%
\pgfpathmoveto{\pgfqpoint{1.822443in}{0.550000in}}%
\pgfpathlineto{\pgfqpoint{1.849964in}{0.550000in}}%
\pgfpathlineto{\pgfqpoint{1.849964in}{2.058294in}}%
\pgfpathlineto{\pgfqpoint{1.822443in}{2.058294in}}%
\pgfpathlineto{\pgfqpoint{1.822443in}{0.550000in}}%
\pgfpathclose%
\pgfusepath{fill}%
\end{pgfscope}%
\begin{pgfscope}%
\pgfpathrectangle{\pgfqpoint{1.250000in}{0.550000in}}{\pgfqpoint{7.750000in}{3.850000in}}%
\pgfusepath{clip}%
\pgfsetbuttcap%
\pgfsetmiterjoin%
\definecolor{currentfill}{rgb}{0.501961,0.501961,0.501961}%
\pgfsetfillcolor{currentfill}%
\pgfsetfillopacity{0.700000}%
\pgfsetlinewidth{0.000000pt}%
\definecolor{currentstroke}{rgb}{0.000000,0.000000,0.000000}%
\pgfsetstrokecolor{currentstroke}%
\pgfsetstrokeopacity{0.700000}%
\pgfsetdash{}{0pt}%
\pgfpathmoveto{\pgfqpoint{1.849964in}{0.550000in}}%
\pgfpathlineto{\pgfqpoint{1.877486in}{0.550000in}}%
\pgfpathlineto{\pgfqpoint{1.877486in}{2.049741in}}%
\pgfpathlineto{\pgfqpoint{1.849964in}{2.049741in}}%
\pgfpathlineto{\pgfqpoint{1.849964in}{0.550000in}}%
\pgfpathclose%
\pgfusepath{fill}%
\end{pgfscope}%
\begin{pgfscope}%
\pgfpathrectangle{\pgfqpoint{1.250000in}{0.550000in}}{\pgfqpoint{7.750000in}{3.850000in}}%
\pgfusepath{clip}%
\pgfsetbuttcap%
\pgfsetmiterjoin%
\definecolor{currentfill}{rgb}{0.501961,0.501961,0.501961}%
\pgfsetfillcolor{currentfill}%
\pgfsetfillopacity{0.700000}%
\pgfsetlinewidth{0.000000pt}%
\definecolor{currentstroke}{rgb}{0.000000,0.000000,0.000000}%
\pgfsetstrokecolor{currentstroke}%
\pgfsetstrokeopacity{0.700000}%
\pgfsetdash{}{0pt}%
\pgfpathmoveto{\pgfqpoint{1.877486in}{0.550000in}}%
\pgfpathlineto{\pgfqpoint{1.905007in}{0.550000in}}%
\pgfpathlineto{\pgfqpoint{1.905007in}{2.001270in}}%
\pgfpathlineto{\pgfqpoint{1.877486in}{2.001270in}}%
\pgfpathlineto{\pgfqpoint{1.877486in}{0.550000in}}%
\pgfpathclose%
\pgfusepath{fill}%
\end{pgfscope}%
\begin{pgfscope}%
\pgfpathrectangle{\pgfqpoint{1.250000in}{0.550000in}}{\pgfqpoint{7.750000in}{3.850000in}}%
\pgfusepath{clip}%
\pgfsetbuttcap%
\pgfsetmiterjoin%
\definecolor{currentfill}{rgb}{0.501961,0.501961,0.501961}%
\pgfsetfillcolor{currentfill}%
\pgfsetfillopacity{0.700000}%
\pgfsetlinewidth{0.000000pt}%
\definecolor{currentstroke}{rgb}{0.000000,0.000000,0.000000}%
\pgfsetstrokecolor{currentstroke}%
\pgfsetstrokeopacity{0.700000}%
\pgfsetdash{}{0pt}%
\pgfpathmoveto{\pgfqpoint{1.905007in}{0.550000in}}%
\pgfpathlineto{\pgfqpoint{1.932528in}{0.550000in}}%
\pgfpathlineto{\pgfqpoint{1.932528in}{2.109616in}}%
\pgfpathlineto{\pgfqpoint{1.905007in}{2.109616in}}%
\pgfpathlineto{\pgfqpoint{1.905007in}{0.550000in}}%
\pgfpathclose%
\pgfusepath{fill}%
\end{pgfscope}%
\begin{pgfscope}%
\pgfpathrectangle{\pgfqpoint{1.250000in}{0.550000in}}{\pgfqpoint{7.750000in}{3.850000in}}%
\pgfusepath{clip}%
\pgfsetbuttcap%
\pgfsetmiterjoin%
\definecolor{currentfill}{rgb}{0.501961,0.501961,0.501961}%
\pgfsetfillcolor{currentfill}%
\pgfsetfillopacity{0.700000}%
\pgfsetlinewidth{0.000000pt}%
\definecolor{currentstroke}{rgb}{0.000000,0.000000,0.000000}%
\pgfsetstrokecolor{currentstroke}%
\pgfsetstrokeopacity{0.700000}%
\pgfsetdash{}{0pt}%
\pgfpathmoveto{\pgfqpoint{1.932528in}{0.550000in}}%
\pgfpathlineto{\pgfqpoint{1.960050in}{0.550000in}}%
\pgfpathlineto{\pgfqpoint{1.960050in}{2.207983in}}%
\pgfpathlineto{\pgfqpoint{1.932528in}{2.207983in}}%
\pgfpathlineto{\pgfqpoint{1.932528in}{0.550000in}}%
\pgfpathclose%
\pgfusepath{fill}%
\end{pgfscope}%
\begin{pgfscope}%
\pgfpathrectangle{\pgfqpoint{1.250000in}{0.550000in}}{\pgfqpoint{7.750000in}{3.850000in}}%
\pgfusepath{clip}%
\pgfsetbuttcap%
\pgfsetmiterjoin%
\definecolor{currentfill}{rgb}{0.501961,0.501961,0.501961}%
\pgfsetfillcolor{currentfill}%
\pgfsetfillopacity{0.700000}%
\pgfsetlinewidth{0.000000pt}%
\definecolor{currentstroke}{rgb}{0.000000,0.000000,0.000000}%
\pgfsetstrokecolor{currentstroke}%
\pgfsetstrokeopacity{0.700000}%
\pgfsetdash{}{0pt}%
\pgfpathmoveto{\pgfqpoint{1.960050in}{0.550000in}}%
\pgfpathlineto{\pgfqpoint{1.987571in}{0.550000in}}%
\pgfpathlineto{\pgfqpoint{1.987571in}{2.302074in}}%
\pgfpathlineto{\pgfqpoint{1.960050in}{2.302074in}}%
\pgfpathlineto{\pgfqpoint{1.960050in}{0.550000in}}%
\pgfpathclose%
\pgfusepath{fill}%
\end{pgfscope}%
\begin{pgfscope}%
\pgfpathrectangle{\pgfqpoint{1.250000in}{0.550000in}}{\pgfqpoint{7.750000in}{3.850000in}}%
\pgfusepath{clip}%
\pgfsetbuttcap%
\pgfsetmiterjoin%
\definecolor{currentfill}{rgb}{0.501961,0.501961,0.501961}%
\pgfsetfillcolor{currentfill}%
\pgfsetfillopacity{0.700000}%
\pgfsetlinewidth{0.000000pt}%
\definecolor{currentstroke}{rgb}{0.000000,0.000000,0.000000}%
\pgfsetstrokecolor{currentstroke}%
\pgfsetstrokeopacity{0.700000}%
\pgfsetdash{}{0pt}%
\pgfpathmoveto{\pgfqpoint{1.987571in}{0.550000in}}%
\pgfpathlineto{\pgfqpoint{2.015092in}{0.550000in}}%
\pgfpathlineto{\pgfqpoint{2.015092in}{2.145257in}}%
\pgfpathlineto{\pgfqpoint{1.987571in}{2.145257in}}%
\pgfpathlineto{\pgfqpoint{1.987571in}{0.550000in}}%
\pgfpathclose%
\pgfusepath{fill}%
\end{pgfscope}%
\begin{pgfscope}%
\pgfpathrectangle{\pgfqpoint{1.250000in}{0.550000in}}{\pgfqpoint{7.750000in}{3.850000in}}%
\pgfusepath{clip}%
\pgfsetbuttcap%
\pgfsetmiterjoin%
\definecolor{currentfill}{rgb}{0.501961,0.501961,0.501961}%
\pgfsetfillcolor{currentfill}%
\pgfsetfillopacity{0.700000}%
\pgfsetlinewidth{0.000000pt}%
\definecolor{currentstroke}{rgb}{0.000000,0.000000,0.000000}%
\pgfsetstrokecolor{currentstroke}%
\pgfsetstrokeopacity{0.700000}%
\pgfsetdash{}{0pt}%
\pgfpathmoveto{\pgfqpoint{2.015092in}{0.550000in}}%
\pgfpathlineto{\pgfqpoint{2.042614in}{0.550000in}}%
\pgfpathlineto{\pgfqpoint{2.042614in}{2.188025in}}%
\pgfpathlineto{\pgfqpoint{2.015092in}{2.188025in}}%
\pgfpathlineto{\pgfqpoint{2.015092in}{0.550000in}}%
\pgfpathclose%
\pgfusepath{fill}%
\end{pgfscope}%
\begin{pgfscope}%
\pgfpathrectangle{\pgfqpoint{1.250000in}{0.550000in}}{\pgfqpoint{7.750000in}{3.850000in}}%
\pgfusepath{clip}%
\pgfsetbuttcap%
\pgfsetmiterjoin%
\definecolor{currentfill}{rgb}{0.501961,0.501961,0.501961}%
\pgfsetfillcolor{currentfill}%
\pgfsetfillopacity{0.700000}%
\pgfsetlinewidth{0.000000pt}%
\definecolor{currentstroke}{rgb}{0.000000,0.000000,0.000000}%
\pgfsetstrokecolor{currentstroke}%
\pgfsetstrokeopacity{0.700000}%
\pgfsetdash{}{0pt}%
\pgfpathmoveto{\pgfqpoint{2.042614in}{0.550000in}}%
\pgfpathlineto{\pgfqpoint{2.070135in}{0.550000in}}%
\pgfpathlineto{\pgfqpoint{2.070135in}{2.222240in}}%
\pgfpathlineto{\pgfqpoint{2.042614in}{2.222240in}}%
\pgfpathlineto{\pgfqpoint{2.042614in}{0.550000in}}%
\pgfpathclose%
\pgfusepath{fill}%
\end{pgfscope}%
\begin{pgfscope}%
\pgfpathrectangle{\pgfqpoint{1.250000in}{0.550000in}}{\pgfqpoint{7.750000in}{3.850000in}}%
\pgfusepath{clip}%
\pgfsetbuttcap%
\pgfsetmiterjoin%
\definecolor{currentfill}{rgb}{0.501961,0.501961,0.501961}%
\pgfsetfillcolor{currentfill}%
\pgfsetfillopacity{0.700000}%
\pgfsetlinewidth{0.000000pt}%
\definecolor{currentstroke}{rgb}{0.000000,0.000000,0.000000}%
\pgfsetstrokecolor{currentstroke}%
\pgfsetstrokeopacity{0.700000}%
\pgfsetdash{}{0pt}%
\pgfpathmoveto{\pgfqpoint{2.070135in}{0.550000in}}%
\pgfpathlineto{\pgfqpoint{2.097656in}{0.550000in}}%
\pgfpathlineto{\pgfqpoint{2.097656in}{2.168066in}}%
\pgfpathlineto{\pgfqpoint{2.070135in}{2.168066in}}%
\pgfpathlineto{\pgfqpoint{2.070135in}{0.550000in}}%
\pgfpathclose%
\pgfusepath{fill}%
\end{pgfscope}%
\begin{pgfscope}%
\pgfpathrectangle{\pgfqpoint{1.250000in}{0.550000in}}{\pgfqpoint{7.750000in}{3.850000in}}%
\pgfusepath{clip}%
\pgfsetbuttcap%
\pgfsetmiterjoin%
\definecolor{currentfill}{rgb}{0.501961,0.501961,0.501961}%
\pgfsetfillcolor{currentfill}%
\pgfsetfillopacity{0.700000}%
\pgfsetlinewidth{0.000000pt}%
\definecolor{currentstroke}{rgb}{0.000000,0.000000,0.000000}%
\pgfsetstrokecolor{currentstroke}%
\pgfsetstrokeopacity{0.700000}%
\pgfsetdash{}{0pt}%
\pgfpathmoveto{\pgfqpoint{2.097656in}{0.550000in}}%
\pgfpathlineto{\pgfqpoint{2.125178in}{0.550000in}}%
\pgfpathlineto{\pgfqpoint{2.125178in}{2.132426in}}%
\pgfpathlineto{\pgfqpoint{2.097656in}{2.132426in}}%
\pgfpathlineto{\pgfqpoint{2.097656in}{0.550000in}}%
\pgfpathclose%
\pgfusepath{fill}%
\end{pgfscope}%
\begin{pgfscope}%
\pgfpathrectangle{\pgfqpoint{1.250000in}{0.550000in}}{\pgfqpoint{7.750000in}{3.850000in}}%
\pgfusepath{clip}%
\pgfsetbuttcap%
\pgfsetmiterjoin%
\definecolor{currentfill}{rgb}{0.501961,0.501961,0.501961}%
\pgfsetfillcolor{currentfill}%
\pgfsetfillopacity{0.700000}%
\pgfsetlinewidth{0.000000pt}%
\definecolor{currentstroke}{rgb}{0.000000,0.000000,0.000000}%
\pgfsetstrokecolor{currentstroke}%
\pgfsetstrokeopacity{0.700000}%
\pgfsetdash{}{0pt}%
\pgfpathmoveto{\pgfqpoint{2.125178in}{0.550000in}}%
\pgfpathlineto{\pgfqpoint{2.152699in}{0.550000in}}%
\pgfpathlineto{\pgfqpoint{2.152699in}{2.276413in}}%
\pgfpathlineto{\pgfqpoint{2.125178in}{2.276413in}}%
\pgfpathlineto{\pgfqpoint{2.125178in}{0.550000in}}%
\pgfpathclose%
\pgfusepath{fill}%
\end{pgfscope}%
\begin{pgfscope}%
\pgfpathrectangle{\pgfqpoint{1.250000in}{0.550000in}}{\pgfqpoint{7.750000in}{3.850000in}}%
\pgfusepath{clip}%
\pgfsetbuttcap%
\pgfsetmiterjoin%
\definecolor{currentfill}{rgb}{0.501961,0.501961,0.501961}%
\pgfsetfillcolor{currentfill}%
\pgfsetfillopacity{0.700000}%
\pgfsetlinewidth{0.000000pt}%
\definecolor{currentstroke}{rgb}{0.000000,0.000000,0.000000}%
\pgfsetstrokecolor{currentstroke}%
\pgfsetstrokeopacity{0.700000}%
\pgfsetdash{}{0pt}%
\pgfpathmoveto{\pgfqpoint{2.152699in}{0.550000in}}%
\pgfpathlineto{\pgfqpoint{2.180220in}{0.550000in}}%
\pgfpathlineto{\pgfqpoint{2.180220in}{2.206558in}}%
\pgfpathlineto{\pgfqpoint{2.152699in}{2.206558in}}%
\pgfpathlineto{\pgfqpoint{2.152699in}{0.550000in}}%
\pgfpathclose%
\pgfusepath{fill}%
\end{pgfscope}%
\begin{pgfscope}%
\pgfpathrectangle{\pgfqpoint{1.250000in}{0.550000in}}{\pgfqpoint{7.750000in}{3.850000in}}%
\pgfusepath{clip}%
\pgfsetbuttcap%
\pgfsetmiterjoin%
\definecolor{currentfill}{rgb}{0.501961,0.501961,0.501961}%
\pgfsetfillcolor{currentfill}%
\pgfsetfillopacity{0.700000}%
\pgfsetlinewidth{0.000000pt}%
\definecolor{currentstroke}{rgb}{0.000000,0.000000,0.000000}%
\pgfsetstrokecolor{currentstroke}%
\pgfsetstrokeopacity{0.700000}%
\pgfsetdash{}{0pt}%
\pgfpathmoveto{\pgfqpoint{2.180220in}{0.550000in}}%
\pgfpathlineto{\pgfqpoint{2.207741in}{0.550000in}}%
\pgfpathlineto{\pgfqpoint{2.207741in}{2.303499in}}%
\pgfpathlineto{\pgfqpoint{2.180220in}{2.303499in}}%
\pgfpathlineto{\pgfqpoint{2.180220in}{0.550000in}}%
\pgfpathclose%
\pgfusepath{fill}%
\end{pgfscope}%
\begin{pgfscope}%
\pgfpathrectangle{\pgfqpoint{1.250000in}{0.550000in}}{\pgfqpoint{7.750000in}{3.850000in}}%
\pgfusepath{clip}%
\pgfsetbuttcap%
\pgfsetmiterjoin%
\definecolor{currentfill}{rgb}{0.501961,0.501961,0.501961}%
\pgfsetfillcolor{currentfill}%
\pgfsetfillopacity{0.700000}%
\pgfsetlinewidth{0.000000pt}%
\definecolor{currentstroke}{rgb}{0.000000,0.000000,0.000000}%
\pgfsetstrokecolor{currentstroke}%
\pgfsetstrokeopacity{0.700000}%
\pgfsetdash{}{0pt}%
\pgfpathmoveto{\pgfqpoint{2.207741in}{0.550000in}}%
\pgfpathlineto{\pgfqpoint{2.235263in}{0.550000in}}%
\pgfpathlineto{\pgfqpoint{2.235263in}{2.363375in}}%
\pgfpathlineto{\pgfqpoint{2.207741in}{2.363375in}}%
\pgfpathlineto{\pgfqpoint{2.207741in}{0.550000in}}%
\pgfpathclose%
\pgfusepath{fill}%
\end{pgfscope}%
\begin{pgfscope}%
\pgfpathrectangle{\pgfqpoint{1.250000in}{0.550000in}}{\pgfqpoint{7.750000in}{3.850000in}}%
\pgfusepath{clip}%
\pgfsetbuttcap%
\pgfsetmiterjoin%
\definecolor{currentfill}{rgb}{0.501961,0.501961,0.501961}%
\pgfsetfillcolor{currentfill}%
\pgfsetfillopacity{0.700000}%
\pgfsetlinewidth{0.000000pt}%
\definecolor{currentstroke}{rgb}{0.000000,0.000000,0.000000}%
\pgfsetstrokecolor{currentstroke}%
\pgfsetstrokeopacity{0.700000}%
\pgfsetdash{}{0pt}%
\pgfpathmoveto{\pgfqpoint{2.235263in}{0.550000in}}%
\pgfpathlineto{\pgfqpoint{2.262784in}{0.550000in}}%
\pgfpathlineto{\pgfqpoint{2.262784in}{2.363375in}}%
\pgfpathlineto{\pgfqpoint{2.235263in}{2.363375in}}%
\pgfpathlineto{\pgfqpoint{2.235263in}{0.550000in}}%
\pgfpathclose%
\pgfusepath{fill}%
\end{pgfscope}%
\begin{pgfscope}%
\pgfpathrectangle{\pgfqpoint{1.250000in}{0.550000in}}{\pgfqpoint{7.750000in}{3.850000in}}%
\pgfusepath{clip}%
\pgfsetbuttcap%
\pgfsetmiterjoin%
\definecolor{currentfill}{rgb}{0.501961,0.501961,0.501961}%
\pgfsetfillcolor{currentfill}%
\pgfsetfillopacity{0.700000}%
\pgfsetlinewidth{0.000000pt}%
\definecolor{currentstroke}{rgb}{0.000000,0.000000,0.000000}%
\pgfsetstrokecolor{currentstroke}%
\pgfsetstrokeopacity{0.700000}%
\pgfsetdash{}{0pt}%
\pgfpathmoveto{\pgfqpoint{2.262784in}{0.550000in}}%
\pgfpathlineto{\pgfqpoint{2.290305in}{0.550000in}}%
\pgfpathlineto{\pgfqpoint{2.290305in}{2.451763in}}%
\pgfpathlineto{\pgfqpoint{2.262784in}{2.451763in}}%
\pgfpathlineto{\pgfqpoint{2.262784in}{0.550000in}}%
\pgfpathclose%
\pgfusepath{fill}%
\end{pgfscope}%
\begin{pgfscope}%
\pgfpathrectangle{\pgfqpoint{1.250000in}{0.550000in}}{\pgfqpoint{7.750000in}{3.850000in}}%
\pgfusepath{clip}%
\pgfsetbuttcap%
\pgfsetmiterjoin%
\definecolor{currentfill}{rgb}{0.501961,0.501961,0.501961}%
\pgfsetfillcolor{currentfill}%
\pgfsetfillopacity{0.700000}%
\pgfsetlinewidth{0.000000pt}%
\definecolor{currentstroke}{rgb}{0.000000,0.000000,0.000000}%
\pgfsetstrokecolor{currentstroke}%
\pgfsetstrokeopacity{0.700000}%
\pgfsetdash{}{0pt}%
\pgfpathmoveto{\pgfqpoint{2.290305in}{0.550000in}}%
\pgfpathlineto{\pgfqpoint{2.317827in}{0.550000in}}%
\pgfpathlineto{\pgfqpoint{2.317827in}{2.527320in}}%
\pgfpathlineto{\pgfqpoint{2.290305in}{2.527320in}}%
\pgfpathlineto{\pgfqpoint{2.290305in}{0.550000in}}%
\pgfpathclose%
\pgfusepath{fill}%
\end{pgfscope}%
\begin{pgfscope}%
\pgfpathrectangle{\pgfqpoint{1.250000in}{0.550000in}}{\pgfqpoint{7.750000in}{3.850000in}}%
\pgfusepath{clip}%
\pgfsetbuttcap%
\pgfsetmiterjoin%
\definecolor{currentfill}{rgb}{0.501961,0.501961,0.501961}%
\pgfsetfillcolor{currentfill}%
\pgfsetfillopacity{0.700000}%
\pgfsetlinewidth{0.000000pt}%
\definecolor{currentstroke}{rgb}{0.000000,0.000000,0.000000}%
\pgfsetstrokecolor{currentstroke}%
\pgfsetstrokeopacity{0.700000}%
\pgfsetdash{}{0pt}%
\pgfpathmoveto{\pgfqpoint{2.317827in}{0.550000in}}%
\pgfpathlineto{\pgfqpoint{2.345348in}{0.550000in}}%
\pgfpathlineto{\pgfqpoint{2.345348in}{2.610005in}}%
\pgfpathlineto{\pgfqpoint{2.317827in}{2.610005in}}%
\pgfpathlineto{\pgfqpoint{2.317827in}{0.550000in}}%
\pgfpathclose%
\pgfusepath{fill}%
\end{pgfscope}%
\begin{pgfscope}%
\pgfpathrectangle{\pgfqpoint{1.250000in}{0.550000in}}{\pgfqpoint{7.750000in}{3.850000in}}%
\pgfusepath{clip}%
\pgfsetbuttcap%
\pgfsetmiterjoin%
\definecolor{currentfill}{rgb}{0.501961,0.501961,0.501961}%
\pgfsetfillcolor{currentfill}%
\pgfsetfillopacity{0.700000}%
\pgfsetlinewidth{0.000000pt}%
\definecolor{currentstroke}{rgb}{0.000000,0.000000,0.000000}%
\pgfsetstrokecolor{currentstroke}%
\pgfsetstrokeopacity{0.700000}%
\pgfsetdash{}{0pt}%
\pgfpathmoveto{\pgfqpoint{2.345348in}{0.550000in}}%
\pgfpathlineto{\pgfqpoint{2.372869in}{0.550000in}}%
\pgfpathlineto{\pgfqpoint{2.372869in}{2.493105in}}%
\pgfpathlineto{\pgfqpoint{2.345348in}{2.493105in}}%
\pgfpathlineto{\pgfqpoint{2.345348in}{0.550000in}}%
\pgfpathclose%
\pgfusepath{fill}%
\end{pgfscope}%
\begin{pgfscope}%
\pgfpathrectangle{\pgfqpoint{1.250000in}{0.550000in}}{\pgfqpoint{7.750000in}{3.850000in}}%
\pgfusepath{clip}%
\pgfsetbuttcap%
\pgfsetmiterjoin%
\definecolor{currentfill}{rgb}{0.501961,0.501961,0.501961}%
\pgfsetfillcolor{currentfill}%
\pgfsetfillopacity{0.700000}%
\pgfsetlinewidth{0.000000pt}%
\definecolor{currentstroke}{rgb}{0.000000,0.000000,0.000000}%
\pgfsetstrokecolor{currentstroke}%
\pgfsetstrokeopacity{0.700000}%
\pgfsetdash{}{0pt}%
\pgfpathmoveto{\pgfqpoint{2.372869in}{0.550000in}}%
\pgfpathlineto{\pgfqpoint{2.400391in}{0.550000in}}%
\pgfpathlineto{\pgfqpoint{2.400391in}{0.550000in}}%
\pgfpathlineto{\pgfqpoint{2.372869in}{0.550000in}}%
\pgfpathlineto{\pgfqpoint{2.372869in}{0.550000in}}%
\pgfpathclose%
\pgfusepath{fill}%
\end{pgfscope}%
\begin{pgfscope}%
\pgfpathrectangle{\pgfqpoint{1.250000in}{0.550000in}}{\pgfqpoint{7.750000in}{3.850000in}}%
\pgfusepath{clip}%
\pgfsetbuttcap%
\pgfsetmiterjoin%
\definecolor{currentfill}{rgb}{0.501961,0.501961,0.501961}%
\pgfsetfillcolor{currentfill}%
\pgfsetfillopacity{0.700000}%
\pgfsetlinewidth{0.000000pt}%
\definecolor{currentstroke}{rgb}{0.000000,0.000000,0.000000}%
\pgfsetstrokecolor{currentstroke}%
\pgfsetstrokeopacity{0.700000}%
\pgfsetdash{}{0pt}%
\pgfpathmoveto{\pgfqpoint{2.400391in}{0.550000in}}%
\pgfpathlineto{\pgfqpoint{2.427912in}{0.550000in}}%
\pgfpathlineto{\pgfqpoint{2.427912in}{2.655625in}}%
\pgfpathlineto{\pgfqpoint{2.400391in}{2.655625in}}%
\pgfpathlineto{\pgfqpoint{2.400391in}{0.550000in}}%
\pgfpathclose%
\pgfusepath{fill}%
\end{pgfscope}%
\begin{pgfscope}%
\pgfpathrectangle{\pgfqpoint{1.250000in}{0.550000in}}{\pgfqpoint{7.750000in}{3.850000in}}%
\pgfusepath{clip}%
\pgfsetbuttcap%
\pgfsetmiterjoin%
\definecolor{currentfill}{rgb}{0.501961,0.501961,0.501961}%
\pgfsetfillcolor{currentfill}%
\pgfsetfillopacity{0.700000}%
\pgfsetlinewidth{0.000000pt}%
\definecolor{currentstroke}{rgb}{0.000000,0.000000,0.000000}%
\pgfsetstrokecolor{currentstroke}%
\pgfsetstrokeopacity{0.700000}%
\pgfsetdash{}{0pt}%
\pgfpathmoveto{\pgfqpoint{2.427912in}{0.550000in}}%
\pgfpathlineto{\pgfqpoint{2.455433in}{0.550000in}}%
\pgfpathlineto{\pgfqpoint{2.455433in}{2.493105in}}%
\pgfpathlineto{\pgfqpoint{2.427912in}{2.493105in}}%
\pgfpathlineto{\pgfqpoint{2.427912in}{0.550000in}}%
\pgfpathclose%
\pgfusepath{fill}%
\end{pgfscope}%
\begin{pgfscope}%
\pgfpathrectangle{\pgfqpoint{1.250000in}{0.550000in}}{\pgfqpoint{7.750000in}{3.850000in}}%
\pgfusepath{clip}%
\pgfsetbuttcap%
\pgfsetmiterjoin%
\definecolor{currentfill}{rgb}{0.501961,0.501961,0.501961}%
\pgfsetfillcolor{currentfill}%
\pgfsetfillopacity{0.700000}%
\pgfsetlinewidth{0.000000pt}%
\definecolor{currentstroke}{rgb}{0.000000,0.000000,0.000000}%
\pgfsetstrokecolor{currentstroke}%
\pgfsetstrokeopacity{0.700000}%
\pgfsetdash{}{0pt}%
\pgfpathmoveto{\pgfqpoint{2.455433in}{0.550000in}}%
\pgfpathlineto{\pgfqpoint{2.482955in}{0.550000in}}%
\pgfpathlineto{\pgfqpoint{2.482955in}{2.574365in}}%
\pgfpathlineto{\pgfqpoint{2.455433in}{2.574365in}}%
\pgfpathlineto{\pgfqpoint{2.455433in}{0.550000in}}%
\pgfpathclose%
\pgfusepath{fill}%
\end{pgfscope}%
\begin{pgfscope}%
\pgfpathrectangle{\pgfqpoint{1.250000in}{0.550000in}}{\pgfqpoint{7.750000in}{3.850000in}}%
\pgfusepath{clip}%
\pgfsetbuttcap%
\pgfsetmiterjoin%
\definecolor{currentfill}{rgb}{0.501961,0.501961,0.501961}%
\pgfsetfillcolor{currentfill}%
\pgfsetfillopacity{0.700000}%
\pgfsetlinewidth{0.000000pt}%
\definecolor{currentstroke}{rgb}{0.000000,0.000000,0.000000}%
\pgfsetstrokecolor{currentstroke}%
\pgfsetstrokeopacity{0.700000}%
\pgfsetdash{}{0pt}%
\pgfpathmoveto{\pgfqpoint{2.482955in}{0.550000in}}%
\pgfpathlineto{\pgfqpoint{2.510476in}{0.550000in}}%
\pgfpathlineto{\pgfqpoint{2.510476in}{2.531597in}}%
\pgfpathlineto{\pgfqpoint{2.482955in}{2.531597in}}%
\pgfpathlineto{\pgfqpoint{2.482955in}{0.550000in}}%
\pgfpathclose%
\pgfusepath{fill}%
\end{pgfscope}%
\begin{pgfscope}%
\pgfpathrectangle{\pgfqpoint{1.250000in}{0.550000in}}{\pgfqpoint{7.750000in}{3.850000in}}%
\pgfusepath{clip}%
\pgfsetbuttcap%
\pgfsetmiterjoin%
\definecolor{currentfill}{rgb}{0.501961,0.501961,0.501961}%
\pgfsetfillcolor{currentfill}%
\pgfsetfillopacity{0.700000}%
\pgfsetlinewidth{0.000000pt}%
\definecolor{currentstroke}{rgb}{0.000000,0.000000,0.000000}%
\pgfsetstrokecolor{currentstroke}%
\pgfsetstrokeopacity{0.700000}%
\pgfsetdash{}{0pt}%
\pgfpathmoveto{\pgfqpoint{2.510476in}{0.550000in}}%
\pgfpathlineto{\pgfqpoint{2.537997in}{0.550000in}}%
\pgfpathlineto{\pgfqpoint{2.537997in}{2.591472in}}%
\pgfpathlineto{\pgfqpoint{2.510476in}{2.591472in}}%
\pgfpathlineto{\pgfqpoint{2.510476in}{0.550000in}}%
\pgfpathclose%
\pgfusepath{fill}%
\end{pgfscope}%
\begin{pgfscope}%
\pgfpathrectangle{\pgfqpoint{1.250000in}{0.550000in}}{\pgfqpoint{7.750000in}{3.850000in}}%
\pgfusepath{clip}%
\pgfsetbuttcap%
\pgfsetmiterjoin%
\definecolor{currentfill}{rgb}{0.501961,0.501961,0.501961}%
\pgfsetfillcolor{currentfill}%
\pgfsetfillopacity{0.700000}%
\pgfsetlinewidth{0.000000pt}%
\definecolor{currentstroke}{rgb}{0.000000,0.000000,0.000000}%
\pgfsetstrokecolor{currentstroke}%
\pgfsetstrokeopacity{0.700000}%
\pgfsetdash{}{0pt}%
\pgfpathmoveto{\pgfqpoint{2.537997in}{0.550000in}}%
\pgfpathlineto{\pgfqpoint{2.565518in}{0.550000in}}%
\pgfpathlineto{\pgfqpoint{2.565518in}{2.456039in}}%
\pgfpathlineto{\pgfqpoint{2.537997in}{2.456039in}}%
\pgfpathlineto{\pgfqpoint{2.537997in}{0.550000in}}%
\pgfpathclose%
\pgfusepath{fill}%
\end{pgfscope}%
\begin{pgfscope}%
\pgfpathrectangle{\pgfqpoint{1.250000in}{0.550000in}}{\pgfqpoint{7.750000in}{3.850000in}}%
\pgfusepath{clip}%
\pgfsetbuttcap%
\pgfsetmiterjoin%
\definecolor{currentfill}{rgb}{0.501961,0.501961,0.501961}%
\pgfsetfillcolor{currentfill}%
\pgfsetfillopacity{0.700000}%
\pgfsetlinewidth{0.000000pt}%
\definecolor{currentstroke}{rgb}{0.000000,0.000000,0.000000}%
\pgfsetstrokecolor{currentstroke}%
\pgfsetstrokeopacity{0.700000}%
\pgfsetdash{}{0pt}%
\pgfpathmoveto{\pgfqpoint{2.565518in}{0.550000in}}%
\pgfpathlineto{\pgfqpoint{2.593040in}{0.550000in}}%
\pgfpathlineto{\pgfqpoint{2.593040in}{2.538725in}}%
\pgfpathlineto{\pgfqpoint{2.565518in}{2.538725in}}%
\pgfpathlineto{\pgfqpoint{2.565518in}{0.550000in}}%
\pgfpathclose%
\pgfusepath{fill}%
\end{pgfscope}%
\begin{pgfscope}%
\pgfpathrectangle{\pgfqpoint{1.250000in}{0.550000in}}{\pgfqpoint{7.750000in}{3.850000in}}%
\pgfusepath{clip}%
\pgfsetbuttcap%
\pgfsetmiterjoin%
\definecolor{currentfill}{rgb}{0.501961,0.501961,0.501961}%
\pgfsetfillcolor{currentfill}%
\pgfsetfillopacity{0.700000}%
\pgfsetlinewidth{0.000000pt}%
\definecolor{currentstroke}{rgb}{0.000000,0.000000,0.000000}%
\pgfsetstrokecolor{currentstroke}%
\pgfsetstrokeopacity{0.700000}%
\pgfsetdash{}{0pt}%
\pgfpathmoveto{\pgfqpoint{2.593040in}{0.550000in}}%
\pgfpathlineto{\pgfqpoint{2.620561in}{0.550000in}}%
\pgfpathlineto{\pgfqpoint{2.620561in}{2.456039in}}%
\pgfpathlineto{\pgfqpoint{2.593040in}{2.456039in}}%
\pgfpathlineto{\pgfqpoint{2.593040in}{0.550000in}}%
\pgfpathclose%
\pgfusepath{fill}%
\end{pgfscope}%
\begin{pgfscope}%
\pgfpathrectangle{\pgfqpoint{1.250000in}{0.550000in}}{\pgfqpoint{7.750000in}{3.850000in}}%
\pgfusepath{clip}%
\pgfsetbuttcap%
\pgfsetmiterjoin%
\definecolor{currentfill}{rgb}{0.501961,0.501961,0.501961}%
\pgfsetfillcolor{currentfill}%
\pgfsetfillopacity{0.700000}%
\pgfsetlinewidth{0.000000pt}%
\definecolor{currentstroke}{rgb}{0.000000,0.000000,0.000000}%
\pgfsetstrokecolor{currentstroke}%
\pgfsetstrokeopacity{0.700000}%
\pgfsetdash{}{0pt}%
\pgfpathmoveto{\pgfqpoint{2.620561in}{0.550000in}}%
\pgfpathlineto{\pgfqpoint{2.648082in}{0.550000in}}%
\pgfpathlineto{\pgfqpoint{2.648082in}{2.438932in}}%
\pgfpathlineto{\pgfqpoint{2.620561in}{2.438932in}}%
\pgfpathlineto{\pgfqpoint{2.620561in}{0.550000in}}%
\pgfpathclose%
\pgfusepath{fill}%
\end{pgfscope}%
\begin{pgfscope}%
\pgfpathrectangle{\pgfqpoint{1.250000in}{0.550000in}}{\pgfqpoint{7.750000in}{3.850000in}}%
\pgfusepath{clip}%
\pgfsetbuttcap%
\pgfsetmiterjoin%
\definecolor{currentfill}{rgb}{0.501961,0.501961,0.501961}%
\pgfsetfillcolor{currentfill}%
\pgfsetfillopacity{0.700000}%
\pgfsetlinewidth{0.000000pt}%
\definecolor{currentstroke}{rgb}{0.000000,0.000000,0.000000}%
\pgfsetstrokecolor{currentstroke}%
\pgfsetstrokeopacity{0.700000}%
\pgfsetdash{}{0pt}%
\pgfpathmoveto{\pgfqpoint{2.648082in}{0.550000in}}%
\pgfpathlineto{\pgfqpoint{2.675604in}{0.550000in}}%
\pgfpathlineto{\pgfqpoint{2.675604in}{2.427527in}}%
\pgfpathlineto{\pgfqpoint{2.648082in}{2.427527in}}%
\pgfpathlineto{\pgfqpoint{2.648082in}{0.550000in}}%
\pgfpathclose%
\pgfusepath{fill}%
\end{pgfscope}%
\begin{pgfscope}%
\pgfpathrectangle{\pgfqpoint{1.250000in}{0.550000in}}{\pgfqpoint{7.750000in}{3.850000in}}%
\pgfusepath{clip}%
\pgfsetbuttcap%
\pgfsetmiterjoin%
\definecolor{currentfill}{rgb}{0.501961,0.501961,0.501961}%
\pgfsetfillcolor{currentfill}%
\pgfsetfillopacity{0.700000}%
\pgfsetlinewidth{0.000000pt}%
\definecolor{currentstroke}{rgb}{0.000000,0.000000,0.000000}%
\pgfsetstrokecolor{currentstroke}%
\pgfsetstrokeopacity{0.700000}%
\pgfsetdash{}{0pt}%
\pgfpathmoveto{\pgfqpoint{2.675604in}{0.550000in}}%
\pgfpathlineto{\pgfqpoint{2.703125in}{0.550000in}}%
\pgfpathlineto{\pgfqpoint{2.703125in}{2.340565in}}%
\pgfpathlineto{\pgfqpoint{2.675604in}{2.340565in}}%
\pgfpathlineto{\pgfqpoint{2.675604in}{0.550000in}}%
\pgfpathclose%
\pgfusepath{fill}%
\end{pgfscope}%
\begin{pgfscope}%
\pgfpathrectangle{\pgfqpoint{1.250000in}{0.550000in}}{\pgfqpoint{7.750000in}{3.850000in}}%
\pgfusepath{clip}%
\pgfsetbuttcap%
\pgfsetmiterjoin%
\definecolor{currentfill}{rgb}{0.501961,0.501961,0.501961}%
\pgfsetfillcolor{currentfill}%
\pgfsetfillopacity{0.700000}%
\pgfsetlinewidth{0.000000pt}%
\definecolor{currentstroke}{rgb}{0.000000,0.000000,0.000000}%
\pgfsetstrokecolor{currentstroke}%
\pgfsetstrokeopacity{0.700000}%
\pgfsetdash{}{0pt}%
\pgfpathmoveto{\pgfqpoint{2.703125in}{0.550000in}}%
\pgfpathlineto{\pgfqpoint{2.730646in}{0.550000in}}%
\pgfpathlineto{\pgfqpoint{2.730646in}{2.427527in}}%
\pgfpathlineto{\pgfqpoint{2.703125in}{2.427527in}}%
\pgfpathlineto{\pgfqpoint{2.703125in}{0.550000in}}%
\pgfpathclose%
\pgfusepath{fill}%
\end{pgfscope}%
\begin{pgfscope}%
\pgfpathrectangle{\pgfqpoint{1.250000in}{0.550000in}}{\pgfqpoint{7.750000in}{3.850000in}}%
\pgfusepath{clip}%
\pgfsetbuttcap%
\pgfsetmiterjoin%
\definecolor{currentfill}{rgb}{0.501961,0.501961,0.501961}%
\pgfsetfillcolor{currentfill}%
\pgfsetfillopacity{0.700000}%
\pgfsetlinewidth{0.000000pt}%
\definecolor{currentstroke}{rgb}{0.000000,0.000000,0.000000}%
\pgfsetstrokecolor{currentstroke}%
\pgfsetstrokeopacity{0.700000}%
\pgfsetdash{}{0pt}%
\pgfpathmoveto{\pgfqpoint{2.730646in}{0.550000in}}%
\pgfpathlineto{\pgfqpoint{2.758168in}{0.550000in}}%
\pgfpathlineto{\pgfqpoint{2.758168in}{2.427527in}}%
\pgfpathlineto{\pgfqpoint{2.730646in}{2.427527in}}%
\pgfpathlineto{\pgfqpoint{2.730646in}{0.550000in}}%
\pgfpathclose%
\pgfusepath{fill}%
\end{pgfscope}%
\begin{pgfscope}%
\pgfpathrectangle{\pgfqpoint{1.250000in}{0.550000in}}{\pgfqpoint{7.750000in}{3.850000in}}%
\pgfusepath{clip}%
\pgfsetbuttcap%
\pgfsetmiterjoin%
\definecolor{currentfill}{rgb}{0.501961,0.501961,0.501961}%
\pgfsetfillcolor{currentfill}%
\pgfsetfillopacity{0.700000}%
\pgfsetlinewidth{0.000000pt}%
\definecolor{currentstroke}{rgb}{0.000000,0.000000,0.000000}%
\pgfsetstrokecolor{currentstroke}%
\pgfsetstrokeopacity{0.700000}%
\pgfsetdash{}{0pt}%
\pgfpathmoveto{\pgfqpoint{2.758168in}{0.550000in}}%
\pgfpathlineto{\pgfqpoint{2.785689in}{0.550000in}}%
\pgfpathlineto{\pgfqpoint{2.785689in}{2.363375in}}%
\pgfpathlineto{\pgfqpoint{2.758168in}{2.363375in}}%
\pgfpathlineto{\pgfqpoint{2.758168in}{0.550000in}}%
\pgfpathclose%
\pgfusepath{fill}%
\end{pgfscope}%
\begin{pgfscope}%
\pgfpathrectangle{\pgfqpoint{1.250000in}{0.550000in}}{\pgfqpoint{7.750000in}{3.850000in}}%
\pgfusepath{clip}%
\pgfsetbuttcap%
\pgfsetmiterjoin%
\definecolor{currentfill}{rgb}{0.501961,0.501961,0.501961}%
\pgfsetfillcolor{currentfill}%
\pgfsetfillopacity{0.700000}%
\pgfsetlinewidth{0.000000pt}%
\definecolor{currentstroke}{rgb}{0.000000,0.000000,0.000000}%
\pgfsetstrokecolor{currentstroke}%
\pgfsetstrokeopacity{0.700000}%
\pgfsetdash{}{0pt}%
\pgfpathmoveto{\pgfqpoint{2.785689in}{0.550000in}}%
\pgfpathlineto{\pgfqpoint{2.813210in}{0.550000in}}%
\pgfpathlineto{\pgfqpoint{2.813210in}{2.387610in}}%
\pgfpathlineto{\pgfqpoint{2.785689in}{2.387610in}}%
\pgfpathlineto{\pgfqpoint{2.785689in}{0.550000in}}%
\pgfpathclose%
\pgfusepath{fill}%
\end{pgfscope}%
\begin{pgfscope}%
\pgfpathrectangle{\pgfqpoint{1.250000in}{0.550000in}}{\pgfqpoint{7.750000in}{3.850000in}}%
\pgfusepath{clip}%
\pgfsetbuttcap%
\pgfsetmiterjoin%
\definecolor{currentfill}{rgb}{0.501961,0.501961,0.501961}%
\pgfsetfillcolor{currentfill}%
\pgfsetfillopacity{0.700000}%
\pgfsetlinewidth{0.000000pt}%
\definecolor{currentstroke}{rgb}{0.000000,0.000000,0.000000}%
\pgfsetstrokecolor{currentstroke}%
\pgfsetstrokeopacity{0.700000}%
\pgfsetdash{}{0pt}%
\pgfpathmoveto{\pgfqpoint{2.813210in}{0.550000in}}%
\pgfpathlineto{\pgfqpoint{2.840732in}{0.550000in}}%
\pgfpathlineto{\pgfqpoint{2.840732in}{4.151089in}}%
\pgfpathlineto{\pgfqpoint{2.813210in}{4.151089in}}%
\pgfpathlineto{\pgfqpoint{2.813210in}{0.550000in}}%
\pgfpathclose%
\pgfusepath{fill}%
\end{pgfscope}%
\begin{pgfscope}%
\pgfpathrectangle{\pgfqpoint{1.250000in}{0.550000in}}{\pgfqpoint{7.750000in}{3.850000in}}%
\pgfusepath{clip}%
\pgfsetbuttcap%
\pgfsetmiterjoin%
\definecolor{currentfill}{rgb}{0.501961,0.501961,0.501961}%
\pgfsetfillcolor{currentfill}%
\pgfsetfillopacity{0.700000}%
\pgfsetlinewidth{0.000000pt}%
\definecolor{currentstroke}{rgb}{0.000000,0.000000,0.000000}%
\pgfsetstrokecolor{currentstroke}%
\pgfsetstrokeopacity{0.700000}%
\pgfsetdash{}{0pt}%
\pgfpathmoveto{\pgfqpoint{2.840732in}{0.550000in}}%
\pgfpathlineto{\pgfqpoint{2.868253in}{0.550000in}}%
\pgfpathlineto{\pgfqpoint{2.868253in}{2.406143in}}%
\pgfpathlineto{\pgfqpoint{2.840732in}{2.406143in}}%
\pgfpathlineto{\pgfqpoint{2.840732in}{0.550000in}}%
\pgfpathclose%
\pgfusepath{fill}%
\end{pgfscope}%
\begin{pgfscope}%
\pgfpathrectangle{\pgfqpoint{1.250000in}{0.550000in}}{\pgfqpoint{7.750000in}{3.850000in}}%
\pgfusepath{clip}%
\pgfsetbuttcap%
\pgfsetmiterjoin%
\definecolor{currentfill}{rgb}{0.501961,0.501961,0.501961}%
\pgfsetfillcolor{currentfill}%
\pgfsetfillopacity{0.700000}%
\pgfsetlinewidth{0.000000pt}%
\definecolor{currentstroke}{rgb}{0.000000,0.000000,0.000000}%
\pgfsetstrokecolor{currentstroke}%
\pgfsetstrokeopacity{0.700000}%
\pgfsetdash{}{0pt}%
\pgfpathmoveto{\pgfqpoint{2.868253in}{0.550000in}}%
\pgfpathlineto{\pgfqpoint{2.895774in}{0.550000in}}%
\pgfpathlineto{\pgfqpoint{2.895774in}{2.424676in}}%
\pgfpathlineto{\pgfqpoint{2.868253in}{2.424676in}}%
\pgfpathlineto{\pgfqpoint{2.868253in}{0.550000in}}%
\pgfpathclose%
\pgfusepath{fill}%
\end{pgfscope}%
\begin{pgfscope}%
\pgfpathrectangle{\pgfqpoint{1.250000in}{0.550000in}}{\pgfqpoint{7.750000in}{3.850000in}}%
\pgfusepath{clip}%
\pgfsetbuttcap%
\pgfsetmiterjoin%
\definecolor{currentfill}{rgb}{0.501961,0.501961,0.501961}%
\pgfsetfillcolor{currentfill}%
\pgfsetfillopacity{0.700000}%
\pgfsetlinewidth{0.000000pt}%
\definecolor{currentstroke}{rgb}{0.000000,0.000000,0.000000}%
\pgfsetstrokecolor{currentstroke}%
\pgfsetstrokeopacity{0.700000}%
\pgfsetdash{}{0pt}%
\pgfpathmoveto{\pgfqpoint{2.895774in}{0.550000in}}%
\pgfpathlineto{\pgfqpoint{2.923295in}{0.550000in}}%
\pgfpathlineto{\pgfqpoint{2.923295in}{2.393313in}}%
\pgfpathlineto{\pgfqpoint{2.895774in}{2.393313in}}%
\pgfpathlineto{\pgfqpoint{2.895774in}{0.550000in}}%
\pgfpathclose%
\pgfusepath{fill}%
\end{pgfscope}%
\begin{pgfscope}%
\pgfpathrectangle{\pgfqpoint{1.250000in}{0.550000in}}{\pgfqpoint{7.750000in}{3.850000in}}%
\pgfusepath{clip}%
\pgfsetbuttcap%
\pgfsetmiterjoin%
\definecolor{currentfill}{rgb}{0.501961,0.501961,0.501961}%
\pgfsetfillcolor{currentfill}%
\pgfsetfillopacity{0.700000}%
\pgfsetlinewidth{0.000000pt}%
\definecolor{currentstroke}{rgb}{0.000000,0.000000,0.000000}%
\pgfsetstrokecolor{currentstroke}%
\pgfsetstrokeopacity{0.700000}%
\pgfsetdash{}{0pt}%
\pgfpathmoveto{\pgfqpoint{2.923295in}{0.550000in}}%
\pgfpathlineto{\pgfqpoint{2.950817in}{0.550000in}}%
\pgfpathlineto{\pgfqpoint{2.950817in}{2.420399in}}%
\pgfpathlineto{\pgfqpoint{2.923295in}{2.420399in}}%
\pgfpathlineto{\pgfqpoint{2.923295in}{0.550000in}}%
\pgfpathclose%
\pgfusepath{fill}%
\end{pgfscope}%
\begin{pgfscope}%
\pgfpathrectangle{\pgfqpoint{1.250000in}{0.550000in}}{\pgfqpoint{7.750000in}{3.850000in}}%
\pgfusepath{clip}%
\pgfsetbuttcap%
\pgfsetmiterjoin%
\definecolor{currentfill}{rgb}{0.501961,0.501961,0.501961}%
\pgfsetfillcolor{currentfill}%
\pgfsetfillopacity{0.700000}%
\pgfsetlinewidth{0.000000pt}%
\definecolor{currentstroke}{rgb}{0.000000,0.000000,0.000000}%
\pgfsetstrokecolor{currentstroke}%
\pgfsetstrokeopacity{0.700000}%
\pgfsetdash{}{0pt}%
\pgfpathmoveto{\pgfqpoint{2.950817in}{0.550000in}}%
\pgfpathlineto{\pgfqpoint{2.978338in}{0.550000in}}%
\pgfpathlineto{\pgfqpoint{2.978338in}{2.467444in}}%
\pgfpathlineto{\pgfqpoint{2.950817in}{2.467444in}}%
\pgfpathlineto{\pgfqpoint{2.950817in}{0.550000in}}%
\pgfpathclose%
\pgfusepath{fill}%
\end{pgfscope}%
\begin{pgfscope}%
\pgfpathrectangle{\pgfqpoint{1.250000in}{0.550000in}}{\pgfqpoint{7.750000in}{3.850000in}}%
\pgfusepath{clip}%
\pgfsetbuttcap%
\pgfsetmiterjoin%
\definecolor{currentfill}{rgb}{0.501961,0.501961,0.501961}%
\pgfsetfillcolor{currentfill}%
\pgfsetfillopacity{0.700000}%
\pgfsetlinewidth{0.000000pt}%
\definecolor{currentstroke}{rgb}{0.000000,0.000000,0.000000}%
\pgfsetstrokecolor{currentstroke}%
\pgfsetstrokeopacity{0.700000}%
\pgfsetdash{}{0pt}%
\pgfpathmoveto{\pgfqpoint{2.978338in}{0.550000in}}%
\pgfpathlineto{\pgfqpoint{3.005859in}{0.550000in}}%
\pgfpathlineto{\pgfqpoint{3.005859in}{2.426102in}}%
\pgfpathlineto{\pgfqpoint{2.978338in}{2.426102in}}%
\pgfpathlineto{\pgfqpoint{2.978338in}{0.550000in}}%
\pgfpathclose%
\pgfusepath{fill}%
\end{pgfscope}%
\begin{pgfscope}%
\pgfpathrectangle{\pgfqpoint{1.250000in}{0.550000in}}{\pgfqpoint{7.750000in}{3.850000in}}%
\pgfusepath{clip}%
\pgfsetbuttcap%
\pgfsetmiterjoin%
\definecolor{currentfill}{rgb}{0.501961,0.501961,0.501961}%
\pgfsetfillcolor{currentfill}%
\pgfsetfillopacity{0.700000}%
\pgfsetlinewidth{0.000000pt}%
\definecolor{currentstroke}{rgb}{0.000000,0.000000,0.000000}%
\pgfsetstrokecolor{currentstroke}%
\pgfsetstrokeopacity{0.700000}%
\pgfsetdash{}{0pt}%
\pgfpathmoveto{\pgfqpoint{3.005859in}{0.550000in}}%
\pgfpathlineto{\pgfqpoint{3.033381in}{0.550000in}}%
\pgfpathlineto{\pgfqpoint{3.033381in}{2.615708in}}%
\pgfpathlineto{\pgfqpoint{3.005859in}{2.615708in}}%
\pgfpathlineto{\pgfqpoint{3.005859in}{0.550000in}}%
\pgfpathclose%
\pgfusepath{fill}%
\end{pgfscope}%
\begin{pgfscope}%
\pgfpathrectangle{\pgfqpoint{1.250000in}{0.550000in}}{\pgfqpoint{7.750000in}{3.850000in}}%
\pgfusepath{clip}%
\pgfsetbuttcap%
\pgfsetmiterjoin%
\definecolor{currentfill}{rgb}{0.501961,0.501961,0.501961}%
\pgfsetfillcolor{currentfill}%
\pgfsetfillopacity{0.700000}%
\pgfsetlinewidth{0.000000pt}%
\definecolor{currentstroke}{rgb}{0.000000,0.000000,0.000000}%
\pgfsetstrokecolor{currentstroke}%
\pgfsetstrokeopacity{0.700000}%
\pgfsetdash{}{0pt}%
\pgfpathmoveto{\pgfqpoint{3.033381in}{0.550000in}}%
\pgfpathlineto{\pgfqpoint{3.060902in}{0.550000in}}%
\pgfpathlineto{\pgfqpoint{3.060902in}{2.421825in}}%
\pgfpathlineto{\pgfqpoint{3.033381in}{2.421825in}}%
\pgfpathlineto{\pgfqpoint{3.033381in}{0.550000in}}%
\pgfpathclose%
\pgfusepath{fill}%
\end{pgfscope}%
\begin{pgfscope}%
\pgfpathrectangle{\pgfqpoint{1.250000in}{0.550000in}}{\pgfqpoint{7.750000in}{3.850000in}}%
\pgfusepath{clip}%
\pgfsetbuttcap%
\pgfsetmiterjoin%
\definecolor{currentfill}{rgb}{0.501961,0.501961,0.501961}%
\pgfsetfillcolor{currentfill}%
\pgfsetfillopacity{0.700000}%
\pgfsetlinewidth{0.000000pt}%
\definecolor{currentstroke}{rgb}{0.000000,0.000000,0.000000}%
\pgfsetstrokecolor{currentstroke}%
\pgfsetstrokeopacity{0.700000}%
\pgfsetdash{}{0pt}%
\pgfpathmoveto{\pgfqpoint{3.060902in}{0.550000in}}%
\pgfpathlineto{\pgfqpoint{3.088423in}{0.550000in}}%
\pgfpathlineto{\pgfqpoint{3.088423in}{2.584344in}}%
\pgfpathlineto{\pgfqpoint{3.060902in}{2.584344in}}%
\pgfpathlineto{\pgfqpoint{3.060902in}{0.550000in}}%
\pgfpathclose%
\pgfusepath{fill}%
\end{pgfscope}%
\begin{pgfscope}%
\pgfpathrectangle{\pgfqpoint{1.250000in}{0.550000in}}{\pgfqpoint{7.750000in}{3.850000in}}%
\pgfusepath{clip}%
\pgfsetbuttcap%
\pgfsetmiterjoin%
\definecolor{currentfill}{rgb}{0.501961,0.501961,0.501961}%
\pgfsetfillcolor{currentfill}%
\pgfsetfillopacity{0.700000}%
\pgfsetlinewidth{0.000000pt}%
\definecolor{currentstroke}{rgb}{0.000000,0.000000,0.000000}%
\pgfsetstrokecolor{currentstroke}%
\pgfsetstrokeopacity{0.700000}%
\pgfsetdash{}{0pt}%
\pgfpathmoveto{\pgfqpoint{3.088423in}{0.550000in}}%
\pgfpathlineto{\pgfqpoint{3.115945in}{0.550000in}}%
\pgfpathlineto{\pgfqpoint{3.115945in}{2.537299in}}%
\pgfpathlineto{\pgfqpoint{3.088423in}{2.537299in}}%
\pgfpathlineto{\pgfqpoint{3.088423in}{0.550000in}}%
\pgfpathclose%
\pgfusepath{fill}%
\end{pgfscope}%
\begin{pgfscope}%
\pgfpathrectangle{\pgfqpoint{1.250000in}{0.550000in}}{\pgfqpoint{7.750000in}{3.850000in}}%
\pgfusepath{clip}%
\pgfsetbuttcap%
\pgfsetmiterjoin%
\definecolor{currentfill}{rgb}{0.501961,0.501961,0.501961}%
\pgfsetfillcolor{currentfill}%
\pgfsetfillopacity{0.700000}%
\pgfsetlinewidth{0.000000pt}%
\definecolor{currentstroke}{rgb}{0.000000,0.000000,0.000000}%
\pgfsetstrokecolor{currentstroke}%
\pgfsetstrokeopacity{0.700000}%
\pgfsetdash{}{0pt}%
\pgfpathmoveto{\pgfqpoint{3.115945in}{0.550000in}}%
\pgfpathlineto{\pgfqpoint{3.143466in}{0.550000in}}%
\pgfpathlineto{\pgfqpoint{3.143466in}{2.658476in}}%
\pgfpathlineto{\pgfqpoint{3.115945in}{2.658476in}}%
\pgfpathlineto{\pgfqpoint{3.115945in}{0.550000in}}%
\pgfpathclose%
\pgfusepath{fill}%
\end{pgfscope}%
\begin{pgfscope}%
\pgfpathrectangle{\pgfqpoint{1.250000in}{0.550000in}}{\pgfqpoint{7.750000in}{3.850000in}}%
\pgfusepath{clip}%
\pgfsetbuttcap%
\pgfsetmiterjoin%
\definecolor{currentfill}{rgb}{0.501961,0.501961,0.501961}%
\pgfsetfillcolor{currentfill}%
\pgfsetfillopacity{0.700000}%
\pgfsetlinewidth{0.000000pt}%
\definecolor{currentstroke}{rgb}{0.000000,0.000000,0.000000}%
\pgfsetstrokecolor{currentstroke}%
\pgfsetstrokeopacity{0.700000}%
\pgfsetdash{}{0pt}%
\pgfpathmoveto{\pgfqpoint{3.143466in}{0.550000in}}%
\pgfpathlineto{\pgfqpoint{3.170987in}{0.550000in}}%
\pgfpathlineto{\pgfqpoint{3.170987in}{2.588621in}}%
\pgfpathlineto{\pgfqpoint{3.143466in}{2.588621in}}%
\pgfpathlineto{\pgfqpoint{3.143466in}{0.550000in}}%
\pgfpathclose%
\pgfusepath{fill}%
\end{pgfscope}%
\begin{pgfscope}%
\pgfpathrectangle{\pgfqpoint{1.250000in}{0.550000in}}{\pgfqpoint{7.750000in}{3.850000in}}%
\pgfusepath{clip}%
\pgfsetbuttcap%
\pgfsetmiterjoin%
\definecolor{currentfill}{rgb}{0.501961,0.501961,0.501961}%
\pgfsetfillcolor{currentfill}%
\pgfsetfillopacity{0.700000}%
\pgfsetlinewidth{0.000000pt}%
\definecolor{currentstroke}{rgb}{0.000000,0.000000,0.000000}%
\pgfsetstrokecolor{currentstroke}%
\pgfsetstrokeopacity{0.700000}%
\pgfsetdash{}{0pt}%
\pgfpathmoveto{\pgfqpoint{3.170987in}{0.550000in}}%
\pgfpathlineto{\pgfqpoint{3.198509in}{0.550000in}}%
\pgfpathlineto{\pgfqpoint{3.198509in}{0.550000in}}%
\pgfpathlineto{\pgfqpoint{3.170987in}{0.550000in}}%
\pgfpathlineto{\pgfqpoint{3.170987in}{0.550000in}}%
\pgfpathclose%
\pgfusepath{fill}%
\end{pgfscope}%
\begin{pgfscope}%
\pgfpathrectangle{\pgfqpoint{1.250000in}{0.550000in}}{\pgfqpoint{7.750000in}{3.850000in}}%
\pgfusepath{clip}%
\pgfsetbuttcap%
\pgfsetmiterjoin%
\definecolor{currentfill}{rgb}{0.501961,0.501961,0.501961}%
\pgfsetfillcolor{currentfill}%
\pgfsetfillopacity{0.700000}%
\pgfsetlinewidth{0.000000pt}%
\definecolor{currentstroke}{rgb}{0.000000,0.000000,0.000000}%
\pgfsetstrokecolor{currentstroke}%
\pgfsetstrokeopacity{0.700000}%
\pgfsetdash{}{0pt}%
\pgfpathmoveto{\pgfqpoint{3.198509in}{0.550000in}}%
\pgfpathlineto{\pgfqpoint{3.226030in}{0.550000in}}%
\pgfpathlineto{\pgfqpoint{3.226030in}{2.631389in}}%
\pgfpathlineto{\pgfqpoint{3.198509in}{2.631389in}}%
\pgfpathlineto{\pgfqpoint{3.198509in}{0.550000in}}%
\pgfpathclose%
\pgfusepath{fill}%
\end{pgfscope}%
\begin{pgfscope}%
\pgfpathrectangle{\pgfqpoint{1.250000in}{0.550000in}}{\pgfqpoint{7.750000in}{3.850000in}}%
\pgfusepath{clip}%
\pgfsetbuttcap%
\pgfsetmiterjoin%
\definecolor{currentfill}{rgb}{0.501961,0.501961,0.501961}%
\pgfsetfillcolor{currentfill}%
\pgfsetfillopacity{0.700000}%
\pgfsetlinewidth{0.000000pt}%
\definecolor{currentstroke}{rgb}{0.000000,0.000000,0.000000}%
\pgfsetstrokecolor{currentstroke}%
\pgfsetstrokeopacity{0.700000}%
\pgfsetdash{}{0pt}%
\pgfpathmoveto{\pgfqpoint{3.226030in}{0.550000in}}%
\pgfpathlineto{\pgfqpoint{3.253551in}{0.550000in}}%
\pgfpathlineto{\pgfqpoint{3.253551in}{2.712649in}}%
\pgfpathlineto{\pgfqpoint{3.226030in}{2.712649in}}%
\pgfpathlineto{\pgfqpoint{3.226030in}{0.550000in}}%
\pgfpathclose%
\pgfusepath{fill}%
\end{pgfscope}%
\begin{pgfscope}%
\pgfpathrectangle{\pgfqpoint{1.250000in}{0.550000in}}{\pgfqpoint{7.750000in}{3.850000in}}%
\pgfusepath{clip}%
\pgfsetbuttcap%
\pgfsetmiterjoin%
\definecolor{currentfill}{rgb}{0.501961,0.501961,0.501961}%
\pgfsetfillcolor{currentfill}%
\pgfsetfillopacity{0.700000}%
\pgfsetlinewidth{0.000000pt}%
\definecolor{currentstroke}{rgb}{0.000000,0.000000,0.000000}%
\pgfsetstrokecolor{currentstroke}%
\pgfsetstrokeopacity{0.700000}%
\pgfsetdash{}{0pt}%
\pgfpathmoveto{\pgfqpoint{3.253551in}{0.550000in}}%
\pgfpathlineto{\pgfqpoint{3.281072in}{0.550000in}}%
\pgfpathlineto{\pgfqpoint{3.281072in}{2.662753in}}%
\pgfpathlineto{\pgfqpoint{3.253551in}{2.662753in}}%
\pgfpathlineto{\pgfqpoint{3.253551in}{0.550000in}}%
\pgfpathclose%
\pgfusepath{fill}%
\end{pgfscope}%
\begin{pgfscope}%
\pgfpathrectangle{\pgfqpoint{1.250000in}{0.550000in}}{\pgfqpoint{7.750000in}{3.850000in}}%
\pgfusepath{clip}%
\pgfsetbuttcap%
\pgfsetmiterjoin%
\definecolor{currentfill}{rgb}{0.501961,0.501961,0.501961}%
\pgfsetfillcolor{currentfill}%
\pgfsetfillopacity{0.700000}%
\pgfsetlinewidth{0.000000pt}%
\definecolor{currentstroke}{rgb}{0.000000,0.000000,0.000000}%
\pgfsetstrokecolor{currentstroke}%
\pgfsetstrokeopacity{0.700000}%
\pgfsetdash{}{0pt}%
\pgfpathmoveto{\pgfqpoint{3.281072in}{0.550000in}}%
\pgfpathlineto{\pgfqpoint{3.308594in}{0.550000in}}%
\pgfpathlineto{\pgfqpoint{3.308594in}{2.627112in}}%
\pgfpathlineto{\pgfqpoint{3.281072in}{2.627112in}}%
\pgfpathlineto{\pgfqpoint{3.281072in}{0.550000in}}%
\pgfpathclose%
\pgfusepath{fill}%
\end{pgfscope}%
\begin{pgfscope}%
\pgfpathrectangle{\pgfqpoint{1.250000in}{0.550000in}}{\pgfqpoint{7.750000in}{3.850000in}}%
\pgfusepath{clip}%
\pgfsetbuttcap%
\pgfsetmiterjoin%
\definecolor{currentfill}{rgb}{0.501961,0.501961,0.501961}%
\pgfsetfillcolor{currentfill}%
\pgfsetfillopacity{0.700000}%
\pgfsetlinewidth{0.000000pt}%
\definecolor{currentstroke}{rgb}{0.000000,0.000000,0.000000}%
\pgfsetstrokecolor{currentstroke}%
\pgfsetstrokeopacity{0.700000}%
\pgfsetdash{}{0pt}%
\pgfpathmoveto{\pgfqpoint{3.308594in}{0.550000in}}%
\pgfpathlineto{\pgfqpoint{3.336115in}{0.550000in}}%
\pgfpathlineto{\pgfqpoint{3.336115in}{2.829549in}}%
\pgfpathlineto{\pgfqpoint{3.308594in}{2.829549in}}%
\pgfpathlineto{\pgfqpoint{3.308594in}{0.550000in}}%
\pgfpathclose%
\pgfusepath{fill}%
\end{pgfscope}%
\begin{pgfscope}%
\pgfpathrectangle{\pgfqpoint{1.250000in}{0.550000in}}{\pgfqpoint{7.750000in}{3.850000in}}%
\pgfusepath{clip}%
\pgfsetbuttcap%
\pgfsetmiterjoin%
\definecolor{currentfill}{rgb}{0.501961,0.501961,0.501961}%
\pgfsetfillcolor{currentfill}%
\pgfsetfillopacity{0.700000}%
\pgfsetlinewidth{0.000000pt}%
\definecolor{currentstroke}{rgb}{0.000000,0.000000,0.000000}%
\pgfsetstrokecolor{currentstroke}%
\pgfsetstrokeopacity{0.700000}%
\pgfsetdash{}{0pt}%
\pgfpathmoveto{\pgfqpoint{3.336115in}{0.550000in}}%
\pgfpathlineto{\pgfqpoint{3.363636in}{0.550000in}}%
\pgfpathlineto{\pgfqpoint{3.363636in}{2.724054in}}%
\pgfpathlineto{\pgfqpoint{3.336115in}{2.724054in}}%
\pgfpathlineto{\pgfqpoint{3.336115in}{0.550000in}}%
\pgfpathclose%
\pgfusepath{fill}%
\end{pgfscope}%
\begin{pgfscope}%
\pgfpathrectangle{\pgfqpoint{1.250000in}{0.550000in}}{\pgfqpoint{7.750000in}{3.850000in}}%
\pgfusepath{clip}%
\pgfsetbuttcap%
\pgfsetmiterjoin%
\definecolor{currentfill}{rgb}{0.501961,0.501961,0.501961}%
\pgfsetfillcolor{currentfill}%
\pgfsetfillopacity{0.700000}%
\pgfsetlinewidth{0.000000pt}%
\definecolor{currentstroke}{rgb}{0.000000,0.000000,0.000000}%
\pgfsetstrokecolor{currentstroke}%
\pgfsetstrokeopacity{0.700000}%
\pgfsetdash{}{0pt}%
\pgfpathmoveto{\pgfqpoint{3.363636in}{0.550000in}}%
\pgfpathlineto{\pgfqpoint{3.391158in}{0.550000in}}%
\pgfpathlineto{\pgfqpoint{3.391158in}{2.806739in}}%
\pgfpathlineto{\pgfqpoint{3.363636in}{2.806739in}}%
\pgfpathlineto{\pgfqpoint{3.363636in}{0.550000in}}%
\pgfpathclose%
\pgfusepath{fill}%
\end{pgfscope}%
\begin{pgfscope}%
\pgfpathrectangle{\pgfqpoint{1.250000in}{0.550000in}}{\pgfqpoint{7.750000in}{3.850000in}}%
\pgfusepath{clip}%
\pgfsetbuttcap%
\pgfsetmiterjoin%
\definecolor{currentfill}{rgb}{0.501961,0.501961,0.501961}%
\pgfsetfillcolor{currentfill}%
\pgfsetfillopacity{0.700000}%
\pgfsetlinewidth{0.000000pt}%
\definecolor{currentstroke}{rgb}{0.000000,0.000000,0.000000}%
\pgfsetstrokecolor{currentstroke}%
\pgfsetstrokeopacity{0.700000}%
\pgfsetdash{}{0pt}%
\pgfpathmoveto{\pgfqpoint{3.391158in}{0.550000in}}%
\pgfpathlineto{\pgfqpoint{3.418679in}{0.550000in}}%
\pgfpathlineto{\pgfqpoint{3.418679in}{2.734033in}}%
\pgfpathlineto{\pgfqpoint{3.391158in}{2.734033in}}%
\pgfpathlineto{\pgfqpoint{3.391158in}{0.550000in}}%
\pgfpathclose%
\pgfusepath{fill}%
\end{pgfscope}%
\begin{pgfscope}%
\pgfpathrectangle{\pgfqpoint{1.250000in}{0.550000in}}{\pgfqpoint{7.750000in}{3.850000in}}%
\pgfusepath{clip}%
\pgfsetbuttcap%
\pgfsetmiterjoin%
\definecolor{currentfill}{rgb}{0.501961,0.501961,0.501961}%
\pgfsetfillcolor{currentfill}%
\pgfsetfillopacity{0.700000}%
\pgfsetlinewidth{0.000000pt}%
\definecolor{currentstroke}{rgb}{0.000000,0.000000,0.000000}%
\pgfsetstrokecolor{currentstroke}%
\pgfsetstrokeopacity{0.700000}%
\pgfsetdash{}{0pt}%
\pgfpathmoveto{\pgfqpoint{3.418679in}{0.550000in}}%
\pgfpathlineto{\pgfqpoint{3.446200in}{0.550000in}}%
\pgfpathlineto{\pgfqpoint{3.446200in}{2.811016in}}%
\pgfpathlineto{\pgfqpoint{3.418679in}{2.811016in}}%
\pgfpathlineto{\pgfqpoint{3.418679in}{0.550000in}}%
\pgfpathclose%
\pgfusepath{fill}%
\end{pgfscope}%
\begin{pgfscope}%
\pgfpathrectangle{\pgfqpoint{1.250000in}{0.550000in}}{\pgfqpoint{7.750000in}{3.850000in}}%
\pgfusepath{clip}%
\pgfsetbuttcap%
\pgfsetmiterjoin%
\definecolor{currentfill}{rgb}{0.501961,0.501961,0.501961}%
\pgfsetfillcolor{currentfill}%
\pgfsetfillopacity{0.700000}%
\pgfsetlinewidth{0.000000pt}%
\definecolor{currentstroke}{rgb}{0.000000,0.000000,0.000000}%
\pgfsetstrokecolor{currentstroke}%
\pgfsetstrokeopacity{0.700000}%
\pgfsetdash{}{0pt}%
\pgfpathmoveto{\pgfqpoint{3.446200in}{0.550000in}}%
\pgfpathlineto{\pgfqpoint{3.473722in}{0.550000in}}%
\pgfpathlineto{\pgfqpoint{3.473722in}{2.752566in}}%
\pgfpathlineto{\pgfqpoint{3.446200in}{2.752566in}}%
\pgfpathlineto{\pgfqpoint{3.446200in}{0.550000in}}%
\pgfpathclose%
\pgfusepath{fill}%
\end{pgfscope}%
\begin{pgfscope}%
\pgfpathrectangle{\pgfqpoint{1.250000in}{0.550000in}}{\pgfqpoint{7.750000in}{3.850000in}}%
\pgfusepath{clip}%
\pgfsetbuttcap%
\pgfsetmiterjoin%
\definecolor{currentfill}{rgb}{0.501961,0.501961,0.501961}%
\pgfsetfillcolor{currentfill}%
\pgfsetfillopacity{0.700000}%
\pgfsetlinewidth{0.000000pt}%
\definecolor{currentstroke}{rgb}{0.000000,0.000000,0.000000}%
\pgfsetstrokecolor{currentstroke}%
\pgfsetstrokeopacity{0.700000}%
\pgfsetdash{}{0pt}%
\pgfpathmoveto{\pgfqpoint{3.473722in}{0.550000in}}%
\pgfpathlineto{\pgfqpoint{3.501243in}{0.550000in}}%
\pgfpathlineto{\pgfqpoint{3.501243in}{2.714075in}}%
\pgfpathlineto{\pgfqpoint{3.473722in}{2.714075in}}%
\pgfpathlineto{\pgfqpoint{3.473722in}{0.550000in}}%
\pgfpathclose%
\pgfusepath{fill}%
\end{pgfscope}%
\begin{pgfscope}%
\pgfpathrectangle{\pgfqpoint{1.250000in}{0.550000in}}{\pgfqpoint{7.750000in}{3.850000in}}%
\pgfusepath{clip}%
\pgfsetbuttcap%
\pgfsetmiterjoin%
\definecolor{currentfill}{rgb}{0.501961,0.501961,0.501961}%
\pgfsetfillcolor{currentfill}%
\pgfsetfillopacity{0.700000}%
\pgfsetlinewidth{0.000000pt}%
\definecolor{currentstroke}{rgb}{0.000000,0.000000,0.000000}%
\pgfsetstrokecolor{currentstroke}%
\pgfsetstrokeopacity{0.700000}%
\pgfsetdash{}{0pt}%
\pgfpathmoveto{\pgfqpoint{3.501243in}{0.550000in}}%
\pgfpathlineto{\pgfqpoint{3.528764in}{0.550000in}}%
\pgfpathlineto{\pgfqpoint{3.528764in}{2.724054in}}%
\pgfpathlineto{\pgfqpoint{3.501243in}{2.724054in}}%
\pgfpathlineto{\pgfqpoint{3.501243in}{0.550000in}}%
\pgfpathclose%
\pgfusepath{fill}%
\end{pgfscope}%
\begin{pgfscope}%
\pgfpathrectangle{\pgfqpoint{1.250000in}{0.550000in}}{\pgfqpoint{7.750000in}{3.850000in}}%
\pgfusepath{clip}%
\pgfsetbuttcap%
\pgfsetmiterjoin%
\definecolor{currentfill}{rgb}{0.501961,0.501961,0.501961}%
\pgfsetfillcolor{currentfill}%
\pgfsetfillopacity{0.700000}%
\pgfsetlinewidth{0.000000pt}%
\definecolor{currentstroke}{rgb}{0.000000,0.000000,0.000000}%
\pgfsetstrokecolor{currentstroke}%
\pgfsetstrokeopacity{0.700000}%
\pgfsetdash{}{0pt}%
\pgfpathmoveto{\pgfqpoint{3.528764in}{0.550000in}}%
\pgfpathlineto{\pgfqpoint{3.556286in}{0.550000in}}%
\pgfpathlineto{\pgfqpoint{3.556286in}{2.719777in}}%
\pgfpathlineto{\pgfqpoint{3.528764in}{2.719777in}}%
\pgfpathlineto{\pgfqpoint{3.528764in}{0.550000in}}%
\pgfpathclose%
\pgfusepath{fill}%
\end{pgfscope}%
\begin{pgfscope}%
\pgfpathrectangle{\pgfqpoint{1.250000in}{0.550000in}}{\pgfqpoint{7.750000in}{3.850000in}}%
\pgfusepath{clip}%
\pgfsetbuttcap%
\pgfsetmiterjoin%
\definecolor{currentfill}{rgb}{0.501961,0.501961,0.501961}%
\pgfsetfillcolor{currentfill}%
\pgfsetfillopacity{0.700000}%
\pgfsetlinewidth{0.000000pt}%
\definecolor{currentstroke}{rgb}{0.000000,0.000000,0.000000}%
\pgfsetstrokecolor{currentstroke}%
\pgfsetstrokeopacity{0.700000}%
\pgfsetdash{}{0pt}%
\pgfpathmoveto{\pgfqpoint{3.556286in}{0.550000in}}%
\pgfpathlineto{\pgfqpoint{3.583807in}{0.550000in}}%
\pgfpathlineto{\pgfqpoint{3.583807in}{2.849508in}}%
\pgfpathlineto{\pgfqpoint{3.556286in}{2.849508in}}%
\pgfpathlineto{\pgfqpoint{3.556286in}{0.550000in}}%
\pgfpathclose%
\pgfusepath{fill}%
\end{pgfscope}%
\begin{pgfscope}%
\pgfpathrectangle{\pgfqpoint{1.250000in}{0.550000in}}{\pgfqpoint{7.750000in}{3.850000in}}%
\pgfusepath{clip}%
\pgfsetbuttcap%
\pgfsetmiterjoin%
\definecolor{currentfill}{rgb}{0.501961,0.501961,0.501961}%
\pgfsetfillcolor{currentfill}%
\pgfsetfillopacity{0.700000}%
\pgfsetlinewidth{0.000000pt}%
\definecolor{currentstroke}{rgb}{0.000000,0.000000,0.000000}%
\pgfsetstrokecolor{currentstroke}%
\pgfsetstrokeopacity{0.700000}%
\pgfsetdash{}{0pt}%
\pgfpathmoveto{\pgfqpoint{3.583807in}{0.550000in}}%
\pgfpathlineto{\pgfqpoint{3.611328in}{0.550000in}}%
\pgfpathlineto{\pgfqpoint{3.611328in}{3.017729in}}%
\pgfpathlineto{\pgfqpoint{3.583807in}{3.017729in}}%
\pgfpathlineto{\pgfqpoint{3.583807in}{0.550000in}}%
\pgfpathclose%
\pgfusepath{fill}%
\end{pgfscope}%
\begin{pgfscope}%
\pgfpathrectangle{\pgfqpoint{1.250000in}{0.550000in}}{\pgfqpoint{7.750000in}{3.850000in}}%
\pgfusepath{clip}%
\pgfsetbuttcap%
\pgfsetmiterjoin%
\definecolor{currentfill}{rgb}{0.501961,0.501961,0.501961}%
\pgfsetfillcolor{currentfill}%
\pgfsetfillopacity{0.700000}%
\pgfsetlinewidth{0.000000pt}%
\definecolor{currentstroke}{rgb}{0.000000,0.000000,0.000000}%
\pgfsetstrokecolor{currentstroke}%
\pgfsetstrokeopacity{0.700000}%
\pgfsetdash{}{0pt}%
\pgfpathmoveto{\pgfqpoint{3.611328in}{0.550000in}}%
\pgfpathlineto{\pgfqpoint{3.638849in}{0.550000in}}%
\pgfpathlineto{\pgfqpoint{3.638849in}{2.993494in}}%
\pgfpathlineto{\pgfqpoint{3.611328in}{2.993494in}}%
\pgfpathlineto{\pgfqpoint{3.611328in}{0.550000in}}%
\pgfpathclose%
\pgfusepath{fill}%
\end{pgfscope}%
\begin{pgfscope}%
\pgfpathrectangle{\pgfqpoint{1.250000in}{0.550000in}}{\pgfqpoint{7.750000in}{3.850000in}}%
\pgfusepath{clip}%
\pgfsetbuttcap%
\pgfsetmiterjoin%
\definecolor{currentfill}{rgb}{0.501961,0.501961,0.501961}%
\pgfsetfillcolor{currentfill}%
\pgfsetfillopacity{0.700000}%
\pgfsetlinewidth{0.000000pt}%
\definecolor{currentstroke}{rgb}{0.000000,0.000000,0.000000}%
\pgfsetstrokecolor{currentstroke}%
\pgfsetstrokeopacity{0.700000}%
\pgfsetdash{}{0pt}%
\pgfpathmoveto{\pgfqpoint{3.638849in}{0.550000in}}%
\pgfpathlineto{\pgfqpoint{3.666371in}{0.550000in}}%
\pgfpathlineto{\pgfqpoint{3.666371in}{2.900829in}}%
\pgfpathlineto{\pgfqpoint{3.638849in}{2.900829in}}%
\pgfpathlineto{\pgfqpoint{3.638849in}{0.550000in}}%
\pgfpathclose%
\pgfusepath{fill}%
\end{pgfscope}%
\begin{pgfscope}%
\pgfpathrectangle{\pgfqpoint{1.250000in}{0.550000in}}{\pgfqpoint{7.750000in}{3.850000in}}%
\pgfusepath{clip}%
\pgfsetbuttcap%
\pgfsetmiterjoin%
\definecolor{currentfill}{rgb}{0.501961,0.501961,0.501961}%
\pgfsetfillcolor{currentfill}%
\pgfsetfillopacity{0.700000}%
\pgfsetlinewidth{0.000000pt}%
\definecolor{currentstroke}{rgb}{0.000000,0.000000,0.000000}%
\pgfsetstrokecolor{currentstroke}%
\pgfsetstrokeopacity{0.700000}%
\pgfsetdash{}{0pt}%
\pgfpathmoveto{\pgfqpoint{3.666371in}{0.550000in}}%
\pgfpathlineto{\pgfqpoint{3.693892in}{0.550000in}}%
\pgfpathlineto{\pgfqpoint{3.693892in}{2.853784in}}%
\pgfpathlineto{\pgfqpoint{3.666371in}{2.853784in}}%
\pgfpathlineto{\pgfqpoint{3.666371in}{0.550000in}}%
\pgfpathclose%
\pgfusepath{fill}%
\end{pgfscope}%
\begin{pgfscope}%
\pgfpathrectangle{\pgfqpoint{1.250000in}{0.550000in}}{\pgfqpoint{7.750000in}{3.850000in}}%
\pgfusepath{clip}%
\pgfsetbuttcap%
\pgfsetmiterjoin%
\definecolor{currentfill}{rgb}{0.501961,0.501961,0.501961}%
\pgfsetfillcolor{currentfill}%
\pgfsetfillopacity{0.700000}%
\pgfsetlinewidth{0.000000pt}%
\definecolor{currentstroke}{rgb}{0.000000,0.000000,0.000000}%
\pgfsetstrokecolor{currentstroke}%
\pgfsetstrokeopacity{0.700000}%
\pgfsetdash{}{0pt}%
\pgfpathmoveto{\pgfqpoint{3.693892in}{0.550000in}}%
\pgfpathlineto{\pgfqpoint{3.721413in}{0.550000in}}%
\pgfpathlineto{\pgfqpoint{3.721413in}{2.791058in}}%
\pgfpathlineto{\pgfqpoint{3.693892in}{2.791058in}}%
\pgfpathlineto{\pgfqpoint{3.693892in}{0.550000in}}%
\pgfpathclose%
\pgfusepath{fill}%
\end{pgfscope}%
\begin{pgfscope}%
\pgfpathrectangle{\pgfqpoint{1.250000in}{0.550000in}}{\pgfqpoint{7.750000in}{3.850000in}}%
\pgfusepath{clip}%
\pgfsetbuttcap%
\pgfsetmiterjoin%
\definecolor{currentfill}{rgb}{0.501961,0.501961,0.501961}%
\pgfsetfillcolor{currentfill}%
\pgfsetfillopacity{0.700000}%
\pgfsetlinewidth{0.000000pt}%
\definecolor{currentstroke}{rgb}{0.000000,0.000000,0.000000}%
\pgfsetstrokecolor{currentstroke}%
\pgfsetstrokeopacity{0.700000}%
\pgfsetdash{}{0pt}%
\pgfpathmoveto{\pgfqpoint{3.721413in}{0.550000in}}%
\pgfpathlineto{\pgfqpoint{3.748935in}{0.550000in}}%
\pgfpathlineto{\pgfqpoint{3.748935in}{2.816719in}}%
\pgfpathlineto{\pgfqpoint{3.721413in}{2.816719in}}%
\pgfpathlineto{\pgfqpoint{3.721413in}{0.550000in}}%
\pgfpathclose%
\pgfusepath{fill}%
\end{pgfscope}%
\begin{pgfscope}%
\pgfpathrectangle{\pgfqpoint{1.250000in}{0.550000in}}{\pgfqpoint{7.750000in}{3.850000in}}%
\pgfusepath{clip}%
\pgfsetbuttcap%
\pgfsetmiterjoin%
\definecolor{currentfill}{rgb}{0.501961,0.501961,0.501961}%
\pgfsetfillcolor{currentfill}%
\pgfsetfillopacity{0.700000}%
\pgfsetlinewidth{0.000000pt}%
\definecolor{currentstroke}{rgb}{0.000000,0.000000,0.000000}%
\pgfsetstrokecolor{currentstroke}%
\pgfsetstrokeopacity{0.700000}%
\pgfsetdash{}{0pt}%
\pgfpathmoveto{\pgfqpoint{3.748935in}{0.550000in}}%
\pgfpathlineto{\pgfqpoint{3.776456in}{0.550000in}}%
\pgfpathlineto{\pgfqpoint{3.776456in}{0.550000in}}%
\pgfpathlineto{\pgfqpoint{3.748935in}{0.550000in}}%
\pgfpathlineto{\pgfqpoint{3.748935in}{0.550000in}}%
\pgfpathclose%
\pgfusepath{fill}%
\end{pgfscope}%
\begin{pgfscope}%
\pgfpathrectangle{\pgfqpoint{1.250000in}{0.550000in}}{\pgfqpoint{7.750000in}{3.850000in}}%
\pgfusepath{clip}%
\pgfsetbuttcap%
\pgfsetmiterjoin%
\definecolor{currentfill}{rgb}{0.501961,0.501961,0.501961}%
\pgfsetfillcolor{currentfill}%
\pgfsetfillopacity{0.700000}%
\pgfsetlinewidth{0.000000pt}%
\definecolor{currentstroke}{rgb}{0.000000,0.000000,0.000000}%
\pgfsetstrokecolor{currentstroke}%
\pgfsetstrokeopacity{0.700000}%
\pgfsetdash{}{0pt}%
\pgfpathmoveto{\pgfqpoint{3.776456in}{0.550000in}}%
\pgfpathlineto{\pgfqpoint{3.803977in}{0.550000in}}%
\pgfpathlineto{\pgfqpoint{3.803977in}{2.843805in}}%
\pgfpathlineto{\pgfqpoint{3.776456in}{2.843805in}}%
\pgfpathlineto{\pgfqpoint{3.776456in}{0.550000in}}%
\pgfpathclose%
\pgfusepath{fill}%
\end{pgfscope}%
\begin{pgfscope}%
\pgfpathrectangle{\pgfqpoint{1.250000in}{0.550000in}}{\pgfqpoint{7.750000in}{3.850000in}}%
\pgfusepath{clip}%
\pgfsetbuttcap%
\pgfsetmiterjoin%
\definecolor{currentfill}{rgb}{0.501961,0.501961,0.501961}%
\pgfsetfillcolor{currentfill}%
\pgfsetfillopacity{0.700000}%
\pgfsetlinewidth{0.000000pt}%
\definecolor{currentstroke}{rgb}{0.000000,0.000000,0.000000}%
\pgfsetstrokecolor{currentstroke}%
\pgfsetstrokeopacity{0.700000}%
\pgfsetdash{}{0pt}%
\pgfpathmoveto{\pgfqpoint{3.803977in}{0.550000in}}%
\pgfpathlineto{\pgfqpoint{3.831499in}{0.550000in}}%
\pgfpathlineto{\pgfqpoint{3.831499in}{2.785355in}}%
\pgfpathlineto{\pgfqpoint{3.803977in}{2.785355in}}%
\pgfpathlineto{\pgfqpoint{3.803977in}{0.550000in}}%
\pgfpathclose%
\pgfusepath{fill}%
\end{pgfscope}%
\begin{pgfscope}%
\pgfpathrectangle{\pgfqpoint{1.250000in}{0.550000in}}{\pgfqpoint{7.750000in}{3.850000in}}%
\pgfusepath{clip}%
\pgfsetbuttcap%
\pgfsetmiterjoin%
\definecolor{currentfill}{rgb}{0.501961,0.501961,0.501961}%
\pgfsetfillcolor{currentfill}%
\pgfsetfillopacity{0.700000}%
\pgfsetlinewidth{0.000000pt}%
\definecolor{currentstroke}{rgb}{0.000000,0.000000,0.000000}%
\pgfsetstrokecolor{currentstroke}%
\pgfsetstrokeopacity{0.700000}%
\pgfsetdash{}{0pt}%
\pgfpathmoveto{\pgfqpoint{3.831499in}{0.550000in}}%
\pgfpathlineto{\pgfqpoint{3.859020in}{0.550000in}}%
\pgfpathlineto{\pgfqpoint{3.859020in}{2.869466in}}%
\pgfpathlineto{\pgfqpoint{3.831499in}{2.869466in}}%
\pgfpathlineto{\pgfqpoint{3.831499in}{0.550000in}}%
\pgfpathclose%
\pgfusepath{fill}%
\end{pgfscope}%
\begin{pgfscope}%
\pgfpathrectangle{\pgfqpoint{1.250000in}{0.550000in}}{\pgfqpoint{7.750000in}{3.850000in}}%
\pgfusepath{clip}%
\pgfsetbuttcap%
\pgfsetmiterjoin%
\definecolor{currentfill}{rgb}{0.501961,0.501961,0.501961}%
\pgfsetfillcolor{currentfill}%
\pgfsetfillopacity{0.700000}%
\pgfsetlinewidth{0.000000pt}%
\definecolor{currentstroke}{rgb}{0.000000,0.000000,0.000000}%
\pgfsetstrokecolor{currentstroke}%
\pgfsetstrokeopacity{0.700000}%
\pgfsetdash{}{0pt}%
\pgfpathmoveto{\pgfqpoint{3.859020in}{0.550000in}}%
\pgfpathlineto{\pgfqpoint{3.886541in}{0.550000in}}%
\pgfpathlineto{\pgfqpoint{3.886541in}{2.759694in}}%
\pgfpathlineto{\pgfqpoint{3.859020in}{2.759694in}}%
\pgfpathlineto{\pgfqpoint{3.859020in}{0.550000in}}%
\pgfpathclose%
\pgfusepath{fill}%
\end{pgfscope}%
\begin{pgfscope}%
\pgfpathrectangle{\pgfqpoint{1.250000in}{0.550000in}}{\pgfqpoint{7.750000in}{3.850000in}}%
\pgfusepath{clip}%
\pgfsetbuttcap%
\pgfsetmiterjoin%
\definecolor{currentfill}{rgb}{0.501961,0.501961,0.501961}%
\pgfsetfillcolor{currentfill}%
\pgfsetfillopacity{0.700000}%
\pgfsetlinewidth{0.000000pt}%
\definecolor{currentstroke}{rgb}{0.000000,0.000000,0.000000}%
\pgfsetstrokecolor{currentstroke}%
\pgfsetstrokeopacity{0.700000}%
\pgfsetdash{}{0pt}%
\pgfpathmoveto{\pgfqpoint{3.886541in}{0.550000in}}%
\pgfpathlineto{\pgfqpoint{3.914062in}{0.550000in}}%
\pgfpathlineto{\pgfqpoint{3.914062in}{2.836677in}}%
\pgfpathlineto{\pgfqpoint{3.886541in}{2.836677in}}%
\pgfpathlineto{\pgfqpoint{3.886541in}{0.550000in}}%
\pgfpathclose%
\pgfusepath{fill}%
\end{pgfscope}%
\begin{pgfscope}%
\pgfpathrectangle{\pgfqpoint{1.250000in}{0.550000in}}{\pgfqpoint{7.750000in}{3.850000in}}%
\pgfusepath{clip}%
\pgfsetbuttcap%
\pgfsetmiterjoin%
\definecolor{currentfill}{rgb}{0.501961,0.501961,0.501961}%
\pgfsetfillcolor{currentfill}%
\pgfsetfillopacity{0.700000}%
\pgfsetlinewidth{0.000000pt}%
\definecolor{currentstroke}{rgb}{0.000000,0.000000,0.000000}%
\pgfsetstrokecolor{currentstroke}%
\pgfsetstrokeopacity{0.700000}%
\pgfsetdash{}{0pt}%
\pgfpathmoveto{\pgfqpoint{3.914062in}{0.550000in}}%
\pgfpathlineto{\pgfqpoint{3.941584in}{0.550000in}}%
\pgfpathlineto{\pgfqpoint{3.941584in}{2.749715in}}%
\pgfpathlineto{\pgfqpoint{3.914062in}{2.749715in}}%
\pgfpathlineto{\pgfqpoint{3.914062in}{0.550000in}}%
\pgfpathclose%
\pgfusepath{fill}%
\end{pgfscope}%
\begin{pgfscope}%
\pgfpathrectangle{\pgfqpoint{1.250000in}{0.550000in}}{\pgfqpoint{7.750000in}{3.850000in}}%
\pgfusepath{clip}%
\pgfsetbuttcap%
\pgfsetmiterjoin%
\definecolor{currentfill}{rgb}{0.501961,0.501961,0.501961}%
\pgfsetfillcolor{currentfill}%
\pgfsetfillopacity{0.700000}%
\pgfsetlinewidth{0.000000pt}%
\definecolor{currentstroke}{rgb}{0.000000,0.000000,0.000000}%
\pgfsetstrokecolor{currentstroke}%
\pgfsetstrokeopacity{0.700000}%
\pgfsetdash{}{0pt}%
\pgfpathmoveto{\pgfqpoint{3.941584in}{0.550000in}}%
\pgfpathlineto{\pgfqpoint{3.969105in}{0.550000in}}%
\pgfpathlineto{\pgfqpoint{3.969105in}{2.762545in}}%
\pgfpathlineto{\pgfqpoint{3.941584in}{2.762545in}}%
\pgfpathlineto{\pgfqpoint{3.941584in}{0.550000in}}%
\pgfpathclose%
\pgfusepath{fill}%
\end{pgfscope}%
\begin{pgfscope}%
\pgfpathrectangle{\pgfqpoint{1.250000in}{0.550000in}}{\pgfqpoint{7.750000in}{3.850000in}}%
\pgfusepath{clip}%
\pgfsetbuttcap%
\pgfsetmiterjoin%
\definecolor{currentfill}{rgb}{0.501961,0.501961,0.501961}%
\pgfsetfillcolor{currentfill}%
\pgfsetfillopacity{0.700000}%
\pgfsetlinewidth{0.000000pt}%
\definecolor{currentstroke}{rgb}{0.000000,0.000000,0.000000}%
\pgfsetstrokecolor{currentstroke}%
\pgfsetstrokeopacity{0.700000}%
\pgfsetdash{}{0pt}%
\pgfpathmoveto{\pgfqpoint{3.969105in}{0.550000in}}%
\pgfpathlineto{\pgfqpoint{3.996626in}{0.550000in}}%
\pgfpathlineto{\pgfqpoint{3.996626in}{2.719777in}}%
\pgfpathlineto{\pgfqpoint{3.969105in}{2.719777in}}%
\pgfpathlineto{\pgfqpoint{3.969105in}{0.550000in}}%
\pgfpathclose%
\pgfusepath{fill}%
\end{pgfscope}%
\begin{pgfscope}%
\pgfpathrectangle{\pgfqpoint{1.250000in}{0.550000in}}{\pgfqpoint{7.750000in}{3.850000in}}%
\pgfusepath{clip}%
\pgfsetbuttcap%
\pgfsetmiterjoin%
\definecolor{currentfill}{rgb}{0.501961,0.501961,0.501961}%
\pgfsetfillcolor{currentfill}%
\pgfsetfillopacity{0.700000}%
\pgfsetlinewidth{0.000000pt}%
\definecolor{currentstroke}{rgb}{0.000000,0.000000,0.000000}%
\pgfsetstrokecolor{currentstroke}%
\pgfsetstrokeopacity{0.700000}%
\pgfsetdash{}{0pt}%
\pgfpathmoveto{\pgfqpoint{3.996626in}{0.550000in}}%
\pgfpathlineto{\pgfqpoint{4.024148in}{0.550000in}}%
\pgfpathlineto{\pgfqpoint{4.024148in}{2.637092in}}%
\pgfpathlineto{\pgfqpoint{3.996626in}{2.637092in}}%
\pgfpathlineto{\pgfqpoint{3.996626in}{0.550000in}}%
\pgfpathclose%
\pgfusepath{fill}%
\end{pgfscope}%
\begin{pgfscope}%
\pgfpathrectangle{\pgfqpoint{1.250000in}{0.550000in}}{\pgfqpoint{7.750000in}{3.850000in}}%
\pgfusepath{clip}%
\pgfsetbuttcap%
\pgfsetmiterjoin%
\definecolor{currentfill}{rgb}{0.501961,0.501961,0.501961}%
\pgfsetfillcolor{currentfill}%
\pgfsetfillopacity{0.700000}%
\pgfsetlinewidth{0.000000pt}%
\definecolor{currentstroke}{rgb}{0.000000,0.000000,0.000000}%
\pgfsetstrokecolor{currentstroke}%
\pgfsetstrokeopacity{0.700000}%
\pgfsetdash{}{0pt}%
\pgfpathmoveto{\pgfqpoint{4.024148in}{0.550000in}}%
\pgfpathlineto{\pgfqpoint{4.051669in}{0.550000in}}%
\pgfpathlineto{\pgfqpoint{4.051669in}{2.608580in}}%
\pgfpathlineto{\pgfqpoint{4.024148in}{2.608580in}}%
\pgfpathlineto{\pgfqpoint{4.024148in}{0.550000in}}%
\pgfpathclose%
\pgfusepath{fill}%
\end{pgfscope}%
\begin{pgfscope}%
\pgfpathrectangle{\pgfqpoint{1.250000in}{0.550000in}}{\pgfqpoint{7.750000in}{3.850000in}}%
\pgfusepath{clip}%
\pgfsetbuttcap%
\pgfsetmiterjoin%
\definecolor{currentfill}{rgb}{0.501961,0.501961,0.501961}%
\pgfsetfillcolor{currentfill}%
\pgfsetfillopacity{0.700000}%
\pgfsetlinewidth{0.000000pt}%
\definecolor{currentstroke}{rgb}{0.000000,0.000000,0.000000}%
\pgfsetstrokecolor{currentstroke}%
\pgfsetstrokeopacity{0.700000}%
\pgfsetdash{}{0pt}%
\pgfpathmoveto{\pgfqpoint{4.051669in}{0.550000in}}%
\pgfpathlineto{\pgfqpoint{4.079190in}{0.550000in}}%
\pgfpathlineto{\pgfqpoint{4.079190in}{2.685562in}}%
\pgfpathlineto{\pgfqpoint{4.051669in}{2.685562in}}%
\pgfpathlineto{\pgfqpoint{4.051669in}{0.550000in}}%
\pgfpathclose%
\pgfusepath{fill}%
\end{pgfscope}%
\begin{pgfscope}%
\pgfpathrectangle{\pgfqpoint{1.250000in}{0.550000in}}{\pgfqpoint{7.750000in}{3.850000in}}%
\pgfusepath{clip}%
\pgfsetbuttcap%
\pgfsetmiterjoin%
\definecolor{currentfill}{rgb}{0.501961,0.501961,0.501961}%
\pgfsetfillcolor{currentfill}%
\pgfsetfillopacity{0.700000}%
\pgfsetlinewidth{0.000000pt}%
\definecolor{currentstroke}{rgb}{0.000000,0.000000,0.000000}%
\pgfsetstrokecolor{currentstroke}%
\pgfsetstrokeopacity{0.700000}%
\pgfsetdash{}{0pt}%
\pgfpathmoveto{\pgfqpoint{4.079190in}{0.550000in}}%
\pgfpathlineto{\pgfqpoint{4.106712in}{0.550000in}}%
\pgfpathlineto{\pgfqpoint{4.106712in}{2.578642in}}%
\pgfpathlineto{\pgfqpoint{4.079190in}{2.578642in}}%
\pgfpathlineto{\pgfqpoint{4.079190in}{0.550000in}}%
\pgfpathclose%
\pgfusepath{fill}%
\end{pgfscope}%
\begin{pgfscope}%
\pgfpathrectangle{\pgfqpoint{1.250000in}{0.550000in}}{\pgfqpoint{7.750000in}{3.850000in}}%
\pgfusepath{clip}%
\pgfsetbuttcap%
\pgfsetmiterjoin%
\definecolor{currentfill}{rgb}{0.501961,0.501961,0.501961}%
\pgfsetfillcolor{currentfill}%
\pgfsetfillopacity{0.700000}%
\pgfsetlinewidth{0.000000pt}%
\definecolor{currentstroke}{rgb}{0.000000,0.000000,0.000000}%
\pgfsetstrokecolor{currentstroke}%
\pgfsetstrokeopacity{0.700000}%
\pgfsetdash{}{0pt}%
\pgfpathmoveto{\pgfqpoint{4.106712in}{0.550000in}}%
\pgfpathlineto{\pgfqpoint{4.134233in}{0.550000in}}%
\pgfpathlineto{\pgfqpoint{4.134233in}{2.448911in}}%
\pgfpathlineto{\pgfqpoint{4.106712in}{2.448911in}}%
\pgfpathlineto{\pgfqpoint{4.106712in}{0.550000in}}%
\pgfpathclose%
\pgfusepath{fill}%
\end{pgfscope}%
\begin{pgfscope}%
\pgfpathrectangle{\pgfqpoint{1.250000in}{0.550000in}}{\pgfqpoint{7.750000in}{3.850000in}}%
\pgfusepath{clip}%
\pgfsetbuttcap%
\pgfsetmiterjoin%
\definecolor{currentfill}{rgb}{0.501961,0.501961,0.501961}%
\pgfsetfillcolor{currentfill}%
\pgfsetfillopacity{0.700000}%
\pgfsetlinewidth{0.000000pt}%
\definecolor{currentstroke}{rgb}{0.000000,0.000000,0.000000}%
\pgfsetstrokecolor{currentstroke}%
\pgfsetstrokeopacity{0.700000}%
\pgfsetdash{}{0pt}%
\pgfpathmoveto{\pgfqpoint{4.134233in}{0.550000in}}%
\pgfpathlineto{\pgfqpoint{4.161754in}{0.550000in}}%
\pgfpathlineto{\pgfqpoint{4.161754in}{4.216667in}}%
\pgfpathlineto{\pgfqpoint{4.134233in}{4.216667in}}%
\pgfpathlineto{\pgfqpoint{4.134233in}{0.550000in}}%
\pgfpathclose%
\pgfusepath{fill}%
\end{pgfscope}%
\begin{pgfscope}%
\pgfpathrectangle{\pgfqpoint{1.250000in}{0.550000in}}{\pgfqpoint{7.750000in}{3.850000in}}%
\pgfusepath{clip}%
\pgfsetbuttcap%
\pgfsetmiterjoin%
\definecolor{currentfill}{rgb}{0.501961,0.501961,0.501961}%
\pgfsetfillcolor{currentfill}%
\pgfsetfillopacity{0.700000}%
\pgfsetlinewidth{0.000000pt}%
\definecolor{currentstroke}{rgb}{0.000000,0.000000,0.000000}%
\pgfsetstrokecolor{currentstroke}%
\pgfsetstrokeopacity{0.700000}%
\pgfsetdash{}{0pt}%
\pgfpathmoveto{\pgfqpoint{4.161754in}{0.550000in}}%
\pgfpathlineto{\pgfqpoint{4.189276in}{0.550000in}}%
\pgfpathlineto{\pgfqpoint{4.189276in}{2.441783in}}%
\pgfpathlineto{\pgfqpoint{4.161754in}{2.441783in}}%
\pgfpathlineto{\pgfqpoint{4.161754in}{0.550000in}}%
\pgfpathclose%
\pgfusepath{fill}%
\end{pgfscope}%
\begin{pgfscope}%
\pgfpathrectangle{\pgfqpoint{1.250000in}{0.550000in}}{\pgfqpoint{7.750000in}{3.850000in}}%
\pgfusepath{clip}%
\pgfsetbuttcap%
\pgfsetmiterjoin%
\definecolor{currentfill}{rgb}{0.501961,0.501961,0.501961}%
\pgfsetfillcolor{currentfill}%
\pgfsetfillopacity{0.700000}%
\pgfsetlinewidth{0.000000pt}%
\definecolor{currentstroke}{rgb}{0.000000,0.000000,0.000000}%
\pgfsetstrokecolor{currentstroke}%
\pgfsetstrokeopacity{0.700000}%
\pgfsetdash{}{0pt}%
\pgfpathmoveto{\pgfqpoint{4.189276in}{0.550000in}}%
\pgfpathlineto{\pgfqpoint{4.216797in}{0.550000in}}%
\pgfpathlineto{\pgfqpoint{4.216797in}{2.304925in}}%
\pgfpathlineto{\pgfqpoint{4.189276in}{2.304925in}}%
\pgfpathlineto{\pgfqpoint{4.189276in}{0.550000in}}%
\pgfpathclose%
\pgfusepath{fill}%
\end{pgfscope}%
\begin{pgfscope}%
\pgfpathrectangle{\pgfqpoint{1.250000in}{0.550000in}}{\pgfqpoint{7.750000in}{3.850000in}}%
\pgfusepath{clip}%
\pgfsetbuttcap%
\pgfsetmiterjoin%
\definecolor{currentfill}{rgb}{0.501961,0.501961,0.501961}%
\pgfsetfillcolor{currentfill}%
\pgfsetfillopacity{0.700000}%
\pgfsetlinewidth{0.000000pt}%
\definecolor{currentstroke}{rgb}{0.000000,0.000000,0.000000}%
\pgfsetstrokecolor{currentstroke}%
\pgfsetstrokeopacity{0.700000}%
\pgfsetdash{}{0pt}%
\pgfpathmoveto{\pgfqpoint{4.216797in}{0.550000in}}%
\pgfpathlineto{\pgfqpoint{4.244318in}{0.550000in}}%
\pgfpathlineto{\pgfqpoint{4.244318in}{2.282115in}}%
\pgfpathlineto{\pgfqpoint{4.216797in}{2.282115in}}%
\pgfpathlineto{\pgfqpoint{4.216797in}{0.550000in}}%
\pgfpathclose%
\pgfusepath{fill}%
\end{pgfscope}%
\begin{pgfscope}%
\pgfpathrectangle{\pgfqpoint{1.250000in}{0.550000in}}{\pgfqpoint{7.750000in}{3.850000in}}%
\pgfusepath{clip}%
\pgfsetbuttcap%
\pgfsetmiterjoin%
\definecolor{currentfill}{rgb}{0.501961,0.501961,0.501961}%
\pgfsetfillcolor{currentfill}%
\pgfsetfillopacity{0.700000}%
\pgfsetlinewidth{0.000000pt}%
\definecolor{currentstroke}{rgb}{0.000000,0.000000,0.000000}%
\pgfsetstrokecolor{currentstroke}%
\pgfsetstrokeopacity{0.700000}%
\pgfsetdash{}{0pt}%
\pgfpathmoveto{\pgfqpoint{4.244318in}{0.550000in}}%
\pgfpathlineto{\pgfqpoint{4.271839in}{0.550000in}}%
\pgfpathlineto{\pgfqpoint{4.271839in}{0.550000in}}%
\pgfpathlineto{\pgfqpoint{4.244318in}{0.550000in}}%
\pgfpathlineto{\pgfqpoint{4.244318in}{0.550000in}}%
\pgfpathclose%
\pgfusepath{fill}%
\end{pgfscope}%
\begin{pgfscope}%
\pgfpathrectangle{\pgfqpoint{1.250000in}{0.550000in}}{\pgfqpoint{7.750000in}{3.850000in}}%
\pgfusepath{clip}%
\pgfsetbuttcap%
\pgfsetmiterjoin%
\definecolor{currentfill}{rgb}{0.501961,0.501961,0.501961}%
\pgfsetfillcolor{currentfill}%
\pgfsetfillopacity{0.700000}%
\pgfsetlinewidth{0.000000pt}%
\definecolor{currentstroke}{rgb}{0.000000,0.000000,0.000000}%
\pgfsetstrokecolor{currentstroke}%
\pgfsetstrokeopacity{0.700000}%
\pgfsetdash{}{0pt}%
\pgfpathmoveto{\pgfqpoint{4.271839in}{0.550000in}}%
\pgfpathlineto{\pgfqpoint{4.299361in}{0.550000in}}%
\pgfpathlineto{\pgfqpoint{4.299361in}{2.289243in}}%
\pgfpathlineto{\pgfqpoint{4.271839in}{2.289243in}}%
\pgfpathlineto{\pgfqpoint{4.271839in}{0.550000in}}%
\pgfpathclose%
\pgfusepath{fill}%
\end{pgfscope}%
\begin{pgfscope}%
\pgfpathrectangle{\pgfqpoint{1.250000in}{0.550000in}}{\pgfqpoint{7.750000in}{3.850000in}}%
\pgfusepath{clip}%
\pgfsetbuttcap%
\pgfsetmiterjoin%
\definecolor{currentfill}{rgb}{0.501961,0.501961,0.501961}%
\pgfsetfillcolor{currentfill}%
\pgfsetfillopacity{0.700000}%
\pgfsetlinewidth{0.000000pt}%
\definecolor{currentstroke}{rgb}{0.000000,0.000000,0.000000}%
\pgfsetstrokecolor{currentstroke}%
\pgfsetstrokeopacity{0.700000}%
\pgfsetdash{}{0pt}%
\pgfpathmoveto{\pgfqpoint{4.299361in}{0.550000in}}%
\pgfpathlineto{\pgfqpoint{4.326882in}{0.550000in}}%
\pgfpathlineto{\pgfqpoint{4.326882in}{0.550000in}}%
\pgfpathlineto{\pgfqpoint{4.299361in}{0.550000in}}%
\pgfpathlineto{\pgfqpoint{4.299361in}{0.550000in}}%
\pgfpathclose%
\pgfusepath{fill}%
\end{pgfscope}%
\begin{pgfscope}%
\pgfpathrectangle{\pgfqpoint{1.250000in}{0.550000in}}{\pgfqpoint{7.750000in}{3.850000in}}%
\pgfusepath{clip}%
\pgfsetbuttcap%
\pgfsetmiterjoin%
\definecolor{currentfill}{rgb}{0.501961,0.501961,0.501961}%
\pgfsetfillcolor{currentfill}%
\pgfsetfillopacity{0.700000}%
\pgfsetlinewidth{0.000000pt}%
\definecolor{currentstroke}{rgb}{0.000000,0.000000,0.000000}%
\pgfsetstrokecolor{currentstroke}%
\pgfsetstrokeopacity{0.700000}%
\pgfsetdash{}{0pt}%
\pgfpathmoveto{\pgfqpoint{4.326882in}{0.550000in}}%
\pgfpathlineto{\pgfqpoint{4.354403in}{0.550000in}}%
\pgfpathlineto{\pgfqpoint{4.354403in}{2.129575in}}%
\pgfpathlineto{\pgfqpoint{4.326882in}{2.129575in}}%
\pgfpathlineto{\pgfqpoint{4.326882in}{0.550000in}}%
\pgfpathclose%
\pgfusepath{fill}%
\end{pgfscope}%
\begin{pgfscope}%
\pgfpathrectangle{\pgfqpoint{1.250000in}{0.550000in}}{\pgfqpoint{7.750000in}{3.850000in}}%
\pgfusepath{clip}%
\pgfsetbuttcap%
\pgfsetmiterjoin%
\definecolor{currentfill}{rgb}{0.501961,0.501961,0.501961}%
\pgfsetfillcolor{currentfill}%
\pgfsetfillopacity{0.700000}%
\pgfsetlinewidth{0.000000pt}%
\definecolor{currentstroke}{rgb}{0.000000,0.000000,0.000000}%
\pgfsetstrokecolor{currentstroke}%
\pgfsetstrokeopacity{0.700000}%
\pgfsetdash{}{0pt}%
\pgfpathmoveto{\pgfqpoint{4.354403in}{0.550000in}}%
\pgfpathlineto{\pgfqpoint{4.381925in}{0.550000in}}%
\pgfpathlineto{\pgfqpoint{4.381925in}{0.550000in}}%
\pgfpathlineto{\pgfqpoint{4.354403in}{0.550000in}}%
\pgfpathlineto{\pgfqpoint{4.354403in}{0.550000in}}%
\pgfpathclose%
\pgfusepath{fill}%
\end{pgfscope}%
\begin{pgfscope}%
\pgfpathrectangle{\pgfqpoint{1.250000in}{0.550000in}}{\pgfqpoint{7.750000in}{3.850000in}}%
\pgfusepath{clip}%
\pgfsetbuttcap%
\pgfsetmiterjoin%
\definecolor{currentfill}{rgb}{0.501961,0.501961,0.501961}%
\pgfsetfillcolor{currentfill}%
\pgfsetfillopacity{0.700000}%
\pgfsetlinewidth{0.000000pt}%
\definecolor{currentstroke}{rgb}{0.000000,0.000000,0.000000}%
\pgfsetstrokecolor{currentstroke}%
\pgfsetstrokeopacity{0.700000}%
\pgfsetdash{}{0pt}%
\pgfpathmoveto{\pgfqpoint{4.381925in}{0.550000in}}%
\pgfpathlineto{\pgfqpoint{4.409446in}{0.550000in}}%
\pgfpathlineto{\pgfqpoint{4.409446in}{2.280689in}}%
\pgfpathlineto{\pgfqpoint{4.381925in}{2.280689in}}%
\pgfpathlineto{\pgfqpoint{4.381925in}{0.550000in}}%
\pgfpathclose%
\pgfusepath{fill}%
\end{pgfscope}%
\begin{pgfscope}%
\pgfpathrectangle{\pgfqpoint{1.250000in}{0.550000in}}{\pgfqpoint{7.750000in}{3.850000in}}%
\pgfusepath{clip}%
\pgfsetbuttcap%
\pgfsetmiterjoin%
\definecolor{currentfill}{rgb}{0.501961,0.501961,0.501961}%
\pgfsetfillcolor{currentfill}%
\pgfsetfillopacity{0.700000}%
\pgfsetlinewidth{0.000000pt}%
\definecolor{currentstroke}{rgb}{0.000000,0.000000,0.000000}%
\pgfsetstrokecolor{currentstroke}%
\pgfsetstrokeopacity{0.700000}%
\pgfsetdash{}{0pt}%
\pgfpathmoveto{\pgfqpoint{4.409446in}{0.550000in}}%
\pgfpathlineto{\pgfqpoint{4.436967in}{0.550000in}}%
\pgfpathlineto{\pgfqpoint{4.436967in}{0.550000in}}%
\pgfpathlineto{\pgfqpoint{4.409446in}{0.550000in}}%
\pgfpathlineto{\pgfqpoint{4.409446in}{0.550000in}}%
\pgfpathclose%
\pgfusepath{fill}%
\end{pgfscope}%
\begin{pgfscope}%
\pgfpathrectangle{\pgfqpoint{1.250000in}{0.550000in}}{\pgfqpoint{7.750000in}{3.850000in}}%
\pgfusepath{clip}%
\pgfsetbuttcap%
\pgfsetmiterjoin%
\definecolor{currentfill}{rgb}{0.501961,0.501961,0.501961}%
\pgfsetfillcolor{currentfill}%
\pgfsetfillopacity{0.700000}%
\pgfsetlinewidth{0.000000pt}%
\definecolor{currentstroke}{rgb}{0.000000,0.000000,0.000000}%
\pgfsetstrokecolor{currentstroke}%
\pgfsetstrokeopacity{0.700000}%
\pgfsetdash{}{0pt}%
\pgfpathmoveto{\pgfqpoint{4.436967in}{0.550000in}}%
\pgfpathlineto{\pgfqpoint{4.464489in}{0.550000in}}%
\pgfpathlineto{\pgfqpoint{4.464489in}{2.073976in}}%
\pgfpathlineto{\pgfqpoint{4.436967in}{2.073976in}}%
\pgfpathlineto{\pgfqpoint{4.436967in}{0.550000in}}%
\pgfpathclose%
\pgfusepath{fill}%
\end{pgfscope}%
\begin{pgfscope}%
\pgfpathrectangle{\pgfqpoint{1.250000in}{0.550000in}}{\pgfqpoint{7.750000in}{3.850000in}}%
\pgfusepath{clip}%
\pgfsetbuttcap%
\pgfsetmiterjoin%
\definecolor{currentfill}{rgb}{0.501961,0.501961,0.501961}%
\pgfsetfillcolor{currentfill}%
\pgfsetfillopacity{0.700000}%
\pgfsetlinewidth{0.000000pt}%
\definecolor{currentstroke}{rgb}{0.000000,0.000000,0.000000}%
\pgfsetstrokecolor{currentstroke}%
\pgfsetstrokeopacity{0.700000}%
\pgfsetdash{}{0pt}%
\pgfpathmoveto{\pgfqpoint{4.464489in}{0.550000in}}%
\pgfpathlineto{\pgfqpoint{4.492010in}{0.550000in}}%
\pgfpathlineto{\pgfqpoint{4.492010in}{0.550000in}}%
\pgfpathlineto{\pgfqpoint{4.464489in}{0.550000in}}%
\pgfpathlineto{\pgfqpoint{4.464489in}{0.550000in}}%
\pgfpathclose%
\pgfusepath{fill}%
\end{pgfscope}%
\begin{pgfscope}%
\pgfpathrectangle{\pgfqpoint{1.250000in}{0.550000in}}{\pgfqpoint{7.750000in}{3.850000in}}%
\pgfusepath{clip}%
\pgfsetbuttcap%
\pgfsetmiterjoin%
\definecolor{currentfill}{rgb}{0.501961,0.501961,0.501961}%
\pgfsetfillcolor{currentfill}%
\pgfsetfillopacity{0.700000}%
\pgfsetlinewidth{0.000000pt}%
\definecolor{currentstroke}{rgb}{0.000000,0.000000,0.000000}%
\pgfsetstrokecolor{currentstroke}%
\pgfsetstrokeopacity{0.700000}%
\pgfsetdash{}{0pt}%
\pgfpathmoveto{\pgfqpoint{4.492010in}{0.550000in}}%
\pgfpathlineto{\pgfqpoint{4.519531in}{0.550000in}}%
\pgfpathlineto{\pgfqpoint{4.519531in}{2.142405in}}%
\pgfpathlineto{\pgfqpoint{4.492010in}{2.142405in}}%
\pgfpathlineto{\pgfqpoint{4.492010in}{0.550000in}}%
\pgfpathclose%
\pgfusepath{fill}%
\end{pgfscope}%
\begin{pgfscope}%
\pgfpathrectangle{\pgfqpoint{1.250000in}{0.550000in}}{\pgfqpoint{7.750000in}{3.850000in}}%
\pgfusepath{clip}%
\pgfsetbuttcap%
\pgfsetmiterjoin%
\definecolor{currentfill}{rgb}{0.501961,0.501961,0.501961}%
\pgfsetfillcolor{currentfill}%
\pgfsetfillopacity{0.700000}%
\pgfsetlinewidth{0.000000pt}%
\definecolor{currentstroke}{rgb}{0.000000,0.000000,0.000000}%
\pgfsetstrokecolor{currentstroke}%
\pgfsetstrokeopacity{0.700000}%
\pgfsetdash{}{0pt}%
\pgfpathmoveto{\pgfqpoint{4.519531in}{0.550000in}}%
\pgfpathlineto{\pgfqpoint{4.547053in}{0.550000in}}%
\pgfpathlineto{\pgfqpoint{4.547053in}{0.550000in}}%
\pgfpathlineto{\pgfqpoint{4.519531in}{0.550000in}}%
\pgfpathlineto{\pgfqpoint{4.519531in}{0.550000in}}%
\pgfpathclose%
\pgfusepath{fill}%
\end{pgfscope}%
\begin{pgfscope}%
\pgfpathrectangle{\pgfqpoint{1.250000in}{0.550000in}}{\pgfqpoint{7.750000in}{3.850000in}}%
\pgfusepath{clip}%
\pgfsetbuttcap%
\pgfsetmiterjoin%
\definecolor{currentfill}{rgb}{0.501961,0.501961,0.501961}%
\pgfsetfillcolor{currentfill}%
\pgfsetfillopacity{0.700000}%
\pgfsetlinewidth{0.000000pt}%
\definecolor{currentstroke}{rgb}{0.000000,0.000000,0.000000}%
\pgfsetstrokecolor{currentstroke}%
\pgfsetstrokeopacity{0.700000}%
\pgfsetdash{}{0pt}%
\pgfpathmoveto{\pgfqpoint{4.547053in}{0.550000in}}%
\pgfpathlineto{\pgfqpoint{4.574574in}{0.550000in}}%
\pgfpathlineto{\pgfqpoint{4.574574in}{2.086807in}}%
\pgfpathlineto{\pgfqpoint{4.547053in}{2.086807in}}%
\pgfpathlineto{\pgfqpoint{4.547053in}{0.550000in}}%
\pgfpathclose%
\pgfusepath{fill}%
\end{pgfscope}%
\begin{pgfscope}%
\pgfpathrectangle{\pgfqpoint{1.250000in}{0.550000in}}{\pgfqpoint{7.750000in}{3.850000in}}%
\pgfusepath{clip}%
\pgfsetbuttcap%
\pgfsetmiterjoin%
\definecolor{currentfill}{rgb}{0.501961,0.501961,0.501961}%
\pgfsetfillcolor{currentfill}%
\pgfsetfillopacity{0.700000}%
\pgfsetlinewidth{0.000000pt}%
\definecolor{currentstroke}{rgb}{0.000000,0.000000,0.000000}%
\pgfsetstrokecolor{currentstroke}%
\pgfsetstrokeopacity{0.700000}%
\pgfsetdash{}{0pt}%
\pgfpathmoveto{\pgfqpoint{4.574574in}{0.550000in}}%
\pgfpathlineto{\pgfqpoint{4.602095in}{0.550000in}}%
\pgfpathlineto{\pgfqpoint{4.602095in}{0.550000in}}%
\pgfpathlineto{\pgfqpoint{4.574574in}{0.550000in}}%
\pgfpathlineto{\pgfqpoint{4.574574in}{0.550000in}}%
\pgfpathclose%
\pgfusepath{fill}%
\end{pgfscope}%
\begin{pgfscope}%
\pgfpathrectangle{\pgfqpoint{1.250000in}{0.550000in}}{\pgfqpoint{7.750000in}{3.850000in}}%
\pgfusepath{clip}%
\pgfsetbuttcap%
\pgfsetmiterjoin%
\definecolor{currentfill}{rgb}{0.501961,0.501961,0.501961}%
\pgfsetfillcolor{currentfill}%
\pgfsetfillopacity{0.700000}%
\pgfsetlinewidth{0.000000pt}%
\definecolor{currentstroke}{rgb}{0.000000,0.000000,0.000000}%
\pgfsetstrokecolor{currentstroke}%
\pgfsetstrokeopacity{0.700000}%
\pgfsetdash{}{0pt}%
\pgfpathmoveto{\pgfqpoint{4.602095in}{0.550000in}}%
\pgfpathlineto{\pgfqpoint{4.629616in}{0.550000in}}%
\pgfpathlineto{\pgfqpoint{4.629616in}{2.039762in}}%
\pgfpathlineto{\pgfqpoint{4.602095in}{2.039762in}}%
\pgfpathlineto{\pgfqpoint{4.602095in}{0.550000in}}%
\pgfpathclose%
\pgfusepath{fill}%
\end{pgfscope}%
\begin{pgfscope}%
\pgfpathrectangle{\pgfqpoint{1.250000in}{0.550000in}}{\pgfqpoint{7.750000in}{3.850000in}}%
\pgfusepath{clip}%
\pgfsetbuttcap%
\pgfsetmiterjoin%
\definecolor{currentfill}{rgb}{0.501961,0.501961,0.501961}%
\pgfsetfillcolor{currentfill}%
\pgfsetfillopacity{0.700000}%
\pgfsetlinewidth{0.000000pt}%
\definecolor{currentstroke}{rgb}{0.000000,0.000000,0.000000}%
\pgfsetstrokecolor{currentstroke}%
\pgfsetstrokeopacity{0.700000}%
\pgfsetdash{}{0pt}%
\pgfpathmoveto{\pgfqpoint{4.629616in}{0.550000in}}%
\pgfpathlineto{\pgfqpoint{4.657138in}{0.550000in}}%
\pgfpathlineto{\pgfqpoint{4.657138in}{0.550000in}}%
\pgfpathlineto{\pgfqpoint{4.629616in}{0.550000in}}%
\pgfpathlineto{\pgfqpoint{4.629616in}{0.550000in}}%
\pgfpathclose%
\pgfusepath{fill}%
\end{pgfscope}%
\begin{pgfscope}%
\pgfpathrectangle{\pgfqpoint{1.250000in}{0.550000in}}{\pgfqpoint{7.750000in}{3.850000in}}%
\pgfusepath{clip}%
\pgfsetbuttcap%
\pgfsetmiterjoin%
\definecolor{currentfill}{rgb}{0.501961,0.501961,0.501961}%
\pgfsetfillcolor{currentfill}%
\pgfsetfillopacity{0.700000}%
\pgfsetlinewidth{0.000000pt}%
\definecolor{currentstroke}{rgb}{0.000000,0.000000,0.000000}%
\pgfsetstrokecolor{currentstroke}%
\pgfsetstrokeopacity{0.700000}%
\pgfsetdash{}{0pt}%
\pgfpathmoveto{\pgfqpoint{4.657138in}{0.550000in}}%
\pgfpathlineto{\pgfqpoint{4.684659in}{0.550000in}}%
\pgfpathlineto{\pgfqpoint{4.684659in}{2.078253in}}%
\pgfpathlineto{\pgfqpoint{4.657138in}{2.078253in}}%
\pgfpathlineto{\pgfqpoint{4.657138in}{0.550000in}}%
\pgfpathclose%
\pgfusepath{fill}%
\end{pgfscope}%
\begin{pgfscope}%
\pgfpathrectangle{\pgfqpoint{1.250000in}{0.550000in}}{\pgfqpoint{7.750000in}{3.850000in}}%
\pgfusepath{clip}%
\pgfsetbuttcap%
\pgfsetmiterjoin%
\definecolor{currentfill}{rgb}{0.501961,0.501961,0.501961}%
\pgfsetfillcolor{currentfill}%
\pgfsetfillopacity{0.700000}%
\pgfsetlinewidth{0.000000pt}%
\definecolor{currentstroke}{rgb}{0.000000,0.000000,0.000000}%
\pgfsetstrokecolor{currentstroke}%
\pgfsetstrokeopacity{0.700000}%
\pgfsetdash{}{0pt}%
\pgfpathmoveto{\pgfqpoint{4.684659in}{0.550000in}}%
\pgfpathlineto{\pgfqpoint{4.712180in}{0.550000in}}%
\pgfpathlineto{\pgfqpoint{4.712180in}{0.550000in}}%
\pgfpathlineto{\pgfqpoint{4.684659in}{0.550000in}}%
\pgfpathlineto{\pgfqpoint{4.684659in}{0.550000in}}%
\pgfpathclose%
\pgfusepath{fill}%
\end{pgfscope}%
\begin{pgfscope}%
\pgfpathrectangle{\pgfqpoint{1.250000in}{0.550000in}}{\pgfqpoint{7.750000in}{3.850000in}}%
\pgfusepath{clip}%
\pgfsetbuttcap%
\pgfsetmiterjoin%
\definecolor{currentfill}{rgb}{0.501961,0.501961,0.501961}%
\pgfsetfillcolor{currentfill}%
\pgfsetfillopacity{0.700000}%
\pgfsetlinewidth{0.000000pt}%
\definecolor{currentstroke}{rgb}{0.000000,0.000000,0.000000}%
\pgfsetstrokecolor{currentstroke}%
\pgfsetstrokeopacity{0.700000}%
\pgfsetdash{}{0pt}%
\pgfpathmoveto{\pgfqpoint{4.712180in}{0.550000in}}%
\pgfpathlineto{\pgfqpoint{4.739702in}{0.550000in}}%
\pgfpathlineto{\pgfqpoint{4.739702in}{2.098212in}}%
\pgfpathlineto{\pgfqpoint{4.712180in}{2.098212in}}%
\pgfpathlineto{\pgfqpoint{4.712180in}{0.550000in}}%
\pgfpathclose%
\pgfusepath{fill}%
\end{pgfscope}%
\begin{pgfscope}%
\pgfpathrectangle{\pgfqpoint{1.250000in}{0.550000in}}{\pgfqpoint{7.750000in}{3.850000in}}%
\pgfusepath{clip}%
\pgfsetbuttcap%
\pgfsetmiterjoin%
\definecolor{currentfill}{rgb}{0.501961,0.501961,0.501961}%
\pgfsetfillcolor{currentfill}%
\pgfsetfillopacity{0.700000}%
\pgfsetlinewidth{0.000000pt}%
\definecolor{currentstroke}{rgb}{0.000000,0.000000,0.000000}%
\pgfsetstrokecolor{currentstroke}%
\pgfsetstrokeopacity{0.700000}%
\pgfsetdash{}{0pt}%
\pgfpathmoveto{\pgfqpoint{4.739702in}{0.550000in}}%
\pgfpathlineto{\pgfqpoint{4.767223in}{0.550000in}}%
\pgfpathlineto{\pgfqpoint{4.767223in}{0.550000in}}%
\pgfpathlineto{\pgfqpoint{4.739702in}{0.550000in}}%
\pgfpathlineto{\pgfqpoint{4.739702in}{0.550000in}}%
\pgfpathclose%
\pgfusepath{fill}%
\end{pgfscope}%
\begin{pgfscope}%
\pgfpathrectangle{\pgfqpoint{1.250000in}{0.550000in}}{\pgfqpoint{7.750000in}{3.850000in}}%
\pgfusepath{clip}%
\pgfsetbuttcap%
\pgfsetmiterjoin%
\definecolor{currentfill}{rgb}{0.501961,0.501961,0.501961}%
\pgfsetfillcolor{currentfill}%
\pgfsetfillopacity{0.700000}%
\pgfsetlinewidth{0.000000pt}%
\definecolor{currentstroke}{rgb}{0.000000,0.000000,0.000000}%
\pgfsetstrokecolor{currentstroke}%
\pgfsetstrokeopacity{0.700000}%
\pgfsetdash{}{0pt}%
\pgfpathmoveto{\pgfqpoint{4.767223in}{0.550000in}}%
\pgfpathlineto{\pgfqpoint{4.794744in}{0.550000in}}%
\pgfpathlineto{\pgfqpoint{4.794744in}{1.937118in}}%
\pgfpathlineto{\pgfqpoint{4.767223in}{1.937118in}}%
\pgfpathlineto{\pgfqpoint{4.767223in}{0.550000in}}%
\pgfpathclose%
\pgfusepath{fill}%
\end{pgfscope}%
\begin{pgfscope}%
\pgfpathrectangle{\pgfqpoint{1.250000in}{0.550000in}}{\pgfqpoint{7.750000in}{3.850000in}}%
\pgfusepath{clip}%
\pgfsetbuttcap%
\pgfsetmiterjoin%
\definecolor{currentfill}{rgb}{0.501961,0.501961,0.501961}%
\pgfsetfillcolor{currentfill}%
\pgfsetfillopacity{0.700000}%
\pgfsetlinewidth{0.000000pt}%
\definecolor{currentstroke}{rgb}{0.000000,0.000000,0.000000}%
\pgfsetstrokecolor{currentstroke}%
\pgfsetstrokeopacity{0.700000}%
\pgfsetdash{}{0pt}%
\pgfpathmoveto{\pgfqpoint{4.794744in}{0.550000in}}%
\pgfpathlineto{\pgfqpoint{4.822266in}{0.550000in}}%
\pgfpathlineto{\pgfqpoint{4.822266in}{0.550000in}}%
\pgfpathlineto{\pgfqpoint{4.794744in}{0.550000in}}%
\pgfpathlineto{\pgfqpoint{4.794744in}{0.550000in}}%
\pgfpathclose%
\pgfusepath{fill}%
\end{pgfscope}%
\begin{pgfscope}%
\pgfpathrectangle{\pgfqpoint{1.250000in}{0.550000in}}{\pgfqpoint{7.750000in}{3.850000in}}%
\pgfusepath{clip}%
\pgfsetbuttcap%
\pgfsetmiterjoin%
\definecolor{currentfill}{rgb}{0.501961,0.501961,0.501961}%
\pgfsetfillcolor{currentfill}%
\pgfsetfillopacity{0.700000}%
\pgfsetlinewidth{0.000000pt}%
\definecolor{currentstroke}{rgb}{0.000000,0.000000,0.000000}%
\pgfsetstrokecolor{currentstroke}%
\pgfsetstrokeopacity{0.700000}%
\pgfsetdash{}{0pt}%
\pgfpathmoveto{\pgfqpoint{4.822266in}{0.550000in}}%
\pgfpathlineto{\pgfqpoint{4.849787in}{0.550000in}}%
\pgfpathlineto{\pgfqpoint{4.849787in}{1.928564in}}%
\pgfpathlineto{\pgfqpoint{4.822266in}{1.928564in}}%
\pgfpathlineto{\pgfqpoint{4.822266in}{0.550000in}}%
\pgfpathclose%
\pgfusepath{fill}%
\end{pgfscope}%
\begin{pgfscope}%
\pgfpathrectangle{\pgfqpoint{1.250000in}{0.550000in}}{\pgfqpoint{7.750000in}{3.850000in}}%
\pgfusepath{clip}%
\pgfsetbuttcap%
\pgfsetmiterjoin%
\definecolor{currentfill}{rgb}{0.501961,0.501961,0.501961}%
\pgfsetfillcolor{currentfill}%
\pgfsetfillopacity{0.700000}%
\pgfsetlinewidth{0.000000pt}%
\definecolor{currentstroke}{rgb}{0.000000,0.000000,0.000000}%
\pgfsetstrokecolor{currentstroke}%
\pgfsetstrokeopacity{0.700000}%
\pgfsetdash{}{0pt}%
\pgfpathmoveto{\pgfqpoint{4.849787in}{0.550000in}}%
\pgfpathlineto{\pgfqpoint{4.877308in}{0.550000in}}%
\pgfpathlineto{\pgfqpoint{4.877308in}{0.550000in}}%
\pgfpathlineto{\pgfqpoint{4.849787in}{0.550000in}}%
\pgfpathlineto{\pgfqpoint{4.849787in}{0.550000in}}%
\pgfpathclose%
\pgfusepath{fill}%
\end{pgfscope}%
\begin{pgfscope}%
\pgfpathrectangle{\pgfqpoint{1.250000in}{0.550000in}}{\pgfqpoint{7.750000in}{3.850000in}}%
\pgfusepath{clip}%
\pgfsetbuttcap%
\pgfsetmiterjoin%
\definecolor{currentfill}{rgb}{0.501961,0.501961,0.501961}%
\pgfsetfillcolor{currentfill}%
\pgfsetfillopacity{0.700000}%
\pgfsetlinewidth{0.000000pt}%
\definecolor{currentstroke}{rgb}{0.000000,0.000000,0.000000}%
\pgfsetstrokecolor{currentstroke}%
\pgfsetstrokeopacity{0.700000}%
\pgfsetdash{}{0pt}%
\pgfpathmoveto{\pgfqpoint{4.877308in}{0.550000in}}%
\pgfpathlineto{\pgfqpoint{4.904830in}{0.550000in}}%
\pgfpathlineto{\pgfqpoint{4.904830in}{1.942820in}}%
\pgfpathlineto{\pgfqpoint{4.877308in}{1.942820in}}%
\pgfpathlineto{\pgfqpoint{4.877308in}{0.550000in}}%
\pgfpathclose%
\pgfusepath{fill}%
\end{pgfscope}%
\begin{pgfscope}%
\pgfpathrectangle{\pgfqpoint{1.250000in}{0.550000in}}{\pgfqpoint{7.750000in}{3.850000in}}%
\pgfusepath{clip}%
\pgfsetbuttcap%
\pgfsetmiterjoin%
\definecolor{currentfill}{rgb}{0.501961,0.501961,0.501961}%
\pgfsetfillcolor{currentfill}%
\pgfsetfillopacity{0.700000}%
\pgfsetlinewidth{0.000000pt}%
\definecolor{currentstroke}{rgb}{0.000000,0.000000,0.000000}%
\pgfsetstrokecolor{currentstroke}%
\pgfsetstrokeopacity{0.700000}%
\pgfsetdash{}{0pt}%
\pgfpathmoveto{\pgfqpoint{4.904830in}{0.550000in}}%
\pgfpathlineto{\pgfqpoint{4.932351in}{0.550000in}}%
\pgfpathlineto{\pgfqpoint{4.932351in}{0.550000in}}%
\pgfpathlineto{\pgfqpoint{4.904830in}{0.550000in}}%
\pgfpathlineto{\pgfqpoint{4.904830in}{0.550000in}}%
\pgfpathclose%
\pgfusepath{fill}%
\end{pgfscope}%
\begin{pgfscope}%
\pgfpathrectangle{\pgfqpoint{1.250000in}{0.550000in}}{\pgfqpoint{7.750000in}{3.850000in}}%
\pgfusepath{clip}%
\pgfsetbuttcap%
\pgfsetmiterjoin%
\definecolor{currentfill}{rgb}{0.501961,0.501961,0.501961}%
\pgfsetfillcolor{currentfill}%
\pgfsetfillopacity{0.700000}%
\pgfsetlinewidth{0.000000pt}%
\definecolor{currentstroke}{rgb}{0.000000,0.000000,0.000000}%
\pgfsetstrokecolor{currentstroke}%
\pgfsetstrokeopacity{0.700000}%
\pgfsetdash{}{0pt}%
\pgfpathmoveto{\pgfqpoint{4.932351in}{0.550000in}}%
\pgfpathlineto{\pgfqpoint{4.959872in}{0.550000in}}%
\pgfpathlineto{\pgfqpoint{4.959872in}{1.854432in}}%
\pgfpathlineto{\pgfqpoint{4.932351in}{1.854432in}}%
\pgfpathlineto{\pgfqpoint{4.932351in}{0.550000in}}%
\pgfpathclose%
\pgfusepath{fill}%
\end{pgfscope}%
\begin{pgfscope}%
\pgfpathrectangle{\pgfqpoint{1.250000in}{0.550000in}}{\pgfqpoint{7.750000in}{3.850000in}}%
\pgfusepath{clip}%
\pgfsetbuttcap%
\pgfsetmiterjoin%
\definecolor{currentfill}{rgb}{0.501961,0.501961,0.501961}%
\pgfsetfillcolor{currentfill}%
\pgfsetfillopacity{0.700000}%
\pgfsetlinewidth{0.000000pt}%
\definecolor{currentstroke}{rgb}{0.000000,0.000000,0.000000}%
\pgfsetstrokecolor{currentstroke}%
\pgfsetstrokeopacity{0.700000}%
\pgfsetdash{}{0pt}%
\pgfpathmoveto{\pgfqpoint{4.959872in}{0.550000in}}%
\pgfpathlineto{\pgfqpoint{4.987393in}{0.550000in}}%
\pgfpathlineto{\pgfqpoint{4.987393in}{0.550000in}}%
\pgfpathlineto{\pgfqpoint{4.959872in}{0.550000in}}%
\pgfpathlineto{\pgfqpoint{4.959872in}{0.550000in}}%
\pgfpathclose%
\pgfusepath{fill}%
\end{pgfscope}%
\begin{pgfscope}%
\pgfpathrectangle{\pgfqpoint{1.250000in}{0.550000in}}{\pgfqpoint{7.750000in}{3.850000in}}%
\pgfusepath{clip}%
\pgfsetbuttcap%
\pgfsetmiterjoin%
\definecolor{currentfill}{rgb}{0.501961,0.501961,0.501961}%
\pgfsetfillcolor{currentfill}%
\pgfsetfillopacity{0.700000}%
\pgfsetlinewidth{0.000000pt}%
\definecolor{currentstroke}{rgb}{0.000000,0.000000,0.000000}%
\pgfsetstrokecolor{currentstroke}%
\pgfsetstrokeopacity{0.700000}%
\pgfsetdash{}{0pt}%
\pgfpathmoveto{\pgfqpoint{4.987393in}{0.550000in}}%
\pgfpathlineto{\pgfqpoint{5.014915in}{0.550000in}}%
\pgfpathlineto{\pgfqpoint{5.014915in}{1.827346in}}%
\pgfpathlineto{\pgfqpoint{4.987393in}{1.827346in}}%
\pgfpathlineto{\pgfqpoint{4.987393in}{0.550000in}}%
\pgfpathclose%
\pgfusepath{fill}%
\end{pgfscope}%
\begin{pgfscope}%
\pgfpathrectangle{\pgfqpoint{1.250000in}{0.550000in}}{\pgfqpoint{7.750000in}{3.850000in}}%
\pgfusepath{clip}%
\pgfsetbuttcap%
\pgfsetmiterjoin%
\definecolor{currentfill}{rgb}{0.501961,0.501961,0.501961}%
\pgfsetfillcolor{currentfill}%
\pgfsetfillopacity{0.700000}%
\pgfsetlinewidth{0.000000pt}%
\definecolor{currentstroke}{rgb}{0.000000,0.000000,0.000000}%
\pgfsetstrokecolor{currentstroke}%
\pgfsetstrokeopacity{0.700000}%
\pgfsetdash{}{0pt}%
\pgfpathmoveto{\pgfqpoint{5.014915in}{0.550000in}}%
\pgfpathlineto{\pgfqpoint{5.042436in}{0.550000in}}%
\pgfpathlineto{\pgfqpoint{5.042436in}{0.550000in}}%
\pgfpathlineto{\pgfqpoint{5.014915in}{0.550000in}}%
\pgfpathlineto{\pgfqpoint{5.014915in}{0.550000in}}%
\pgfpathclose%
\pgfusepath{fill}%
\end{pgfscope}%
\begin{pgfscope}%
\pgfpathrectangle{\pgfqpoint{1.250000in}{0.550000in}}{\pgfqpoint{7.750000in}{3.850000in}}%
\pgfusepath{clip}%
\pgfsetbuttcap%
\pgfsetmiterjoin%
\definecolor{currentfill}{rgb}{0.501961,0.501961,0.501961}%
\pgfsetfillcolor{currentfill}%
\pgfsetfillopacity{0.700000}%
\pgfsetlinewidth{0.000000pt}%
\definecolor{currentstroke}{rgb}{0.000000,0.000000,0.000000}%
\pgfsetstrokecolor{currentstroke}%
\pgfsetstrokeopacity{0.700000}%
\pgfsetdash{}{0pt}%
\pgfpathmoveto{\pgfqpoint{5.042436in}{0.550000in}}%
\pgfpathlineto{\pgfqpoint{5.069957in}{0.550000in}}%
\pgfpathlineto{\pgfqpoint{5.069957in}{1.766045in}}%
\pgfpathlineto{\pgfqpoint{5.042436in}{1.766045in}}%
\pgfpathlineto{\pgfqpoint{5.042436in}{0.550000in}}%
\pgfpathclose%
\pgfusepath{fill}%
\end{pgfscope}%
\begin{pgfscope}%
\pgfpathrectangle{\pgfqpoint{1.250000in}{0.550000in}}{\pgfqpoint{7.750000in}{3.850000in}}%
\pgfusepath{clip}%
\pgfsetbuttcap%
\pgfsetmiterjoin%
\definecolor{currentfill}{rgb}{0.501961,0.501961,0.501961}%
\pgfsetfillcolor{currentfill}%
\pgfsetfillopacity{0.700000}%
\pgfsetlinewidth{0.000000pt}%
\definecolor{currentstroke}{rgb}{0.000000,0.000000,0.000000}%
\pgfsetstrokecolor{currentstroke}%
\pgfsetstrokeopacity{0.700000}%
\pgfsetdash{}{0pt}%
\pgfpathmoveto{\pgfqpoint{5.069957in}{0.550000in}}%
\pgfpathlineto{\pgfqpoint{5.097479in}{0.550000in}}%
\pgfpathlineto{\pgfqpoint{5.097479in}{0.550000in}}%
\pgfpathlineto{\pgfqpoint{5.069957in}{0.550000in}}%
\pgfpathlineto{\pgfqpoint{5.069957in}{0.550000in}}%
\pgfpathclose%
\pgfusepath{fill}%
\end{pgfscope}%
\begin{pgfscope}%
\pgfpathrectangle{\pgfqpoint{1.250000in}{0.550000in}}{\pgfqpoint{7.750000in}{3.850000in}}%
\pgfusepath{clip}%
\pgfsetbuttcap%
\pgfsetmiterjoin%
\definecolor{currentfill}{rgb}{0.501961,0.501961,0.501961}%
\pgfsetfillcolor{currentfill}%
\pgfsetfillopacity{0.700000}%
\pgfsetlinewidth{0.000000pt}%
\definecolor{currentstroke}{rgb}{0.000000,0.000000,0.000000}%
\pgfsetstrokecolor{currentstroke}%
\pgfsetstrokeopacity{0.700000}%
\pgfsetdash{}{0pt}%
\pgfpathmoveto{\pgfqpoint{5.097479in}{0.550000in}}%
\pgfpathlineto{\pgfqpoint{5.125000in}{0.550000in}}%
\pgfpathlineto{\pgfqpoint{5.125000in}{1.784578in}}%
\pgfpathlineto{\pgfqpoint{5.097479in}{1.784578in}}%
\pgfpathlineto{\pgfqpoint{5.097479in}{0.550000in}}%
\pgfpathclose%
\pgfusepath{fill}%
\end{pgfscope}%
\begin{pgfscope}%
\pgfpathrectangle{\pgfqpoint{1.250000in}{0.550000in}}{\pgfqpoint{7.750000in}{3.850000in}}%
\pgfusepath{clip}%
\pgfsetbuttcap%
\pgfsetmiterjoin%
\definecolor{currentfill}{rgb}{0.501961,0.501961,0.501961}%
\pgfsetfillcolor{currentfill}%
\pgfsetfillopacity{0.700000}%
\pgfsetlinewidth{0.000000pt}%
\definecolor{currentstroke}{rgb}{0.000000,0.000000,0.000000}%
\pgfsetstrokecolor{currentstroke}%
\pgfsetstrokeopacity{0.700000}%
\pgfsetdash{}{0pt}%
\pgfpathmoveto{\pgfqpoint{5.125000in}{0.550000in}}%
\pgfpathlineto{\pgfqpoint{5.152521in}{0.550000in}}%
\pgfpathlineto{\pgfqpoint{5.152521in}{0.550000in}}%
\pgfpathlineto{\pgfqpoint{5.125000in}{0.550000in}}%
\pgfpathlineto{\pgfqpoint{5.125000in}{0.550000in}}%
\pgfpathclose%
\pgfusepath{fill}%
\end{pgfscope}%
\begin{pgfscope}%
\pgfpathrectangle{\pgfqpoint{1.250000in}{0.550000in}}{\pgfqpoint{7.750000in}{3.850000in}}%
\pgfusepath{clip}%
\pgfsetbuttcap%
\pgfsetmiterjoin%
\definecolor{currentfill}{rgb}{0.501961,0.501961,0.501961}%
\pgfsetfillcolor{currentfill}%
\pgfsetfillopacity{0.700000}%
\pgfsetlinewidth{0.000000pt}%
\definecolor{currentstroke}{rgb}{0.000000,0.000000,0.000000}%
\pgfsetstrokecolor{currentstroke}%
\pgfsetstrokeopacity{0.700000}%
\pgfsetdash{}{0pt}%
\pgfpathmoveto{\pgfqpoint{5.152521in}{0.550000in}}%
\pgfpathlineto{\pgfqpoint{5.180043in}{0.550000in}}%
\pgfpathlineto{\pgfqpoint{5.180043in}{1.734681in}}%
\pgfpathlineto{\pgfqpoint{5.152521in}{1.734681in}}%
\pgfpathlineto{\pgfqpoint{5.152521in}{0.550000in}}%
\pgfpathclose%
\pgfusepath{fill}%
\end{pgfscope}%
\begin{pgfscope}%
\pgfpathrectangle{\pgfqpoint{1.250000in}{0.550000in}}{\pgfqpoint{7.750000in}{3.850000in}}%
\pgfusepath{clip}%
\pgfsetbuttcap%
\pgfsetmiterjoin%
\definecolor{currentfill}{rgb}{0.501961,0.501961,0.501961}%
\pgfsetfillcolor{currentfill}%
\pgfsetfillopacity{0.700000}%
\pgfsetlinewidth{0.000000pt}%
\definecolor{currentstroke}{rgb}{0.000000,0.000000,0.000000}%
\pgfsetstrokecolor{currentstroke}%
\pgfsetstrokeopacity{0.700000}%
\pgfsetdash{}{0pt}%
\pgfpathmoveto{\pgfqpoint{5.180043in}{0.550000in}}%
\pgfpathlineto{\pgfqpoint{5.207564in}{0.550000in}}%
\pgfpathlineto{\pgfqpoint{5.207564in}{0.550000in}}%
\pgfpathlineto{\pgfqpoint{5.180043in}{0.550000in}}%
\pgfpathlineto{\pgfqpoint{5.180043in}{0.550000in}}%
\pgfpathclose%
\pgfusepath{fill}%
\end{pgfscope}%
\begin{pgfscope}%
\pgfpathrectangle{\pgfqpoint{1.250000in}{0.550000in}}{\pgfqpoint{7.750000in}{3.850000in}}%
\pgfusepath{clip}%
\pgfsetbuttcap%
\pgfsetmiterjoin%
\definecolor{currentfill}{rgb}{0.501961,0.501961,0.501961}%
\pgfsetfillcolor{currentfill}%
\pgfsetfillopacity{0.700000}%
\pgfsetlinewidth{0.000000pt}%
\definecolor{currentstroke}{rgb}{0.000000,0.000000,0.000000}%
\pgfsetstrokecolor{currentstroke}%
\pgfsetstrokeopacity{0.700000}%
\pgfsetdash{}{0pt}%
\pgfpathmoveto{\pgfqpoint{5.207564in}{0.550000in}}%
\pgfpathlineto{\pgfqpoint{5.235085in}{0.550000in}}%
\pgfpathlineto{\pgfqpoint{5.235085in}{1.690487in}}%
\pgfpathlineto{\pgfqpoint{5.207564in}{1.690487in}}%
\pgfpathlineto{\pgfqpoint{5.207564in}{0.550000in}}%
\pgfpathclose%
\pgfusepath{fill}%
\end{pgfscope}%
\begin{pgfscope}%
\pgfpathrectangle{\pgfqpoint{1.250000in}{0.550000in}}{\pgfqpoint{7.750000in}{3.850000in}}%
\pgfusepath{clip}%
\pgfsetbuttcap%
\pgfsetmiterjoin%
\definecolor{currentfill}{rgb}{0.501961,0.501961,0.501961}%
\pgfsetfillcolor{currentfill}%
\pgfsetfillopacity{0.700000}%
\pgfsetlinewidth{0.000000pt}%
\definecolor{currentstroke}{rgb}{0.000000,0.000000,0.000000}%
\pgfsetstrokecolor{currentstroke}%
\pgfsetstrokeopacity{0.700000}%
\pgfsetdash{}{0pt}%
\pgfpathmoveto{\pgfqpoint{5.235085in}{0.550000in}}%
\pgfpathlineto{\pgfqpoint{5.262607in}{0.550000in}}%
\pgfpathlineto{\pgfqpoint{5.262607in}{0.550000in}}%
\pgfpathlineto{\pgfqpoint{5.235085in}{0.550000in}}%
\pgfpathlineto{\pgfqpoint{5.235085in}{0.550000in}}%
\pgfpathclose%
\pgfusepath{fill}%
\end{pgfscope}%
\begin{pgfscope}%
\pgfpathrectangle{\pgfqpoint{1.250000in}{0.550000in}}{\pgfqpoint{7.750000in}{3.850000in}}%
\pgfusepath{clip}%
\pgfsetbuttcap%
\pgfsetmiterjoin%
\definecolor{currentfill}{rgb}{0.501961,0.501961,0.501961}%
\pgfsetfillcolor{currentfill}%
\pgfsetfillopacity{0.700000}%
\pgfsetlinewidth{0.000000pt}%
\definecolor{currentstroke}{rgb}{0.000000,0.000000,0.000000}%
\pgfsetstrokecolor{currentstroke}%
\pgfsetstrokeopacity{0.700000}%
\pgfsetdash{}{0pt}%
\pgfpathmoveto{\pgfqpoint{5.262607in}{0.550000in}}%
\pgfpathlineto{\pgfqpoint{5.290128in}{0.550000in}}%
\pgfpathlineto{\pgfqpoint{5.290128in}{1.644868in}}%
\pgfpathlineto{\pgfqpoint{5.262607in}{1.644868in}}%
\pgfpathlineto{\pgfqpoint{5.262607in}{0.550000in}}%
\pgfpathclose%
\pgfusepath{fill}%
\end{pgfscope}%
\begin{pgfscope}%
\pgfpathrectangle{\pgfqpoint{1.250000in}{0.550000in}}{\pgfqpoint{7.750000in}{3.850000in}}%
\pgfusepath{clip}%
\pgfsetbuttcap%
\pgfsetmiterjoin%
\definecolor{currentfill}{rgb}{0.501961,0.501961,0.501961}%
\pgfsetfillcolor{currentfill}%
\pgfsetfillopacity{0.700000}%
\pgfsetlinewidth{0.000000pt}%
\definecolor{currentstroke}{rgb}{0.000000,0.000000,0.000000}%
\pgfsetstrokecolor{currentstroke}%
\pgfsetstrokeopacity{0.700000}%
\pgfsetdash{}{0pt}%
\pgfpathmoveto{\pgfqpoint{5.290128in}{0.550000in}}%
\pgfpathlineto{\pgfqpoint{5.317649in}{0.550000in}}%
\pgfpathlineto{\pgfqpoint{5.317649in}{0.550000in}}%
\pgfpathlineto{\pgfqpoint{5.290128in}{0.550000in}}%
\pgfpathlineto{\pgfqpoint{5.290128in}{0.550000in}}%
\pgfpathclose%
\pgfusepath{fill}%
\end{pgfscope}%
\begin{pgfscope}%
\pgfpathrectangle{\pgfqpoint{1.250000in}{0.550000in}}{\pgfqpoint{7.750000in}{3.850000in}}%
\pgfusepath{clip}%
\pgfsetbuttcap%
\pgfsetmiterjoin%
\definecolor{currentfill}{rgb}{0.501961,0.501961,0.501961}%
\pgfsetfillcolor{currentfill}%
\pgfsetfillopacity{0.700000}%
\pgfsetlinewidth{0.000000pt}%
\definecolor{currentstroke}{rgb}{0.000000,0.000000,0.000000}%
\pgfsetstrokecolor{currentstroke}%
\pgfsetstrokeopacity{0.700000}%
\pgfsetdash{}{0pt}%
\pgfpathmoveto{\pgfqpoint{5.317649in}{0.550000in}}%
\pgfpathlineto{\pgfqpoint{5.345170in}{0.550000in}}%
\pgfpathlineto{\pgfqpoint{5.345170in}{1.602100in}}%
\pgfpathlineto{\pgfqpoint{5.317649in}{1.602100in}}%
\pgfpathlineto{\pgfqpoint{5.317649in}{0.550000in}}%
\pgfpathclose%
\pgfusepath{fill}%
\end{pgfscope}%
\begin{pgfscope}%
\pgfpathrectangle{\pgfqpoint{1.250000in}{0.550000in}}{\pgfqpoint{7.750000in}{3.850000in}}%
\pgfusepath{clip}%
\pgfsetbuttcap%
\pgfsetmiterjoin%
\definecolor{currentfill}{rgb}{0.501961,0.501961,0.501961}%
\pgfsetfillcolor{currentfill}%
\pgfsetfillopacity{0.700000}%
\pgfsetlinewidth{0.000000pt}%
\definecolor{currentstroke}{rgb}{0.000000,0.000000,0.000000}%
\pgfsetstrokecolor{currentstroke}%
\pgfsetstrokeopacity{0.700000}%
\pgfsetdash{}{0pt}%
\pgfpathmoveto{\pgfqpoint{5.345170in}{0.550000in}}%
\pgfpathlineto{\pgfqpoint{5.372692in}{0.550000in}}%
\pgfpathlineto{\pgfqpoint{5.372692in}{0.550000in}}%
\pgfpathlineto{\pgfqpoint{5.345170in}{0.550000in}}%
\pgfpathlineto{\pgfqpoint{5.345170in}{0.550000in}}%
\pgfpathclose%
\pgfusepath{fill}%
\end{pgfscope}%
\begin{pgfscope}%
\pgfpathrectangle{\pgfqpoint{1.250000in}{0.550000in}}{\pgfqpoint{7.750000in}{3.850000in}}%
\pgfusepath{clip}%
\pgfsetbuttcap%
\pgfsetmiterjoin%
\definecolor{currentfill}{rgb}{0.501961,0.501961,0.501961}%
\pgfsetfillcolor{currentfill}%
\pgfsetfillopacity{0.700000}%
\pgfsetlinewidth{0.000000pt}%
\definecolor{currentstroke}{rgb}{0.000000,0.000000,0.000000}%
\pgfsetstrokecolor{currentstroke}%
\pgfsetstrokeopacity{0.700000}%
\pgfsetdash{}{0pt}%
\pgfpathmoveto{\pgfqpoint{5.372692in}{0.550000in}}%
\pgfpathlineto{\pgfqpoint{5.400213in}{0.550000in}}%
\pgfpathlineto{\pgfqpoint{5.400213in}{1.592120in}}%
\pgfpathlineto{\pgfqpoint{5.372692in}{1.592120in}}%
\pgfpathlineto{\pgfqpoint{5.372692in}{0.550000in}}%
\pgfpathclose%
\pgfusepath{fill}%
\end{pgfscope}%
\begin{pgfscope}%
\pgfpathrectangle{\pgfqpoint{1.250000in}{0.550000in}}{\pgfqpoint{7.750000in}{3.850000in}}%
\pgfusepath{clip}%
\pgfsetbuttcap%
\pgfsetmiterjoin%
\definecolor{currentfill}{rgb}{0.501961,0.501961,0.501961}%
\pgfsetfillcolor{currentfill}%
\pgfsetfillopacity{0.700000}%
\pgfsetlinewidth{0.000000pt}%
\definecolor{currentstroke}{rgb}{0.000000,0.000000,0.000000}%
\pgfsetstrokecolor{currentstroke}%
\pgfsetstrokeopacity{0.700000}%
\pgfsetdash{}{0pt}%
\pgfpathmoveto{\pgfqpoint{5.400213in}{0.550000in}}%
\pgfpathlineto{\pgfqpoint{5.427734in}{0.550000in}}%
\pgfpathlineto{\pgfqpoint{5.427734in}{0.550000in}}%
\pgfpathlineto{\pgfqpoint{5.400213in}{0.550000in}}%
\pgfpathlineto{\pgfqpoint{5.400213in}{0.550000in}}%
\pgfpathclose%
\pgfusepath{fill}%
\end{pgfscope}%
\begin{pgfscope}%
\pgfpathrectangle{\pgfqpoint{1.250000in}{0.550000in}}{\pgfqpoint{7.750000in}{3.850000in}}%
\pgfusepath{clip}%
\pgfsetbuttcap%
\pgfsetmiterjoin%
\definecolor{currentfill}{rgb}{0.501961,0.501961,0.501961}%
\pgfsetfillcolor{currentfill}%
\pgfsetfillopacity{0.700000}%
\pgfsetlinewidth{0.000000pt}%
\definecolor{currentstroke}{rgb}{0.000000,0.000000,0.000000}%
\pgfsetstrokecolor{currentstroke}%
\pgfsetstrokeopacity{0.700000}%
\pgfsetdash{}{0pt}%
\pgfpathmoveto{\pgfqpoint{5.427734in}{0.550000in}}%
\pgfpathlineto{\pgfqpoint{5.455256in}{0.550000in}}%
\pgfpathlineto{\pgfqpoint{5.455256in}{1.510861in}}%
\pgfpathlineto{\pgfqpoint{5.427734in}{1.510861in}}%
\pgfpathlineto{\pgfqpoint{5.427734in}{0.550000in}}%
\pgfpathclose%
\pgfusepath{fill}%
\end{pgfscope}%
\begin{pgfscope}%
\pgfpathrectangle{\pgfqpoint{1.250000in}{0.550000in}}{\pgfqpoint{7.750000in}{3.850000in}}%
\pgfusepath{clip}%
\pgfsetbuttcap%
\pgfsetmiterjoin%
\definecolor{currentfill}{rgb}{0.501961,0.501961,0.501961}%
\pgfsetfillcolor{currentfill}%
\pgfsetfillopacity{0.700000}%
\pgfsetlinewidth{0.000000pt}%
\definecolor{currentstroke}{rgb}{0.000000,0.000000,0.000000}%
\pgfsetstrokecolor{currentstroke}%
\pgfsetstrokeopacity{0.700000}%
\pgfsetdash{}{0pt}%
\pgfpathmoveto{\pgfqpoint{5.455256in}{0.550000in}}%
\pgfpathlineto{\pgfqpoint{5.482777in}{0.550000in}}%
\pgfpathlineto{\pgfqpoint{5.482777in}{0.550000in}}%
\pgfpathlineto{\pgfqpoint{5.455256in}{0.550000in}}%
\pgfpathlineto{\pgfqpoint{5.455256in}{0.550000in}}%
\pgfpathclose%
\pgfusepath{fill}%
\end{pgfscope}%
\begin{pgfscope}%
\pgfpathrectangle{\pgfqpoint{1.250000in}{0.550000in}}{\pgfqpoint{7.750000in}{3.850000in}}%
\pgfusepath{clip}%
\pgfsetbuttcap%
\pgfsetmiterjoin%
\definecolor{currentfill}{rgb}{0.501961,0.501961,0.501961}%
\pgfsetfillcolor{currentfill}%
\pgfsetfillopacity{0.700000}%
\pgfsetlinewidth{0.000000pt}%
\definecolor{currentstroke}{rgb}{0.000000,0.000000,0.000000}%
\pgfsetstrokecolor{currentstroke}%
\pgfsetstrokeopacity{0.700000}%
\pgfsetdash{}{0pt}%
\pgfpathmoveto{\pgfqpoint{5.482777in}{0.550000in}}%
\pgfpathlineto{\pgfqpoint{5.510298in}{0.550000in}}%
\pgfpathlineto{\pgfqpoint{5.510298in}{1.508009in}}%
\pgfpathlineto{\pgfqpoint{5.482777in}{1.508009in}}%
\pgfpathlineto{\pgfqpoint{5.482777in}{0.550000in}}%
\pgfpathclose%
\pgfusepath{fill}%
\end{pgfscope}%
\begin{pgfscope}%
\pgfpathrectangle{\pgfqpoint{1.250000in}{0.550000in}}{\pgfqpoint{7.750000in}{3.850000in}}%
\pgfusepath{clip}%
\pgfsetbuttcap%
\pgfsetmiterjoin%
\definecolor{currentfill}{rgb}{0.501961,0.501961,0.501961}%
\pgfsetfillcolor{currentfill}%
\pgfsetfillopacity{0.700000}%
\pgfsetlinewidth{0.000000pt}%
\definecolor{currentstroke}{rgb}{0.000000,0.000000,0.000000}%
\pgfsetstrokecolor{currentstroke}%
\pgfsetstrokeopacity{0.700000}%
\pgfsetdash{}{0pt}%
\pgfpathmoveto{\pgfqpoint{5.510298in}{0.550000in}}%
\pgfpathlineto{\pgfqpoint{5.537820in}{0.550000in}}%
\pgfpathlineto{\pgfqpoint{5.537820in}{0.550000in}}%
\pgfpathlineto{\pgfqpoint{5.510298in}{0.550000in}}%
\pgfpathlineto{\pgfqpoint{5.510298in}{0.550000in}}%
\pgfpathclose%
\pgfusepath{fill}%
\end{pgfscope}%
\begin{pgfscope}%
\pgfpathrectangle{\pgfqpoint{1.250000in}{0.550000in}}{\pgfqpoint{7.750000in}{3.850000in}}%
\pgfusepath{clip}%
\pgfsetbuttcap%
\pgfsetmiterjoin%
\definecolor{currentfill}{rgb}{0.501961,0.501961,0.501961}%
\pgfsetfillcolor{currentfill}%
\pgfsetfillopacity{0.700000}%
\pgfsetlinewidth{0.000000pt}%
\definecolor{currentstroke}{rgb}{0.000000,0.000000,0.000000}%
\pgfsetstrokecolor{currentstroke}%
\pgfsetstrokeopacity{0.700000}%
\pgfsetdash{}{0pt}%
\pgfpathmoveto{\pgfqpoint{5.537820in}{0.550000in}}%
\pgfpathlineto{\pgfqpoint{5.565341in}{0.550000in}}%
\pgfpathlineto{\pgfqpoint{5.565341in}{1.382556in}}%
\pgfpathlineto{\pgfqpoint{5.537820in}{1.382556in}}%
\pgfpathlineto{\pgfqpoint{5.537820in}{0.550000in}}%
\pgfpathclose%
\pgfusepath{fill}%
\end{pgfscope}%
\begin{pgfscope}%
\pgfpathrectangle{\pgfqpoint{1.250000in}{0.550000in}}{\pgfqpoint{7.750000in}{3.850000in}}%
\pgfusepath{clip}%
\pgfsetbuttcap%
\pgfsetmiterjoin%
\definecolor{currentfill}{rgb}{0.501961,0.501961,0.501961}%
\pgfsetfillcolor{currentfill}%
\pgfsetfillopacity{0.700000}%
\pgfsetlinewidth{0.000000pt}%
\definecolor{currentstroke}{rgb}{0.000000,0.000000,0.000000}%
\pgfsetstrokecolor{currentstroke}%
\pgfsetstrokeopacity{0.700000}%
\pgfsetdash{}{0pt}%
\pgfpathmoveto{\pgfqpoint{5.565341in}{0.550000in}}%
\pgfpathlineto{\pgfqpoint{5.592862in}{0.550000in}}%
\pgfpathlineto{\pgfqpoint{5.592862in}{0.550000in}}%
\pgfpathlineto{\pgfqpoint{5.565341in}{0.550000in}}%
\pgfpathlineto{\pgfqpoint{5.565341in}{0.550000in}}%
\pgfpathclose%
\pgfusepath{fill}%
\end{pgfscope}%
\begin{pgfscope}%
\pgfpathrectangle{\pgfqpoint{1.250000in}{0.550000in}}{\pgfqpoint{7.750000in}{3.850000in}}%
\pgfusepath{clip}%
\pgfsetbuttcap%
\pgfsetmiterjoin%
\definecolor{currentfill}{rgb}{0.501961,0.501961,0.501961}%
\pgfsetfillcolor{currentfill}%
\pgfsetfillopacity{0.700000}%
\pgfsetlinewidth{0.000000pt}%
\definecolor{currentstroke}{rgb}{0.000000,0.000000,0.000000}%
\pgfsetstrokecolor{currentstroke}%
\pgfsetstrokeopacity{0.700000}%
\pgfsetdash{}{0pt}%
\pgfpathmoveto{\pgfqpoint{5.592862in}{0.550000in}}%
\pgfpathlineto{\pgfqpoint{5.620384in}{0.550000in}}%
\pgfpathlineto{\pgfqpoint{5.620384in}{1.449559in}}%
\pgfpathlineto{\pgfqpoint{5.592862in}{1.449559in}}%
\pgfpathlineto{\pgfqpoint{5.592862in}{0.550000in}}%
\pgfpathclose%
\pgfusepath{fill}%
\end{pgfscope}%
\begin{pgfscope}%
\pgfpathrectangle{\pgfqpoint{1.250000in}{0.550000in}}{\pgfqpoint{7.750000in}{3.850000in}}%
\pgfusepath{clip}%
\pgfsetbuttcap%
\pgfsetmiterjoin%
\definecolor{currentfill}{rgb}{0.501961,0.501961,0.501961}%
\pgfsetfillcolor{currentfill}%
\pgfsetfillopacity{0.700000}%
\pgfsetlinewidth{0.000000pt}%
\definecolor{currentstroke}{rgb}{0.000000,0.000000,0.000000}%
\pgfsetstrokecolor{currentstroke}%
\pgfsetstrokeopacity{0.700000}%
\pgfsetdash{}{0pt}%
\pgfpathmoveto{\pgfqpoint{5.620384in}{0.550000in}}%
\pgfpathlineto{\pgfqpoint{5.647905in}{0.550000in}}%
\pgfpathlineto{\pgfqpoint{5.647905in}{0.550000in}}%
\pgfpathlineto{\pgfqpoint{5.620384in}{0.550000in}}%
\pgfpathlineto{\pgfqpoint{5.620384in}{0.550000in}}%
\pgfpathclose%
\pgfusepath{fill}%
\end{pgfscope}%
\begin{pgfscope}%
\pgfpathrectangle{\pgfqpoint{1.250000in}{0.550000in}}{\pgfqpoint{7.750000in}{3.850000in}}%
\pgfusepath{clip}%
\pgfsetbuttcap%
\pgfsetmiterjoin%
\definecolor{currentfill}{rgb}{0.501961,0.501961,0.501961}%
\pgfsetfillcolor{currentfill}%
\pgfsetfillopacity{0.700000}%
\pgfsetlinewidth{0.000000pt}%
\definecolor{currentstroke}{rgb}{0.000000,0.000000,0.000000}%
\pgfsetstrokecolor{currentstroke}%
\pgfsetstrokeopacity{0.700000}%
\pgfsetdash{}{0pt}%
\pgfpathmoveto{\pgfqpoint{5.647905in}{0.550000in}}%
\pgfpathlineto{\pgfqpoint{5.675426in}{0.550000in}}%
\pgfpathlineto{\pgfqpoint{5.675426in}{1.425324in}}%
\pgfpathlineto{\pgfqpoint{5.647905in}{1.425324in}}%
\pgfpathlineto{\pgfqpoint{5.647905in}{0.550000in}}%
\pgfpathclose%
\pgfusepath{fill}%
\end{pgfscope}%
\begin{pgfscope}%
\pgfpathrectangle{\pgfqpoint{1.250000in}{0.550000in}}{\pgfqpoint{7.750000in}{3.850000in}}%
\pgfusepath{clip}%
\pgfsetbuttcap%
\pgfsetmiterjoin%
\definecolor{currentfill}{rgb}{0.501961,0.501961,0.501961}%
\pgfsetfillcolor{currentfill}%
\pgfsetfillopacity{0.700000}%
\pgfsetlinewidth{0.000000pt}%
\definecolor{currentstroke}{rgb}{0.000000,0.000000,0.000000}%
\pgfsetstrokecolor{currentstroke}%
\pgfsetstrokeopacity{0.700000}%
\pgfsetdash{}{0pt}%
\pgfpathmoveto{\pgfqpoint{5.675426in}{0.550000in}}%
\pgfpathlineto{\pgfqpoint{5.702947in}{0.550000in}}%
\pgfpathlineto{\pgfqpoint{5.702947in}{0.550000in}}%
\pgfpathlineto{\pgfqpoint{5.675426in}{0.550000in}}%
\pgfpathlineto{\pgfqpoint{5.675426in}{0.550000in}}%
\pgfpathclose%
\pgfusepath{fill}%
\end{pgfscope}%
\begin{pgfscope}%
\pgfpathrectangle{\pgfqpoint{1.250000in}{0.550000in}}{\pgfqpoint{7.750000in}{3.850000in}}%
\pgfusepath{clip}%
\pgfsetbuttcap%
\pgfsetmiterjoin%
\definecolor{currentfill}{rgb}{0.501961,0.501961,0.501961}%
\pgfsetfillcolor{currentfill}%
\pgfsetfillopacity{0.700000}%
\pgfsetlinewidth{0.000000pt}%
\definecolor{currentstroke}{rgb}{0.000000,0.000000,0.000000}%
\pgfsetstrokecolor{currentstroke}%
\pgfsetstrokeopacity{0.700000}%
\pgfsetdash{}{0pt}%
\pgfpathmoveto{\pgfqpoint{5.702947in}{0.550000in}}%
\pgfpathlineto{\pgfqpoint{5.730469in}{0.550000in}}%
\pgfpathlineto{\pgfqpoint{5.730469in}{1.445283in}}%
\pgfpathlineto{\pgfqpoint{5.702947in}{1.445283in}}%
\pgfpathlineto{\pgfqpoint{5.702947in}{0.550000in}}%
\pgfpathclose%
\pgfusepath{fill}%
\end{pgfscope}%
\begin{pgfscope}%
\pgfpathrectangle{\pgfqpoint{1.250000in}{0.550000in}}{\pgfqpoint{7.750000in}{3.850000in}}%
\pgfusepath{clip}%
\pgfsetbuttcap%
\pgfsetmiterjoin%
\definecolor{currentfill}{rgb}{0.501961,0.501961,0.501961}%
\pgfsetfillcolor{currentfill}%
\pgfsetfillopacity{0.700000}%
\pgfsetlinewidth{0.000000pt}%
\definecolor{currentstroke}{rgb}{0.000000,0.000000,0.000000}%
\pgfsetstrokecolor{currentstroke}%
\pgfsetstrokeopacity{0.700000}%
\pgfsetdash{}{0pt}%
\pgfpathmoveto{\pgfqpoint{5.730469in}{0.550000in}}%
\pgfpathlineto{\pgfqpoint{5.757990in}{0.550000in}}%
\pgfpathlineto{\pgfqpoint{5.757990in}{0.550000in}}%
\pgfpathlineto{\pgfqpoint{5.730469in}{0.550000in}}%
\pgfpathlineto{\pgfqpoint{5.730469in}{0.550000in}}%
\pgfpathclose%
\pgfusepath{fill}%
\end{pgfscope}%
\begin{pgfscope}%
\pgfpathrectangle{\pgfqpoint{1.250000in}{0.550000in}}{\pgfqpoint{7.750000in}{3.850000in}}%
\pgfusepath{clip}%
\pgfsetbuttcap%
\pgfsetmiterjoin%
\definecolor{currentfill}{rgb}{0.501961,0.501961,0.501961}%
\pgfsetfillcolor{currentfill}%
\pgfsetfillopacity{0.700000}%
\pgfsetlinewidth{0.000000pt}%
\definecolor{currentstroke}{rgb}{0.000000,0.000000,0.000000}%
\pgfsetstrokecolor{currentstroke}%
\pgfsetstrokeopacity{0.700000}%
\pgfsetdash{}{0pt}%
\pgfpathmoveto{\pgfqpoint{5.757990in}{0.550000in}}%
\pgfpathlineto{\pgfqpoint{5.785511in}{0.550000in}}%
\pgfpathlineto{\pgfqpoint{5.785511in}{1.374002in}}%
\pgfpathlineto{\pgfqpoint{5.757990in}{1.374002in}}%
\pgfpathlineto{\pgfqpoint{5.757990in}{0.550000in}}%
\pgfpathclose%
\pgfusepath{fill}%
\end{pgfscope}%
\begin{pgfscope}%
\pgfpathrectangle{\pgfqpoint{1.250000in}{0.550000in}}{\pgfqpoint{7.750000in}{3.850000in}}%
\pgfusepath{clip}%
\pgfsetbuttcap%
\pgfsetmiterjoin%
\definecolor{currentfill}{rgb}{0.501961,0.501961,0.501961}%
\pgfsetfillcolor{currentfill}%
\pgfsetfillopacity{0.700000}%
\pgfsetlinewidth{0.000000pt}%
\definecolor{currentstroke}{rgb}{0.000000,0.000000,0.000000}%
\pgfsetstrokecolor{currentstroke}%
\pgfsetstrokeopacity{0.700000}%
\pgfsetdash{}{0pt}%
\pgfpathmoveto{\pgfqpoint{5.785511in}{0.550000in}}%
\pgfpathlineto{\pgfqpoint{5.813033in}{0.550000in}}%
\pgfpathlineto{\pgfqpoint{5.813033in}{0.550000in}}%
\pgfpathlineto{\pgfqpoint{5.785511in}{0.550000in}}%
\pgfpathlineto{\pgfqpoint{5.785511in}{0.550000in}}%
\pgfpathclose%
\pgfusepath{fill}%
\end{pgfscope}%
\begin{pgfscope}%
\pgfpathrectangle{\pgfqpoint{1.250000in}{0.550000in}}{\pgfqpoint{7.750000in}{3.850000in}}%
\pgfusepath{clip}%
\pgfsetbuttcap%
\pgfsetmiterjoin%
\definecolor{currentfill}{rgb}{0.501961,0.501961,0.501961}%
\pgfsetfillcolor{currentfill}%
\pgfsetfillopacity{0.700000}%
\pgfsetlinewidth{0.000000pt}%
\definecolor{currentstroke}{rgb}{0.000000,0.000000,0.000000}%
\pgfsetstrokecolor{currentstroke}%
\pgfsetstrokeopacity{0.700000}%
\pgfsetdash{}{0pt}%
\pgfpathmoveto{\pgfqpoint{5.813033in}{0.550000in}}%
\pgfpathlineto{\pgfqpoint{5.840554in}{0.550000in}}%
\pgfpathlineto{\pgfqpoint{5.840554in}{1.379705in}}%
\pgfpathlineto{\pgfqpoint{5.813033in}{1.379705in}}%
\pgfpathlineto{\pgfqpoint{5.813033in}{0.550000in}}%
\pgfpathclose%
\pgfusepath{fill}%
\end{pgfscope}%
\begin{pgfscope}%
\pgfpathrectangle{\pgfqpoint{1.250000in}{0.550000in}}{\pgfqpoint{7.750000in}{3.850000in}}%
\pgfusepath{clip}%
\pgfsetbuttcap%
\pgfsetmiterjoin%
\definecolor{currentfill}{rgb}{0.501961,0.501961,0.501961}%
\pgfsetfillcolor{currentfill}%
\pgfsetfillopacity{0.700000}%
\pgfsetlinewidth{0.000000pt}%
\definecolor{currentstroke}{rgb}{0.000000,0.000000,0.000000}%
\pgfsetstrokecolor{currentstroke}%
\pgfsetstrokeopacity{0.700000}%
\pgfsetdash{}{0pt}%
\pgfpathmoveto{\pgfqpoint{5.840554in}{0.550000in}}%
\pgfpathlineto{\pgfqpoint{5.868075in}{0.550000in}}%
\pgfpathlineto{\pgfqpoint{5.868075in}{0.550000in}}%
\pgfpathlineto{\pgfqpoint{5.840554in}{0.550000in}}%
\pgfpathlineto{\pgfqpoint{5.840554in}{0.550000in}}%
\pgfpathclose%
\pgfusepath{fill}%
\end{pgfscope}%
\begin{pgfscope}%
\pgfpathrectangle{\pgfqpoint{1.250000in}{0.550000in}}{\pgfqpoint{7.750000in}{3.850000in}}%
\pgfusepath{clip}%
\pgfsetbuttcap%
\pgfsetmiterjoin%
\definecolor{currentfill}{rgb}{0.501961,0.501961,0.501961}%
\pgfsetfillcolor{currentfill}%
\pgfsetfillopacity{0.700000}%
\pgfsetlinewidth{0.000000pt}%
\definecolor{currentstroke}{rgb}{0.000000,0.000000,0.000000}%
\pgfsetstrokecolor{currentstroke}%
\pgfsetstrokeopacity{0.700000}%
\pgfsetdash{}{0pt}%
\pgfpathmoveto{\pgfqpoint{5.868075in}{0.550000in}}%
\pgfpathlineto{\pgfqpoint{5.895597in}{0.550000in}}%
\pgfpathlineto{\pgfqpoint{5.895597in}{1.278486in}}%
\pgfpathlineto{\pgfqpoint{5.868075in}{1.278486in}}%
\pgfpathlineto{\pgfqpoint{5.868075in}{0.550000in}}%
\pgfpathclose%
\pgfusepath{fill}%
\end{pgfscope}%
\begin{pgfscope}%
\pgfpathrectangle{\pgfqpoint{1.250000in}{0.550000in}}{\pgfqpoint{7.750000in}{3.850000in}}%
\pgfusepath{clip}%
\pgfsetbuttcap%
\pgfsetmiterjoin%
\definecolor{currentfill}{rgb}{0.501961,0.501961,0.501961}%
\pgfsetfillcolor{currentfill}%
\pgfsetfillopacity{0.700000}%
\pgfsetlinewidth{0.000000pt}%
\definecolor{currentstroke}{rgb}{0.000000,0.000000,0.000000}%
\pgfsetstrokecolor{currentstroke}%
\pgfsetstrokeopacity{0.700000}%
\pgfsetdash{}{0pt}%
\pgfpathmoveto{\pgfqpoint{5.895597in}{0.550000in}}%
\pgfpathlineto{\pgfqpoint{5.923118in}{0.550000in}}%
\pgfpathlineto{\pgfqpoint{5.923118in}{0.550000in}}%
\pgfpathlineto{\pgfqpoint{5.895597in}{0.550000in}}%
\pgfpathlineto{\pgfqpoint{5.895597in}{0.550000in}}%
\pgfpathclose%
\pgfusepath{fill}%
\end{pgfscope}%
\begin{pgfscope}%
\pgfpathrectangle{\pgfqpoint{1.250000in}{0.550000in}}{\pgfqpoint{7.750000in}{3.850000in}}%
\pgfusepath{clip}%
\pgfsetbuttcap%
\pgfsetmiterjoin%
\definecolor{currentfill}{rgb}{0.501961,0.501961,0.501961}%
\pgfsetfillcolor{currentfill}%
\pgfsetfillopacity{0.700000}%
\pgfsetlinewidth{0.000000pt}%
\definecolor{currentstroke}{rgb}{0.000000,0.000000,0.000000}%
\pgfsetstrokecolor{currentstroke}%
\pgfsetstrokeopacity{0.700000}%
\pgfsetdash{}{0pt}%
\pgfpathmoveto{\pgfqpoint{5.923118in}{0.550000in}}%
\pgfpathlineto{\pgfqpoint{5.950639in}{0.550000in}}%
\pgfpathlineto{\pgfqpoint{5.950639in}{1.272784in}}%
\pgfpathlineto{\pgfqpoint{5.923118in}{1.272784in}}%
\pgfpathlineto{\pgfqpoint{5.923118in}{0.550000in}}%
\pgfpathclose%
\pgfusepath{fill}%
\end{pgfscope}%
\begin{pgfscope}%
\pgfpathrectangle{\pgfqpoint{1.250000in}{0.550000in}}{\pgfqpoint{7.750000in}{3.850000in}}%
\pgfusepath{clip}%
\pgfsetbuttcap%
\pgfsetmiterjoin%
\definecolor{currentfill}{rgb}{0.501961,0.501961,0.501961}%
\pgfsetfillcolor{currentfill}%
\pgfsetfillopacity{0.700000}%
\pgfsetlinewidth{0.000000pt}%
\definecolor{currentstroke}{rgb}{0.000000,0.000000,0.000000}%
\pgfsetstrokecolor{currentstroke}%
\pgfsetstrokeopacity{0.700000}%
\pgfsetdash{}{0pt}%
\pgfpathmoveto{\pgfqpoint{5.950639in}{0.550000in}}%
\pgfpathlineto{\pgfqpoint{5.978161in}{0.550000in}}%
\pgfpathlineto{\pgfqpoint{5.978161in}{1.314126in}}%
\pgfpathlineto{\pgfqpoint{5.950639in}{1.314126in}}%
\pgfpathlineto{\pgfqpoint{5.950639in}{0.550000in}}%
\pgfpathclose%
\pgfusepath{fill}%
\end{pgfscope}%
\begin{pgfscope}%
\pgfpathrectangle{\pgfqpoint{1.250000in}{0.550000in}}{\pgfqpoint{7.750000in}{3.850000in}}%
\pgfusepath{clip}%
\pgfsetbuttcap%
\pgfsetmiterjoin%
\definecolor{currentfill}{rgb}{0.501961,0.501961,0.501961}%
\pgfsetfillcolor{currentfill}%
\pgfsetfillopacity{0.700000}%
\pgfsetlinewidth{0.000000pt}%
\definecolor{currentstroke}{rgb}{0.000000,0.000000,0.000000}%
\pgfsetstrokecolor{currentstroke}%
\pgfsetstrokeopacity{0.700000}%
\pgfsetdash{}{0pt}%
\pgfpathmoveto{\pgfqpoint{5.978161in}{0.550000in}}%
\pgfpathlineto{\pgfqpoint{6.005682in}{0.550000in}}%
\pgfpathlineto{\pgfqpoint{6.005682in}{0.550000in}}%
\pgfpathlineto{\pgfqpoint{5.978161in}{0.550000in}}%
\pgfpathlineto{\pgfqpoint{5.978161in}{0.550000in}}%
\pgfpathclose%
\pgfusepath{fill}%
\end{pgfscope}%
\begin{pgfscope}%
\pgfpathrectangle{\pgfqpoint{1.250000in}{0.550000in}}{\pgfqpoint{7.750000in}{3.850000in}}%
\pgfusepath{clip}%
\pgfsetbuttcap%
\pgfsetmiterjoin%
\definecolor{currentfill}{rgb}{0.501961,0.501961,0.501961}%
\pgfsetfillcolor{currentfill}%
\pgfsetfillopacity{0.700000}%
\pgfsetlinewidth{0.000000pt}%
\definecolor{currentstroke}{rgb}{0.000000,0.000000,0.000000}%
\pgfsetstrokecolor{currentstroke}%
\pgfsetstrokeopacity{0.700000}%
\pgfsetdash{}{0pt}%
\pgfpathmoveto{\pgfqpoint{6.005682in}{0.550000in}}%
\pgfpathlineto{\pgfqpoint{6.033203in}{0.550000in}}%
\pgfpathlineto{\pgfqpoint{6.033203in}{1.261379in}}%
\pgfpathlineto{\pgfqpoint{6.005682in}{1.261379in}}%
\pgfpathlineto{\pgfqpoint{6.005682in}{0.550000in}}%
\pgfpathclose%
\pgfusepath{fill}%
\end{pgfscope}%
\begin{pgfscope}%
\pgfpathrectangle{\pgfqpoint{1.250000in}{0.550000in}}{\pgfqpoint{7.750000in}{3.850000in}}%
\pgfusepath{clip}%
\pgfsetbuttcap%
\pgfsetmiterjoin%
\definecolor{currentfill}{rgb}{0.501961,0.501961,0.501961}%
\pgfsetfillcolor{currentfill}%
\pgfsetfillopacity{0.700000}%
\pgfsetlinewidth{0.000000pt}%
\definecolor{currentstroke}{rgb}{0.000000,0.000000,0.000000}%
\pgfsetstrokecolor{currentstroke}%
\pgfsetstrokeopacity{0.700000}%
\pgfsetdash{}{0pt}%
\pgfpathmoveto{\pgfqpoint{6.033203in}{0.550000in}}%
\pgfpathlineto{\pgfqpoint{6.060724in}{0.550000in}}%
\pgfpathlineto{\pgfqpoint{6.060724in}{0.550000in}}%
\pgfpathlineto{\pgfqpoint{6.033203in}{0.550000in}}%
\pgfpathlineto{\pgfqpoint{6.033203in}{0.550000in}}%
\pgfpathclose%
\pgfusepath{fill}%
\end{pgfscope}%
\begin{pgfscope}%
\pgfpathrectangle{\pgfqpoint{1.250000in}{0.550000in}}{\pgfqpoint{7.750000in}{3.850000in}}%
\pgfusepath{clip}%
\pgfsetbuttcap%
\pgfsetmiterjoin%
\definecolor{currentfill}{rgb}{0.501961,0.501961,0.501961}%
\pgfsetfillcolor{currentfill}%
\pgfsetfillopacity{0.700000}%
\pgfsetlinewidth{0.000000pt}%
\definecolor{currentstroke}{rgb}{0.000000,0.000000,0.000000}%
\pgfsetstrokecolor{currentstroke}%
\pgfsetstrokeopacity{0.700000}%
\pgfsetdash{}{0pt}%
\pgfpathmoveto{\pgfqpoint{6.060724in}{0.550000in}}%
\pgfpathlineto{\pgfqpoint{6.088246in}{0.550000in}}%
\pgfpathlineto{\pgfqpoint{6.088246in}{1.207206in}}%
\pgfpathlineto{\pgfqpoint{6.060724in}{1.207206in}}%
\pgfpathlineto{\pgfqpoint{6.060724in}{0.550000in}}%
\pgfpathclose%
\pgfusepath{fill}%
\end{pgfscope}%
\begin{pgfscope}%
\pgfpathrectangle{\pgfqpoint{1.250000in}{0.550000in}}{\pgfqpoint{7.750000in}{3.850000in}}%
\pgfusepath{clip}%
\pgfsetbuttcap%
\pgfsetmiterjoin%
\definecolor{currentfill}{rgb}{0.501961,0.501961,0.501961}%
\pgfsetfillcolor{currentfill}%
\pgfsetfillopacity{0.700000}%
\pgfsetlinewidth{0.000000pt}%
\definecolor{currentstroke}{rgb}{0.000000,0.000000,0.000000}%
\pgfsetstrokecolor{currentstroke}%
\pgfsetstrokeopacity{0.700000}%
\pgfsetdash{}{0pt}%
\pgfpathmoveto{\pgfqpoint{6.088246in}{0.550000in}}%
\pgfpathlineto{\pgfqpoint{6.115767in}{0.550000in}}%
\pgfpathlineto{\pgfqpoint{6.115767in}{0.550000in}}%
\pgfpathlineto{\pgfqpoint{6.088246in}{0.550000in}}%
\pgfpathlineto{\pgfqpoint{6.088246in}{0.550000in}}%
\pgfpathclose%
\pgfusepath{fill}%
\end{pgfscope}%
\begin{pgfscope}%
\pgfpathrectangle{\pgfqpoint{1.250000in}{0.550000in}}{\pgfqpoint{7.750000in}{3.850000in}}%
\pgfusepath{clip}%
\pgfsetbuttcap%
\pgfsetmiterjoin%
\definecolor{currentfill}{rgb}{0.501961,0.501961,0.501961}%
\pgfsetfillcolor{currentfill}%
\pgfsetfillopacity{0.700000}%
\pgfsetlinewidth{0.000000pt}%
\definecolor{currentstroke}{rgb}{0.000000,0.000000,0.000000}%
\pgfsetstrokecolor{currentstroke}%
\pgfsetstrokeopacity{0.700000}%
\pgfsetdash{}{0pt}%
\pgfpathmoveto{\pgfqpoint{6.115767in}{0.550000in}}%
\pgfpathlineto{\pgfqpoint{6.143288in}{0.550000in}}%
\pgfpathlineto{\pgfqpoint{6.143288in}{1.194375in}}%
\pgfpathlineto{\pgfqpoint{6.115767in}{1.194375in}}%
\pgfpathlineto{\pgfqpoint{6.115767in}{0.550000in}}%
\pgfpathclose%
\pgfusepath{fill}%
\end{pgfscope}%
\begin{pgfscope}%
\pgfpathrectangle{\pgfqpoint{1.250000in}{0.550000in}}{\pgfqpoint{7.750000in}{3.850000in}}%
\pgfusepath{clip}%
\pgfsetbuttcap%
\pgfsetmiterjoin%
\definecolor{currentfill}{rgb}{0.501961,0.501961,0.501961}%
\pgfsetfillcolor{currentfill}%
\pgfsetfillopacity{0.700000}%
\pgfsetlinewidth{0.000000pt}%
\definecolor{currentstroke}{rgb}{0.000000,0.000000,0.000000}%
\pgfsetstrokecolor{currentstroke}%
\pgfsetstrokeopacity{0.700000}%
\pgfsetdash{}{0pt}%
\pgfpathmoveto{\pgfqpoint{6.143288in}{0.550000in}}%
\pgfpathlineto{\pgfqpoint{6.170810in}{0.550000in}}%
\pgfpathlineto{\pgfqpoint{6.170810in}{0.550000in}}%
\pgfpathlineto{\pgfqpoint{6.143288in}{0.550000in}}%
\pgfpathlineto{\pgfqpoint{6.143288in}{0.550000in}}%
\pgfpathclose%
\pgfusepath{fill}%
\end{pgfscope}%
\begin{pgfscope}%
\pgfpathrectangle{\pgfqpoint{1.250000in}{0.550000in}}{\pgfqpoint{7.750000in}{3.850000in}}%
\pgfusepath{clip}%
\pgfsetbuttcap%
\pgfsetmiterjoin%
\definecolor{currentfill}{rgb}{0.501961,0.501961,0.501961}%
\pgfsetfillcolor{currentfill}%
\pgfsetfillopacity{0.700000}%
\pgfsetlinewidth{0.000000pt}%
\definecolor{currentstroke}{rgb}{0.000000,0.000000,0.000000}%
\pgfsetstrokecolor{currentstroke}%
\pgfsetstrokeopacity{0.700000}%
\pgfsetdash{}{0pt}%
\pgfpathmoveto{\pgfqpoint{6.170810in}{0.550000in}}%
\pgfpathlineto{\pgfqpoint{6.198331in}{0.550000in}}%
\pgfpathlineto{\pgfqpoint{6.198331in}{1.222888in}}%
\pgfpathlineto{\pgfqpoint{6.170810in}{1.222888in}}%
\pgfpathlineto{\pgfqpoint{6.170810in}{0.550000in}}%
\pgfpathclose%
\pgfusepath{fill}%
\end{pgfscope}%
\begin{pgfscope}%
\pgfpathrectangle{\pgfqpoint{1.250000in}{0.550000in}}{\pgfqpoint{7.750000in}{3.850000in}}%
\pgfusepath{clip}%
\pgfsetbuttcap%
\pgfsetmiterjoin%
\definecolor{currentfill}{rgb}{0.501961,0.501961,0.501961}%
\pgfsetfillcolor{currentfill}%
\pgfsetfillopacity{0.700000}%
\pgfsetlinewidth{0.000000pt}%
\definecolor{currentstroke}{rgb}{0.000000,0.000000,0.000000}%
\pgfsetstrokecolor{currentstroke}%
\pgfsetstrokeopacity{0.700000}%
\pgfsetdash{}{0pt}%
\pgfpathmoveto{\pgfqpoint{6.198331in}{0.550000in}}%
\pgfpathlineto{\pgfqpoint{6.225852in}{0.550000in}}%
\pgfpathlineto{\pgfqpoint{6.225852in}{0.550000in}}%
\pgfpathlineto{\pgfqpoint{6.198331in}{0.550000in}}%
\pgfpathlineto{\pgfqpoint{6.198331in}{0.550000in}}%
\pgfpathclose%
\pgfusepath{fill}%
\end{pgfscope}%
\begin{pgfscope}%
\pgfpathrectangle{\pgfqpoint{1.250000in}{0.550000in}}{\pgfqpoint{7.750000in}{3.850000in}}%
\pgfusepath{clip}%
\pgfsetbuttcap%
\pgfsetmiterjoin%
\definecolor{currentfill}{rgb}{0.501961,0.501961,0.501961}%
\pgfsetfillcolor{currentfill}%
\pgfsetfillopacity{0.700000}%
\pgfsetlinewidth{0.000000pt}%
\definecolor{currentstroke}{rgb}{0.000000,0.000000,0.000000}%
\pgfsetstrokecolor{currentstroke}%
\pgfsetstrokeopacity{0.700000}%
\pgfsetdash{}{0pt}%
\pgfpathmoveto{\pgfqpoint{6.225852in}{0.550000in}}%
\pgfpathlineto{\pgfqpoint{6.253374in}{0.550000in}}%
\pgfpathlineto{\pgfqpoint{6.253374in}{1.191524in}}%
\pgfpathlineto{\pgfqpoint{6.225852in}{1.191524in}}%
\pgfpathlineto{\pgfqpoint{6.225852in}{0.550000in}}%
\pgfpathclose%
\pgfusepath{fill}%
\end{pgfscope}%
\begin{pgfscope}%
\pgfpathrectangle{\pgfqpoint{1.250000in}{0.550000in}}{\pgfqpoint{7.750000in}{3.850000in}}%
\pgfusepath{clip}%
\pgfsetbuttcap%
\pgfsetmiterjoin%
\definecolor{currentfill}{rgb}{0.501961,0.501961,0.501961}%
\pgfsetfillcolor{currentfill}%
\pgfsetfillopacity{0.700000}%
\pgfsetlinewidth{0.000000pt}%
\definecolor{currentstroke}{rgb}{0.000000,0.000000,0.000000}%
\pgfsetstrokecolor{currentstroke}%
\pgfsetstrokeopacity{0.700000}%
\pgfsetdash{}{0pt}%
\pgfpathmoveto{\pgfqpoint{6.253374in}{0.550000in}}%
\pgfpathlineto{\pgfqpoint{6.280895in}{0.550000in}}%
\pgfpathlineto{\pgfqpoint{6.280895in}{0.550000in}}%
\pgfpathlineto{\pgfqpoint{6.253374in}{0.550000in}}%
\pgfpathlineto{\pgfqpoint{6.253374in}{0.550000in}}%
\pgfpathclose%
\pgfusepath{fill}%
\end{pgfscope}%
\begin{pgfscope}%
\pgfpathrectangle{\pgfqpoint{1.250000in}{0.550000in}}{\pgfqpoint{7.750000in}{3.850000in}}%
\pgfusepath{clip}%
\pgfsetbuttcap%
\pgfsetmiterjoin%
\definecolor{currentfill}{rgb}{0.501961,0.501961,0.501961}%
\pgfsetfillcolor{currentfill}%
\pgfsetfillopacity{0.700000}%
\pgfsetlinewidth{0.000000pt}%
\definecolor{currentstroke}{rgb}{0.000000,0.000000,0.000000}%
\pgfsetstrokecolor{currentstroke}%
\pgfsetstrokeopacity{0.700000}%
\pgfsetdash{}{0pt}%
\pgfpathmoveto{\pgfqpoint{6.280895in}{0.550000in}}%
\pgfpathlineto{\pgfqpoint{6.308416in}{0.550000in}}%
\pgfpathlineto{\pgfqpoint{6.308416in}{1.161586in}}%
\pgfpathlineto{\pgfqpoint{6.280895in}{1.161586in}}%
\pgfpathlineto{\pgfqpoint{6.280895in}{0.550000in}}%
\pgfpathclose%
\pgfusepath{fill}%
\end{pgfscope}%
\begin{pgfscope}%
\pgfpathrectangle{\pgfqpoint{1.250000in}{0.550000in}}{\pgfqpoint{7.750000in}{3.850000in}}%
\pgfusepath{clip}%
\pgfsetbuttcap%
\pgfsetmiterjoin%
\definecolor{currentfill}{rgb}{0.501961,0.501961,0.501961}%
\pgfsetfillcolor{currentfill}%
\pgfsetfillopacity{0.700000}%
\pgfsetlinewidth{0.000000pt}%
\definecolor{currentstroke}{rgb}{0.000000,0.000000,0.000000}%
\pgfsetstrokecolor{currentstroke}%
\pgfsetstrokeopacity{0.700000}%
\pgfsetdash{}{0pt}%
\pgfpathmoveto{\pgfqpoint{6.308416in}{0.550000in}}%
\pgfpathlineto{\pgfqpoint{6.335937in}{0.550000in}}%
\pgfpathlineto{\pgfqpoint{6.335937in}{0.550000in}}%
\pgfpathlineto{\pgfqpoint{6.308416in}{0.550000in}}%
\pgfpathlineto{\pgfqpoint{6.308416in}{0.550000in}}%
\pgfpathclose%
\pgfusepath{fill}%
\end{pgfscope}%
\begin{pgfscope}%
\pgfpathrectangle{\pgfqpoint{1.250000in}{0.550000in}}{\pgfqpoint{7.750000in}{3.850000in}}%
\pgfusepath{clip}%
\pgfsetbuttcap%
\pgfsetmiterjoin%
\definecolor{currentfill}{rgb}{0.501961,0.501961,0.501961}%
\pgfsetfillcolor{currentfill}%
\pgfsetfillopacity{0.700000}%
\pgfsetlinewidth{0.000000pt}%
\definecolor{currentstroke}{rgb}{0.000000,0.000000,0.000000}%
\pgfsetstrokecolor{currentstroke}%
\pgfsetstrokeopacity{0.700000}%
\pgfsetdash{}{0pt}%
\pgfpathmoveto{\pgfqpoint{6.335938in}{0.550000in}}%
\pgfpathlineto{\pgfqpoint{6.363459in}{0.550000in}}%
\pgfpathlineto{\pgfqpoint{6.363459in}{1.110264in}}%
\pgfpathlineto{\pgfqpoint{6.335938in}{1.110264in}}%
\pgfpathlineto{\pgfqpoint{6.335938in}{0.550000in}}%
\pgfpathclose%
\pgfusepath{fill}%
\end{pgfscope}%
\begin{pgfscope}%
\pgfpathrectangle{\pgfqpoint{1.250000in}{0.550000in}}{\pgfqpoint{7.750000in}{3.850000in}}%
\pgfusepath{clip}%
\pgfsetbuttcap%
\pgfsetmiterjoin%
\definecolor{currentfill}{rgb}{0.501961,0.501961,0.501961}%
\pgfsetfillcolor{currentfill}%
\pgfsetfillopacity{0.700000}%
\pgfsetlinewidth{0.000000pt}%
\definecolor{currentstroke}{rgb}{0.000000,0.000000,0.000000}%
\pgfsetstrokecolor{currentstroke}%
\pgfsetstrokeopacity{0.700000}%
\pgfsetdash{}{0pt}%
\pgfpathmoveto{\pgfqpoint{6.363459in}{0.550000in}}%
\pgfpathlineto{\pgfqpoint{6.390980in}{0.550000in}}%
\pgfpathlineto{\pgfqpoint{6.390980in}{0.550000in}}%
\pgfpathlineto{\pgfqpoint{6.363459in}{0.550000in}}%
\pgfpathlineto{\pgfqpoint{6.363459in}{0.550000in}}%
\pgfpathclose%
\pgfusepath{fill}%
\end{pgfscope}%
\begin{pgfscope}%
\pgfpathrectangle{\pgfqpoint{1.250000in}{0.550000in}}{\pgfqpoint{7.750000in}{3.850000in}}%
\pgfusepath{clip}%
\pgfsetbuttcap%
\pgfsetmiterjoin%
\definecolor{currentfill}{rgb}{0.501961,0.501961,0.501961}%
\pgfsetfillcolor{currentfill}%
\pgfsetfillopacity{0.700000}%
\pgfsetlinewidth{0.000000pt}%
\definecolor{currentstroke}{rgb}{0.000000,0.000000,0.000000}%
\pgfsetstrokecolor{currentstroke}%
\pgfsetstrokeopacity{0.700000}%
\pgfsetdash{}{0pt}%
\pgfpathmoveto{\pgfqpoint{6.390980in}{0.550000in}}%
\pgfpathlineto{\pgfqpoint{6.418501in}{0.550000in}}%
\pgfpathlineto{\pgfqpoint{6.418501in}{1.125946in}}%
\pgfpathlineto{\pgfqpoint{6.390980in}{1.125946in}}%
\pgfpathlineto{\pgfqpoint{6.390980in}{0.550000in}}%
\pgfpathclose%
\pgfusepath{fill}%
\end{pgfscope}%
\begin{pgfscope}%
\pgfpathrectangle{\pgfqpoint{1.250000in}{0.550000in}}{\pgfqpoint{7.750000in}{3.850000in}}%
\pgfusepath{clip}%
\pgfsetbuttcap%
\pgfsetmiterjoin%
\definecolor{currentfill}{rgb}{0.501961,0.501961,0.501961}%
\pgfsetfillcolor{currentfill}%
\pgfsetfillopacity{0.700000}%
\pgfsetlinewidth{0.000000pt}%
\definecolor{currentstroke}{rgb}{0.000000,0.000000,0.000000}%
\pgfsetstrokecolor{currentstroke}%
\pgfsetstrokeopacity{0.700000}%
\pgfsetdash{}{0pt}%
\pgfpathmoveto{\pgfqpoint{6.418501in}{0.550000in}}%
\pgfpathlineto{\pgfqpoint{6.446023in}{0.550000in}}%
\pgfpathlineto{\pgfqpoint{6.446023in}{0.550000in}}%
\pgfpathlineto{\pgfqpoint{6.418501in}{0.550000in}}%
\pgfpathlineto{\pgfqpoint{6.418501in}{0.550000in}}%
\pgfpathclose%
\pgfusepath{fill}%
\end{pgfscope}%
\begin{pgfscope}%
\pgfpathrectangle{\pgfqpoint{1.250000in}{0.550000in}}{\pgfqpoint{7.750000in}{3.850000in}}%
\pgfusepath{clip}%
\pgfsetbuttcap%
\pgfsetmiterjoin%
\definecolor{currentfill}{rgb}{0.501961,0.501961,0.501961}%
\pgfsetfillcolor{currentfill}%
\pgfsetfillopacity{0.700000}%
\pgfsetlinewidth{0.000000pt}%
\definecolor{currentstroke}{rgb}{0.000000,0.000000,0.000000}%
\pgfsetstrokecolor{currentstroke}%
\pgfsetstrokeopacity{0.700000}%
\pgfsetdash{}{0pt}%
\pgfpathmoveto{\pgfqpoint{6.446023in}{0.550000in}}%
\pgfpathlineto{\pgfqpoint{6.473544in}{0.550000in}}%
\pgfpathlineto{\pgfqpoint{6.473544in}{1.086029in}}%
\pgfpathlineto{\pgfqpoint{6.446023in}{1.086029in}}%
\pgfpathlineto{\pgfqpoint{6.446023in}{0.550000in}}%
\pgfpathclose%
\pgfusepath{fill}%
\end{pgfscope}%
\begin{pgfscope}%
\pgfpathrectangle{\pgfqpoint{1.250000in}{0.550000in}}{\pgfqpoint{7.750000in}{3.850000in}}%
\pgfusepath{clip}%
\pgfsetbuttcap%
\pgfsetmiterjoin%
\definecolor{currentfill}{rgb}{0.501961,0.501961,0.501961}%
\pgfsetfillcolor{currentfill}%
\pgfsetfillopacity{0.700000}%
\pgfsetlinewidth{0.000000pt}%
\definecolor{currentstroke}{rgb}{0.000000,0.000000,0.000000}%
\pgfsetstrokecolor{currentstroke}%
\pgfsetstrokeopacity{0.700000}%
\pgfsetdash{}{0pt}%
\pgfpathmoveto{\pgfqpoint{6.473544in}{0.550000in}}%
\pgfpathlineto{\pgfqpoint{6.501065in}{0.550000in}}%
\pgfpathlineto{\pgfqpoint{6.501065in}{0.550000in}}%
\pgfpathlineto{\pgfqpoint{6.473544in}{0.550000in}}%
\pgfpathlineto{\pgfqpoint{6.473544in}{0.550000in}}%
\pgfpathclose%
\pgfusepath{fill}%
\end{pgfscope}%
\begin{pgfscope}%
\pgfpathrectangle{\pgfqpoint{1.250000in}{0.550000in}}{\pgfqpoint{7.750000in}{3.850000in}}%
\pgfusepath{clip}%
\pgfsetbuttcap%
\pgfsetmiterjoin%
\definecolor{currentfill}{rgb}{0.501961,0.501961,0.501961}%
\pgfsetfillcolor{currentfill}%
\pgfsetfillopacity{0.700000}%
\pgfsetlinewidth{0.000000pt}%
\definecolor{currentstroke}{rgb}{0.000000,0.000000,0.000000}%
\pgfsetstrokecolor{currentstroke}%
\pgfsetstrokeopacity{0.700000}%
\pgfsetdash{}{0pt}%
\pgfpathmoveto{\pgfqpoint{6.501065in}{0.550000in}}%
\pgfpathlineto{\pgfqpoint{6.528587in}{0.550000in}}%
\pgfpathlineto{\pgfqpoint{6.528587in}{1.103136in}}%
\pgfpathlineto{\pgfqpoint{6.501065in}{1.103136in}}%
\pgfpathlineto{\pgfqpoint{6.501065in}{0.550000in}}%
\pgfpathclose%
\pgfusepath{fill}%
\end{pgfscope}%
\begin{pgfscope}%
\pgfpathrectangle{\pgfqpoint{1.250000in}{0.550000in}}{\pgfqpoint{7.750000in}{3.850000in}}%
\pgfusepath{clip}%
\pgfsetbuttcap%
\pgfsetmiterjoin%
\definecolor{currentfill}{rgb}{0.501961,0.501961,0.501961}%
\pgfsetfillcolor{currentfill}%
\pgfsetfillopacity{0.700000}%
\pgfsetlinewidth{0.000000pt}%
\definecolor{currentstroke}{rgb}{0.000000,0.000000,0.000000}%
\pgfsetstrokecolor{currentstroke}%
\pgfsetstrokeopacity{0.700000}%
\pgfsetdash{}{0pt}%
\pgfpathmoveto{\pgfqpoint{6.528587in}{0.550000in}}%
\pgfpathlineto{\pgfqpoint{6.556108in}{0.550000in}}%
\pgfpathlineto{\pgfqpoint{6.556108in}{0.550000in}}%
\pgfpathlineto{\pgfqpoint{6.528587in}{0.550000in}}%
\pgfpathlineto{\pgfqpoint{6.528587in}{0.550000in}}%
\pgfpathclose%
\pgfusepath{fill}%
\end{pgfscope}%
\begin{pgfscope}%
\pgfpathrectangle{\pgfqpoint{1.250000in}{0.550000in}}{\pgfqpoint{7.750000in}{3.850000in}}%
\pgfusepath{clip}%
\pgfsetbuttcap%
\pgfsetmiterjoin%
\definecolor{currentfill}{rgb}{0.501961,0.501961,0.501961}%
\pgfsetfillcolor{currentfill}%
\pgfsetfillopacity{0.700000}%
\pgfsetlinewidth{0.000000pt}%
\definecolor{currentstroke}{rgb}{0.000000,0.000000,0.000000}%
\pgfsetstrokecolor{currentstroke}%
\pgfsetstrokeopacity{0.700000}%
\pgfsetdash{}{0pt}%
\pgfpathmoveto{\pgfqpoint{6.556108in}{0.550000in}}%
\pgfpathlineto{\pgfqpoint{6.583629in}{0.550000in}}%
\pgfpathlineto{\pgfqpoint{6.583629in}{1.044686in}}%
\pgfpathlineto{\pgfqpoint{6.556108in}{1.044686in}}%
\pgfpathlineto{\pgfqpoint{6.556108in}{0.550000in}}%
\pgfpathclose%
\pgfusepath{fill}%
\end{pgfscope}%
\begin{pgfscope}%
\pgfpathrectangle{\pgfqpoint{1.250000in}{0.550000in}}{\pgfqpoint{7.750000in}{3.850000in}}%
\pgfusepath{clip}%
\pgfsetbuttcap%
\pgfsetmiterjoin%
\definecolor{currentfill}{rgb}{0.501961,0.501961,0.501961}%
\pgfsetfillcolor{currentfill}%
\pgfsetfillopacity{0.700000}%
\pgfsetlinewidth{0.000000pt}%
\definecolor{currentstroke}{rgb}{0.000000,0.000000,0.000000}%
\pgfsetstrokecolor{currentstroke}%
\pgfsetstrokeopacity{0.700000}%
\pgfsetdash{}{0pt}%
\pgfpathmoveto{\pgfqpoint{6.583629in}{0.550000in}}%
\pgfpathlineto{\pgfqpoint{6.611151in}{0.550000in}}%
\pgfpathlineto{\pgfqpoint{6.611151in}{0.550000in}}%
\pgfpathlineto{\pgfqpoint{6.583629in}{0.550000in}}%
\pgfpathlineto{\pgfqpoint{6.583629in}{0.550000in}}%
\pgfpathclose%
\pgfusepath{fill}%
\end{pgfscope}%
\begin{pgfscope}%
\pgfpathrectangle{\pgfqpoint{1.250000in}{0.550000in}}{\pgfqpoint{7.750000in}{3.850000in}}%
\pgfusepath{clip}%
\pgfsetbuttcap%
\pgfsetmiterjoin%
\definecolor{currentfill}{rgb}{0.501961,0.501961,0.501961}%
\pgfsetfillcolor{currentfill}%
\pgfsetfillopacity{0.700000}%
\pgfsetlinewidth{0.000000pt}%
\definecolor{currentstroke}{rgb}{0.000000,0.000000,0.000000}%
\pgfsetstrokecolor{currentstroke}%
\pgfsetstrokeopacity{0.700000}%
\pgfsetdash{}{0pt}%
\pgfpathmoveto{\pgfqpoint{6.611151in}{0.550000in}}%
\pgfpathlineto{\pgfqpoint{6.638672in}{0.550000in}}%
\pgfpathlineto{\pgfqpoint{6.638672in}{1.033281in}}%
\pgfpathlineto{\pgfqpoint{6.611151in}{1.033281in}}%
\pgfpathlineto{\pgfqpoint{6.611151in}{0.550000in}}%
\pgfpathclose%
\pgfusepath{fill}%
\end{pgfscope}%
\begin{pgfscope}%
\pgfpathrectangle{\pgfqpoint{1.250000in}{0.550000in}}{\pgfqpoint{7.750000in}{3.850000in}}%
\pgfusepath{clip}%
\pgfsetbuttcap%
\pgfsetmiterjoin%
\definecolor{currentfill}{rgb}{0.501961,0.501961,0.501961}%
\pgfsetfillcolor{currentfill}%
\pgfsetfillopacity{0.700000}%
\pgfsetlinewidth{0.000000pt}%
\definecolor{currentstroke}{rgb}{0.000000,0.000000,0.000000}%
\pgfsetstrokecolor{currentstroke}%
\pgfsetstrokeopacity{0.700000}%
\pgfsetdash{}{0pt}%
\pgfpathmoveto{\pgfqpoint{6.638672in}{0.550000in}}%
\pgfpathlineto{\pgfqpoint{6.666193in}{0.550000in}}%
\pgfpathlineto{\pgfqpoint{6.666193in}{0.550000in}}%
\pgfpathlineto{\pgfqpoint{6.638672in}{0.550000in}}%
\pgfpathlineto{\pgfqpoint{6.638672in}{0.550000in}}%
\pgfpathclose%
\pgfusepath{fill}%
\end{pgfscope}%
\begin{pgfscope}%
\pgfpathrectangle{\pgfqpoint{1.250000in}{0.550000in}}{\pgfqpoint{7.750000in}{3.850000in}}%
\pgfusepath{clip}%
\pgfsetbuttcap%
\pgfsetmiterjoin%
\definecolor{currentfill}{rgb}{0.501961,0.501961,0.501961}%
\pgfsetfillcolor{currentfill}%
\pgfsetfillopacity{0.700000}%
\pgfsetlinewidth{0.000000pt}%
\definecolor{currentstroke}{rgb}{0.000000,0.000000,0.000000}%
\pgfsetstrokecolor{currentstroke}%
\pgfsetstrokeopacity{0.700000}%
\pgfsetdash{}{0pt}%
\pgfpathmoveto{\pgfqpoint{6.666193in}{0.550000in}}%
\pgfpathlineto{\pgfqpoint{6.693714in}{0.550000in}}%
\pgfpathlineto{\pgfqpoint{6.693714in}{1.063219in}}%
\pgfpathlineto{\pgfqpoint{6.666193in}{1.063219in}}%
\pgfpathlineto{\pgfqpoint{6.666193in}{0.550000in}}%
\pgfpathclose%
\pgfusepath{fill}%
\end{pgfscope}%
\begin{pgfscope}%
\pgfpathrectangle{\pgfqpoint{1.250000in}{0.550000in}}{\pgfqpoint{7.750000in}{3.850000in}}%
\pgfusepath{clip}%
\pgfsetbuttcap%
\pgfsetmiterjoin%
\definecolor{currentfill}{rgb}{0.501961,0.501961,0.501961}%
\pgfsetfillcolor{currentfill}%
\pgfsetfillopacity{0.700000}%
\pgfsetlinewidth{0.000000pt}%
\definecolor{currentstroke}{rgb}{0.000000,0.000000,0.000000}%
\pgfsetstrokecolor{currentstroke}%
\pgfsetstrokeopacity{0.700000}%
\pgfsetdash{}{0pt}%
\pgfpathmoveto{\pgfqpoint{6.693714in}{0.550000in}}%
\pgfpathlineto{\pgfqpoint{6.721236in}{0.550000in}}%
\pgfpathlineto{\pgfqpoint{6.721236in}{0.550000in}}%
\pgfpathlineto{\pgfqpoint{6.693714in}{0.550000in}}%
\pgfpathlineto{\pgfqpoint{6.693714in}{0.550000in}}%
\pgfpathclose%
\pgfusepath{fill}%
\end{pgfscope}%
\begin{pgfscope}%
\pgfpathrectangle{\pgfqpoint{1.250000in}{0.550000in}}{\pgfqpoint{7.750000in}{3.850000in}}%
\pgfusepath{clip}%
\pgfsetbuttcap%
\pgfsetmiterjoin%
\definecolor{currentfill}{rgb}{0.501961,0.501961,0.501961}%
\pgfsetfillcolor{currentfill}%
\pgfsetfillopacity{0.700000}%
\pgfsetlinewidth{0.000000pt}%
\definecolor{currentstroke}{rgb}{0.000000,0.000000,0.000000}%
\pgfsetstrokecolor{currentstroke}%
\pgfsetstrokeopacity{0.700000}%
\pgfsetdash{}{0pt}%
\pgfpathmoveto{\pgfqpoint{6.721236in}{0.550000in}}%
\pgfpathlineto{\pgfqpoint{6.748757in}{0.550000in}}%
\pgfpathlineto{\pgfqpoint{6.748757in}{1.017600in}}%
\pgfpathlineto{\pgfqpoint{6.721236in}{1.017600in}}%
\pgfpathlineto{\pgfqpoint{6.721236in}{0.550000in}}%
\pgfpathclose%
\pgfusepath{fill}%
\end{pgfscope}%
\begin{pgfscope}%
\pgfpathrectangle{\pgfqpoint{1.250000in}{0.550000in}}{\pgfqpoint{7.750000in}{3.850000in}}%
\pgfusepath{clip}%
\pgfsetbuttcap%
\pgfsetmiterjoin%
\definecolor{currentfill}{rgb}{0.501961,0.501961,0.501961}%
\pgfsetfillcolor{currentfill}%
\pgfsetfillopacity{0.700000}%
\pgfsetlinewidth{0.000000pt}%
\definecolor{currentstroke}{rgb}{0.000000,0.000000,0.000000}%
\pgfsetstrokecolor{currentstroke}%
\pgfsetstrokeopacity{0.700000}%
\pgfsetdash{}{0pt}%
\pgfpathmoveto{\pgfqpoint{6.748757in}{0.550000in}}%
\pgfpathlineto{\pgfqpoint{6.776278in}{0.550000in}}%
\pgfpathlineto{\pgfqpoint{6.776278in}{0.550000in}}%
\pgfpathlineto{\pgfqpoint{6.748757in}{0.550000in}}%
\pgfpathlineto{\pgfqpoint{6.748757in}{0.550000in}}%
\pgfpathclose%
\pgfusepath{fill}%
\end{pgfscope}%
\begin{pgfscope}%
\pgfpathrectangle{\pgfqpoint{1.250000in}{0.550000in}}{\pgfqpoint{7.750000in}{3.850000in}}%
\pgfusepath{clip}%
\pgfsetbuttcap%
\pgfsetmiterjoin%
\definecolor{currentfill}{rgb}{0.501961,0.501961,0.501961}%
\pgfsetfillcolor{currentfill}%
\pgfsetfillopacity{0.700000}%
\pgfsetlinewidth{0.000000pt}%
\definecolor{currentstroke}{rgb}{0.000000,0.000000,0.000000}%
\pgfsetstrokecolor{currentstroke}%
\pgfsetstrokeopacity{0.700000}%
\pgfsetdash{}{0pt}%
\pgfpathmoveto{\pgfqpoint{6.776278in}{0.550000in}}%
\pgfpathlineto{\pgfqpoint{6.803800in}{0.550000in}}%
\pgfpathlineto{\pgfqpoint{6.803800in}{0.977683in}}%
\pgfpathlineto{\pgfqpoint{6.776278in}{0.977683in}}%
\pgfpathlineto{\pgfqpoint{6.776278in}{0.550000in}}%
\pgfpathclose%
\pgfusepath{fill}%
\end{pgfscope}%
\begin{pgfscope}%
\pgfpathrectangle{\pgfqpoint{1.250000in}{0.550000in}}{\pgfqpoint{7.750000in}{3.850000in}}%
\pgfusepath{clip}%
\pgfsetbuttcap%
\pgfsetmiterjoin%
\definecolor{currentfill}{rgb}{0.501961,0.501961,0.501961}%
\pgfsetfillcolor{currentfill}%
\pgfsetfillopacity{0.700000}%
\pgfsetlinewidth{0.000000pt}%
\definecolor{currentstroke}{rgb}{0.000000,0.000000,0.000000}%
\pgfsetstrokecolor{currentstroke}%
\pgfsetstrokeopacity{0.700000}%
\pgfsetdash{}{0pt}%
\pgfpathmoveto{\pgfqpoint{6.803800in}{0.550000in}}%
\pgfpathlineto{\pgfqpoint{6.831321in}{0.550000in}}%
\pgfpathlineto{\pgfqpoint{6.831321in}{0.550000in}}%
\pgfpathlineto{\pgfqpoint{6.803800in}{0.550000in}}%
\pgfpathlineto{\pgfqpoint{6.803800in}{0.550000in}}%
\pgfpathclose%
\pgfusepath{fill}%
\end{pgfscope}%
\begin{pgfscope}%
\pgfpathrectangle{\pgfqpoint{1.250000in}{0.550000in}}{\pgfqpoint{7.750000in}{3.850000in}}%
\pgfusepath{clip}%
\pgfsetbuttcap%
\pgfsetmiterjoin%
\definecolor{currentfill}{rgb}{0.501961,0.501961,0.501961}%
\pgfsetfillcolor{currentfill}%
\pgfsetfillopacity{0.700000}%
\pgfsetlinewidth{0.000000pt}%
\definecolor{currentstroke}{rgb}{0.000000,0.000000,0.000000}%
\pgfsetstrokecolor{currentstroke}%
\pgfsetstrokeopacity{0.700000}%
\pgfsetdash{}{0pt}%
\pgfpathmoveto{\pgfqpoint{6.831321in}{0.550000in}}%
\pgfpathlineto{\pgfqpoint{6.858842in}{0.550000in}}%
\pgfpathlineto{\pgfqpoint{6.858842in}{1.001918in}}%
\pgfpathlineto{\pgfqpoint{6.831321in}{1.001918in}}%
\pgfpathlineto{\pgfqpoint{6.831321in}{0.550000in}}%
\pgfpathclose%
\pgfusepath{fill}%
\end{pgfscope}%
\begin{pgfscope}%
\pgfpathrectangle{\pgfqpoint{1.250000in}{0.550000in}}{\pgfqpoint{7.750000in}{3.850000in}}%
\pgfusepath{clip}%
\pgfsetbuttcap%
\pgfsetmiterjoin%
\definecolor{currentfill}{rgb}{0.501961,0.501961,0.501961}%
\pgfsetfillcolor{currentfill}%
\pgfsetfillopacity{0.700000}%
\pgfsetlinewidth{0.000000pt}%
\definecolor{currentstroke}{rgb}{0.000000,0.000000,0.000000}%
\pgfsetstrokecolor{currentstroke}%
\pgfsetstrokeopacity{0.700000}%
\pgfsetdash{}{0pt}%
\pgfpathmoveto{\pgfqpoint{6.858842in}{0.550000in}}%
\pgfpathlineto{\pgfqpoint{6.886364in}{0.550000in}}%
\pgfpathlineto{\pgfqpoint{6.886364in}{0.550000in}}%
\pgfpathlineto{\pgfqpoint{6.858842in}{0.550000in}}%
\pgfpathlineto{\pgfqpoint{6.858842in}{0.550000in}}%
\pgfpathclose%
\pgfusepath{fill}%
\end{pgfscope}%
\begin{pgfscope}%
\pgfpathrectangle{\pgfqpoint{1.250000in}{0.550000in}}{\pgfqpoint{7.750000in}{3.850000in}}%
\pgfusepath{clip}%
\pgfsetbuttcap%
\pgfsetmiterjoin%
\definecolor{currentfill}{rgb}{0.501961,0.501961,0.501961}%
\pgfsetfillcolor{currentfill}%
\pgfsetfillopacity{0.700000}%
\pgfsetlinewidth{0.000000pt}%
\definecolor{currentstroke}{rgb}{0.000000,0.000000,0.000000}%
\pgfsetstrokecolor{currentstroke}%
\pgfsetstrokeopacity{0.700000}%
\pgfsetdash{}{0pt}%
\pgfpathmoveto{\pgfqpoint{6.886364in}{0.550000in}}%
\pgfpathlineto{\pgfqpoint{6.913885in}{0.550000in}}%
\pgfpathlineto{\pgfqpoint{6.913885in}{1.009046in}}%
\pgfpathlineto{\pgfqpoint{6.886364in}{1.009046in}}%
\pgfpathlineto{\pgfqpoint{6.886364in}{0.550000in}}%
\pgfpathclose%
\pgfusepath{fill}%
\end{pgfscope}%
\begin{pgfscope}%
\pgfpathrectangle{\pgfqpoint{1.250000in}{0.550000in}}{\pgfqpoint{7.750000in}{3.850000in}}%
\pgfusepath{clip}%
\pgfsetbuttcap%
\pgfsetmiterjoin%
\definecolor{currentfill}{rgb}{0.501961,0.501961,0.501961}%
\pgfsetfillcolor{currentfill}%
\pgfsetfillopacity{0.700000}%
\pgfsetlinewidth{0.000000pt}%
\definecolor{currentstroke}{rgb}{0.000000,0.000000,0.000000}%
\pgfsetstrokecolor{currentstroke}%
\pgfsetstrokeopacity{0.700000}%
\pgfsetdash{}{0pt}%
\pgfpathmoveto{\pgfqpoint{6.913885in}{0.550000in}}%
\pgfpathlineto{\pgfqpoint{6.941406in}{0.550000in}}%
\pgfpathlineto{\pgfqpoint{6.941406in}{0.550000in}}%
\pgfpathlineto{\pgfqpoint{6.913885in}{0.550000in}}%
\pgfpathlineto{\pgfqpoint{6.913885in}{0.550000in}}%
\pgfpathclose%
\pgfusepath{fill}%
\end{pgfscope}%
\begin{pgfscope}%
\pgfpathrectangle{\pgfqpoint{1.250000in}{0.550000in}}{\pgfqpoint{7.750000in}{3.850000in}}%
\pgfusepath{clip}%
\pgfsetbuttcap%
\pgfsetmiterjoin%
\definecolor{currentfill}{rgb}{0.501961,0.501961,0.501961}%
\pgfsetfillcolor{currentfill}%
\pgfsetfillopacity{0.700000}%
\pgfsetlinewidth{0.000000pt}%
\definecolor{currentstroke}{rgb}{0.000000,0.000000,0.000000}%
\pgfsetstrokecolor{currentstroke}%
\pgfsetstrokeopacity{0.700000}%
\pgfsetdash{}{0pt}%
\pgfpathmoveto{\pgfqpoint{6.941406in}{0.550000in}}%
\pgfpathlineto{\pgfqpoint{6.968928in}{0.550000in}}%
\pgfpathlineto{\pgfqpoint{6.968928in}{0.990513in}}%
\pgfpathlineto{\pgfqpoint{6.941406in}{0.990513in}}%
\pgfpathlineto{\pgfqpoint{6.941406in}{0.550000in}}%
\pgfpathclose%
\pgfusepath{fill}%
\end{pgfscope}%
\begin{pgfscope}%
\pgfpathrectangle{\pgfqpoint{1.250000in}{0.550000in}}{\pgfqpoint{7.750000in}{3.850000in}}%
\pgfusepath{clip}%
\pgfsetbuttcap%
\pgfsetmiterjoin%
\definecolor{currentfill}{rgb}{0.501961,0.501961,0.501961}%
\pgfsetfillcolor{currentfill}%
\pgfsetfillopacity{0.700000}%
\pgfsetlinewidth{0.000000pt}%
\definecolor{currentstroke}{rgb}{0.000000,0.000000,0.000000}%
\pgfsetstrokecolor{currentstroke}%
\pgfsetstrokeopacity{0.700000}%
\pgfsetdash{}{0pt}%
\pgfpathmoveto{\pgfqpoint{6.968928in}{0.550000in}}%
\pgfpathlineto{\pgfqpoint{6.996449in}{0.550000in}}%
\pgfpathlineto{\pgfqpoint{6.996449in}{0.550000in}}%
\pgfpathlineto{\pgfqpoint{6.968928in}{0.550000in}}%
\pgfpathlineto{\pgfqpoint{6.968928in}{0.550000in}}%
\pgfpathclose%
\pgfusepath{fill}%
\end{pgfscope}%
\begin{pgfscope}%
\pgfpathrectangle{\pgfqpoint{1.250000in}{0.550000in}}{\pgfqpoint{7.750000in}{3.850000in}}%
\pgfusepath{clip}%
\pgfsetbuttcap%
\pgfsetmiterjoin%
\definecolor{currentfill}{rgb}{0.501961,0.501961,0.501961}%
\pgfsetfillcolor{currentfill}%
\pgfsetfillopacity{0.700000}%
\pgfsetlinewidth{0.000000pt}%
\definecolor{currentstroke}{rgb}{0.000000,0.000000,0.000000}%
\pgfsetstrokecolor{currentstroke}%
\pgfsetstrokeopacity{0.700000}%
\pgfsetdash{}{0pt}%
\pgfpathmoveto{\pgfqpoint{6.996449in}{0.550000in}}%
\pgfpathlineto{\pgfqpoint{7.023970in}{0.550000in}}%
\pgfpathlineto{\pgfqpoint{7.023970in}{0.943468in}}%
\pgfpathlineto{\pgfqpoint{6.996449in}{0.943468in}}%
\pgfpathlineto{\pgfqpoint{6.996449in}{0.550000in}}%
\pgfpathclose%
\pgfusepath{fill}%
\end{pgfscope}%
\begin{pgfscope}%
\pgfpathrectangle{\pgfqpoint{1.250000in}{0.550000in}}{\pgfqpoint{7.750000in}{3.850000in}}%
\pgfusepath{clip}%
\pgfsetbuttcap%
\pgfsetmiterjoin%
\definecolor{currentfill}{rgb}{0.501961,0.501961,0.501961}%
\pgfsetfillcolor{currentfill}%
\pgfsetfillopacity{0.700000}%
\pgfsetlinewidth{0.000000pt}%
\definecolor{currentstroke}{rgb}{0.000000,0.000000,0.000000}%
\pgfsetstrokecolor{currentstroke}%
\pgfsetstrokeopacity{0.700000}%
\pgfsetdash{}{0pt}%
\pgfpathmoveto{\pgfqpoint{7.023970in}{0.550000in}}%
\pgfpathlineto{\pgfqpoint{7.051491in}{0.550000in}}%
\pgfpathlineto{\pgfqpoint{7.051491in}{0.550000in}}%
\pgfpathlineto{\pgfqpoint{7.023970in}{0.550000in}}%
\pgfpathlineto{\pgfqpoint{7.023970in}{0.550000in}}%
\pgfpathclose%
\pgfusepath{fill}%
\end{pgfscope}%
\begin{pgfscope}%
\pgfpathrectangle{\pgfqpoint{1.250000in}{0.550000in}}{\pgfqpoint{7.750000in}{3.850000in}}%
\pgfusepath{clip}%
\pgfsetbuttcap%
\pgfsetmiterjoin%
\definecolor{currentfill}{rgb}{0.501961,0.501961,0.501961}%
\pgfsetfillcolor{currentfill}%
\pgfsetfillopacity{0.700000}%
\pgfsetlinewidth{0.000000pt}%
\definecolor{currentstroke}{rgb}{0.000000,0.000000,0.000000}%
\pgfsetstrokecolor{currentstroke}%
\pgfsetstrokeopacity{0.700000}%
\pgfsetdash{}{0pt}%
\pgfpathmoveto{\pgfqpoint{7.051491in}{0.550000in}}%
\pgfpathlineto{\pgfqpoint{7.079013in}{0.550000in}}%
\pgfpathlineto{\pgfqpoint{7.079013in}{0.974832in}}%
\pgfpathlineto{\pgfqpoint{7.051491in}{0.974832in}}%
\pgfpathlineto{\pgfqpoint{7.051491in}{0.550000in}}%
\pgfpathclose%
\pgfusepath{fill}%
\end{pgfscope}%
\begin{pgfscope}%
\pgfpathrectangle{\pgfqpoint{1.250000in}{0.550000in}}{\pgfqpoint{7.750000in}{3.850000in}}%
\pgfusepath{clip}%
\pgfsetbuttcap%
\pgfsetmiterjoin%
\definecolor{currentfill}{rgb}{0.501961,0.501961,0.501961}%
\pgfsetfillcolor{currentfill}%
\pgfsetfillopacity{0.700000}%
\pgfsetlinewidth{0.000000pt}%
\definecolor{currentstroke}{rgb}{0.000000,0.000000,0.000000}%
\pgfsetstrokecolor{currentstroke}%
\pgfsetstrokeopacity{0.700000}%
\pgfsetdash{}{0pt}%
\pgfpathmoveto{\pgfqpoint{7.079013in}{0.550000in}}%
\pgfpathlineto{\pgfqpoint{7.106534in}{0.550000in}}%
\pgfpathlineto{\pgfqpoint{7.106534in}{0.550000in}}%
\pgfpathlineto{\pgfqpoint{7.079013in}{0.550000in}}%
\pgfpathlineto{\pgfqpoint{7.079013in}{0.550000in}}%
\pgfpathclose%
\pgfusepath{fill}%
\end{pgfscope}%
\begin{pgfscope}%
\pgfpathrectangle{\pgfqpoint{1.250000in}{0.550000in}}{\pgfqpoint{7.750000in}{3.850000in}}%
\pgfusepath{clip}%
\pgfsetbuttcap%
\pgfsetmiterjoin%
\definecolor{currentfill}{rgb}{0.501961,0.501961,0.501961}%
\pgfsetfillcolor{currentfill}%
\pgfsetfillopacity{0.700000}%
\pgfsetlinewidth{0.000000pt}%
\definecolor{currentstroke}{rgb}{0.000000,0.000000,0.000000}%
\pgfsetstrokecolor{currentstroke}%
\pgfsetstrokeopacity{0.700000}%
\pgfsetdash{}{0pt}%
\pgfpathmoveto{\pgfqpoint{7.106534in}{0.550000in}}%
\pgfpathlineto{\pgfqpoint{7.134055in}{0.550000in}}%
\pgfpathlineto{\pgfqpoint{7.134055in}{0.953447in}}%
\pgfpathlineto{\pgfqpoint{7.106534in}{0.953447in}}%
\pgfpathlineto{\pgfqpoint{7.106534in}{0.550000in}}%
\pgfpathclose%
\pgfusepath{fill}%
\end{pgfscope}%
\begin{pgfscope}%
\pgfpathrectangle{\pgfqpoint{1.250000in}{0.550000in}}{\pgfqpoint{7.750000in}{3.850000in}}%
\pgfusepath{clip}%
\pgfsetbuttcap%
\pgfsetmiterjoin%
\definecolor{currentfill}{rgb}{0.501961,0.501961,0.501961}%
\pgfsetfillcolor{currentfill}%
\pgfsetfillopacity{0.700000}%
\pgfsetlinewidth{0.000000pt}%
\definecolor{currentstroke}{rgb}{0.000000,0.000000,0.000000}%
\pgfsetstrokecolor{currentstroke}%
\pgfsetstrokeopacity{0.700000}%
\pgfsetdash{}{0pt}%
\pgfpathmoveto{\pgfqpoint{7.134055in}{0.550000in}}%
\pgfpathlineto{\pgfqpoint{7.161577in}{0.550000in}}%
\pgfpathlineto{\pgfqpoint{7.161577in}{0.550000in}}%
\pgfpathlineto{\pgfqpoint{7.134055in}{0.550000in}}%
\pgfpathlineto{\pgfqpoint{7.134055in}{0.550000in}}%
\pgfpathclose%
\pgfusepath{fill}%
\end{pgfscope}%
\begin{pgfscope}%
\pgfpathrectangle{\pgfqpoint{1.250000in}{0.550000in}}{\pgfqpoint{7.750000in}{3.850000in}}%
\pgfusepath{clip}%
\pgfsetbuttcap%
\pgfsetmiterjoin%
\definecolor{currentfill}{rgb}{0.501961,0.501961,0.501961}%
\pgfsetfillcolor{currentfill}%
\pgfsetfillopacity{0.700000}%
\pgfsetlinewidth{0.000000pt}%
\definecolor{currentstroke}{rgb}{0.000000,0.000000,0.000000}%
\pgfsetstrokecolor{currentstroke}%
\pgfsetstrokeopacity{0.700000}%
\pgfsetdash{}{0pt}%
\pgfpathmoveto{\pgfqpoint{7.161577in}{0.550000in}}%
\pgfpathlineto{\pgfqpoint{7.189098in}{0.550000in}}%
\pgfpathlineto{\pgfqpoint{7.189098in}{0.949171in}}%
\pgfpathlineto{\pgfqpoint{7.161577in}{0.949171in}}%
\pgfpathlineto{\pgfqpoint{7.161577in}{0.550000in}}%
\pgfpathclose%
\pgfusepath{fill}%
\end{pgfscope}%
\begin{pgfscope}%
\pgfpathrectangle{\pgfqpoint{1.250000in}{0.550000in}}{\pgfqpoint{7.750000in}{3.850000in}}%
\pgfusepath{clip}%
\pgfsetbuttcap%
\pgfsetmiterjoin%
\definecolor{currentfill}{rgb}{0.501961,0.501961,0.501961}%
\pgfsetfillcolor{currentfill}%
\pgfsetfillopacity{0.700000}%
\pgfsetlinewidth{0.000000pt}%
\definecolor{currentstroke}{rgb}{0.000000,0.000000,0.000000}%
\pgfsetstrokecolor{currentstroke}%
\pgfsetstrokeopacity{0.700000}%
\pgfsetdash{}{0pt}%
\pgfpathmoveto{\pgfqpoint{7.189098in}{0.550000in}}%
\pgfpathlineto{\pgfqpoint{7.216619in}{0.550000in}}%
\pgfpathlineto{\pgfqpoint{7.216619in}{0.550000in}}%
\pgfpathlineto{\pgfqpoint{7.189098in}{0.550000in}}%
\pgfpathlineto{\pgfqpoint{7.189098in}{0.550000in}}%
\pgfpathclose%
\pgfusepath{fill}%
\end{pgfscope}%
\begin{pgfscope}%
\pgfpathrectangle{\pgfqpoint{1.250000in}{0.550000in}}{\pgfqpoint{7.750000in}{3.850000in}}%
\pgfusepath{clip}%
\pgfsetbuttcap%
\pgfsetmiterjoin%
\definecolor{currentfill}{rgb}{0.501961,0.501961,0.501961}%
\pgfsetfillcolor{currentfill}%
\pgfsetfillopacity{0.700000}%
\pgfsetlinewidth{0.000000pt}%
\definecolor{currentstroke}{rgb}{0.000000,0.000000,0.000000}%
\pgfsetstrokecolor{currentstroke}%
\pgfsetstrokeopacity{0.700000}%
\pgfsetdash{}{0pt}%
\pgfpathmoveto{\pgfqpoint{7.216619in}{0.550000in}}%
\pgfpathlineto{\pgfqpoint{7.244141in}{0.550000in}}%
\pgfpathlineto{\pgfqpoint{7.244141in}{0.934914in}}%
\pgfpathlineto{\pgfqpoint{7.216619in}{0.934914in}}%
\pgfpathlineto{\pgfqpoint{7.216619in}{0.550000in}}%
\pgfpathclose%
\pgfusepath{fill}%
\end{pgfscope}%
\begin{pgfscope}%
\pgfpathrectangle{\pgfqpoint{1.250000in}{0.550000in}}{\pgfqpoint{7.750000in}{3.850000in}}%
\pgfusepath{clip}%
\pgfsetbuttcap%
\pgfsetmiterjoin%
\definecolor{currentfill}{rgb}{0.501961,0.501961,0.501961}%
\pgfsetfillcolor{currentfill}%
\pgfsetfillopacity{0.700000}%
\pgfsetlinewidth{0.000000pt}%
\definecolor{currentstroke}{rgb}{0.000000,0.000000,0.000000}%
\pgfsetstrokecolor{currentstroke}%
\pgfsetstrokeopacity{0.700000}%
\pgfsetdash{}{0pt}%
\pgfpathmoveto{\pgfqpoint{7.244141in}{0.550000in}}%
\pgfpathlineto{\pgfqpoint{7.271662in}{0.550000in}}%
\pgfpathlineto{\pgfqpoint{7.271662in}{0.550000in}}%
\pgfpathlineto{\pgfqpoint{7.244141in}{0.550000in}}%
\pgfpathlineto{\pgfqpoint{7.244141in}{0.550000in}}%
\pgfpathclose%
\pgfusepath{fill}%
\end{pgfscope}%
\begin{pgfscope}%
\pgfpathrectangle{\pgfqpoint{1.250000in}{0.550000in}}{\pgfqpoint{7.750000in}{3.850000in}}%
\pgfusepath{clip}%
\pgfsetbuttcap%
\pgfsetmiterjoin%
\definecolor{currentfill}{rgb}{0.501961,0.501961,0.501961}%
\pgfsetfillcolor{currentfill}%
\pgfsetfillopacity{0.700000}%
\pgfsetlinewidth{0.000000pt}%
\definecolor{currentstroke}{rgb}{0.000000,0.000000,0.000000}%
\pgfsetstrokecolor{currentstroke}%
\pgfsetstrokeopacity{0.700000}%
\pgfsetdash{}{0pt}%
\pgfpathmoveto{\pgfqpoint{7.271662in}{0.550000in}}%
\pgfpathlineto{\pgfqpoint{7.299183in}{0.550000in}}%
\pgfpathlineto{\pgfqpoint{7.299183in}{0.855080in}}%
\pgfpathlineto{\pgfqpoint{7.271662in}{0.855080in}}%
\pgfpathlineto{\pgfqpoint{7.271662in}{0.550000in}}%
\pgfpathclose%
\pgfusepath{fill}%
\end{pgfscope}%
\begin{pgfscope}%
\pgfpathrectangle{\pgfqpoint{1.250000in}{0.550000in}}{\pgfqpoint{7.750000in}{3.850000in}}%
\pgfusepath{clip}%
\pgfsetbuttcap%
\pgfsetmiterjoin%
\definecolor{currentfill}{rgb}{0.501961,0.501961,0.501961}%
\pgfsetfillcolor{currentfill}%
\pgfsetfillopacity{0.700000}%
\pgfsetlinewidth{0.000000pt}%
\definecolor{currentstroke}{rgb}{0.000000,0.000000,0.000000}%
\pgfsetstrokecolor{currentstroke}%
\pgfsetstrokeopacity{0.700000}%
\pgfsetdash{}{0pt}%
\pgfpathmoveto{\pgfqpoint{7.299183in}{0.550000in}}%
\pgfpathlineto{\pgfqpoint{7.326705in}{0.550000in}}%
\pgfpathlineto{\pgfqpoint{7.326705in}{0.550000in}}%
\pgfpathlineto{\pgfqpoint{7.299183in}{0.550000in}}%
\pgfpathlineto{\pgfqpoint{7.299183in}{0.550000in}}%
\pgfpathclose%
\pgfusepath{fill}%
\end{pgfscope}%
\begin{pgfscope}%
\pgfpathrectangle{\pgfqpoint{1.250000in}{0.550000in}}{\pgfqpoint{7.750000in}{3.850000in}}%
\pgfusepath{clip}%
\pgfsetbuttcap%
\pgfsetmiterjoin%
\definecolor{currentfill}{rgb}{0.501961,0.501961,0.501961}%
\pgfsetfillcolor{currentfill}%
\pgfsetfillopacity{0.700000}%
\pgfsetlinewidth{0.000000pt}%
\definecolor{currentstroke}{rgb}{0.000000,0.000000,0.000000}%
\pgfsetstrokecolor{currentstroke}%
\pgfsetstrokeopacity{0.700000}%
\pgfsetdash{}{0pt}%
\pgfpathmoveto{\pgfqpoint{7.326705in}{0.550000in}}%
\pgfpathlineto{\pgfqpoint{7.354226in}{0.550000in}}%
\pgfpathlineto{\pgfqpoint{7.354226in}{0.904977in}}%
\pgfpathlineto{\pgfqpoint{7.326705in}{0.904977in}}%
\pgfpathlineto{\pgfqpoint{7.326705in}{0.550000in}}%
\pgfpathclose%
\pgfusepath{fill}%
\end{pgfscope}%
\begin{pgfscope}%
\pgfpathrectangle{\pgfqpoint{1.250000in}{0.550000in}}{\pgfqpoint{7.750000in}{3.850000in}}%
\pgfusepath{clip}%
\pgfsetbuttcap%
\pgfsetmiterjoin%
\definecolor{currentfill}{rgb}{0.501961,0.501961,0.501961}%
\pgfsetfillcolor{currentfill}%
\pgfsetfillopacity{0.700000}%
\pgfsetlinewidth{0.000000pt}%
\definecolor{currentstroke}{rgb}{0.000000,0.000000,0.000000}%
\pgfsetstrokecolor{currentstroke}%
\pgfsetstrokeopacity{0.700000}%
\pgfsetdash{}{0pt}%
\pgfpathmoveto{\pgfqpoint{7.354226in}{0.550000in}}%
\pgfpathlineto{\pgfqpoint{7.381747in}{0.550000in}}%
\pgfpathlineto{\pgfqpoint{7.381747in}{0.880741in}}%
\pgfpathlineto{\pgfqpoint{7.354226in}{0.880741in}}%
\pgfpathlineto{\pgfqpoint{7.354226in}{0.550000in}}%
\pgfpathclose%
\pgfusepath{fill}%
\end{pgfscope}%
\begin{pgfscope}%
\pgfpathrectangle{\pgfqpoint{1.250000in}{0.550000in}}{\pgfqpoint{7.750000in}{3.850000in}}%
\pgfusepath{clip}%
\pgfsetbuttcap%
\pgfsetmiterjoin%
\definecolor{currentfill}{rgb}{0.501961,0.501961,0.501961}%
\pgfsetfillcolor{currentfill}%
\pgfsetfillopacity{0.700000}%
\pgfsetlinewidth{0.000000pt}%
\definecolor{currentstroke}{rgb}{0.000000,0.000000,0.000000}%
\pgfsetstrokecolor{currentstroke}%
\pgfsetstrokeopacity{0.700000}%
\pgfsetdash{}{0pt}%
\pgfpathmoveto{\pgfqpoint{7.381747in}{0.550000in}}%
\pgfpathlineto{\pgfqpoint{7.409268in}{0.550000in}}%
\pgfpathlineto{\pgfqpoint{7.409268in}{0.550000in}}%
\pgfpathlineto{\pgfqpoint{7.381747in}{0.550000in}}%
\pgfpathlineto{\pgfqpoint{7.381747in}{0.550000in}}%
\pgfpathclose%
\pgfusepath{fill}%
\end{pgfscope}%
\begin{pgfscope}%
\pgfpathrectangle{\pgfqpoint{1.250000in}{0.550000in}}{\pgfqpoint{7.750000in}{3.850000in}}%
\pgfusepath{clip}%
\pgfsetbuttcap%
\pgfsetmiterjoin%
\definecolor{currentfill}{rgb}{0.501961,0.501961,0.501961}%
\pgfsetfillcolor{currentfill}%
\pgfsetfillopacity{0.700000}%
\pgfsetlinewidth{0.000000pt}%
\definecolor{currentstroke}{rgb}{0.000000,0.000000,0.000000}%
\pgfsetstrokecolor{currentstroke}%
\pgfsetstrokeopacity{0.700000}%
\pgfsetdash{}{0pt}%
\pgfpathmoveto{\pgfqpoint{7.409268in}{0.550000in}}%
\pgfpathlineto{\pgfqpoint{7.436790in}{0.550000in}}%
\pgfpathlineto{\pgfqpoint{7.436790in}{0.875039in}}%
\pgfpathlineto{\pgfqpoint{7.409268in}{0.875039in}}%
\pgfpathlineto{\pgfqpoint{7.409268in}{0.550000in}}%
\pgfpathclose%
\pgfusepath{fill}%
\end{pgfscope}%
\begin{pgfscope}%
\pgfpathrectangle{\pgfqpoint{1.250000in}{0.550000in}}{\pgfqpoint{7.750000in}{3.850000in}}%
\pgfusepath{clip}%
\pgfsetbuttcap%
\pgfsetmiterjoin%
\definecolor{currentfill}{rgb}{0.501961,0.501961,0.501961}%
\pgfsetfillcolor{currentfill}%
\pgfsetfillopacity{0.700000}%
\pgfsetlinewidth{0.000000pt}%
\definecolor{currentstroke}{rgb}{0.000000,0.000000,0.000000}%
\pgfsetstrokecolor{currentstroke}%
\pgfsetstrokeopacity{0.700000}%
\pgfsetdash{}{0pt}%
\pgfpathmoveto{\pgfqpoint{7.436790in}{0.550000in}}%
\pgfpathlineto{\pgfqpoint{7.464311in}{0.550000in}}%
\pgfpathlineto{\pgfqpoint{7.464311in}{0.550000in}}%
\pgfpathlineto{\pgfqpoint{7.436790in}{0.550000in}}%
\pgfpathlineto{\pgfqpoint{7.436790in}{0.550000in}}%
\pgfpathclose%
\pgfusepath{fill}%
\end{pgfscope}%
\begin{pgfscope}%
\pgfpathrectangle{\pgfqpoint{1.250000in}{0.550000in}}{\pgfqpoint{7.750000in}{3.850000in}}%
\pgfusepath{clip}%
\pgfsetbuttcap%
\pgfsetmiterjoin%
\definecolor{currentfill}{rgb}{0.501961,0.501961,0.501961}%
\pgfsetfillcolor{currentfill}%
\pgfsetfillopacity{0.700000}%
\pgfsetlinewidth{0.000000pt}%
\definecolor{currentstroke}{rgb}{0.000000,0.000000,0.000000}%
\pgfsetstrokecolor{currentstroke}%
\pgfsetstrokeopacity{0.700000}%
\pgfsetdash{}{0pt}%
\pgfpathmoveto{\pgfqpoint{7.464311in}{0.550000in}}%
\pgfpathlineto{\pgfqpoint{7.491832in}{0.550000in}}%
\pgfpathlineto{\pgfqpoint{7.491832in}{0.929212in}}%
\pgfpathlineto{\pgfqpoint{7.464311in}{0.929212in}}%
\pgfpathlineto{\pgfqpoint{7.464311in}{0.550000in}}%
\pgfpathclose%
\pgfusepath{fill}%
\end{pgfscope}%
\begin{pgfscope}%
\pgfpathrectangle{\pgfqpoint{1.250000in}{0.550000in}}{\pgfqpoint{7.750000in}{3.850000in}}%
\pgfusepath{clip}%
\pgfsetbuttcap%
\pgfsetmiterjoin%
\definecolor{currentfill}{rgb}{0.501961,0.501961,0.501961}%
\pgfsetfillcolor{currentfill}%
\pgfsetfillopacity{0.700000}%
\pgfsetlinewidth{0.000000pt}%
\definecolor{currentstroke}{rgb}{0.000000,0.000000,0.000000}%
\pgfsetstrokecolor{currentstroke}%
\pgfsetstrokeopacity{0.700000}%
\pgfsetdash{}{0pt}%
\pgfpathmoveto{\pgfqpoint{7.491832in}{0.550000in}}%
\pgfpathlineto{\pgfqpoint{7.519354in}{0.550000in}}%
\pgfpathlineto{\pgfqpoint{7.519354in}{0.550000in}}%
\pgfpathlineto{\pgfqpoint{7.491832in}{0.550000in}}%
\pgfpathlineto{\pgfqpoint{7.491832in}{0.550000in}}%
\pgfpathclose%
\pgfusepath{fill}%
\end{pgfscope}%
\begin{pgfscope}%
\pgfpathrectangle{\pgfqpoint{1.250000in}{0.550000in}}{\pgfqpoint{7.750000in}{3.850000in}}%
\pgfusepath{clip}%
\pgfsetbuttcap%
\pgfsetmiterjoin%
\definecolor{currentfill}{rgb}{0.501961,0.501961,0.501961}%
\pgfsetfillcolor{currentfill}%
\pgfsetfillopacity{0.700000}%
\pgfsetlinewidth{0.000000pt}%
\definecolor{currentstroke}{rgb}{0.000000,0.000000,0.000000}%
\pgfsetstrokecolor{currentstroke}%
\pgfsetstrokeopacity{0.700000}%
\pgfsetdash{}{0pt}%
\pgfpathmoveto{\pgfqpoint{7.519354in}{0.550000in}}%
\pgfpathlineto{\pgfqpoint{7.546875in}{0.550000in}}%
\pgfpathlineto{\pgfqpoint{7.546875in}{0.856506in}}%
\pgfpathlineto{\pgfqpoint{7.519354in}{0.856506in}}%
\pgfpathlineto{\pgfqpoint{7.519354in}{0.550000in}}%
\pgfpathclose%
\pgfusepath{fill}%
\end{pgfscope}%
\begin{pgfscope}%
\pgfpathrectangle{\pgfqpoint{1.250000in}{0.550000in}}{\pgfqpoint{7.750000in}{3.850000in}}%
\pgfusepath{clip}%
\pgfsetbuttcap%
\pgfsetmiterjoin%
\definecolor{currentfill}{rgb}{0.501961,0.501961,0.501961}%
\pgfsetfillcolor{currentfill}%
\pgfsetfillopacity{0.700000}%
\pgfsetlinewidth{0.000000pt}%
\definecolor{currentstroke}{rgb}{0.000000,0.000000,0.000000}%
\pgfsetstrokecolor{currentstroke}%
\pgfsetstrokeopacity{0.700000}%
\pgfsetdash{}{0pt}%
\pgfpathmoveto{\pgfqpoint{7.546875in}{0.550000in}}%
\pgfpathlineto{\pgfqpoint{7.574396in}{0.550000in}}%
\pgfpathlineto{\pgfqpoint{7.574396in}{0.805184in}}%
\pgfpathlineto{\pgfqpoint{7.546875in}{0.805184in}}%
\pgfpathlineto{\pgfqpoint{7.546875in}{0.550000in}}%
\pgfpathclose%
\pgfusepath{fill}%
\end{pgfscope}%
\begin{pgfscope}%
\pgfpathrectangle{\pgfqpoint{1.250000in}{0.550000in}}{\pgfqpoint{7.750000in}{3.850000in}}%
\pgfusepath{clip}%
\pgfsetbuttcap%
\pgfsetmiterjoin%
\definecolor{currentfill}{rgb}{0.501961,0.501961,0.501961}%
\pgfsetfillcolor{currentfill}%
\pgfsetfillopacity{0.700000}%
\pgfsetlinewidth{0.000000pt}%
\definecolor{currentstroke}{rgb}{0.000000,0.000000,0.000000}%
\pgfsetstrokecolor{currentstroke}%
\pgfsetstrokeopacity{0.700000}%
\pgfsetdash{}{0pt}%
\pgfpathmoveto{\pgfqpoint{7.574396in}{0.550000in}}%
\pgfpathlineto{\pgfqpoint{7.601918in}{0.550000in}}%
\pgfpathlineto{\pgfqpoint{7.601918in}{0.799482in}}%
\pgfpathlineto{\pgfqpoint{7.574396in}{0.799482in}}%
\pgfpathlineto{\pgfqpoint{7.574396in}{0.550000in}}%
\pgfpathclose%
\pgfusepath{fill}%
\end{pgfscope}%
\begin{pgfscope}%
\pgfpathrectangle{\pgfqpoint{1.250000in}{0.550000in}}{\pgfqpoint{7.750000in}{3.850000in}}%
\pgfusepath{clip}%
\pgfsetbuttcap%
\pgfsetmiterjoin%
\definecolor{currentfill}{rgb}{0.501961,0.501961,0.501961}%
\pgfsetfillcolor{currentfill}%
\pgfsetfillopacity{0.700000}%
\pgfsetlinewidth{0.000000pt}%
\definecolor{currentstroke}{rgb}{0.000000,0.000000,0.000000}%
\pgfsetstrokecolor{currentstroke}%
\pgfsetstrokeopacity{0.700000}%
\pgfsetdash{}{0pt}%
\pgfpathmoveto{\pgfqpoint{7.601918in}{0.550000in}}%
\pgfpathlineto{\pgfqpoint{7.629439in}{0.550000in}}%
\pgfpathlineto{\pgfqpoint{7.629439in}{0.550000in}}%
\pgfpathlineto{\pgfqpoint{7.601918in}{0.550000in}}%
\pgfpathlineto{\pgfqpoint{7.601918in}{0.550000in}}%
\pgfpathclose%
\pgfusepath{fill}%
\end{pgfscope}%
\begin{pgfscope}%
\pgfpathrectangle{\pgfqpoint{1.250000in}{0.550000in}}{\pgfqpoint{7.750000in}{3.850000in}}%
\pgfusepath{clip}%
\pgfsetbuttcap%
\pgfsetmiterjoin%
\definecolor{currentfill}{rgb}{0.501961,0.501961,0.501961}%
\pgfsetfillcolor{currentfill}%
\pgfsetfillopacity{0.700000}%
\pgfsetlinewidth{0.000000pt}%
\definecolor{currentstroke}{rgb}{0.000000,0.000000,0.000000}%
\pgfsetstrokecolor{currentstroke}%
\pgfsetstrokeopacity{0.700000}%
\pgfsetdash{}{0pt}%
\pgfpathmoveto{\pgfqpoint{7.629439in}{0.550000in}}%
\pgfpathlineto{\pgfqpoint{7.656960in}{0.550000in}}%
\pgfpathlineto{\pgfqpoint{7.656960in}{0.809461in}}%
\pgfpathlineto{\pgfqpoint{7.629439in}{0.809461in}}%
\pgfpathlineto{\pgfqpoint{7.629439in}{0.550000in}}%
\pgfpathclose%
\pgfusepath{fill}%
\end{pgfscope}%
\begin{pgfscope}%
\pgfpathrectangle{\pgfqpoint{1.250000in}{0.550000in}}{\pgfqpoint{7.750000in}{3.850000in}}%
\pgfusepath{clip}%
\pgfsetbuttcap%
\pgfsetmiterjoin%
\definecolor{currentfill}{rgb}{0.501961,0.501961,0.501961}%
\pgfsetfillcolor{currentfill}%
\pgfsetfillopacity{0.700000}%
\pgfsetlinewidth{0.000000pt}%
\definecolor{currentstroke}{rgb}{0.000000,0.000000,0.000000}%
\pgfsetstrokecolor{currentstroke}%
\pgfsetstrokeopacity{0.700000}%
\pgfsetdash{}{0pt}%
\pgfpathmoveto{\pgfqpoint{7.656960in}{0.550000in}}%
\pgfpathlineto{\pgfqpoint{7.684482in}{0.550000in}}%
\pgfpathlineto{\pgfqpoint{7.684482in}{0.805184in}}%
\pgfpathlineto{\pgfqpoint{7.656960in}{0.805184in}}%
\pgfpathlineto{\pgfqpoint{7.656960in}{0.550000in}}%
\pgfpathclose%
\pgfusepath{fill}%
\end{pgfscope}%
\begin{pgfscope}%
\pgfpathrectangle{\pgfqpoint{1.250000in}{0.550000in}}{\pgfqpoint{7.750000in}{3.850000in}}%
\pgfusepath{clip}%
\pgfsetbuttcap%
\pgfsetmiterjoin%
\definecolor{currentfill}{rgb}{0.501961,0.501961,0.501961}%
\pgfsetfillcolor{currentfill}%
\pgfsetfillopacity{0.700000}%
\pgfsetlinewidth{0.000000pt}%
\definecolor{currentstroke}{rgb}{0.000000,0.000000,0.000000}%
\pgfsetstrokecolor{currentstroke}%
\pgfsetstrokeopacity{0.700000}%
\pgfsetdash{}{0pt}%
\pgfpathmoveto{\pgfqpoint{7.684482in}{0.550000in}}%
\pgfpathlineto{\pgfqpoint{7.712003in}{0.550000in}}%
\pgfpathlineto{\pgfqpoint{7.712003in}{0.813738in}}%
\pgfpathlineto{\pgfqpoint{7.684482in}{0.813738in}}%
\pgfpathlineto{\pgfqpoint{7.684482in}{0.550000in}}%
\pgfpathclose%
\pgfusepath{fill}%
\end{pgfscope}%
\begin{pgfscope}%
\pgfpathrectangle{\pgfqpoint{1.250000in}{0.550000in}}{\pgfqpoint{7.750000in}{3.850000in}}%
\pgfusepath{clip}%
\pgfsetbuttcap%
\pgfsetmiterjoin%
\definecolor{currentfill}{rgb}{0.501961,0.501961,0.501961}%
\pgfsetfillcolor{currentfill}%
\pgfsetfillopacity{0.700000}%
\pgfsetlinewidth{0.000000pt}%
\definecolor{currentstroke}{rgb}{0.000000,0.000000,0.000000}%
\pgfsetstrokecolor{currentstroke}%
\pgfsetstrokeopacity{0.700000}%
\pgfsetdash{}{0pt}%
\pgfpathmoveto{\pgfqpoint{7.712003in}{0.550000in}}%
\pgfpathlineto{\pgfqpoint{7.739524in}{0.550000in}}%
\pgfpathlineto{\pgfqpoint{7.739524in}{0.550000in}}%
\pgfpathlineto{\pgfqpoint{7.712003in}{0.550000in}}%
\pgfpathlineto{\pgfqpoint{7.712003in}{0.550000in}}%
\pgfpathclose%
\pgfusepath{fill}%
\end{pgfscope}%
\begin{pgfscope}%
\pgfpathrectangle{\pgfqpoint{1.250000in}{0.550000in}}{\pgfqpoint{7.750000in}{3.850000in}}%
\pgfusepath{clip}%
\pgfsetbuttcap%
\pgfsetmiterjoin%
\definecolor{currentfill}{rgb}{0.501961,0.501961,0.501961}%
\pgfsetfillcolor{currentfill}%
\pgfsetfillopacity{0.700000}%
\pgfsetlinewidth{0.000000pt}%
\definecolor{currentstroke}{rgb}{0.000000,0.000000,0.000000}%
\pgfsetstrokecolor{currentstroke}%
\pgfsetstrokeopacity{0.700000}%
\pgfsetdash{}{0pt}%
\pgfpathmoveto{\pgfqpoint{7.739524in}{0.550000in}}%
\pgfpathlineto{\pgfqpoint{7.767045in}{0.550000in}}%
\pgfpathlineto{\pgfqpoint{7.767045in}{0.782374in}}%
\pgfpathlineto{\pgfqpoint{7.739524in}{0.782374in}}%
\pgfpathlineto{\pgfqpoint{7.739524in}{0.550000in}}%
\pgfpathclose%
\pgfusepath{fill}%
\end{pgfscope}%
\begin{pgfscope}%
\pgfpathrectangle{\pgfqpoint{1.250000in}{0.550000in}}{\pgfqpoint{7.750000in}{3.850000in}}%
\pgfusepath{clip}%
\pgfsetbuttcap%
\pgfsetmiterjoin%
\definecolor{currentfill}{rgb}{0.501961,0.501961,0.501961}%
\pgfsetfillcolor{currentfill}%
\pgfsetfillopacity{0.700000}%
\pgfsetlinewidth{0.000000pt}%
\definecolor{currentstroke}{rgb}{0.000000,0.000000,0.000000}%
\pgfsetstrokecolor{currentstroke}%
\pgfsetstrokeopacity{0.700000}%
\pgfsetdash{}{0pt}%
\pgfpathmoveto{\pgfqpoint{7.767045in}{0.550000in}}%
\pgfpathlineto{\pgfqpoint{7.794567in}{0.550000in}}%
\pgfpathlineto{\pgfqpoint{7.794567in}{0.768118in}}%
\pgfpathlineto{\pgfqpoint{7.767045in}{0.768118in}}%
\pgfpathlineto{\pgfqpoint{7.767045in}{0.550000in}}%
\pgfpathclose%
\pgfusepath{fill}%
\end{pgfscope}%
\begin{pgfscope}%
\pgfpathrectangle{\pgfqpoint{1.250000in}{0.550000in}}{\pgfqpoint{7.750000in}{3.850000in}}%
\pgfusepath{clip}%
\pgfsetbuttcap%
\pgfsetmiterjoin%
\definecolor{currentfill}{rgb}{0.501961,0.501961,0.501961}%
\pgfsetfillcolor{currentfill}%
\pgfsetfillopacity{0.700000}%
\pgfsetlinewidth{0.000000pt}%
\definecolor{currentstroke}{rgb}{0.000000,0.000000,0.000000}%
\pgfsetstrokecolor{currentstroke}%
\pgfsetstrokeopacity{0.700000}%
\pgfsetdash{}{0pt}%
\pgfpathmoveto{\pgfqpoint{7.794567in}{0.550000in}}%
\pgfpathlineto{\pgfqpoint{7.822088in}{0.550000in}}%
\pgfpathlineto{\pgfqpoint{7.822088in}{0.775246in}}%
\pgfpathlineto{\pgfqpoint{7.794567in}{0.775246in}}%
\pgfpathlineto{\pgfqpoint{7.794567in}{0.550000in}}%
\pgfpathclose%
\pgfusepath{fill}%
\end{pgfscope}%
\begin{pgfscope}%
\pgfpathrectangle{\pgfqpoint{1.250000in}{0.550000in}}{\pgfqpoint{7.750000in}{3.850000in}}%
\pgfusepath{clip}%
\pgfsetbuttcap%
\pgfsetmiterjoin%
\definecolor{currentfill}{rgb}{0.501961,0.501961,0.501961}%
\pgfsetfillcolor{currentfill}%
\pgfsetfillopacity{0.700000}%
\pgfsetlinewidth{0.000000pt}%
\definecolor{currentstroke}{rgb}{0.000000,0.000000,0.000000}%
\pgfsetstrokecolor{currentstroke}%
\pgfsetstrokeopacity{0.700000}%
\pgfsetdash{}{0pt}%
\pgfpathmoveto{\pgfqpoint{7.822088in}{0.550000in}}%
\pgfpathlineto{\pgfqpoint{7.849609in}{0.550000in}}%
\pgfpathlineto{\pgfqpoint{7.849609in}{0.773821in}}%
\pgfpathlineto{\pgfqpoint{7.822088in}{0.773821in}}%
\pgfpathlineto{\pgfqpoint{7.822088in}{0.550000in}}%
\pgfpathclose%
\pgfusepath{fill}%
\end{pgfscope}%
\begin{pgfscope}%
\pgfpathrectangle{\pgfqpoint{1.250000in}{0.550000in}}{\pgfqpoint{7.750000in}{3.850000in}}%
\pgfusepath{clip}%
\pgfsetbuttcap%
\pgfsetmiterjoin%
\definecolor{currentfill}{rgb}{0.501961,0.501961,0.501961}%
\pgfsetfillcolor{currentfill}%
\pgfsetfillopacity{0.700000}%
\pgfsetlinewidth{0.000000pt}%
\definecolor{currentstroke}{rgb}{0.000000,0.000000,0.000000}%
\pgfsetstrokecolor{currentstroke}%
\pgfsetstrokeopacity{0.700000}%
\pgfsetdash{}{0pt}%
\pgfpathmoveto{\pgfqpoint{7.849609in}{0.550000in}}%
\pgfpathlineto{\pgfqpoint{7.877131in}{0.550000in}}%
\pgfpathlineto{\pgfqpoint{7.877131in}{0.745308in}}%
\pgfpathlineto{\pgfqpoint{7.849609in}{0.745308in}}%
\pgfpathlineto{\pgfqpoint{7.849609in}{0.550000in}}%
\pgfpathclose%
\pgfusepath{fill}%
\end{pgfscope}%
\begin{pgfscope}%
\pgfpathrectangle{\pgfqpoint{1.250000in}{0.550000in}}{\pgfqpoint{7.750000in}{3.850000in}}%
\pgfusepath{clip}%
\pgfsetbuttcap%
\pgfsetmiterjoin%
\definecolor{currentfill}{rgb}{0.501961,0.501961,0.501961}%
\pgfsetfillcolor{currentfill}%
\pgfsetfillopacity{0.700000}%
\pgfsetlinewidth{0.000000pt}%
\definecolor{currentstroke}{rgb}{0.000000,0.000000,0.000000}%
\pgfsetstrokecolor{currentstroke}%
\pgfsetstrokeopacity{0.700000}%
\pgfsetdash{}{0pt}%
\pgfpathmoveto{\pgfqpoint{7.877131in}{0.550000in}}%
\pgfpathlineto{\pgfqpoint{7.904652in}{0.550000in}}%
\pgfpathlineto{\pgfqpoint{7.904652in}{0.749585in}}%
\pgfpathlineto{\pgfqpoint{7.877131in}{0.749585in}}%
\pgfpathlineto{\pgfqpoint{7.877131in}{0.550000in}}%
\pgfpathclose%
\pgfusepath{fill}%
\end{pgfscope}%
\begin{pgfscope}%
\pgfpathrectangle{\pgfqpoint{1.250000in}{0.550000in}}{\pgfqpoint{7.750000in}{3.850000in}}%
\pgfusepath{clip}%
\pgfsetbuttcap%
\pgfsetmiterjoin%
\definecolor{currentfill}{rgb}{0.501961,0.501961,0.501961}%
\pgfsetfillcolor{currentfill}%
\pgfsetfillopacity{0.700000}%
\pgfsetlinewidth{0.000000pt}%
\definecolor{currentstroke}{rgb}{0.000000,0.000000,0.000000}%
\pgfsetstrokecolor{currentstroke}%
\pgfsetstrokeopacity{0.700000}%
\pgfsetdash{}{0pt}%
\pgfpathmoveto{\pgfqpoint{7.904652in}{0.550000in}}%
\pgfpathlineto{\pgfqpoint{7.932173in}{0.550000in}}%
\pgfpathlineto{\pgfqpoint{7.932173in}{0.709668in}}%
\pgfpathlineto{\pgfqpoint{7.904652in}{0.709668in}}%
\pgfpathlineto{\pgfqpoint{7.904652in}{0.550000in}}%
\pgfpathclose%
\pgfusepath{fill}%
\end{pgfscope}%
\begin{pgfscope}%
\pgfpathrectangle{\pgfqpoint{1.250000in}{0.550000in}}{\pgfqpoint{7.750000in}{3.850000in}}%
\pgfusepath{clip}%
\pgfsetbuttcap%
\pgfsetmiterjoin%
\definecolor{currentfill}{rgb}{0.501961,0.501961,0.501961}%
\pgfsetfillcolor{currentfill}%
\pgfsetfillopacity{0.700000}%
\pgfsetlinewidth{0.000000pt}%
\definecolor{currentstroke}{rgb}{0.000000,0.000000,0.000000}%
\pgfsetstrokecolor{currentstroke}%
\pgfsetstrokeopacity{0.700000}%
\pgfsetdash{}{0pt}%
\pgfpathmoveto{\pgfqpoint{7.932173in}{0.550000in}}%
\pgfpathlineto{\pgfqpoint{7.959695in}{0.550000in}}%
\pgfpathlineto{\pgfqpoint{7.959695in}{0.729627in}}%
\pgfpathlineto{\pgfqpoint{7.932173in}{0.729627in}}%
\pgfpathlineto{\pgfqpoint{7.932173in}{0.550000in}}%
\pgfpathclose%
\pgfusepath{fill}%
\end{pgfscope}%
\begin{pgfscope}%
\pgfpathrectangle{\pgfqpoint{1.250000in}{0.550000in}}{\pgfqpoint{7.750000in}{3.850000in}}%
\pgfusepath{clip}%
\pgfsetbuttcap%
\pgfsetmiterjoin%
\definecolor{currentfill}{rgb}{0.501961,0.501961,0.501961}%
\pgfsetfillcolor{currentfill}%
\pgfsetfillopacity{0.700000}%
\pgfsetlinewidth{0.000000pt}%
\definecolor{currentstroke}{rgb}{0.000000,0.000000,0.000000}%
\pgfsetstrokecolor{currentstroke}%
\pgfsetstrokeopacity{0.700000}%
\pgfsetdash{}{0pt}%
\pgfpathmoveto{\pgfqpoint{7.959695in}{0.550000in}}%
\pgfpathlineto{\pgfqpoint{7.987216in}{0.550000in}}%
\pgfpathlineto{\pgfqpoint{7.987216in}{0.726776in}}%
\pgfpathlineto{\pgfqpoint{7.959695in}{0.726776in}}%
\pgfpathlineto{\pgfqpoint{7.959695in}{0.550000in}}%
\pgfpathclose%
\pgfusepath{fill}%
\end{pgfscope}%
\begin{pgfscope}%
\pgfpathrectangle{\pgfqpoint{1.250000in}{0.550000in}}{\pgfqpoint{7.750000in}{3.850000in}}%
\pgfusepath{clip}%
\pgfsetbuttcap%
\pgfsetmiterjoin%
\definecolor{currentfill}{rgb}{0.501961,0.501961,0.501961}%
\pgfsetfillcolor{currentfill}%
\pgfsetfillopacity{0.700000}%
\pgfsetlinewidth{0.000000pt}%
\definecolor{currentstroke}{rgb}{0.000000,0.000000,0.000000}%
\pgfsetstrokecolor{currentstroke}%
\pgfsetstrokeopacity{0.700000}%
\pgfsetdash{}{0pt}%
\pgfpathmoveto{\pgfqpoint{7.987216in}{0.550000in}}%
\pgfpathlineto{\pgfqpoint{8.014737in}{0.550000in}}%
\pgfpathlineto{\pgfqpoint{8.014737in}{0.709668in}}%
\pgfpathlineto{\pgfqpoint{7.987216in}{0.709668in}}%
\pgfpathlineto{\pgfqpoint{7.987216in}{0.550000in}}%
\pgfpathclose%
\pgfusepath{fill}%
\end{pgfscope}%
\begin{pgfscope}%
\pgfpathrectangle{\pgfqpoint{1.250000in}{0.550000in}}{\pgfqpoint{7.750000in}{3.850000in}}%
\pgfusepath{clip}%
\pgfsetbuttcap%
\pgfsetmiterjoin%
\definecolor{currentfill}{rgb}{0.501961,0.501961,0.501961}%
\pgfsetfillcolor{currentfill}%
\pgfsetfillopacity{0.700000}%
\pgfsetlinewidth{0.000000pt}%
\definecolor{currentstroke}{rgb}{0.000000,0.000000,0.000000}%
\pgfsetstrokecolor{currentstroke}%
\pgfsetstrokeopacity{0.700000}%
\pgfsetdash{}{0pt}%
\pgfpathmoveto{\pgfqpoint{8.014737in}{0.550000in}}%
\pgfpathlineto{\pgfqpoint{8.042259in}{0.550000in}}%
\pgfpathlineto{\pgfqpoint{8.042259in}{0.693987in}}%
\pgfpathlineto{\pgfqpoint{8.014737in}{0.693987in}}%
\pgfpathlineto{\pgfqpoint{8.014737in}{0.550000in}}%
\pgfpathclose%
\pgfusepath{fill}%
\end{pgfscope}%
\begin{pgfscope}%
\pgfpathrectangle{\pgfqpoint{1.250000in}{0.550000in}}{\pgfqpoint{7.750000in}{3.850000in}}%
\pgfusepath{clip}%
\pgfsetbuttcap%
\pgfsetmiterjoin%
\definecolor{currentfill}{rgb}{0.501961,0.501961,0.501961}%
\pgfsetfillcolor{currentfill}%
\pgfsetfillopacity{0.700000}%
\pgfsetlinewidth{0.000000pt}%
\definecolor{currentstroke}{rgb}{0.000000,0.000000,0.000000}%
\pgfsetstrokecolor{currentstroke}%
\pgfsetstrokeopacity{0.700000}%
\pgfsetdash{}{0pt}%
\pgfpathmoveto{\pgfqpoint{8.042259in}{0.550000in}}%
\pgfpathlineto{\pgfqpoint{8.069780in}{0.550000in}}%
\pgfpathlineto{\pgfqpoint{8.069780in}{0.856506in}}%
\pgfpathlineto{\pgfqpoint{8.042259in}{0.856506in}}%
\pgfpathlineto{\pgfqpoint{8.042259in}{0.550000in}}%
\pgfpathclose%
\pgfusepath{fill}%
\end{pgfscope}%
\begin{pgfscope}%
\pgfpathrectangle{\pgfqpoint{1.250000in}{0.550000in}}{\pgfqpoint{7.750000in}{3.850000in}}%
\pgfusepath{clip}%
\pgfsetbuttcap%
\pgfsetmiterjoin%
\definecolor{currentfill}{rgb}{0.501961,0.501961,0.501961}%
\pgfsetfillcolor{currentfill}%
\pgfsetfillopacity{0.700000}%
\pgfsetlinewidth{0.000000pt}%
\definecolor{currentstroke}{rgb}{0.000000,0.000000,0.000000}%
\pgfsetstrokecolor{currentstroke}%
\pgfsetstrokeopacity{0.700000}%
\pgfsetdash{}{0pt}%
\pgfpathmoveto{\pgfqpoint{8.069780in}{0.550000in}}%
\pgfpathlineto{\pgfqpoint{8.097301in}{0.550000in}}%
\pgfpathlineto{\pgfqpoint{8.097301in}{0.684007in}}%
\pgfpathlineto{\pgfqpoint{8.069780in}{0.684007in}}%
\pgfpathlineto{\pgfqpoint{8.069780in}{0.550000in}}%
\pgfpathclose%
\pgfusepath{fill}%
\end{pgfscope}%
\begin{pgfscope}%
\pgfpathrectangle{\pgfqpoint{1.250000in}{0.550000in}}{\pgfqpoint{7.750000in}{3.850000in}}%
\pgfusepath{clip}%
\pgfsetbuttcap%
\pgfsetmiterjoin%
\definecolor{currentfill}{rgb}{0.501961,0.501961,0.501961}%
\pgfsetfillcolor{currentfill}%
\pgfsetfillopacity{0.700000}%
\pgfsetlinewidth{0.000000pt}%
\definecolor{currentstroke}{rgb}{0.000000,0.000000,0.000000}%
\pgfsetstrokecolor{currentstroke}%
\pgfsetstrokeopacity{0.700000}%
\pgfsetdash{}{0pt}%
\pgfpathmoveto{\pgfqpoint{8.097301in}{0.550000in}}%
\pgfpathlineto{\pgfqpoint{8.124822in}{0.550000in}}%
\pgfpathlineto{\pgfqpoint{8.124822in}{0.800907in}}%
\pgfpathlineto{\pgfqpoint{8.097301in}{0.800907in}}%
\pgfpathlineto{\pgfqpoint{8.097301in}{0.550000in}}%
\pgfpathclose%
\pgfusepath{fill}%
\end{pgfscope}%
\begin{pgfscope}%
\pgfpathrectangle{\pgfqpoint{1.250000in}{0.550000in}}{\pgfqpoint{7.750000in}{3.850000in}}%
\pgfusepath{clip}%
\pgfsetbuttcap%
\pgfsetmiterjoin%
\definecolor{currentfill}{rgb}{0.501961,0.501961,0.501961}%
\pgfsetfillcolor{currentfill}%
\pgfsetfillopacity{0.700000}%
\pgfsetlinewidth{0.000000pt}%
\definecolor{currentstroke}{rgb}{0.000000,0.000000,0.000000}%
\pgfsetstrokecolor{currentstroke}%
\pgfsetstrokeopacity{0.700000}%
\pgfsetdash{}{0pt}%
\pgfpathmoveto{\pgfqpoint{8.124822in}{0.550000in}}%
\pgfpathlineto{\pgfqpoint{8.152344in}{0.550000in}}%
\pgfpathlineto{\pgfqpoint{8.152344in}{0.691135in}}%
\pgfpathlineto{\pgfqpoint{8.124822in}{0.691135in}}%
\pgfpathlineto{\pgfqpoint{8.124822in}{0.550000in}}%
\pgfpathclose%
\pgfusepath{fill}%
\end{pgfscope}%
\begin{pgfscope}%
\pgfpathrectangle{\pgfqpoint{1.250000in}{0.550000in}}{\pgfqpoint{7.750000in}{3.850000in}}%
\pgfusepath{clip}%
\pgfsetbuttcap%
\pgfsetmiterjoin%
\definecolor{currentfill}{rgb}{0.501961,0.501961,0.501961}%
\pgfsetfillcolor{currentfill}%
\pgfsetfillopacity{0.700000}%
\pgfsetlinewidth{0.000000pt}%
\definecolor{currentstroke}{rgb}{0.000000,0.000000,0.000000}%
\pgfsetstrokecolor{currentstroke}%
\pgfsetstrokeopacity{0.700000}%
\pgfsetdash{}{0pt}%
\pgfpathmoveto{\pgfqpoint{8.152344in}{0.550000in}}%
\pgfpathlineto{\pgfqpoint{8.179865in}{0.550000in}}%
\pgfpathlineto{\pgfqpoint{8.179865in}{0.776672in}}%
\pgfpathlineto{\pgfqpoint{8.152344in}{0.776672in}}%
\pgfpathlineto{\pgfqpoint{8.152344in}{0.550000in}}%
\pgfpathclose%
\pgfusepath{fill}%
\end{pgfscope}%
\begin{pgfscope}%
\pgfpathrectangle{\pgfqpoint{1.250000in}{0.550000in}}{\pgfqpoint{7.750000in}{3.850000in}}%
\pgfusepath{clip}%
\pgfsetbuttcap%
\pgfsetmiterjoin%
\definecolor{currentfill}{rgb}{0.501961,0.501961,0.501961}%
\pgfsetfillcolor{currentfill}%
\pgfsetfillopacity{0.700000}%
\pgfsetlinewidth{0.000000pt}%
\definecolor{currentstroke}{rgb}{0.000000,0.000000,0.000000}%
\pgfsetstrokecolor{currentstroke}%
\pgfsetstrokeopacity{0.700000}%
\pgfsetdash{}{0pt}%
\pgfpathmoveto{\pgfqpoint{8.179865in}{0.550000in}}%
\pgfpathlineto{\pgfqpoint{8.207386in}{0.550000in}}%
\pgfpathlineto{\pgfqpoint{8.207386in}{0.782374in}}%
\pgfpathlineto{\pgfqpoint{8.179865in}{0.782374in}}%
\pgfpathlineto{\pgfqpoint{8.179865in}{0.550000in}}%
\pgfpathclose%
\pgfusepath{fill}%
\end{pgfscope}%
\begin{pgfscope}%
\pgfpathrectangle{\pgfqpoint{1.250000in}{0.550000in}}{\pgfqpoint{7.750000in}{3.850000in}}%
\pgfusepath{clip}%
\pgfsetbuttcap%
\pgfsetmiterjoin%
\definecolor{currentfill}{rgb}{0.501961,0.501961,0.501961}%
\pgfsetfillcolor{currentfill}%
\pgfsetfillopacity{0.700000}%
\pgfsetlinewidth{0.000000pt}%
\definecolor{currentstroke}{rgb}{0.000000,0.000000,0.000000}%
\pgfsetstrokecolor{currentstroke}%
\pgfsetstrokeopacity{0.700000}%
\pgfsetdash{}{0pt}%
\pgfpathmoveto{\pgfqpoint{8.207386in}{0.550000in}}%
\pgfpathlineto{\pgfqpoint{8.234908in}{0.550000in}}%
\pgfpathlineto{\pgfqpoint{8.234908in}{0.718222in}}%
\pgfpathlineto{\pgfqpoint{8.207386in}{0.718222in}}%
\pgfpathlineto{\pgfqpoint{8.207386in}{0.550000in}}%
\pgfpathclose%
\pgfusepath{fill}%
\end{pgfscope}%
\begin{pgfscope}%
\pgfpathrectangle{\pgfqpoint{1.250000in}{0.550000in}}{\pgfqpoint{7.750000in}{3.850000in}}%
\pgfusepath{clip}%
\pgfsetbuttcap%
\pgfsetmiterjoin%
\definecolor{currentfill}{rgb}{0.501961,0.501961,0.501961}%
\pgfsetfillcolor{currentfill}%
\pgfsetfillopacity{0.700000}%
\pgfsetlinewidth{0.000000pt}%
\definecolor{currentstroke}{rgb}{0.000000,0.000000,0.000000}%
\pgfsetstrokecolor{currentstroke}%
\pgfsetstrokeopacity{0.700000}%
\pgfsetdash{}{0pt}%
\pgfpathmoveto{\pgfqpoint{8.234908in}{0.550000in}}%
\pgfpathlineto{\pgfqpoint{8.262429in}{0.550000in}}%
\pgfpathlineto{\pgfqpoint{8.262429in}{0.679730in}}%
\pgfpathlineto{\pgfqpoint{8.234908in}{0.679730in}}%
\pgfpathlineto{\pgfqpoint{8.234908in}{0.550000in}}%
\pgfpathclose%
\pgfusepath{fill}%
\end{pgfscope}%
\begin{pgfscope}%
\pgfpathrectangle{\pgfqpoint{1.250000in}{0.550000in}}{\pgfqpoint{7.750000in}{3.850000in}}%
\pgfusepath{clip}%
\pgfsetbuttcap%
\pgfsetmiterjoin%
\definecolor{currentfill}{rgb}{0.501961,0.501961,0.501961}%
\pgfsetfillcolor{currentfill}%
\pgfsetfillopacity{0.700000}%
\pgfsetlinewidth{0.000000pt}%
\definecolor{currentstroke}{rgb}{0.000000,0.000000,0.000000}%
\pgfsetstrokecolor{currentstroke}%
\pgfsetstrokeopacity{0.700000}%
\pgfsetdash{}{0pt}%
\pgfpathmoveto{\pgfqpoint{8.262429in}{0.550000in}}%
\pgfpathlineto{\pgfqpoint{8.289950in}{0.550000in}}%
\pgfpathlineto{\pgfqpoint{8.289950in}{0.749585in}}%
\pgfpathlineto{\pgfqpoint{8.262429in}{0.749585in}}%
\pgfpathlineto{\pgfqpoint{8.262429in}{0.550000in}}%
\pgfpathclose%
\pgfusepath{fill}%
\end{pgfscope}%
\begin{pgfscope}%
\pgfpathrectangle{\pgfqpoint{1.250000in}{0.550000in}}{\pgfqpoint{7.750000in}{3.850000in}}%
\pgfusepath{clip}%
\pgfsetbuttcap%
\pgfsetmiterjoin%
\definecolor{currentfill}{rgb}{0.501961,0.501961,0.501961}%
\pgfsetfillcolor{currentfill}%
\pgfsetfillopacity{0.700000}%
\pgfsetlinewidth{0.000000pt}%
\definecolor{currentstroke}{rgb}{0.000000,0.000000,0.000000}%
\pgfsetstrokecolor{currentstroke}%
\pgfsetstrokeopacity{0.700000}%
\pgfsetdash{}{0pt}%
\pgfpathmoveto{\pgfqpoint{8.289950in}{0.550000in}}%
\pgfpathlineto{\pgfqpoint{8.317472in}{0.550000in}}%
\pgfpathlineto{\pgfqpoint{8.317472in}{0.792354in}}%
\pgfpathlineto{\pgfqpoint{8.289950in}{0.792354in}}%
\pgfpathlineto{\pgfqpoint{8.289950in}{0.550000in}}%
\pgfpathclose%
\pgfusepath{fill}%
\end{pgfscope}%
\begin{pgfscope}%
\pgfpathrectangle{\pgfqpoint{1.250000in}{0.550000in}}{\pgfqpoint{7.750000in}{3.850000in}}%
\pgfusepath{clip}%
\pgfsetbuttcap%
\pgfsetmiterjoin%
\definecolor{currentfill}{rgb}{0.501961,0.501961,0.501961}%
\pgfsetfillcolor{currentfill}%
\pgfsetfillopacity{0.700000}%
\pgfsetlinewidth{0.000000pt}%
\definecolor{currentstroke}{rgb}{0.000000,0.000000,0.000000}%
\pgfsetstrokecolor{currentstroke}%
\pgfsetstrokeopacity{0.700000}%
\pgfsetdash{}{0pt}%
\pgfpathmoveto{\pgfqpoint{8.317472in}{0.550000in}}%
\pgfpathlineto{\pgfqpoint{8.344993in}{0.550000in}}%
\pgfpathlineto{\pgfqpoint{8.344993in}{0.709668in}}%
\pgfpathlineto{\pgfqpoint{8.317472in}{0.709668in}}%
\pgfpathlineto{\pgfqpoint{8.317472in}{0.550000in}}%
\pgfpathclose%
\pgfusepath{fill}%
\end{pgfscope}%
\begin{pgfscope}%
\pgfpathrectangle{\pgfqpoint{1.250000in}{0.550000in}}{\pgfqpoint{7.750000in}{3.850000in}}%
\pgfusepath{clip}%
\pgfsetbuttcap%
\pgfsetmiterjoin%
\definecolor{currentfill}{rgb}{0.501961,0.501961,0.501961}%
\pgfsetfillcolor{currentfill}%
\pgfsetfillopacity{0.700000}%
\pgfsetlinewidth{0.000000pt}%
\definecolor{currentstroke}{rgb}{0.000000,0.000000,0.000000}%
\pgfsetstrokecolor{currentstroke}%
\pgfsetstrokeopacity{0.700000}%
\pgfsetdash{}{0pt}%
\pgfpathmoveto{\pgfqpoint{8.344993in}{0.550000in}}%
\pgfpathlineto{\pgfqpoint{8.372514in}{0.550000in}}%
\pgfpathlineto{\pgfqpoint{8.372514in}{0.785226in}}%
\pgfpathlineto{\pgfqpoint{8.344993in}{0.785226in}}%
\pgfpathlineto{\pgfqpoint{8.344993in}{0.550000in}}%
\pgfpathclose%
\pgfusepath{fill}%
\end{pgfscope}%
\begin{pgfscope}%
\pgfpathrectangle{\pgfqpoint{1.250000in}{0.550000in}}{\pgfqpoint{7.750000in}{3.850000in}}%
\pgfusepath{clip}%
\pgfsetbuttcap%
\pgfsetmiterjoin%
\definecolor{currentfill}{rgb}{0.501961,0.501961,0.501961}%
\pgfsetfillcolor{currentfill}%
\pgfsetfillopacity{0.700000}%
\pgfsetlinewidth{0.000000pt}%
\definecolor{currentstroke}{rgb}{0.000000,0.000000,0.000000}%
\pgfsetstrokecolor{currentstroke}%
\pgfsetstrokeopacity{0.700000}%
\pgfsetdash{}{0pt}%
\pgfpathmoveto{\pgfqpoint{8.372514in}{0.550000in}}%
\pgfpathlineto{\pgfqpoint{8.400036in}{0.550000in}}%
\pgfpathlineto{\pgfqpoint{8.400036in}{0.705391in}}%
\pgfpathlineto{\pgfqpoint{8.372514in}{0.705391in}}%
\pgfpathlineto{\pgfqpoint{8.372514in}{0.550000in}}%
\pgfpathclose%
\pgfusepath{fill}%
\end{pgfscope}%
\begin{pgfscope}%
\pgfpathrectangle{\pgfqpoint{1.250000in}{0.550000in}}{\pgfqpoint{7.750000in}{3.850000in}}%
\pgfusepath{clip}%
\pgfsetbuttcap%
\pgfsetmiterjoin%
\definecolor{currentfill}{rgb}{0.501961,0.501961,0.501961}%
\pgfsetfillcolor{currentfill}%
\pgfsetfillopacity{0.700000}%
\pgfsetlinewidth{0.000000pt}%
\definecolor{currentstroke}{rgb}{0.000000,0.000000,0.000000}%
\pgfsetstrokecolor{currentstroke}%
\pgfsetstrokeopacity{0.700000}%
\pgfsetdash{}{0pt}%
\pgfpathmoveto{\pgfqpoint{8.400036in}{0.550000in}}%
\pgfpathlineto{\pgfqpoint{8.427557in}{0.550000in}}%
\pgfpathlineto{\pgfqpoint{8.427557in}{0.772395in}}%
\pgfpathlineto{\pgfqpoint{8.400036in}{0.772395in}}%
\pgfpathlineto{\pgfqpoint{8.400036in}{0.550000in}}%
\pgfpathclose%
\pgfusepath{fill}%
\end{pgfscope}%
\begin{pgfscope}%
\pgfpathrectangle{\pgfqpoint{1.250000in}{0.550000in}}{\pgfqpoint{7.750000in}{3.850000in}}%
\pgfusepath{clip}%
\pgfsetbuttcap%
\pgfsetmiterjoin%
\definecolor{currentfill}{rgb}{0.501961,0.501961,0.501961}%
\pgfsetfillcolor{currentfill}%
\pgfsetfillopacity{0.700000}%
\pgfsetlinewidth{0.000000pt}%
\definecolor{currentstroke}{rgb}{0.000000,0.000000,0.000000}%
\pgfsetstrokecolor{currentstroke}%
\pgfsetstrokeopacity{0.700000}%
\pgfsetdash{}{0pt}%
\pgfpathmoveto{\pgfqpoint{8.427557in}{0.550000in}}%
\pgfpathlineto{\pgfqpoint{8.455078in}{0.550000in}}%
\pgfpathlineto{\pgfqpoint{8.455078in}{0.742457in}}%
\pgfpathlineto{\pgfqpoint{8.427557in}{0.742457in}}%
\pgfpathlineto{\pgfqpoint{8.427557in}{0.550000in}}%
\pgfpathclose%
\pgfusepath{fill}%
\end{pgfscope}%
\begin{pgfscope}%
\pgfpathrectangle{\pgfqpoint{1.250000in}{0.550000in}}{\pgfqpoint{7.750000in}{3.850000in}}%
\pgfusepath{clip}%
\pgfsetbuttcap%
\pgfsetmiterjoin%
\definecolor{currentfill}{rgb}{0.501961,0.501961,0.501961}%
\pgfsetfillcolor{currentfill}%
\pgfsetfillopacity{0.700000}%
\pgfsetlinewidth{0.000000pt}%
\definecolor{currentstroke}{rgb}{0.000000,0.000000,0.000000}%
\pgfsetstrokecolor{currentstroke}%
\pgfsetstrokeopacity{0.700000}%
\pgfsetdash{}{0pt}%
\pgfpathmoveto{\pgfqpoint{8.455078in}{0.550000in}}%
\pgfpathlineto{\pgfqpoint{8.482599in}{0.550000in}}%
\pgfpathlineto{\pgfqpoint{8.482599in}{0.738180in}}%
\pgfpathlineto{\pgfqpoint{8.455078in}{0.738180in}}%
\pgfpathlineto{\pgfqpoint{8.455078in}{0.550000in}}%
\pgfpathclose%
\pgfusepath{fill}%
\end{pgfscope}%
\begin{pgfscope}%
\pgfpathrectangle{\pgfqpoint{1.250000in}{0.550000in}}{\pgfqpoint{7.750000in}{3.850000in}}%
\pgfusepath{clip}%
\pgfsetbuttcap%
\pgfsetmiterjoin%
\definecolor{currentfill}{rgb}{0.501961,0.501961,0.501961}%
\pgfsetfillcolor{currentfill}%
\pgfsetfillopacity{0.700000}%
\pgfsetlinewidth{0.000000pt}%
\definecolor{currentstroke}{rgb}{0.000000,0.000000,0.000000}%
\pgfsetstrokecolor{currentstroke}%
\pgfsetstrokeopacity{0.700000}%
\pgfsetdash{}{0pt}%
\pgfpathmoveto{\pgfqpoint{8.482599in}{0.550000in}}%
\pgfpathlineto{\pgfqpoint{8.510121in}{0.550000in}}%
\pgfpathlineto{\pgfqpoint{8.510121in}{0.760990in}}%
\pgfpathlineto{\pgfqpoint{8.482599in}{0.760990in}}%
\pgfpathlineto{\pgfqpoint{8.482599in}{0.550000in}}%
\pgfpathclose%
\pgfusepath{fill}%
\end{pgfscope}%
\begin{pgfscope}%
\pgfpathrectangle{\pgfqpoint{1.250000in}{0.550000in}}{\pgfqpoint{7.750000in}{3.850000in}}%
\pgfusepath{clip}%
\pgfsetbuttcap%
\pgfsetmiterjoin%
\definecolor{currentfill}{rgb}{0.501961,0.501961,0.501961}%
\pgfsetfillcolor{currentfill}%
\pgfsetfillopacity{0.700000}%
\pgfsetlinewidth{0.000000pt}%
\definecolor{currentstroke}{rgb}{0.000000,0.000000,0.000000}%
\pgfsetstrokecolor{currentstroke}%
\pgfsetstrokeopacity{0.700000}%
\pgfsetdash{}{0pt}%
\pgfpathmoveto{\pgfqpoint{8.510121in}{0.550000in}}%
\pgfpathlineto{\pgfqpoint{8.537642in}{0.550000in}}%
\pgfpathlineto{\pgfqpoint{8.537642in}{0.733904in}}%
\pgfpathlineto{\pgfqpoint{8.510121in}{0.733904in}}%
\pgfpathlineto{\pgfqpoint{8.510121in}{0.550000in}}%
\pgfpathclose%
\pgfusepath{fill}%
\end{pgfscope}%
\begin{pgfscope}%
\pgfpathrectangle{\pgfqpoint{1.250000in}{0.550000in}}{\pgfqpoint{7.750000in}{3.850000in}}%
\pgfusepath{clip}%
\pgfsetbuttcap%
\pgfsetmiterjoin%
\definecolor{currentfill}{rgb}{0.501961,0.501961,0.501961}%
\pgfsetfillcolor{currentfill}%
\pgfsetfillopacity{0.700000}%
\pgfsetlinewidth{0.000000pt}%
\definecolor{currentstroke}{rgb}{0.000000,0.000000,0.000000}%
\pgfsetstrokecolor{currentstroke}%
\pgfsetstrokeopacity{0.700000}%
\pgfsetdash{}{0pt}%
\pgfpathmoveto{\pgfqpoint{8.537642in}{0.550000in}}%
\pgfpathlineto{\pgfqpoint{8.565163in}{0.550000in}}%
\pgfpathlineto{\pgfqpoint{8.565163in}{0.741032in}}%
\pgfpathlineto{\pgfqpoint{8.537642in}{0.741032in}}%
\pgfpathlineto{\pgfqpoint{8.537642in}{0.550000in}}%
\pgfpathclose%
\pgfusepath{fill}%
\end{pgfscope}%
\begin{pgfscope}%
\pgfpathrectangle{\pgfqpoint{1.250000in}{0.550000in}}{\pgfqpoint{7.750000in}{3.850000in}}%
\pgfusepath{clip}%
\pgfsetbuttcap%
\pgfsetmiterjoin%
\definecolor{currentfill}{rgb}{0.501961,0.501961,0.501961}%
\pgfsetfillcolor{currentfill}%
\pgfsetfillopacity{0.700000}%
\pgfsetlinewidth{0.000000pt}%
\definecolor{currentstroke}{rgb}{0.000000,0.000000,0.000000}%
\pgfsetstrokecolor{currentstroke}%
\pgfsetstrokeopacity{0.700000}%
\pgfsetdash{}{0pt}%
\pgfpathmoveto{\pgfqpoint{8.565163in}{0.550000in}}%
\pgfpathlineto{\pgfqpoint{8.592685in}{0.550000in}}%
\pgfpathlineto{\pgfqpoint{8.592685in}{0.728201in}}%
\pgfpathlineto{\pgfqpoint{8.565163in}{0.728201in}}%
\pgfpathlineto{\pgfqpoint{8.565163in}{0.550000in}}%
\pgfpathclose%
\pgfusepath{fill}%
\end{pgfscope}%
\begin{pgfscope}%
\pgfpathrectangle{\pgfqpoint{1.250000in}{0.550000in}}{\pgfqpoint{7.750000in}{3.850000in}}%
\pgfusepath{clip}%
\pgfsetbuttcap%
\pgfsetmiterjoin%
\definecolor{currentfill}{rgb}{0.501961,0.501961,0.501961}%
\pgfsetfillcolor{currentfill}%
\pgfsetfillopacity{0.700000}%
\pgfsetlinewidth{0.000000pt}%
\definecolor{currentstroke}{rgb}{0.000000,0.000000,0.000000}%
\pgfsetstrokecolor{currentstroke}%
\pgfsetstrokeopacity{0.700000}%
\pgfsetdash{}{0pt}%
\pgfpathmoveto{\pgfqpoint{8.592685in}{0.550000in}}%
\pgfpathlineto{\pgfqpoint{8.620206in}{0.550000in}}%
\pgfpathlineto{\pgfqpoint{8.620206in}{0.735329in}}%
\pgfpathlineto{\pgfqpoint{8.592685in}{0.735329in}}%
\pgfpathlineto{\pgfqpoint{8.592685in}{0.550000in}}%
\pgfpathclose%
\pgfusepath{fill}%
\end{pgfscope}%
\begin{pgfscope}%
\pgfpathrectangle{\pgfqpoint{1.250000in}{0.550000in}}{\pgfqpoint{7.750000in}{3.850000in}}%
\pgfusepath{clip}%
\pgfsetbuttcap%
\pgfsetmiterjoin%
\definecolor{currentfill}{rgb}{0.501961,0.501961,0.501961}%
\pgfsetfillcolor{currentfill}%
\pgfsetfillopacity{0.700000}%
\pgfsetlinewidth{0.000000pt}%
\definecolor{currentstroke}{rgb}{0.000000,0.000000,0.000000}%
\pgfsetstrokecolor{currentstroke}%
\pgfsetstrokeopacity{0.700000}%
\pgfsetdash{}{0pt}%
\pgfpathmoveto{\pgfqpoint{8.620206in}{0.550000in}}%
\pgfpathlineto{\pgfqpoint{8.647727in}{0.550000in}}%
\pgfpathlineto{\pgfqpoint{8.647727in}{0.666900in}}%
\pgfpathlineto{\pgfqpoint{8.620206in}{0.666900in}}%
\pgfpathlineto{\pgfqpoint{8.620206in}{0.550000in}}%
\pgfpathclose%
\pgfusepath{fill}%
\end{pgfscope}%
\begin{pgfscope}%
\pgfsetbuttcap%
\pgfsetroundjoin%
\definecolor{currentfill}{rgb}{0.000000,0.000000,0.000000}%
\pgfsetfillcolor{currentfill}%
\pgfsetlinewidth{0.803000pt}%
\definecolor{currentstroke}{rgb}{0.000000,0.000000,0.000000}%
\pgfsetstrokecolor{currentstroke}%
\pgfsetdash{}{0pt}%
\pgfsys@defobject{currentmarker}{\pgfqpoint{0.000000in}{-0.048611in}}{\pgfqpoint{0.000000in}{0.000000in}}{%
\pgfpathmoveto{\pgfqpoint{0.000000in}{0.000000in}}%
\pgfpathlineto{\pgfqpoint{0.000000in}{-0.048611in}}%
\pgfusepath{stroke,fill}%
}%
\begin{pgfscope}%
\pgfsys@transformshift{1.602273in}{0.550000in}%
\pgfsys@useobject{currentmarker}{}%
\end{pgfscope}%
\end{pgfscope}%
\begin{pgfscope}%
\definecolor{textcolor}{rgb}{0.000000,0.000000,0.000000}%
\pgfsetstrokecolor{textcolor}%
\pgfsetfillcolor{textcolor}%
\pgftext[x=1.602273in,y=0.452778in,,top]{\color{textcolor}{\sffamily\fontsize{10.000000}{12.000000}\selectfont\catcode`\^=\active\def^{\ifmmode\sp\else\^{}\fi}\catcode`\%=\active\def%{\%}0}}%
\end{pgfscope}%
\begin{pgfscope}%
\pgfsetbuttcap%
\pgfsetroundjoin%
\definecolor{currentfill}{rgb}{0.000000,0.000000,0.000000}%
\pgfsetfillcolor{currentfill}%
\pgfsetlinewidth{0.803000pt}%
\definecolor{currentstroke}{rgb}{0.000000,0.000000,0.000000}%
\pgfsetstrokecolor{currentstroke}%
\pgfsetdash{}{0pt}%
\pgfsys@defobject{currentmarker}{\pgfqpoint{0.000000in}{-0.048611in}}{\pgfqpoint{0.000000in}{0.000000in}}{%
\pgfpathmoveto{\pgfqpoint{0.000000in}{0.000000in}}%
\pgfpathlineto{\pgfqpoint{0.000000in}{-0.048611in}}%
\pgfusepath{stroke,fill}%
}%
\begin{pgfscope}%
\pgfsys@transformshift{2.978338in}{0.550000in}%
\pgfsys@useobject{currentmarker}{}%
\end{pgfscope}%
\end{pgfscope}%
\begin{pgfscope}%
\definecolor{textcolor}{rgb}{0.000000,0.000000,0.000000}%
\pgfsetstrokecolor{textcolor}%
\pgfsetfillcolor{textcolor}%
\pgftext[x=2.978338in,y=0.452778in,,top]{\color{textcolor}{\sffamily\fontsize{10.000000}{12.000000}\selectfont\catcode`\^=\active\def^{\ifmmode\sp\else\^{}\fi}\catcode`\%=\active\def%{\%}50}}%
\end{pgfscope}%
\begin{pgfscope}%
\pgfsetbuttcap%
\pgfsetroundjoin%
\definecolor{currentfill}{rgb}{0.000000,0.000000,0.000000}%
\pgfsetfillcolor{currentfill}%
\pgfsetlinewidth{0.803000pt}%
\definecolor{currentstroke}{rgb}{0.000000,0.000000,0.000000}%
\pgfsetstrokecolor{currentstroke}%
\pgfsetdash{}{0pt}%
\pgfsys@defobject{currentmarker}{\pgfqpoint{0.000000in}{-0.048611in}}{\pgfqpoint{0.000000in}{0.000000in}}{%
\pgfpathmoveto{\pgfqpoint{0.000000in}{0.000000in}}%
\pgfpathlineto{\pgfqpoint{0.000000in}{-0.048611in}}%
\pgfusepath{stroke,fill}%
}%
\begin{pgfscope}%
\pgfsys@transformshift{4.354403in}{0.550000in}%
\pgfsys@useobject{currentmarker}{}%
\end{pgfscope}%
\end{pgfscope}%
\begin{pgfscope}%
\definecolor{textcolor}{rgb}{0.000000,0.000000,0.000000}%
\pgfsetstrokecolor{textcolor}%
\pgfsetfillcolor{textcolor}%
\pgftext[x=4.354403in,y=0.452778in,,top]{\color{textcolor}{\sffamily\fontsize{10.000000}{12.000000}\selectfont\catcode`\^=\active\def^{\ifmmode\sp\else\^{}\fi}\catcode`\%=\active\def%{\%}100}}%
\end{pgfscope}%
\begin{pgfscope}%
\pgfsetbuttcap%
\pgfsetroundjoin%
\definecolor{currentfill}{rgb}{0.000000,0.000000,0.000000}%
\pgfsetfillcolor{currentfill}%
\pgfsetlinewidth{0.803000pt}%
\definecolor{currentstroke}{rgb}{0.000000,0.000000,0.000000}%
\pgfsetstrokecolor{currentstroke}%
\pgfsetdash{}{0pt}%
\pgfsys@defobject{currentmarker}{\pgfqpoint{0.000000in}{-0.048611in}}{\pgfqpoint{0.000000in}{0.000000in}}{%
\pgfpathmoveto{\pgfqpoint{0.000000in}{0.000000in}}%
\pgfpathlineto{\pgfqpoint{0.000000in}{-0.048611in}}%
\pgfusepath{stroke,fill}%
}%
\begin{pgfscope}%
\pgfsys@transformshift{5.730469in}{0.550000in}%
\pgfsys@useobject{currentmarker}{}%
\end{pgfscope}%
\end{pgfscope}%
\begin{pgfscope}%
\definecolor{textcolor}{rgb}{0.000000,0.000000,0.000000}%
\pgfsetstrokecolor{textcolor}%
\pgfsetfillcolor{textcolor}%
\pgftext[x=5.730469in,y=0.452778in,,top]{\color{textcolor}{\sffamily\fontsize{10.000000}{12.000000}\selectfont\catcode`\^=\active\def^{\ifmmode\sp\else\^{}\fi}\catcode`\%=\active\def%{\%}150}}%
\end{pgfscope}%
\begin{pgfscope}%
\pgfsetbuttcap%
\pgfsetroundjoin%
\definecolor{currentfill}{rgb}{0.000000,0.000000,0.000000}%
\pgfsetfillcolor{currentfill}%
\pgfsetlinewidth{0.803000pt}%
\definecolor{currentstroke}{rgb}{0.000000,0.000000,0.000000}%
\pgfsetstrokecolor{currentstroke}%
\pgfsetdash{}{0pt}%
\pgfsys@defobject{currentmarker}{\pgfqpoint{0.000000in}{-0.048611in}}{\pgfqpoint{0.000000in}{0.000000in}}{%
\pgfpathmoveto{\pgfqpoint{0.000000in}{0.000000in}}%
\pgfpathlineto{\pgfqpoint{0.000000in}{-0.048611in}}%
\pgfusepath{stroke,fill}%
}%
\begin{pgfscope}%
\pgfsys@transformshift{7.106534in}{0.550000in}%
\pgfsys@useobject{currentmarker}{}%
\end{pgfscope}%
\end{pgfscope}%
\begin{pgfscope}%
\definecolor{textcolor}{rgb}{0.000000,0.000000,0.000000}%
\pgfsetstrokecolor{textcolor}%
\pgfsetfillcolor{textcolor}%
\pgftext[x=7.106534in,y=0.452778in,,top]{\color{textcolor}{\sffamily\fontsize{10.000000}{12.000000}\selectfont\catcode`\^=\active\def^{\ifmmode\sp\else\^{}\fi}\catcode`\%=\active\def%{\%}200}}%
\end{pgfscope}%
\begin{pgfscope}%
\pgfsetbuttcap%
\pgfsetroundjoin%
\definecolor{currentfill}{rgb}{0.000000,0.000000,0.000000}%
\pgfsetfillcolor{currentfill}%
\pgfsetlinewidth{0.803000pt}%
\definecolor{currentstroke}{rgb}{0.000000,0.000000,0.000000}%
\pgfsetstrokecolor{currentstroke}%
\pgfsetdash{}{0pt}%
\pgfsys@defobject{currentmarker}{\pgfqpoint{0.000000in}{-0.048611in}}{\pgfqpoint{0.000000in}{0.000000in}}{%
\pgfpathmoveto{\pgfqpoint{0.000000in}{0.000000in}}%
\pgfpathlineto{\pgfqpoint{0.000000in}{-0.048611in}}%
\pgfusepath{stroke,fill}%
}%
\begin{pgfscope}%
\pgfsys@transformshift{8.482599in}{0.550000in}%
\pgfsys@useobject{currentmarker}{}%
\end{pgfscope}%
\end{pgfscope}%
\begin{pgfscope}%
\definecolor{textcolor}{rgb}{0.000000,0.000000,0.000000}%
\pgfsetstrokecolor{textcolor}%
\pgfsetfillcolor{textcolor}%
\pgftext[x=8.482599in,y=0.452778in,,top]{\color{textcolor}{\sffamily\fontsize{10.000000}{12.000000}\selectfont\catcode`\^=\active\def^{\ifmmode\sp\else\^{}\fi}\catcode`\%=\active\def%{\%}250}}%
\end{pgfscope}%
\begin{pgfscope}%
\definecolor{textcolor}{rgb}{0.000000,0.000000,0.000000}%
\pgfsetstrokecolor{textcolor}%
\pgfsetfillcolor{textcolor}%
\pgftext[x=5.125000in,y=0.262809in,,top]{\color{textcolor}{\sffamily\fontsize{10.000000}{12.000000}\selectfont\catcode`\^=\active\def^{\ifmmode\sp\else\^{}\fi}\catcode`\%=\active\def%{\%}Intensidad}}%
\end{pgfscope}%
\begin{pgfscope}%
\pgfsetbuttcap%
\pgfsetroundjoin%
\definecolor{currentfill}{rgb}{0.000000,0.000000,0.000000}%
\pgfsetfillcolor{currentfill}%
\pgfsetlinewidth{0.803000pt}%
\definecolor{currentstroke}{rgb}{0.000000,0.000000,0.000000}%
\pgfsetstrokecolor{currentstroke}%
\pgfsetdash{}{0pt}%
\pgfsys@defobject{currentmarker}{\pgfqpoint{-0.048611in}{0.000000in}}{\pgfqpoint{-0.000000in}{0.000000in}}{%
\pgfpathmoveto{\pgfqpoint{-0.000000in}{0.000000in}}%
\pgfpathlineto{\pgfqpoint{-0.048611in}{0.000000in}}%
\pgfusepath{stroke,fill}%
}%
\begin{pgfscope}%
\pgfsys@transformshift{1.250000in}{0.550000in}%
\pgfsys@useobject{currentmarker}{}%
\end{pgfscope}%
\end{pgfscope}%
\begin{pgfscope}%
\definecolor{textcolor}{rgb}{0.000000,0.000000,0.000000}%
\pgfsetstrokecolor{textcolor}%
\pgfsetfillcolor{textcolor}%
\pgftext[x=1.064412in, y=0.497238in, left, base]{\color{textcolor}{\sffamily\fontsize{10.000000}{12.000000}\selectfont\catcode`\^=\active\def^{\ifmmode\sp\else\^{}\fi}\catcode`\%=\active\def%{\%}0}}%
\end{pgfscope}%
\begin{pgfscope}%
\pgfsetbuttcap%
\pgfsetroundjoin%
\definecolor{currentfill}{rgb}{0.000000,0.000000,0.000000}%
\pgfsetfillcolor{currentfill}%
\pgfsetlinewidth{0.803000pt}%
\definecolor{currentstroke}{rgb}{0.000000,0.000000,0.000000}%
\pgfsetstrokecolor{currentstroke}%
\pgfsetdash{}{0pt}%
\pgfsys@defobject{currentmarker}{\pgfqpoint{-0.048611in}{0.000000in}}{\pgfqpoint{-0.000000in}{0.000000in}}{%
\pgfpathmoveto{\pgfqpoint{-0.000000in}{0.000000in}}%
\pgfpathlineto{\pgfqpoint{-0.048611in}{0.000000in}}%
\pgfusepath{stroke,fill}%
}%
\begin{pgfscope}%
\pgfsys@transformshift{1.250000in}{1.262805in}%
\pgfsys@useobject{currentmarker}{}%
\end{pgfscope}%
\end{pgfscope}%
\begin{pgfscope}%
\definecolor{textcolor}{rgb}{0.000000,0.000000,0.000000}%
\pgfsetstrokecolor{textcolor}%
\pgfsetfillcolor{textcolor}%
\pgftext[x=0.887682in, y=1.210043in, left, base]{\color{textcolor}{\sffamily\fontsize{10.000000}{12.000000}\selectfont\catcode`\^=\active\def^{\ifmmode\sp\else\^{}\fi}\catcode`\%=\active\def%{\%}500}}%
\end{pgfscope}%
\begin{pgfscope}%
\pgfsetbuttcap%
\pgfsetroundjoin%
\definecolor{currentfill}{rgb}{0.000000,0.000000,0.000000}%
\pgfsetfillcolor{currentfill}%
\pgfsetlinewidth{0.803000pt}%
\definecolor{currentstroke}{rgb}{0.000000,0.000000,0.000000}%
\pgfsetstrokecolor{currentstroke}%
\pgfsetdash{}{0pt}%
\pgfsys@defobject{currentmarker}{\pgfqpoint{-0.048611in}{0.000000in}}{\pgfqpoint{-0.000000in}{0.000000in}}{%
\pgfpathmoveto{\pgfqpoint{-0.000000in}{0.000000in}}%
\pgfpathlineto{\pgfqpoint{-0.048611in}{0.000000in}}%
\pgfusepath{stroke,fill}%
}%
\begin{pgfscope}%
\pgfsys@transformshift{1.250000in}{1.975609in}%
\pgfsys@useobject{currentmarker}{}%
\end{pgfscope}%
\end{pgfscope}%
\begin{pgfscope}%
\definecolor{textcolor}{rgb}{0.000000,0.000000,0.000000}%
\pgfsetstrokecolor{textcolor}%
\pgfsetfillcolor{textcolor}%
\pgftext[x=0.799316in, y=1.922848in, left, base]{\color{textcolor}{\sffamily\fontsize{10.000000}{12.000000}\selectfont\catcode`\^=\active\def^{\ifmmode\sp\else\^{}\fi}\catcode`\%=\active\def%{\%}1000}}%
\end{pgfscope}%
\begin{pgfscope}%
\pgfsetbuttcap%
\pgfsetroundjoin%
\definecolor{currentfill}{rgb}{0.000000,0.000000,0.000000}%
\pgfsetfillcolor{currentfill}%
\pgfsetlinewidth{0.803000pt}%
\definecolor{currentstroke}{rgb}{0.000000,0.000000,0.000000}%
\pgfsetstrokecolor{currentstroke}%
\pgfsetdash{}{0pt}%
\pgfsys@defobject{currentmarker}{\pgfqpoint{-0.048611in}{0.000000in}}{\pgfqpoint{-0.000000in}{0.000000in}}{%
\pgfpathmoveto{\pgfqpoint{-0.000000in}{0.000000in}}%
\pgfpathlineto{\pgfqpoint{-0.048611in}{0.000000in}}%
\pgfusepath{stroke,fill}%
}%
\begin{pgfscope}%
\pgfsys@transformshift{1.250000in}{2.688414in}%
\pgfsys@useobject{currentmarker}{}%
\end{pgfscope}%
\end{pgfscope}%
\begin{pgfscope}%
\definecolor{textcolor}{rgb}{0.000000,0.000000,0.000000}%
\pgfsetstrokecolor{textcolor}%
\pgfsetfillcolor{textcolor}%
\pgftext[x=0.799316in, y=2.635652in, left, base]{\color{textcolor}{\sffamily\fontsize{10.000000}{12.000000}\selectfont\catcode`\^=\active\def^{\ifmmode\sp\else\^{}\fi}\catcode`\%=\active\def%{\%}1500}}%
\end{pgfscope}%
\begin{pgfscope}%
\pgfsetbuttcap%
\pgfsetroundjoin%
\definecolor{currentfill}{rgb}{0.000000,0.000000,0.000000}%
\pgfsetfillcolor{currentfill}%
\pgfsetlinewidth{0.803000pt}%
\definecolor{currentstroke}{rgb}{0.000000,0.000000,0.000000}%
\pgfsetstrokecolor{currentstroke}%
\pgfsetdash{}{0pt}%
\pgfsys@defobject{currentmarker}{\pgfqpoint{-0.048611in}{0.000000in}}{\pgfqpoint{-0.000000in}{0.000000in}}{%
\pgfpathmoveto{\pgfqpoint{-0.000000in}{0.000000in}}%
\pgfpathlineto{\pgfqpoint{-0.048611in}{0.000000in}}%
\pgfusepath{stroke,fill}%
}%
\begin{pgfscope}%
\pgfsys@transformshift{1.250000in}{3.401218in}%
\pgfsys@useobject{currentmarker}{}%
\end{pgfscope}%
\end{pgfscope}%
\begin{pgfscope}%
\definecolor{textcolor}{rgb}{0.000000,0.000000,0.000000}%
\pgfsetstrokecolor{textcolor}%
\pgfsetfillcolor{textcolor}%
\pgftext[x=0.799316in, y=3.348457in, left, base]{\color{textcolor}{\sffamily\fontsize{10.000000}{12.000000}\selectfont\catcode`\^=\active\def^{\ifmmode\sp\else\^{}\fi}\catcode`\%=\active\def%{\%}2000}}%
\end{pgfscope}%
\begin{pgfscope}%
\pgfsetbuttcap%
\pgfsetroundjoin%
\definecolor{currentfill}{rgb}{0.000000,0.000000,0.000000}%
\pgfsetfillcolor{currentfill}%
\pgfsetlinewidth{0.803000pt}%
\definecolor{currentstroke}{rgb}{0.000000,0.000000,0.000000}%
\pgfsetstrokecolor{currentstroke}%
\pgfsetdash{}{0pt}%
\pgfsys@defobject{currentmarker}{\pgfqpoint{-0.048611in}{0.000000in}}{\pgfqpoint{-0.000000in}{0.000000in}}{%
\pgfpathmoveto{\pgfqpoint{-0.000000in}{0.000000in}}%
\pgfpathlineto{\pgfqpoint{-0.048611in}{0.000000in}}%
\pgfusepath{stroke,fill}%
}%
\begin{pgfscope}%
\pgfsys@transformshift{1.250000in}{4.114023in}%
\pgfsys@useobject{currentmarker}{}%
\end{pgfscope}%
\end{pgfscope}%
\begin{pgfscope}%
\definecolor{textcolor}{rgb}{0.000000,0.000000,0.000000}%
\pgfsetstrokecolor{textcolor}%
\pgfsetfillcolor{textcolor}%
\pgftext[x=0.799316in, y=4.061261in, left, base]{\color{textcolor}{\sffamily\fontsize{10.000000}{12.000000}\selectfont\catcode`\^=\active\def^{\ifmmode\sp\else\^{}\fi}\catcode`\%=\active\def%{\%}2500}}%
\end{pgfscope}%
\begin{pgfscope}%
\definecolor{textcolor}{rgb}{0.000000,0.000000,0.000000}%
\pgfsetstrokecolor{textcolor}%
\pgfsetfillcolor{textcolor}%
\pgftext[x=0.743761in,y=2.475000in,,bottom,rotate=90.000000]{\color{textcolor}{\sffamily\fontsize{10.000000}{12.000000}\selectfont\catcode`\^=\active\def^{\ifmmode\sp\else\^{}\fi}\catcode`\%=\active\def%{\%}Frecuencia}}%
\end{pgfscope}%
\begin{pgfscope}%
\pgfsetrectcap%
\pgfsetmiterjoin%
\pgfsetlinewidth{0.803000pt}%
\definecolor{currentstroke}{rgb}{0.000000,0.000000,0.000000}%
\pgfsetstrokecolor{currentstroke}%
\pgfsetdash{}{0pt}%
\pgfpathmoveto{\pgfqpoint{1.250000in}{0.550000in}}%
\pgfpathlineto{\pgfqpoint{1.250000in}{4.400000in}}%
\pgfusepath{stroke}%
\end{pgfscope}%
\begin{pgfscope}%
\pgfsetrectcap%
\pgfsetmiterjoin%
\pgfsetlinewidth{0.803000pt}%
\definecolor{currentstroke}{rgb}{0.000000,0.000000,0.000000}%
\pgfsetstrokecolor{currentstroke}%
\pgfsetdash{}{0pt}%
\pgfpathmoveto{\pgfqpoint{9.000000in}{0.550000in}}%
\pgfpathlineto{\pgfqpoint{9.000000in}{4.400000in}}%
\pgfusepath{stroke}%
\end{pgfscope}%
\begin{pgfscope}%
\pgfsetrectcap%
\pgfsetmiterjoin%
\pgfsetlinewidth{0.803000pt}%
\definecolor{currentstroke}{rgb}{0.000000,0.000000,0.000000}%
\pgfsetstrokecolor{currentstroke}%
\pgfsetdash{}{0pt}%
\pgfpathmoveto{\pgfqpoint{1.250000in}{0.550000in}}%
\pgfpathlineto{\pgfqpoint{9.000000in}{0.550000in}}%
\pgfusepath{stroke}%
\end{pgfscope}%
\begin{pgfscope}%
\pgfsetrectcap%
\pgfsetmiterjoin%
\pgfsetlinewidth{0.803000pt}%
\definecolor{currentstroke}{rgb}{0.000000,0.000000,0.000000}%
\pgfsetstrokecolor{currentstroke}%
\pgfsetdash{}{0pt}%
\pgfpathmoveto{\pgfqpoint{1.250000in}{4.400000in}}%
\pgfpathlineto{\pgfqpoint{9.000000in}{4.400000in}}%
\pgfusepath{stroke}%
\end{pgfscope}%
\begin{pgfscope}%
\definecolor{textcolor}{rgb}{0.000000,0.000000,0.000000}%
\pgfsetstrokecolor{textcolor}%
\pgfsetfillcolor{textcolor}%
\pgftext[x=5.125000in,y=4.483333in,,base]{\color{textcolor}{\sffamily\fontsize{12.000000}{14.400000}\selectfont\catcode`\^=\active\def^{\ifmmode\sp\else\^{}\fi}\catcode`\%=\active\def%{\%}Histograma de DQHEPL}}%
\end{pgfscope}%
\end{pgfpicture}%
\makeatother%
\endgroup%
}
		\caption{DQHEPL}
		\label{fig:dqhepl_1}
	\end{subfigure}

	% Espacio entre filas
	\vspace{5mm}

	% Segunda fila (2 histogramas)
	\begin{subfigure}[b]{0.45\textwidth}
		\resizebox{\linewidth}{!}{%% Creator: Matplotlib, PGF backend
%%
%% To include the figure in your LaTeX document, write
%%   \input{<filename>.pgf}
%%
%% Make sure the required packages are loaded in your preamble
%%   \usepackage{pgf}
%%
%% Also ensure that all the required font packages are loaded; for instance,
%% the lmodern package is sometimes necessary when using math font.
%%   \usepackage{lmodern}
%%
%% Figures using additional raster images can only be included by \input if
%% they are in the same directory as the main LaTeX file. For loading figures
%% from other directories you can use the `import` package
%%   \usepackage{import}
%%
%% and then include the figures with
%%   \import{<path to file>}{<filename>.pgf}
%%
%% Matplotlib used the following preamble
%%   \def\mathdefault#1{#1}
%%   \everymath=\expandafter{\the\everymath\displaystyle}
%%   \IfFileExists{scrextend.sty}{
%%     \usepackage[fontsize=10.000000pt]{scrextend}
%%   }{
%%     \renewcommand{\normalsize}{\fontsize{10.000000}{12.000000}\selectfont}
%%     \normalsize
%%   }
%%   
%%   \ifdefined\pdftexversion\else  % non-pdftex case.
%%     \usepackage{fontspec}
%%     \setmainfont{DejaVuSerif.ttf}[Path=\detokenize{/home/elias/.virtualenvs/src-pway/lib/python3.13/site-packages/matplotlib/mpl-data/fonts/ttf/}]
%%     \setsansfont{DejaVuSans.ttf}[Path=\detokenize{/home/elias/.virtualenvs/src-pway/lib/python3.13/site-packages/matplotlib/mpl-data/fonts/ttf/}]
%%     \setmonofont{DejaVuSansMono.ttf}[Path=\detokenize{/home/elias/.virtualenvs/src-pway/lib/python3.13/site-packages/matplotlib/mpl-data/fonts/ttf/}]
%%   \fi
%%   \makeatletter\@ifpackageloaded{underscore}{}{\usepackage[strings]{underscore}}\makeatother
%%
\begingroup%
\makeatletter%
\begin{pgfpicture}%
\pgfpathrectangle{\pgfpointorigin}{\pgfqpoint{10.000000in}{5.000000in}}%
\pgfusepath{use as bounding box, clip}%
\begin{pgfscope}%
\pgfsetbuttcap%
\pgfsetmiterjoin%
\definecolor{currentfill}{rgb}{1.000000,1.000000,1.000000}%
\pgfsetfillcolor{currentfill}%
\pgfsetlinewidth{0.000000pt}%
\definecolor{currentstroke}{rgb}{1.000000,1.000000,1.000000}%
\pgfsetstrokecolor{currentstroke}%
\pgfsetdash{}{0pt}%
\pgfpathmoveto{\pgfqpoint{0.000000in}{0.000000in}}%
\pgfpathlineto{\pgfqpoint{10.000000in}{0.000000in}}%
\pgfpathlineto{\pgfqpoint{10.000000in}{5.000000in}}%
\pgfpathlineto{\pgfqpoint{0.000000in}{5.000000in}}%
\pgfpathlineto{\pgfqpoint{0.000000in}{0.000000in}}%
\pgfpathclose%
\pgfusepath{fill}%
\end{pgfscope}%
\begin{pgfscope}%
\pgfsetbuttcap%
\pgfsetmiterjoin%
\definecolor{currentfill}{rgb}{1.000000,1.000000,1.000000}%
\pgfsetfillcolor{currentfill}%
\pgfsetlinewidth{0.000000pt}%
\definecolor{currentstroke}{rgb}{0.000000,0.000000,0.000000}%
\pgfsetstrokecolor{currentstroke}%
\pgfsetstrokeopacity{0.000000}%
\pgfsetdash{}{0pt}%
\pgfpathmoveto{\pgfqpoint{1.250000in}{0.550000in}}%
\pgfpathlineto{\pgfqpoint{9.000000in}{0.550000in}}%
\pgfpathlineto{\pgfqpoint{9.000000in}{4.400000in}}%
\pgfpathlineto{\pgfqpoint{1.250000in}{4.400000in}}%
\pgfpathlineto{\pgfqpoint{1.250000in}{0.550000in}}%
\pgfpathclose%
\pgfusepath{fill}%
\end{pgfscope}%
\begin{pgfscope}%
\pgfpathrectangle{\pgfqpoint{1.250000in}{0.550000in}}{\pgfqpoint{7.750000in}{3.850000in}}%
\pgfusepath{clip}%
\pgfsetbuttcap%
\pgfsetmiterjoin%
\definecolor{currentfill}{rgb}{0.501961,0.501961,0.501961}%
\pgfsetfillcolor{currentfill}%
\pgfsetfillopacity{0.700000}%
\pgfsetlinewidth{0.000000pt}%
\definecolor{currentstroke}{rgb}{0.000000,0.000000,0.000000}%
\pgfsetstrokecolor{currentstroke}%
\pgfsetstrokeopacity{0.700000}%
\pgfsetdash{}{0pt}%
\pgfpathmoveto{\pgfqpoint{1.602273in}{0.550000in}}%
\pgfpathlineto{\pgfqpoint{1.629794in}{0.550000in}}%
\pgfpathlineto{\pgfqpoint{1.629794in}{4.216667in}}%
\pgfpathlineto{\pgfqpoint{1.602273in}{4.216667in}}%
\pgfpathlineto{\pgfqpoint{1.602273in}{0.550000in}}%
\pgfpathclose%
\pgfusepath{fill}%
\end{pgfscope}%
\begin{pgfscope}%
\pgfpathrectangle{\pgfqpoint{1.250000in}{0.550000in}}{\pgfqpoint{7.750000in}{3.850000in}}%
\pgfusepath{clip}%
\pgfsetbuttcap%
\pgfsetmiterjoin%
\definecolor{currentfill}{rgb}{0.501961,0.501961,0.501961}%
\pgfsetfillcolor{currentfill}%
\pgfsetfillopacity{0.700000}%
\pgfsetlinewidth{0.000000pt}%
\definecolor{currentstroke}{rgb}{0.000000,0.000000,0.000000}%
\pgfsetstrokecolor{currentstroke}%
\pgfsetstrokeopacity{0.700000}%
\pgfsetdash{}{0pt}%
\pgfpathmoveto{\pgfqpoint{1.629794in}{0.550000in}}%
\pgfpathlineto{\pgfqpoint{1.657315in}{0.550000in}}%
\pgfpathlineto{\pgfqpoint{1.657315in}{0.550000in}}%
\pgfpathlineto{\pgfqpoint{1.629794in}{0.550000in}}%
\pgfpathlineto{\pgfqpoint{1.629794in}{0.550000in}}%
\pgfpathclose%
\pgfusepath{fill}%
\end{pgfscope}%
\begin{pgfscope}%
\pgfpathrectangle{\pgfqpoint{1.250000in}{0.550000in}}{\pgfqpoint{7.750000in}{3.850000in}}%
\pgfusepath{clip}%
\pgfsetbuttcap%
\pgfsetmiterjoin%
\definecolor{currentfill}{rgb}{0.501961,0.501961,0.501961}%
\pgfsetfillcolor{currentfill}%
\pgfsetfillopacity{0.700000}%
\pgfsetlinewidth{0.000000pt}%
\definecolor{currentstroke}{rgb}{0.000000,0.000000,0.000000}%
\pgfsetstrokecolor{currentstroke}%
\pgfsetstrokeopacity{0.700000}%
\pgfsetdash{}{0pt}%
\pgfpathmoveto{\pgfqpoint{1.657315in}{0.550000in}}%
\pgfpathlineto{\pgfqpoint{1.684837in}{0.550000in}}%
\pgfpathlineto{\pgfqpoint{1.684837in}{0.550000in}}%
\pgfpathlineto{\pgfqpoint{1.657315in}{0.550000in}}%
\pgfpathlineto{\pgfqpoint{1.657315in}{0.550000in}}%
\pgfpathclose%
\pgfusepath{fill}%
\end{pgfscope}%
\begin{pgfscope}%
\pgfpathrectangle{\pgfqpoint{1.250000in}{0.550000in}}{\pgfqpoint{7.750000in}{3.850000in}}%
\pgfusepath{clip}%
\pgfsetbuttcap%
\pgfsetmiterjoin%
\definecolor{currentfill}{rgb}{0.501961,0.501961,0.501961}%
\pgfsetfillcolor{currentfill}%
\pgfsetfillopacity{0.700000}%
\pgfsetlinewidth{0.000000pt}%
\definecolor{currentstroke}{rgb}{0.000000,0.000000,0.000000}%
\pgfsetstrokecolor{currentstroke}%
\pgfsetstrokeopacity{0.700000}%
\pgfsetdash{}{0pt}%
\pgfpathmoveto{\pgfqpoint{1.684837in}{0.550000in}}%
\pgfpathlineto{\pgfqpoint{1.712358in}{0.550000in}}%
\pgfpathlineto{\pgfqpoint{1.712358in}{0.550000in}}%
\pgfpathlineto{\pgfqpoint{1.684837in}{0.550000in}}%
\pgfpathlineto{\pgfqpoint{1.684837in}{0.550000in}}%
\pgfpathclose%
\pgfusepath{fill}%
\end{pgfscope}%
\begin{pgfscope}%
\pgfpathrectangle{\pgfqpoint{1.250000in}{0.550000in}}{\pgfqpoint{7.750000in}{3.850000in}}%
\pgfusepath{clip}%
\pgfsetbuttcap%
\pgfsetmiterjoin%
\definecolor{currentfill}{rgb}{0.501961,0.501961,0.501961}%
\pgfsetfillcolor{currentfill}%
\pgfsetfillopacity{0.700000}%
\pgfsetlinewidth{0.000000pt}%
\definecolor{currentstroke}{rgb}{0.000000,0.000000,0.000000}%
\pgfsetstrokecolor{currentstroke}%
\pgfsetstrokeopacity{0.700000}%
\pgfsetdash{}{0pt}%
\pgfpathmoveto{\pgfqpoint{1.712358in}{0.550000in}}%
\pgfpathlineto{\pgfqpoint{1.739879in}{0.550000in}}%
\pgfpathlineto{\pgfqpoint{1.739879in}{0.550000in}}%
\pgfpathlineto{\pgfqpoint{1.712358in}{0.550000in}}%
\pgfpathlineto{\pgfqpoint{1.712358in}{0.550000in}}%
\pgfpathclose%
\pgfusepath{fill}%
\end{pgfscope}%
\begin{pgfscope}%
\pgfpathrectangle{\pgfqpoint{1.250000in}{0.550000in}}{\pgfqpoint{7.750000in}{3.850000in}}%
\pgfusepath{clip}%
\pgfsetbuttcap%
\pgfsetmiterjoin%
\definecolor{currentfill}{rgb}{0.501961,0.501961,0.501961}%
\pgfsetfillcolor{currentfill}%
\pgfsetfillopacity{0.700000}%
\pgfsetlinewidth{0.000000pt}%
\definecolor{currentstroke}{rgb}{0.000000,0.000000,0.000000}%
\pgfsetstrokecolor{currentstroke}%
\pgfsetstrokeopacity{0.700000}%
\pgfsetdash{}{0pt}%
\pgfpathmoveto{\pgfqpoint{1.739879in}{0.550000in}}%
\pgfpathlineto{\pgfqpoint{1.767401in}{0.550000in}}%
\pgfpathlineto{\pgfqpoint{1.767401in}{0.550000in}}%
\pgfpathlineto{\pgfqpoint{1.739879in}{0.550000in}}%
\pgfpathlineto{\pgfqpoint{1.739879in}{0.550000in}}%
\pgfpathclose%
\pgfusepath{fill}%
\end{pgfscope}%
\begin{pgfscope}%
\pgfpathrectangle{\pgfqpoint{1.250000in}{0.550000in}}{\pgfqpoint{7.750000in}{3.850000in}}%
\pgfusepath{clip}%
\pgfsetbuttcap%
\pgfsetmiterjoin%
\definecolor{currentfill}{rgb}{0.501961,0.501961,0.501961}%
\pgfsetfillcolor{currentfill}%
\pgfsetfillopacity{0.700000}%
\pgfsetlinewidth{0.000000pt}%
\definecolor{currentstroke}{rgb}{0.000000,0.000000,0.000000}%
\pgfsetstrokecolor{currentstroke}%
\pgfsetstrokeopacity{0.700000}%
\pgfsetdash{}{0pt}%
\pgfpathmoveto{\pgfqpoint{1.767401in}{0.550000in}}%
\pgfpathlineto{\pgfqpoint{1.794922in}{0.550000in}}%
\pgfpathlineto{\pgfqpoint{1.794922in}{0.550000in}}%
\pgfpathlineto{\pgfqpoint{1.767401in}{0.550000in}}%
\pgfpathlineto{\pgfqpoint{1.767401in}{0.550000in}}%
\pgfpathclose%
\pgfusepath{fill}%
\end{pgfscope}%
\begin{pgfscope}%
\pgfpathrectangle{\pgfqpoint{1.250000in}{0.550000in}}{\pgfqpoint{7.750000in}{3.850000in}}%
\pgfusepath{clip}%
\pgfsetbuttcap%
\pgfsetmiterjoin%
\definecolor{currentfill}{rgb}{0.501961,0.501961,0.501961}%
\pgfsetfillcolor{currentfill}%
\pgfsetfillopacity{0.700000}%
\pgfsetlinewidth{0.000000pt}%
\definecolor{currentstroke}{rgb}{0.000000,0.000000,0.000000}%
\pgfsetstrokecolor{currentstroke}%
\pgfsetstrokeopacity{0.700000}%
\pgfsetdash{}{0pt}%
\pgfpathmoveto{\pgfqpoint{1.794922in}{0.550000in}}%
\pgfpathlineto{\pgfqpoint{1.822443in}{0.550000in}}%
\pgfpathlineto{\pgfqpoint{1.822443in}{0.550000in}}%
\pgfpathlineto{\pgfqpoint{1.794922in}{0.550000in}}%
\pgfpathlineto{\pgfqpoint{1.794922in}{0.550000in}}%
\pgfpathclose%
\pgfusepath{fill}%
\end{pgfscope}%
\begin{pgfscope}%
\pgfpathrectangle{\pgfqpoint{1.250000in}{0.550000in}}{\pgfqpoint{7.750000in}{3.850000in}}%
\pgfusepath{clip}%
\pgfsetbuttcap%
\pgfsetmiterjoin%
\definecolor{currentfill}{rgb}{0.501961,0.501961,0.501961}%
\pgfsetfillcolor{currentfill}%
\pgfsetfillopacity{0.700000}%
\pgfsetlinewidth{0.000000pt}%
\definecolor{currentstroke}{rgb}{0.000000,0.000000,0.000000}%
\pgfsetstrokecolor{currentstroke}%
\pgfsetstrokeopacity{0.700000}%
\pgfsetdash{}{0pt}%
\pgfpathmoveto{\pgfqpoint{1.822443in}{0.550000in}}%
\pgfpathlineto{\pgfqpoint{1.849964in}{0.550000in}}%
\pgfpathlineto{\pgfqpoint{1.849964in}{0.550000in}}%
\pgfpathlineto{\pgfqpoint{1.822443in}{0.550000in}}%
\pgfpathlineto{\pgfqpoint{1.822443in}{0.550000in}}%
\pgfpathclose%
\pgfusepath{fill}%
\end{pgfscope}%
\begin{pgfscope}%
\pgfpathrectangle{\pgfqpoint{1.250000in}{0.550000in}}{\pgfqpoint{7.750000in}{3.850000in}}%
\pgfusepath{clip}%
\pgfsetbuttcap%
\pgfsetmiterjoin%
\definecolor{currentfill}{rgb}{0.501961,0.501961,0.501961}%
\pgfsetfillcolor{currentfill}%
\pgfsetfillopacity{0.700000}%
\pgfsetlinewidth{0.000000pt}%
\definecolor{currentstroke}{rgb}{0.000000,0.000000,0.000000}%
\pgfsetstrokecolor{currentstroke}%
\pgfsetstrokeopacity{0.700000}%
\pgfsetdash{}{0pt}%
\pgfpathmoveto{\pgfqpoint{1.849964in}{0.550000in}}%
\pgfpathlineto{\pgfqpoint{1.877486in}{0.550000in}}%
\pgfpathlineto{\pgfqpoint{1.877486in}{0.550000in}}%
\pgfpathlineto{\pgfqpoint{1.849964in}{0.550000in}}%
\pgfpathlineto{\pgfqpoint{1.849964in}{0.550000in}}%
\pgfpathclose%
\pgfusepath{fill}%
\end{pgfscope}%
\begin{pgfscope}%
\pgfpathrectangle{\pgfqpoint{1.250000in}{0.550000in}}{\pgfqpoint{7.750000in}{3.850000in}}%
\pgfusepath{clip}%
\pgfsetbuttcap%
\pgfsetmiterjoin%
\definecolor{currentfill}{rgb}{0.501961,0.501961,0.501961}%
\pgfsetfillcolor{currentfill}%
\pgfsetfillopacity{0.700000}%
\pgfsetlinewidth{0.000000pt}%
\definecolor{currentstroke}{rgb}{0.000000,0.000000,0.000000}%
\pgfsetstrokecolor{currentstroke}%
\pgfsetstrokeopacity{0.700000}%
\pgfsetdash{}{0pt}%
\pgfpathmoveto{\pgfqpoint{1.877486in}{0.550000in}}%
\pgfpathlineto{\pgfqpoint{1.905007in}{0.550000in}}%
\pgfpathlineto{\pgfqpoint{1.905007in}{0.550000in}}%
\pgfpathlineto{\pgfqpoint{1.877486in}{0.550000in}}%
\pgfpathlineto{\pgfqpoint{1.877486in}{0.550000in}}%
\pgfpathclose%
\pgfusepath{fill}%
\end{pgfscope}%
\begin{pgfscope}%
\pgfpathrectangle{\pgfqpoint{1.250000in}{0.550000in}}{\pgfqpoint{7.750000in}{3.850000in}}%
\pgfusepath{clip}%
\pgfsetbuttcap%
\pgfsetmiterjoin%
\definecolor{currentfill}{rgb}{0.501961,0.501961,0.501961}%
\pgfsetfillcolor{currentfill}%
\pgfsetfillopacity{0.700000}%
\pgfsetlinewidth{0.000000pt}%
\definecolor{currentstroke}{rgb}{0.000000,0.000000,0.000000}%
\pgfsetstrokecolor{currentstroke}%
\pgfsetstrokeopacity{0.700000}%
\pgfsetdash{}{0pt}%
\pgfpathmoveto{\pgfqpoint{1.905007in}{0.550000in}}%
\pgfpathlineto{\pgfqpoint{1.932528in}{0.550000in}}%
\pgfpathlineto{\pgfqpoint{1.932528in}{0.550000in}}%
\pgfpathlineto{\pgfqpoint{1.905007in}{0.550000in}}%
\pgfpathlineto{\pgfqpoint{1.905007in}{0.550000in}}%
\pgfpathclose%
\pgfusepath{fill}%
\end{pgfscope}%
\begin{pgfscope}%
\pgfpathrectangle{\pgfqpoint{1.250000in}{0.550000in}}{\pgfqpoint{7.750000in}{3.850000in}}%
\pgfusepath{clip}%
\pgfsetbuttcap%
\pgfsetmiterjoin%
\definecolor{currentfill}{rgb}{0.501961,0.501961,0.501961}%
\pgfsetfillcolor{currentfill}%
\pgfsetfillopacity{0.700000}%
\pgfsetlinewidth{0.000000pt}%
\definecolor{currentstroke}{rgb}{0.000000,0.000000,0.000000}%
\pgfsetstrokecolor{currentstroke}%
\pgfsetstrokeopacity{0.700000}%
\pgfsetdash{}{0pt}%
\pgfpathmoveto{\pgfqpoint{1.932528in}{0.550000in}}%
\pgfpathlineto{\pgfqpoint{1.960050in}{0.550000in}}%
\pgfpathlineto{\pgfqpoint{1.960050in}{0.550000in}}%
\pgfpathlineto{\pgfqpoint{1.932528in}{0.550000in}}%
\pgfpathlineto{\pgfqpoint{1.932528in}{0.550000in}}%
\pgfpathclose%
\pgfusepath{fill}%
\end{pgfscope}%
\begin{pgfscope}%
\pgfpathrectangle{\pgfqpoint{1.250000in}{0.550000in}}{\pgfqpoint{7.750000in}{3.850000in}}%
\pgfusepath{clip}%
\pgfsetbuttcap%
\pgfsetmiterjoin%
\definecolor{currentfill}{rgb}{0.501961,0.501961,0.501961}%
\pgfsetfillcolor{currentfill}%
\pgfsetfillopacity{0.700000}%
\pgfsetlinewidth{0.000000pt}%
\definecolor{currentstroke}{rgb}{0.000000,0.000000,0.000000}%
\pgfsetstrokecolor{currentstroke}%
\pgfsetstrokeopacity{0.700000}%
\pgfsetdash{}{0pt}%
\pgfpathmoveto{\pgfqpoint{1.960050in}{0.550000in}}%
\pgfpathlineto{\pgfqpoint{1.987571in}{0.550000in}}%
\pgfpathlineto{\pgfqpoint{1.987571in}{0.550000in}}%
\pgfpathlineto{\pgfqpoint{1.960050in}{0.550000in}}%
\pgfpathlineto{\pgfqpoint{1.960050in}{0.550000in}}%
\pgfpathclose%
\pgfusepath{fill}%
\end{pgfscope}%
\begin{pgfscope}%
\pgfpathrectangle{\pgfqpoint{1.250000in}{0.550000in}}{\pgfqpoint{7.750000in}{3.850000in}}%
\pgfusepath{clip}%
\pgfsetbuttcap%
\pgfsetmiterjoin%
\definecolor{currentfill}{rgb}{0.501961,0.501961,0.501961}%
\pgfsetfillcolor{currentfill}%
\pgfsetfillopacity{0.700000}%
\pgfsetlinewidth{0.000000pt}%
\definecolor{currentstroke}{rgb}{0.000000,0.000000,0.000000}%
\pgfsetstrokecolor{currentstroke}%
\pgfsetstrokeopacity{0.700000}%
\pgfsetdash{}{0pt}%
\pgfpathmoveto{\pgfqpoint{1.987571in}{0.550000in}}%
\pgfpathlineto{\pgfqpoint{2.015092in}{0.550000in}}%
\pgfpathlineto{\pgfqpoint{2.015092in}{0.550000in}}%
\pgfpathlineto{\pgfqpoint{1.987571in}{0.550000in}}%
\pgfpathlineto{\pgfqpoint{1.987571in}{0.550000in}}%
\pgfpathclose%
\pgfusepath{fill}%
\end{pgfscope}%
\begin{pgfscope}%
\pgfpathrectangle{\pgfqpoint{1.250000in}{0.550000in}}{\pgfqpoint{7.750000in}{3.850000in}}%
\pgfusepath{clip}%
\pgfsetbuttcap%
\pgfsetmiterjoin%
\definecolor{currentfill}{rgb}{0.501961,0.501961,0.501961}%
\pgfsetfillcolor{currentfill}%
\pgfsetfillopacity{0.700000}%
\pgfsetlinewidth{0.000000pt}%
\definecolor{currentstroke}{rgb}{0.000000,0.000000,0.000000}%
\pgfsetstrokecolor{currentstroke}%
\pgfsetstrokeopacity{0.700000}%
\pgfsetdash{}{0pt}%
\pgfpathmoveto{\pgfqpoint{2.015092in}{0.550000in}}%
\pgfpathlineto{\pgfqpoint{2.042614in}{0.550000in}}%
\pgfpathlineto{\pgfqpoint{2.042614in}{0.550000in}}%
\pgfpathlineto{\pgfqpoint{2.015092in}{0.550000in}}%
\pgfpathlineto{\pgfqpoint{2.015092in}{0.550000in}}%
\pgfpathclose%
\pgfusepath{fill}%
\end{pgfscope}%
\begin{pgfscope}%
\pgfpathrectangle{\pgfqpoint{1.250000in}{0.550000in}}{\pgfqpoint{7.750000in}{3.850000in}}%
\pgfusepath{clip}%
\pgfsetbuttcap%
\pgfsetmiterjoin%
\definecolor{currentfill}{rgb}{0.501961,0.501961,0.501961}%
\pgfsetfillcolor{currentfill}%
\pgfsetfillopacity{0.700000}%
\pgfsetlinewidth{0.000000pt}%
\definecolor{currentstroke}{rgb}{0.000000,0.000000,0.000000}%
\pgfsetstrokecolor{currentstroke}%
\pgfsetstrokeopacity{0.700000}%
\pgfsetdash{}{0pt}%
\pgfpathmoveto{\pgfqpoint{2.042614in}{0.550000in}}%
\pgfpathlineto{\pgfqpoint{2.070135in}{0.550000in}}%
\pgfpathlineto{\pgfqpoint{2.070135in}{0.550000in}}%
\pgfpathlineto{\pgfqpoint{2.042614in}{0.550000in}}%
\pgfpathlineto{\pgfqpoint{2.042614in}{0.550000in}}%
\pgfpathclose%
\pgfusepath{fill}%
\end{pgfscope}%
\begin{pgfscope}%
\pgfpathrectangle{\pgfqpoint{1.250000in}{0.550000in}}{\pgfqpoint{7.750000in}{3.850000in}}%
\pgfusepath{clip}%
\pgfsetbuttcap%
\pgfsetmiterjoin%
\definecolor{currentfill}{rgb}{0.501961,0.501961,0.501961}%
\pgfsetfillcolor{currentfill}%
\pgfsetfillopacity{0.700000}%
\pgfsetlinewidth{0.000000pt}%
\definecolor{currentstroke}{rgb}{0.000000,0.000000,0.000000}%
\pgfsetstrokecolor{currentstroke}%
\pgfsetstrokeopacity{0.700000}%
\pgfsetdash{}{0pt}%
\pgfpathmoveto{\pgfqpoint{2.070135in}{0.550000in}}%
\pgfpathlineto{\pgfqpoint{2.097656in}{0.550000in}}%
\pgfpathlineto{\pgfqpoint{2.097656in}{0.550000in}}%
\pgfpathlineto{\pgfqpoint{2.070135in}{0.550000in}}%
\pgfpathlineto{\pgfqpoint{2.070135in}{0.550000in}}%
\pgfpathclose%
\pgfusepath{fill}%
\end{pgfscope}%
\begin{pgfscope}%
\pgfpathrectangle{\pgfqpoint{1.250000in}{0.550000in}}{\pgfqpoint{7.750000in}{3.850000in}}%
\pgfusepath{clip}%
\pgfsetbuttcap%
\pgfsetmiterjoin%
\definecolor{currentfill}{rgb}{0.501961,0.501961,0.501961}%
\pgfsetfillcolor{currentfill}%
\pgfsetfillopacity{0.700000}%
\pgfsetlinewidth{0.000000pt}%
\definecolor{currentstroke}{rgb}{0.000000,0.000000,0.000000}%
\pgfsetstrokecolor{currentstroke}%
\pgfsetstrokeopacity{0.700000}%
\pgfsetdash{}{0pt}%
\pgfpathmoveto{\pgfqpoint{2.097656in}{0.550000in}}%
\pgfpathlineto{\pgfqpoint{2.125178in}{0.550000in}}%
\pgfpathlineto{\pgfqpoint{2.125178in}{0.550000in}}%
\pgfpathlineto{\pgfqpoint{2.097656in}{0.550000in}}%
\pgfpathlineto{\pgfqpoint{2.097656in}{0.550000in}}%
\pgfpathclose%
\pgfusepath{fill}%
\end{pgfscope}%
\begin{pgfscope}%
\pgfpathrectangle{\pgfqpoint{1.250000in}{0.550000in}}{\pgfqpoint{7.750000in}{3.850000in}}%
\pgfusepath{clip}%
\pgfsetbuttcap%
\pgfsetmiterjoin%
\definecolor{currentfill}{rgb}{0.501961,0.501961,0.501961}%
\pgfsetfillcolor{currentfill}%
\pgfsetfillopacity{0.700000}%
\pgfsetlinewidth{0.000000pt}%
\definecolor{currentstroke}{rgb}{0.000000,0.000000,0.000000}%
\pgfsetstrokecolor{currentstroke}%
\pgfsetstrokeopacity{0.700000}%
\pgfsetdash{}{0pt}%
\pgfpathmoveto{\pgfqpoint{2.125178in}{0.550000in}}%
\pgfpathlineto{\pgfqpoint{2.152699in}{0.550000in}}%
\pgfpathlineto{\pgfqpoint{2.152699in}{0.550000in}}%
\pgfpathlineto{\pgfqpoint{2.125178in}{0.550000in}}%
\pgfpathlineto{\pgfqpoint{2.125178in}{0.550000in}}%
\pgfpathclose%
\pgfusepath{fill}%
\end{pgfscope}%
\begin{pgfscope}%
\pgfpathrectangle{\pgfqpoint{1.250000in}{0.550000in}}{\pgfqpoint{7.750000in}{3.850000in}}%
\pgfusepath{clip}%
\pgfsetbuttcap%
\pgfsetmiterjoin%
\definecolor{currentfill}{rgb}{0.501961,0.501961,0.501961}%
\pgfsetfillcolor{currentfill}%
\pgfsetfillopacity{0.700000}%
\pgfsetlinewidth{0.000000pt}%
\definecolor{currentstroke}{rgb}{0.000000,0.000000,0.000000}%
\pgfsetstrokecolor{currentstroke}%
\pgfsetstrokeopacity{0.700000}%
\pgfsetdash{}{0pt}%
\pgfpathmoveto{\pgfqpoint{2.152699in}{0.550000in}}%
\pgfpathlineto{\pgfqpoint{2.180220in}{0.550000in}}%
\pgfpathlineto{\pgfqpoint{2.180220in}{3.683059in}}%
\pgfpathlineto{\pgfqpoint{2.152699in}{3.683059in}}%
\pgfpathlineto{\pgfqpoint{2.152699in}{0.550000in}}%
\pgfpathclose%
\pgfusepath{fill}%
\end{pgfscope}%
\begin{pgfscope}%
\pgfpathrectangle{\pgfqpoint{1.250000in}{0.550000in}}{\pgfqpoint{7.750000in}{3.850000in}}%
\pgfusepath{clip}%
\pgfsetbuttcap%
\pgfsetmiterjoin%
\definecolor{currentfill}{rgb}{0.501961,0.501961,0.501961}%
\pgfsetfillcolor{currentfill}%
\pgfsetfillopacity{0.700000}%
\pgfsetlinewidth{0.000000pt}%
\definecolor{currentstroke}{rgb}{0.000000,0.000000,0.000000}%
\pgfsetstrokecolor{currentstroke}%
\pgfsetstrokeopacity{0.700000}%
\pgfsetdash{}{0pt}%
\pgfpathmoveto{\pgfqpoint{2.180220in}{0.550000in}}%
\pgfpathlineto{\pgfqpoint{2.207741in}{0.550000in}}%
\pgfpathlineto{\pgfqpoint{2.207741in}{0.550000in}}%
\pgfpathlineto{\pgfqpoint{2.180220in}{0.550000in}}%
\pgfpathlineto{\pgfqpoint{2.180220in}{0.550000in}}%
\pgfpathclose%
\pgfusepath{fill}%
\end{pgfscope}%
\begin{pgfscope}%
\pgfpathrectangle{\pgfqpoint{1.250000in}{0.550000in}}{\pgfqpoint{7.750000in}{3.850000in}}%
\pgfusepath{clip}%
\pgfsetbuttcap%
\pgfsetmiterjoin%
\definecolor{currentfill}{rgb}{0.501961,0.501961,0.501961}%
\pgfsetfillcolor{currentfill}%
\pgfsetfillopacity{0.700000}%
\pgfsetlinewidth{0.000000pt}%
\definecolor{currentstroke}{rgb}{0.000000,0.000000,0.000000}%
\pgfsetstrokecolor{currentstroke}%
\pgfsetstrokeopacity{0.700000}%
\pgfsetdash{}{0pt}%
\pgfpathmoveto{\pgfqpoint{2.207741in}{0.550000in}}%
\pgfpathlineto{\pgfqpoint{2.235263in}{0.550000in}}%
\pgfpathlineto{\pgfqpoint{2.235263in}{0.550000in}}%
\pgfpathlineto{\pgfqpoint{2.207741in}{0.550000in}}%
\pgfpathlineto{\pgfqpoint{2.207741in}{0.550000in}}%
\pgfpathclose%
\pgfusepath{fill}%
\end{pgfscope}%
\begin{pgfscope}%
\pgfpathrectangle{\pgfqpoint{1.250000in}{0.550000in}}{\pgfqpoint{7.750000in}{3.850000in}}%
\pgfusepath{clip}%
\pgfsetbuttcap%
\pgfsetmiterjoin%
\definecolor{currentfill}{rgb}{0.501961,0.501961,0.501961}%
\pgfsetfillcolor{currentfill}%
\pgfsetfillopacity{0.700000}%
\pgfsetlinewidth{0.000000pt}%
\definecolor{currentstroke}{rgb}{0.000000,0.000000,0.000000}%
\pgfsetstrokecolor{currentstroke}%
\pgfsetstrokeopacity{0.700000}%
\pgfsetdash{}{0pt}%
\pgfpathmoveto{\pgfqpoint{2.235263in}{0.550000in}}%
\pgfpathlineto{\pgfqpoint{2.262784in}{0.550000in}}%
\pgfpathlineto{\pgfqpoint{2.262784in}{0.550000in}}%
\pgfpathlineto{\pgfqpoint{2.235263in}{0.550000in}}%
\pgfpathlineto{\pgfqpoint{2.235263in}{0.550000in}}%
\pgfpathclose%
\pgfusepath{fill}%
\end{pgfscope}%
\begin{pgfscope}%
\pgfpathrectangle{\pgfqpoint{1.250000in}{0.550000in}}{\pgfqpoint{7.750000in}{3.850000in}}%
\pgfusepath{clip}%
\pgfsetbuttcap%
\pgfsetmiterjoin%
\definecolor{currentfill}{rgb}{0.501961,0.501961,0.501961}%
\pgfsetfillcolor{currentfill}%
\pgfsetfillopacity{0.700000}%
\pgfsetlinewidth{0.000000pt}%
\definecolor{currentstroke}{rgb}{0.000000,0.000000,0.000000}%
\pgfsetstrokecolor{currentstroke}%
\pgfsetstrokeopacity{0.700000}%
\pgfsetdash{}{0pt}%
\pgfpathmoveto{\pgfqpoint{2.262784in}{0.550000in}}%
\pgfpathlineto{\pgfqpoint{2.290305in}{0.550000in}}%
\pgfpathlineto{\pgfqpoint{2.290305in}{0.550000in}}%
\pgfpathlineto{\pgfqpoint{2.262784in}{0.550000in}}%
\pgfpathlineto{\pgfqpoint{2.262784in}{0.550000in}}%
\pgfpathclose%
\pgfusepath{fill}%
\end{pgfscope}%
\begin{pgfscope}%
\pgfpathrectangle{\pgfqpoint{1.250000in}{0.550000in}}{\pgfqpoint{7.750000in}{3.850000in}}%
\pgfusepath{clip}%
\pgfsetbuttcap%
\pgfsetmiterjoin%
\definecolor{currentfill}{rgb}{0.501961,0.501961,0.501961}%
\pgfsetfillcolor{currentfill}%
\pgfsetfillopacity{0.700000}%
\pgfsetlinewidth{0.000000pt}%
\definecolor{currentstroke}{rgb}{0.000000,0.000000,0.000000}%
\pgfsetstrokecolor{currentstroke}%
\pgfsetstrokeopacity{0.700000}%
\pgfsetdash{}{0pt}%
\pgfpathmoveto{\pgfqpoint{2.290305in}{0.550000in}}%
\pgfpathlineto{\pgfqpoint{2.317827in}{0.550000in}}%
\pgfpathlineto{\pgfqpoint{2.317827in}{0.550000in}}%
\pgfpathlineto{\pgfqpoint{2.290305in}{0.550000in}}%
\pgfpathlineto{\pgfqpoint{2.290305in}{0.550000in}}%
\pgfpathclose%
\pgfusepath{fill}%
\end{pgfscope}%
\begin{pgfscope}%
\pgfpathrectangle{\pgfqpoint{1.250000in}{0.550000in}}{\pgfqpoint{7.750000in}{3.850000in}}%
\pgfusepath{clip}%
\pgfsetbuttcap%
\pgfsetmiterjoin%
\definecolor{currentfill}{rgb}{0.501961,0.501961,0.501961}%
\pgfsetfillcolor{currentfill}%
\pgfsetfillopacity{0.700000}%
\pgfsetlinewidth{0.000000pt}%
\definecolor{currentstroke}{rgb}{0.000000,0.000000,0.000000}%
\pgfsetstrokecolor{currentstroke}%
\pgfsetstrokeopacity{0.700000}%
\pgfsetdash{}{0pt}%
\pgfpathmoveto{\pgfqpoint{2.317827in}{0.550000in}}%
\pgfpathlineto{\pgfqpoint{2.345348in}{0.550000in}}%
\pgfpathlineto{\pgfqpoint{2.345348in}{0.550000in}}%
\pgfpathlineto{\pgfqpoint{2.317827in}{0.550000in}}%
\pgfpathlineto{\pgfqpoint{2.317827in}{0.550000in}}%
\pgfpathclose%
\pgfusepath{fill}%
\end{pgfscope}%
\begin{pgfscope}%
\pgfpathrectangle{\pgfqpoint{1.250000in}{0.550000in}}{\pgfqpoint{7.750000in}{3.850000in}}%
\pgfusepath{clip}%
\pgfsetbuttcap%
\pgfsetmiterjoin%
\definecolor{currentfill}{rgb}{0.501961,0.501961,0.501961}%
\pgfsetfillcolor{currentfill}%
\pgfsetfillopacity{0.700000}%
\pgfsetlinewidth{0.000000pt}%
\definecolor{currentstroke}{rgb}{0.000000,0.000000,0.000000}%
\pgfsetstrokecolor{currentstroke}%
\pgfsetstrokeopacity{0.700000}%
\pgfsetdash{}{0pt}%
\pgfpathmoveto{\pgfqpoint{2.345348in}{0.550000in}}%
\pgfpathlineto{\pgfqpoint{2.372869in}{0.550000in}}%
\pgfpathlineto{\pgfqpoint{2.372869in}{0.550000in}}%
\pgfpathlineto{\pgfqpoint{2.345348in}{0.550000in}}%
\pgfpathlineto{\pgfqpoint{2.345348in}{0.550000in}}%
\pgfpathclose%
\pgfusepath{fill}%
\end{pgfscope}%
\begin{pgfscope}%
\pgfpathrectangle{\pgfqpoint{1.250000in}{0.550000in}}{\pgfqpoint{7.750000in}{3.850000in}}%
\pgfusepath{clip}%
\pgfsetbuttcap%
\pgfsetmiterjoin%
\definecolor{currentfill}{rgb}{0.501961,0.501961,0.501961}%
\pgfsetfillcolor{currentfill}%
\pgfsetfillopacity{0.700000}%
\pgfsetlinewidth{0.000000pt}%
\definecolor{currentstroke}{rgb}{0.000000,0.000000,0.000000}%
\pgfsetstrokecolor{currentstroke}%
\pgfsetstrokeopacity{0.700000}%
\pgfsetdash{}{0pt}%
\pgfpathmoveto{\pgfqpoint{2.372869in}{0.550000in}}%
\pgfpathlineto{\pgfqpoint{2.400391in}{0.550000in}}%
\pgfpathlineto{\pgfqpoint{2.400391in}{0.550000in}}%
\pgfpathlineto{\pgfqpoint{2.372869in}{0.550000in}}%
\pgfpathlineto{\pgfqpoint{2.372869in}{0.550000in}}%
\pgfpathclose%
\pgfusepath{fill}%
\end{pgfscope}%
\begin{pgfscope}%
\pgfpathrectangle{\pgfqpoint{1.250000in}{0.550000in}}{\pgfqpoint{7.750000in}{3.850000in}}%
\pgfusepath{clip}%
\pgfsetbuttcap%
\pgfsetmiterjoin%
\definecolor{currentfill}{rgb}{0.501961,0.501961,0.501961}%
\pgfsetfillcolor{currentfill}%
\pgfsetfillopacity{0.700000}%
\pgfsetlinewidth{0.000000pt}%
\definecolor{currentstroke}{rgb}{0.000000,0.000000,0.000000}%
\pgfsetstrokecolor{currentstroke}%
\pgfsetstrokeopacity{0.700000}%
\pgfsetdash{}{0pt}%
\pgfpathmoveto{\pgfqpoint{2.400391in}{0.550000in}}%
\pgfpathlineto{\pgfqpoint{2.427912in}{0.550000in}}%
\pgfpathlineto{\pgfqpoint{2.427912in}{0.550000in}}%
\pgfpathlineto{\pgfqpoint{2.400391in}{0.550000in}}%
\pgfpathlineto{\pgfqpoint{2.400391in}{0.550000in}}%
\pgfpathclose%
\pgfusepath{fill}%
\end{pgfscope}%
\begin{pgfscope}%
\pgfpathrectangle{\pgfqpoint{1.250000in}{0.550000in}}{\pgfqpoint{7.750000in}{3.850000in}}%
\pgfusepath{clip}%
\pgfsetbuttcap%
\pgfsetmiterjoin%
\definecolor{currentfill}{rgb}{0.501961,0.501961,0.501961}%
\pgfsetfillcolor{currentfill}%
\pgfsetfillopacity{0.700000}%
\pgfsetlinewidth{0.000000pt}%
\definecolor{currentstroke}{rgb}{0.000000,0.000000,0.000000}%
\pgfsetstrokecolor{currentstroke}%
\pgfsetstrokeopacity{0.700000}%
\pgfsetdash{}{0pt}%
\pgfpathmoveto{\pgfqpoint{2.427912in}{0.550000in}}%
\pgfpathlineto{\pgfqpoint{2.455433in}{0.550000in}}%
\pgfpathlineto{\pgfqpoint{2.455433in}{0.550000in}}%
\pgfpathlineto{\pgfqpoint{2.427912in}{0.550000in}}%
\pgfpathlineto{\pgfqpoint{2.427912in}{0.550000in}}%
\pgfpathclose%
\pgfusepath{fill}%
\end{pgfscope}%
\begin{pgfscope}%
\pgfpathrectangle{\pgfqpoint{1.250000in}{0.550000in}}{\pgfqpoint{7.750000in}{3.850000in}}%
\pgfusepath{clip}%
\pgfsetbuttcap%
\pgfsetmiterjoin%
\definecolor{currentfill}{rgb}{0.501961,0.501961,0.501961}%
\pgfsetfillcolor{currentfill}%
\pgfsetfillopacity{0.700000}%
\pgfsetlinewidth{0.000000pt}%
\definecolor{currentstroke}{rgb}{0.000000,0.000000,0.000000}%
\pgfsetstrokecolor{currentstroke}%
\pgfsetstrokeopacity{0.700000}%
\pgfsetdash{}{0pt}%
\pgfpathmoveto{\pgfqpoint{2.455433in}{0.550000in}}%
\pgfpathlineto{\pgfqpoint{2.482955in}{0.550000in}}%
\pgfpathlineto{\pgfqpoint{2.482955in}{2.356000in}}%
\pgfpathlineto{\pgfqpoint{2.455433in}{2.356000in}}%
\pgfpathlineto{\pgfqpoint{2.455433in}{0.550000in}}%
\pgfpathclose%
\pgfusepath{fill}%
\end{pgfscope}%
\begin{pgfscope}%
\pgfpathrectangle{\pgfqpoint{1.250000in}{0.550000in}}{\pgfqpoint{7.750000in}{3.850000in}}%
\pgfusepath{clip}%
\pgfsetbuttcap%
\pgfsetmiterjoin%
\definecolor{currentfill}{rgb}{0.501961,0.501961,0.501961}%
\pgfsetfillcolor{currentfill}%
\pgfsetfillopacity{0.700000}%
\pgfsetlinewidth{0.000000pt}%
\definecolor{currentstroke}{rgb}{0.000000,0.000000,0.000000}%
\pgfsetstrokecolor{currentstroke}%
\pgfsetstrokeopacity{0.700000}%
\pgfsetdash{}{0pt}%
\pgfpathmoveto{\pgfqpoint{2.482955in}{0.550000in}}%
\pgfpathlineto{\pgfqpoint{2.510476in}{0.550000in}}%
\pgfpathlineto{\pgfqpoint{2.510476in}{0.550000in}}%
\pgfpathlineto{\pgfqpoint{2.482955in}{0.550000in}}%
\pgfpathlineto{\pgfqpoint{2.482955in}{0.550000in}}%
\pgfpathclose%
\pgfusepath{fill}%
\end{pgfscope}%
\begin{pgfscope}%
\pgfpathrectangle{\pgfqpoint{1.250000in}{0.550000in}}{\pgfqpoint{7.750000in}{3.850000in}}%
\pgfusepath{clip}%
\pgfsetbuttcap%
\pgfsetmiterjoin%
\definecolor{currentfill}{rgb}{0.501961,0.501961,0.501961}%
\pgfsetfillcolor{currentfill}%
\pgfsetfillopacity{0.700000}%
\pgfsetlinewidth{0.000000pt}%
\definecolor{currentstroke}{rgb}{0.000000,0.000000,0.000000}%
\pgfsetstrokecolor{currentstroke}%
\pgfsetstrokeopacity{0.700000}%
\pgfsetdash{}{0pt}%
\pgfpathmoveto{\pgfqpoint{2.510476in}{0.550000in}}%
\pgfpathlineto{\pgfqpoint{2.537997in}{0.550000in}}%
\pgfpathlineto{\pgfqpoint{2.537997in}{0.550000in}}%
\pgfpathlineto{\pgfqpoint{2.510476in}{0.550000in}}%
\pgfpathlineto{\pgfqpoint{2.510476in}{0.550000in}}%
\pgfpathclose%
\pgfusepath{fill}%
\end{pgfscope}%
\begin{pgfscope}%
\pgfpathrectangle{\pgfqpoint{1.250000in}{0.550000in}}{\pgfqpoint{7.750000in}{3.850000in}}%
\pgfusepath{clip}%
\pgfsetbuttcap%
\pgfsetmiterjoin%
\definecolor{currentfill}{rgb}{0.501961,0.501961,0.501961}%
\pgfsetfillcolor{currentfill}%
\pgfsetfillopacity{0.700000}%
\pgfsetlinewidth{0.000000pt}%
\definecolor{currentstroke}{rgb}{0.000000,0.000000,0.000000}%
\pgfsetstrokecolor{currentstroke}%
\pgfsetstrokeopacity{0.700000}%
\pgfsetdash{}{0pt}%
\pgfpathmoveto{\pgfqpoint{2.537997in}{0.550000in}}%
\pgfpathlineto{\pgfqpoint{2.565518in}{0.550000in}}%
\pgfpathlineto{\pgfqpoint{2.565518in}{0.550000in}}%
\pgfpathlineto{\pgfqpoint{2.537997in}{0.550000in}}%
\pgfpathlineto{\pgfqpoint{2.537997in}{0.550000in}}%
\pgfpathclose%
\pgfusepath{fill}%
\end{pgfscope}%
\begin{pgfscope}%
\pgfpathrectangle{\pgfqpoint{1.250000in}{0.550000in}}{\pgfqpoint{7.750000in}{3.850000in}}%
\pgfusepath{clip}%
\pgfsetbuttcap%
\pgfsetmiterjoin%
\definecolor{currentfill}{rgb}{0.501961,0.501961,0.501961}%
\pgfsetfillcolor{currentfill}%
\pgfsetfillopacity{0.700000}%
\pgfsetlinewidth{0.000000pt}%
\definecolor{currentstroke}{rgb}{0.000000,0.000000,0.000000}%
\pgfsetstrokecolor{currentstroke}%
\pgfsetstrokeopacity{0.700000}%
\pgfsetdash{}{0pt}%
\pgfpathmoveto{\pgfqpoint{2.565518in}{0.550000in}}%
\pgfpathlineto{\pgfqpoint{2.593040in}{0.550000in}}%
\pgfpathlineto{\pgfqpoint{2.593040in}{0.550000in}}%
\pgfpathlineto{\pgfqpoint{2.565518in}{0.550000in}}%
\pgfpathlineto{\pgfqpoint{2.565518in}{0.550000in}}%
\pgfpathclose%
\pgfusepath{fill}%
\end{pgfscope}%
\begin{pgfscope}%
\pgfpathrectangle{\pgfqpoint{1.250000in}{0.550000in}}{\pgfqpoint{7.750000in}{3.850000in}}%
\pgfusepath{clip}%
\pgfsetbuttcap%
\pgfsetmiterjoin%
\definecolor{currentfill}{rgb}{0.501961,0.501961,0.501961}%
\pgfsetfillcolor{currentfill}%
\pgfsetfillopacity{0.700000}%
\pgfsetlinewidth{0.000000pt}%
\definecolor{currentstroke}{rgb}{0.000000,0.000000,0.000000}%
\pgfsetstrokecolor{currentstroke}%
\pgfsetstrokeopacity{0.700000}%
\pgfsetdash{}{0pt}%
\pgfpathmoveto{\pgfqpoint{2.593040in}{0.550000in}}%
\pgfpathlineto{\pgfqpoint{2.620561in}{0.550000in}}%
\pgfpathlineto{\pgfqpoint{2.620561in}{0.550000in}}%
\pgfpathlineto{\pgfqpoint{2.593040in}{0.550000in}}%
\pgfpathlineto{\pgfqpoint{2.593040in}{0.550000in}}%
\pgfpathclose%
\pgfusepath{fill}%
\end{pgfscope}%
\begin{pgfscope}%
\pgfpathrectangle{\pgfqpoint{1.250000in}{0.550000in}}{\pgfqpoint{7.750000in}{3.850000in}}%
\pgfusepath{clip}%
\pgfsetbuttcap%
\pgfsetmiterjoin%
\definecolor{currentfill}{rgb}{0.501961,0.501961,0.501961}%
\pgfsetfillcolor{currentfill}%
\pgfsetfillopacity{0.700000}%
\pgfsetlinewidth{0.000000pt}%
\definecolor{currentstroke}{rgb}{0.000000,0.000000,0.000000}%
\pgfsetstrokecolor{currentstroke}%
\pgfsetstrokeopacity{0.700000}%
\pgfsetdash{}{0pt}%
\pgfpathmoveto{\pgfqpoint{2.620561in}{0.550000in}}%
\pgfpathlineto{\pgfqpoint{2.648082in}{0.550000in}}%
\pgfpathlineto{\pgfqpoint{2.648082in}{0.550000in}}%
\pgfpathlineto{\pgfqpoint{2.620561in}{0.550000in}}%
\pgfpathlineto{\pgfqpoint{2.620561in}{0.550000in}}%
\pgfpathclose%
\pgfusepath{fill}%
\end{pgfscope}%
\begin{pgfscope}%
\pgfpathrectangle{\pgfqpoint{1.250000in}{0.550000in}}{\pgfqpoint{7.750000in}{3.850000in}}%
\pgfusepath{clip}%
\pgfsetbuttcap%
\pgfsetmiterjoin%
\definecolor{currentfill}{rgb}{0.501961,0.501961,0.501961}%
\pgfsetfillcolor{currentfill}%
\pgfsetfillopacity{0.700000}%
\pgfsetlinewidth{0.000000pt}%
\definecolor{currentstroke}{rgb}{0.000000,0.000000,0.000000}%
\pgfsetstrokecolor{currentstroke}%
\pgfsetstrokeopacity{0.700000}%
\pgfsetdash{}{0pt}%
\pgfpathmoveto{\pgfqpoint{2.648082in}{0.550000in}}%
\pgfpathlineto{\pgfqpoint{2.675604in}{0.550000in}}%
\pgfpathlineto{\pgfqpoint{2.675604in}{0.550000in}}%
\pgfpathlineto{\pgfqpoint{2.648082in}{0.550000in}}%
\pgfpathlineto{\pgfqpoint{2.648082in}{0.550000in}}%
\pgfpathclose%
\pgfusepath{fill}%
\end{pgfscope}%
\begin{pgfscope}%
\pgfpathrectangle{\pgfqpoint{1.250000in}{0.550000in}}{\pgfqpoint{7.750000in}{3.850000in}}%
\pgfusepath{clip}%
\pgfsetbuttcap%
\pgfsetmiterjoin%
\definecolor{currentfill}{rgb}{0.501961,0.501961,0.501961}%
\pgfsetfillcolor{currentfill}%
\pgfsetfillopacity{0.700000}%
\pgfsetlinewidth{0.000000pt}%
\definecolor{currentstroke}{rgb}{0.000000,0.000000,0.000000}%
\pgfsetstrokecolor{currentstroke}%
\pgfsetstrokeopacity{0.700000}%
\pgfsetdash{}{0pt}%
\pgfpathmoveto{\pgfqpoint{2.675604in}{0.550000in}}%
\pgfpathlineto{\pgfqpoint{2.703125in}{0.550000in}}%
\pgfpathlineto{\pgfqpoint{2.703125in}{0.550000in}}%
\pgfpathlineto{\pgfqpoint{2.675604in}{0.550000in}}%
\pgfpathlineto{\pgfqpoint{2.675604in}{0.550000in}}%
\pgfpathclose%
\pgfusepath{fill}%
\end{pgfscope}%
\begin{pgfscope}%
\pgfpathrectangle{\pgfqpoint{1.250000in}{0.550000in}}{\pgfqpoint{7.750000in}{3.850000in}}%
\pgfusepath{clip}%
\pgfsetbuttcap%
\pgfsetmiterjoin%
\definecolor{currentfill}{rgb}{0.501961,0.501961,0.501961}%
\pgfsetfillcolor{currentfill}%
\pgfsetfillopacity{0.700000}%
\pgfsetlinewidth{0.000000pt}%
\definecolor{currentstroke}{rgb}{0.000000,0.000000,0.000000}%
\pgfsetstrokecolor{currentstroke}%
\pgfsetstrokeopacity{0.700000}%
\pgfsetdash{}{0pt}%
\pgfpathmoveto{\pgfqpoint{2.703125in}{0.550000in}}%
\pgfpathlineto{\pgfqpoint{2.730646in}{0.550000in}}%
\pgfpathlineto{\pgfqpoint{2.730646in}{2.100507in}}%
\pgfpathlineto{\pgfqpoint{2.703125in}{2.100507in}}%
\pgfpathlineto{\pgfqpoint{2.703125in}{0.550000in}}%
\pgfpathclose%
\pgfusepath{fill}%
\end{pgfscope}%
\begin{pgfscope}%
\pgfpathrectangle{\pgfqpoint{1.250000in}{0.550000in}}{\pgfqpoint{7.750000in}{3.850000in}}%
\pgfusepath{clip}%
\pgfsetbuttcap%
\pgfsetmiterjoin%
\definecolor{currentfill}{rgb}{0.501961,0.501961,0.501961}%
\pgfsetfillcolor{currentfill}%
\pgfsetfillopacity{0.700000}%
\pgfsetlinewidth{0.000000pt}%
\definecolor{currentstroke}{rgb}{0.000000,0.000000,0.000000}%
\pgfsetstrokecolor{currentstroke}%
\pgfsetstrokeopacity{0.700000}%
\pgfsetdash{}{0pt}%
\pgfpathmoveto{\pgfqpoint{2.730646in}{0.550000in}}%
\pgfpathlineto{\pgfqpoint{2.758168in}{0.550000in}}%
\pgfpathlineto{\pgfqpoint{2.758168in}{0.550000in}}%
\pgfpathlineto{\pgfqpoint{2.730646in}{0.550000in}}%
\pgfpathlineto{\pgfqpoint{2.730646in}{0.550000in}}%
\pgfpathclose%
\pgfusepath{fill}%
\end{pgfscope}%
\begin{pgfscope}%
\pgfpathrectangle{\pgfqpoint{1.250000in}{0.550000in}}{\pgfqpoint{7.750000in}{3.850000in}}%
\pgfusepath{clip}%
\pgfsetbuttcap%
\pgfsetmiterjoin%
\definecolor{currentfill}{rgb}{0.501961,0.501961,0.501961}%
\pgfsetfillcolor{currentfill}%
\pgfsetfillopacity{0.700000}%
\pgfsetlinewidth{0.000000pt}%
\definecolor{currentstroke}{rgb}{0.000000,0.000000,0.000000}%
\pgfsetstrokecolor{currentstroke}%
\pgfsetstrokeopacity{0.700000}%
\pgfsetdash{}{0pt}%
\pgfpathmoveto{\pgfqpoint{2.758168in}{0.550000in}}%
\pgfpathlineto{\pgfqpoint{2.785689in}{0.550000in}}%
\pgfpathlineto{\pgfqpoint{2.785689in}{0.550000in}}%
\pgfpathlineto{\pgfqpoint{2.758168in}{0.550000in}}%
\pgfpathlineto{\pgfqpoint{2.758168in}{0.550000in}}%
\pgfpathclose%
\pgfusepath{fill}%
\end{pgfscope}%
\begin{pgfscope}%
\pgfpathrectangle{\pgfqpoint{1.250000in}{0.550000in}}{\pgfqpoint{7.750000in}{3.850000in}}%
\pgfusepath{clip}%
\pgfsetbuttcap%
\pgfsetmiterjoin%
\definecolor{currentfill}{rgb}{0.501961,0.501961,0.501961}%
\pgfsetfillcolor{currentfill}%
\pgfsetfillopacity{0.700000}%
\pgfsetlinewidth{0.000000pt}%
\definecolor{currentstroke}{rgb}{0.000000,0.000000,0.000000}%
\pgfsetstrokecolor{currentstroke}%
\pgfsetstrokeopacity{0.700000}%
\pgfsetdash{}{0pt}%
\pgfpathmoveto{\pgfqpoint{2.785689in}{0.550000in}}%
\pgfpathlineto{\pgfqpoint{2.813210in}{0.550000in}}%
\pgfpathlineto{\pgfqpoint{2.813210in}{0.550000in}}%
\pgfpathlineto{\pgfqpoint{2.785689in}{0.550000in}}%
\pgfpathlineto{\pgfqpoint{2.785689in}{0.550000in}}%
\pgfpathclose%
\pgfusepath{fill}%
\end{pgfscope}%
\begin{pgfscope}%
\pgfpathrectangle{\pgfqpoint{1.250000in}{0.550000in}}{\pgfqpoint{7.750000in}{3.850000in}}%
\pgfusepath{clip}%
\pgfsetbuttcap%
\pgfsetmiterjoin%
\definecolor{currentfill}{rgb}{0.501961,0.501961,0.501961}%
\pgfsetfillcolor{currentfill}%
\pgfsetfillopacity{0.700000}%
\pgfsetlinewidth{0.000000pt}%
\definecolor{currentstroke}{rgb}{0.000000,0.000000,0.000000}%
\pgfsetstrokecolor{currentstroke}%
\pgfsetstrokeopacity{0.700000}%
\pgfsetdash{}{0pt}%
\pgfpathmoveto{\pgfqpoint{2.813210in}{0.550000in}}%
\pgfpathlineto{\pgfqpoint{2.840732in}{0.550000in}}%
\pgfpathlineto{\pgfqpoint{2.840732in}{0.550000in}}%
\pgfpathlineto{\pgfqpoint{2.813210in}{0.550000in}}%
\pgfpathlineto{\pgfqpoint{2.813210in}{0.550000in}}%
\pgfpathclose%
\pgfusepath{fill}%
\end{pgfscope}%
\begin{pgfscope}%
\pgfpathrectangle{\pgfqpoint{1.250000in}{0.550000in}}{\pgfqpoint{7.750000in}{3.850000in}}%
\pgfusepath{clip}%
\pgfsetbuttcap%
\pgfsetmiterjoin%
\definecolor{currentfill}{rgb}{0.501961,0.501961,0.501961}%
\pgfsetfillcolor{currentfill}%
\pgfsetfillopacity{0.700000}%
\pgfsetlinewidth{0.000000pt}%
\definecolor{currentstroke}{rgb}{0.000000,0.000000,0.000000}%
\pgfsetstrokecolor{currentstroke}%
\pgfsetstrokeopacity{0.700000}%
\pgfsetdash{}{0pt}%
\pgfpathmoveto{\pgfqpoint{2.840732in}{0.550000in}}%
\pgfpathlineto{\pgfqpoint{2.868253in}{0.550000in}}%
\pgfpathlineto{\pgfqpoint{2.868253in}{0.550000in}}%
\pgfpathlineto{\pgfqpoint{2.840732in}{0.550000in}}%
\pgfpathlineto{\pgfqpoint{2.840732in}{0.550000in}}%
\pgfpathclose%
\pgfusepath{fill}%
\end{pgfscope}%
\begin{pgfscope}%
\pgfpathrectangle{\pgfqpoint{1.250000in}{0.550000in}}{\pgfqpoint{7.750000in}{3.850000in}}%
\pgfusepath{clip}%
\pgfsetbuttcap%
\pgfsetmiterjoin%
\definecolor{currentfill}{rgb}{0.501961,0.501961,0.501961}%
\pgfsetfillcolor{currentfill}%
\pgfsetfillopacity{0.700000}%
\pgfsetlinewidth{0.000000pt}%
\definecolor{currentstroke}{rgb}{0.000000,0.000000,0.000000}%
\pgfsetstrokecolor{currentstroke}%
\pgfsetstrokeopacity{0.700000}%
\pgfsetdash{}{0pt}%
\pgfpathmoveto{\pgfqpoint{2.868253in}{0.550000in}}%
\pgfpathlineto{\pgfqpoint{2.895774in}{0.550000in}}%
\pgfpathlineto{\pgfqpoint{2.895774in}{0.550000in}}%
\pgfpathlineto{\pgfqpoint{2.868253in}{0.550000in}}%
\pgfpathlineto{\pgfqpoint{2.868253in}{0.550000in}}%
\pgfpathclose%
\pgfusepath{fill}%
\end{pgfscope}%
\begin{pgfscope}%
\pgfpathrectangle{\pgfqpoint{1.250000in}{0.550000in}}{\pgfqpoint{7.750000in}{3.850000in}}%
\pgfusepath{clip}%
\pgfsetbuttcap%
\pgfsetmiterjoin%
\definecolor{currentfill}{rgb}{0.501961,0.501961,0.501961}%
\pgfsetfillcolor{currentfill}%
\pgfsetfillopacity{0.700000}%
\pgfsetlinewidth{0.000000pt}%
\definecolor{currentstroke}{rgb}{0.000000,0.000000,0.000000}%
\pgfsetstrokecolor{currentstroke}%
\pgfsetstrokeopacity{0.700000}%
\pgfsetdash{}{0pt}%
\pgfpathmoveto{\pgfqpoint{2.895774in}{0.550000in}}%
\pgfpathlineto{\pgfqpoint{2.923295in}{0.550000in}}%
\pgfpathlineto{\pgfqpoint{2.923295in}{0.550000in}}%
\pgfpathlineto{\pgfqpoint{2.895774in}{0.550000in}}%
\pgfpathlineto{\pgfqpoint{2.895774in}{0.550000in}}%
\pgfpathclose%
\pgfusepath{fill}%
\end{pgfscope}%
\begin{pgfscope}%
\pgfpathrectangle{\pgfqpoint{1.250000in}{0.550000in}}{\pgfqpoint{7.750000in}{3.850000in}}%
\pgfusepath{clip}%
\pgfsetbuttcap%
\pgfsetmiterjoin%
\definecolor{currentfill}{rgb}{0.501961,0.501961,0.501961}%
\pgfsetfillcolor{currentfill}%
\pgfsetfillopacity{0.700000}%
\pgfsetlinewidth{0.000000pt}%
\definecolor{currentstroke}{rgb}{0.000000,0.000000,0.000000}%
\pgfsetstrokecolor{currentstroke}%
\pgfsetstrokeopacity{0.700000}%
\pgfsetdash{}{0pt}%
\pgfpathmoveto{\pgfqpoint{2.923295in}{0.550000in}}%
\pgfpathlineto{\pgfqpoint{2.950817in}{0.550000in}}%
\pgfpathlineto{\pgfqpoint{2.950817in}{0.550000in}}%
\pgfpathlineto{\pgfqpoint{2.923295in}{0.550000in}}%
\pgfpathlineto{\pgfqpoint{2.923295in}{0.550000in}}%
\pgfpathclose%
\pgfusepath{fill}%
\end{pgfscope}%
\begin{pgfscope}%
\pgfpathrectangle{\pgfqpoint{1.250000in}{0.550000in}}{\pgfqpoint{7.750000in}{3.850000in}}%
\pgfusepath{clip}%
\pgfsetbuttcap%
\pgfsetmiterjoin%
\definecolor{currentfill}{rgb}{0.501961,0.501961,0.501961}%
\pgfsetfillcolor{currentfill}%
\pgfsetfillopacity{0.700000}%
\pgfsetlinewidth{0.000000pt}%
\definecolor{currentstroke}{rgb}{0.000000,0.000000,0.000000}%
\pgfsetstrokecolor{currentstroke}%
\pgfsetstrokeopacity{0.700000}%
\pgfsetdash{}{0pt}%
\pgfpathmoveto{\pgfqpoint{2.950817in}{0.550000in}}%
\pgfpathlineto{\pgfqpoint{2.978338in}{0.550000in}}%
\pgfpathlineto{\pgfqpoint{2.978338in}{1.845448in}}%
\pgfpathlineto{\pgfqpoint{2.950817in}{1.845448in}}%
\pgfpathlineto{\pgfqpoint{2.950817in}{0.550000in}}%
\pgfpathclose%
\pgfusepath{fill}%
\end{pgfscope}%
\begin{pgfscope}%
\pgfpathrectangle{\pgfqpoint{1.250000in}{0.550000in}}{\pgfqpoint{7.750000in}{3.850000in}}%
\pgfusepath{clip}%
\pgfsetbuttcap%
\pgfsetmiterjoin%
\definecolor{currentfill}{rgb}{0.501961,0.501961,0.501961}%
\pgfsetfillcolor{currentfill}%
\pgfsetfillopacity{0.700000}%
\pgfsetlinewidth{0.000000pt}%
\definecolor{currentstroke}{rgb}{0.000000,0.000000,0.000000}%
\pgfsetstrokecolor{currentstroke}%
\pgfsetstrokeopacity{0.700000}%
\pgfsetdash{}{0pt}%
\pgfpathmoveto{\pgfqpoint{2.978338in}{0.550000in}}%
\pgfpathlineto{\pgfqpoint{3.005859in}{0.550000in}}%
\pgfpathlineto{\pgfqpoint{3.005859in}{0.550000in}}%
\pgfpathlineto{\pgfqpoint{2.978338in}{0.550000in}}%
\pgfpathlineto{\pgfqpoint{2.978338in}{0.550000in}}%
\pgfpathclose%
\pgfusepath{fill}%
\end{pgfscope}%
\begin{pgfscope}%
\pgfpathrectangle{\pgfqpoint{1.250000in}{0.550000in}}{\pgfqpoint{7.750000in}{3.850000in}}%
\pgfusepath{clip}%
\pgfsetbuttcap%
\pgfsetmiterjoin%
\definecolor{currentfill}{rgb}{0.501961,0.501961,0.501961}%
\pgfsetfillcolor{currentfill}%
\pgfsetfillopacity{0.700000}%
\pgfsetlinewidth{0.000000pt}%
\definecolor{currentstroke}{rgb}{0.000000,0.000000,0.000000}%
\pgfsetstrokecolor{currentstroke}%
\pgfsetstrokeopacity{0.700000}%
\pgfsetdash{}{0pt}%
\pgfpathmoveto{\pgfqpoint{3.005859in}{0.550000in}}%
\pgfpathlineto{\pgfqpoint{3.033381in}{0.550000in}}%
\pgfpathlineto{\pgfqpoint{3.033381in}{0.550000in}}%
\pgfpathlineto{\pgfqpoint{3.005859in}{0.550000in}}%
\pgfpathlineto{\pgfqpoint{3.005859in}{0.550000in}}%
\pgfpathclose%
\pgfusepath{fill}%
\end{pgfscope}%
\begin{pgfscope}%
\pgfpathrectangle{\pgfqpoint{1.250000in}{0.550000in}}{\pgfqpoint{7.750000in}{3.850000in}}%
\pgfusepath{clip}%
\pgfsetbuttcap%
\pgfsetmiterjoin%
\definecolor{currentfill}{rgb}{0.501961,0.501961,0.501961}%
\pgfsetfillcolor{currentfill}%
\pgfsetfillopacity{0.700000}%
\pgfsetlinewidth{0.000000pt}%
\definecolor{currentstroke}{rgb}{0.000000,0.000000,0.000000}%
\pgfsetstrokecolor{currentstroke}%
\pgfsetstrokeopacity{0.700000}%
\pgfsetdash{}{0pt}%
\pgfpathmoveto{\pgfqpoint{3.033381in}{0.550000in}}%
\pgfpathlineto{\pgfqpoint{3.060902in}{0.550000in}}%
\pgfpathlineto{\pgfqpoint{3.060902in}{0.550000in}}%
\pgfpathlineto{\pgfqpoint{3.033381in}{0.550000in}}%
\pgfpathlineto{\pgfqpoint{3.033381in}{0.550000in}}%
\pgfpathclose%
\pgfusepath{fill}%
\end{pgfscope}%
\begin{pgfscope}%
\pgfpathrectangle{\pgfqpoint{1.250000in}{0.550000in}}{\pgfqpoint{7.750000in}{3.850000in}}%
\pgfusepath{clip}%
\pgfsetbuttcap%
\pgfsetmiterjoin%
\definecolor{currentfill}{rgb}{0.501961,0.501961,0.501961}%
\pgfsetfillcolor{currentfill}%
\pgfsetfillopacity{0.700000}%
\pgfsetlinewidth{0.000000pt}%
\definecolor{currentstroke}{rgb}{0.000000,0.000000,0.000000}%
\pgfsetstrokecolor{currentstroke}%
\pgfsetstrokeopacity{0.700000}%
\pgfsetdash{}{0pt}%
\pgfpathmoveto{\pgfqpoint{3.060902in}{0.550000in}}%
\pgfpathlineto{\pgfqpoint{3.088423in}{0.550000in}}%
\pgfpathlineto{\pgfqpoint{3.088423in}{0.550000in}}%
\pgfpathlineto{\pgfqpoint{3.060902in}{0.550000in}}%
\pgfpathlineto{\pgfqpoint{3.060902in}{0.550000in}}%
\pgfpathclose%
\pgfusepath{fill}%
\end{pgfscope}%
\begin{pgfscope}%
\pgfpathrectangle{\pgfqpoint{1.250000in}{0.550000in}}{\pgfqpoint{7.750000in}{3.850000in}}%
\pgfusepath{clip}%
\pgfsetbuttcap%
\pgfsetmiterjoin%
\definecolor{currentfill}{rgb}{0.501961,0.501961,0.501961}%
\pgfsetfillcolor{currentfill}%
\pgfsetfillopacity{0.700000}%
\pgfsetlinewidth{0.000000pt}%
\definecolor{currentstroke}{rgb}{0.000000,0.000000,0.000000}%
\pgfsetstrokecolor{currentstroke}%
\pgfsetstrokeopacity{0.700000}%
\pgfsetdash{}{0pt}%
\pgfpathmoveto{\pgfqpoint{3.088423in}{0.550000in}}%
\pgfpathlineto{\pgfqpoint{3.115945in}{0.550000in}}%
\pgfpathlineto{\pgfqpoint{3.115945in}{0.550000in}}%
\pgfpathlineto{\pgfqpoint{3.088423in}{0.550000in}}%
\pgfpathlineto{\pgfqpoint{3.088423in}{0.550000in}}%
\pgfpathclose%
\pgfusepath{fill}%
\end{pgfscope}%
\begin{pgfscope}%
\pgfpathrectangle{\pgfqpoint{1.250000in}{0.550000in}}{\pgfqpoint{7.750000in}{3.850000in}}%
\pgfusepath{clip}%
\pgfsetbuttcap%
\pgfsetmiterjoin%
\definecolor{currentfill}{rgb}{0.501961,0.501961,0.501961}%
\pgfsetfillcolor{currentfill}%
\pgfsetfillopacity{0.700000}%
\pgfsetlinewidth{0.000000pt}%
\definecolor{currentstroke}{rgb}{0.000000,0.000000,0.000000}%
\pgfsetstrokecolor{currentstroke}%
\pgfsetstrokeopacity{0.700000}%
\pgfsetdash{}{0pt}%
\pgfpathmoveto{\pgfqpoint{3.115945in}{0.550000in}}%
\pgfpathlineto{\pgfqpoint{3.143466in}{0.550000in}}%
\pgfpathlineto{\pgfqpoint{3.143466in}{1.619100in}}%
\pgfpathlineto{\pgfqpoint{3.115945in}{1.619100in}}%
\pgfpathlineto{\pgfqpoint{3.115945in}{0.550000in}}%
\pgfpathclose%
\pgfusepath{fill}%
\end{pgfscope}%
\begin{pgfscope}%
\pgfpathrectangle{\pgfqpoint{1.250000in}{0.550000in}}{\pgfqpoint{7.750000in}{3.850000in}}%
\pgfusepath{clip}%
\pgfsetbuttcap%
\pgfsetmiterjoin%
\definecolor{currentfill}{rgb}{0.501961,0.501961,0.501961}%
\pgfsetfillcolor{currentfill}%
\pgfsetfillopacity{0.700000}%
\pgfsetlinewidth{0.000000pt}%
\definecolor{currentstroke}{rgb}{0.000000,0.000000,0.000000}%
\pgfsetstrokecolor{currentstroke}%
\pgfsetstrokeopacity{0.700000}%
\pgfsetdash{}{0pt}%
\pgfpathmoveto{\pgfqpoint{3.143466in}{0.550000in}}%
\pgfpathlineto{\pgfqpoint{3.170987in}{0.550000in}}%
\pgfpathlineto{\pgfqpoint{3.170987in}{0.550000in}}%
\pgfpathlineto{\pgfqpoint{3.143466in}{0.550000in}}%
\pgfpathlineto{\pgfqpoint{3.143466in}{0.550000in}}%
\pgfpathclose%
\pgfusepath{fill}%
\end{pgfscope}%
\begin{pgfscope}%
\pgfpathrectangle{\pgfqpoint{1.250000in}{0.550000in}}{\pgfqpoint{7.750000in}{3.850000in}}%
\pgfusepath{clip}%
\pgfsetbuttcap%
\pgfsetmiterjoin%
\definecolor{currentfill}{rgb}{0.501961,0.501961,0.501961}%
\pgfsetfillcolor{currentfill}%
\pgfsetfillopacity{0.700000}%
\pgfsetlinewidth{0.000000pt}%
\definecolor{currentstroke}{rgb}{0.000000,0.000000,0.000000}%
\pgfsetstrokecolor{currentstroke}%
\pgfsetstrokeopacity{0.700000}%
\pgfsetdash{}{0pt}%
\pgfpathmoveto{\pgfqpoint{3.170987in}{0.550000in}}%
\pgfpathlineto{\pgfqpoint{3.198509in}{0.550000in}}%
\pgfpathlineto{\pgfqpoint{3.198509in}{0.550000in}}%
\pgfpathlineto{\pgfqpoint{3.170987in}{0.550000in}}%
\pgfpathlineto{\pgfqpoint{3.170987in}{0.550000in}}%
\pgfpathclose%
\pgfusepath{fill}%
\end{pgfscope}%
\begin{pgfscope}%
\pgfpathrectangle{\pgfqpoint{1.250000in}{0.550000in}}{\pgfqpoint{7.750000in}{3.850000in}}%
\pgfusepath{clip}%
\pgfsetbuttcap%
\pgfsetmiterjoin%
\definecolor{currentfill}{rgb}{0.501961,0.501961,0.501961}%
\pgfsetfillcolor{currentfill}%
\pgfsetfillopacity{0.700000}%
\pgfsetlinewidth{0.000000pt}%
\definecolor{currentstroke}{rgb}{0.000000,0.000000,0.000000}%
\pgfsetstrokecolor{currentstroke}%
\pgfsetstrokeopacity{0.700000}%
\pgfsetdash{}{0pt}%
\pgfpathmoveto{\pgfqpoint{3.198509in}{0.550000in}}%
\pgfpathlineto{\pgfqpoint{3.226030in}{0.550000in}}%
\pgfpathlineto{\pgfqpoint{3.226030in}{0.550000in}}%
\pgfpathlineto{\pgfqpoint{3.198509in}{0.550000in}}%
\pgfpathlineto{\pgfqpoint{3.198509in}{0.550000in}}%
\pgfpathclose%
\pgfusepath{fill}%
\end{pgfscope}%
\begin{pgfscope}%
\pgfpathrectangle{\pgfqpoint{1.250000in}{0.550000in}}{\pgfqpoint{7.750000in}{3.850000in}}%
\pgfusepath{clip}%
\pgfsetbuttcap%
\pgfsetmiterjoin%
\definecolor{currentfill}{rgb}{0.501961,0.501961,0.501961}%
\pgfsetfillcolor{currentfill}%
\pgfsetfillopacity{0.700000}%
\pgfsetlinewidth{0.000000pt}%
\definecolor{currentstroke}{rgb}{0.000000,0.000000,0.000000}%
\pgfsetstrokecolor{currentstroke}%
\pgfsetstrokeopacity{0.700000}%
\pgfsetdash{}{0pt}%
\pgfpathmoveto{\pgfqpoint{3.226030in}{0.550000in}}%
\pgfpathlineto{\pgfqpoint{3.253551in}{0.550000in}}%
\pgfpathlineto{\pgfqpoint{3.253551in}{0.550000in}}%
\pgfpathlineto{\pgfqpoint{3.226030in}{0.550000in}}%
\pgfpathlineto{\pgfqpoint{3.226030in}{0.550000in}}%
\pgfpathclose%
\pgfusepath{fill}%
\end{pgfscope}%
\begin{pgfscope}%
\pgfpathrectangle{\pgfqpoint{1.250000in}{0.550000in}}{\pgfqpoint{7.750000in}{3.850000in}}%
\pgfusepath{clip}%
\pgfsetbuttcap%
\pgfsetmiterjoin%
\definecolor{currentfill}{rgb}{0.501961,0.501961,0.501961}%
\pgfsetfillcolor{currentfill}%
\pgfsetfillopacity{0.700000}%
\pgfsetlinewidth{0.000000pt}%
\definecolor{currentstroke}{rgb}{0.000000,0.000000,0.000000}%
\pgfsetstrokecolor{currentstroke}%
\pgfsetstrokeopacity{0.700000}%
\pgfsetdash{}{0pt}%
\pgfpathmoveto{\pgfqpoint{3.253551in}{0.550000in}}%
\pgfpathlineto{\pgfqpoint{3.281072in}{0.550000in}}%
\pgfpathlineto{\pgfqpoint{3.281072in}{0.550000in}}%
\pgfpathlineto{\pgfqpoint{3.253551in}{0.550000in}}%
\pgfpathlineto{\pgfqpoint{3.253551in}{0.550000in}}%
\pgfpathclose%
\pgfusepath{fill}%
\end{pgfscope}%
\begin{pgfscope}%
\pgfpathrectangle{\pgfqpoint{1.250000in}{0.550000in}}{\pgfqpoint{7.750000in}{3.850000in}}%
\pgfusepath{clip}%
\pgfsetbuttcap%
\pgfsetmiterjoin%
\definecolor{currentfill}{rgb}{0.501961,0.501961,0.501961}%
\pgfsetfillcolor{currentfill}%
\pgfsetfillopacity{0.700000}%
\pgfsetlinewidth{0.000000pt}%
\definecolor{currentstroke}{rgb}{0.000000,0.000000,0.000000}%
\pgfsetstrokecolor{currentstroke}%
\pgfsetstrokeopacity{0.700000}%
\pgfsetdash{}{0pt}%
\pgfpathmoveto{\pgfqpoint{3.281072in}{0.550000in}}%
\pgfpathlineto{\pgfqpoint{3.308594in}{0.550000in}}%
\pgfpathlineto{\pgfqpoint{3.308594in}{1.464963in}}%
\pgfpathlineto{\pgfqpoint{3.281072in}{1.464963in}}%
\pgfpathlineto{\pgfqpoint{3.281072in}{0.550000in}}%
\pgfpathclose%
\pgfusepath{fill}%
\end{pgfscope}%
\begin{pgfscope}%
\pgfpathrectangle{\pgfqpoint{1.250000in}{0.550000in}}{\pgfqpoint{7.750000in}{3.850000in}}%
\pgfusepath{clip}%
\pgfsetbuttcap%
\pgfsetmiterjoin%
\definecolor{currentfill}{rgb}{0.501961,0.501961,0.501961}%
\pgfsetfillcolor{currentfill}%
\pgfsetfillopacity{0.700000}%
\pgfsetlinewidth{0.000000pt}%
\definecolor{currentstroke}{rgb}{0.000000,0.000000,0.000000}%
\pgfsetstrokecolor{currentstroke}%
\pgfsetstrokeopacity{0.700000}%
\pgfsetdash{}{0pt}%
\pgfpathmoveto{\pgfqpoint{3.308594in}{0.550000in}}%
\pgfpathlineto{\pgfqpoint{3.336115in}{0.550000in}}%
\pgfpathlineto{\pgfqpoint{3.336115in}{0.550000in}}%
\pgfpathlineto{\pgfqpoint{3.308594in}{0.550000in}}%
\pgfpathlineto{\pgfqpoint{3.308594in}{0.550000in}}%
\pgfpathclose%
\pgfusepath{fill}%
\end{pgfscope}%
\begin{pgfscope}%
\pgfpathrectangle{\pgfqpoint{1.250000in}{0.550000in}}{\pgfqpoint{7.750000in}{3.850000in}}%
\pgfusepath{clip}%
\pgfsetbuttcap%
\pgfsetmiterjoin%
\definecolor{currentfill}{rgb}{0.501961,0.501961,0.501961}%
\pgfsetfillcolor{currentfill}%
\pgfsetfillopacity{0.700000}%
\pgfsetlinewidth{0.000000pt}%
\definecolor{currentstroke}{rgb}{0.000000,0.000000,0.000000}%
\pgfsetstrokecolor{currentstroke}%
\pgfsetstrokeopacity{0.700000}%
\pgfsetdash{}{0pt}%
\pgfpathmoveto{\pgfqpoint{3.336115in}{0.550000in}}%
\pgfpathlineto{\pgfqpoint{3.363636in}{0.550000in}}%
\pgfpathlineto{\pgfqpoint{3.363636in}{0.550000in}}%
\pgfpathlineto{\pgfqpoint{3.336115in}{0.550000in}}%
\pgfpathlineto{\pgfqpoint{3.336115in}{0.550000in}}%
\pgfpathclose%
\pgfusepath{fill}%
\end{pgfscope}%
\begin{pgfscope}%
\pgfpathrectangle{\pgfqpoint{1.250000in}{0.550000in}}{\pgfqpoint{7.750000in}{3.850000in}}%
\pgfusepath{clip}%
\pgfsetbuttcap%
\pgfsetmiterjoin%
\definecolor{currentfill}{rgb}{0.501961,0.501961,0.501961}%
\pgfsetfillcolor{currentfill}%
\pgfsetfillopacity{0.700000}%
\pgfsetlinewidth{0.000000pt}%
\definecolor{currentstroke}{rgb}{0.000000,0.000000,0.000000}%
\pgfsetstrokecolor{currentstroke}%
\pgfsetstrokeopacity{0.700000}%
\pgfsetdash{}{0pt}%
\pgfpathmoveto{\pgfqpoint{3.363636in}{0.550000in}}%
\pgfpathlineto{\pgfqpoint{3.391158in}{0.550000in}}%
\pgfpathlineto{\pgfqpoint{3.391158in}{0.550000in}}%
\pgfpathlineto{\pgfqpoint{3.363636in}{0.550000in}}%
\pgfpathlineto{\pgfqpoint{3.363636in}{0.550000in}}%
\pgfpathclose%
\pgfusepath{fill}%
\end{pgfscope}%
\begin{pgfscope}%
\pgfpathrectangle{\pgfqpoint{1.250000in}{0.550000in}}{\pgfqpoint{7.750000in}{3.850000in}}%
\pgfusepath{clip}%
\pgfsetbuttcap%
\pgfsetmiterjoin%
\definecolor{currentfill}{rgb}{0.501961,0.501961,0.501961}%
\pgfsetfillcolor{currentfill}%
\pgfsetfillopacity{0.700000}%
\pgfsetlinewidth{0.000000pt}%
\definecolor{currentstroke}{rgb}{0.000000,0.000000,0.000000}%
\pgfsetstrokecolor{currentstroke}%
\pgfsetstrokeopacity{0.700000}%
\pgfsetdash{}{0pt}%
\pgfpathmoveto{\pgfqpoint{3.391158in}{0.550000in}}%
\pgfpathlineto{\pgfqpoint{3.418679in}{0.550000in}}%
\pgfpathlineto{\pgfqpoint{3.418679in}{0.550000in}}%
\pgfpathlineto{\pgfqpoint{3.391158in}{0.550000in}}%
\pgfpathlineto{\pgfqpoint{3.391158in}{0.550000in}}%
\pgfpathclose%
\pgfusepath{fill}%
\end{pgfscope}%
\begin{pgfscope}%
\pgfpathrectangle{\pgfqpoint{1.250000in}{0.550000in}}{\pgfqpoint{7.750000in}{3.850000in}}%
\pgfusepath{clip}%
\pgfsetbuttcap%
\pgfsetmiterjoin%
\definecolor{currentfill}{rgb}{0.501961,0.501961,0.501961}%
\pgfsetfillcolor{currentfill}%
\pgfsetfillopacity{0.700000}%
\pgfsetlinewidth{0.000000pt}%
\definecolor{currentstroke}{rgb}{0.000000,0.000000,0.000000}%
\pgfsetstrokecolor{currentstroke}%
\pgfsetstrokeopacity{0.700000}%
\pgfsetdash{}{0pt}%
\pgfpathmoveto{\pgfqpoint{3.418679in}{0.550000in}}%
\pgfpathlineto{\pgfqpoint{3.446200in}{0.550000in}}%
\pgfpathlineto{\pgfqpoint{3.446200in}{1.319671in}}%
\pgfpathlineto{\pgfqpoint{3.418679in}{1.319671in}}%
\pgfpathlineto{\pgfqpoint{3.418679in}{0.550000in}}%
\pgfpathclose%
\pgfusepath{fill}%
\end{pgfscope}%
\begin{pgfscope}%
\pgfpathrectangle{\pgfqpoint{1.250000in}{0.550000in}}{\pgfqpoint{7.750000in}{3.850000in}}%
\pgfusepath{clip}%
\pgfsetbuttcap%
\pgfsetmiterjoin%
\definecolor{currentfill}{rgb}{0.501961,0.501961,0.501961}%
\pgfsetfillcolor{currentfill}%
\pgfsetfillopacity{0.700000}%
\pgfsetlinewidth{0.000000pt}%
\definecolor{currentstroke}{rgb}{0.000000,0.000000,0.000000}%
\pgfsetstrokecolor{currentstroke}%
\pgfsetstrokeopacity{0.700000}%
\pgfsetdash{}{0pt}%
\pgfpathmoveto{\pgfqpoint{3.446200in}{0.550000in}}%
\pgfpathlineto{\pgfqpoint{3.473722in}{0.550000in}}%
\pgfpathlineto{\pgfqpoint{3.473722in}{0.550000in}}%
\pgfpathlineto{\pgfqpoint{3.446200in}{0.550000in}}%
\pgfpathlineto{\pgfqpoint{3.446200in}{0.550000in}}%
\pgfpathclose%
\pgfusepath{fill}%
\end{pgfscope}%
\begin{pgfscope}%
\pgfpathrectangle{\pgfqpoint{1.250000in}{0.550000in}}{\pgfqpoint{7.750000in}{3.850000in}}%
\pgfusepath{clip}%
\pgfsetbuttcap%
\pgfsetmiterjoin%
\definecolor{currentfill}{rgb}{0.501961,0.501961,0.501961}%
\pgfsetfillcolor{currentfill}%
\pgfsetfillopacity{0.700000}%
\pgfsetlinewidth{0.000000pt}%
\definecolor{currentstroke}{rgb}{0.000000,0.000000,0.000000}%
\pgfsetstrokecolor{currentstroke}%
\pgfsetstrokeopacity{0.700000}%
\pgfsetdash{}{0pt}%
\pgfpathmoveto{\pgfqpoint{3.473722in}{0.550000in}}%
\pgfpathlineto{\pgfqpoint{3.501243in}{0.550000in}}%
\pgfpathlineto{\pgfqpoint{3.501243in}{0.550000in}}%
\pgfpathlineto{\pgfqpoint{3.473722in}{0.550000in}}%
\pgfpathlineto{\pgfqpoint{3.473722in}{0.550000in}}%
\pgfpathclose%
\pgfusepath{fill}%
\end{pgfscope}%
\begin{pgfscope}%
\pgfpathrectangle{\pgfqpoint{1.250000in}{0.550000in}}{\pgfqpoint{7.750000in}{3.850000in}}%
\pgfusepath{clip}%
\pgfsetbuttcap%
\pgfsetmiterjoin%
\definecolor{currentfill}{rgb}{0.501961,0.501961,0.501961}%
\pgfsetfillcolor{currentfill}%
\pgfsetfillopacity{0.700000}%
\pgfsetlinewidth{0.000000pt}%
\definecolor{currentstroke}{rgb}{0.000000,0.000000,0.000000}%
\pgfsetstrokecolor{currentstroke}%
\pgfsetstrokeopacity{0.700000}%
\pgfsetdash{}{0pt}%
\pgfpathmoveto{\pgfqpoint{3.501243in}{0.550000in}}%
\pgfpathlineto{\pgfqpoint{3.528764in}{0.550000in}}%
\pgfpathlineto{\pgfqpoint{3.528764in}{0.550000in}}%
\pgfpathlineto{\pgfqpoint{3.501243in}{0.550000in}}%
\pgfpathlineto{\pgfqpoint{3.501243in}{0.550000in}}%
\pgfpathclose%
\pgfusepath{fill}%
\end{pgfscope}%
\begin{pgfscope}%
\pgfpathrectangle{\pgfqpoint{1.250000in}{0.550000in}}{\pgfqpoint{7.750000in}{3.850000in}}%
\pgfusepath{clip}%
\pgfsetbuttcap%
\pgfsetmiterjoin%
\definecolor{currentfill}{rgb}{0.501961,0.501961,0.501961}%
\pgfsetfillcolor{currentfill}%
\pgfsetfillopacity{0.700000}%
\pgfsetlinewidth{0.000000pt}%
\definecolor{currentstroke}{rgb}{0.000000,0.000000,0.000000}%
\pgfsetstrokecolor{currentstroke}%
\pgfsetstrokeopacity{0.700000}%
\pgfsetdash{}{0pt}%
\pgfpathmoveto{\pgfqpoint{3.528764in}{0.550000in}}%
\pgfpathlineto{\pgfqpoint{3.556286in}{0.550000in}}%
\pgfpathlineto{\pgfqpoint{3.556286in}{1.243110in}}%
\pgfpathlineto{\pgfqpoint{3.528764in}{1.243110in}}%
\pgfpathlineto{\pgfqpoint{3.528764in}{0.550000in}}%
\pgfpathclose%
\pgfusepath{fill}%
\end{pgfscope}%
\begin{pgfscope}%
\pgfpathrectangle{\pgfqpoint{1.250000in}{0.550000in}}{\pgfqpoint{7.750000in}{3.850000in}}%
\pgfusepath{clip}%
\pgfsetbuttcap%
\pgfsetmiterjoin%
\definecolor{currentfill}{rgb}{0.501961,0.501961,0.501961}%
\pgfsetfillcolor{currentfill}%
\pgfsetfillopacity{0.700000}%
\pgfsetlinewidth{0.000000pt}%
\definecolor{currentstroke}{rgb}{0.000000,0.000000,0.000000}%
\pgfsetstrokecolor{currentstroke}%
\pgfsetstrokeopacity{0.700000}%
\pgfsetdash{}{0pt}%
\pgfpathmoveto{\pgfqpoint{3.556286in}{0.550000in}}%
\pgfpathlineto{\pgfqpoint{3.583807in}{0.550000in}}%
\pgfpathlineto{\pgfqpoint{3.583807in}{0.550000in}}%
\pgfpathlineto{\pgfqpoint{3.556286in}{0.550000in}}%
\pgfpathlineto{\pgfqpoint{3.556286in}{0.550000in}}%
\pgfpathclose%
\pgfusepath{fill}%
\end{pgfscope}%
\begin{pgfscope}%
\pgfpathrectangle{\pgfqpoint{1.250000in}{0.550000in}}{\pgfqpoint{7.750000in}{3.850000in}}%
\pgfusepath{clip}%
\pgfsetbuttcap%
\pgfsetmiterjoin%
\definecolor{currentfill}{rgb}{0.501961,0.501961,0.501961}%
\pgfsetfillcolor{currentfill}%
\pgfsetfillopacity{0.700000}%
\pgfsetlinewidth{0.000000pt}%
\definecolor{currentstroke}{rgb}{0.000000,0.000000,0.000000}%
\pgfsetstrokecolor{currentstroke}%
\pgfsetstrokeopacity{0.700000}%
\pgfsetdash{}{0pt}%
\pgfpathmoveto{\pgfqpoint{3.583807in}{0.550000in}}%
\pgfpathlineto{\pgfqpoint{3.611328in}{0.550000in}}%
\pgfpathlineto{\pgfqpoint{3.611328in}{0.550000in}}%
\pgfpathlineto{\pgfqpoint{3.583807in}{0.550000in}}%
\pgfpathlineto{\pgfqpoint{3.583807in}{0.550000in}}%
\pgfpathclose%
\pgfusepath{fill}%
\end{pgfscope}%
\begin{pgfscope}%
\pgfpathrectangle{\pgfqpoint{1.250000in}{0.550000in}}{\pgfqpoint{7.750000in}{3.850000in}}%
\pgfusepath{clip}%
\pgfsetbuttcap%
\pgfsetmiterjoin%
\definecolor{currentfill}{rgb}{0.501961,0.501961,0.501961}%
\pgfsetfillcolor{currentfill}%
\pgfsetfillopacity{0.700000}%
\pgfsetlinewidth{0.000000pt}%
\definecolor{currentstroke}{rgb}{0.000000,0.000000,0.000000}%
\pgfsetstrokecolor{currentstroke}%
\pgfsetstrokeopacity{0.700000}%
\pgfsetdash{}{0pt}%
\pgfpathmoveto{\pgfqpoint{3.611328in}{0.550000in}}%
\pgfpathlineto{\pgfqpoint{3.638849in}{0.550000in}}%
\pgfpathlineto{\pgfqpoint{3.638849in}{0.550000in}}%
\pgfpathlineto{\pgfqpoint{3.611328in}{0.550000in}}%
\pgfpathlineto{\pgfqpoint{3.611328in}{0.550000in}}%
\pgfpathclose%
\pgfusepath{fill}%
\end{pgfscope}%
\begin{pgfscope}%
\pgfpathrectangle{\pgfqpoint{1.250000in}{0.550000in}}{\pgfqpoint{7.750000in}{3.850000in}}%
\pgfusepath{clip}%
\pgfsetbuttcap%
\pgfsetmiterjoin%
\definecolor{currentfill}{rgb}{0.501961,0.501961,0.501961}%
\pgfsetfillcolor{currentfill}%
\pgfsetfillopacity{0.700000}%
\pgfsetlinewidth{0.000000pt}%
\definecolor{currentstroke}{rgb}{0.000000,0.000000,0.000000}%
\pgfsetstrokecolor{currentstroke}%
\pgfsetstrokeopacity{0.700000}%
\pgfsetdash{}{0pt}%
\pgfpathmoveto{\pgfqpoint{3.638849in}{0.550000in}}%
\pgfpathlineto{\pgfqpoint{3.666371in}{0.550000in}}%
\pgfpathlineto{\pgfqpoint{3.666371in}{1.179454in}}%
\pgfpathlineto{\pgfqpoint{3.638849in}{1.179454in}}%
\pgfpathlineto{\pgfqpoint{3.638849in}{0.550000in}}%
\pgfpathclose%
\pgfusepath{fill}%
\end{pgfscope}%
\begin{pgfscope}%
\pgfpathrectangle{\pgfqpoint{1.250000in}{0.550000in}}{\pgfqpoint{7.750000in}{3.850000in}}%
\pgfusepath{clip}%
\pgfsetbuttcap%
\pgfsetmiterjoin%
\definecolor{currentfill}{rgb}{0.501961,0.501961,0.501961}%
\pgfsetfillcolor{currentfill}%
\pgfsetfillopacity{0.700000}%
\pgfsetlinewidth{0.000000pt}%
\definecolor{currentstroke}{rgb}{0.000000,0.000000,0.000000}%
\pgfsetstrokecolor{currentstroke}%
\pgfsetstrokeopacity{0.700000}%
\pgfsetdash{}{0pt}%
\pgfpathmoveto{\pgfqpoint{3.666371in}{0.550000in}}%
\pgfpathlineto{\pgfqpoint{3.693892in}{0.550000in}}%
\pgfpathlineto{\pgfqpoint{3.693892in}{0.550000in}}%
\pgfpathlineto{\pgfqpoint{3.666371in}{0.550000in}}%
\pgfpathlineto{\pgfqpoint{3.666371in}{0.550000in}}%
\pgfpathclose%
\pgfusepath{fill}%
\end{pgfscope}%
\begin{pgfscope}%
\pgfpathrectangle{\pgfqpoint{1.250000in}{0.550000in}}{\pgfqpoint{7.750000in}{3.850000in}}%
\pgfusepath{clip}%
\pgfsetbuttcap%
\pgfsetmiterjoin%
\definecolor{currentfill}{rgb}{0.501961,0.501961,0.501961}%
\pgfsetfillcolor{currentfill}%
\pgfsetfillopacity{0.700000}%
\pgfsetlinewidth{0.000000pt}%
\definecolor{currentstroke}{rgb}{0.000000,0.000000,0.000000}%
\pgfsetstrokecolor{currentstroke}%
\pgfsetstrokeopacity{0.700000}%
\pgfsetdash{}{0pt}%
\pgfpathmoveto{\pgfqpoint{3.693892in}{0.550000in}}%
\pgfpathlineto{\pgfqpoint{3.721413in}{0.550000in}}%
\pgfpathlineto{\pgfqpoint{3.721413in}{0.550000in}}%
\pgfpathlineto{\pgfqpoint{3.693892in}{0.550000in}}%
\pgfpathlineto{\pgfqpoint{3.693892in}{0.550000in}}%
\pgfpathclose%
\pgfusepath{fill}%
\end{pgfscope}%
\begin{pgfscope}%
\pgfpathrectangle{\pgfqpoint{1.250000in}{0.550000in}}{\pgfqpoint{7.750000in}{3.850000in}}%
\pgfusepath{clip}%
\pgfsetbuttcap%
\pgfsetmiterjoin%
\definecolor{currentfill}{rgb}{0.501961,0.501961,0.501961}%
\pgfsetfillcolor{currentfill}%
\pgfsetfillopacity{0.700000}%
\pgfsetlinewidth{0.000000pt}%
\definecolor{currentstroke}{rgb}{0.000000,0.000000,0.000000}%
\pgfsetstrokecolor{currentstroke}%
\pgfsetstrokeopacity{0.700000}%
\pgfsetdash{}{0pt}%
\pgfpathmoveto{\pgfqpoint{3.721413in}{0.550000in}}%
\pgfpathlineto{\pgfqpoint{3.748935in}{0.550000in}}%
\pgfpathlineto{\pgfqpoint{3.748935in}{0.550000in}}%
\pgfpathlineto{\pgfqpoint{3.721413in}{0.550000in}}%
\pgfpathlineto{\pgfqpoint{3.721413in}{0.550000in}}%
\pgfpathclose%
\pgfusepath{fill}%
\end{pgfscope}%
\begin{pgfscope}%
\pgfpathrectangle{\pgfqpoint{1.250000in}{0.550000in}}{\pgfqpoint{7.750000in}{3.850000in}}%
\pgfusepath{clip}%
\pgfsetbuttcap%
\pgfsetmiterjoin%
\definecolor{currentfill}{rgb}{0.501961,0.501961,0.501961}%
\pgfsetfillcolor{currentfill}%
\pgfsetfillopacity{0.700000}%
\pgfsetlinewidth{0.000000pt}%
\definecolor{currentstroke}{rgb}{0.000000,0.000000,0.000000}%
\pgfsetstrokecolor{currentstroke}%
\pgfsetstrokeopacity{0.700000}%
\pgfsetdash{}{0pt}%
\pgfpathmoveto{\pgfqpoint{3.748935in}{0.550000in}}%
\pgfpathlineto{\pgfqpoint{3.776456in}{0.550000in}}%
\pgfpathlineto{\pgfqpoint{3.776456in}{1.167274in}}%
\pgfpathlineto{\pgfqpoint{3.748935in}{1.167274in}}%
\pgfpathlineto{\pgfqpoint{3.748935in}{0.550000in}}%
\pgfpathclose%
\pgfusepath{fill}%
\end{pgfscope}%
\begin{pgfscope}%
\pgfpathrectangle{\pgfqpoint{1.250000in}{0.550000in}}{\pgfqpoint{7.750000in}{3.850000in}}%
\pgfusepath{clip}%
\pgfsetbuttcap%
\pgfsetmiterjoin%
\definecolor{currentfill}{rgb}{0.501961,0.501961,0.501961}%
\pgfsetfillcolor{currentfill}%
\pgfsetfillopacity{0.700000}%
\pgfsetlinewidth{0.000000pt}%
\definecolor{currentstroke}{rgb}{0.000000,0.000000,0.000000}%
\pgfsetstrokecolor{currentstroke}%
\pgfsetstrokeopacity{0.700000}%
\pgfsetdash{}{0pt}%
\pgfpathmoveto{\pgfqpoint{3.776456in}{0.550000in}}%
\pgfpathlineto{\pgfqpoint{3.803977in}{0.550000in}}%
\pgfpathlineto{\pgfqpoint{3.803977in}{0.550000in}}%
\pgfpathlineto{\pgfqpoint{3.776456in}{0.550000in}}%
\pgfpathlineto{\pgfqpoint{3.776456in}{0.550000in}}%
\pgfpathclose%
\pgfusepath{fill}%
\end{pgfscope}%
\begin{pgfscope}%
\pgfpathrectangle{\pgfqpoint{1.250000in}{0.550000in}}{\pgfqpoint{7.750000in}{3.850000in}}%
\pgfusepath{clip}%
\pgfsetbuttcap%
\pgfsetmiterjoin%
\definecolor{currentfill}{rgb}{0.501961,0.501961,0.501961}%
\pgfsetfillcolor{currentfill}%
\pgfsetfillopacity{0.700000}%
\pgfsetlinewidth{0.000000pt}%
\definecolor{currentstroke}{rgb}{0.000000,0.000000,0.000000}%
\pgfsetstrokecolor{currentstroke}%
\pgfsetstrokeopacity{0.700000}%
\pgfsetdash{}{0pt}%
\pgfpathmoveto{\pgfqpoint{3.803977in}{0.550000in}}%
\pgfpathlineto{\pgfqpoint{3.831499in}{0.550000in}}%
\pgfpathlineto{\pgfqpoint{3.831499in}{0.550000in}}%
\pgfpathlineto{\pgfqpoint{3.803977in}{0.550000in}}%
\pgfpathlineto{\pgfqpoint{3.803977in}{0.550000in}}%
\pgfpathclose%
\pgfusepath{fill}%
\end{pgfscope}%
\begin{pgfscope}%
\pgfpathrectangle{\pgfqpoint{1.250000in}{0.550000in}}{\pgfqpoint{7.750000in}{3.850000in}}%
\pgfusepath{clip}%
\pgfsetbuttcap%
\pgfsetmiterjoin%
\definecolor{currentfill}{rgb}{0.501961,0.501961,0.501961}%
\pgfsetfillcolor{currentfill}%
\pgfsetfillopacity{0.700000}%
\pgfsetlinewidth{0.000000pt}%
\definecolor{currentstroke}{rgb}{0.000000,0.000000,0.000000}%
\pgfsetstrokecolor{currentstroke}%
\pgfsetstrokeopacity{0.700000}%
\pgfsetdash{}{0pt}%
\pgfpathmoveto{\pgfqpoint{3.831499in}{0.550000in}}%
\pgfpathlineto{\pgfqpoint{3.859020in}{0.550000in}}%
\pgfpathlineto{\pgfqpoint{3.859020in}{1.126383in}}%
\pgfpathlineto{\pgfqpoint{3.831499in}{1.126383in}}%
\pgfpathlineto{\pgfqpoint{3.831499in}{0.550000in}}%
\pgfpathclose%
\pgfusepath{fill}%
\end{pgfscope}%
\begin{pgfscope}%
\pgfpathrectangle{\pgfqpoint{1.250000in}{0.550000in}}{\pgfqpoint{7.750000in}{3.850000in}}%
\pgfusepath{clip}%
\pgfsetbuttcap%
\pgfsetmiterjoin%
\definecolor{currentfill}{rgb}{0.501961,0.501961,0.501961}%
\pgfsetfillcolor{currentfill}%
\pgfsetfillopacity{0.700000}%
\pgfsetlinewidth{0.000000pt}%
\definecolor{currentstroke}{rgb}{0.000000,0.000000,0.000000}%
\pgfsetstrokecolor{currentstroke}%
\pgfsetstrokeopacity{0.700000}%
\pgfsetdash{}{0pt}%
\pgfpathmoveto{\pgfqpoint{3.859020in}{0.550000in}}%
\pgfpathlineto{\pgfqpoint{3.886541in}{0.550000in}}%
\pgfpathlineto{\pgfqpoint{3.886541in}{0.550000in}}%
\pgfpathlineto{\pgfqpoint{3.859020in}{0.550000in}}%
\pgfpathlineto{\pgfqpoint{3.859020in}{0.550000in}}%
\pgfpathclose%
\pgfusepath{fill}%
\end{pgfscope}%
\begin{pgfscope}%
\pgfpathrectangle{\pgfqpoint{1.250000in}{0.550000in}}{\pgfqpoint{7.750000in}{3.850000in}}%
\pgfusepath{clip}%
\pgfsetbuttcap%
\pgfsetmiterjoin%
\definecolor{currentfill}{rgb}{0.501961,0.501961,0.501961}%
\pgfsetfillcolor{currentfill}%
\pgfsetfillopacity{0.700000}%
\pgfsetlinewidth{0.000000pt}%
\definecolor{currentstroke}{rgb}{0.000000,0.000000,0.000000}%
\pgfsetstrokecolor{currentstroke}%
\pgfsetstrokeopacity{0.700000}%
\pgfsetdash{}{0pt}%
\pgfpathmoveto{\pgfqpoint{3.886541in}{0.550000in}}%
\pgfpathlineto{\pgfqpoint{3.914062in}{0.550000in}}%
\pgfpathlineto{\pgfqpoint{3.914062in}{0.550000in}}%
\pgfpathlineto{\pgfqpoint{3.886541in}{0.550000in}}%
\pgfpathlineto{\pgfqpoint{3.886541in}{0.550000in}}%
\pgfpathclose%
\pgfusepath{fill}%
\end{pgfscope}%
\begin{pgfscope}%
\pgfpathrectangle{\pgfqpoint{1.250000in}{0.550000in}}{\pgfqpoint{7.750000in}{3.850000in}}%
\pgfusepath{clip}%
\pgfsetbuttcap%
\pgfsetmiterjoin%
\definecolor{currentfill}{rgb}{0.501961,0.501961,0.501961}%
\pgfsetfillcolor{currentfill}%
\pgfsetfillopacity{0.700000}%
\pgfsetlinewidth{0.000000pt}%
\definecolor{currentstroke}{rgb}{0.000000,0.000000,0.000000}%
\pgfsetstrokecolor{currentstroke}%
\pgfsetstrokeopacity{0.700000}%
\pgfsetdash{}{0pt}%
\pgfpathmoveto{\pgfqpoint{3.914062in}{0.550000in}}%
\pgfpathlineto{\pgfqpoint{3.941584in}{0.550000in}}%
\pgfpathlineto{\pgfqpoint{3.941584in}{0.550000in}}%
\pgfpathlineto{\pgfqpoint{3.914062in}{0.550000in}}%
\pgfpathlineto{\pgfqpoint{3.914062in}{0.550000in}}%
\pgfpathclose%
\pgfusepath{fill}%
\end{pgfscope}%
\begin{pgfscope}%
\pgfpathrectangle{\pgfqpoint{1.250000in}{0.550000in}}{\pgfqpoint{7.750000in}{3.850000in}}%
\pgfusepath{clip}%
\pgfsetbuttcap%
\pgfsetmiterjoin%
\definecolor{currentfill}{rgb}{0.501961,0.501961,0.501961}%
\pgfsetfillcolor{currentfill}%
\pgfsetfillopacity{0.700000}%
\pgfsetlinewidth{0.000000pt}%
\definecolor{currentstroke}{rgb}{0.000000,0.000000,0.000000}%
\pgfsetstrokecolor{currentstroke}%
\pgfsetstrokeopacity{0.700000}%
\pgfsetdash{}{0pt}%
\pgfpathmoveto{\pgfqpoint{3.941584in}{0.550000in}}%
\pgfpathlineto{\pgfqpoint{3.969105in}{0.550000in}}%
\pgfpathlineto{\pgfqpoint{3.969105in}{1.112463in}}%
\pgfpathlineto{\pgfqpoint{3.941584in}{1.112463in}}%
\pgfpathlineto{\pgfqpoint{3.941584in}{0.550000in}}%
\pgfpathclose%
\pgfusepath{fill}%
\end{pgfscope}%
\begin{pgfscope}%
\pgfpathrectangle{\pgfqpoint{1.250000in}{0.550000in}}{\pgfqpoint{7.750000in}{3.850000in}}%
\pgfusepath{clip}%
\pgfsetbuttcap%
\pgfsetmiterjoin%
\definecolor{currentfill}{rgb}{0.501961,0.501961,0.501961}%
\pgfsetfillcolor{currentfill}%
\pgfsetfillopacity{0.700000}%
\pgfsetlinewidth{0.000000pt}%
\definecolor{currentstroke}{rgb}{0.000000,0.000000,0.000000}%
\pgfsetstrokecolor{currentstroke}%
\pgfsetstrokeopacity{0.700000}%
\pgfsetdash{}{0pt}%
\pgfpathmoveto{\pgfqpoint{3.969105in}{0.550000in}}%
\pgfpathlineto{\pgfqpoint{3.996626in}{0.550000in}}%
\pgfpathlineto{\pgfqpoint{3.996626in}{0.550000in}}%
\pgfpathlineto{\pgfqpoint{3.969105in}{0.550000in}}%
\pgfpathlineto{\pgfqpoint{3.969105in}{0.550000in}}%
\pgfpathclose%
\pgfusepath{fill}%
\end{pgfscope}%
\begin{pgfscope}%
\pgfpathrectangle{\pgfqpoint{1.250000in}{0.550000in}}{\pgfqpoint{7.750000in}{3.850000in}}%
\pgfusepath{clip}%
\pgfsetbuttcap%
\pgfsetmiterjoin%
\definecolor{currentfill}{rgb}{0.501961,0.501961,0.501961}%
\pgfsetfillcolor{currentfill}%
\pgfsetfillopacity{0.700000}%
\pgfsetlinewidth{0.000000pt}%
\definecolor{currentstroke}{rgb}{0.000000,0.000000,0.000000}%
\pgfsetstrokecolor{currentstroke}%
\pgfsetstrokeopacity{0.700000}%
\pgfsetdash{}{0pt}%
\pgfpathmoveto{\pgfqpoint{3.996626in}{0.550000in}}%
\pgfpathlineto{\pgfqpoint{4.024148in}{0.550000in}}%
\pgfpathlineto{\pgfqpoint{4.024148in}{0.550000in}}%
\pgfpathlineto{\pgfqpoint{3.996626in}{0.550000in}}%
\pgfpathlineto{\pgfqpoint{3.996626in}{0.550000in}}%
\pgfpathclose%
\pgfusepath{fill}%
\end{pgfscope}%
\begin{pgfscope}%
\pgfpathrectangle{\pgfqpoint{1.250000in}{0.550000in}}{\pgfqpoint{7.750000in}{3.850000in}}%
\pgfusepath{clip}%
\pgfsetbuttcap%
\pgfsetmiterjoin%
\definecolor{currentfill}{rgb}{0.501961,0.501961,0.501961}%
\pgfsetfillcolor{currentfill}%
\pgfsetfillopacity{0.700000}%
\pgfsetlinewidth{0.000000pt}%
\definecolor{currentstroke}{rgb}{0.000000,0.000000,0.000000}%
\pgfsetstrokecolor{currentstroke}%
\pgfsetstrokeopacity{0.700000}%
\pgfsetdash{}{0pt}%
\pgfpathmoveto{\pgfqpoint{4.024148in}{0.550000in}}%
\pgfpathlineto{\pgfqpoint{4.051669in}{0.550000in}}%
\pgfpathlineto{\pgfqpoint{4.051669in}{1.099848in}}%
\pgfpathlineto{\pgfqpoint{4.024148in}{1.099848in}}%
\pgfpathlineto{\pgfqpoint{4.024148in}{0.550000in}}%
\pgfpathclose%
\pgfusepath{fill}%
\end{pgfscope}%
\begin{pgfscope}%
\pgfpathrectangle{\pgfqpoint{1.250000in}{0.550000in}}{\pgfqpoint{7.750000in}{3.850000in}}%
\pgfusepath{clip}%
\pgfsetbuttcap%
\pgfsetmiterjoin%
\definecolor{currentfill}{rgb}{0.501961,0.501961,0.501961}%
\pgfsetfillcolor{currentfill}%
\pgfsetfillopacity{0.700000}%
\pgfsetlinewidth{0.000000pt}%
\definecolor{currentstroke}{rgb}{0.000000,0.000000,0.000000}%
\pgfsetstrokecolor{currentstroke}%
\pgfsetstrokeopacity{0.700000}%
\pgfsetdash{}{0pt}%
\pgfpathmoveto{\pgfqpoint{4.051669in}{0.550000in}}%
\pgfpathlineto{\pgfqpoint{4.079190in}{0.550000in}}%
\pgfpathlineto{\pgfqpoint{4.079190in}{0.550000in}}%
\pgfpathlineto{\pgfqpoint{4.051669in}{0.550000in}}%
\pgfpathlineto{\pgfqpoint{4.051669in}{0.550000in}}%
\pgfpathclose%
\pgfusepath{fill}%
\end{pgfscope}%
\begin{pgfscope}%
\pgfpathrectangle{\pgfqpoint{1.250000in}{0.550000in}}{\pgfqpoint{7.750000in}{3.850000in}}%
\pgfusepath{clip}%
\pgfsetbuttcap%
\pgfsetmiterjoin%
\definecolor{currentfill}{rgb}{0.501961,0.501961,0.501961}%
\pgfsetfillcolor{currentfill}%
\pgfsetfillopacity{0.700000}%
\pgfsetlinewidth{0.000000pt}%
\definecolor{currentstroke}{rgb}{0.000000,0.000000,0.000000}%
\pgfsetstrokecolor{currentstroke}%
\pgfsetstrokeopacity{0.700000}%
\pgfsetdash{}{0pt}%
\pgfpathmoveto{\pgfqpoint{4.079190in}{0.550000in}}%
\pgfpathlineto{\pgfqpoint{4.106712in}{0.550000in}}%
\pgfpathlineto{\pgfqpoint{4.106712in}{0.550000in}}%
\pgfpathlineto{\pgfqpoint{4.079190in}{0.550000in}}%
\pgfpathlineto{\pgfqpoint{4.079190in}{0.550000in}}%
\pgfpathclose%
\pgfusepath{fill}%
\end{pgfscope}%
\begin{pgfscope}%
\pgfpathrectangle{\pgfqpoint{1.250000in}{0.550000in}}{\pgfqpoint{7.750000in}{3.850000in}}%
\pgfusepath{clip}%
\pgfsetbuttcap%
\pgfsetmiterjoin%
\definecolor{currentfill}{rgb}{0.501961,0.501961,0.501961}%
\pgfsetfillcolor{currentfill}%
\pgfsetfillopacity{0.700000}%
\pgfsetlinewidth{0.000000pt}%
\definecolor{currentstroke}{rgb}{0.000000,0.000000,0.000000}%
\pgfsetstrokecolor{currentstroke}%
\pgfsetstrokeopacity{0.700000}%
\pgfsetdash{}{0pt}%
\pgfpathmoveto{\pgfqpoint{4.106712in}{0.550000in}}%
\pgfpathlineto{\pgfqpoint{4.134233in}{0.550000in}}%
\pgfpathlineto{\pgfqpoint{4.134233in}{0.550000in}}%
\pgfpathlineto{\pgfqpoint{4.106712in}{0.550000in}}%
\pgfpathlineto{\pgfqpoint{4.106712in}{0.550000in}}%
\pgfpathclose%
\pgfusepath{fill}%
\end{pgfscope}%
\begin{pgfscope}%
\pgfpathrectangle{\pgfqpoint{1.250000in}{0.550000in}}{\pgfqpoint{7.750000in}{3.850000in}}%
\pgfusepath{clip}%
\pgfsetbuttcap%
\pgfsetmiterjoin%
\definecolor{currentfill}{rgb}{0.501961,0.501961,0.501961}%
\pgfsetfillcolor{currentfill}%
\pgfsetfillopacity{0.700000}%
\pgfsetlinewidth{0.000000pt}%
\definecolor{currentstroke}{rgb}{0.000000,0.000000,0.000000}%
\pgfsetstrokecolor{currentstroke}%
\pgfsetstrokeopacity{0.700000}%
\pgfsetdash{}{0pt}%
\pgfpathmoveto{\pgfqpoint{4.134233in}{0.550000in}}%
\pgfpathlineto{\pgfqpoint{4.161754in}{0.550000in}}%
\pgfpathlineto{\pgfqpoint{4.161754in}{1.064902in}}%
\pgfpathlineto{\pgfqpoint{4.134233in}{1.064902in}}%
\pgfpathlineto{\pgfqpoint{4.134233in}{0.550000in}}%
\pgfpathclose%
\pgfusepath{fill}%
\end{pgfscope}%
\begin{pgfscope}%
\pgfpathrectangle{\pgfqpoint{1.250000in}{0.550000in}}{\pgfqpoint{7.750000in}{3.850000in}}%
\pgfusepath{clip}%
\pgfsetbuttcap%
\pgfsetmiterjoin%
\definecolor{currentfill}{rgb}{0.501961,0.501961,0.501961}%
\pgfsetfillcolor{currentfill}%
\pgfsetfillopacity{0.700000}%
\pgfsetlinewidth{0.000000pt}%
\definecolor{currentstroke}{rgb}{0.000000,0.000000,0.000000}%
\pgfsetstrokecolor{currentstroke}%
\pgfsetstrokeopacity{0.700000}%
\pgfsetdash{}{0pt}%
\pgfpathmoveto{\pgfqpoint{4.161754in}{0.550000in}}%
\pgfpathlineto{\pgfqpoint{4.189276in}{0.550000in}}%
\pgfpathlineto{\pgfqpoint{4.189276in}{0.550000in}}%
\pgfpathlineto{\pgfqpoint{4.161754in}{0.550000in}}%
\pgfpathlineto{\pgfqpoint{4.161754in}{0.550000in}}%
\pgfpathclose%
\pgfusepath{fill}%
\end{pgfscope}%
\begin{pgfscope}%
\pgfpathrectangle{\pgfqpoint{1.250000in}{0.550000in}}{\pgfqpoint{7.750000in}{3.850000in}}%
\pgfusepath{clip}%
\pgfsetbuttcap%
\pgfsetmiterjoin%
\definecolor{currentfill}{rgb}{0.501961,0.501961,0.501961}%
\pgfsetfillcolor{currentfill}%
\pgfsetfillopacity{0.700000}%
\pgfsetlinewidth{0.000000pt}%
\definecolor{currentstroke}{rgb}{0.000000,0.000000,0.000000}%
\pgfsetstrokecolor{currentstroke}%
\pgfsetstrokeopacity{0.700000}%
\pgfsetdash{}{0pt}%
\pgfpathmoveto{\pgfqpoint{4.189276in}{0.550000in}}%
\pgfpathlineto{\pgfqpoint{4.216797in}{0.550000in}}%
\pgfpathlineto{\pgfqpoint{4.216797in}{0.550000in}}%
\pgfpathlineto{\pgfqpoint{4.189276in}{0.550000in}}%
\pgfpathlineto{\pgfqpoint{4.189276in}{0.550000in}}%
\pgfpathclose%
\pgfusepath{fill}%
\end{pgfscope}%
\begin{pgfscope}%
\pgfpathrectangle{\pgfqpoint{1.250000in}{0.550000in}}{\pgfqpoint{7.750000in}{3.850000in}}%
\pgfusepath{clip}%
\pgfsetbuttcap%
\pgfsetmiterjoin%
\definecolor{currentfill}{rgb}{0.501961,0.501961,0.501961}%
\pgfsetfillcolor{currentfill}%
\pgfsetfillopacity{0.700000}%
\pgfsetlinewidth{0.000000pt}%
\definecolor{currentstroke}{rgb}{0.000000,0.000000,0.000000}%
\pgfsetstrokecolor{currentstroke}%
\pgfsetstrokeopacity{0.700000}%
\pgfsetdash{}{0pt}%
\pgfpathmoveto{\pgfqpoint{4.216797in}{0.550000in}}%
\pgfpathlineto{\pgfqpoint{4.244318in}{0.550000in}}%
\pgfpathlineto{\pgfqpoint{4.244318in}{1.074327in}}%
\pgfpathlineto{\pgfqpoint{4.216797in}{1.074327in}}%
\pgfpathlineto{\pgfqpoint{4.216797in}{0.550000in}}%
\pgfpathclose%
\pgfusepath{fill}%
\end{pgfscope}%
\begin{pgfscope}%
\pgfpathrectangle{\pgfqpoint{1.250000in}{0.550000in}}{\pgfqpoint{7.750000in}{3.850000in}}%
\pgfusepath{clip}%
\pgfsetbuttcap%
\pgfsetmiterjoin%
\definecolor{currentfill}{rgb}{0.501961,0.501961,0.501961}%
\pgfsetfillcolor{currentfill}%
\pgfsetfillopacity{0.700000}%
\pgfsetlinewidth{0.000000pt}%
\definecolor{currentstroke}{rgb}{0.000000,0.000000,0.000000}%
\pgfsetstrokecolor{currentstroke}%
\pgfsetstrokeopacity{0.700000}%
\pgfsetdash{}{0pt}%
\pgfpathmoveto{\pgfqpoint{4.244318in}{0.550000in}}%
\pgfpathlineto{\pgfqpoint{4.271839in}{0.550000in}}%
\pgfpathlineto{\pgfqpoint{4.271839in}{0.550000in}}%
\pgfpathlineto{\pgfqpoint{4.244318in}{0.550000in}}%
\pgfpathlineto{\pgfqpoint{4.244318in}{0.550000in}}%
\pgfpathclose%
\pgfusepath{fill}%
\end{pgfscope}%
\begin{pgfscope}%
\pgfpathrectangle{\pgfqpoint{1.250000in}{0.550000in}}{\pgfqpoint{7.750000in}{3.850000in}}%
\pgfusepath{clip}%
\pgfsetbuttcap%
\pgfsetmiterjoin%
\definecolor{currentfill}{rgb}{0.501961,0.501961,0.501961}%
\pgfsetfillcolor{currentfill}%
\pgfsetfillopacity{0.700000}%
\pgfsetlinewidth{0.000000pt}%
\definecolor{currentstroke}{rgb}{0.000000,0.000000,0.000000}%
\pgfsetstrokecolor{currentstroke}%
\pgfsetstrokeopacity{0.700000}%
\pgfsetdash{}{0pt}%
\pgfpathmoveto{\pgfqpoint{4.271839in}{0.550000in}}%
\pgfpathlineto{\pgfqpoint{4.299361in}{0.550000in}}%
\pgfpathlineto{\pgfqpoint{4.299361in}{0.550000in}}%
\pgfpathlineto{\pgfqpoint{4.271839in}{0.550000in}}%
\pgfpathlineto{\pgfqpoint{4.271839in}{0.550000in}}%
\pgfpathclose%
\pgfusepath{fill}%
\end{pgfscope}%
\begin{pgfscope}%
\pgfpathrectangle{\pgfqpoint{1.250000in}{0.550000in}}{\pgfqpoint{7.750000in}{3.850000in}}%
\pgfusepath{clip}%
\pgfsetbuttcap%
\pgfsetmiterjoin%
\definecolor{currentfill}{rgb}{0.501961,0.501961,0.501961}%
\pgfsetfillcolor{currentfill}%
\pgfsetfillopacity{0.700000}%
\pgfsetlinewidth{0.000000pt}%
\definecolor{currentstroke}{rgb}{0.000000,0.000000,0.000000}%
\pgfsetstrokecolor{currentstroke}%
\pgfsetstrokeopacity{0.700000}%
\pgfsetdash{}{0pt}%
\pgfpathmoveto{\pgfqpoint{4.299361in}{0.550000in}}%
\pgfpathlineto{\pgfqpoint{4.326882in}{0.550000in}}%
\pgfpathlineto{\pgfqpoint{4.326882in}{1.044312in}}%
\pgfpathlineto{\pgfqpoint{4.299361in}{1.044312in}}%
\pgfpathlineto{\pgfqpoint{4.299361in}{0.550000in}}%
\pgfpathclose%
\pgfusepath{fill}%
\end{pgfscope}%
\begin{pgfscope}%
\pgfpathrectangle{\pgfqpoint{1.250000in}{0.550000in}}{\pgfqpoint{7.750000in}{3.850000in}}%
\pgfusepath{clip}%
\pgfsetbuttcap%
\pgfsetmiterjoin%
\definecolor{currentfill}{rgb}{0.501961,0.501961,0.501961}%
\pgfsetfillcolor{currentfill}%
\pgfsetfillopacity{0.700000}%
\pgfsetlinewidth{0.000000pt}%
\definecolor{currentstroke}{rgb}{0.000000,0.000000,0.000000}%
\pgfsetstrokecolor{currentstroke}%
\pgfsetstrokeopacity{0.700000}%
\pgfsetdash{}{0pt}%
\pgfpathmoveto{\pgfqpoint{4.326882in}{0.550000in}}%
\pgfpathlineto{\pgfqpoint{4.354403in}{0.550000in}}%
\pgfpathlineto{\pgfqpoint{4.354403in}{0.550000in}}%
\pgfpathlineto{\pgfqpoint{4.326882in}{0.550000in}}%
\pgfpathlineto{\pgfqpoint{4.326882in}{0.550000in}}%
\pgfpathclose%
\pgfusepath{fill}%
\end{pgfscope}%
\begin{pgfscope}%
\pgfpathrectangle{\pgfqpoint{1.250000in}{0.550000in}}{\pgfqpoint{7.750000in}{3.850000in}}%
\pgfusepath{clip}%
\pgfsetbuttcap%
\pgfsetmiterjoin%
\definecolor{currentfill}{rgb}{0.501961,0.501961,0.501961}%
\pgfsetfillcolor{currentfill}%
\pgfsetfillopacity{0.700000}%
\pgfsetlinewidth{0.000000pt}%
\definecolor{currentstroke}{rgb}{0.000000,0.000000,0.000000}%
\pgfsetstrokecolor{currentstroke}%
\pgfsetstrokeopacity{0.700000}%
\pgfsetdash{}{0pt}%
\pgfpathmoveto{\pgfqpoint{4.354403in}{0.550000in}}%
\pgfpathlineto{\pgfqpoint{4.381925in}{0.550000in}}%
\pgfpathlineto{\pgfqpoint{4.381925in}{0.550000in}}%
\pgfpathlineto{\pgfqpoint{4.354403in}{0.550000in}}%
\pgfpathlineto{\pgfqpoint{4.354403in}{0.550000in}}%
\pgfpathclose%
\pgfusepath{fill}%
\end{pgfscope}%
\begin{pgfscope}%
\pgfpathrectangle{\pgfqpoint{1.250000in}{0.550000in}}{\pgfqpoint{7.750000in}{3.850000in}}%
\pgfusepath{clip}%
\pgfsetbuttcap%
\pgfsetmiterjoin%
\definecolor{currentfill}{rgb}{0.501961,0.501961,0.501961}%
\pgfsetfillcolor{currentfill}%
\pgfsetfillopacity{0.700000}%
\pgfsetlinewidth{0.000000pt}%
\definecolor{currentstroke}{rgb}{0.000000,0.000000,0.000000}%
\pgfsetstrokecolor{currentstroke}%
\pgfsetstrokeopacity{0.700000}%
\pgfsetdash{}{0pt}%
\pgfpathmoveto{\pgfqpoint{4.381925in}{0.550000in}}%
\pgfpathlineto{\pgfqpoint{4.409446in}{0.550000in}}%
\pgfpathlineto{\pgfqpoint{4.409446in}{1.041267in}}%
\pgfpathlineto{\pgfqpoint{4.381925in}{1.041267in}}%
\pgfpathlineto{\pgfqpoint{4.381925in}{0.550000in}}%
\pgfpathclose%
\pgfusepath{fill}%
\end{pgfscope}%
\begin{pgfscope}%
\pgfpathrectangle{\pgfqpoint{1.250000in}{0.550000in}}{\pgfqpoint{7.750000in}{3.850000in}}%
\pgfusepath{clip}%
\pgfsetbuttcap%
\pgfsetmiterjoin%
\definecolor{currentfill}{rgb}{0.501961,0.501961,0.501961}%
\pgfsetfillcolor{currentfill}%
\pgfsetfillopacity{0.700000}%
\pgfsetlinewidth{0.000000pt}%
\definecolor{currentstroke}{rgb}{0.000000,0.000000,0.000000}%
\pgfsetstrokecolor{currentstroke}%
\pgfsetstrokeopacity{0.700000}%
\pgfsetdash{}{0pt}%
\pgfpathmoveto{\pgfqpoint{4.409446in}{0.550000in}}%
\pgfpathlineto{\pgfqpoint{4.436967in}{0.550000in}}%
\pgfpathlineto{\pgfqpoint{4.436967in}{0.550000in}}%
\pgfpathlineto{\pgfqpoint{4.409446in}{0.550000in}}%
\pgfpathlineto{\pgfqpoint{4.409446in}{0.550000in}}%
\pgfpathclose%
\pgfusepath{fill}%
\end{pgfscope}%
\begin{pgfscope}%
\pgfpathrectangle{\pgfqpoint{1.250000in}{0.550000in}}{\pgfqpoint{7.750000in}{3.850000in}}%
\pgfusepath{clip}%
\pgfsetbuttcap%
\pgfsetmiterjoin%
\definecolor{currentfill}{rgb}{0.501961,0.501961,0.501961}%
\pgfsetfillcolor{currentfill}%
\pgfsetfillopacity{0.700000}%
\pgfsetlinewidth{0.000000pt}%
\definecolor{currentstroke}{rgb}{0.000000,0.000000,0.000000}%
\pgfsetstrokecolor{currentstroke}%
\pgfsetstrokeopacity{0.700000}%
\pgfsetdash{}{0pt}%
\pgfpathmoveto{\pgfqpoint{4.436967in}{0.550000in}}%
\pgfpathlineto{\pgfqpoint{4.464489in}{0.550000in}}%
\pgfpathlineto{\pgfqpoint{4.464489in}{0.550000in}}%
\pgfpathlineto{\pgfqpoint{4.436967in}{0.550000in}}%
\pgfpathlineto{\pgfqpoint{4.436967in}{0.550000in}}%
\pgfpathclose%
\pgfusepath{fill}%
\end{pgfscope}%
\begin{pgfscope}%
\pgfpathrectangle{\pgfqpoint{1.250000in}{0.550000in}}{\pgfqpoint{7.750000in}{3.850000in}}%
\pgfusepath{clip}%
\pgfsetbuttcap%
\pgfsetmiterjoin%
\definecolor{currentfill}{rgb}{0.501961,0.501961,0.501961}%
\pgfsetfillcolor{currentfill}%
\pgfsetfillopacity{0.700000}%
\pgfsetlinewidth{0.000000pt}%
\definecolor{currentstroke}{rgb}{0.000000,0.000000,0.000000}%
\pgfsetstrokecolor{currentstroke}%
\pgfsetstrokeopacity{0.700000}%
\pgfsetdash{}{0pt}%
\pgfpathmoveto{\pgfqpoint{4.464489in}{0.550000in}}%
\pgfpathlineto{\pgfqpoint{4.492010in}{0.550000in}}%
\pgfpathlineto{\pgfqpoint{4.492010in}{1.042282in}}%
\pgfpathlineto{\pgfqpoint{4.464489in}{1.042282in}}%
\pgfpathlineto{\pgfqpoint{4.464489in}{0.550000in}}%
\pgfpathclose%
\pgfusepath{fill}%
\end{pgfscope}%
\begin{pgfscope}%
\pgfpathrectangle{\pgfqpoint{1.250000in}{0.550000in}}{\pgfqpoint{7.750000in}{3.850000in}}%
\pgfusepath{clip}%
\pgfsetbuttcap%
\pgfsetmiterjoin%
\definecolor{currentfill}{rgb}{0.501961,0.501961,0.501961}%
\pgfsetfillcolor{currentfill}%
\pgfsetfillopacity{0.700000}%
\pgfsetlinewidth{0.000000pt}%
\definecolor{currentstroke}{rgb}{0.000000,0.000000,0.000000}%
\pgfsetstrokecolor{currentstroke}%
\pgfsetstrokeopacity{0.700000}%
\pgfsetdash{}{0pt}%
\pgfpathmoveto{\pgfqpoint{4.492010in}{0.550000in}}%
\pgfpathlineto{\pgfqpoint{4.519531in}{0.550000in}}%
\pgfpathlineto{\pgfqpoint{4.519531in}{0.550000in}}%
\pgfpathlineto{\pgfqpoint{4.492010in}{0.550000in}}%
\pgfpathlineto{\pgfqpoint{4.492010in}{0.550000in}}%
\pgfpathclose%
\pgfusepath{fill}%
\end{pgfscope}%
\begin{pgfscope}%
\pgfpathrectangle{\pgfqpoint{1.250000in}{0.550000in}}{\pgfqpoint{7.750000in}{3.850000in}}%
\pgfusepath{clip}%
\pgfsetbuttcap%
\pgfsetmiterjoin%
\definecolor{currentfill}{rgb}{0.501961,0.501961,0.501961}%
\pgfsetfillcolor{currentfill}%
\pgfsetfillopacity{0.700000}%
\pgfsetlinewidth{0.000000pt}%
\definecolor{currentstroke}{rgb}{0.000000,0.000000,0.000000}%
\pgfsetstrokecolor{currentstroke}%
\pgfsetstrokeopacity{0.700000}%
\pgfsetdash{}{0pt}%
\pgfpathmoveto{\pgfqpoint{4.519531in}{0.550000in}}%
\pgfpathlineto{\pgfqpoint{4.547053in}{0.550000in}}%
\pgfpathlineto{\pgfqpoint{4.547053in}{0.550000in}}%
\pgfpathlineto{\pgfqpoint{4.519531in}{0.550000in}}%
\pgfpathlineto{\pgfqpoint{4.519531in}{0.550000in}}%
\pgfpathclose%
\pgfusepath{fill}%
\end{pgfscope}%
\begin{pgfscope}%
\pgfpathrectangle{\pgfqpoint{1.250000in}{0.550000in}}{\pgfqpoint{7.750000in}{3.850000in}}%
\pgfusepath{clip}%
\pgfsetbuttcap%
\pgfsetmiterjoin%
\definecolor{currentfill}{rgb}{0.501961,0.501961,0.501961}%
\pgfsetfillcolor{currentfill}%
\pgfsetfillopacity{0.700000}%
\pgfsetlinewidth{0.000000pt}%
\definecolor{currentstroke}{rgb}{0.000000,0.000000,0.000000}%
\pgfsetstrokecolor{currentstroke}%
\pgfsetstrokeopacity{0.700000}%
\pgfsetdash{}{0pt}%
\pgfpathmoveto{\pgfqpoint{4.547053in}{0.550000in}}%
\pgfpathlineto{\pgfqpoint{4.574574in}{0.550000in}}%
\pgfpathlineto{\pgfqpoint{4.574574in}{1.032422in}}%
\pgfpathlineto{\pgfqpoint{4.547053in}{1.032422in}}%
\pgfpathlineto{\pgfqpoint{4.547053in}{0.550000in}}%
\pgfpathclose%
\pgfusepath{fill}%
\end{pgfscope}%
\begin{pgfscope}%
\pgfpathrectangle{\pgfqpoint{1.250000in}{0.550000in}}{\pgfqpoint{7.750000in}{3.850000in}}%
\pgfusepath{clip}%
\pgfsetbuttcap%
\pgfsetmiterjoin%
\definecolor{currentfill}{rgb}{0.501961,0.501961,0.501961}%
\pgfsetfillcolor{currentfill}%
\pgfsetfillopacity{0.700000}%
\pgfsetlinewidth{0.000000pt}%
\definecolor{currentstroke}{rgb}{0.000000,0.000000,0.000000}%
\pgfsetstrokecolor{currentstroke}%
\pgfsetstrokeopacity{0.700000}%
\pgfsetdash{}{0pt}%
\pgfpathmoveto{\pgfqpoint{4.574574in}{0.550000in}}%
\pgfpathlineto{\pgfqpoint{4.602095in}{0.550000in}}%
\pgfpathlineto{\pgfqpoint{4.602095in}{0.550000in}}%
\pgfpathlineto{\pgfqpoint{4.574574in}{0.550000in}}%
\pgfpathlineto{\pgfqpoint{4.574574in}{0.550000in}}%
\pgfpathclose%
\pgfusepath{fill}%
\end{pgfscope}%
\begin{pgfscope}%
\pgfpathrectangle{\pgfqpoint{1.250000in}{0.550000in}}{\pgfqpoint{7.750000in}{3.850000in}}%
\pgfusepath{clip}%
\pgfsetbuttcap%
\pgfsetmiterjoin%
\definecolor{currentfill}{rgb}{0.501961,0.501961,0.501961}%
\pgfsetfillcolor{currentfill}%
\pgfsetfillopacity{0.700000}%
\pgfsetlinewidth{0.000000pt}%
\definecolor{currentstroke}{rgb}{0.000000,0.000000,0.000000}%
\pgfsetstrokecolor{currentstroke}%
\pgfsetstrokeopacity{0.700000}%
\pgfsetdash{}{0pt}%
\pgfpathmoveto{\pgfqpoint{4.602095in}{0.550000in}}%
\pgfpathlineto{\pgfqpoint{4.629616in}{0.550000in}}%
\pgfpathlineto{\pgfqpoint{4.629616in}{0.550000in}}%
\pgfpathlineto{\pgfqpoint{4.602095in}{0.550000in}}%
\pgfpathlineto{\pgfqpoint{4.602095in}{0.550000in}}%
\pgfpathclose%
\pgfusepath{fill}%
\end{pgfscope}%
\begin{pgfscope}%
\pgfpathrectangle{\pgfqpoint{1.250000in}{0.550000in}}{\pgfqpoint{7.750000in}{3.850000in}}%
\pgfusepath{clip}%
\pgfsetbuttcap%
\pgfsetmiterjoin%
\definecolor{currentfill}{rgb}{0.501961,0.501961,0.501961}%
\pgfsetfillcolor{currentfill}%
\pgfsetfillopacity{0.700000}%
\pgfsetlinewidth{0.000000pt}%
\definecolor{currentstroke}{rgb}{0.000000,0.000000,0.000000}%
\pgfsetstrokecolor{currentstroke}%
\pgfsetstrokeopacity{0.700000}%
\pgfsetdash{}{0pt}%
\pgfpathmoveto{\pgfqpoint{4.629616in}{0.550000in}}%
\pgfpathlineto{\pgfqpoint{4.657138in}{0.550000in}}%
\pgfpathlineto{\pgfqpoint{4.657138in}{1.008351in}}%
\pgfpathlineto{\pgfqpoint{4.629616in}{1.008351in}}%
\pgfpathlineto{\pgfqpoint{4.629616in}{0.550000in}}%
\pgfpathclose%
\pgfusepath{fill}%
\end{pgfscope}%
\begin{pgfscope}%
\pgfpathrectangle{\pgfqpoint{1.250000in}{0.550000in}}{\pgfqpoint{7.750000in}{3.850000in}}%
\pgfusepath{clip}%
\pgfsetbuttcap%
\pgfsetmiterjoin%
\definecolor{currentfill}{rgb}{0.501961,0.501961,0.501961}%
\pgfsetfillcolor{currentfill}%
\pgfsetfillopacity{0.700000}%
\pgfsetlinewidth{0.000000pt}%
\definecolor{currentstroke}{rgb}{0.000000,0.000000,0.000000}%
\pgfsetstrokecolor{currentstroke}%
\pgfsetstrokeopacity{0.700000}%
\pgfsetdash{}{0pt}%
\pgfpathmoveto{\pgfqpoint{4.657138in}{0.550000in}}%
\pgfpathlineto{\pgfqpoint{4.684659in}{0.550000in}}%
\pgfpathlineto{\pgfqpoint{4.684659in}{0.550000in}}%
\pgfpathlineto{\pgfqpoint{4.657138in}{0.550000in}}%
\pgfpathlineto{\pgfqpoint{4.657138in}{0.550000in}}%
\pgfpathclose%
\pgfusepath{fill}%
\end{pgfscope}%
\begin{pgfscope}%
\pgfpathrectangle{\pgfqpoint{1.250000in}{0.550000in}}{\pgfqpoint{7.750000in}{3.850000in}}%
\pgfusepath{clip}%
\pgfsetbuttcap%
\pgfsetmiterjoin%
\definecolor{currentfill}{rgb}{0.501961,0.501961,0.501961}%
\pgfsetfillcolor{currentfill}%
\pgfsetfillopacity{0.700000}%
\pgfsetlinewidth{0.000000pt}%
\definecolor{currentstroke}{rgb}{0.000000,0.000000,0.000000}%
\pgfsetstrokecolor{currentstroke}%
\pgfsetstrokeopacity{0.700000}%
\pgfsetdash{}{0pt}%
\pgfpathmoveto{\pgfqpoint{4.684659in}{0.550000in}}%
\pgfpathlineto{\pgfqpoint{4.712180in}{0.550000in}}%
\pgfpathlineto{\pgfqpoint{4.712180in}{0.550000in}}%
\pgfpathlineto{\pgfqpoint{4.684659in}{0.550000in}}%
\pgfpathlineto{\pgfqpoint{4.684659in}{0.550000in}}%
\pgfpathclose%
\pgfusepath{fill}%
\end{pgfscope}%
\begin{pgfscope}%
\pgfpathrectangle{\pgfqpoint{1.250000in}{0.550000in}}{\pgfqpoint{7.750000in}{3.850000in}}%
\pgfusepath{clip}%
\pgfsetbuttcap%
\pgfsetmiterjoin%
\definecolor{currentfill}{rgb}{0.501961,0.501961,0.501961}%
\pgfsetfillcolor{currentfill}%
\pgfsetfillopacity{0.700000}%
\pgfsetlinewidth{0.000000pt}%
\definecolor{currentstroke}{rgb}{0.000000,0.000000,0.000000}%
\pgfsetstrokecolor{currentstroke}%
\pgfsetstrokeopacity{0.700000}%
\pgfsetdash{}{0pt}%
\pgfpathmoveto{\pgfqpoint{4.712180in}{0.550000in}}%
\pgfpathlineto{\pgfqpoint{4.739702in}{0.550000in}}%
\pgfpathlineto{\pgfqpoint{4.739702in}{1.021402in}}%
\pgfpathlineto{\pgfqpoint{4.712180in}{1.021402in}}%
\pgfpathlineto{\pgfqpoint{4.712180in}{0.550000in}}%
\pgfpathclose%
\pgfusepath{fill}%
\end{pgfscope}%
\begin{pgfscope}%
\pgfpathrectangle{\pgfqpoint{1.250000in}{0.550000in}}{\pgfqpoint{7.750000in}{3.850000in}}%
\pgfusepath{clip}%
\pgfsetbuttcap%
\pgfsetmiterjoin%
\definecolor{currentfill}{rgb}{0.501961,0.501961,0.501961}%
\pgfsetfillcolor{currentfill}%
\pgfsetfillopacity{0.700000}%
\pgfsetlinewidth{0.000000pt}%
\definecolor{currentstroke}{rgb}{0.000000,0.000000,0.000000}%
\pgfsetstrokecolor{currentstroke}%
\pgfsetstrokeopacity{0.700000}%
\pgfsetdash{}{0pt}%
\pgfpathmoveto{\pgfqpoint{4.739702in}{0.550000in}}%
\pgfpathlineto{\pgfqpoint{4.767223in}{0.550000in}}%
\pgfpathlineto{\pgfqpoint{4.767223in}{0.550000in}}%
\pgfpathlineto{\pgfqpoint{4.739702in}{0.550000in}}%
\pgfpathlineto{\pgfqpoint{4.739702in}{0.550000in}}%
\pgfpathclose%
\pgfusepath{fill}%
\end{pgfscope}%
\begin{pgfscope}%
\pgfpathrectangle{\pgfqpoint{1.250000in}{0.550000in}}{\pgfqpoint{7.750000in}{3.850000in}}%
\pgfusepath{clip}%
\pgfsetbuttcap%
\pgfsetmiterjoin%
\definecolor{currentfill}{rgb}{0.501961,0.501961,0.501961}%
\pgfsetfillcolor{currentfill}%
\pgfsetfillopacity{0.700000}%
\pgfsetlinewidth{0.000000pt}%
\definecolor{currentstroke}{rgb}{0.000000,0.000000,0.000000}%
\pgfsetstrokecolor{currentstroke}%
\pgfsetstrokeopacity{0.700000}%
\pgfsetdash{}{0pt}%
\pgfpathmoveto{\pgfqpoint{4.767223in}{0.550000in}}%
\pgfpathlineto{\pgfqpoint{4.794744in}{0.550000in}}%
\pgfpathlineto{\pgfqpoint{4.794744in}{0.550000in}}%
\pgfpathlineto{\pgfqpoint{4.767223in}{0.550000in}}%
\pgfpathlineto{\pgfqpoint{4.767223in}{0.550000in}}%
\pgfpathclose%
\pgfusepath{fill}%
\end{pgfscope}%
\begin{pgfscope}%
\pgfpathrectangle{\pgfqpoint{1.250000in}{0.550000in}}{\pgfqpoint{7.750000in}{3.850000in}}%
\pgfusepath{clip}%
\pgfsetbuttcap%
\pgfsetmiterjoin%
\definecolor{currentfill}{rgb}{0.501961,0.501961,0.501961}%
\pgfsetfillcolor{currentfill}%
\pgfsetfillopacity{0.700000}%
\pgfsetlinewidth{0.000000pt}%
\definecolor{currentstroke}{rgb}{0.000000,0.000000,0.000000}%
\pgfsetstrokecolor{currentstroke}%
\pgfsetstrokeopacity{0.700000}%
\pgfsetdash{}{0pt}%
\pgfpathmoveto{\pgfqpoint{4.794744in}{0.550000in}}%
\pgfpathlineto{\pgfqpoint{4.822266in}{0.550000in}}%
\pgfpathlineto{\pgfqpoint{4.822266in}{1.002986in}}%
\pgfpathlineto{\pgfqpoint{4.794744in}{1.002986in}}%
\pgfpathlineto{\pgfqpoint{4.794744in}{0.550000in}}%
\pgfpathclose%
\pgfusepath{fill}%
\end{pgfscope}%
\begin{pgfscope}%
\pgfpathrectangle{\pgfqpoint{1.250000in}{0.550000in}}{\pgfqpoint{7.750000in}{3.850000in}}%
\pgfusepath{clip}%
\pgfsetbuttcap%
\pgfsetmiterjoin%
\definecolor{currentfill}{rgb}{0.501961,0.501961,0.501961}%
\pgfsetfillcolor{currentfill}%
\pgfsetfillopacity{0.700000}%
\pgfsetlinewidth{0.000000pt}%
\definecolor{currentstroke}{rgb}{0.000000,0.000000,0.000000}%
\pgfsetstrokecolor{currentstroke}%
\pgfsetstrokeopacity{0.700000}%
\pgfsetdash{}{0pt}%
\pgfpathmoveto{\pgfqpoint{4.822266in}{0.550000in}}%
\pgfpathlineto{\pgfqpoint{4.849787in}{0.550000in}}%
\pgfpathlineto{\pgfqpoint{4.849787in}{0.550000in}}%
\pgfpathlineto{\pgfqpoint{4.822266in}{0.550000in}}%
\pgfpathlineto{\pgfqpoint{4.822266in}{0.550000in}}%
\pgfpathclose%
\pgfusepath{fill}%
\end{pgfscope}%
\begin{pgfscope}%
\pgfpathrectangle{\pgfqpoint{1.250000in}{0.550000in}}{\pgfqpoint{7.750000in}{3.850000in}}%
\pgfusepath{clip}%
\pgfsetbuttcap%
\pgfsetmiterjoin%
\definecolor{currentfill}{rgb}{0.501961,0.501961,0.501961}%
\pgfsetfillcolor{currentfill}%
\pgfsetfillopacity{0.700000}%
\pgfsetlinewidth{0.000000pt}%
\definecolor{currentstroke}{rgb}{0.000000,0.000000,0.000000}%
\pgfsetstrokecolor{currentstroke}%
\pgfsetstrokeopacity{0.700000}%
\pgfsetdash{}{0pt}%
\pgfpathmoveto{\pgfqpoint{4.849787in}{0.550000in}}%
\pgfpathlineto{\pgfqpoint{4.877308in}{0.550000in}}%
\pgfpathlineto{\pgfqpoint{4.877308in}{0.990516in}}%
\pgfpathlineto{\pgfqpoint{4.849787in}{0.990516in}}%
\pgfpathlineto{\pgfqpoint{4.849787in}{0.550000in}}%
\pgfpathclose%
\pgfusepath{fill}%
\end{pgfscope}%
\begin{pgfscope}%
\pgfpathrectangle{\pgfqpoint{1.250000in}{0.550000in}}{\pgfqpoint{7.750000in}{3.850000in}}%
\pgfusepath{clip}%
\pgfsetbuttcap%
\pgfsetmiterjoin%
\definecolor{currentfill}{rgb}{0.501961,0.501961,0.501961}%
\pgfsetfillcolor{currentfill}%
\pgfsetfillopacity{0.700000}%
\pgfsetlinewidth{0.000000pt}%
\definecolor{currentstroke}{rgb}{0.000000,0.000000,0.000000}%
\pgfsetstrokecolor{currentstroke}%
\pgfsetstrokeopacity{0.700000}%
\pgfsetdash{}{0pt}%
\pgfpathmoveto{\pgfqpoint{4.877308in}{0.550000in}}%
\pgfpathlineto{\pgfqpoint{4.904830in}{0.550000in}}%
\pgfpathlineto{\pgfqpoint{4.904830in}{0.550000in}}%
\pgfpathlineto{\pgfqpoint{4.877308in}{0.550000in}}%
\pgfpathlineto{\pgfqpoint{4.877308in}{0.550000in}}%
\pgfpathclose%
\pgfusepath{fill}%
\end{pgfscope}%
\begin{pgfscope}%
\pgfpathrectangle{\pgfqpoint{1.250000in}{0.550000in}}{\pgfqpoint{7.750000in}{3.850000in}}%
\pgfusepath{clip}%
\pgfsetbuttcap%
\pgfsetmiterjoin%
\definecolor{currentfill}{rgb}{0.501961,0.501961,0.501961}%
\pgfsetfillcolor{currentfill}%
\pgfsetfillopacity{0.700000}%
\pgfsetlinewidth{0.000000pt}%
\definecolor{currentstroke}{rgb}{0.000000,0.000000,0.000000}%
\pgfsetstrokecolor{currentstroke}%
\pgfsetstrokeopacity{0.700000}%
\pgfsetdash{}{0pt}%
\pgfpathmoveto{\pgfqpoint{4.904830in}{0.550000in}}%
\pgfpathlineto{\pgfqpoint{4.932351in}{0.550000in}}%
\pgfpathlineto{\pgfqpoint{4.932351in}{0.550000in}}%
\pgfpathlineto{\pgfqpoint{4.904830in}{0.550000in}}%
\pgfpathlineto{\pgfqpoint{4.904830in}{0.550000in}}%
\pgfpathclose%
\pgfusepath{fill}%
\end{pgfscope}%
\begin{pgfscope}%
\pgfpathrectangle{\pgfqpoint{1.250000in}{0.550000in}}{\pgfqpoint{7.750000in}{3.850000in}}%
\pgfusepath{clip}%
\pgfsetbuttcap%
\pgfsetmiterjoin%
\definecolor{currentfill}{rgb}{0.501961,0.501961,0.501961}%
\pgfsetfillcolor{currentfill}%
\pgfsetfillopacity{0.700000}%
\pgfsetlinewidth{0.000000pt}%
\definecolor{currentstroke}{rgb}{0.000000,0.000000,0.000000}%
\pgfsetstrokecolor{currentstroke}%
\pgfsetstrokeopacity{0.700000}%
\pgfsetdash{}{0pt}%
\pgfpathmoveto{\pgfqpoint{4.932351in}{0.550000in}}%
\pgfpathlineto{\pgfqpoint{4.959872in}{0.550000in}}%
\pgfpathlineto{\pgfqpoint{4.959872in}{0.968766in}}%
\pgfpathlineto{\pgfqpoint{4.932351in}{0.968766in}}%
\pgfpathlineto{\pgfqpoint{4.932351in}{0.550000in}}%
\pgfpathclose%
\pgfusepath{fill}%
\end{pgfscope}%
\begin{pgfscope}%
\pgfpathrectangle{\pgfqpoint{1.250000in}{0.550000in}}{\pgfqpoint{7.750000in}{3.850000in}}%
\pgfusepath{clip}%
\pgfsetbuttcap%
\pgfsetmiterjoin%
\definecolor{currentfill}{rgb}{0.501961,0.501961,0.501961}%
\pgfsetfillcolor{currentfill}%
\pgfsetfillopacity{0.700000}%
\pgfsetlinewidth{0.000000pt}%
\definecolor{currentstroke}{rgb}{0.000000,0.000000,0.000000}%
\pgfsetstrokecolor{currentstroke}%
\pgfsetstrokeopacity{0.700000}%
\pgfsetdash{}{0pt}%
\pgfpathmoveto{\pgfqpoint{4.959872in}{0.550000in}}%
\pgfpathlineto{\pgfqpoint{4.987393in}{0.550000in}}%
\pgfpathlineto{\pgfqpoint{4.987393in}{0.550000in}}%
\pgfpathlineto{\pgfqpoint{4.959872in}{0.550000in}}%
\pgfpathlineto{\pgfqpoint{4.959872in}{0.550000in}}%
\pgfpathclose%
\pgfusepath{fill}%
\end{pgfscope}%
\begin{pgfscope}%
\pgfpathrectangle{\pgfqpoint{1.250000in}{0.550000in}}{\pgfqpoint{7.750000in}{3.850000in}}%
\pgfusepath{clip}%
\pgfsetbuttcap%
\pgfsetmiterjoin%
\definecolor{currentfill}{rgb}{0.501961,0.501961,0.501961}%
\pgfsetfillcolor{currentfill}%
\pgfsetfillopacity{0.700000}%
\pgfsetlinewidth{0.000000pt}%
\definecolor{currentstroke}{rgb}{0.000000,0.000000,0.000000}%
\pgfsetstrokecolor{currentstroke}%
\pgfsetstrokeopacity{0.700000}%
\pgfsetdash{}{0pt}%
\pgfpathmoveto{\pgfqpoint{4.987393in}{0.550000in}}%
\pgfpathlineto{\pgfqpoint{5.014915in}{0.550000in}}%
\pgfpathlineto{\pgfqpoint{5.014915in}{0.550000in}}%
\pgfpathlineto{\pgfqpoint{4.987393in}{0.550000in}}%
\pgfpathlineto{\pgfqpoint{4.987393in}{0.550000in}}%
\pgfpathclose%
\pgfusepath{fill}%
\end{pgfscope}%
\begin{pgfscope}%
\pgfpathrectangle{\pgfqpoint{1.250000in}{0.550000in}}{\pgfqpoint{7.750000in}{3.850000in}}%
\pgfusepath{clip}%
\pgfsetbuttcap%
\pgfsetmiterjoin%
\definecolor{currentfill}{rgb}{0.501961,0.501961,0.501961}%
\pgfsetfillcolor{currentfill}%
\pgfsetfillopacity{0.700000}%
\pgfsetlinewidth{0.000000pt}%
\definecolor{currentstroke}{rgb}{0.000000,0.000000,0.000000}%
\pgfsetstrokecolor{currentstroke}%
\pgfsetstrokeopacity{0.700000}%
\pgfsetdash{}{0pt}%
\pgfpathmoveto{\pgfqpoint{5.014915in}{0.550000in}}%
\pgfpathlineto{\pgfqpoint{5.042436in}{0.550000in}}%
\pgfpathlineto{\pgfqpoint{5.042436in}{0.967461in}}%
\pgfpathlineto{\pgfqpoint{5.014915in}{0.967461in}}%
\pgfpathlineto{\pgfqpoint{5.014915in}{0.550000in}}%
\pgfpathclose%
\pgfusepath{fill}%
\end{pgfscope}%
\begin{pgfscope}%
\pgfpathrectangle{\pgfqpoint{1.250000in}{0.550000in}}{\pgfqpoint{7.750000in}{3.850000in}}%
\pgfusepath{clip}%
\pgfsetbuttcap%
\pgfsetmiterjoin%
\definecolor{currentfill}{rgb}{0.501961,0.501961,0.501961}%
\pgfsetfillcolor{currentfill}%
\pgfsetfillopacity{0.700000}%
\pgfsetlinewidth{0.000000pt}%
\definecolor{currentstroke}{rgb}{0.000000,0.000000,0.000000}%
\pgfsetstrokecolor{currentstroke}%
\pgfsetstrokeopacity{0.700000}%
\pgfsetdash{}{0pt}%
\pgfpathmoveto{\pgfqpoint{5.042436in}{0.550000in}}%
\pgfpathlineto{\pgfqpoint{5.069957in}{0.550000in}}%
\pgfpathlineto{\pgfqpoint{5.069957in}{0.550000in}}%
\pgfpathlineto{\pgfqpoint{5.042436in}{0.550000in}}%
\pgfpathlineto{\pgfqpoint{5.042436in}{0.550000in}}%
\pgfpathclose%
\pgfusepath{fill}%
\end{pgfscope}%
\begin{pgfscope}%
\pgfpathrectangle{\pgfqpoint{1.250000in}{0.550000in}}{\pgfqpoint{7.750000in}{3.850000in}}%
\pgfusepath{clip}%
\pgfsetbuttcap%
\pgfsetmiterjoin%
\definecolor{currentfill}{rgb}{0.501961,0.501961,0.501961}%
\pgfsetfillcolor{currentfill}%
\pgfsetfillopacity{0.700000}%
\pgfsetlinewidth{0.000000pt}%
\definecolor{currentstroke}{rgb}{0.000000,0.000000,0.000000}%
\pgfsetstrokecolor{currentstroke}%
\pgfsetstrokeopacity{0.700000}%
\pgfsetdash{}{0pt}%
\pgfpathmoveto{\pgfqpoint{5.069957in}{0.550000in}}%
\pgfpathlineto{\pgfqpoint{5.097479in}{0.550000in}}%
\pgfpathlineto{\pgfqpoint{5.097479in}{0.950206in}}%
\pgfpathlineto{\pgfqpoint{5.069957in}{0.950206in}}%
\pgfpathlineto{\pgfqpoint{5.069957in}{0.550000in}}%
\pgfpathclose%
\pgfusepath{fill}%
\end{pgfscope}%
\begin{pgfscope}%
\pgfpathrectangle{\pgfqpoint{1.250000in}{0.550000in}}{\pgfqpoint{7.750000in}{3.850000in}}%
\pgfusepath{clip}%
\pgfsetbuttcap%
\pgfsetmiterjoin%
\definecolor{currentfill}{rgb}{0.501961,0.501961,0.501961}%
\pgfsetfillcolor{currentfill}%
\pgfsetfillopacity{0.700000}%
\pgfsetlinewidth{0.000000pt}%
\definecolor{currentstroke}{rgb}{0.000000,0.000000,0.000000}%
\pgfsetstrokecolor{currentstroke}%
\pgfsetstrokeopacity{0.700000}%
\pgfsetdash{}{0pt}%
\pgfpathmoveto{\pgfqpoint{5.097479in}{0.550000in}}%
\pgfpathlineto{\pgfqpoint{5.125000in}{0.550000in}}%
\pgfpathlineto{\pgfqpoint{5.125000in}{0.550000in}}%
\pgfpathlineto{\pgfqpoint{5.097479in}{0.550000in}}%
\pgfpathlineto{\pgfqpoint{5.097479in}{0.550000in}}%
\pgfpathclose%
\pgfusepath{fill}%
\end{pgfscope}%
\begin{pgfscope}%
\pgfpathrectangle{\pgfqpoint{1.250000in}{0.550000in}}{\pgfqpoint{7.750000in}{3.850000in}}%
\pgfusepath{clip}%
\pgfsetbuttcap%
\pgfsetmiterjoin%
\definecolor{currentfill}{rgb}{0.501961,0.501961,0.501961}%
\pgfsetfillcolor{currentfill}%
\pgfsetfillopacity{0.700000}%
\pgfsetlinewidth{0.000000pt}%
\definecolor{currentstroke}{rgb}{0.000000,0.000000,0.000000}%
\pgfsetstrokecolor{currentstroke}%
\pgfsetstrokeopacity{0.700000}%
\pgfsetdash{}{0pt}%
\pgfpathmoveto{\pgfqpoint{5.125000in}{0.550000in}}%
\pgfpathlineto{\pgfqpoint{5.152521in}{0.550000in}}%
\pgfpathlineto{\pgfqpoint{5.152521in}{0.550000in}}%
\pgfpathlineto{\pgfqpoint{5.125000in}{0.550000in}}%
\pgfpathlineto{\pgfqpoint{5.125000in}{0.550000in}}%
\pgfpathclose%
\pgfusepath{fill}%
\end{pgfscope}%
\begin{pgfscope}%
\pgfpathrectangle{\pgfqpoint{1.250000in}{0.550000in}}{\pgfqpoint{7.750000in}{3.850000in}}%
\pgfusepath{clip}%
\pgfsetbuttcap%
\pgfsetmiterjoin%
\definecolor{currentfill}{rgb}{0.501961,0.501961,0.501961}%
\pgfsetfillcolor{currentfill}%
\pgfsetfillopacity{0.700000}%
\pgfsetlinewidth{0.000000pt}%
\definecolor{currentstroke}{rgb}{0.000000,0.000000,0.000000}%
\pgfsetstrokecolor{currentstroke}%
\pgfsetstrokeopacity{0.700000}%
\pgfsetdash{}{0pt}%
\pgfpathmoveto{\pgfqpoint{5.152521in}{0.550000in}}%
\pgfpathlineto{\pgfqpoint{5.180043in}{0.550000in}}%
\pgfpathlineto{\pgfqpoint{5.180043in}{0.950061in}}%
\pgfpathlineto{\pgfqpoint{5.152521in}{0.950061in}}%
\pgfpathlineto{\pgfqpoint{5.152521in}{0.550000in}}%
\pgfpathclose%
\pgfusepath{fill}%
\end{pgfscope}%
\begin{pgfscope}%
\pgfpathrectangle{\pgfqpoint{1.250000in}{0.550000in}}{\pgfqpoint{7.750000in}{3.850000in}}%
\pgfusepath{clip}%
\pgfsetbuttcap%
\pgfsetmiterjoin%
\definecolor{currentfill}{rgb}{0.501961,0.501961,0.501961}%
\pgfsetfillcolor{currentfill}%
\pgfsetfillopacity{0.700000}%
\pgfsetlinewidth{0.000000pt}%
\definecolor{currentstroke}{rgb}{0.000000,0.000000,0.000000}%
\pgfsetstrokecolor{currentstroke}%
\pgfsetstrokeopacity{0.700000}%
\pgfsetdash{}{0pt}%
\pgfpathmoveto{\pgfqpoint{5.180043in}{0.550000in}}%
\pgfpathlineto{\pgfqpoint{5.207564in}{0.550000in}}%
\pgfpathlineto{\pgfqpoint{5.207564in}{0.550000in}}%
\pgfpathlineto{\pgfqpoint{5.180043in}{0.550000in}}%
\pgfpathlineto{\pgfqpoint{5.180043in}{0.550000in}}%
\pgfpathclose%
\pgfusepath{fill}%
\end{pgfscope}%
\begin{pgfscope}%
\pgfpathrectangle{\pgfqpoint{1.250000in}{0.550000in}}{\pgfqpoint{7.750000in}{3.850000in}}%
\pgfusepath{clip}%
\pgfsetbuttcap%
\pgfsetmiterjoin%
\definecolor{currentfill}{rgb}{0.501961,0.501961,0.501961}%
\pgfsetfillcolor{currentfill}%
\pgfsetfillopacity{0.700000}%
\pgfsetlinewidth{0.000000pt}%
\definecolor{currentstroke}{rgb}{0.000000,0.000000,0.000000}%
\pgfsetstrokecolor{currentstroke}%
\pgfsetstrokeopacity{0.700000}%
\pgfsetdash{}{0pt}%
\pgfpathmoveto{\pgfqpoint{5.207564in}{0.550000in}}%
\pgfpathlineto{\pgfqpoint{5.235085in}{0.550000in}}%
\pgfpathlineto{\pgfqpoint{5.235085in}{0.932225in}}%
\pgfpathlineto{\pgfqpoint{5.207564in}{0.932225in}}%
\pgfpathlineto{\pgfqpoint{5.207564in}{0.550000in}}%
\pgfpathclose%
\pgfusepath{fill}%
\end{pgfscope}%
\begin{pgfscope}%
\pgfpathrectangle{\pgfqpoint{1.250000in}{0.550000in}}{\pgfqpoint{7.750000in}{3.850000in}}%
\pgfusepath{clip}%
\pgfsetbuttcap%
\pgfsetmiterjoin%
\definecolor{currentfill}{rgb}{0.501961,0.501961,0.501961}%
\pgfsetfillcolor{currentfill}%
\pgfsetfillopacity{0.700000}%
\pgfsetlinewidth{0.000000pt}%
\definecolor{currentstroke}{rgb}{0.000000,0.000000,0.000000}%
\pgfsetstrokecolor{currentstroke}%
\pgfsetstrokeopacity{0.700000}%
\pgfsetdash{}{0pt}%
\pgfpathmoveto{\pgfqpoint{5.235085in}{0.550000in}}%
\pgfpathlineto{\pgfqpoint{5.262607in}{0.550000in}}%
\pgfpathlineto{\pgfqpoint{5.262607in}{0.550000in}}%
\pgfpathlineto{\pgfqpoint{5.235085in}{0.550000in}}%
\pgfpathlineto{\pgfqpoint{5.235085in}{0.550000in}}%
\pgfpathclose%
\pgfusepath{fill}%
\end{pgfscope}%
\begin{pgfscope}%
\pgfpathrectangle{\pgfqpoint{1.250000in}{0.550000in}}{\pgfqpoint{7.750000in}{3.850000in}}%
\pgfusepath{clip}%
\pgfsetbuttcap%
\pgfsetmiterjoin%
\definecolor{currentfill}{rgb}{0.501961,0.501961,0.501961}%
\pgfsetfillcolor{currentfill}%
\pgfsetfillopacity{0.700000}%
\pgfsetlinewidth{0.000000pt}%
\definecolor{currentstroke}{rgb}{0.000000,0.000000,0.000000}%
\pgfsetstrokecolor{currentstroke}%
\pgfsetstrokeopacity{0.700000}%
\pgfsetdash{}{0pt}%
\pgfpathmoveto{\pgfqpoint{5.262607in}{0.550000in}}%
\pgfpathlineto{\pgfqpoint{5.290128in}{0.550000in}}%
\pgfpathlineto{\pgfqpoint{5.290128in}{0.928745in}}%
\pgfpathlineto{\pgfqpoint{5.262607in}{0.928745in}}%
\pgfpathlineto{\pgfqpoint{5.262607in}{0.550000in}}%
\pgfpathclose%
\pgfusepath{fill}%
\end{pgfscope}%
\begin{pgfscope}%
\pgfpathrectangle{\pgfqpoint{1.250000in}{0.550000in}}{\pgfqpoint{7.750000in}{3.850000in}}%
\pgfusepath{clip}%
\pgfsetbuttcap%
\pgfsetmiterjoin%
\definecolor{currentfill}{rgb}{0.501961,0.501961,0.501961}%
\pgfsetfillcolor{currentfill}%
\pgfsetfillopacity{0.700000}%
\pgfsetlinewidth{0.000000pt}%
\definecolor{currentstroke}{rgb}{0.000000,0.000000,0.000000}%
\pgfsetstrokecolor{currentstroke}%
\pgfsetstrokeopacity{0.700000}%
\pgfsetdash{}{0pt}%
\pgfpathmoveto{\pgfqpoint{5.290128in}{0.550000in}}%
\pgfpathlineto{\pgfqpoint{5.317649in}{0.550000in}}%
\pgfpathlineto{\pgfqpoint{5.317649in}{0.550000in}}%
\pgfpathlineto{\pgfqpoint{5.290128in}{0.550000in}}%
\pgfpathlineto{\pgfqpoint{5.290128in}{0.550000in}}%
\pgfpathclose%
\pgfusepath{fill}%
\end{pgfscope}%
\begin{pgfscope}%
\pgfpathrectangle{\pgfqpoint{1.250000in}{0.550000in}}{\pgfqpoint{7.750000in}{3.850000in}}%
\pgfusepath{clip}%
\pgfsetbuttcap%
\pgfsetmiterjoin%
\definecolor{currentfill}{rgb}{0.501961,0.501961,0.501961}%
\pgfsetfillcolor{currentfill}%
\pgfsetfillopacity{0.700000}%
\pgfsetlinewidth{0.000000pt}%
\definecolor{currentstroke}{rgb}{0.000000,0.000000,0.000000}%
\pgfsetstrokecolor{currentstroke}%
\pgfsetstrokeopacity{0.700000}%
\pgfsetdash{}{0pt}%
\pgfpathmoveto{\pgfqpoint{5.317649in}{0.550000in}}%
\pgfpathlineto{\pgfqpoint{5.345170in}{0.550000in}}%
\pgfpathlineto{\pgfqpoint{5.345170in}{0.550000in}}%
\pgfpathlineto{\pgfqpoint{5.317649in}{0.550000in}}%
\pgfpathlineto{\pgfqpoint{5.317649in}{0.550000in}}%
\pgfpathclose%
\pgfusepath{fill}%
\end{pgfscope}%
\begin{pgfscope}%
\pgfpathrectangle{\pgfqpoint{1.250000in}{0.550000in}}{\pgfqpoint{7.750000in}{3.850000in}}%
\pgfusepath{clip}%
\pgfsetbuttcap%
\pgfsetmiterjoin%
\definecolor{currentfill}{rgb}{0.501961,0.501961,0.501961}%
\pgfsetfillcolor{currentfill}%
\pgfsetfillopacity{0.700000}%
\pgfsetlinewidth{0.000000pt}%
\definecolor{currentstroke}{rgb}{0.000000,0.000000,0.000000}%
\pgfsetstrokecolor{currentstroke}%
\pgfsetstrokeopacity{0.700000}%
\pgfsetdash{}{0pt}%
\pgfpathmoveto{\pgfqpoint{5.345170in}{0.550000in}}%
\pgfpathlineto{\pgfqpoint{5.372692in}{0.550000in}}%
\pgfpathlineto{\pgfqpoint{5.372692in}{0.922365in}}%
\pgfpathlineto{\pgfqpoint{5.345170in}{0.922365in}}%
\pgfpathlineto{\pgfqpoint{5.345170in}{0.550000in}}%
\pgfpathclose%
\pgfusepath{fill}%
\end{pgfscope}%
\begin{pgfscope}%
\pgfpathrectangle{\pgfqpoint{1.250000in}{0.550000in}}{\pgfqpoint{7.750000in}{3.850000in}}%
\pgfusepath{clip}%
\pgfsetbuttcap%
\pgfsetmiterjoin%
\definecolor{currentfill}{rgb}{0.501961,0.501961,0.501961}%
\pgfsetfillcolor{currentfill}%
\pgfsetfillopacity{0.700000}%
\pgfsetlinewidth{0.000000pt}%
\definecolor{currentstroke}{rgb}{0.000000,0.000000,0.000000}%
\pgfsetstrokecolor{currentstroke}%
\pgfsetstrokeopacity{0.700000}%
\pgfsetdash{}{0pt}%
\pgfpathmoveto{\pgfqpoint{5.372692in}{0.550000in}}%
\pgfpathlineto{\pgfqpoint{5.400213in}{0.550000in}}%
\pgfpathlineto{\pgfqpoint{5.400213in}{0.550000in}}%
\pgfpathlineto{\pgfqpoint{5.372692in}{0.550000in}}%
\pgfpathlineto{\pgfqpoint{5.372692in}{0.550000in}}%
\pgfpathclose%
\pgfusepath{fill}%
\end{pgfscope}%
\begin{pgfscope}%
\pgfpathrectangle{\pgfqpoint{1.250000in}{0.550000in}}{\pgfqpoint{7.750000in}{3.850000in}}%
\pgfusepath{clip}%
\pgfsetbuttcap%
\pgfsetmiterjoin%
\definecolor{currentfill}{rgb}{0.501961,0.501961,0.501961}%
\pgfsetfillcolor{currentfill}%
\pgfsetfillopacity{0.700000}%
\pgfsetlinewidth{0.000000pt}%
\definecolor{currentstroke}{rgb}{0.000000,0.000000,0.000000}%
\pgfsetstrokecolor{currentstroke}%
\pgfsetstrokeopacity{0.700000}%
\pgfsetdash{}{0pt}%
\pgfpathmoveto{\pgfqpoint{5.400213in}{0.550000in}}%
\pgfpathlineto{\pgfqpoint{5.427734in}{0.550000in}}%
\pgfpathlineto{\pgfqpoint{5.427734in}{0.908445in}}%
\pgfpathlineto{\pgfqpoint{5.400213in}{0.908445in}}%
\pgfpathlineto{\pgfqpoint{5.400213in}{0.550000in}}%
\pgfpathclose%
\pgfusepath{fill}%
\end{pgfscope}%
\begin{pgfscope}%
\pgfpathrectangle{\pgfqpoint{1.250000in}{0.550000in}}{\pgfqpoint{7.750000in}{3.850000in}}%
\pgfusepath{clip}%
\pgfsetbuttcap%
\pgfsetmiterjoin%
\definecolor{currentfill}{rgb}{0.501961,0.501961,0.501961}%
\pgfsetfillcolor{currentfill}%
\pgfsetfillopacity{0.700000}%
\pgfsetlinewidth{0.000000pt}%
\definecolor{currentstroke}{rgb}{0.000000,0.000000,0.000000}%
\pgfsetstrokecolor{currentstroke}%
\pgfsetstrokeopacity{0.700000}%
\pgfsetdash{}{0pt}%
\pgfpathmoveto{\pgfqpoint{5.427734in}{0.550000in}}%
\pgfpathlineto{\pgfqpoint{5.455256in}{0.550000in}}%
\pgfpathlineto{\pgfqpoint{5.455256in}{0.550000in}}%
\pgfpathlineto{\pgfqpoint{5.427734in}{0.550000in}}%
\pgfpathlineto{\pgfqpoint{5.427734in}{0.550000in}}%
\pgfpathclose%
\pgfusepath{fill}%
\end{pgfscope}%
\begin{pgfscope}%
\pgfpathrectangle{\pgfqpoint{1.250000in}{0.550000in}}{\pgfqpoint{7.750000in}{3.850000in}}%
\pgfusepath{clip}%
\pgfsetbuttcap%
\pgfsetmiterjoin%
\definecolor{currentfill}{rgb}{0.501961,0.501961,0.501961}%
\pgfsetfillcolor{currentfill}%
\pgfsetfillopacity{0.700000}%
\pgfsetlinewidth{0.000000pt}%
\definecolor{currentstroke}{rgb}{0.000000,0.000000,0.000000}%
\pgfsetstrokecolor{currentstroke}%
\pgfsetstrokeopacity{0.700000}%
\pgfsetdash{}{0pt}%
\pgfpathmoveto{\pgfqpoint{5.455256in}{0.550000in}}%
\pgfpathlineto{\pgfqpoint{5.482777in}{0.550000in}}%
\pgfpathlineto{\pgfqpoint{5.482777in}{0.896700in}}%
\pgfpathlineto{\pgfqpoint{5.455256in}{0.896700in}}%
\pgfpathlineto{\pgfqpoint{5.455256in}{0.550000in}}%
\pgfpathclose%
\pgfusepath{fill}%
\end{pgfscope}%
\begin{pgfscope}%
\pgfpathrectangle{\pgfqpoint{1.250000in}{0.550000in}}{\pgfqpoint{7.750000in}{3.850000in}}%
\pgfusepath{clip}%
\pgfsetbuttcap%
\pgfsetmiterjoin%
\definecolor{currentfill}{rgb}{0.501961,0.501961,0.501961}%
\pgfsetfillcolor{currentfill}%
\pgfsetfillopacity{0.700000}%
\pgfsetlinewidth{0.000000pt}%
\definecolor{currentstroke}{rgb}{0.000000,0.000000,0.000000}%
\pgfsetstrokecolor{currentstroke}%
\pgfsetstrokeopacity{0.700000}%
\pgfsetdash{}{0pt}%
\pgfpathmoveto{\pgfqpoint{5.482777in}{0.550000in}}%
\pgfpathlineto{\pgfqpoint{5.510298in}{0.550000in}}%
\pgfpathlineto{\pgfqpoint{5.510298in}{0.550000in}}%
\pgfpathlineto{\pgfqpoint{5.482777in}{0.550000in}}%
\pgfpathlineto{\pgfqpoint{5.482777in}{0.550000in}}%
\pgfpathclose%
\pgfusepath{fill}%
\end{pgfscope}%
\begin{pgfscope}%
\pgfpathrectangle{\pgfqpoint{1.250000in}{0.550000in}}{\pgfqpoint{7.750000in}{3.850000in}}%
\pgfusepath{clip}%
\pgfsetbuttcap%
\pgfsetmiterjoin%
\definecolor{currentfill}{rgb}{0.501961,0.501961,0.501961}%
\pgfsetfillcolor{currentfill}%
\pgfsetfillopacity{0.700000}%
\pgfsetlinewidth{0.000000pt}%
\definecolor{currentstroke}{rgb}{0.000000,0.000000,0.000000}%
\pgfsetstrokecolor{currentstroke}%
\pgfsetstrokeopacity{0.700000}%
\pgfsetdash{}{0pt}%
\pgfpathmoveto{\pgfqpoint{5.510298in}{0.550000in}}%
\pgfpathlineto{\pgfqpoint{5.537820in}{0.550000in}}%
\pgfpathlineto{\pgfqpoint{5.537820in}{0.883360in}}%
\pgfpathlineto{\pgfqpoint{5.510298in}{0.883360in}}%
\pgfpathlineto{\pgfqpoint{5.510298in}{0.550000in}}%
\pgfpathclose%
\pgfusepath{fill}%
\end{pgfscope}%
\begin{pgfscope}%
\pgfpathrectangle{\pgfqpoint{1.250000in}{0.550000in}}{\pgfqpoint{7.750000in}{3.850000in}}%
\pgfusepath{clip}%
\pgfsetbuttcap%
\pgfsetmiterjoin%
\definecolor{currentfill}{rgb}{0.501961,0.501961,0.501961}%
\pgfsetfillcolor{currentfill}%
\pgfsetfillopacity{0.700000}%
\pgfsetlinewidth{0.000000pt}%
\definecolor{currentstroke}{rgb}{0.000000,0.000000,0.000000}%
\pgfsetstrokecolor{currentstroke}%
\pgfsetstrokeopacity{0.700000}%
\pgfsetdash{}{0pt}%
\pgfpathmoveto{\pgfqpoint{5.537820in}{0.550000in}}%
\pgfpathlineto{\pgfqpoint{5.565341in}{0.550000in}}%
\pgfpathlineto{\pgfqpoint{5.565341in}{0.550000in}}%
\pgfpathlineto{\pgfqpoint{5.537820in}{0.550000in}}%
\pgfpathlineto{\pgfqpoint{5.537820in}{0.550000in}}%
\pgfpathclose%
\pgfusepath{fill}%
\end{pgfscope}%
\begin{pgfscope}%
\pgfpathrectangle{\pgfqpoint{1.250000in}{0.550000in}}{\pgfqpoint{7.750000in}{3.850000in}}%
\pgfusepath{clip}%
\pgfsetbuttcap%
\pgfsetmiterjoin%
\definecolor{currentfill}{rgb}{0.501961,0.501961,0.501961}%
\pgfsetfillcolor{currentfill}%
\pgfsetfillopacity{0.700000}%
\pgfsetlinewidth{0.000000pt}%
\definecolor{currentstroke}{rgb}{0.000000,0.000000,0.000000}%
\pgfsetstrokecolor{currentstroke}%
\pgfsetstrokeopacity{0.700000}%
\pgfsetdash{}{0pt}%
\pgfpathmoveto{\pgfqpoint{5.565341in}{0.550000in}}%
\pgfpathlineto{\pgfqpoint{5.592862in}{0.550000in}}%
\pgfpathlineto{\pgfqpoint{5.592862in}{0.869149in}}%
\pgfpathlineto{\pgfqpoint{5.565341in}{0.869149in}}%
\pgfpathlineto{\pgfqpoint{5.565341in}{0.550000in}}%
\pgfpathclose%
\pgfusepath{fill}%
\end{pgfscope}%
\begin{pgfscope}%
\pgfpathrectangle{\pgfqpoint{1.250000in}{0.550000in}}{\pgfqpoint{7.750000in}{3.850000in}}%
\pgfusepath{clip}%
\pgfsetbuttcap%
\pgfsetmiterjoin%
\definecolor{currentfill}{rgb}{0.501961,0.501961,0.501961}%
\pgfsetfillcolor{currentfill}%
\pgfsetfillopacity{0.700000}%
\pgfsetlinewidth{0.000000pt}%
\definecolor{currentstroke}{rgb}{0.000000,0.000000,0.000000}%
\pgfsetstrokecolor{currentstroke}%
\pgfsetstrokeopacity{0.700000}%
\pgfsetdash{}{0pt}%
\pgfpathmoveto{\pgfqpoint{5.592862in}{0.550000in}}%
\pgfpathlineto{\pgfqpoint{5.620384in}{0.550000in}}%
\pgfpathlineto{\pgfqpoint{5.620384in}{0.550000in}}%
\pgfpathlineto{\pgfqpoint{5.592862in}{0.550000in}}%
\pgfpathlineto{\pgfqpoint{5.592862in}{0.550000in}}%
\pgfpathclose%
\pgfusepath{fill}%
\end{pgfscope}%
\begin{pgfscope}%
\pgfpathrectangle{\pgfqpoint{1.250000in}{0.550000in}}{\pgfqpoint{7.750000in}{3.850000in}}%
\pgfusepath{clip}%
\pgfsetbuttcap%
\pgfsetmiterjoin%
\definecolor{currentfill}{rgb}{0.501961,0.501961,0.501961}%
\pgfsetfillcolor{currentfill}%
\pgfsetfillopacity{0.700000}%
\pgfsetlinewidth{0.000000pt}%
\definecolor{currentstroke}{rgb}{0.000000,0.000000,0.000000}%
\pgfsetstrokecolor{currentstroke}%
\pgfsetstrokeopacity{0.700000}%
\pgfsetdash{}{0pt}%
\pgfpathmoveto{\pgfqpoint{5.620384in}{0.550000in}}%
\pgfpathlineto{\pgfqpoint{5.647905in}{0.550000in}}%
\pgfpathlineto{\pgfqpoint{5.647905in}{0.854794in}}%
\pgfpathlineto{\pgfqpoint{5.620384in}{0.854794in}}%
\pgfpathlineto{\pgfqpoint{5.620384in}{0.550000in}}%
\pgfpathclose%
\pgfusepath{fill}%
\end{pgfscope}%
\begin{pgfscope}%
\pgfpathrectangle{\pgfqpoint{1.250000in}{0.550000in}}{\pgfqpoint{7.750000in}{3.850000in}}%
\pgfusepath{clip}%
\pgfsetbuttcap%
\pgfsetmiterjoin%
\definecolor{currentfill}{rgb}{0.501961,0.501961,0.501961}%
\pgfsetfillcolor{currentfill}%
\pgfsetfillopacity{0.700000}%
\pgfsetlinewidth{0.000000pt}%
\definecolor{currentstroke}{rgb}{0.000000,0.000000,0.000000}%
\pgfsetstrokecolor{currentstroke}%
\pgfsetstrokeopacity{0.700000}%
\pgfsetdash{}{0pt}%
\pgfpathmoveto{\pgfqpoint{5.647905in}{0.550000in}}%
\pgfpathlineto{\pgfqpoint{5.675426in}{0.550000in}}%
\pgfpathlineto{\pgfqpoint{5.675426in}{0.550000in}}%
\pgfpathlineto{\pgfqpoint{5.647905in}{0.550000in}}%
\pgfpathlineto{\pgfqpoint{5.647905in}{0.550000in}}%
\pgfpathclose%
\pgfusepath{fill}%
\end{pgfscope}%
\begin{pgfscope}%
\pgfpathrectangle{\pgfqpoint{1.250000in}{0.550000in}}{\pgfqpoint{7.750000in}{3.850000in}}%
\pgfusepath{clip}%
\pgfsetbuttcap%
\pgfsetmiterjoin%
\definecolor{currentfill}{rgb}{0.501961,0.501961,0.501961}%
\pgfsetfillcolor{currentfill}%
\pgfsetfillopacity{0.700000}%
\pgfsetlinewidth{0.000000pt}%
\definecolor{currentstroke}{rgb}{0.000000,0.000000,0.000000}%
\pgfsetstrokecolor{currentstroke}%
\pgfsetstrokeopacity{0.700000}%
\pgfsetdash{}{0pt}%
\pgfpathmoveto{\pgfqpoint{5.675426in}{0.550000in}}%
\pgfpathlineto{\pgfqpoint{5.702947in}{0.550000in}}%
\pgfpathlineto{\pgfqpoint{5.702947in}{0.838264in}}%
\pgfpathlineto{\pgfqpoint{5.675426in}{0.838264in}}%
\pgfpathlineto{\pgfqpoint{5.675426in}{0.550000in}}%
\pgfpathclose%
\pgfusepath{fill}%
\end{pgfscope}%
\begin{pgfscope}%
\pgfpathrectangle{\pgfqpoint{1.250000in}{0.550000in}}{\pgfqpoint{7.750000in}{3.850000in}}%
\pgfusepath{clip}%
\pgfsetbuttcap%
\pgfsetmiterjoin%
\definecolor{currentfill}{rgb}{0.501961,0.501961,0.501961}%
\pgfsetfillcolor{currentfill}%
\pgfsetfillopacity{0.700000}%
\pgfsetlinewidth{0.000000pt}%
\definecolor{currentstroke}{rgb}{0.000000,0.000000,0.000000}%
\pgfsetstrokecolor{currentstroke}%
\pgfsetstrokeopacity{0.700000}%
\pgfsetdash{}{0pt}%
\pgfpathmoveto{\pgfqpoint{5.702947in}{0.550000in}}%
\pgfpathlineto{\pgfqpoint{5.730469in}{0.550000in}}%
\pgfpathlineto{\pgfqpoint{5.730469in}{0.550000in}}%
\pgfpathlineto{\pgfqpoint{5.702947in}{0.550000in}}%
\pgfpathlineto{\pgfqpoint{5.702947in}{0.550000in}}%
\pgfpathclose%
\pgfusepath{fill}%
\end{pgfscope}%
\begin{pgfscope}%
\pgfpathrectangle{\pgfqpoint{1.250000in}{0.550000in}}{\pgfqpoint{7.750000in}{3.850000in}}%
\pgfusepath{clip}%
\pgfsetbuttcap%
\pgfsetmiterjoin%
\definecolor{currentfill}{rgb}{0.501961,0.501961,0.501961}%
\pgfsetfillcolor{currentfill}%
\pgfsetfillopacity{0.700000}%
\pgfsetlinewidth{0.000000pt}%
\definecolor{currentstroke}{rgb}{0.000000,0.000000,0.000000}%
\pgfsetstrokecolor{currentstroke}%
\pgfsetstrokeopacity{0.700000}%
\pgfsetdash{}{0pt}%
\pgfpathmoveto{\pgfqpoint{5.730469in}{0.550000in}}%
\pgfpathlineto{\pgfqpoint{5.757990in}{0.550000in}}%
\pgfpathlineto{\pgfqpoint{5.757990in}{0.842469in}}%
\pgfpathlineto{\pgfqpoint{5.730469in}{0.842469in}}%
\pgfpathlineto{\pgfqpoint{5.730469in}{0.550000in}}%
\pgfpathclose%
\pgfusepath{fill}%
\end{pgfscope}%
\begin{pgfscope}%
\pgfpathrectangle{\pgfqpoint{1.250000in}{0.550000in}}{\pgfqpoint{7.750000in}{3.850000in}}%
\pgfusepath{clip}%
\pgfsetbuttcap%
\pgfsetmiterjoin%
\definecolor{currentfill}{rgb}{0.501961,0.501961,0.501961}%
\pgfsetfillcolor{currentfill}%
\pgfsetfillopacity{0.700000}%
\pgfsetlinewidth{0.000000pt}%
\definecolor{currentstroke}{rgb}{0.000000,0.000000,0.000000}%
\pgfsetstrokecolor{currentstroke}%
\pgfsetstrokeopacity{0.700000}%
\pgfsetdash{}{0pt}%
\pgfpathmoveto{\pgfqpoint{5.757990in}{0.550000in}}%
\pgfpathlineto{\pgfqpoint{5.785511in}{0.550000in}}%
\pgfpathlineto{\pgfqpoint{5.785511in}{0.823764in}}%
\pgfpathlineto{\pgfqpoint{5.757990in}{0.823764in}}%
\pgfpathlineto{\pgfqpoint{5.757990in}{0.550000in}}%
\pgfpathclose%
\pgfusepath{fill}%
\end{pgfscope}%
\begin{pgfscope}%
\pgfpathrectangle{\pgfqpoint{1.250000in}{0.550000in}}{\pgfqpoint{7.750000in}{3.850000in}}%
\pgfusepath{clip}%
\pgfsetbuttcap%
\pgfsetmiterjoin%
\definecolor{currentfill}{rgb}{0.501961,0.501961,0.501961}%
\pgfsetfillcolor{currentfill}%
\pgfsetfillopacity{0.700000}%
\pgfsetlinewidth{0.000000pt}%
\definecolor{currentstroke}{rgb}{0.000000,0.000000,0.000000}%
\pgfsetstrokecolor{currentstroke}%
\pgfsetstrokeopacity{0.700000}%
\pgfsetdash{}{0pt}%
\pgfpathmoveto{\pgfqpoint{5.785511in}{0.550000in}}%
\pgfpathlineto{\pgfqpoint{5.813033in}{0.550000in}}%
\pgfpathlineto{\pgfqpoint{5.813033in}{0.550000in}}%
\pgfpathlineto{\pgfqpoint{5.785511in}{0.550000in}}%
\pgfpathlineto{\pgfqpoint{5.785511in}{0.550000in}}%
\pgfpathclose%
\pgfusepath{fill}%
\end{pgfscope}%
\begin{pgfscope}%
\pgfpathrectangle{\pgfqpoint{1.250000in}{0.550000in}}{\pgfqpoint{7.750000in}{3.850000in}}%
\pgfusepath{clip}%
\pgfsetbuttcap%
\pgfsetmiterjoin%
\definecolor{currentfill}{rgb}{0.501961,0.501961,0.501961}%
\pgfsetfillcolor{currentfill}%
\pgfsetfillopacity{0.700000}%
\pgfsetlinewidth{0.000000pt}%
\definecolor{currentstroke}{rgb}{0.000000,0.000000,0.000000}%
\pgfsetstrokecolor{currentstroke}%
\pgfsetstrokeopacity{0.700000}%
\pgfsetdash{}{0pt}%
\pgfpathmoveto{\pgfqpoint{5.813033in}{0.550000in}}%
\pgfpathlineto{\pgfqpoint{5.840554in}{0.550000in}}%
\pgfpathlineto{\pgfqpoint{5.840554in}{0.816949in}}%
\pgfpathlineto{\pgfqpoint{5.813033in}{0.816949in}}%
\pgfpathlineto{\pgfqpoint{5.813033in}{0.550000in}}%
\pgfpathclose%
\pgfusepath{fill}%
\end{pgfscope}%
\begin{pgfscope}%
\pgfpathrectangle{\pgfqpoint{1.250000in}{0.550000in}}{\pgfqpoint{7.750000in}{3.850000in}}%
\pgfusepath{clip}%
\pgfsetbuttcap%
\pgfsetmiterjoin%
\definecolor{currentfill}{rgb}{0.501961,0.501961,0.501961}%
\pgfsetfillcolor{currentfill}%
\pgfsetfillopacity{0.700000}%
\pgfsetlinewidth{0.000000pt}%
\definecolor{currentstroke}{rgb}{0.000000,0.000000,0.000000}%
\pgfsetstrokecolor{currentstroke}%
\pgfsetstrokeopacity{0.700000}%
\pgfsetdash{}{0pt}%
\pgfpathmoveto{\pgfqpoint{5.840554in}{0.550000in}}%
\pgfpathlineto{\pgfqpoint{5.868075in}{0.550000in}}%
\pgfpathlineto{\pgfqpoint{5.868075in}{0.550000in}}%
\pgfpathlineto{\pgfqpoint{5.840554in}{0.550000in}}%
\pgfpathlineto{\pgfqpoint{5.840554in}{0.550000in}}%
\pgfpathclose%
\pgfusepath{fill}%
\end{pgfscope}%
\begin{pgfscope}%
\pgfpathrectangle{\pgfqpoint{1.250000in}{0.550000in}}{\pgfqpoint{7.750000in}{3.850000in}}%
\pgfusepath{clip}%
\pgfsetbuttcap%
\pgfsetmiterjoin%
\definecolor{currentfill}{rgb}{0.501961,0.501961,0.501961}%
\pgfsetfillcolor{currentfill}%
\pgfsetfillopacity{0.700000}%
\pgfsetlinewidth{0.000000pt}%
\definecolor{currentstroke}{rgb}{0.000000,0.000000,0.000000}%
\pgfsetstrokecolor{currentstroke}%
\pgfsetstrokeopacity{0.700000}%
\pgfsetdash{}{0pt}%
\pgfpathmoveto{\pgfqpoint{5.868075in}{0.550000in}}%
\pgfpathlineto{\pgfqpoint{5.895597in}{0.550000in}}%
\pgfpathlineto{\pgfqpoint{5.895597in}{0.804914in}}%
\pgfpathlineto{\pgfqpoint{5.868075in}{0.804914in}}%
\pgfpathlineto{\pgfqpoint{5.868075in}{0.550000in}}%
\pgfpathclose%
\pgfusepath{fill}%
\end{pgfscope}%
\begin{pgfscope}%
\pgfpathrectangle{\pgfqpoint{1.250000in}{0.550000in}}{\pgfqpoint{7.750000in}{3.850000in}}%
\pgfusepath{clip}%
\pgfsetbuttcap%
\pgfsetmiterjoin%
\definecolor{currentfill}{rgb}{0.501961,0.501961,0.501961}%
\pgfsetfillcolor{currentfill}%
\pgfsetfillopacity{0.700000}%
\pgfsetlinewidth{0.000000pt}%
\definecolor{currentstroke}{rgb}{0.000000,0.000000,0.000000}%
\pgfsetstrokecolor{currentstroke}%
\pgfsetstrokeopacity{0.700000}%
\pgfsetdash{}{0pt}%
\pgfpathmoveto{\pgfqpoint{5.895597in}{0.550000in}}%
\pgfpathlineto{\pgfqpoint{5.923118in}{0.550000in}}%
\pgfpathlineto{\pgfqpoint{5.923118in}{0.794763in}}%
\pgfpathlineto{\pgfqpoint{5.895597in}{0.794763in}}%
\pgfpathlineto{\pgfqpoint{5.895597in}{0.550000in}}%
\pgfpathclose%
\pgfusepath{fill}%
\end{pgfscope}%
\begin{pgfscope}%
\pgfpathrectangle{\pgfqpoint{1.250000in}{0.550000in}}{\pgfqpoint{7.750000in}{3.850000in}}%
\pgfusepath{clip}%
\pgfsetbuttcap%
\pgfsetmiterjoin%
\definecolor{currentfill}{rgb}{0.501961,0.501961,0.501961}%
\pgfsetfillcolor{currentfill}%
\pgfsetfillopacity{0.700000}%
\pgfsetlinewidth{0.000000pt}%
\definecolor{currentstroke}{rgb}{0.000000,0.000000,0.000000}%
\pgfsetstrokecolor{currentstroke}%
\pgfsetstrokeopacity{0.700000}%
\pgfsetdash{}{0pt}%
\pgfpathmoveto{\pgfqpoint{5.923118in}{0.550000in}}%
\pgfpathlineto{\pgfqpoint{5.950639in}{0.550000in}}%
\pgfpathlineto{\pgfqpoint{5.950639in}{0.550000in}}%
\pgfpathlineto{\pgfqpoint{5.923118in}{0.550000in}}%
\pgfpathlineto{\pgfqpoint{5.923118in}{0.550000in}}%
\pgfpathclose%
\pgfusepath{fill}%
\end{pgfscope}%
\begin{pgfscope}%
\pgfpathrectangle{\pgfqpoint{1.250000in}{0.550000in}}{\pgfqpoint{7.750000in}{3.850000in}}%
\pgfusepath{clip}%
\pgfsetbuttcap%
\pgfsetmiterjoin%
\definecolor{currentfill}{rgb}{0.501961,0.501961,0.501961}%
\pgfsetfillcolor{currentfill}%
\pgfsetfillopacity{0.700000}%
\pgfsetlinewidth{0.000000pt}%
\definecolor{currentstroke}{rgb}{0.000000,0.000000,0.000000}%
\pgfsetstrokecolor{currentstroke}%
\pgfsetstrokeopacity{0.700000}%
\pgfsetdash{}{0pt}%
\pgfpathmoveto{\pgfqpoint{5.950639in}{0.550000in}}%
\pgfpathlineto{\pgfqpoint{5.978161in}{0.550000in}}%
\pgfpathlineto{\pgfqpoint{5.978161in}{0.796503in}}%
\pgfpathlineto{\pgfqpoint{5.950639in}{0.796503in}}%
\pgfpathlineto{\pgfqpoint{5.950639in}{0.550000in}}%
\pgfpathclose%
\pgfusepath{fill}%
\end{pgfscope}%
\begin{pgfscope}%
\pgfpathrectangle{\pgfqpoint{1.250000in}{0.550000in}}{\pgfqpoint{7.750000in}{3.850000in}}%
\pgfusepath{clip}%
\pgfsetbuttcap%
\pgfsetmiterjoin%
\definecolor{currentfill}{rgb}{0.501961,0.501961,0.501961}%
\pgfsetfillcolor{currentfill}%
\pgfsetfillopacity{0.700000}%
\pgfsetlinewidth{0.000000pt}%
\definecolor{currentstroke}{rgb}{0.000000,0.000000,0.000000}%
\pgfsetstrokecolor{currentstroke}%
\pgfsetstrokeopacity{0.700000}%
\pgfsetdash{}{0pt}%
\pgfpathmoveto{\pgfqpoint{5.978161in}{0.550000in}}%
\pgfpathlineto{\pgfqpoint{6.005682in}{0.550000in}}%
\pgfpathlineto{\pgfqpoint{6.005682in}{0.782728in}}%
\pgfpathlineto{\pgfqpoint{5.978161in}{0.782728in}}%
\pgfpathlineto{\pgfqpoint{5.978161in}{0.550000in}}%
\pgfpathclose%
\pgfusepath{fill}%
\end{pgfscope}%
\begin{pgfscope}%
\pgfpathrectangle{\pgfqpoint{1.250000in}{0.550000in}}{\pgfqpoint{7.750000in}{3.850000in}}%
\pgfusepath{clip}%
\pgfsetbuttcap%
\pgfsetmiterjoin%
\definecolor{currentfill}{rgb}{0.501961,0.501961,0.501961}%
\pgfsetfillcolor{currentfill}%
\pgfsetfillopacity{0.700000}%
\pgfsetlinewidth{0.000000pt}%
\definecolor{currentstroke}{rgb}{0.000000,0.000000,0.000000}%
\pgfsetstrokecolor{currentstroke}%
\pgfsetstrokeopacity{0.700000}%
\pgfsetdash{}{0pt}%
\pgfpathmoveto{\pgfqpoint{6.005682in}{0.550000in}}%
\pgfpathlineto{\pgfqpoint{6.033203in}{0.550000in}}%
\pgfpathlineto{\pgfqpoint{6.033203in}{0.550000in}}%
\pgfpathlineto{\pgfqpoint{6.005682in}{0.550000in}}%
\pgfpathlineto{\pgfqpoint{6.005682in}{0.550000in}}%
\pgfpathclose%
\pgfusepath{fill}%
\end{pgfscope}%
\begin{pgfscope}%
\pgfpathrectangle{\pgfqpoint{1.250000in}{0.550000in}}{\pgfqpoint{7.750000in}{3.850000in}}%
\pgfusepath{clip}%
\pgfsetbuttcap%
\pgfsetmiterjoin%
\definecolor{currentfill}{rgb}{0.501961,0.501961,0.501961}%
\pgfsetfillcolor{currentfill}%
\pgfsetfillopacity{0.700000}%
\pgfsetlinewidth{0.000000pt}%
\definecolor{currentstroke}{rgb}{0.000000,0.000000,0.000000}%
\pgfsetstrokecolor{currentstroke}%
\pgfsetstrokeopacity{0.700000}%
\pgfsetdash{}{0pt}%
\pgfpathmoveto{\pgfqpoint{6.033203in}{0.550000in}}%
\pgfpathlineto{\pgfqpoint{6.060724in}{0.550000in}}%
\pgfpathlineto{\pgfqpoint{6.060724in}{0.779393in}}%
\pgfpathlineto{\pgfqpoint{6.033203in}{0.779393in}}%
\pgfpathlineto{\pgfqpoint{6.033203in}{0.550000in}}%
\pgfpathclose%
\pgfusepath{fill}%
\end{pgfscope}%
\begin{pgfscope}%
\pgfpathrectangle{\pgfqpoint{1.250000in}{0.550000in}}{\pgfqpoint{7.750000in}{3.850000in}}%
\pgfusepath{clip}%
\pgfsetbuttcap%
\pgfsetmiterjoin%
\definecolor{currentfill}{rgb}{0.501961,0.501961,0.501961}%
\pgfsetfillcolor{currentfill}%
\pgfsetfillopacity{0.700000}%
\pgfsetlinewidth{0.000000pt}%
\definecolor{currentstroke}{rgb}{0.000000,0.000000,0.000000}%
\pgfsetstrokecolor{currentstroke}%
\pgfsetstrokeopacity{0.700000}%
\pgfsetdash{}{0pt}%
\pgfpathmoveto{\pgfqpoint{6.060724in}{0.550000in}}%
\pgfpathlineto{\pgfqpoint{6.088246in}{0.550000in}}%
\pgfpathlineto{\pgfqpoint{6.088246in}{0.763733in}}%
\pgfpathlineto{\pgfqpoint{6.060724in}{0.763733in}}%
\pgfpathlineto{\pgfqpoint{6.060724in}{0.550000in}}%
\pgfpathclose%
\pgfusepath{fill}%
\end{pgfscope}%
\begin{pgfscope}%
\pgfpathrectangle{\pgfqpoint{1.250000in}{0.550000in}}{\pgfqpoint{7.750000in}{3.850000in}}%
\pgfusepath{clip}%
\pgfsetbuttcap%
\pgfsetmiterjoin%
\definecolor{currentfill}{rgb}{0.501961,0.501961,0.501961}%
\pgfsetfillcolor{currentfill}%
\pgfsetfillopacity{0.700000}%
\pgfsetlinewidth{0.000000pt}%
\definecolor{currentstroke}{rgb}{0.000000,0.000000,0.000000}%
\pgfsetstrokecolor{currentstroke}%
\pgfsetstrokeopacity{0.700000}%
\pgfsetdash{}{0pt}%
\pgfpathmoveto{\pgfqpoint{6.088246in}{0.550000in}}%
\pgfpathlineto{\pgfqpoint{6.115767in}{0.550000in}}%
\pgfpathlineto{\pgfqpoint{6.115767in}{0.762573in}}%
\pgfpathlineto{\pgfqpoint{6.088246in}{0.762573in}}%
\pgfpathlineto{\pgfqpoint{6.088246in}{0.550000in}}%
\pgfpathclose%
\pgfusepath{fill}%
\end{pgfscope}%
\begin{pgfscope}%
\pgfpathrectangle{\pgfqpoint{1.250000in}{0.550000in}}{\pgfqpoint{7.750000in}{3.850000in}}%
\pgfusepath{clip}%
\pgfsetbuttcap%
\pgfsetmiterjoin%
\definecolor{currentfill}{rgb}{0.501961,0.501961,0.501961}%
\pgfsetfillcolor{currentfill}%
\pgfsetfillopacity{0.700000}%
\pgfsetlinewidth{0.000000pt}%
\definecolor{currentstroke}{rgb}{0.000000,0.000000,0.000000}%
\pgfsetstrokecolor{currentstroke}%
\pgfsetstrokeopacity{0.700000}%
\pgfsetdash{}{0pt}%
\pgfpathmoveto{\pgfqpoint{6.115767in}{0.550000in}}%
\pgfpathlineto{\pgfqpoint{6.143288in}{0.550000in}}%
\pgfpathlineto{\pgfqpoint{6.143288in}{0.550000in}}%
\pgfpathlineto{\pgfqpoint{6.115767in}{0.550000in}}%
\pgfpathlineto{\pgfqpoint{6.115767in}{0.550000in}}%
\pgfpathclose%
\pgfusepath{fill}%
\end{pgfscope}%
\begin{pgfscope}%
\pgfpathrectangle{\pgfqpoint{1.250000in}{0.550000in}}{\pgfqpoint{7.750000in}{3.850000in}}%
\pgfusepath{clip}%
\pgfsetbuttcap%
\pgfsetmiterjoin%
\definecolor{currentfill}{rgb}{0.501961,0.501961,0.501961}%
\pgfsetfillcolor{currentfill}%
\pgfsetfillopacity{0.700000}%
\pgfsetlinewidth{0.000000pt}%
\definecolor{currentstroke}{rgb}{0.000000,0.000000,0.000000}%
\pgfsetstrokecolor{currentstroke}%
\pgfsetstrokeopacity{0.700000}%
\pgfsetdash{}{0pt}%
\pgfpathmoveto{\pgfqpoint{6.143288in}{0.550000in}}%
\pgfpathlineto{\pgfqpoint{6.170810in}{0.550000in}}%
\pgfpathlineto{\pgfqpoint{6.170810in}{0.749958in}}%
\pgfpathlineto{\pgfqpoint{6.143288in}{0.749958in}}%
\pgfpathlineto{\pgfqpoint{6.143288in}{0.550000in}}%
\pgfpathclose%
\pgfusepath{fill}%
\end{pgfscope}%
\begin{pgfscope}%
\pgfpathrectangle{\pgfqpoint{1.250000in}{0.550000in}}{\pgfqpoint{7.750000in}{3.850000in}}%
\pgfusepath{clip}%
\pgfsetbuttcap%
\pgfsetmiterjoin%
\definecolor{currentfill}{rgb}{0.501961,0.501961,0.501961}%
\pgfsetfillcolor{currentfill}%
\pgfsetfillopacity{0.700000}%
\pgfsetlinewidth{0.000000pt}%
\definecolor{currentstroke}{rgb}{0.000000,0.000000,0.000000}%
\pgfsetstrokecolor{currentstroke}%
\pgfsetstrokeopacity{0.700000}%
\pgfsetdash{}{0pt}%
\pgfpathmoveto{\pgfqpoint{6.170810in}{0.550000in}}%
\pgfpathlineto{\pgfqpoint{6.198331in}{0.550000in}}%
\pgfpathlineto{\pgfqpoint{6.198331in}{0.736328in}}%
\pgfpathlineto{\pgfqpoint{6.170810in}{0.736328in}}%
\pgfpathlineto{\pgfqpoint{6.170810in}{0.550000in}}%
\pgfpathclose%
\pgfusepath{fill}%
\end{pgfscope}%
\begin{pgfscope}%
\pgfpathrectangle{\pgfqpoint{1.250000in}{0.550000in}}{\pgfqpoint{7.750000in}{3.850000in}}%
\pgfusepath{clip}%
\pgfsetbuttcap%
\pgfsetmiterjoin%
\definecolor{currentfill}{rgb}{0.501961,0.501961,0.501961}%
\pgfsetfillcolor{currentfill}%
\pgfsetfillopacity{0.700000}%
\pgfsetlinewidth{0.000000pt}%
\definecolor{currentstroke}{rgb}{0.000000,0.000000,0.000000}%
\pgfsetstrokecolor{currentstroke}%
\pgfsetstrokeopacity{0.700000}%
\pgfsetdash{}{0pt}%
\pgfpathmoveto{\pgfqpoint{6.198331in}{0.550000in}}%
\pgfpathlineto{\pgfqpoint{6.225852in}{0.550000in}}%
\pgfpathlineto{\pgfqpoint{6.225852in}{0.726467in}}%
\pgfpathlineto{\pgfqpoint{6.198331in}{0.726467in}}%
\pgfpathlineto{\pgfqpoint{6.198331in}{0.550000in}}%
\pgfpathclose%
\pgfusepath{fill}%
\end{pgfscope}%
\begin{pgfscope}%
\pgfpathrectangle{\pgfqpoint{1.250000in}{0.550000in}}{\pgfqpoint{7.750000in}{3.850000in}}%
\pgfusepath{clip}%
\pgfsetbuttcap%
\pgfsetmiterjoin%
\definecolor{currentfill}{rgb}{0.501961,0.501961,0.501961}%
\pgfsetfillcolor{currentfill}%
\pgfsetfillopacity{0.700000}%
\pgfsetlinewidth{0.000000pt}%
\definecolor{currentstroke}{rgb}{0.000000,0.000000,0.000000}%
\pgfsetstrokecolor{currentstroke}%
\pgfsetstrokeopacity{0.700000}%
\pgfsetdash{}{0pt}%
\pgfpathmoveto{\pgfqpoint{6.225852in}{0.550000in}}%
\pgfpathlineto{\pgfqpoint{6.253374in}{0.550000in}}%
\pgfpathlineto{\pgfqpoint{6.253374in}{0.732413in}}%
\pgfpathlineto{\pgfqpoint{6.225852in}{0.732413in}}%
\pgfpathlineto{\pgfqpoint{6.225852in}{0.550000in}}%
\pgfpathclose%
\pgfusepath{fill}%
\end{pgfscope}%
\begin{pgfscope}%
\pgfpathrectangle{\pgfqpoint{1.250000in}{0.550000in}}{\pgfqpoint{7.750000in}{3.850000in}}%
\pgfusepath{clip}%
\pgfsetbuttcap%
\pgfsetmiterjoin%
\definecolor{currentfill}{rgb}{0.501961,0.501961,0.501961}%
\pgfsetfillcolor{currentfill}%
\pgfsetfillopacity{0.700000}%
\pgfsetlinewidth{0.000000pt}%
\definecolor{currentstroke}{rgb}{0.000000,0.000000,0.000000}%
\pgfsetstrokecolor{currentstroke}%
\pgfsetstrokeopacity{0.700000}%
\pgfsetdash{}{0pt}%
\pgfpathmoveto{\pgfqpoint{6.253374in}{0.550000in}}%
\pgfpathlineto{\pgfqpoint{6.280895in}{0.550000in}}%
\pgfpathlineto{\pgfqpoint{6.280895in}{0.721827in}}%
\pgfpathlineto{\pgfqpoint{6.253374in}{0.721827in}}%
\pgfpathlineto{\pgfqpoint{6.253374in}{0.550000in}}%
\pgfpathclose%
\pgfusepath{fill}%
\end{pgfscope}%
\begin{pgfscope}%
\pgfpathrectangle{\pgfqpoint{1.250000in}{0.550000in}}{\pgfqpoint{7.750000in}{3.850000in}}%
\pgfusepath{clip}%
\pgfsetbuttcap%
\pgfsetmiterjoin%
\definecolor{currentfill}{rgb}{0.501961,0.501961,0.501961}%
\pgfsetfillcolor{currentfill}%
\pgfsetfillopacity{0.700000}%
\pgfsetlinewidth{0.000000pt}%
\definecolor{currentstroke}{rgb}{0.000000,0.000000,0.000000}%
\pgfsetstrokecolor{currentstroke}%
\pgfsetstrokeopacity{0.700000}%
\pgfsetdash{}{0pt}%
\pgfpathmoveto{\pgfqpoint{6.280895in}{0.550000in}}%
\pgfpathlineto{\pgfqpoint{6.308416in}{0.550000in}}%
\pgfpathlineto{\pgfqpoint{6.308416in}{0.708487in}}%
\pgfpathlineto{\pgfqpoint{6.280895in}{0.708487in}}%
\pgfpathlineto{\pgfqpoint{6.280895in}{0.550000in}}%
\pgfpathclose%
\pgfusepath{fill}%
\end{pgfscope}%
\begin{pgfscope}%
\pgfpathrectangle{\pgfqpoint{1.250000in}{0.550000in}}{\pgfqpoint{7.750000in}{3.850000in}}%
\pgfusepath{clip}%
\pgfsetbuttcap%
\pgfsetmiterjoin%
\definecolor{currentfill}{rgb}{0.501961,0.501961,0.501961}%
\pgfsetfillcolor{currentfill}%
\pgfsetfillopacity{0.700000}%
\pgfsetlinewidth{0.000000pt}%
\definecolor{currentstroke}{rgb}{0.000000,0.000000,0.000000}%
\pgfsetstrokecolor{currentstroke}%
\pgfsetstrokeopacity{0.700000}%
\pgfsetdash{}{0pt}%
\pgfpathmoveto{\pgfqpoint{6.308416in}{0.550000in}}%
\pgfpathlineto{\pgfqpoint{6.335937in}{0.550000in}}%
\pgfpathlineto{\pgfqpoint{6.335937in}{0.704862in}}%
\pgfpathlineto{\pgfqpoint{6.308416in}{0.704862in}}%
\pgfpathlineto{\pgfqpoint{6.308416in}{0.550000in}}%
\pgfpathclose%
\pgfusepath{fill}%
\end{pgfscope}%
\begin{pgfscope}%
\pgfpathrectangle{\pgfqpoint{1.250000in}{0.550000in}}{\pgfqpoint{7.750000in}{3.850000in}}%
\pgfusepath{clip}%
\pgfsetbuttcap%
\pgfsetmiterjoin%
\definecolor{currentfill}{rgb}{0.501961,0.501961,0.501961}%
\pgfsetfillcolor{currentfill}%
\pgfsetfillopacity{0.700000}%
\pgfsetlinewidth{0.000000pt}%
\definecolor{currentstroke}{rgb}{0.000000,0.000000,0.000000}%
\pgfsetstrokecolor{currentstroke}%
\pgfsetstrokeopacity{0.700000}%
\pgfsetdash{}{0pt}%
\pgfpathmoveto{\pgfqpoint{6.335938in}{0.550000in}}%
\pgfpathlineto{\pgfqpoint{6.363459in}{0.550000in}}%
\pgfpathlineto{\pgfqpoint{6.363459in}{0.696017in}}%
\pgfpathlineto{\pgfqpoint{6.335938in}{0.696017in}}%
\pgfpathlineto{\pgfqpoint{6.335938in}{0.550000in}}%
\pgfpathclose%
\pgfusepath{fill}%
\end{pgfscope}%
\begin{pgfscope}%
\pgfpathrectangle{\pgfqpoint{1.250000in}{0.550000in}}{\pgfqpoint{7.750000in}{3.850000in}}%
\pgfusepath{clip}%
\pgfsetbuttcap%
\pgfsetmiterjoin%
\definecolor{currentfill}{rgb}{0.501961,0.501961,0.501961}%
\pgfsetfillcolor{currentfill}%
\pgfsetfillopacity{0.700000}%
\pgfsetlinewidth{0.000000pt}%
\definecolor{currentstroke}{rgb}{0.000000,0.000000,0.000000}%
\pgfsetstrokecolor{currentstroke}%
\pgfsetstrokeopacity{0.700000}%
\pgfsetdash{}{0pt}%
\pgfpathmoveto{\pgfqpoint{6.363459in}{0.550000in}}%
\pgfpathlineto{\pgfqpoint{6.390980in}{0.550000in}}%
\pgfpathlineto{\pgfqpoint{6.390980in}{0.699642in}}%
\pgfpathlineto{\pgfqpoint{6.363459in}{0.699642in}}%
\pgfpathlineto{\pgfqpoint{6.363459in}{0.550000in}}%
\pgfpathclose%
\pgfusepath{fill}%
\end{pgfscope}%
\begin{pgfscope}%
\pgfpathrectangle{\pgfqpoint{1.250000in}{0.550000in}}{\pgfqpoint{7.750000in}{3.850000in}}%
\pgfusepath{clip}%
\pgfsetbuttcap%
\pgfsetmiterjoin%
\definecolor{currentfill}{rgb}{0.501961,0.501961,0.501961}%
\pgfsetfillcolor{currentfill}%
\pgfsetfillopacity{0.700000}%
\pgfsetlinewidth{0.000000pt}%
\definecolor{currentstroke}{rgb}{0.000000,0.000000,0.000000}%
\pgfsetstrokecolor{currentstroke}%
\pgfsetstrokeopacity{0.700000}%
\pgfsetdash{}{0pt}%
\pgfpathmoveto{\pgfqpoint{6.390980in}{0.550000in}}%
\pgfpathlineto{\pgfqpoint{6.418501in}{0.550000in}}%
\pgfpathlineto{\pgfqpoint{6.418501in}{0.690797in}}%
\pgfpathlineto{\pgfqpoint{6.390980in}{0.690797in}}%
\pgfpathlineto{\pgfqpoint{6.390980in}{0.550000in}}%
\pgfpathclose%
\pgfusepath{fill}%
\end{pgfscope}%
\begin{pgfscope}%
\pgfpathrectangle{\pgfqpoint{1.250000in}{0.550000in}}{\pgfqpoint{7.750000in}{3.850000in}}%
\pgfusepath{clip}%
\pgfsetbuttcap%
\pgfsetmiterjoin%
\definecolor{currentfill}{rgb}{0.501961,0.501961,0.501961}%
\pgfsetfillcolor{currentfill}%
\pgfsetfillopacity{0.700000}%
\pgfsetlinewidth{0.000000pt}%
\definecolor{currentstroke}{rgb}{0.000000,0.000000,0.000000}%
\pgfsetstrokecolor{currentstroke}%
\pgfsetstrokeopacity{0.700000}%
\pgfsetdash{}{0pt}%
\pgfpathmoveto{\pgfqpoint{6.418501in}{0.550000in}}%
\pgfpathlineto{\pgfqpoint{6.446023in}{0.550000in}}%
\pgfpathlineto{\pgfqpoint{6.446023in}{0.822169in}}%
\pgfpathlineto{\pgfqpoint{6.418501in}{0.822169in}}%
\pgfpathlineto{\pgfqpoint{6.418501in}{0.550000in}}%
\pgfpathclose%
\pgfusepath{fill}%
\end{pgfscope}%
\begin{pgfscope}%
\pgfpathrectangle{\pgfqpoint{1.250000in}{0.550000in}}{\pgfqpoint{7.750000in}{3.850000in}}%
\pgfusepath{clip}%
\pgfsetbuttcap%
\pgfsetmiterjoin%
\definecolor{currentfill}{rgb}{0.501961,0.501961,0.501961}%
\pgfsetfillcolor{currentfill}%
\pgfsetfillopacity{0.700000}%
\pgfsetlinewidth{0.000000pt}%
\definecolor{currentstroke}{rgb}{0.000000,0.000000,0.000000}%
\pgfsetstrokecolor{currentstroke}%
\pgfsetstrokeopacity{0.700000}%
\pgfsetdash{}{0pt}%
\pgfpathmoveto{\pgfqpoint{6.446023in}{0.550000in}}%
\pgfpathlineto{\pgfqpoint{6.473544in}{0.550000in}}%
\pgfpathlineto{\pgfqpoint{6.473544in}{0.677602in}}%
\pgfpathlineto{\pgfqpoint{6.446023in}{0.677602in}}%
\pgfpathlineto{\pgfqpoint{6.446023in}{0.550000in}}%
\pgfpathclose%
\pgfusepath{fill}%
\end{pgfscope}%
\begin{pgfscope}%
\pgfpathrectangle{\pgfqpoint{1.250000in}{0.550000in}}{\pgfqpoint{7.750000in}{3.850000in}}%
\pgfusepath{clip}%
\pgfsetbuttcap%
\pgfsetmiterjoin%
\definecolor{currentfill}{rgb}{0.501961,0.501961,0.501961}%
\pgfsetfillcolor{currentfill}%
\pgfsetfillopacity{0.700000}%
\pgfsetlinewidth{0.000000pt}%
\definecolor{currentstroke}{rgb}{0.000000,0.000000,0.000000}%
\pgfsetstrokecolor{currentstroke}%
\pgfsetstrokeopacity{0.700000}%
\pgfsetdash{}{0pt}%
\pgfpathmoveto{\pgfqpoint{6.473544in}{0.550000in}}%
\pgfpathlineto{\pgfqpoint{6.501065in}{0.550000in}}%
\pgfpathlineto{\pgfqpoint{6.501065in}{0.678617in}}%
\pgfpathlineto{\pgfqpoint{6.473544in}{0.678617in}}%
\pgfpathlineto{\pgfqpoint{6.473544in}{0.550000in}}%
\pgfpathclose%
\pgfusepath{fill}%
\end{pgfscope}%
\begin{pgfscope}%
\pgfpathrectangle{\pgfqpoint{1.250000in}{0.550000in}}{\pgfqpoint{7.750000in}{3.850000in}}%
\pgfusepath{clip}%
\pgfsetbuttcap%
\pgfsetmiterjoin%
\definecolor{currentfill}{rgb}{0.501961,0.501961,0.501961}%
\pgfsetfillcolor{currentfill}%
\pgfsetfillopacity{0.700000}%
\pgfsetlinewidth{0.000000pt}%
\definecolor{currentstroke}{rgb}{0.000000,0.000000,0.000000}%
\pgfsetstrokecolor{currentstroke}%
\pgfsetstrokeopacity{0.700000}%
\pgfsetdash{}{0pt}%
\pgfpathmoveto{\pgfqpoint{6.501065in}{0.550000in}}%
\pgfpathlineto{\pgfqpoint{6.528587in}{0.550000in}}%
\pgfpathlineto{\pgfqpoint{6.528587in}{0.677312in}}%
\pgfpathlineto{\pgfqpoint{6.501065in}{0.677312in}}%
\pgfpathlineto{\pgfqpoint{6.501065in}{0.550000in}}%
\pgfpathclose%
\pgfusepath{fill}%
\end{pgfscope}%
\begin{pgfscope}%
\pgfpathrectangle{\pgfqpoint{1.250000in}{0.550000in}}{\pgfqpoint{7.750000in}{3.850000in}}%
\pgfusepath{clip}%
\pgfsetbuttcap%
\pgfsetmiterjoin%
\definecolor{currentfill}{rgb}{0.501961,0.501961,0.501961}%
\pgfsetfillcolor{currentfill}%
\pgfsetfillopacity{0.700000}%
\pgfsetlinewidth{0.000000pt}%
\definecolor{currentstroke}{rgb}{0.000000,0.000000,0.000000}%
\pgfsetstrokecolor{currentstroke}%
\pgfsetstrokeopacity{0.700000}%
\pgfsetdash{}{0pt}%
\pgfpathmoveto{\pgfqpoint{6.528587in}{0.550000in}}%
\pgfpathlineto{\pgfqpoint{6.556108in}{0.550000in}}%
\pgfpathlineto{\pgfqpoint{6.556108in}{0.799839in}}%
\pgfpathlineto{\pgfqpoint{6.528587in}{0.799839in}}%
\pgfpathlineto{\pgfqpoint{6.528587in}{0.550000in}}%
\pgfpathclose%
\pgfusepath{fill}%
\end{pgfscope}%
\begin{pgfscope}%
\pgfpathrectangle{\pgfqpoint{1.250000in}{0.550000in}}{\pgfqpoint{7.750000in}{3.850000in}}%
\pgfusepath{clip}%
\pgfsetbuttcap%
\pgfsetmiterjoin%
\definecolor{currentfill}{rgb}{0.501961,0.501961,0.501961}%
\pgfsetfillcolor{currentfill}%
\pgfsetfillopacity{0.700000}%
\pgfsetlinewidth{0.000000pt}%
\definecolor{currentstroke}{rgb}{0.000000,0.000000,0.000000}%
\pgfsetstrokecolor{currentstroke}%
\pgfsetstrokeopacity{0.700000}%
\pgfsetdash{}{0pt}%
\pgfpathmoveto{\pgfqpoint{6.556108in}{0.550000in}}%
\pgfpathlineto{\pgfqpoint{6.583629in}{0.550000in}}%
\pgfpathlineto{\pgfqpoint{6.583629in}{0.675282in}}%
\pgfpathlineto{\pgfqpoint{6.556108in}{0.675282in}}%
\pgfpathlineto{\pgfqpoint{6.556108in}{0.550000in}}%
\pgfpathclose%
\pgfusepath{fill}%
\end{pgfscope}%
\begin{pgfscope}%
\pgfpathrectangle{\pgfqpoint{1.250000in}{0.550000in}}{\pgfqpoint{7.750000in}{3.850000in}}%
\pgfusepath{clip}%
\pgfsetbuttcap%
\pgfsetmiterjoin%
\definecolor{currentfill}{rgb}{0.501961,0.501961,0.501961}%
\pgfsetfillcolor{currentfill}%
\pgfsetfillopacity{0.700000}%
\pgfsetlinewidth{0.000000pt}%
\definecolor{currentstroke}{rgb}{0.000000,0.000000,0.000000}%
\pgfsetstrokecolor{currentstroke}%
\pgfsetstrokeopacity{0.700000}%
\pgfsetdash{}{0pt}%
\pgfpathmoveto{\pgfqpoint{6.583629in}{0.550000in}}%
\pgfpathlineto{\pgfqpoint{6.611151in}{0.550000in}}%
\pgfpathlineto{\pgfqpoint{6.611151in}{0.663827in}}%
\pgfpathlineto{\pgfqpoint{6.583629in}{0.663827in}}%
\pgfpathlineto{\pgfqpoint{6.583629in}{0.550000in}}%
\pgfpathclose%
\pgfusepath{fill}%
\end{pgfscope}%
\begin{pgfscope}%
\pgfpathrectangle{\pgfqpoint{1.250000in}{0.550000in}}{\pgfqpoint{7.750000in}{3.850000in}}%
\pgfusepath{clip}%
\pgfsetbuttcap%
\pgfsetmiterjoin%
\definecolor{currentfill}{rgb}{0.501961,0.501961,0.501961}%
\pgfsetfillcolor{currentfill}%
\pgfsetfillopacity{0.700000}%
\pgfsetlinewidth{0.000000pt}%
\definecolor{currentstroke}{rgb}{0.000000,0.000000,0.000000}%
\pgfsetstrokecolor{currentstroke}%
\pgfsetstrokeopacity{0.700000}%
\pgfsetdash{}{0pt}%
\pgfpathmoveto{\pgfqpoint{6.611151in}{0.550000in}}%
\pgfpathlineto{\pgfqpoint{6.638672in}{0.550000in}}%
\pgfpathlineto{\pgfqpoint{6.638672in}{0.779828in}}%
\pgfpathlineto{\pgfqpoint{6.611151in}{0.779828in}}%
\pgfpathlineto{\pgfqpoint{6.611151in}{0.550000in}}%
\pgfpathclose%
\pgfusepath{fill}%
\end{pgfscope}%
\begin{pgfscope}%
\pgfpathrectangle{\pgfqpoint{1.250000in}{0.550000in}}{\pgfqpoint{7.750000in}{3.850000in}}%
\pgfusepath{clip}%
\pgfsetbuttcap%
\pgfsetmiterjoin%
\definecolor{currentfill}{rgb}{0.501961,0.501961,0.501961}%
\pgfsetfillcolor{currentfill}%
\pgfsetfillopacity{0.700000}%
\pgfsetlinewidth{0.000000pt}%
\definecolor{currentstroke}{rgb}{0.000000,0.000000,0.000000}%
\pgfsetstrokecolor{currentstroke}%
\pgfsetstrokeopacity{0.700000}%
\pgfsetdash{}{0pt}%
\pgfpathmoveto{\pgfqpoint{6.638672in}{0.550000in}}%
\pgfpathlineto{\pgfqpoint{6.666193in}{0.550000in}}%
\pgfpathlineto{\pgfqpoint{6.666193in}{0.664697in}}%
\pgfpathlineto{\pgfqpoint{6.638672in}{0.664697in}}%
\pgfpathlineto{\pgfqpoint{6.638672in}{0.550000in}}%
\pgfpathclose%
\pgfusepath{fill}%
\end{pgfscope}%
\begin{pgfscope}%
\pgfpathrectangle{\pgfqpoint{1.250000in}{0.550000in}}{\pgfqpoint{7.750000in}{3.850000in}}%
\pgfusepath{clip}%
\pgfsetbuttcap%
\pgfsetmiterjoin%
\definecolor{currentfill}{rgb}{0.501961,0.501961,0.501961}%
\pgfsetfillcolor{currentfill}%
\pgfsetfillopacity{0.700000}%
\pgfsetlinewidth{0.000000pt}%
\definecolor{currentstroke}{rgb}{0.000000,0.000000,0.000000}%
\pgfsetstrokecolor{currentstroke}%
\pgfsetstrokeopacity{0.700000}%
\pgfsetdash{}{0pt}%
\pgfpathmoveto{\pgfqpoint{6.666193in}{0.550000in}}%
\pgfpathlineto{\pgfqpoint{6.693714in}{0.550000in}}%
\pgfpathlineto{\pgfqpoint{6.693714in}{0.763153in}}%
\pgfpathlineto{\pgfqpoint{6.666193in}{0.763153in}}%
\pgfpathlineto{\pgfqpoint{6.666193in}{0.550000in}}%
\pgfpathclose%
\pgfusepath{fill}%
\end{pgfscope}%
\begin{pgfscope}%
\pgfpathrectangle{\pgfqpoint{1.250000in}{0.550000in}}{\pgfqpoint{7.750000in}{3.850000in}}%
\pgfusepath{clip}%
\pgfsetbuttcap%
\pgfsetmiterjoin%
\definecolor{currentfill}{rgb}{0.501961,0.501961,0.501961}%
\pgfsetfillcolor{currentfill}%
\pgfsetfillopacity{0.700000}%
\pgfsetlinewidth{0.000000pt}%
\definecolor{currentstroke}{rgb}{0.000000,0.000000,0.000000}%
\pgfsetstrokecolor{currentstroke}%
\pgfsetstrokeopacity{0.700000}%
\pgfsetdash{}{0pt}%
\pgfpathmoveto{\pgfqpoint{6.693714in}{0.550000in}}%
\pgfpathlineto{\pgfqpoint{6.721236in}{0.550000in}}%
\pgfpathlineto{\pgfqpoint{6.721236in}{0.658172in}}%
\pgfpathlineto{\pgfqpoint{6.693714in}{0.658172in}}%
\pgfpathlineto{\pgfqpoint{6.693714in}{0.550000in}}%
\pgfpathclose%
\pgfusepath{fill}%
\end{pgfscope}%
\begin{pgfscope}%
\pgfpathrectangle{\pgfqpoint{1.250000in}{0.550000in}}{\pgfqpoint{7.750000in}{3.850000in}}%
\pgfusepath{clip}%
\pgfsetbuttcap%
\pgfsetmiterjoin%
\definecolor{currentfill}{rgb}{0.501961,0.501961,0.501961}%
\pgfsetfillcolor{currentfill}%
\pgfsetfillopacity{0.700000}%
\pgfsetlinewidth{0.000000pt}%
\definecolor{currentstroke}{rgb}{0.000000,0.000000,0.000000}%
\pgfsetstrokecolor{currentstroke}%
\pgfsetstrokeopacity{0.700000}%
\pgfsetdash{}{0pt}%
\pgfpathmoveto{\pgfqpoint{6.721236in}{0.550000in}}%
\pgfpathlineto{\pgfqpoint{6.748757in}{0.550000in}}%
\pgfpathlineto{\pgfqpoint{6.748757in}{0.767503in}}%
\pgfpathlineto{\pgfqpoint{6.721236in}{0.767503in}}%
\pgfpathlineto{\pgfqpoint{6.721236in}{0.550000in}}%
\pgfpathclose%
\pgfusepath{fill}%
\end{pgfscope}%
\begin{pgfscope}%
\pgfpathrectangle{\pgfqpoint{1.250000in}{0.550000in}}{\pgfqpoint{7.750000in}{3.850000in}}%
\pgfusepath{clip}%
\pgfsetbuttcap%
\pgfsetmiterjoin%
\definecolor{currentfill}{rgb}{0.501961,0.501961,0.501961}%
\pgfsetfillcolor{currentfill}%
\pgfsetfillopacity{0.700000}%
\pgfsetlinewidth{0.000000pt}%
\definecolor{currentstroke}{rgb}{0.000000,0.000000,0.000000}%
\pgfsetstrokecolor{currentstroke}%
\pgfsetstrokeopacity{0.700000}%
\pgfsetdash{}{0pt}%
\pgfpathmoveto{\pgfqpoint{6.748757in}{0.550000in}}%
\pgfpathlineto{\pgfqpoint{6.776278in}{0.550000in}}%
\pgfpathlineto{\pgfqpoint{6.776278in}{0.651501in}}%
\pgfpathlineto{\pgfqpoint{6.748757in}{0.651501in}}%
\pgfpathlineto{\pgfqpoint{6.748757in}{0.550000in}}%
\pgfpathclose%
\pgfusepath{fill}%
\end{pgfscope}%
\begin{pgfscope}%
\pgfpathrectangle{\pgfqpoint{1.250000in}{0.550000in}}{\pgfqpoint{7.750000in}{3.850000in}}%
\pgfusepath{clip}%
\pgfsetbuttcap%
\pgfsetmiterjoin%
\definecolor{currentfill}{rgb}{0.501961,0.501961,0.501961}%
\pgfsetfillcolor{currentfill}%
\pgfsetfillopacity{0.700000}%
\pgfsetlinewidth{0.000000pt}%
\definecolor{currentstroke}{rgb}{0.000000,0.000000,0.000000}%
\pgfsetstrokecolor{currentstroke}%
\pgfsetstrokeopacity{0.700000}%
\pgfsetdash{}{0pt}%
\pgfpathmoveto{\pgfqpoint{6.776278in}{0.550000in}}%
\pgfpathlineto{\pgfqpoint{6.803800in}{0.550000in}}%
\pgfpathlineto{\pgfqpoint{6.803800in}{0.767503in}}%
\pgfpathlineto{\pgfqpoint{6.776278in}{0.767503in}}%
\pgfpathlineto{\pgfqpoint{6.776278in}{0.550000in}}%
\pgfpathclose%
\pgfusepath{fill}%
\end{pgfscope}%
\begin{pgfscope}%
\pgfpathrectangle{\pgfqpoint{1.250000in}{0.550000in}}{\pgfqpoint{7.750000in}{3.850000in}}%
\pgfusepath{clip}%
\pgfsetbuttcap%
\pgfsetmiterjoin%
\definecolor{currentfill}{rgb}{0.501961,0.501961,0.501961}%
\pgfsetfillcolor{currentfill}%
\pgfsetfillopacity{0.700000}%
\pgfsetlinewidth{0.000000pt}%
\definecolor{currentstroke}{rgb}{0.000000,0.000000,0.000000}%
\pgfsetstrokecolor{currentstroke}%
\pgfsetstrokeopacity{0.700000}%
\pgfsetdash{}{0pt}%
\pgfpathmoveto{\pgfqpoint{6.803800in}{0.550000in}}%
\pgfpathlineto{\pgfqpoint{6.831321in}{0.550000in}}%
\pgfpathlineto{\pgfqpoint{6.831321in}{0.649906in}}%
\pgfpathlineto{\pgfqpoint{6.803800in}{0.649906in}}%
\pgfpathlineto{\pgfqpoint{6.803800in}{0.550000in}}%
\pgfpathclose%
\pgfusepath{fill}%
\end{pgfscope}%
\begin{pgfscope}%
\pgfpathrectangle{\pgfqpoint{1.250000in}{0.550000in}}{\pgfqpoint{7.750000in}{3.850000in}}%
\pgfusepath{clip}%
\pgfsetbuttcap%
\pgfsetmiterjoin%
\definecolor{currentfill}{rgb}{0.501961,0.501961,0.501961}%
\pgfsetfillcolor{currentfill}%
\pgfsetfillopacity{0.700000}%
\pgfsetlinewidth{0.000000pt}%
\definecolor{currentstroke}{rgb}{0.000000,0.000000,0.000000}%
\pgfsetstrokecolor{currentstroke}%
\pgfsetstrokeopacity{0.700000}%
\pgfsetdash{}{0pt}%
\pgfpathmoveto{\pgfqpoint{6.831321in}{0.550000in}}%
\pgfpathlineto{\pgfqpoint{6.858842in}{0.550000in}}%
\pgfpathlineto{\pgfqpoint{6.858842in}{0.755033in}}%
\pgfpathlineto{\pgfqpoint{6.831321in}{0.755033in}}%
\pgfpathlineto{\pgfqpoint{6.831321in}{0.550000in}}%
\pgfpathclose%
\pgfusepath{fill}%
\end{pgfscope}%
\begin{pgfscope}%
\pgfpathrectangle{\pgfqpoint{1.250000in}{0.550000in}}{\pgfqpoint{7.750000in}{3.850000in}}%
\pgfusepath{clip}%
\pgfsetbuttcap%
\pgfsetmiterjoin%
\definecolor{currentfill}{rgb}{0.501961,0.501961,0.501961}%
\pgfsetfillcolor{currentfill}%
\pgfsetfillopacity{0.700000}%
\pgfsetlinewidth{0.000000pt}%
\definecolor{currentstroke}{rgb}{0.000000,0.000000,0.000000}%
\pgfsetstrokecolor{currentstroke}%
\pgfsetstrokeopacity{0.700000}%
\pgfsetdash{}{0pt}%
\pgfpathmoveto{\pgfqpoint{6.858842in}{0.550000in}}%
\pgfpathlineto{\pgfqpoint{6.886364in}{0.550000in}}%
\pgfpathlineto{\pgfqpoint{6.886364in}{0.650196in}}%
\pgfpathlineto{\pgfqpoint{6.858842in}{0.650196in}}%
\pgfpathlineto{\pgfqpoint{6.858842in}{0.550000in}}%
\pgfpathclose%
\pgfusepath{fill}%
\end{pgfscope}%
\begin{pgfscope}%
\pgfpathrectangle{\pgfqpoint{1.250000in}{0.550000in}}{\pgfqpoint{7.750000in}{3.850000in}}%
\pgfusepath{clip}%
\pgfsetbuttcap%
\pgfsetmiterjoin%
\definecolor{currentfill}{rgb}{0.501961,0.501961,0.501961}%
\pgfsetfillcolor{currentfill}%
\pgfsetfillopacity{0.700000}%
\pgfsetlinewidth{0.000000pt}%
\definecolor{currentstroke}{rgb}{0.000000,0.000000,0.000000}%
\pgfsetstrokecolor{currentstroke}%
\pgfsetstrokeopacity{0.700000}%
\pgfsetdash{}{0pt}%
\pgfpathmoveto{\pgfqpoint{6.886364in}{0.550000in}}%
\pgfpathlineto{\pgfqpoint{6.913885in}{0.550000in}}%
\pgfpathlineto{\pgfqpoint{6.913885in}{0.751408in}}%
\pgfpathlineto{\pgfqpoint{6.886364in}{0.751408in}}%
\pgfpathlineto{\pgfqpoint{6.886364in}{0.550000in}}%
\pgfpathclose%
\pgfusepath{fill}%
\end{pgfscope}%
\begin{pgfscope}%
\pgfpathrectangle{\pgfqpoint{1.250000in}{0.550000in}}{\pgfqpoint{7.750000in}{3.850000in}}%
\pgfusepath{clip}%
\pgfsetbuttcap%
\pgfsetmiterjoin%
\definecolor{currentfill}{rgb}{0.501961,0.501961,0.501961}%
\pgfsetfillcolor{currentfill}%
\pgfsetfillopacity{0.700000}%
\pgfsetlinewidth{0.000000pt}%
\definecolor{currentstroke}{rgb}{0.000000,0.000000,0.000000}%
\pgfsetstrokecolor{currentstroke}%
\pgfsetstrokeopacity{0.700000}%
\pgfsetdash{}{0pt}%
\pgfpathmoveto{\pgfqpoint{6.913885in}{0.550000in}}%
\pgfpathlineto{\pgfqpoint{6.941406in}{0.550000in}}%
\pgfpathlineto{\pgfqpoint{6.941406in}{0.746768in}}%
\pgfpathlineto{\pgfqpoint{6.913885in}{0.746768in}}%
\pgfpathlineto{\pgfqpoint{6.913885in}{0.550000in}}%
\pgfpathclose%
\pgfusepath{fill}%
\end{pgfscope}%
\begin{pgfscope}%
\pgfpathrectangle{\pgfqpoint{1.250000in}{0.550000in}}{\pgfqpoint{7.750000in}{3.850000in}}%
\pgfusepath{clip}%
\pgfsetbuttcap%
\pgfsetmiterjoin%
\definecolor{currentfill}{rgb}{0.501961,0.501961,0.501961}%
\pgfsetfillcolor{currentfill}%
\pgfsetfillopacity{0.700000}%
\pgfsetlinewidth{0.000000pt}%
\definecolor{currentstroke}{rgb}{0.000000,0.000000,0.000000}%
\pgfsetstrokecolor{currentstroke}%
\pgfsetstrokeopacity{0.700000}%
\pgfsetdash{}{0pt}%
\pgfpathmoveto{\pgfqpoint{6.941406in}{0.550000in}}%
\pgfpathlineto{\pgfqpoint{6.968928in}{0.550000in}}%
\pgfpathlineto{\pgfqpoint{6.968928in}{0.642221in}}%
\pgfpathlineto{\pgfqpoint{6.941406in}{0.642221in}}%
\pgfpathlineto{\pgfqpoint{6.941406in}{0.550000in}}%
\pgfpathclose%
\pgfusepath{fill}%
\end{pgfscope}%
\begin{pgfscope}%
\pgfpathrectangle{\pgfqpoint{1.250000in}{0.550000in}}{\pgfqpoint{7.750000in}{3.850000in}}%
\pgfusepath{clip}%
\pgfsetbuttcap%
\pgfsetmiterjoin%
\definecolor{currentfill}{rgb}{0.501961,0.501961,0.501961}%
\pgfsetfillcolor{currentfill}%
\pgfsetfillopacity{0.700000}%
\pgfsetlinewidth{0.000000pt}%
\definecolor{currentstroke}{rgb}{0.000000,0.000000,0.000000}%
\pgfsetstrokecolor{currentstroke}%
\pgfsetstrokeopacity{0.700000}%
\pgfsetdash{}{0pt}%
\pgfpathmoveto{\pgfqpoint{6.968928in}{0.550000in}}%
\pgfpathlineto{\pgfqpoint{6.996449in}{0.550000in}}%
\pgfpathlineto{\pgfqpoint{6.996449in}{0.751118in}}%
\pgfpathlineto{\pgfqpoint{6.968928in}{0.751118in}}%
\pgfpathlineto{\pgfqpoint{6.968928in}{0.550000in}}%
\pgfpathclose%
\pgfusepath{fill}%
\end{pgfscope}%
\begin{pgfscope}%
\pgfpathrectangle{\pgfqpoint{1.250000in}{0.550000in}}{\pgfqpoint{7.750000in}{3.850000in}}%
\pgfusepath{clip}%
\pgfsetbuttcap%
\pgfsetmiterjoin%
\definecolor{currentfill}{rgb}{0.501961,0.501961,0.501961}%
\pgfsetfillcolor{currentfill}%
\pgfsetfillopacity{0.700000}%
\pgfsetlinewidth{0.000000pt}%
\definecolor{currentstroke}{rgb}{0.000000,0.000000,0.000000}%
\pgfsetstrokecolor{currentstroke}%
\pgfsetstrokeopacity{0.700000}%
\pgfsetdash{}{0pt}%
\pgfpathmoveto{\pgfqpoint{6.996449in}{0.550000in}}%
\pgfpathlineto{\pgfqpoint{7.023970in}{0.550000in}}%
\pgfpathlineto{\pgfqpoint{7.023970in}{0.740533in}}%
\pgfpathlineto{\pgfqpoint{6.996449in}{0.740533in}}%
\pgfpathlineto{\pgfqpoint{6.996449in}{0.550000in}}%
\pgfpathclose%
\pgfusepath{fill}%
\end{pgfscope}%
\begin{pgfscope}%
\pgfpathrectangle{\pgfqpoint{1.250000in}{0.550000in}}{\pgfqpoint{7.750000in}{3.850000in}}%
\pgfusepath{clip}%
\pgfsetbuttcap%
\pgfsetmiterjoin%
\definecolor{currentfill}{rgb}{0.501961,0.501961,0.501961}%
\pgfsetfillcolor{currentfill}%
\pgfsetfillopacity{0.700000}%
\pgfsetlinewidth{0.000000pt}%
\definecolor{currentstroke}{rgb}{0.000000,0.000000,0.000000}%
\pgfsetstrokecolor{currentstroke}%
\pgfsetstrokeopacity{0.700000}%
\pgfsetdash{}{0pt}%
\pgfpathmoveto{\pgfqpoint{7.023970in}{0.550000in}}%
\pgfpathlineto{\pgfqpoint{7.051491in}{0.550000in}}%
\pgfpathlineto{\pgfqpoint{7.051491in}{0.647731in}}%
\pgfpathlineto{\pgfqpoint{7.023970in}{0.647731in}}%
\pgfpathlineto{\pgfqpoint{7.023970in}{0.550000in}}%
\pgfpathclose%
\pgfusepath{fill}%
\end{pgfscope}%
\begin{pgfscope}%
\pgfpathrectangle{\pgfqpoint{1.250000in}{0.550000in}}{\pgfqpoint{7.750000in}{3.850000in}}%
\pgfusepath{clip}%
\pgfsetbuttcap%
\pgfsetmiterjoin%
\definecolor{currentfill}{rgb}{0.501961,0.501961,0.501961}%
\pgfsetfillcolor{currentfill}%
\pgfsetfillopacity{0.700000}%
\pgfsetlinewidth{0.000000pt}%
\definecolor{currentstroke}{rgb}{0.000000,0.000000,0.000000}%
\pgfsetstrokecolor{currentstroke}%
\pgfsetstrokeopacity{0.700000}%
\pgfsetdash{}{0pt}%
\pgfpathmoveto{\pgfqpoint{7.051491in}{0.550000in}}%
\pgfpathlineto{\pgfqpoint{7.079013in}{0.550000in}}%
\pgfpathlineto{\pgfqpoint{7.079013in}{0.748508in}}%
\pgfpathlineto{\pgfqpoint{7.051491in}{0.748508in}}%
\pgfpathlineto{\pgfqpoint{7.051491in}{0.550000in}}%
\pgfpathclose%
\pgfusepath{fill}%
\end{pgfscope}%
\begin{pgfscope}%
\pgfpathrectangle{\pgfqpoint{1.250000in}{0.550000in}}{\pgfqpoint{7.750000in}{3.850000in}}%
\pgfusepath{clip}%
\pgfsetbuttcap%
\pgfsetmiterjoin%
\definecolor{currentfill}{rgb}{0.501961,0.501961,0.501961}%
\pgfsetfillcolor{currentfill}%
\pgfsetfillopacity{0.700000}%
\pgfsetlinewidth{0.000000pt}%
\definecolor{currentstroke}{rgb}{0.000000,0.000000,0.000000}%
\pgfsetstrokecolor{currentstroke}%
\pgfsetstrokeopacity{0.700000}%
\pgfsetdash{}{0pt}%
\pgfpathmoveto{\pgfqpoint{7.079013in}{0.550000in}}%
\pgfpathlineto{\pgfqpoint{7.106534in}{0.550000in}}%
\pgfpathlineto{\pgfqpoint{7.106534in}{0.749813in}}%
\pgfpathlineto{\pgfqpoint{7.079013in}{0.749813in}}%
\pgfpathlineto{\pgfqpoint{7.079013in}{0.550000in}}%
\pgfpathclose%
\pgfusepath{fill}%
\end{pgfscope}%
\begin{pgfscope}%
\pgfpathrectangle{\pgfqpoint{1.250000in}{0.550000in}}{\pgfqpoint{7.750000in}{3.850000in}}%
\pgfusepath{clip}%
\pgfsetbuttcap%
\pgfsetmiterjoin%
\definecolor{currentfill}{rgb}{0.501961,0.501961,0.501961}%
\pgfsetfillcolor{currentfill}%
\pgfsetfillopacity{0.700000}%
\pgfsetlinewidth{0.000000pt}%
\definecolor{currentstroke}{rgb}{0.000000,0.000000,0.000000}%
\pgfsetstrokecolor{currentstroke}%
\pgfsetstrokeopacity{0.700000}%
\pgfsetdash{}{0pt}%
\pgfpathmoveto{\pgfqpoint{7.106534in}{0.550000in}}%
\pgfpathlineto{\pgfqpoint{7.134055in}{0.550000in}}%
\pgfpathlineto{\pgfqpoint{7.134055in}{0.657012in}}%
\pgfpathlineto{\pgfqpoint{7.106534in}{0.657012in}}%
\pgfpathlineto{\pgfqpoint{7.106534in}{0.550000in}}%
\pgfpathclose%
\pgfusepath{fill}%
\end{pgfscope}%
\begin{pgfscope}%
\pgfpathrectangle{\pgfqpoint{1.250000in}{0.550000in}}{\pgfqpoint{7.750000in}{3.850000in}}%
\pgfusepath{clip}%
\pgfsetbuttcap%
\pgfsetmiterjoin%
\definecolor{currentfill}{rgb}{0.501961,0.501961,0.501961}%
\pgfsetfillcolor{currentfill}%
\pgfsetfillopacity{0.700000}%
\pgfsetlinewidth{0.000000pt}%
\definecolor{currentstroke}{rgb}{0.000000,0.000000,0.000000}%
\pgfsetstrokecolor{currentstroke}%
\pgfsetstrokeopacity{0.700000}%
\pgfsetdash{}{0pt}%
\pgfpathmoveto{\pgfqpoint{7.134055in}{0.550000in}}%
\pgfpathlineto{\pgfqpoint{7.161577in}{0.550000in}}%
\pgfpathlineto{\pgfqpoint{7.161577in}{0.749233in}}%
\pgfpathlineto{\pgfqpoint{7.134055in}{0.749233in}}%
\pgfpathlineto{\pgfqpoint{7.134055in}{0.550000in}}%
\pgfpathclose%
\pgfusepath{fill}%
\end{pgfscope}%
\begin{pgfscope}%
\pgfpathrectangle{\pgfqpoint{1.250000in}{0.550000in}}{\pgfqpoint{7.750000in}{3.850000in}}%
\pgfusepath{clip}%
\pgfsetbuttcap%
\pgfsetmiterjoin%
\definecolor{currentfill}{rgb}{0.501961,0.501961,0.501961}%
\pgfsetfillcolor{currentfill}%
\pgfsetfillopacity{0.700000}%
\pgfsetlinewidth{0.000000pt}%
\definecolor{currentstroke}{rgb}{0.000000,0.000000,0.000000}%
\pgfsetstrokecolor{currentstroke}%
\pgfsetstrokeopacity{0.700000}%
\pgfsetdash{}{0pt}%
\pgfpathmoveto{\pgfqpoint{7.161577in}{0.550000in}}%
\pgfpathlineto{\pgfqpoint{7.189098in}{0.550000in}}%
\pgfpathlineto{\pgfqpoint{7.189098in}{0.649906in}}%
\pgfpathlineto{\pgfqpoint{7.161577in}{0.649906in}}%
\pgfpathlineto{\pgfqpoint{7.161577in}{0.550000in}}%
\pgfpathclose%
\pgfusepath{fill}%
\end{pgfscope}%
\begin{pgfscope}%
\pgfpathrectangle{\pgfqpoint{1.250000in}{0.550000in}}{\pgfqpoint{7.750000in}{3.850000in}}%
\pgfusepath{clip}%
\pgfsetbuttcap%
\pgfsetmiterjoin%
\definecolor{currentfill}{rgb}{0.501961,0.501961,0.501961}%
\pgfsetfillcolor{currentfill}%
\pgfsetfillopacity{0.700000}%
\pgfsetlinewidth{0.000000pt}%
\definecolor{currentstroke}{rgb}{0.000000,0.000000,0.000000}%
\pgfsetstrokecolor{currentstroke}%
\pgfsetstrokeopacity{0.700000}%
\pgfsetdash{}{0pt}%
\pgfpathmoveto{\pgfqpoint{7.189098in}{0.550000in}}%
\pgfpathlineto{\pgfqpoint{7.216619in}{0.550000in}}%
\pgfpathlineto{\pgfqpoint{7.216619in}{0.752858in}}%
\pgfpathlineto{\pgfqpoint{7.189098in}{0.752858in}}%
\pgfpathlineto{\pgfqpoint{7.189098in}{0.550000in}}%
\pgfpathclose%
\pgfusepath{fill}%
\end{pgfscope}%
\begin{pgfscope}%
\pgfpathrectangle{\pgfqpoint{1.250000in}{0.550000in}}{\pgfqpoint{7.750000in}{3.850000in}}%
\pgfusepath{clip}%
\pgfsetbuttcap%
\pgfsetmiterjoin%
\definecolor{currentfill}{rgb}{0.501961,0.501961,0.501961}%
\pgfsetfillcolor{currentfill}%
\pgfsetfillopacity{0.700000}%
\pgfsetlinewidth{0.000000pt}%
\definecolor{currentstroke}{rgb}{0.000000,0.000000,0.000000}%
\pgfsetstrokecolor{currentstroke}%
\pgfsetstrokeopacity{0.700000}%
\pgfsetdash{}{0pt}%
\pgfpathmoveto{\pgfqpoint{7.216619in}{0.550000in}}%
\pgfpathlineto{\pgfqpoint{7.244141in}{0.550000in}}%
\pgfpathlineto{\pgfqpoint{7.244141in}{0.744158in}}%
\pgfpathlineto{\pgfqpoint{7.216619in}{0.744158in}}%
\pgfpathlineto{\pgfqpoint{7.216619in}{0.550000in}}%
\pgfpathclose%
\pgfusepath{fill}%
\end{pgfscope}%
\begin{pgfscope}%
\pgfpathrectangle{\pgfqpoint{1.250000in}{0.550000in}}{\pgfqpoint{7.750000in}{3.850000in}}%
\pgfusepath{clip}%
\pgfsetbuttcap%
\pgfsetmiterjoin%
\definecolor{currentfill}{rgb}{0.501961,0.501961,0.501961}%
\pgfsetfillcolor{currentfill}%
\pgfsetfillopacity{0.700000}%
\pgfsetlinewidth{0.000000pt}%
\definecolor{currentstroke}{rgb}{0.000000,0.000000,0.000000}%
\pgfsetstrokecolor{currentstroke}%
\pgfsetstrokeopacity{0.700000}%
\pgfsetdash{}{0pt}%
\pgfpathmoveto{\pgfqpoint{7.244141in}{0.550000in}}%
\pgfpathlineto{\pgfqpoint{7.271662in}{0.550000in}}%
\pgfpathlineto{\pgfqpoint{7.271662in}{0.646281in}}%
\pgfpathlineto{\pgfqpoint{7.244141in}{0.646281in}}%
\pgfpathlineto{\pgfqpoint{7.244141in}{0.550000in}}%
\pgfpathclose%
\pgfusepath{fill}%
\end{pgfscope}%
\begin{pgfscope}%
\pgfpathrectangle{\pgfqpoint{1.250000in}{0.550000in}}{\pgfqpoint{7.750000in}{3.850000in}}%
\pgfusepath{clip}%
\pgfsetbuttcap%
\pgfsetmiterjoin%
\definecolor{currentfill}{rgb}{0.501961,0.501961,0.501961}%
\pgfsetfillcolor{currentfill}%
\pgfsetfillopacity{0.700000}%
\pgfsetlinewidth{0.000000pt}%
\definecolor{currentstroke}{rgb}{0.000000,0.000000,0.000000}%
\pgfsetstrokecolor{currentstroke}%
\pgfsetstrokeopacity{0.700000}%
\pgfsetdash{}{0pt}%
\pgfpathmoveto{\pgfqpoint{7.271662in}{0.550000in}}%
\pgfpathlineto{\pgfqpoint{7.299183in}{0.550000in}}%
\pgfpathlineto{\pgfqpoint{7.299183in}{0.744738in}}%
\pgfpathlineto{\pgfqpoint{7.271662in}{0.744738in}}%
\pgfpathlineto{\pgfqpoint{7.271662in}{0.550000in}}%
\pgfpathclose%
\pgfusepath{fill}%
\end{pgfscope}%
\begin{pgfscope}%
\pgfpathrectangle{\pgfqpoint{1.250000in}{0.550000in}}{\pgfqpoint{7.750000in}{3.850000in}}%
\pgfusepath{clip}%
\pgfsetbuttcap%
\pgfsetmiterjoin%
\definecolor{currentfill}{rgb}{0.501961,0.501961,0.501961}%
\pgfsetfillcolor{currentfill}%
\pgfsetfillopacity{0.700000}%
\pgfsetlinewidth{0.000000pt}%
\definecolor{currentstroke}{rgb}{0.000000,0.000000,0.000000}%
\pgfsetstrokecolor{currentstroke}%
\pgfsetstrokeopacity{0.700000}%
\pgfsetdash{}{0pt}%
\pgfpathmoveto{\pgfqpoint{7.299183in}{0.550000in}}%
\pgfpathlineto{\pgfqpoint{7.326705in}{0.550000in}}%
\pgfpathlineto{\pgfqpoint{7.326705in}{0.656141in}}%
\pgfpathlineto{\pgfqpoint{7.299183in}{0.656141in}}%
\pgfpathlineto{\pgfqpoint{7.299183in}{0.550000in}}%
\pgfpathclose%
\pgfusepath{fill}%
\end{pgfscope}%
\begin{pgfscope}%
\pgfpathrectangle{\pgfqpoint{1.250000in}{0.550000in}}{\pgfqpoint{7.750000in}{3.850000in}}%
\pgfusepath{clip}%
\pgfsetbuttcap%
\pgfsetmiterjoin%
\definecolor{currentfill}{rgb}{0.501961,0.501961,0.501961}%
\pgfsetfillcolor{currentfill}%
\pgfsetfillopacity{0.700000}%
\pgfsetlinewidth{0.000000pt}%
\definecolor{currentstroke}{rgb}{0.000000,0.000000,0.000000}%
\pgfsetstrokecolor{currentstroke}%
\pgfsetstrokeopacity{0.700000}%
\pgfsetdash{}{0pt}%
\pgfpathmoveto{\pgfqpoint{7.326705in}{0.550000in}}%
\pgfpathlineto{\pgfqpoint{7.354226in}{0.550000in}}%
\pgfpathlineto{\pgfqpoint{7.354226in}{0.741693in}}%
\pgfpathlineto{\pgfqpoint{7.326705in}{0.741693in}}%
\pgfpathlineto{\pgfqpoint{7.326705in}{0.550000in}}%
\pgfpathclose%
\pgfusepath{fill}%
\end{pgfscope}%
\begin{pgfscope}%
\pgfpathrectangle{\pgfqpoint{1.250000in}{0.550000in}}{\pgfqpoint{7.750000in}{3.850000in}}%
\pgfusepath{clip}%
\pgfsetbuttcap%
\pgfsetmiterjoin%
\definecolor{currentfill}{rgb}{0.501961,0.501961,0.501961}%
\pgfsetfillcolor{currentfill}%
\pgfsetfillopacity{0.700000}%
\pgfsetlinewidth{0.000000pt}%
\definecolor{currentstroke}{rgb}{0.000000,0.000000,0.000000}%
\pgfsetstrokecolor{currentstroke}%
\pgfsetstrokeopacity{0.700000}%
\pgfsetdash{}{0pt}%
\pgfpathmoveto{\pgfqpoint{7.354226in}{0.550000in}}%
\pgfpathlineto{\pgfqpoint{7.381747in}{0.550000in}}%
\pgfpathlineto{\pgfqpoint{7.381747in}{0.737633in}}%
\pgfpathlineto{\pgfqpoint{7.354226in}{0.737633in}}%
\pgfpathlineto{\pgfqpoint{7.354226in}{0.550000in}}%
\pgfpathclose%
\pgfusepath{fill}%
\end{pgfscope}%
\begin{pgfscope}%
\pgfpathrectangle{\pgfqpoint{1.250000in}{0.550000in}}{\pgfqpoint{7.750000in}{3.850000in}}%
\pgfusepath{clip}%
\pgfsetbuttcap%
\pgfsetmiterjoin%
\definecolor{currentfill}{rgb}{0.501961,0.501961,0.501961}%
\pgfsetfillcolor{currentfill}%
\pgfsetfillopacity{0.700000}%
\pgfsetlinewidth{0.000000pt}%
\definecolor{currentstroke}{rgb}{0.000000,0.000000,0.000000}%
\pgfsetstrokecolor{currentstroke}%
\pgfsetstrokeopacity{0.700000}%
\pgfsetdash{}{0pt}%
\pgfpathmoveto{\pgfqpoint{7.381747in}{0.550000in}}%
\pgfpathlineto{\pgfqpoint{7.409268in}{0.550000in}}%
\pgfpathlineto{\pgfqpoint{7.409268in}{0.732848in}}%
\pgfpathlineto{\pgfqpoint{7.381747in}{0.732848in}}%
\pgfpathlineto{\pgfqpoint{7.381747in}{0.550000in}}%
\pgfpathclose%
\pgfusepath{fill}%
\end{pgfscope}%
\begin{pgfscope}%
\pgfpathrectangle{\pgfqpoint{1.250000in}{0.550000in}}{\pgfqpoint{7.750000in}{3.850000in}}%
\pgfusepath{clip}%
\pgfsetbuttcap%
\pgfsetmiterjoin%
\definecolor{currentfill}{rgb}{0.501961,0.501961,0.501961}%
\pgfsetfillcolor{currentfill}%
\pgfsetfillopacity{0.700000}%
\pgfsetlinewidth{0.000000pt}%
\definecolor{currentstroke}{rgb}{0.000000,0.000000,0.000000}%
\pgfsetstrokecolor{currentstroke}%
\pgfsetstrokeopacity{0.700000}%
\pgfsetdash{}{0pt}%
\pgfpathmoveto{\pgfqpoint{7.409268in}{0.550000in}}%
\pgfpathlineto{\pgfqpoint{7.436790in}{0.550000in}}%
\pgfpathlineto{\pgfqpoint{7.436790in}{0.635116in}}%
\pgfpathlineto{\pgfqpoint{7.409268in}{0.635116in}}%
\pgfpathlineto{\pgfqpoint{7.409268in}{0.550000in}}%
\pgfpathclose%
\pgfusepath{fill}%
\end{pgfscope}%
\begin{pgfscope}%
\pgfpathrectangle{\pgfqpoint{1.250000in}{0.550000in}}{\pgfqpoint{7.750000in}{3.850000in}}%
\pgfusepath{clip}%
\pgfsetbuttcap%
\pgfsetmiterjoin%
\definecolor{currentfill}{rgb}{0.501961,0.501961,0.501961}%
\pgfsetfillcolor{currentfill}%
\pgfsetfillopacity{0.700000}%
\pgfsetlinewidth{0.000000pt}%
\definecolor{currentstroke}{rgb}{0.000000,0.000000,0.000000}%
\pgfsetstrokecolor{currentstroke}%
\pgfsetstrokeopacity{0.700000}%
\pgfsetdash{}{0pt}%
\pgfpathmoveto{\pgfqpoint{7.436790in}{0.550000in}}%
\pgfpathlineto{\pgfqpoint{7.464311in}{0.550000in}}%
\pgfpathlineto{\pgfqpoint{7.464311in}{0.722697in}}%
\pgfpathlineto{\pgfqpoint{7.436790in}{0.722697in}}%
\pgfpathlineto{\pgfqpoint{7.436790in}{0.550000in}}%
\pgfpathclose%
\pgfusepath{fill}%
\end{pgfscope}%
\begin{pgfscope}%
\pgfpathrectangle{\pgfqpoint{1.250000in}{0.550000in}}{\pgfqpoint{7.750000in}{3.850000in}}%
\pgfusepath{clip}%
\pgfsetbuttcap%
\pgfsetmiterjoin%
\definecolor{currentfill}{rgb}{0.501961,0.501961,0.501961}%
\pgfsetfillcolor{currentfill}%
\pgfsetfillopacity{0.700000}%
\pgfsetlinewidth{0.000000pt}%
\definecolor{currentstroke}{rgb}{0.000000,0.000000,0.000000}%
\pgfsetstrokecolor{currentstroke}%
\pgfsetstrokeopacity{0.700000}%
\pgfsetdash{}{0pt}%
\pgfpathmoveto{\pgfqpoint{7.464311in}{0.550000in}}%
\pgfpathlineto{\pgfqpoint{7.491832in}{0.550000in}}%
\pgfpathlineto{\pgfqpoint{7.491832in}{0.712402in}}%
\pgfpathlineto{\pgfqpoint{7.464311in}{0.712402in}}%
\pgfpathlineto{\pgfqpoint{7.464311in}{0.550000in}}%
\pgfpathclose%
\pgfusepath{fill}%
\end{pgfscope}%
\begin{pgfscope}%
\pgfpathrectangle{\pgfqpoint{1.250000in}{0.550000in}}{\pgfqpoint{7.750000in}{3.850000in}}%
\pgfusepath{clip}%
\pgfsetbuttcap%
\pgfsetmiterjoin%
\definecolor{currentfill}{rgb}{0.501961,0.501961,0.501961}%
\pgfsetfillcolor{currentfill}%
\pgfsetfillopacity{0.700000}%
\pgfsetlinewidth{0.000000pt}%
\definecolor{currentstroke}{rgb}{0.000000,0.000000,0.000000}%
\pgfsetstrokecolor{currentstroke}%
\pgfsetstrokeopacity{0.700000}%
\pgfsetdash{}{0pt}%
\pgfpathmoveto{\pgfqpoint{7.491832in}{0.550000in}}%
\pgfpathlineto{\pgfqpoint{7.519354in}{0.550000in}}%
\pgfpathlineto{\pgfqpoint{7.519354in}{0.711967in}}%
\pgfpathlineto{\pgfqpoint{7.491832in}{0.711967in}}%
\pgfpathlineto{\pgfqpoint{7.491832in}{0.550000in}}%
\pgfpathclose%
\pgfusepath{fill}%
\end{pgfscope}%
\begin{pgfscope}%
\pgfpathrectangle{\pgfqpoint{1.250000in}{0.550000in}}{\pgfqpoint{7.750000in}{3.850000in}}%
\pgfusepath{clip}%
\pgfsetbuttcap%
\pgfsetmiterjoin%
\definecolor{currentfill}{rgb}{0.501961,0.501961,0.501961}%
\pgfsetfillcolor{currentfill}%
\pgfsetfillopacity{0.700000}%
\pgfsetlinewidth{0.000000pt}%
\definecolor{currentstroke}{rgb}{0.000000,0.000000,0.000000}%
\pgfsetstrokecolor{currentstroke}%
\pgfsetstrokeopacity{0.700000}%
\pgfsetdash{}{0pt}%
\pgfpathmoveto{\pgfqpoint{7.519354in}{0.550000in}}%
\pgfpathlineto{\pgfqpoint{7.546875in}{0.550000in}}%
\pgfpathlineto{\pgfqpoint{7.546875in}{0.767503in}}%
\pgfpathlineto{\pgfqpoint{7.519354in}{0.767503in}}%
\pgfpathlineto{\pgfqpoint{7.519354in}{0.550000in}}%
\pgfpathclose%
\pgfusepath{fill}%
\end{pgfscope}%
\begin{pgfscope}%
\pgfpathrectangle{\pgfqpoint{1.250000in}{0.550000in}}{\pgfqpoint{7.750000in}{3.850000in}}%
\pgfusepath{clip}%
\pgfsetbuttcap%
\pgfsetmiterjoin%
\definecolor{currentfill}{rgb}{0.501961,0.501961,0.501961}%
\pgfsetfillcolor{currentfill}%
\pgfsetfillopacity{0.700000}%
\pgfsetlinewidth{0.000000pt}%
\definecolor{currentstroke}{rgb}{0.000000,0.000000,0.000000}%
\pgfsetstrokecolor{currentstroke}%
\pgfsetstrokeopacity{0.700000}%
\pgfsetdash{}{0pt}%
\pgfpathmoveto{\pgfqpoint{7.546875in}{0.550000in}}%
\pgfpathlineto{\pgfqpoint{7.574396in}{0.550000in}}%
\pgfpathlineto{\pgfqpoint{7.574396in}{0.686012in}}%
\pgfpathlineto{\pgfqpoint{7.546875in}{0.686012in}}%
\pgfpathlineto{\pgfqpoint{7.546875in}{0.550000in}}%
\pgfpathclose%
\pgfusepath{fill}%
\end{pgfscope}%
\begin{pgfscope}%
\pgfpathrectangle{\pgfqpoint{1.250000in}{0.550000in}}{\pgfqpoint{7.750000in}{3.850000in}}%
\pgfusepath{clip}%
\pgfsetbuttcap%
\pgfsetmiterjoin%
\definecolor{currentfill}{rgb}{0.501961,0.501961,0.501961}%
\pgfsetfillcolor{currentfill}%
\pgfsetfillopacity{0.700000}%
\pgfsetlinewidth{0.000000pt}%
\definecolor{currentstroke}{rgb}{0.000000,0.000000,0.000000}%
\pgfsetstrokecolor{currentstroke}%
\pgfsetstrokeopacity{0.700000}%
\pgfsetdash{}{0pt}%
\pgfpathmoveto{\pgfqpoint{7.574396in}{0.550000in}}%
\pgfpathlineto{\pgfqpoint{7.601918in}{0.550000in}}%
\pgfpathlineto{\pgfqpoint{7.601918in}{0.686737in}}%
\pgfpathlineto{\pgfqpoint{7.574396in}{0.686737in}}%
\pgfpathlineto{\pgfqpoint{7.574396in}{0.550000in}}%
\pgfpathclose%
\pgfusepath{fill}%
\end{pgfscope}%
\begin{pgfscope}%
\pgfpathrectangle{\pgfqpoint{1.250000in}{0.550000in}}{\pgfqpoint{7.750000in}{3.850000in}}%
\pgfusepath{clip}%
\pgfsetbuttcap%
\pgfsetmiterjoin%
\definecolor{currentfill}{rgb}{0.501961,0.501961,0.501961}%
\pgfsetfillcolor{currentfill}%
\pgfsetfillopacity{0.700000}%
\pgfsetlinewidth{0.000000pt}%
\definecolor{currentstroke}{rgb}{0.000000,0.000000,0.000000}%
\pgfsetstrokecolor{currentstroke}%
\pgfsetstrokeopacity{0.700000}%
\pgfsetdash{}{0pt}%
\pgfpathmoveto{\pgfqpoint{7.601918in}{0.550000in}}%
\pgfpathlineto{\pgfqpoint{7.629439in}{0.550000in}}%
\pgfpathlineto{\pgfqpoint{7.629439in}{0.741983in}}%
\pgfpathlineto{\pgfqpoint{7.601918in}{0.741983in}}%
\pgfpathlineto{\pgfqpoint{7.601918in}{0.550000in}}%
\pgfpathclose%
\pgfusepath{fill}%
\end{pgfscope}%
\begin{pgfscope}%
\pgfpathrectangle{\pgfqpoint{1.250000in}{0.550000in}}{\pgfqpoint{7.750000in}{3.850000in}}%
\pgfusepath{clip}%
\pgfsetbuttcap%
\pgfsetmiterjoin%
\definecolor{currentfill}{rgb}{0.501961,0.501961,0.501961}%
\pgfsetfillcolor{currentfill}%
\pgfsetfillopacity{0.700000}%
\pgfsetlinewidth{0.000000pt}%
\definecolor{currentstroke}{rgb}{0.000000,0.000000,0.000000}%
\pgfsetstrokecolor{currentstroke}%
\pgfsetstrokeopacity{0.700000}%
\pgfsetdash{}{0pt}%
\pgfpathmoveto{\pgfqpoint{7.629439in}{0.550000in}}%
\pgfpathlineto{\pgfqpoint{7.656960in}{0.550000in}}%
\pgfpathlineto{\pgfqpoint{7.656960in}{0.679052in}}%
\pgfpathlineto{\pgfqpoint{7.629439in}{0.679052in}}%
\pgfpathlineto{\pgfqpoint{7.629439in}{0.550000in}}%
\pgfpathclose%
\pgfusepath{fill}%
\end{pgfscope}%
\begin{pgfscope}%
\pgfpathrectangle{\pgfqpoint{1.250000in}{0.550000in}}{\pgfqpoint{7.750000in}{3.850000in}}%
\pgfusepath{clip}%
\pgfsetbuttcap%
\pgfsetmiterjoin%
\definecolor{currentfill}{rgb}{0.501961,0.501961,0.501961}%
\pgfsetfillcolor{currentfill}%
\pgfsetfillopacity{0.700000}%
\pgfsetlinewidth{0.000000pt}%
\definecolor{currentstroke}{rgb}{0.000000,0.000000,0.000000}%
\pgfsetstrokecolor{currentstroke}%
\pgfsetstrokeopacity{0.700000}%
\pgfsetdash{}{0pt}%
\pgfpathmoveto{\pgfqpoint{7.656960in}{0.550000in}}%
\pgfpathlineto{\pgfqpoint{7.684482in}{0.550000in}}%
\pgfpathlineto{\pgfqpoint{7.684482in}{0.741113in}}%
\pgfpathlineto{\pgfqpoint{7.656960in}{0.741113in}}%
\pgfpathlineto{\pgfqpoint{7.656960in}{0.550000in}}%
\pgfpathclose%
\pgfusepath{fill}%
\end{pgfscope}%
\begin{pgfscope}%
\pgfpathrectangle{\pgfqpoint{1.250000in}{0.550000in}}{\pgfqpoint{7.750000in}{3.850000in}}%
\pgfusepath{clip}%
\pgfsetbuttcap%
\pgfsetmiterjoin%
\definecolor{currentfill}{rgb}{0.501961,0.501961,0.501961}%
\pgfsetfillcolor{currentfill}%
\pgfsetfillopacity{0.700000}%
\pgfsetlinewidth{0.000000pt}%
\definecolor{currentstroke}{rgb}{0.000000,0.000000,0.000000}%
\pgfsetstrokecolor{currentstroke}%
\pgfsetstrokeopacity{0.700000}%
\pgfsetdash{}{0pt}%
\pgfpathmoveto{\pgfqpoint{7.684482in}{0.550000in}}%
\pgfpathlineto{\pgfqpoint{7.712003in}{0.550000in}}%
\pgfpathlineto{\pgfqpoint{7.712003in}{0.732558in}}%
\pgfpathlineto{\pgfqpoint{7.684482in}{0.732558in}}%
\pgfpathlineto{\pgfqpoint{7.684482in}{0.550000in}}%
\pgfpathclose%
\pgfusepath{fill}%
\end{pgfscope}%
\begin{pgfscope}%
\pgfpathrectangle{\pgfqpoint{1.250000in}{0.550000in}}{\pgfqpoint{7.750000in}{3.850000in}}%
\pgfusepath{clip}%
\pgfsetbuttcap%
\pgfsetmiterjoin%
\definecolor{currentfill}{rgb}{0.501961,0.501961,0.501961}%
\pgfsetfillcolor{currentfill}%
\pgfsetfillopacity{0.700000}%
\pgfsetlinewidth{0.000000pt}%
\definecolor{currentstroke}{rgb}{0.000000,0.000000,0.000000}%
\pgfsetstrokecolor{currentstroke}%
\pgfsetstrokeopacity{0.700000}%
\pgfsetdash{}{0pt}%
\pgfpathmoveto{\pgfqpoint{7.712003in}{0.550000in}}%
\pgfpathlineto{\pgfqpoint{7.739524in}{0.550000in}}%
\pgfpathlineto{\pgfqpoint{7.739524in}{0.671367in}}%
\pgfpathlineto{\pgfqpoint{7.712003in}{0.671367in}}%
\pgfpathlineto{\pgfqpoint{7.712003in}{0.550000in}}%
\pgfpathclose%
\pgfusepath{fill}%
\end{pgfscope}%
\begin{pgfscope}%
\pgfpathrectangle{\pgfqpoint{1.250000in}{0.550000in}}{\pgfqpoint{7.750000in}{3.850000in}}%
\pgfusepath{clip}%
\pgfsetbuttcap%
\pgfsetmiterjoin%
\definecolor{currentfill}{rgb}{0.501961,0.501961,0.501961}%
\pgfsetfillcolor{currentfill}%
\pgfsetfillopacity{0.700000}%
\pgfsetlinewidth{0.000000pt}%
\definecolor{currentstroke}{rgb}{0.000000,0.000000,0.000000}%
\pgfsetstrokecolor{currentstroke}%
\pgfsetstrokeopacity{0.700000}%
\pgfsetdash{}{0pt}%
\pgfpathmoveto{\pgfqpoint{7.739524in}{0.550000in}}%
\pgfpathlineto{\pgfqpoint{7.767045in}{0.550000in}}%
\pgfpathlineto{\pgfqpoint{7.767045in}{0.734588in}}%
\pgfpathlineto{\pgfqpoint{7.739524in}{0.734588in}}%
\pgfpathlineto{\pgfqpoint{7.739524in}{0.550000in}}%
\pgfpathclose%
\pgfusepath{fill}%
\end{pgfscope}%
\begin{pgfscope}%
\pgfpathrectangle{\pgfqpoint{1.250000in}{0.550000in}}{\pgfqpoint{7.750000in}{3.850000in}}%
\pgfusepath{clip}%
\pgfsetbuttcap%
\pgfsetmiterjoin%
\definecolor{currentfill}{rgb}{0.501961,0.501961,0.501961}%
\pgfsetfillcolor{currentfill}%
\pgfsetfillopacity{0.700000}%
\pgfsetlinewidth{0.000000pt}%
\definecolor{currentstroke}{rgb}{0.000000,0.000000,0.000000}%
\pgfsetstrokecolor{currentstroke}%
\pgfsetstrokeopacity{0.700000}%
\pgfsetdash{}{0pt}%
\pgfpathmoveto{\pgfqpoint{7.767045in}{0.550000in}}%
\pgfpathlineto{\pgfqpoint{7.794567in}{0.550000in}}%
\pgfpathlineto{\pgfqpoint{7.794567in}{0.726612in}}%
\pgfpathlineto{\pgfqpoint{7.767045in}{0.726612in}}%
\pgfpathlineto{\pgfqpoint{7.767045in}{0.550000in}}%
\pgfpathclose%
\pgfusepath{fill}%
\end{pgfscope}%
\begin{pgfscope}%
\pgfpathrectangle{\pgfqpoint{1.250000in}{0.550000in}}{\pgfqpoint{7.750000in}{3.850000in}}%
\pgfusepath{clip}%
\pgfsetbuttcap%
\pgfsetmiterjoin%
\definecolor{currentfill}{rgb}{0.501961,0.501961,0.501961}%
\pgfsetfillcolor{currentfill}%
\pgfsetfillopacity{0.700000}%
\pgfsetlinewidth{0.000000pt}%
\definecolor{currentstroke}{rgb}{0.000000,0.000000,0.000000}%
\pgfsetstrokecolor{currentstroke}%
\pgfsetstrokeopacity{0.700000}%
\pgfsetdash{}{0pt}%
\pgfpathmoveto{\pgfqpoint{7.794567in}{0.550000in}}%
\pgfpathlineto{\pgfqpoint{7.822088in}{0.550000in}}%
\pgfpathlineto{\pgfqpoint{7.822088in}{0.673252in}}%
\pgfpathlineto{\pgfqpoint{7.794567in}{0.673252in}}%
\pgfpathlineto{\pgfqpoint{7.794567in}{0.550000in}}%
\pgfpathclose%
\pgfusepath{fill}%
\end{pgfscope}%
\begin{pgfscope}%
\pgfpathrectangle{\pgfqpoint{1.250000in}{0.550000in}}{\pgfqpoint{7.750000in}{3.850000in}}%
\pgfusepath{clip}%
\pgfsetbuttcap%
\pgfsetmiterjoin%
\definecolor{currentfill}{rgb}{0.501961,0.501961,0.501961}%
\pgfsetfillcolor{currentfill}%
\pgfsetfillopacity{0.700000}%
\pgfsetlinewidth{0.000000pt}%
\definecolor{currentstroke}{rgb}{0.000000,0.000000,0.000000}%
\pgfsetstrokecolor{currentstroke}%
\pgfsetstrokeopacity{0.700000}%
\pgfsetdash{}{0pt}%
\pgfpathmoveto{\pgfqpoint{7.822088in}{0.550000in}}%
\pgfpathlineto{\pgfqpoint{7.849609in}{0.550000in}}%
\pgfpathlineto{\pgfqpoint{7.849609in}{0.706892in}}%
\pgfpathlineto{\pgfqpoint{7.822088in}{0.706892in}}%
\pgfpathlineto{\pgfqpoint{7.822088in}{0.550000in}}%
\pgfpathclose%
\pgfusepath{fill}%
\end{pgfscope}%
\begin{pgfscope}%
\pgfpathrectangle{\pgfqpoint{1.250000in}{0.550000in}}{\pgfqpoint{7.750000in}{3.850000in}}%
\pgfusepath{clip}%
\pgfsetbuttcap%
\pgfsetmiterjoin%
\definecolor{currentfill}{rgb}{0.501961,0.501961,0.501961}%
\pgfsetfillcolor{currentfill}%
\pgfsetfillopacity{0.700000}%
\pgfsetlinewidth{0.000000pt}%
\definecolor{currentstroke}{rgb}{0.000000,0.000000,0.000000}%
\pgfsetstrokecolor{currentstroke}%
\pgfsetstrokeopacity{0.700000}%
\pgfsetdash{}{0pt}%
\pgfpathmoveto{\pgfqpoint{7.849609in}{0.550000in}}%
\pgfpathlineto{\pgfqpoint{7.877131in}{0.550000in}}%
\pgfpathlineto{\pgfqpoint{7.877131in}{0.745608in}}%
\pgfpathlineto{\pgfqpoint{7.849609in}{0.745608in}}%
\pgfpathlineto{\pgfqpoint{7.849609in}{0.550000in}}%
\pgfpathclose%
\pgfusepath{fill}%
\end{pgfscope}%
\begin{pgfscope}%
\pgfpathrectangle{\pgfqpoint{1.250000in}{0.550000in}}{\pgfqpoint{7.750000in}{3.850000in}}%
\pgfusepath{clip}%
\pgfsetbuttcap%
\pgfsetmiterjoin%
\definecolor{currentfill}{rgb}{0.501961,0.501961,0.501961}%
\pgfsetfillcolor{currentfill}%
\pgfsetfillopacity{0.700000}%
\pgfsetlinewidth{0.000000pt}%
\definecolor{currentstroke}{rgb}{0.000000,0.000000,0.000000}%
\pgfsetstrokecolor{currentstroke}%
\pgfsetstrokeopacity{0.700000}%
\pgfsetdash{}{0pt}%
\pgfpathmoveto{\pgfqpoint{7.877131in}{0.550000in}}%
\pgfpathlineto{\pgfqpoint{7.904652in}{0.550000in}}%
\pgfpathlineto{\pgfqpoint{7.904652in}{0.690072in}}%
\pgfpathlineto{\pgfqpoint{7.877131in}{0.690072in}}%
\pgfpathlineto{\pgfqpoint{7.877131in}{0.550000in}}%
\pgfpathclose%
\pgfusepath{fill}%
\end{pgfscope}%
\begin{pgfscope}%
\pgfpathrectangle{\pgfqpoint{1.250000in}{0.550000in}}{\pgfqpoint{7.750000in}{3.850000in}}%
\pgfusepath{clip}%
\pgfsetbuttcap%
\pgfsetmiterjoin%
\definecolor{currentfill}{rgb}{0.501961,0.501961,0.501961}%
\pgfsetfillcolor{currentfill}%
\pgfsetfillopacity{0.700000}%
\pgfsetlinewidth{0.000000pt}%
\definecolor{currentstroke}{rgb}{0.000000,0.000000,0.000000}%
\pgfsetstrokecolor{currentstroke}%
\pgfsetstrokeopacity{0.700000}%
\pgfsetdash{}{0pt}%
\pgfpathmoveto{\pgfqpoint{7.904652in}{0.550000in}}%
\pgfpathlineto{\pgfqpoint{7.932173in}{0.550000in}}%
\pgfpathlineto{\pgfqpoint{7.932173in}{0.722407in}}%
\pgfpathlineto{\pgfqpoint{7.904652in}{0.722407in}}%
\pgfpathlineto{\pgfqpoint{7.904652in}{0.550000in}}%
\pgfpathclose%
\pgfusepath{fill}%
\end{pgfscope}%
\begin{pgfscope}%
\pgfpathrectangle{\pgfqpoint{1.250000in}{0.550000in}}{\pgfqpoint{7.750000in}{3.850000in}}%
\pgfusepath{clip}%
\pgfsetbuttcap%
\pgfsetmiterjoin%
\definecolor{currentfill}{rgb}{0.501961,0.501961,0.501961}%
\pgfsetfillcolor{currentfill}%
\pgfsetfillopacity{0.700000}%
\pgfsetlinewidth{0.000000pt}%
\definecolor{currentstroke}{rgb}{0.000000,0.000000,0.000000}%
\pgfsetstrokecolor{currentstroke}%
\pgfsetstrokeopacity{0.700000}%
\pgfsetdash{}{0pt}%
\pgfpathmoveto{\pgfqpoint{7.932173in}{0.550000in}}%
\pgfpathlineto{\pgfqpoint{7.959695in}{0.550000in}}%
\pgfpathlineto{\pgfqpoint{7.959695in}{0.706892in}}%
\pgfpathlineto{\pgfqpoint{7.932173in}{0.706892in}}%
\pgfpathlineto{\pgfqpoint{7.932173in}{0.550000in}}%
\pgfpathclose%
\pgfusepath{fill}%
\end{pgfscope}%
\begin{pgfscope}%
\pgfpathrectangle{\pgfqpoint{1.250000in}{0.550000in}}{\pgfqpoint{7.750000in}{3.850000in}}%
\pgfusepath{clip}%
\pgfsetbuttcap%
\pgfsetmiterjoin%
\definecolor{currentfill}{rgb}{0.501961,0.501961,0.501961}%
\pgfsetfillcolor{currentfill}%
\pgfsetfillopacity{0.700000}%
\pgfsetlinewidth{0.000000pt}%
\definecolor{currentstroke}{rgb}{0.000000,0.000000,0.000000}%
\pgfsetstrokecolor{currentstroke}%
\pgfsetstrokeopacity{0.700000}%
\pgfsetdash{}{0pt}%
\pgfpathmoveto{\pgfqpoint{7.959695in}{0.550000in}}%
\pgfpathlineto{\pgfqpoint{7.987216in}{0.550000in}}%
\pgfpathlineto{\pgfqpoint{7.987216in}{0.703992in}}%
\pgfpathlineto{\pgfqpoint{7.959695in}{0.703992in}}%
\pgfpathlineto{\pgfqpoint{7.959695in}{0.550000in}}%
\pgfpathclose%
\pgfusepath{fill}%
\end{pgfscope}%
\begin{pgfscope}%
\pgfpathrectangle{\pgfqpoint{1.250000in}{0.550000in}}{\pgfqpoint{7.750000in}{3.850000in}}%
\pgfusepath{clip}%
\pgfsetbuttcap%
\pgfsetmiterjoin%
\definecolor{currentfill}{rgb}{0.501961,0.501961,0.501961}%
\pgfsetfillcolor{currentfill}%
\pgfsetfillopacity{0.700000}%
\pgfsetlinewidth{0.000000pt}%
\definecolor{currentstroke}{rgb}{0.000000,0.000000,0.000000}%
\pgfsetstrokecolor{currentstroke}%
\pgfsetstrokeopacity{0.700000}%
\pgfsetdash{}{0pt}%
\pgfpathmoveto{\pgfqpoint{7.987216in}{0.550000in}}%
\pgfpathlineto{\pgfqpoint{8.014737in}{0.550000in}}%
\pgfpathlineto{\pgfqpoint{8.014737in}{0.726902in}}%
\pgfpathlineto{\pgfqpoint{7.987216in}{0.726902in}}%
\pgfpathlineto{\pgfqpoint{7.987216in}{0.550000in}}%
\pgfpathclose%
\pgfusepath{fill}%
\end{pgfscope}%
\begin{pgfscope}%
\pgfpathrectangle{\pgfqpoint{1.250000in}{0.550000in}}{\pgfqpoint{7.750000in}{3.850000in}}%
\pgfusepath{clip}%
\pgfsetbuttcap%
\pgfsetmiterjoin%
\definecolor{currentfill}{rgb}{0.501961,0.501961,0.501961}%
\pgfsetfillcolor{currentfill}%
\pgfsetfillopacity{0.700000}%
\pgfsetlinewidth{0.000000pt}%
\definecolor{currentstroke}{rgb}{0.000000,0.000000,0.000000}%
\pgfsetstrokecolor{currentstroke}%
\pgfsetstrokeopacity{0.700000}%
\pgfsetdash{}{0pt}%
\pgfpathmoveto{\pgfqpoint{8.014737in}{0.550000in}}%
\pgfpathlineto{\pgfqpoint{8.042259in}{0.550000in}}%
\pgfpathlineto{\pgfqpoint{8.042259in}{0.692682in}}%
\pgfpathlineto{\pgfqpoint{8.014737in}{0.692682in}}%
\pgfpathlineto{\pgfqpoint{8.014737in}{0.550000in}}%
\pgfpathclose%
\pgfusepath{fill}%
\end{pgfscope}%
\begin{pgfscope}%
\pgfpathrectangle{\pgfqpoint{1.250000in}{0.550000in}}{\pgfqpoint{7.750000in}{3.850000in}}%
\pgfusepath{clip}%
\pgfsetbuttcap%
\pgfsetmiterjoin%
\definecolor{currentfill}{rgb}{0.501961,0.501961,0.501961}%
\pgfsetfillcolor{currentfill}%
\pgfsetfillopacity{0.700000}%
\pgfsetlinewidth{0.000000pt}%
\definecolor{currentstroke}{rgb}{0.000000,0.000000,0.000000}%
\pgfsetstrokecolor{currentstroke}%
\pgfsetstrokeopacity{0.700000}%
\pgfsetdash{}{0pt}%
\pgfpathmoveto{\pgfqpoint{8.042259in}{0.550000in}}%
\pgfpathlineto{\pgfqpoint{8.069780in}{0.550000in}}%
\pgfpathlineto{\pgfqpoint{8.069780in}{0.718637in}}%
\pgfpathlineto{\pgfqpoint{8.042259in}{0.718637in}}%
\pgfpathlineto{\pgfqpoint{8.042259in}{0.550000in}}%
\pgfpathclose%
\pgfusepath{fill}%
\end{pgfscope}%
\begin{pgfscope}%
\pgfpathrectangle{\pgfqpoint{1.250000in}{0.550000in}}{\pgfqpoint{7.750000in}{3.850000in}}%
\pgfusepath{clip}%
\pgfsetbuttcap%
\pgfsetmiterjoin%
\definecolor{currentfill}{rgb}{0.501961,0.501961,0.501961}%
\pgfsetfillcolor{currentfill}%
\pgfsetfillopacity{0.700000}%
\pgfsetlinewidth{0.000000pt}%
\definecolor{currentstroke}{rgb}{0.000000,0.000000,0.000000}%
\pgfsetstrokecolor{currentstroke}%
\pgfsetstrokeopacity{0.700000}%
\pgfsetdash{}{0pt}%
\pgfpathmoveto{\pgfqpoint{8.069780in}{0.550000in}}%
\pgfpathlineto{\pgfqpoint{8.097301in}{0.550000in}}%
\pgfpathlineto{\pgfqpoint{8.097301in}{0.702542in}}%
\pgfpathlineto{\pgfqpoint{8.069780in}{0.702542in}}%
\pgfpathlineto{\pgfqpoint{8.069780in}{0.550000in}}%
\pgfpathclose%
\pgfusepath{fill}%
\end{pgfscope}%
\begin{pgfscope}%
\pgfpathrectangle{\pgfqpoint{1.250000in}{0.550000in}}{\pgfqpoint{7.750000in}{3.850000in}}%
\pgfusepath{clip}%
\pgfsetbuttcap%
\pgfsetmiterjoin%
\definecolor{currentfill}{rgb}{0.501961,0.501961,0.501961}%
\pgfsetfillcolor{currentfill}%
\pgfsetfillopacity{0.700000}%
\pgfsetlinewidth{0.000000pt}%
\definecolor{currentstroke}{rgb}{0.000000,0.000000,0.000000}%
\pgfsetstrokecolor{currentstroke}%
\pgfsetstrokeopacity{0.700000}%
\pgfsetdash{}{0pt}%
\pgfpathmoveto{\pgfqpoint{8.097301in}{0.550000in}}%
\pgfpathlineto{\pgfqpoint{8.124822in}{0.550000in}}%
\pgfpathlineto{\pgfqpoint{8.124822in}{0.712547in}}%
\pgfpathlineto{\pgfqpoint{8.097301in}{0.712547in}}%
\pgfpathlineto{\pgfqpoint{8.097301in}{0.550000in}}%
\pgfpathclose%
\pgfusepath{fill}%
\end{pgfscope}%
\begin{pgfscope}%
\pgfpathrectangle{\pgfqpoint{1.250000in}{0.550000in}}{\pgfqpoint{7.750000in}{3.850000in}}%
\pgfusepath{clip}%
\pgfsetbuttcap%
\pgfsetmiterjoin%
\definecolor{currentfill}{rgb}{0.501961,0.501961,0.501961}%
\pgfsetfillcolor{currentfill}%
\pgfsetfillopacity{0.700000}%
\pgfsetlinewidth{0.000000pt}%
\definecolor{currentstroke}{rgb}{0.000000,0.000000,0.000000}%
\pgfsetstrokecolor{currentstroke}%
\pgfsetstrokeopacity{0.700000}%
\pgfsetdash{}{0pt}%
\pgfpathmoveto{\pgfqpoint{8.124822in}{0.550000in}}%
\pgfpathlineto{\pgfqpoint{8.152344in}{0.550000in}}%
\pgfpathlineto{\pgfqpoint{8.152344in}{0.700657in}}%
\pgfpathlineto{\pgfqpoint{8.124822in}{0.700657in}}%
\pgfpathlineto{\pgfqpoint{8.124822in}{0.550000in}}%
\pgfpathclose%
\pgfusepath{fill}%
\end{pgfscope}%
\begin{pgfscope}%
\pgfpathrectangle{\pgfqpoint{1.250000in}{0.550000in}}{\pgfqpoint{7.750000in}{3.850000in}}%
\pgfusepath{clip}%
\pgfsetbuttcap%
\pgfsetmiterjoin%
\definecolor{currentfill}{rgb}{0.501961,0.501961,0.501961}%
\pgfsetfillcolor{currentfill}%
\pgfsetfillopacity{0.700000}%
\pgfsetlinewidth{0.000000pt}%
\definecolor{currentstroke}{rgb}{0.000000,0.000000,0.000000}%
\pgfsetstrokecolor{currentstroke}%
\pgfsetstrokeopacity{0.700000}%
\pgfsetdash{}{0pt}%
\pgfpathmoveto{\pgfqpoint{8.152344in}{0.550000in}}%
\pgfpathlineto{\pgfqpoint{8.179865in}{0.550000in}}%
\pgfpathlineto{\pgfqpoint{8.179865in}{0.713272in}}%
\pgfpathlineto{\pgfqpoint{8.152344in}{0.713272in}}%
\pgfpathlineto{\pgfqpoint{8.152344in}{0.550000in}}%
\pgfpathclose%
\pgfusepath{fill}%
\end{pgfscope}%
\begin{pgfscope}%
\pgfpathrectangle{\pgfqpoint{1.250000in}{0.550000in}}{\pgfqpoint{7.750000in}{3.850000in}}%
\pgfusepath{clip}%
\pgfsetbuttcap%
\pgfsetmiterjoin%
\definecolor{currentfill}{rgb}{0.501961,0.501961,0.501961}%
\pgfsetfillcolor{currentfill}%
\pgfsetfillopacity{0.700000}%
\pgfsetlinewidth{0.000000pt}%
\definecolor{currentstroke}{rgb}{0.000000,0.000000,0.000000}%
\pgfsetstrokecolor{currentstroke}%
\pgfsetstrokeopacity{0.700000}%
\pgfsetdash{}{0pt}%
\pgfpathmoveto{\pgfqpoint{8.179865in}{0.550000in}}%
\pgfpathlineto{\pgfqpoint{8.207386in}{0.550000in}}%
\pgfpathlineto{\pgfqpoint{8.207386in}{0.722697in}}%
\pgfpathlineto{\pgfqpoint{8.179865in}{0.722697in}}%
\pgfpathlineto{\pgfqpoint{8.179865in}{0.550000in}}%
\pgfpathclose%
\pgfusepath{fill}%
\end{pgfscope}%
\begin{pgfscope}%
\pgfpathrectangle{\pgfqpoint{1.250000in}{0.550000in}}{\pgfqpoint{7.750000in}{3.850000in}}%
\pgfusepath{clip}%
\pgfsetbuttcap%
\pgfsetmiterjoin%
\definecolor{currentfill}{rgb}{0.501961,0.501961,0.501961}%
\pgfsetfillcolor{currentfill}%
\pgfsetfillopacity{0.700000}%
\pgfsetlinewidth{0.000000pt}%
\definecolor{currentstroke}{rgb}{0.000000,0.000000,0.000000}%
\pgfsetstrokecolor{currentstroke}%
\pgfsetstrokeopacity{0.700000}%
\pgfsetdash{}{0pt}%
\pgfpathmoveto{\pgfqpoint{8.207386in}{0.550000in}}%
\pgfpathlineto{\pgfqpoint{8.234908in}{0.550000in}}%
\pgfpathlineto{\pgfqpoint{8.234908in}{0.716462in}}%
\pgfpathlineto{\pgfqpoint{8.207386in}{0.716462in}}%
\pgfpathlineto{\pgfqpoint{8.207386in}{0.550000in}}%
\pgfpathclose%
\pgfusepath{fill}%
\end{pgfscope}%
\begin{pgfscope}%
\pgfpathrectangle{\pgfqpoint{1.250000in}{0.550000in}}{\pgfqpoint{7.750000in}{3.850000in}}%
\pgfusepath{clip}%
\pgfsetbuttcap%
\pgfsetmiterjoin%
\definecolor{currentfill}{rgb}{0.501961,0.501961,0.501961}%
\pgfsetfillcolor{currentfill}%
\pgfsetfillopacity{0.700000}%
\pgfsetlinewidth{0.000000pt}%
\definecolor{currentstroke}{rgb}{0.000000,0.000000,0.000000}%
\pgfsetstrokecolor{currentstroke}%
\pgfsetstrokeopacity{0.700000}%
\pgfsetdash{}{0pt}%
\pgfpathmoveto{\pgfqpoint{8.234908in}{0.550000in}}%
\pgfpathlineto{\pgfqpoint{8.262429in}{0.550000in}}%
\pgfpathlineto{\pgfqpoint{8.262429in}{0.694857in}}%
\pgfpathlineto{\pgfqpoint{8.234908in}{0.694857in}}%
\pgfpathlineto{\pgfqpoint{8.234908in}{0.550000in}}%
\pgfpathclose%
\pgfusepath{fill}%
\end{pgfscope}%
\begin{pgfscope}%
\pgfpathrectangle{\pgfqpoint{1.250000in}{0.550000in}}{\pgfqpoint{7.750000in}{3.850000in}}%
\pgfusepath{clip}%
\pgfsetbuttcap%
\pgfsetmiterjoin%
\definecolor{currentfill}{rgb}{0.501961,0.501961,0.501961}%
\pgfsetfillcolor{currentfill}%
\pgfsetfillopacity{0.700000}%
\pgfsetlinewidth{0.000000pt}%
\definecolor{currentstroke}{rgb}{0.000000,0.000000,0.000000}%
\pgfsetstrokecolor{currentstroke}%
\pgfsetstrokeopacity{0.700000}%
\pgfsetdash{}{0pt}%
\pgfpathmoveto{\pgfqpoint{8.262429in}{0.550000in}}%
\pgfpathlineto{\pgfqpoint{8.289950in}{0.550000in}}%
\pgfpathlineto{\pgfqpoint{8.289950in}{0.732268in}}%
\pgfpathlineto{\pgfqpoint{8.262429in}{0.732268in}}%
\pgfpathlineto{\pgfqpoint{8.262429in}{0.550000in}}%
\pgfpathclose%
\pgfusepath{fill}%
\end{pgfscope}%
\begin{pgfscope}%
\pgfpathrectangle{\pgfqpoint{1.250000in}{0.550000in}}{\pgfqpoint{7.750000in}{3.850000in}}%
\pgfusepath{clip}%
\pgfsetbuttcap%
\pgfsetmiterjoin%
\definecolor{currentfill}{rgb}{0.501961,0.501961,0.501961}%
\pgfsetfillcolor{currentfill}%
\pgfsetfillopacity{0.700000}%
\pgfsetlinewidth{0.000000pt}%
\definecolor{currentstroke}{rgb}{0.000000,0.000000,0.000000}%
\pgfsetstrokecolor{currentstroke}%
\pgfsetstrokeopacity{0.700000}%
\pgfsetdash{}{0pt}%
\pgfpathmoveto{\pgfqpoint{8.289950in}{0.550000in}}%
\pgfpathlineto{\pgfqpoint{8.317472in}{0.550000in}}%
\pgfpathlineto{\pgfqpoint{8.317472in}{0.709212in}}%
\pgfpathlineto{\pgfqpoint{8.289950in}{0.709212in}}%
\pgfpathlineto{\pgfqpoint{8.289950in}{0.550000in}}%
\pgfpathclose%
\pgfusepath{fill}%
\end{pgfscope}%
\begin{pgfscope}%
\pgfpathrectangle{\pgfqpoint{1.250000in}{0.550000in}}{\pgfqpoint{7.750000in}{3.850000in}}%
\pgfusepath{clip}%
\pgfsetbuttcap%
\pgfsetmiterjoin%
\definecolor{currentfill}{rgb}{0.501961,0.501961,0.501961}%
\pgfsetfillcolor{currentfill}%
\pgfsetfillopacity{0.700000}%
\pgfsetlinewidth{0.000000pt}%
\definecolor{currentstroke}{rgb}{0.000000,0.000000,0.000000}%
\pgfsetstrokecolor{currentstroke}%
\pgfsetstrokeopacity{0.700000}%
\pgfsetdash{}{0pt}%
\pgfpathmoveto{\pgfqpoint{8.317472in}{0.550000in}}%
\pgfpathlineto{\pgfqpoint{8.344993in}{0.550000in}}%
\pgfpathlineto{\pgfqpoint{8.344993in}{0.709647in}}%
\pgfpathlineto{\pgfqpoint{8.317472in}{0.709647in}}%
\pgfpathlineto{\pgfqpoint{8.317472in}{0.550000in}}%
\pgfpathclose%
\pgfusepath{fill}%
\end{pgfscope}%
\begin{pgfscope}%
\pgfpathrectangle{\pgfqpoint{1.250000in}{0.550000in}}{\pgfqpoint{7.750000in}{3.850000in}}%
\pgfusepath{clip}%
\pgfsetbuttcap%
\pgfsetmiterjoin%
\definecolor{currentfill}{rgb}{0.501961,0.501961,0.501961}%
\pgfsetfillcolor{currentfill}%
\pgfsetfillopacity{0.700000}%
\pgfsetlinewidth{0.000000pt}%
\definecolor{currentstroke}{rgb}{0.000000,0.000000,0.000000}%
\pgfsetstrokecolor{currentstroke}%
\pgfsetstrokeopacity{0.700000}%
\pgfsetdash{}{0pt}%
\pgfpathmoveto{\pgfqpoint{8.344993in}{0.550000in}}%
\pgfpathlineto{\pgfqpoint{8.372514in}{0.550000in}}%
\pgfpathlineto{\pgfqpoint{8.372514in}{0.711822in}}%
\pgfpathlineto{\pgfqpoint{8.344993in}{0.711822in}}%
\pgfpathlineto{\pgfqpoint{8.344993in}{0.550000in}}%
\pgfpathclose%
\pgfusepath{fill}%
\end{pgfscope}%
\begin{pgfscope}%
\pgfpathrectangle{\pgfqpoint{1.250000in}{0.550000in}}{\pgfqpoint{7.750000in}{3.850000in}}%
\pgfusepath{clip}%
\pgfsetbuttcap%
\pgfsetmiterjoin%
\definecolor{currentfill}{rgb}{0.501961,0.501961,0.501961}%
\pgfsetfillcolor{currentfill}%
\pgfsetfillopacity{0.700000}%
\pgfsetlinewidth{0.000000pt}%
\definecolor{currentstroke}{rgb}{0.000000,0.000000,0.000000}%
\pgfsetstrokecolor{currentstroke}%
\pgfsetstrokeopacity{0.700000}%
\pgfsetdash{}{0pt}%
\pgfpathmoveto{\pgfqpoint{8.372514in}{0.550000in}}%
\pgfpathlineto{\pgfqpoint{8.400036in}{0.550000in}}%
\pgfpathlineto{\pgfqpoint{8.400036in}{0.701962in}}%
\pgfpathlineto{\pgfqpoint{8.372514in}{0.701962in}}%
\pgfpathlineto{\pgfqpoint{8.372514in}{0.550000in}}%
\pgfpathclose%
\pgfusepath{fill}%
\end{pgfscope}%
\begin{pgfscope}%
\pgfpathrectangle{\pgfqpoint{1.250000in}{0.550000in}}{\pgfqpoint{7.750000in}{3.850000in}}%
\pgfusepath{clip}%
\pgfsetbuttcap%
\pgfsetmiterjoin%
\definecolor{currentfill}{rgb}{0.501961,0.501961,0.501961}%
\pgfsetfillcolor{currentfill}%
\pgfsetfillopacity{0.700000}%
\pgfsetlinewidth{0.000000pt}%
\definecolor{currentstroke}{rgb}{0.000000,0.000000,0.000000}%
\pgfsetstrokecolor{currentstroke}%
\pgfsetstrokeopacity{0.700000}%
\pgfsetdash{}{0pt}%
\pgfpathmoveto{\pgfqpoint{8.400036in}{0.550000in}}%
\pgfpathlineto{\pgfqpoint{8.427557in}{0.550000in}}%
\pgfpathlineto{\pgfqpoint{8.427557in}{0.706602in}}%
\pgfpathlineto{\pgfqpoint{8.400036in}{0.706602in}}%
\pgfpathlineto{\pgfqpoint{8.400036in}{0.550000in}}%
\pgfpathclose%
\pgfusepath{fill}%
\end{pgfscope}%
\begin{pgfscope}%
\pgfpathrectangle{\pgfqpoint{1.250000in}{0.550000in}}{\pgfqpoint{7.750000in}{3.850000in}}%
\pgfusepath{clip}%
\pgfsetbuttcap%
\pgfsetmiterjoin%
\definecolor{currentfill}{rgb}{0.501961,0.501961,0.501961}%
\pgfsetfillcolor{currentfill}%
\pgfsetfillopacity{0.700000}%
\pgfsetlinewidth{0.000000pt}%
\definecolor{currentstroke}{rgb}{0.000000,0.000000,0.000000}%
\pgfsetstrokecolor{currentstroke}%
\pgfsetstrokeopacity{0.700000}%
\pgfsetdash{}{0pt}%
\pgfpathmoveto{\pgfqpoint{8.427557in}{0.550000in}}%
\pgfpathlineto{\pgfqpoint{8.455078in}{0.550000in}}%
\pgfpathlineto{\pgfqpoint{8.455078in}{0.710082in}}%
\pgfpathlineto{\pgfqpoint{8.427557in}{0.710082in}}%
\pgfpathlineto{\pgfqpoint{8.427557in}{0.550000in}}%
\pgfpathclose%
\pgfusepath{fill}%
\end{pgfscope}%
\begin{pgfscope}%
\pgfpathrectangle{\pgfqpoint{1.250000in}{0.550000in}}{\pgfqpoint{7.750000in}{3.850000in}}%
\pgfusepath{clip}%
\pgfsetbuttcap%
\pgfsetmiterjoin%
\definecolor{currentfill}{rgb}{0.501961,0.501961,0.501961}%
\pgfsetfillcolor{currentfill}%
\pgfsetfillopacity{0.700000}%
\pgfsetlinewidth{0.000000pt}%
\definecolor{currentstroke}{rgb}{0.000000,0.000000,0.000000}%
\pgfsetstrokecolor{currentstroke}%
\pgfsetstrokeopacity{0.700000}%
\pgfsetdash{}{0pt}%
\pgfpathmoveto{\pgfqpoint{8.455078in}{0.550000in}}%
\pgfpathlineto{\pgfqpoint{8.482599in}{0.550000in}}%
\pgfpathlineto{\pgfqpoint{8.482599in}{0.692972in}}%
\pgfpathlineto{\pgfqpoint{8.455078in}{0.692972in}}%
\pgfpathlineto{\pgfqpoint{8.455078in}{0.550000in}}%
\pgfpathclose%
\pgfusepath{fill}%
\end{pgfscope}%
\begin{pgfscope}%
\pgfpathrectangle{\pgfqpoint{1.250000in}{0.550000in}}{\pgfqpoint{7.750000in}{3.850000in}}%
\pgfusepath{clip}%
\pgfsetbuttcap%
\pgfsetmiterjoin%
\definecolor{currentfill}{rgb}{0.501961,0.501961,0.501961}%
\pgfsetfillcolor{currentfill}%
\pgfsetfillopacity{0.700000}%
\pgfsetlinewidth{0.000000pt}%
\definecolor{currentstroke}{rgb}{0.000000,0.000000,0.000000}%
\pgfsetstrokecolor{currentstroke}%
\pgfsetstrokeopacity{0.700000}%
\pgfsetdash{}{0pt}%
\pgfpathmoveto{\pgfqpoint{8.482599in}{0.550000in}}%
\pgfpathlineto{\pgfqpoint{8.510121in}{0.550000in}}%
\pgfpathlineto{\pgfqpoint{8.510121in}{0.693842in}}%
\pgfpathlineto{\pgfqpoint{8.482599in}{0.693842in}}%
\pgfpathlineto{\pgfqpoint{8.482599in}{0.550000in}}%
\pgfpathclose%
\pgfusepath{fill}%
\end{pgfscope}%
\begin{pgfscope}%
\pgfpathrectangle{\pgfqpoint{1.250000in}{0.550000in}}{\pgfqpoint{7.750000in}{3.850000in}}%
\pgfusepath{clip}%
\pgfsetbuttcap%
\pgfsetmiterjoin%
\definecolor{currentfill}{rgb}{0.501961,0.501961,0.501961}%
\pgfsetfillcolor{currentfill}%
\pgfsetfillopacity{0.700000}%
\pgfsetlinewidth{0.000000pt}%
\definecolor{currentstroke}{rgb}{0.000000,0.000000,0.000000}%
\pgfsetstrokecolor{currentstroke}%
\pgfsetstrokeopacity{0.700000}%
\pgfsetdash{}{0pt}%
\pgfpathmoveto{\pgfqpoint{8.510121in}{0.550000in}}%
\pgfpathlineto{\pgfqpoint{8.537642in}{0.550000in}}%
\pgfpathlineto{\pgfqpoint{8.537642in}{0.655561in}}%
\pgfpathlineto{\pgfqpoint{8.510121in}{0.655561in}}%
\pgfpathlineto{\pgfqpoint{8.510121in}{0.550000in}}%
\pgfpathclose%
\pgfusepath{fill}%
\end{pgfscope}%
\begin{pgfscope}%
\pgfpathrectangle{\pgfqpoint{1.250000in}{0.550000in}}{\pgfqpoint{7.750000in}{3.850000in}}%
\pgfusepath{clip}%
\pgfsetbuttcap%
\pgfsetmiterjoin%
\definecolor{currentfill}{rgb}{0.501961,0.501961,0.501961}%
\pgfsetfillcolor{currentfill}%
\pgfsetfillopacity{0.700000}%
\pgfsetlinewidth{0.000000pt}%
\definecolor{currentstroke}{rgb}{0.000000,0.000000,0.000000}%
\pgfsetstrokecolor{currentstroke}%
\pgfsetstrokeopacity{0.700000}%
\pgfsetdash{}{0pt}%
\pgfpathmoveto{\pgfqpoint{8.537642in}{0.550000in}}%
\pgfpathlineto{\pgfqpoint{8.565163in}{0.550000in}}%
\pgfpathlineto{\pgfqpoint{8.565163in}{0.694857in}}%
\pgfpathlineto{\pgfqpoint{8.537642in}{0.694857in}}%
\pgfpathlineto{\pgfqpoint{8.537642in}{0.550000in}}%
\pgfpathclose%
\pgfusepath{fill}%
\end{pgfscope}%
\begin{pgfscope}%
\pgfpathrectangle{\pgfqpoint{1.250000in}{0.550000in}}{\pgfqpoint{7.750000in}{3.850000in}}%
\pgfusepath{clip}%
\pgfsetbuttcap%
\pgfsetmiterjoin%
\definecolor{currentfill}{rgb}{0.501961,0.501961,0.501961}%
\pgfsetfillcolor{currentfill}%
\pgfsetfillopacity{0.700000}%
\pgfsetlinewidth{0.000000pt}%
\definecolor{currentstroke}{rgb}{0.000000,0.000000,0.000000}%
\pgfsetstrokecolor{currentstroke}%
\pgfsetstrokeopacity{0.700000}%
\pgfsetdash{}{0pt}%
\pgfpathmoveto{\pgfqpoint{8.565163in}{0.550000in}}%
\pgfpathlineto{\pgfqpoint{8.592685in}{0.550000in}}%
\pgfpathlineto{\pgfqpoint{8.592685in}{0.741113in}}%
\pgfpathlineto{\pgfqpoint{8.565163in}{0.741113in}}%
\pgfpathlineto{\pgfqpoint{8.565163in}{0.550000in}}%
\pgfpathclose%
\pgfusepath{fill}%
\end{pgfscope}%
\begin{pgfscope}%
\pgfpathrectangle{\pgfqpoint{1.250000in}{0.550000in}}{\pgfqpoint{7.750000in}{3.850000in}}%
\pgfusepath{clip}%
\pgfsetbuttcap%
\pgfsetmiterjoin%
\definecolor{currentfill}{rgb}{0.501961,0.501961,0.501961}%
\pgfsetfillcolor{currentfill}%
\pgfsetfillopacity{0.700000}%
\pgfsetlinewidth{0.000000pt}%
\definecolor{currentstroke}{rgb}{0.000000,0.000000,0.000000}%
\pgfsetstrokecolor{currentstroke}%
\pgfsetstrokeopacity{0.700000}%
\pgfsetdash{}{0pt}%
\pgfpathmoveto{\pgfqpoint{8.592685in}{0.550000in}}%
\pgfpathlineto{\pgfqpoint{8.620206in}{0.550000in}}%
\pgfpathlineto{\pgfqpoint{8.620206in}{0.550000in}}%
\pgfpathlineto{\pgfqpoint{8.592685in}{0.550000in}}%
\pgfpathlineto{\pgfqpoint{8.592685in}{0.550000in}}%
\pgfpathclose%
\pgfusepath{fill}%
\end{pgfscope}%
\begin{pgfscope}%
\pgfpathrectangle{\pgfqpoint{1.250000in}{0.550000in}}{\pgfqpoint{7.750000in}{3.850000in}}%
\pgfusepath{clip}%
\pgfsetbuttcap%
\pgfsetmiterjoin%
\definecolor{currentfill}{rgb}{0.501961,0.501961,0.501961}%
\pgfsetfillcolor{currentfill}%
\pgfsetfillopacity{0.700000}%
\pgfsetlinewidth{0.000000pt}%
\definecolor{currentstroke}{rgb}{0.000000,0.000000,0.000000}%
\pgfsetstrokecolor{currentstroke}%
\pgfsetstrokeopacity{0.700000}%
\pgfsetdash{}{0pt}%
\pgfpathmoveto{\pgfqpoint{8.620206in}{0.550000in}}%
\pgfpathlineto{\pgfqpoint{8.647727in}{0.550000in}}%
\pgfpathlineto{\pgfqpoint{8.647727in}{0.880025in}}%
\pgfpathlineto{\pgfqpoint{8.620206in}{0.880025in}}%
\pgfpathlineto{\pgfqpoint{8.620206in}{0.550000in}}%
\pgfpathclose%
\pgfusepath{fill}%
\end{pgfscope}%
\begin{pgfscope}%
\pgfsetbuttcap%
\pgfsetroundjoin%
\definecolor{currentfill}{rgb}{0.000000,0.000000,0.000000}%
\pgfsetfillcolor{currentfill}%
\pgfsetlinewidth{0.803000pt}%
\definecolor{currentstroke}{rgb}{0.000000,0.000000,0.000000}%
\pgfsetstrokecolor{currentstroke}%
\pgfsetdash{}{0pt}%
\pgfsys@defobject{currentmarker}{\pgfqpoint{0.000000in}{-0.048611in}}{\pgfqpoint{0.000000in}{0.000000in}}{%
\pgfpathmoveto{\pgfqpoint{0.000000in}{0.000000in}}%
\pgfpathlineto{\pgfqpoint{0.000000in}{-0.048611in}}%
\pgfusepath{stroke,fill}%
}%
\begin{pgfscope}%
\pgfsys@transformshift{1.602273in}{0.550000in}%
\pgfsys@useobject{currentmarker}{}%
\end{pgfscope}%
\end{pgfscope}%
\begin{pgfscope}%
\definecolor{textcolor}{rgb}{0.000000,0.000000,0.000000}%
\pgfsetstrokecolor{textcolor}%
\pgfsetfillcolor{textcolor}%
\pgftext[x=1.602273in,y=0.452778in,,top]{\color{textcolor}{\sffamily\fontsize{10.000000}{12.000000}\selectfont\catcode`\^=\active\def^{\ifmmode\sp\else\^{}\fi}\catcode`\%=\active\def%{\%}0}}%
\end{pgfscope}%
\begin{pgfscope}%
\pgfsetbuttcap%
\pgfsetroundjoin%
\definecolor{currentfill}{rgb}{0.000000,0.000000,0.000000}%
\pgfsetfillcolor{currentfill}%
\pgfsetlinewidth{0.803000pt}%
\definecolor{currentstroke}{rgb}{0.000000,0.000000,0.000000}%
\pgfsetstrokecolor{currentstroke}%
\pgfsetdash{}{0pt}%
\pgfsys@defobject{currentmarker}{\pgfqpoint{0.000000in}{-0.048611in}}{\pgfqpoint{0.000000in}{0.000000in}}{%
\pgfpathmoveto{\pgfqpoint{0.000000in}{0.000000in}}%
\pgfpathlineto{\pgfqpoint{0.000000in}{-0.048611in}}%
\pgfusepath{stroke,fill}%
}%
\begin{pgfscope}%
\pgfsys@transformshift{2.978338in}{0.550000in}%
\pgfsys@useobject{currentmarker}{}%
\end{pgfscope}%
\end{pgfscope}%
\begin{pgfscope}%
\definecolor{textcolor}{rgb}{0.000000,0.000000,0.000000}%
\pgfsetstrokecolor{textcolor}%
\pgfsetfillcolor{textcolor}%
\pgftext[x=2.978338in,y=0.452778in,,top]{\color{textcolor}{\sffamily\fontsize{10.000000}{12.000000}\selectfont\catcode`\^=\active\def^{\ifmmode\sp\else\^{}\fi}\catcode`\%=\active\def%{\%}50}}%
\end{pgfscope}%
\begin{pgfscope}%
\pgfsetbuttcap%
\pgfsetroundjoin%
\definecolor{currentfill}{rgb}{0.000000,0.000000,0.000000}%
\pgfsetfillcolor{currentfill}%
\pgfsetlinewidth{0.803000pt}%
\definecolor{currentstroke}{rgb}{0.000000,0.000000,0.000000}%
\pgfsetstrokecolor{currentstroke}%
\pgfsetdash{}{0pt}%
\pgfsys@defobject{currentmarker}{\pgfqpoint{0.000000in}{-0.048611in}}{\pgfqpoint{0.000000in}{0.000000in}}{%
\pgfpathmoveto{\pgfqpoint{0.000000in}{0.000000in}}%
\pgfpathlineto{\pgfqpoint{0.000000in}{-0.048611in}}%
\pgfusepath{stroke,fill}%
}%
\begin{pgfscope}%
\pgfsys@transformshift{4.354403in}{0.550000in}%
\pgfsys@useobject{currentmarker}{}%
\end{pgfscope}%
\end{pgfscope}%
\begin{pgfscope}%
\definecolor{textcolor}{rgb}{0.000000,0.000000,0.000000}%
\pgfsetstrokecolor{textcolor}%
\pgfsetfillcolor{textcolor}%
\pgftext[x=4.354403in,y=0.452778in,,top]{\color{textcolor}{\sffamily\fontsize{10.000000}{12.000000}\selectfont\catcode`\^=\active\def^{\ifmmode\sp\else\^{}\fi}\catcode`\%=\active\def%{\%}100}}%
\end{pgfscope}%
\begin{pgfscope}%
\pgfsetbuttcap%
\pgfsetroundjoin%
\definecolor{currentfill}{rgb}{0.000000,0.000000,0.000000}%
\pgfsetfillcolor{currentfill}%
\pgfsetlinewidth{0.803000pt}%
\definecolor{currentstroke}{rgb}{0.000000,0.000000,0.000000}%
\pgfsetstrokecolor{currentstroke}%
\pgfsetdash{}{0pt}%
\pgfsys@defobject{currentmarker}{\pgfqpoint{0.000000in}{-0.048611in}}{\pgfqpoint{0.000000in}{0.000000in}}{%
\pgfpathmoveto{\pgfqpoint{0.000000in}{0.000000in}}%
\pgfpathlineto{\pgfqpoint{0.000000in}{-0.048611in}}%
\pgfusepath{stroke,fill}%
}%
\begin{pgfscope}%
\pgfsys@transformshift{5.730469in}{0.550000in}%
\pgfsys@useobject{currentmarker}{}%
\end{pgfscope}%
\end{pgfscope}%
\begin{pgfscope}%
\definecolor{textcolor}{rgb}{0.000000,0.000000,0.000000}%
\pgfsetstrokecolor{textcolor}%
\pgfsetfillcolor{textcolor}%
\pgftext[x=5.730469in,y=0.452778in,,top]{\color{textcolor}{\sffamily\fontsize{10.000000}{12.000000}\selectfont\catcode`\^=\active\def^{\ifmmode\sp\else\^{}\fi}\catcode`\%=\active\def%{\%}150}}%
\end{pgfscope}%
\begin{pgfscope}%
\pgfsetbuttcap%
\pgfsetroundjoin%
\definecolor{currentfill}{rgb}{0.000000,0.000000,0.000000}%
\pgfsetfillcolor{currentfill}%
\pgfsetlinewidth{0.803000pt}%
\definecolor{currentstroke}{rgb}{0.000000,0.000000,0.000000}%
\pgfsetstrokecolor{currentstroke}%
\pgfsetdash{}{0pt}%
\pgfsys@defobject{currentmarker}{\pgfqpoint{0.000000in}{-0.048611in}}{\pgfqpoint{0.000000in}{0.000000in}}{%
\pgfpathmoveto{\pgfqpoint{0.000000in}{0.000000in}}%
\pgfpathlineto{\pgfqpoint{0.000000in}{-0.048611in}}%
\pgfusepath{stroke,fill}%
}%
\begin{pgfscope}%
\pgfsys@transformshift{7.106534in}{0.550000in}%
\pgfsys@useobject{currentmarker}{}%
\end{pgfscope}%
\end{pgfscope}%
\begin{pgfscope}%
\definecolor{textcolor}{rgb}{0.000000,0.000000,0.000000}%
\pgfsetstrokecolor{textcolor}%
\pgfsetfillcolor{textcolor}%
\pgftext[x=7.106534in,y=0.452778in,,top]{\color{textcolor}{\sffamily\fontsize{10.000000}{12.000000}\selectfont\catcode`\^=\active\def^{\ifmmode\sp\else\^{}\fi}\catcode`\%=\active\def%{\%}200}}%
\end{pgfscope}%
\begin{pgfscope}%
\pgfsetbuttcap%
\pgfsetroundjoin%
\definecolor{currentfill}{rgb}{0.000000,0.000000,0.000000}%
\pgfsetfillcolor{currentfill}%
\pgfsetlinewidth{0.803000pt}%
\definecolor{currentstroke}{rgb}{0.000000,0.000000,0.000000}%
\pgfsetstrokecolor{currentstroke}%
\pgfsetdash{}{0pt}%
\pgfsys@defobject{currentmarker}{\pgfqpoint{0.000000in}{-0.048611in}}{\pgfqpoint{0.000000in}{0.000000in}}{%
\pgfpathmoveto{\pgfqpoint{0.000000in}{0.000000in}}%
\pgfpathlineto{\pgfqpoint{0.000000in}{-0.048611in}}%
\pgfusepath{stroke,fill}%
}%
\begin{pgfscope}%
\pgfsys@transformshift{8.482599in}{0.550000in}%
\pgfsys@useobject{currentmarker}{}%
\end{pgfscope}%
\end{pgfscope}%
\begin{pgfscope}%
\definecolor{textcolor}{rgb}{0.000000,0.000000,0.000000}%
\pgfsetstrokecolor{textcolor}%
\pgfsetfillcolor{textcolor}%
\pgftext[x=8.482599in,y=0.452778in,,top]{\color{textcolor}{\sffamily\fontsize{10.000000}{12.000000}\selectfont\catcode`\^=\active\def^{\ifmmode\sp\else\^{}\fi}\catcode`\%=\active\def%{\%}250}}%
\end{pgfscope}%
\begin{pgfscope}%
\definecolor{textcolor}{rgb}{0.000000,0.000000,0.000000}%
\pgfsetstrokecolor{textcolor}%
\pgfsetfillcolor{textcolor}%
\pgftext[x=5.125000in,y=0.262809in,,top]{\color{textcolor}{\sffamily\fontsize{10.000000}{12.000000}\selectfont\catcode`\^=\active\def^{\ifmmode\sp\else\^{}\fi}\catcode`\%=\active\def%{\%}Intensidad}}%
\end{pgfscope}%
\begin{pgfscope}%
\pgfsetbuttcap%
\pgfsetroundjoin%
\definecolor{currentfill}{rgb}{0.000000,0.000000,0.000000}%
\pgfsetfillcolor{currentfill}%
\pgfsetlinewidth{0.803000pt}%
\definecolor{currentstroke}{rgb}{0.000000,0.000000,0.000000}%
\pgfsetstrokecolor{currentstroke}%
\pgfsetdash{}{0pt}%
\pgfsys@defobject{currentmarker}{\pgfqpoint{-0.048611in}{0.000000in}}{\pgfqpoint{-0.000000in}{0.000000in}}{%
\pgfpathmoveto{\pgfqpoint{-0.000000in}{0.000000in}}%
\pgfpathlineto{\pgfqpoint{-0.048611in}{0.000000in}}%
\pgfusepath{stroke,fill}%
}%
\begin{pgfscope}%
\pgfsys@transformshift{1.250000in}{0.550000in}%
\pgfsys@useobject{currentmarker}{}%
\end{pgfscope}%
\end{pgfscope}%
\begin{pgfscope}%
\definecolor{textcolor}{rgb}{0.000000,0.000000,0.000000}%
\pgfsetstrokecolor{textcolor}%
\pgfsetfillcolor{textcolor}%
\pgftext[x=1.064412in, y=0.497238in, left, base]{\color{textcolor}{\sffamily\fontsize{10.000000}{12.000000}\selectfont\catcode`\^=\active\def^{\ifmmode\sp\else\^{}\fi}\catcode`\%=\active\def%{\%}0}}%
\end{pgfscope}%
\begin{pgfscope}%
\pgfsetbuttcap%
\pgfsetroundjoin%
\definecolor{currentfill}{rgb}{0.000000,0.000000,0.000000}%
\pgfsetfillcolor{currentfill}%
\pgfsetlinewidth{0.803000pt}%
\definecolor{currentstroke}{rgb}{0.000000,0.000000,0.000000}%
\pgfsetstrokecolor{currentstroke}%
\pgfsetdash{}{0pt}%
\pgfsys@defobject{currentmarker}{\pgfqpoint{-0.048611in}{0.000000in}}{\pgfqpoint{-0.000000in}{0.000000in}}{%
\pgfpathmoveto{\pgfqpoint{-0.000000in}{0.000000in}}%
\pgfpathlineto{\pgfqpoint{-0.048611in}{0.000000in}}%
\pgfusepath{stroke,fill}%
}%
\begin{pgfscope}%
\pgfsys@transformshift{1.250000in}{1.275010in}%
\pgfsys@useobject{currentmarker}{}%
\end{pgfscope}%
\end{pgfscope}%
\begin{pgfscope}%
\definecolor{textcolor}{rgb}{0.000000,0.000000,0.000000}%
\pgfsetstrokecolor{textcolor}%
\pgfsetfillcolor{textcolor}%
\pgftext[x=0.799316in, y=1.222249in, left, base]{\color{textcolor}{\sffamily\fontsize{10.000000}{12.000000}\selectfont\catcode`\^=\active\def^{\ifmmode\sp\else\^{}\fi}\catcode`\%=\active\def%{\%}5000}}%
\end{pgfscope}%
\begin{pgfscope}%
\pgfsetbuttcap%
\pgfsetroundjoin%
\definecolor{currentfill}{rgb}{0.000000,0.000000,0.000000}%
\pgfsetfillcolor{currentfill}%
\pgfsetlinewidth{0.803000pt}%
\definecolor{currentstroke}{rgb}{0.000000,0.000000,0.000000}%
\pgfsetstrokecolor{currentstroke}%
\pgfsetdash{}{0pt}%
\pgfsys@defobject{currentmarker}{\pgfqpoint{-0.048611in}{0.000000in}}{\pgfqpoint{-0.000000in}{0.000000in}}{%
\pgfpathmoveto{\pgfqpoint{-0.000000in}{0.000000in}}%
\pgfpathlineto{\pgfqpoint{-0.048611in}{0.000000in}}%
\pgfusepath{stroke,fill}%
}%
\begin{pgfscope}%
\pgfsys@transformshift{1.250000in}{2.000020in}%
\pgfsys@useobject{currentmarker}{}%
\end{pgfscope}%
\end{pgfscope}%
\begin{pgfscope}%
\definecolor{textcolor}{rgb}{0.000000,0.000000,0.000000}%
\pgfsetstrokecolor{textcolor}%
\pgfsetfillcolor{textcolor}%
\pgftext[x=0.710951in, y=1.947259in, left, base]{\color{textcolor}{\sffamily\fontsize{10.000000}{12.000000}\selectfont\catcode`\^=\active\def^{\ifmmode\sp\else\^{}\fi}\catcode`\%=\active\def%{\%}10000}}%
\end{pgfscope}%
\begin{pgfscope}%
\pgfsetbuttcap%
\pgfsetroundjoin%
\definecolor{currentfill}{rgb}{0.000000,0.000000,0.000000}%
\pgfsetfillcolor{currentfill}%
\pgfsetlinewidth{0.803000pt}%
\definecolor{currentstroke}{rgb}{0.000000,0.000000,0.000000}%
\pgfsetstrokecolor{currentstroke}%
\pgfsetdash{}{0pt}%
\pgfsys@defobject{currentmarker}{\pgfqpoint{-0.048611in}{0.000000in}}{\pgfqpoint{-0.000000in}{0.000000in}}{%
\pgfpathmoveto{\pgfqpoint{-0.000000in}{0.000000in}}%
\pgfpathlineto{\pgfqpoint{-0.048611in}{0.000000in}}%
\pgfusepath{stroke,fill}%
}%
\begin{pgfscope}%
\pgfsys@transformshift{1.250000in}{2.725031in}%
\pgfsys@useobject{currentmarker}{}%
\end{pgfscope}%
\end{pgfscope}%
\begin{pgfscope}%
\definecolor{textcolor}{rgb}{0.000000,0.000000,0.000000}%
\pgfsetstrokecolor{textcolor}%
\pgfsetfillcolor{textcolor}%
\pgftext[x=0.710951in, y=2.672269in, left, base]{\color{textcolor}{\sffamily\fontsize{10.000000}{12.000000}\selectfont\catcode`\^=\active\def^{\ifmmode\sp\else\^{}\fi}\catcode`\%=\active\def%{\%}15000}}%
\end{pgfscope}%
\begin{pgfscope}%
\pgfsetbuttcap%
\pgfsetroundjoin%
\definecolor{currentfill}{rgb}{0.000000,0.000000,0.000000}%
\pgfsetfillcolor{currentfill}%
\pgfsetlinewidth{0.803000pt}%
\definecolor{currentstroke}{rgb}{0.000000,0.000000,0.000000}%
\pgfsetstrokecolor{currentstroke}%
\pgfsetdash{}{0pt}%
\pgfsys@defobject{currentmarker}{\pgfqpoint{-0.048611in}{0.000000in}}{\pgfqpoint{-0.000000in}{0.000000in}}{%
\pgfpathmoveto{\pgfqpoint{-0.000000in}{0.000000in}}%
\pgfpathlineto{\pgfqpoint{-0.048611in}{0.000000in}}%
\pgfusepath{stroke,fill}%
}%
\begin{pgfscope}%
\pgfsys@transformshift{1.250000in}{3.450041in}%
\pgfsys@useobject{currentmarker}{}%
\end{pgfscope}%
\end{pgfscope}%
\begin{pgfscope}%
\definecolor{textcolor}{rgb}{0.000000,0.000000,0.000000}%
\pgfsetstrokecolor{textcolor}%
\pgfsetfillcolor{textcolor}%
\pgftext[x=0.710951in, y=3.397279in, left, base]{\color{textcolor}{\sffamily\fontsize{10.000000}{12.000000}\selectfont\catcode`\^=\active\def^{\ifmmode\sp\else\^{}\fi}\catcode`\%=\active\def%{\%}20000}}%
\end{pgfscope}%
\begin{pgfscope}%
\pgfsetbuttcap%
\pgfsetroundjoin%
\definecolor{currentfill}{rgb}{0.000000,0.000000,0.000000}%
\pgfsetfillcolor{currentfill}%
\pgfsetlinewidth{0.803000pt}%
\definecolor{currentstroke}{rgb}{0.000000,0.000000,0.000000}%
\pgfsetstrokecolor{currentstroke}%
\pgfsetdash{}{0pt}%
\pgfsys@defobject{currentmarker}{\pgfqpoint{-0.048611in}{0.000000in}}{\pgfqpoint{-0.000000in}{0.000000in}}{%
\pgfpathmoveto{\pgfqpoint{-0.000000in}{0.000000in}}%
\pgfpathlineto{\pgfqpoint{-0.048611in}{0.000000in}}%
\pgfusepath{stroke,fill}%
}%
\begin{pgfscope}%
\pgfsys@transformshift{1.250000in}{4.175051in}%
\pgfsys@useobject{currentmarker}{}%
\end{pgfscope}%
\end{pgfscope}%
\begin{pgfscope}%
\definecolor{textcolor}{rgb}{0.000000,0.000000,0.000000}%
\pgfsetstrokecolor{textcolor}%
\pgfsetfillcolor{textcolor}%
\pgftext[x=0.710951in, y=4.122290in, left, base]{\color{textcolor}{\sffamily\fontsize{10.000000}{12.000000}\selectfont\catcode`\^=\active\def^{\ifmmode\sp\else\^{}\fi}\catcode`\%=\active\def%{\%}25000}}%
\end{pgfscope}%
\begin{pgfscope}%
\definecolor{textcolor}{rgb}{0.000000,0.000000,0.000000}%
\pgfsetstrokecolor{textcolor}%
\pgfsetfillcolor{textcolor}%
\pgftext[x=0.655396in,y=2.475000in,,bottom,rotate=90.000000]{\color{textcolor}{\sffamily\fontsize{10.000000}{12.000000}\selectfont\catcode`\^=\active\def^{\ifmmode\sp\else\^{}\fi}\catcode`\%=\active\def%{\%}Frecuencia}}%
\end{pgfscope}%
\begin{pgfscope}%
\pgfsetrectcap%
\pgfsetmiterjoin%
\pgfsetlinewidth{0.803000pt}%
\definecolor{currentstroke}{rgb}{0.000000,0.000000,0.000000}%
\pgfsetstrokecolor{currentstroke}%
\pgfsetdash{}{0pt}%
\pgfpathmoveto{\pgfqpoint{1.250000in}{0.550000in}}%
\pgfpathlineto{\pgfqpoint{1.250000in}{4.400000in}}%
\pgfusepath{stroke}%
\end{pgfscope}%
\begin{pgfscope}%
\pgfsetrectcap%
\pgfsetmiterjoin%
\pgfsetlinewidth{0.803000pt}%
\definecolor{currentstroke}{rgb}{0.000000,0.000000,0.000000}%
\pgfsetstrokecolor{currentstroke}%
\pgfsetdash{}{0pt}%
\pgfpathmoveto{\pgfqpoint{9.000000in}{0.550000in}}%
\pgfpathlineto{\pgfqpoint{9.000000in}{4.400000in}}%
\pgfusepath{stroke}%
\end{pgfscope}%
\begin{pgfscope}%
\pgfsetrectcap%
\pgfsetmiterjoin%
\pgfsetlinewidth{0.803000pt}%
\definecolor{currentstroke}{rgb}{0.000000,0.000000,0.000000}%
\pgfsetstrokecolor{currentstroke}%
\pgfsetdash{}{0pt}%
\pgfpathmoveto{\pgfqpoint{1.250000in}{0.550000in}}%
\pgfpathlineto{\pgfqpoint{9.000000in}{0.550000in}}%
\pgfusepath{stroke}%
\end{pgfscope}%
\begin{pgfscope}%
\pgfsetrectcap%
\pgfsetmiterjoin%
\pgfsetlinewidth{0.803000pt}%
\definecolor{currentstroke}{rgb}{0.000000,0.000000,0.000000}%
\pgfsetstrokecolor{currentstroke}%
\pgfsetdash{}{0pt}%
\pgfpathmoveto{\pgfqpoint{1.250000in}{4.400000in}}%
\pgfpathlineto{\pgfqpoint{9.000000in}{4.400000in}}%
\pgfusepath{stroke}%
\end{pgfscope}%
\begin{pgfscope}%
\definecolor{textcolor}{rgb}{0.000000,0.000000,0.000000}%
\pgfsetstrokecolor{textcolor}%
\pgfsetfillcolor{textcolor}%
\pgftext[x=5.125000in,y=4.483333in,,base]{\color{textcolor}{\sffamily\fontsize{12.000000}{14.400000}\selectfont\catcode`\^=\active\def^{\ifmmode\sp\else\^{}\fi}\catcode`\%=\active\def%{\%}Histograma de HE}}%
\end{pgfscope}%
\end{pgfpicture}%
\makeatother%
\endgroup%
}
		\caption{HE}
		\label{fig:he_1}
	\end{subfigure}
	\hfill
	\begin{subfigure}[b]{0.45\textwidth}
		\resizebox{\linewidth}{!}{%% Creator: Matplotlib, PGF backend
%%
%% To include the figure in your LaTeX document, write
%%   \input{<filename>.pgf}
%%
%% Make sure the required packages are loaded in your preamble
%%   \usepackage{pgf}
%%
%% Also ensure that all the required font packages are loaded; for instance,
%% the lmodern package is sometimes necessary when using math font.
%%   \usepackage{lmodern}
%%
%% Figures using additional raster images can only be included by \input if
%% they are in the same directory as the main LaTeX file. For loading figures
%% from other directories you can use the `import` package
%%   \usepackage{import}
%%
%% and then include the figures with
%%   \import{<path to file>}{<filename>.pgf}
%%
%% Matplotlib used the following preamble
%%   \def\mathdefault#1{#1}
%%   \everymath=\expandafter{\the\everymath\displaystyle}
%%   \IfFileExists{scrextend.sty}{
%%     \usepackage[fontsize=10.000000pt]{scrextend}
%%   }{
%%     \renewcommand{\normalsize}{\fontsize{10.000000}{12.000000}\selectfont}
%%     \normalsize
%%   }
%%   
%%   \ifdefined\pdftexversion\else  % non-pdftex case.
%%     \usepackage{fontspec}
%%     \setmainfont{DejaVuSerif.ttf}[Path=\detokenize{/home/elias/.virtualenvs/src-pway/lib/python3.13/site-packages/matplotlib/mpl-data/fonts/ttf/}]
%%     \setsansfont{DejaVuSans.ttf}[Path=\detokenize{/home/elias/.virtualenvs/src-pway/lib/python3.13/site-packages/matplotlib/mpl-data/fonts/ttf/}]
%%     \setmonofont{DejaVuSansMono.ttf}[Path=\detokenize{/home/elias/.virtualenvs/src-pway/lib/python3.13/site-packages/matplotlib/mpl-data/fonts/ttf/}]
%%   \fi
%%   \makeatletter\@ifpackageloaded{underscore}{}{\usepackage[strings]{underscore}}\makeatother
%%
\begingroup%
\makeatletter%
\begin{pgfpicture}%
\pgfpathrectangle{\pgfpointorigin}{\pgfqpoint{10.000000in}{5.000000in}}%
\pgfusepath{use as bounding box, clip}%
\begin{pgfscope}%
\pgfsetbuttcap%
\pgfsetmiterjoin%
\definecolor{currentfill}{rgb}{1.000000,1.000000,1.000000}%
\pgfsetfillcolor{currentfill}%
\pgfsetlinewidth{0.000000pt}%
\definecolor{currentstroke}{rgb}{1.000000,1.000000,1.000000}%
\pgfsetstrokecolor{currentstroke}%
\pgfsetdash{}{0pt}%
\pgfpathmoveto{\pgfqpoint{0.000000in}{0.000000in}}%
\pgfpathlineto{\pgfqpoint{10.000000in}{0.000000in}}%
\pgfpathlineto{\pgfqpoint{10.000000in}{5.000000in}}%
\pgfpathlineto{\pgfqpoint{0.000000in}{5.000000in}}%
\pgfpathlineto{\pgfqpoint{0.000000in}{0.000000in}}%
\pgfpathclose%
\pgfusepath{fill}%
\end{pgfscope}%
\begin{pgfscope}%
\pgfsetbuttcap%
\pgfsetmiterjoin%
\definecolor{currentfill}{rgb}{1.000000,1.000000,1.000000}%
\pgfsetfillcolor{currentfill}%
\pgfsetlinewidth{0.000000pt}%
\definecolor{currentstroke}{rgb}{0.000000,0.000000,0.000000}%
\pgfsetstrokecolor{currentstroke}%
\pgfsetstrokeopacity{0.000000}%
\pgfsetdash{}{0pt}%
\pgfpathmoveto{\pgfqpoint{1.250000in}{0.550000in}}%
\pgfpathlineto{\pgfqpoint{9.000000in}{0.550000in}}%
\pgfpathlineto{\pgfqpoint{9.000000in}{4.400000in}}%
\pgfpathlineto{\pgfqpoint{1.250000in}{4.400000in}}%
\pgfpathlineto{\pgfqpoint{1.250000in}{0.550000in}}%
\pgfpathclose%
\pgfusepath{fill}%
\end{pgfscope}%
\begin{pgfscope}%
\pgfpathrectangle{\pgfqpoint{1.250000in}{0.550000in}}{\pgfqpoint{7.750000in}{3.850000in}}%
\pgfusepath{clip}%
\pgfsetbuttcap%
\pgfsetmiterjoin%
\definecolor{currentfill}{rgb}{0.501961,0.501961,0.501961}%
\pgfsetfillcolor{currentfill}%
\pgfsetfillopacity{0.700000}%
\pgfsetlinewidth{0.000000pt}%
\definecolor{currentstroke}{rgb}{0.000000,0.000000,0.000000}%
\pgfsetstrokecolor{currentstroke}%
\pgfsetstrokeopacity{0.700000}%
\pgfsetdash{}{0pt}%
\pgfpathmoveto{\pgfqpoint{1.602273in}{0.550000in}}%
\pgfpathlineto{\pgfqpoint{1.629794in}{0.550000in}}%
\pgfpathlineto{\pgfqpoint{1.629794in}{0.550000in}}%
\pgfpathlineto{\pgfqpoint{1.602273in}{0.550000in}}%
\pgfpathlineto{\pgfqpoint{1.602273in}{0.550000in}}%
\pgfpathclose%
\pgfusepath{fill}%
\end{pgfscope}%
\begin{pgfscope}%
\pgfpathrectangle{\pgfqpoint{1.250000in}{0.550000in}}{\pgfqpoint{7.750000in}{3.850000in}}%
\pgfusepath{clip}%
\pgfsetbuttcap%
\pgfsetmiterjoin%
\definecolor{currentfill}{rgb}{0.501961,0.501961,0.501961}%
\pgfsetfillcolor{currentfill}%
\pgfsetfillopacity{0.700000}%
\pgfsetlinewidth{0.000000pt}%
\definecolor{currentstroke}{rgb}{0.000000,0.000000,0.000000}%
\pgfsetstrokecolor{currentstroke}%
\pgfsetstrokeopacity{0.700000}%
\pgfsetdash{}{0pt}%
\pgfpathmoveto{\pgfqpoint{1.629794in}{0.550000in}}%
\pgfpathlineto{\pgfqpoint{1.657315in}{0.550000in}}%
\pgfpathlineto{\pgfqpoint{1.657315in}{0.555206in}}%
\pgfpathlineto{\pgfqpoint{1.629794in}{0.555206in}}%
\pgfpathlineto{\pgfqpoint{1.629794in}{0.550000in}}%
\pgfpathclose%
\pgfusepath{fill}%
\end{pgfscope}%
\begin{pgfscope}%
\pgfpathrectangle{\pgfqpoint{1.250000in}{0.550000in}}{\pgfqpoint{7.750000in}{3.850000in}}%
\pgfusepath{clip}%
\pgfsetbuttcap%
\pgfsetmiterjoin%
\definecolor{currentfill}{rgb}{0.501961,0.501961,0.501961}%
\pgfsetfillcolor{currentfill}%
\pgfsetfillopacity{0.700000}%
\pgfsetlinewidth{0.000000pt}%
\definecolor{currentstroke}{rgb}{0.000000,0.000000,0.000000}%
\pgfsetstrokecolor{currentstroke}%
\pgfsetstrokeopacity{0.700000}%
\pgfsetdash{}{0pt}%
\pgfpathmoveto{\pgfqpoint{1.657315in}{0.550000in}}%
\pgfpathlineto{\pgfqpoint{1.684837in}{0.550000in}}%
\pgfpathlineto{\pgfqpoint{1.684837in}{0.634516in}}%
\pgfpathlineto{\pgfqpoint{1.657315in}{0.634516in}}%
\pgfpathlineto{\pgfqpoint{1.657315in}{0.550000in}}%
\pgfpathclose%
\pgfusepath{fill}%
\end{pgfscope}%
\begin{pgfscope}%
\pgfpathrectangle{\pgfqpoint{1.250000in}{0.550000in}}{\pgfqpoint{7.750000in}{3.850000in}}%
\pgfusepath{clip}%
\pgfsetbuttcap%
\pgfsetmiterjoin%
\definecolor{currentfill}{rgb}{0.501961,0.501961,0.501961}%
\pgfsetfillcolor{currentfill}%
\pgfsetfillopacity{0.700000}%
\pgfsetlinewidth{0.000000pt}%
\definecolor{currentstroke}{rgb}{0.000000,0.000000,0.000000}%
\pgfsetstrokecolor{currentstroke}%
\pgfsetstrokeopacity{0.700000}%
\pgfsetdash{}{0pt}%
\pgfpathmoveto{\pgfqpoint{1.684837in}{0.550000in}}%
\pgfpathlineto{\pgfqpoint{1.712358in}{0.550000in}}%
\pgfpathlineto{\pgfqpoint{1.712358in}{4.216667in}}%
\pgfpathlineto{\pgfqpoint{1.684837in}{4.216667in}}%
\pgfpathlineto{\pgfqpoint{1.684837in}{0.550000in}}%
\pgfpathclose%
\pgfusepath{fill}%
\end{pgfscope}%
\begin{pgfscope}%
\pgfpathrectangle{\pgfqpoint{1.250000in}{0.550000in}}{\pgfqpoint{7.750000in}{3.850000in}}%
\pgfusepath{clip}%
\pgfsetbuttcap%
\pgfsetmiterjoin%
\definecolor{currentfill}{rgb}{0.501961,0.501961,0.501961}%
\pgfsetfillcolor{currentfill}%
\pgfsetfillopacity{0.700000}%
\pgfsetlinewidth{0.000000pt}%
\definecolor{currentstroke}{rgb}{0.000000,0.000000,0.000000}%
\pgfsetstrokecolor{currentstroke}%
\pgfsetstrokeopacity{0.700000}%
\pgfsetdash{}{0pt}%
\pgfpathmoveto{\pgfqpoint{1.712358in}{0.550000in}}%
\pgfpathlineto{\pgfqpoint{1.739879in}{0.550000in}}%
\pgfpathlineto{\pgfqpoint{1.739879in}{0.597158in}}%
\pgfpathlineto{\pgfqpoint{1.712358in}{0.597158in}}%
\pgfpathlineto{\pgfqpoint{1.712358in}{0.550000in}}%
\pgfpathclose%
\pgfusepath{fill}%
\end{pgfscope}%
\begin{pgfscope}%
\pgfpathrectangle{\pgfqpoint{1.250000in}{0.550000in}}{\pgfqpoint{7.750000in}{3.850000in}}%
\pgfusepath{clip}%
\pgfsetbuttcap%
\pgfsetmiterjoin%
\definecolor{currentfill}{rgb}{0.501961,0.501961,0.501961}%
\pgfsetfillcolor{currentfill}%
\pgfsetfillopacity{0.700000}%
\pgfsetlinewidth{0.000000pt}%
\definecolor{currentstroke}{rgb}{0.000000,0.000000,0.000000}%
\pgfsetstrokecolor{currentstroke}%
\pgfsetstrokeopacity{0.700000}%
\pgfsetdash{}{0pt}%
\pgfpathmoveto{\pgfqpoint{1.739879in}{0.550000in}}%
\pgfpathlineto{\pgfqpoint{1.767401in}{0.550000in}}%
\pgfpathlineto{\pgfqpoint{1.767401in}{1.176524in}}%
\pgfpathlineto{\pgfqpoint{1.739879in}{1.176524in}}%
\pgfpathlineto{\pgfqpoint{1.739879in}{0.550000in}}%
\pgfpathclose%
\pgfusepath{fill}%
\end{pgfscope}%
\begin{pgfscope}%
\pgfpathrectangle{\pgfqpoint{1.250000in}{0.550000in}}{\pgfqpoint{7.750000in}{3.850000in}}%
\pgfusepath{clip}%
\pgfsetbuttcap%
\pgfsetmiterjoin%
\definecolor{currentfill}{rgb}{0.501961,0.501961,0.501961}%
\pgfsetfillcolor{currentfill}%
\pgfsetfillopacity{0.700000}%
\pgfsetlinewidth{0.000000pt}%
\definecolor{currentstroke}{rgb}{0.000000,0.000000,0.000000}%
\pgfsetstrokecolor{currentstroke}%
\pgfsetstrokeopacity{0.700000}%
\pgfsetdash{}{0pt}%
\pgfpathmoveto{\pgfqpoint{1.767401in}{0.550000in}}%
\pgfpathlineto{\pgfqpoint{1.794922in}{0.550000in}}%
\pgfpathlineto{\pgfqpoint{1.794922in}{2.075430in}}%
\pgfpathlineto{\pgfqpoint{1.767401in}{2.075430in}}%
\pgfpathlineto{\pgfqpoint{1.767401in}{0.550000in}}%
\pgfpathclose%
\pgfusepath{fill}%
\end{pgfscope}%
\begin{pgfscope}%
\pgfpathrectangle{\pgfqpoint{1.250000in}{0.550000in}}{\pgfqpoint{7.750000in}{3.850000in}}%
\pgfusepath{clip}%
\pgfsetbuttcap%
\pgfsetmiterjoin%
\definecolor{currentfill}{rgb}{0.501961,0.501961,0.501961}%
\pgfsetfillcolor{currentfill}%
\pgfsetfillopacity{0.700000}%
\pgfsetlinewidth{0.000000pt}%
\definecolor{currentstroke}{rgb}{0.000000,0.000000,0.000000}%
\pgfsetstrokecolor{currentstroke}%
\pgfsetstrokeopacity{0.700000}%
\pgfsetdash{}{0pt}%
\pgfpathmoveto{\pgfqpoint{1.794922in}{0.550000in}}%
\pgfpathlineto{\pgfqpoint{1.822443in}{0.550000in}}%
\pgfpathlineto{\pgfqpoint{1.822443in}{0.638957in}}%
\pgfpathlineto{\pgfqpoint{1.794922in}{0.638957in}}%
\pgfpathlineto{\pgfqpoint{1.794922in}{0.550000in}}%
\pgfpathclose%
\pgfusepath{fill}%
\end{pgfscope}%
\begin{pgfscope}%
\pgfpathrectangle{\pgfqpoint{1.250000in}{0.550000in}}{\pgfqpoint{7.750000in}{3.850000in}}%
\pgfusepath{clip}%
\pgfsetbuttcap%
\pgfsetmiterjoin%
\definecolor{currentfill}{rgb}{0.501961,0.501961,0.501961}%
\pgfsetfillcolor{currentfill}%
\pgfsetfillopacity{0.700000}%
\pgfsetlinewidth{0.000000pt}%
\definecolor{currentstroke}{rgb}{0.000000,0.000000,0.000000}%
\pgfsetstrokecolor{currentstroke}%
\pgfsetstrokeopacity{0.700000}%
\pgfsetdash{}{0pt}%
\pgfpathmoveto{\pgfqpoint{1.822443in}{0.550000in}}%
\pgfpathlineto{\pgfqpoint{1.849964in}{0.550000in}}%
\pgfpathlineto{\pgfqpoint{1.849964in}{1.248026in}}%
\pgfpathlineto{\pgfqpoint{1.822443in}{1.248026in}}%
\pgfpathlineto{\pgfqpoint{1.822443in}{0.550000in}}%
\pgfpathclose%
\pgfusepath{fill}%
\end{pgfscope}%
\begin{pgfscope}%
\pgfpathrectangle{\pgfqpoint{1.250000in}{0.550000in}}{\pgfqpoint{7.750000in}{3.850000in}}%
\pgfusepath{clip}%
\pgfsetbuttcap%
\pgfsetmiterjoin%
\definecolor{currentfill}{rgb}{0.501961,0.501961,0.501961}%
\pgfsetfillcolor{currentfill}%
\pgfsetfillopacity{0.700000}%
\pgfsetlinewidth{0.000000pt}%
\definecolor{currentstroke}{rgb}{0.000000,0.000000,0.000000}%
\pgfsetstrokecolor{currentstroke}%
\pgfsetstrokeopacity{0.700000}%
\pgfsetdash{}{0pt}%
\pgfpathmoveto{\pgfqpoint{1.849964in}{0.550000in}}%
\pgfpathlineto{\pgfqpoint{1.877486in}{0.550000in}}%
\pgfpathlineto{\pgfqpoint{1.877486in}{1.045462in}}%
\pgfpathlineto{\pgfqpoint{1.849964in}{1.045462in}}%
\pgfpathlineto{\pgfqpoint{1.849964in}{0.550000in}}%
\pgfpathclose%
\pgfusepath{fill}%
\end{pgfscope}%
\begin{pgfscope}%
\pgfpathrectangle{\pgfqpoint{1.250000in}{0.550000in}}{\pgfqpoint{7.750000in}{3.850000in}}%
\pgfusepath{clip}%
\pgfsetbuttcap%
\pgfsetmiterjoin%
\definecolor{currentfill}{rgb}{0.501961,0.501961,0.501961}%
\pgfsetfillcolor{currentfill}%
\pgfsetfillopacity{0.700000}%
\pgfsetlinewidth{0.000000pt}%
\definecolor{currentstroke}{rgb}{0.000000,0.000000,0.000000}%
\pgfsetstrokecolor{currentstroke}%
\pgfsetstrokeopacity{0.700000}%
\pgfsetdash{}{0pt}%
\pgfpathmoveto{\pgfqpoint{1.877486in}{0.550000in}}%
\pgfpathlineto{\pgfqpoint{1.905007in}{0.550000in}}%
\pgfpathlineto{\pgfqpoint{1.905007in}{0.679684in}}%
\pgfpathlineto{\pgfqpoint{1.877486in}{0.679684in}}%
\pgfpathlineto{\pgfqpoint{1.877486in}{0.550000in}}%
\pgfpathclose%
\pgfusepath{fill}%
\end{pgfscope}%
\begin{pgfscope}%
\pgfpathrectangle{\pgfqpoint{1.250000in}{0.550000in}}{\pgfqpoint{7.750000in}{3.850000in}}%
\pgfusepath{clip}%
\pgfsetbuttcap%
\pgfsetmiterjoin%
\definecolor{currentfill}{rgb}{0.501961,0.501961,0.501961}%
\pgfsetfillcolor{currentfill}%
\pgfsetfillopacity{0.700000}%
\pgfsetlinewidth{0.000000pt}%
\definecolor{currentstroke}{rgb}{0.000000,0.000000,0.000000}%
\pgfsetstrokecolor{currentstroke}%
\pgfsetstrokeopacity{0.700000}%
\pgfsetdash{}{0pt}%
\pgfpathmoveto{\pgfqpoint{1.905007in}{0.550000in}}%
\pgfpathlineto{\pgfqpoint{1.932528in}{0.550000in}}%
\pgfpathlineto{\pgfqpoint{1.932528in}{1.116046in}}%
\pgfpathlineto{\pgfqpoint{1.905007in}{1.116046in}}%
\pgfpathlineto{\pgfqpoint{1.905007in}{0.550000in}}%
\pgfpathclose%
\pgfusepath{fill}%
\end{pgfscope}%
\begin{pgfscope}%
\pgfpathrectangle{\pgfqpoint{1.250000in}{0.550000in}}{\pgfqpoint{7.750000in}{3.850000in}}%
\pgfusepath{clip}%
\pgfsetbuttcap%
\pgfsetmiterjoin%
\definecolor{currentfill}{rgb}{0.501961,0.501961,0.501961}%
\pgfsetfillcolor{currentfill}%
\pgfsetfillopacity{0.700000}%
\pgfsetlinewidth{0.000000pt}%
\definecolor{currentstroke}{rgb}{0.000000,0.000000,0.000000}%
\pgfsetstrokecolor{currentstroke}%
\pgfsetstrokeopacity{0.700000}%
\pgfsetdash{}{0pt}%
\pgfpathmoveto{\pgfqpoint{1.932528in}{0.550000in}}%
\pgfpathlineto{\pgfqpoint{1.960050in}{0.550000in}}%
\pgfpathlineto{\pgfqpoint{1.960050in}{0.752870in}}%
\pgfpathlineto{\pgfqpoint{1.932528in}{0.752870in}}%
\pgfpathlineto{\pgfqpoint{1.932528in}{0.550000in}}%
\pgfpathclose%
\pgfusepath{fill}%
\end{pgfscope}%
\begin{pgfscope}%
\pgfpathrectangle{\pgfqpoint{1.250000in}{0.550000in}}{\pgfqpoint{7.750000in}{3.850000in}}%
\pgfusepath{clip}%
\pgfsetbuttcap%
\pgfsetmiterjoin%
\definecolor{currentfill}{rgb}{0.501961,0.501961,0.501961}%
\pgfsetfillcolor{currentfill}%
\pgfsetfillopacity{0.700000}%
\pgfsetlinewidth{0.000000pt}%
\definecolor{currentstroke}{rgb}{0.000000,0.000000,0.000000}%
\pgfsetstrokecolor{currentstroke}%
\pgfsetstrokeopacity{0.700000}%
\pgfsetdash{}{0pt}%
\pgfpathmoveto{\pgfqpoint{1.960050in}{0.550000in}}%
\pgfpathlineto{\pgfqpoint{1.987571in}{0.550000in}}%
\pgfpathlineto{\pgfqpoint{1.987571in}{0.732353in}}%
\pgfpathlineto{\pgfqpoint{1.960050in}{0.732353in}}%
\pgfpathlineto{\pgfqpoint{1.960050in}{0.550000in}}%
\pgfpathclose%
\pgfusepath{fill}%
\end{pgfscope}%
\begin{pgfscope}%
\pgfpathrectangle{\pgfqpoint{1.250000in}{0.550000in}}{\pgfqpoint{7.750000in}{3.850000in}}%
\pgfusepath{clip}%
\pgfsetbuttcap%
\pgfsetmiterjoin%
\definecolor{currentfill}{rgb}{0.501961,0.501961,0.501961}%
\pgfsetfillcolor{currentfill}%
\pgfsetfillopacity{0.700000}%
\pgfsetlinewidth{0.000000pt}%
\definecolor{currentstroke}{rgb}{0.000000,0.000000,0.000000}%
\pgfsetstrokecolor{currentstroke}%
\pgfsetstrokeopacity{0.700000}%
\pgfsetdash{}{0pt}%
\pgfpathmoveto{\pgfqpoint{1.987571in}{0.550000in}}%
\pgfpathlineto{\pgfqpoint{2.015092in}{0.550000in}}%
\pgfpathlineto{\pgfqpoint{2.015092in}{1.029998in}}%
\pgfpathlineto{\pgfqpoint{1.987571in}{1.029998in}}%
\pgfpathlineto{\pgfqpoint{1.987571in}{0.550000in}}%
\pgfpathclose%
\pgfusepath{fill}%
\end{pgfscope}%
\begin{pgfscope}%
\pgfpathrectangle{\pgfqpoint{1.250000in}{0.550000in}}{\pgfqpoint{7.750000in}{3.850000in}}%
\pgfusepath{clip}%
\pgfsetbuttcap%
\pgfsetmiterjoin%
\definecolor{currentfill}{rgb}{0.501961,0.501961,0.501961}%
\pgfsetfillcolor{currentfill}%
\pgfsetfillopacity{0.700000}%
\pgfsetlinewidth{0.000000pt}%
\definecolor{currentstroke}{rgb}{0.000000,0.000000,0.000000}%
\pgfsetstrokecolor{currentstroke}%
\pgfsetstrokeopacity{0.700000}%
\pgfsetdash{}{0pt}%
\pgfpathmoveto{\pgfqpoint{2.015092in}{0.550000in}}%
\pgfpathlineto{\pgfqpoint{2.042614in}{0.550000in}}%
\pgfpathlineto{\pgfqpoint{2.042614in}{0.722095in}}%
\pgfpathlineto{\pgfqpoint{2.015092in}{0.722095in}}%
\pgfpathlineto{\pgfqpoint{2.015092in}{0.550000in}}%
\pgfpathclose%
\pgfusepath{fill}%
\end{pgfscope}%
\begin{pgfscope}%
\pgfpathrectangle{\pgfqpoint{1.250000in}{0.550000in}}{\pgfqpoint{7.750000in}{3.850000in}}%
\pgfusepath{clip}%
\pgfsetbuttcap%
\pgfsetmiterjoin%
\definecolor{currentfill}{rgb}{0.501961,0.501961,0.501961}%
\pgfsetfillcolor{currentfill}%
\pgfsetfillopacity{0.700000}%
\pgfsetlinewidth{0.000000pt}%
\definecolor{currentstroke}{rgb}{0.000000,0.000000,0.000000}%
\pgfsetstrokecolor{currentstroke}%
\pgfsetstrokeopacity{0.700000}%
\pgfsetdash{}{0pt}%
\pgfpathmoveto{\pgfqpoint{2.042614in}{0.550000in}}%
\pgfpathlineto{\pgfqpoint{2.070135in}{0.550000in}}%
\pgfpathlineto{\pgfqpoint{2.070135in}{0.843358in}}%
\pgfpathlineto{\pgfqpoint{2.042614in}{0.843358in}}%
\pgfpathlineto{\pgfqpoint{2.042614in}{0.550000in}}%
\pgfpathclose%
\pgfusepath{fill}%
\end{pgfscope}%
\begin{pgfscope}%
\pgfpathrectangle{\pgfqpoint{1.250000in}{0.550000in}}{\pgfqpoint{7.750000in}{3.850000in}}%
\pgfusepath{clip}%
\pgfsetbuttcap%
\pgfsetmiterjoin%
\definecolor{currentfill}{rgb}{0.501961,0.501961,0.501961}%
\pgfsetfillcolor{currentfill}%
\pgfsetfillopacity{0.700000}%
\pgfsetlinewidth{0.000000pt}%
\definecolor{currentstroke}{rgb}{0.000000,0.000000,0.000000}%
\pgfsetstrokecolor{currentstroke}%
\pgfsetstrokeopacity{0.700000}%
\pgfsetdash{}{0pt}%
\pgfpathmoveto{\pgfqpoint{2.070135in}{0.550000in}}%
\pgfpathlineto{\pgfqpoint{2.097656in}{0.550000in}}%
\pgfpathlineto{\pgfqpoint{2.097656in}{0.925884in}}%
\pgfpathlineto{\pgfqpoint{2.070135in}{0.925884in}}%
\pgfpathlineto{\pgfqpoint{2.070135in}{0.550000in}}%
\pgfpathclose%
\pgfusepath{fill}%
\end{pgfscope}%
\begin{pgfscope}%
\pgfpathrectangle{\pgfqpoint{1.250000in}{0.550000in}}{\pgfqpoint{7.750000in}{3.850000in}}%
\pgfusepath{clip}%
\pgfsetbuttcap%
\pgfsetmiterjoin%
\definecolor{currentfill}{rgb}{0.501961,0.501961,0.501961}%
\pgfsetfillcolor{currentfill}%
\pgfsetfillopacity{0.700000}%
\pgfsetlinewidth{0.000000pt}%
\definecolor{currentstroke}{rgb}{0.000000,0.000000,0.000000}%
\pgfsetstrokecolor{currentstroke}%
\pgfsetstrokeopacity{0.700000}%
\pgfsetdash{}{0pt}%
\pgfpathmoveto{\pgfqpoint{2.097656in}{0.550000in}}%
\pgfpathlineto{\pgfqpoint{2.125178in}{0.550000in}}%
\pgfpathlineto{\pgfqpoint{2.125178in}{0.796659in}}%
\pgfpathlineto{\pgfqpoint{2.097656in}{0.796659in}}%
\pgfpathlineto{\pgfqpoint{2.097656in}{0.550000in}}%
\pgfpathclose%
\pgfusepath{fill}%
\end{pgfscope}%
\begin{pgfscope}%
\pgfpathrectangle{\pgfqpoint{1.250000in}{0.550000in}}{\pgfqpoint{7.750000in}{3.850000in}}%
\pgfusepath{clip}%
\pgfsetbuttcap%
\pgfsetmiterjoin%
\definecolor{currentfill}{rgb}{0.501961,0.501961,0.501961}%
\pgfsetfillcolor{currentfill}%
\pgfsetfillopacity{0.700000}%
\pgfsetlinewidth{0.000000pt}%
\definecolor{currentstroke}{rgb}{0.000000,0.000000,0.000000}%
\pgfsetstrokecolor{currentstroke}%
\pgfsetstrokeopacity{0.700000}%
\pgfsetdash{}{0pt}%
\pgfpathmoveto{\pgfqpoint{2.125178in}{0.550000in}}%
\pgfpathlineto{\pgfqpoint{2.152699in}{0.550000in}}%
\pgfpathlineto{\pgfqpoint{2.152699in}{1.010400in}}%
\pgfpathlineto{\pgfqpoint{2.125178in}{1.010400in}}%
\pgfpathlineto{\pgfqpoint{2.125178in}{0.550000in}}%
\pgfpathclose%
\pgfusepath{fill}%
\end{pgfscope}%
\begin{pgfscope}%
\pgfpathrectangle{\pgfqpoint{1.250000in}{0.550000in}}{\pgfqpoint{7.750000in}{3.850000in}}%
\pgfusepath{clip}%
\pgfsetbuttcap%
\pgfsetmiterjoin%
\definecolor{currentfill}{rgb}{0.501961,0.501961,0.501961}%
\pgfsetfillcolor{currentfill}%
\pgfsetfillopacity{0.700000}%
\pgfsetlinewidth{0.000000pt}%
\definecolor{currentstroke}{rgb}{0.000000,0.000000,0.000000}%
\pgfsetstrokecolor{currentstroke}%
\pgfsetstrokeopacity{0.700000}%
\pgfsetdash{}{0pt}%
\pgfpathmoveto{\pgfqpoint{2.152699in}{0.550000in}}%
\pgfpathlineto{\pgfqpoint{2.180220in}{0.550000in}}%
\pgfpathlineto{\pgfqpoint{2.180220in}{0.932927in}}%
\pgfpathlineto{\pgfqpoint{2.152699in}{0.932927in}}%
\pgfpathlineto{\pgfqpoint{2.152699in}{0.550000in}}%
\pgfpathclose%
\pgfusepath{fill}%
\end{pgfscope}%
\begin{pgfscope}%
\pgfpathrectangle{\pgfqpoint{1.250000in}{0.550000in}}{\pgfqpoint{7.750000in}{3.850000in}}%
\pgfusepath{clip}%
\pgfsetbuttcap%
\pgfsetmiterjoin%
\definecolor{currentfill}{rgb}{0.501961,0.501961,0.501961}%
\pgfsetfillcolor{currentfill}%
\pgfsetfillopacity{0.700000}%
\pgfsetlinewidth{0.000000pt}%
\definecolor{currentstroke}{rgb}{0.000000,0.000000,0.000000}%
\pgfsetstrokecolor{currentstroke}%
\pgfsetstrokeopacity{0.700000}%
\pgfsetdash{}{0pt}%
\pgfpathmoveto{\pgfqpoint{2.180220in}{0.550000in}}%
\pgfpathlineto{\pgfqpoint{2.207741in}{0.550000in}}%
\pgfpathlineto{\pgfqpoint{2.207741in}{0.906592in}}%
\pgfpathlineto{\pgfqpoint{2.180220in}{0.906592in}}%
\pgfpathlineto{\pgfqpoint{2.180220in}{0.550000in}}%
\pgfpathclose%
\pgfusepath{fill}%
\end{pgfscope}%
\begin{pgfscope}%
\pgfpathrectangle{\pgfqpoint{1.250000in}{0.550000in}}{\pgfqpoint{7.750000in}{3.850000in}}%
\pgfusepath{clip}%
\pgfsetbuttcap%
\pgfsetmiterjoin%
\definecolor{currentfill}{rgb}{0.501961,0.501961,0.501961}%
\pgfsetfillcolor{currentfill}%
\pgfsetfillopacity{0.700000}%
\pgfsetlinewidth{0.000000pt}%
\definecolor{currentstroke}{rgb}{0.000000,0.000000,0.000000}%
\pgfsetstrokecolor{currentstroke}%
\pgfsetstrokeopacity{0.700000}%
\pgfsetdash{}{0pt}%
\pgfpathmoveto{\pgfqpoint{2.207741in}{0.550000in}}%
\pgfpathlineto{\pgfqpoint{2.235263in}{0.550000in}}%
\pgfpathlineto{\pgfqpoint{2.235263in}{1.190304in}}%
\pgfpathlineto{\pgfqpoint{2.207741in}{1.190304in}}%
\pgfpathlineto{\pgfqpoint{2.207741in}{0.550000in}}%
\pgfpathclose%
\pgfusepath{fill}%
\end{pgfscope}%
\begin{pgfscope}%
\pgfpathrectangle{\pgfqpoint{1.250000in}{0.550000in}}{\pgfqpoint{7.750000in}{3.850000in}}%
\pgfusepath{clip}%
\pgfsetbuttcap%
\pgfsetmiterjoin%
\definecolor{currentfill}{rgb}{0.501961,0.501961,0.501961}%
\pgfsetfillcolor{currentfill}%
\pgfsetfillopacity{0.700000}%
\pgfsetlinewidth{0.000000pt}%
\definecolor{currentstroke}{rgb}{0.000000,0.000000,0.000000}%
\pgfsetstrokecolor{currentstroke}%
\pgfsetstrokeopacity{0.700000}%
\pgfsetdash{}{0pt}%
\pgfpathmoveto{\pgfqpoint{2.235263in}{0.550000in}}%
\pgfpathlineto{\pgfqpoint{2.262784in}{0.550000in}}%
\pgfpathlineto{\pgfqpoint{2.262784in}{1.226285in}}%
\pgfpathlineto{\pgfqpoint{2.235263in}{1.226285in}}%
\pgfpathlineto{\pgfqpoint{2.235263in}{0.550000in}}%
\pgfpathclose%
\pgfusepath{fill}%
\end{pgfscope}%
\begin{pgfscope}%
\pgfpathrectangle{\pgfqpoint{1.250000in}{0.550000in}}{\pgfqpoint{7.750000in}{3.850000in}}%
\pgfusepath{clip}%
\pgfsetbuttcap%
\pgfsetmiterjoin%
\definecolor{currentfill}{rgb}{0.501961,0.501961,0.501961}%
\pgfsetfillcolor{currentfill}%
\pgfsetfillopacity{0.700000}%
\pgfsetlinewidth{0.000000pt}%
\definecolor{currentstroke}{rgb}{0.000000,0.000000,0.000000}%
\pgfsetstrokecolor{currentstroke}%
\pgfsetstrokeopacity{0.700000}%
\pgfsetdash{}{0pt}%
\pgfpathmoveto{\pgfqpoint{2.262784in}{0.550000in}}%
\pgfpathlineto{\pgfqpoint{2.290305in}{0.550000in}}%
\pgfpathlineto{\pgfqpoint{2.290305in}{1.089405in}}%
\pgfpathlineto{\pgfqpoint{2.262784in}{1.089405in}}%
\pgfpathlineto{\pgfqpoint{2.262784in}{0.550000in}}%
\pgfpathclose%
\pgfusepath{fill}%
\end{pgfscope}%
\begin{pgfscope}%
\pgfpathrectangle{\pgfqpoint{1.250000in}{0.550000in}}{\pgfqpoint{7.750000in}{3.850000in}}%
\pgfusepath{clip}%
\pgfsetbuttcap%
\pgfsetmiterjoin%
\definecolor{currentfill}{rgb}{0.501961,0.501961,0.501961}%
\pgfsetfillcolor{currentfill}%
\pgfsetfillopacity{0.700000}%
\pgfsetlinewidth{0.000000pt}%
\definecolor{currentstroke}{rgb}{0.000000,0.000000,0.000000}%
\pgfsetstrokecolor{currentstroke}%
\pgfsetstrokeopacity{0.700000}%
\pgfsetdash{}{0pt}%
\pgfpathmoveto{\pgfqpoint{2.290305in}{0.550000in}}%
\pgfpathlineto{\pgfqpoint{2.317827in}{0.550000in}}%
\pgfpathlineto{\pgfqpoint{2.317827in}{1.623604in}}%
\pgfpathlineto{\pgfqpoint{2.290305in}{1.623604in}}%
\pgfpathlineto{\pgfqpoint{2.290305in}{0.550000in}}%
\pgfpathclose%
\pgfusepath{fill}%
\end{pgfscope}%
\begin{pgfscope}%
\pgfpathrectangle{\pgfqpoint{1.250000in}{0.550000in}}{\pgfqpoint{7.750000in}{3.850000in}}%
\pgfusepath{clip}%
\pgfsetbuttcap%
\pgfsetmiterjoin%
\definecolor{currentfill}{rgb}{0.501961,0.501961,0.501961}%
\pgfsetfillcolor{currentfill}%
\pgfsetfillopacity{0.700000}%
\pgfsetlinewidth{0.000000pt}%
\definecolor{currentstroke}{rgb}{0.000000,0.000000,0.000000}%
\pgfsetstrokecolor{currentstroke}%
\pgfsetstrokeopacity{0.700000}%
\pgfsetdash{}{0pt}%
\pgfpathmoveto{\pgfqpoint{2.317827in}{0.550000in}}%
\pgfpathlineto{\pgfqpoint{2.345348in}{0.550000in}}%
\pgfpathlineto{\pgfqpoint{2.345348in}{1.356122in}}%
\pgfpathlineto{\pgfqpoint{2.317827in}{1.356122in}}%
\pgfpathlineto{\pgfqpoint{2.317827in}{0.550000in}}%
\pgfpathclose%
\pgfusepath{fill}%
\end{pgfscope}%
\begin{pgfscope}%
\pgfpathrectangle{\pgfqpoint{1.250000in}{0.550000in}}{\pgfqpoint{7.750000in}{3.850000in}}%
\pgfusepath{clip}%
\pgfsetbuttcap%
\pgfsetmiterjoin%
\definecolor{currentfill}{rgb}{0.501961,0.501961,0.501961}%
\pgfsetfillcolor{currentfill}%
\pgfsetfillopacity{0.700000}%
\pgfsetlinewidth{0.000000pt}%
\definecolor{currentstroke}{rgb}{0.000000,0.000000,0.000000}%
\pgfsetstrokecolor{currentstroke}%
\pgfsetstrokeopacity{0.700000}%
\pgfsetdash{}{0pt}%
\pgfpathmoveto{\pgfqpoint{2.345348in}{0.550000in}}%
\pgfpathlineto{\pgfqpoint{2.372869in}{0.550000in}}%
\pgfpathlineto{\pgfqpoint{2.372869in}{1.344332in}}%
\pgfpathlineto{\pgfqpoint{2.345348in}{1.344332in}}%
\pgfpathlineto{\pgfqpoint{2.345348in}{0.550000in}}%
\pgfpathclose%
\pgfusepath{fill}%
\end{pgfscope}%
\begin{pgfscope}%
\pgfpathrectangle{\pgfqpoint{1.250000in}{0.550000in}}{\pgfqpoint{7.750000in}{3.850000in}}%
\pgfusepath{clip}%
\pgfsetbuttcap%
\pgfsetmiterjoin%
\definecolor{currentfill}{rgb}{0.501961,0.501961,0.501961}%
\pgfsetfillcolor{currentfill}%
\pgfsetfillopacity{0.700000}%
\pgfsetlinewidth{0.000000pt}%
\definecolor{currentstroke}{rgb}{0.000000,0.000000,0.000000}%
\pgfsetstrokecolor{currentstroke}%
\pgfsetstrokeopacity{0.700000}%
\pgfsetdash{}{0pt}%
\pgfpathmoveto{\pgfqpoint{2.372869in}{0.550000in}}%
\pgfpathlineto{\pgfqpoint{2.400391in}{0.550000in}}%
\pgfpathlineto{\pgfqpoint{2.400391in}{1.777785in}}%
\pgfpathlineto{\pgfqpoint{2.372869in}{1.777785in}}%
\pgfpathlineto{\pgfqpoint{2.372869in}{0.550000in}}%
\pgfpathclose%
\pgfusepath{fill}%
\end{pgfscope}%
\begin{pgfscope}%
\pgfpathrectangle{\pgfqpoint{1.250000in}{0.550000in}}{\pgfqpoint{7.750000in}{3.850000in}}%
\pgfusepath{clip}%
\pgfsetbuttcap%
\pgfsetmiterjoin%
\definecolor{currentfill}{rgb}{0.501961,0.501961,0.501961}%
\pgfsetfillcolor{currentfill}%
\pgfsetfillopacity{0.700000}%
\pgfsetlinewidth{0.000000pt}%
\definecolor{currentstroke}{rgb}{0.000000,0.000000,0.000000}%
\pgfsetstrokecolor{currentstroke}%
\pgfsetstrokeopacity{0.700000}%
\pgfsetdash{}{0pt}%
\pgfpathmoveto{\pgfqpoint{2.400391in}{0.550000in}}%
\pgfpathlineto{\pgfqpoint{2.427912in}{0.550000in}}%
\pgfpathlineto{\pgfqpoint{2.427912in}{1.537403in}}%
\pgfpathlineto{\pgfqpoint{2.400391in}{1.537403in}}%
\pgfpathlineto{\pgfqpoint{2.400391in}{0.550000in}}%
\pgfpathclose%
\pgfusepath{fill}%
\end{pgfscope}%
\begin{pgfscope}%
\pgfpathrectangle{\pgfqpoint{1.250000in}{0.550000in}}{\pgfqpoint{7.750000in}{3.850000in}}%
\pgfusepath{clip}%
\pgfsetbuttcap%
\pgfsetmiterjoin%
\definecolor{currentfill}{rgb}{0.501961,0.501961,0.501961}%
\pgfsetfillcolor{currentfill}%
\pgfsetfillopacity{0.700000}%
\pgfsetlinewidth{0.000000pt}%
\definecolor{currentstroke}{rgb}{0.000000,0.000000,0.000000}%
\pgfsetstrokecolor{currentstroke}%
\pgfsetstrokeopacity{0.700000}%
\pgfsetdash{}{0pt}%
\pgfpathmoveto{\pgfqpoint{2.427912in}{0.550000in}}%
\pgfpathlineto{\pgfqpoint{2.455433in}{0.550000in}}%
\pgfpathlineto{\pgfqpoint{2.455433in}{1.513978in}}%
\pgfpathlineto{\pgfqpoint{2.427912in}{1.513978in}}%
\pgfpathlineto{\pgfqpoint{2.427912in}{0.550000in}}%
\pgfpathclose%
\pgfusepath{fill}%
\end{pgfscope}%
\begin{pgfscope}%
\pgfpathrectangle{\pgfqpoint{1.250000in}{0.550000in}}{\pgfqpoint{7.750000in}{3.850000in}}%
\pgfusepath{clip}%
\pgfsetbuttcap%
\pgfsetmiterjoin%
\definecolor{currentfill}{rgb}{0.501961,0.501961,0.501961}%
\pgfsetfillcolor{currentfill}%
\pgfsetfillopacity{0.700000}%
\pgfsetlinewidth{0.000000pt}%
\definecolor{currentstroke}{rgb}{0.000000,0.000000,0.000000}%
\pgfsetstrokecolor{currentstroke}%
\pgfsetstrokeopacity{0.700000}%
\pgfsetdash{}{0pt}%
\pgfpathmoveto{\pgfqpoint{2.455433in}{0.550000in}}%
\pgfpathlineto{\pgfqpoint{2.482955in}{0.550000in}}%
\pgfpathlineto{\pgfqpoint{2.482955in}{1.840101in}}%
\pgfpathlineto{\pgfqpoint{2.455433in}{1.840101in}}%
\pgfpathlineto{\pgfqpoint{2.455433in}{0.550000in}}%
\pgfpathclose%
\pgfusepath{fill}%
\end{pgfscope}%
\begin{pgfscope}%
\pgfpathrectangle{\pgfqpoint{1.250000in}{0.550000in}}{\pgfqpoint{7.750000in}{3.850000in}}%
\pgfusepath{clip}%
\pgfsetbuttcap%
\pgfsetmiterjoin%
\definecolor{currentfill}{rgb}{0.501961,0.501961,0.501961}%
\pgfsetfillcolor{currentfill}%
\pgfsetfillopacity{0.700000}%
\pgfsetlinewidth{0.000000pt}%
\definecolor{currentstroke}{rgb}{0.000000,0.000000,0.000000}%
\pgfsetstrokecolor{currentstroke}%
\pgfsetstrokeopacity{0.700000}%
\pgfsetdash{}{0pt}%
\pgfpathmoveto{\pgfqpoint{2.482955in}{0.550000in}}%
\pgfpathlineto{\pgfqpoint{2.510476in}{0.550000in}}%
\pgfpathlineto{\pgfqpoint{2.510476in}{1.723125in}}%
\pgfpathlineto{\pgfqpoint{2.482955in}{1.723125in}}%
\pgfpathlineto{\pgfqpoint{2.482955in}{0.550000in}}%
\pgfpathclose%
\pgfusepath{fill}%
\end{pgfscope}%
\begin{pgfscope}%
\pgfpathrectangle{\pgfqpoint{1.250000in}{0.550000in}}{\pgfqpoint{7.750000in}{3.850000in}}%
\pgfusepath{clip}%
\pgfsetbuttcap%
\pgfsetmiterjoin%
\definecolor{currentfill}{rgb}{0.501961,0.501961,0.501961}%
\pgfsetfillcolor{currentfill}%
\pgfsetfillopacity{0.700000}%
\pgfsetlinewidth{0.000000pt}%
\definecolor{currentstroke}{rgb}{0.000000,0.000000,0.000000}%
\pgfsetstrokecolor{currentstroke}%
\pgfsetstrokeopacity{0.700000}%
\pgfsetdash{}{0pt}%
\pgfpathmoveto{\pgfqpoint{2.510476in}{0.550000in}}%
\pgfpathlineto{\pgfqpoint{2.537997in}{0.550000in}}%
\pgfpathlineto{\pgfqpoint{2.537997in}{1.680867in}}%
\pgfpathlineto{\pgfqpoint{2.510476in}{1.680867in}}%
\pgfpathlineto{\pgfqpoint{2.510476in}{0.550000in}}%
\pgfpathclose%
\pgfusepath{fill}%
\end{pgfscope}%
\begin{pgfscope}%
\pgfpathrectangle{\pgfqpoint{1.250000in}{0.550000in}}{\pgfqpoint{7.750000in}{3.850000in}}%
\pgfusepath{clip}%
\pgfsetbuttcap%
\pgfsetmiterjoin%
\definecolor{currentfill}{rgb}{0.501961,0.501961,0.501961}%
\pgfsetfillcolor{currentfill}%
\pgfsetfillopacity{0.700000}%
\pgfsetlinewidth{0.000000pt}%
\definecolor{currentstroke}{rgb}{0.000000,0.000000,0.000000}%
\pgfsetstrokecolor{currentstroke}%
\pgfsetstrokeopacity{0.700000}%
\pgfsetdash{}{0pt}%
\pgfpathmoveto{\pgfqpoint{2.537997in}{0.550000in}}%
\pgfpathlineto{\pgfqpoint{2.565518in}{0.550000in}}%
\pgfpathlineto{\pgfqpoint{2.565518in}{1.612580in}}%
\pgfpathlineto{\pgfqpoint{2.537997in}{1.612580in}}%
\pgfpathlineto{\pgfqpoint{2.537997in}{0.550000in}}%
\pgfpathclose%
\pgfusepath{fill}%
\end{pgfscope}%
\begin{pgfscope}%
\pgfpathrectangle{\pgfqpoint{1.250000in}{0.550000in}}{\pgfqpoint{7.750000in}{3.850000in}}%
\pgfusepath{clip}%
\pgfsetbuttcap%
\pgfsetmiterjoin%
\definecolor{currentfill}{rgb}{0.501961,0.501961,0.501961}%
\pgfsetfillcolor{currentfill}%
\pgfsetfillopacity{0.700000}%
\pgfsetlinewidth{0.000000pt}%
\definecolor{currentstroke}{rgb}{0.000000,0.000000,0.000000}%
\pgfsetstrokecolor{currentstroke}%
\pgfsetstrokeopacity{0.700000}%
\pgfsetdash{}{0pt}%
\pgfpathmoveto{\pgfqpoint{2.565518in}{0.550000in}}%
\pgfpathlineto{\pgfqpoint{2.593040in}{0.550000in}}%
\pgfpathlineto{\pgfqpoint{2.593040in}{1.534341in}}%
\pgfpathlineto{\pgfqpoint{2.565518in}{1.534341in}}%
\pgfpathlineto{\pgfqpoint{2.565518in}{0.550000in}}%
\pgfpathclose%
\pgfusepath{fill}%
\end{pgfscope}%
\begin{pgfscope}%
\pgfpathrectangle{\pgfqpoint{1.250000in}{0.550000in}}{\pgfqpoint{7.750000in}{3.850000in}}%
\pgfusepath{clip}%
\pgfsetbuttcap%
\pgfsetmiterjoin%
\definecolor{currentfill}{rgb}{0.501961,0.501961,0.501961}%
\pgfsetfillcolor{currentfill}%
\pgfsetfillopacity{0.700000}%
\pgfsetlinewidth{0.000000pt}%
\definecolor{currentstroke}{rgb}{0.000000,0.000000,0.000000}%
\pgfsetstrokecolor{currentstroke}%
\pgfsetstrokeopacity{0.700000}%
\pgfsetdash{}{0pt}%
\pgfpathmoveto{\pgfqpoint{2.593040in}{0.550000in}}%
\pgfpathlineto{\pgfqpoint{2.620561in}{0.550000in}}%
\pgfpathlineto{\pgfqpoint{2.620561in}{1.638609in}}%
\pgfpathlineto{\pgfqpoint{2.593040in}{1.638609in}}%
\pgfpathlineto{\pgfqpoint{2.593040in}{0.550000in}}%
\pgfpathclose%
\pgfusepath{fill}%
\end{pgfscope}%
\begin{pgfscope}%
\pgfpathrectangle{\pgfqpoint{1.250000in}{0.550000in}}{\pgfqpoint{7.750000in}{3.850000in}}%
\pgfusepath{clip}%
\pgfsetbuttcap%
\pgfsetmiterjoin%
\definecolor{currentfill}{rgb}{0.501961,0.501961,0.501961}%
\pgfsetfillcolor{currentfill}%
\pgfsetfillopacity{0.700000}%
\pgfsetlinewidth{0.000000pt}%
\definecolor{currentstroke}{rgb}{0.000000,0.000000,0.000000}%
\pgfsetstrokecolor{currentstroke}%
\pgfsetstrokeopacity{0.700000}%
\pgfsetdash{}{0pt}%
\pgfpathmoveto{\pgfqpoint{2.620561in}{0.550000in}}%
\pgfpathlineto{\pgfqpoint{2.648082in}{0.550000in}}%
\pgfpathlineto{\pgfqpoint{2.648082in}{1.574915in}}%
\pgfpathlineto{\pgfqpoint{2.620561in}{1.574915in}}%
\pgfpathlineto{\pgfqpoint{2.620561in}{0.550000in}}%
\pgfpathclose%
\pgfusepath{fill}%
\end{pgfscope}%
\begin{pgfscope}%
\pgfpathrectangle{\pgfqpoint{1.250000in}{0.550000in}}{\pgfqpoint{7.750000in}{3.850000in}}%
\pgfusepath{clip}%
\pgfsetbuttcap%
\pgfsetmiterjoin%
\definecolor{currentfill}{rgb}{0.501961,0.501961,0.501961}%
\pgfsetfillcolor{currentfill}%
\pgfsetfillopacity{0.700000}%
\pgfsetlinewidth{0.000000pt}%
\definecolor{currentstroke}{rgb}{0.000000,0.000000,0.000000}%
\pgfsetstrokecolor{currentstroke}%
\pgfsetstrokeopacity{0.700000}%
\pgfsetdash{}{0pt}%
\pgfpathmoveto{\pgfqpoint{2.648082in}{0.550000in}}%
\pgfpathlineto{\pgfqpoint{2.675604in}{0.550000in}}%
\pgfpathlineto{\pgfqpoint{2.675604in}{1.368677in}}%
\pgfpathlineto{\pgfqpoint{2.648082in}{1.368677in}}%
\pgfpathlineto{\pgfqpoint{2.648082in}{0.550000in}}%
\pgfpathclose%
\pgfusepath{fill}%
\end{pgfscope}%
\begin{pgfscope}%
\pgfpathrectangle{\pgfqpoint{1.250000in}{0.550000in}}{\pgfqpoint{7.750000in}{3.850000in}}%
\pgfusepath{clip}%
\pgfsetbuttcap%
\pgfsetmiterjoin%
\definecolor{currentfill}{rgb}{0.501961,0.501961,0.501961}%
\pgfsetfillcolor{currentfill}%
\pgfsetfillopacity{0.700000}%
\pgfsetlinewidth{0.000000pt}%
\definecolor{currentstroke}{rgb}{0.000000,0.000000,0.000000}%
\pgfsetstrokecolor{currentstroke}%
\pgfsetstrokeopacity{0.700000}%
\pgfsetdash{}{0pt}%
\pgfpathmoveto{\pgfqpoint{2.675604in}{0.550000in}}%
\pgfpathlineto{\pgfqpoint{2.703125in}{0.550000in}}%
\pgfpathlineto{\pgfqpoint{2.703125in}{1.402054in}}%
\pgfpathlineto{\pgfqpoint{2.675604in}{1.402054in}}%
\pgfpathlineto{\pgfqpoint{2.675604in}{0.550000in}}%
\pgfpathclose%
\pgfusepath{fill}%
\end{pgfscope}%
\begin{pgfscope}%
\pgfpathrectangle{\pgfqpoint{1.250000in}{0.550000in}}{\pgfqpoint{7.750000in}{3.850000in}}%
\pgfusepath{clip}%
\pgfsetbuttcap%
\pgfsetmiterjoin%
\definecolor{currentfill}{rgb}{0.501961,0.501961,0.501961}%
\pgfsetfillcolor{currentfill}%
\pgfsetfillopacity{0.700000}%
\pgfsetlinewidth{0.000000pt}%
\definecolor{currentstroke}{rgb}{0.000000,0.000000,0.000000}%
\pgfsetstrokecolor{currentstroke}%
\pgfsetstrokeopacity{0.700000}%
\pgfsetdash{}{0pt}%
\pgfpathmoveto{\pgfqpoint{2.703125in}{0.550000in}}%
\pgfpathlineto{\pgfqpoint{2.730646in}{0.550000in}}%
\pgfpathlineto{\pgfqpoint{2.730646in}{1.303605in}}%
\pgfpathlineto{\pgfqpoint{2.703125in}{1.303605in}}%
\pgfpathlineto{\pgfqpoint{2.703125in}{0.550000in}}%
\pgfpathclose%
\pgfusepath{fill}%
\end{pgfscope}%
\begin{pgfscope}%
\pgfpathrectangle{\pgfqpoint{1.250000in}{0.550000in}}{\pgfqpoint{7.750000in}{3.850000in}}%
\pgfusepath{clip}%
\pgfsetbuttcap%
\pgfsetmiterjoin%
\definecolor{currentfill}{rgb}{0.501961,0.501961,0.501961}%
\pgfsetfillcolor{currentfill}%
\pgfsetfillopacity{0.700000}%
\pgfsetlinewidth{0.000000pt}%
\definecolor{currentstroke}{rgb}{0.000000,0.000000,0.000000}%
\pgfsetstrokecolor{currentstroke}%
\pgfsetstrokeopacity{0.700000}%
\pgfsetdash{}{0pt}%
\pgfpathmoveto{\pgfqpoint{2.730646in}{0.550000in}}%
\pgfpathlineto{\pgfqpoint{2.758168in}{0.550000in}}%
\pgfpathlineto{\pgfqpoint{2.758168in}{1.292734in}}%
\pgfpathlineto{\pgfqpoint{2.730646in}{1.292734in}}%
\pgfpathlineto{\pgfqpoint{2.730646in}{0.550000in}}%
\pgfpathclose%
\pgfusepath{fill}%
\end{pgfscope}%
\begin{pgfscope}%
\pgfpathrectangle{\pgfqpoint{1.250000in}{0.550000in}}{\pgfqpoint{7.750000in}{3.850000in}}%
\pgfusepath{clip}%
\pgfsetbuttcap%
\pgfsetmiterjoin%
\definecolor{currentfill}{rgb}{0.501961,0.501961,0.501961}%
\pgfsetfillcolor{currentfill}%
\pgfsetfillopacity{0.700000}%
\pgfsetlinewidth{0.000000pt}%
\definecolor{currentstroke}{rgb}{0.000000,0.000000,0.000000}%
\pgfsetstrokecolor{currentstroke}%
\pgfsetstrokeopacity{0.700000}%
\pgfsetdash{}{0pt}%
\pgfpathmoveto{\pgfqpoint{2.758168in}{0.550000in}}%
\pgfpathlineto{\pgfqpoint{2.785689in}{0.550000in}}%
\pgfpathlineto{\pgfqpoint{2.785689in}{1.114974in}}%
\pgfpathlineto{\pgfqpoint{2.758168in}{1.114974in}}%
\pgfpathlineto{\pgfqpoint{2.758168in}{0.550000in}}%
\pgfpathclose%
\pgfusepath{fill}%
\end{pgfscope}%
\begin{pgfscope}%
\pgfpathrectangle{\pgfqpoint{1.250000in}{0.550000in}}{\pgfqpoint{7.750000in}{3.850000in}}%
\pgfusepath{clip}%
\pgfsetbuttcap%
\pgfsetmiterjoin%
\definecolor{currentfill}{rgb}{0.501961,0.501961,0.501961}%
\pgfsetfillcolor{currentfill}%
\pgfsetfillopacity{0.700000}%
\pgfsetlinewidth{0.000000pt}%
\definecolor{currentstroke}{rgb}{0.000000,0.000000,0.000000}%
\pgfsetstrokecolor{currentstroke}%
\pgfsetstrokeopacity{0.700000}%
\pgfsetdash{}{0pt}%
\pgfpathmoveto{\pgfqpoint{2.785689in}{0.550000in}}%
\pgfpathlineto{\pgfqpoint{2.813210in}{0.550000in}}%
\pgfpathlineto{\pgfqpoint{2.813210in}{1.037194in}}%
\pgfpathlineto{\pgfqpoint{2.785689in}{1.037194in}}%
\pgfpathlineto{\pgfqpoint{2.785689in}{0.550000in}}%
\pgfpathclose%
\pgfusepath{fill}%
\end{pgfscope}%
\begin{pgfscope}%
\pgfpathrectangle{\pgfqpoint{1.250000in}{0.550000in}}{\pgfqpoint{7.750000in}{3.850000in}}%
\pgfusepath{clip}%
\pgfsetbuttcap%
\pgfsetmiterjoin%
\definecolor{currentfill}{rgb}{0.501961,0.501961,0.501961}%
\pgfsetfillcolor{currentfill}%
\pgfsetfillopacity{0.700000}%
\pgfsetlinewidth{0.000000pt}%
\definecolor{currentstroke}{rgb}{0.000000,0.000000,0.000000}%
\pgfsetstrokecolor{currentstroke}%
\pgfsetstrokeopacity{0.700000}%
\pgfsetdash{}{0pt}%
\pgfpathmoveto{\pgfqpoint{2.813210in}{0.550000in}}%
\pgfpathlineto{\pgfqpoint{2.840732in}{0.550000in}}%
\pgfpathlineto{\pgfqpoint{2.840732in}{1.109462in}}%
\pgfpathlineto{\pgfqpoint{2.813210in}{1.109462in}}%
\pgfpathlineto{\pgfqpoint{2.813210in}{0.550000in}}%
\pgfpathclose%
\pgfusepath{fill}%
\end{pgfscope}%
\begin{pgfscope}%
\pgfpathrectangle{\pgfqpoint{1.250000in}{0.550000in}}{\pgfqpoint{7.750000in}{3.850000in}}%
\pgfusepath{clip}%
\pgfsetbuttcap%
\pgfsetmiterjoin%
\definecolor{currentfill}{rgb}{0.501961,0.501961,0.501961}%
\pgfsetfillcolor{currentfill}%
\pgfsetfillopacity{0.700000}%
\pgfsetlinewidth{0.000000pt}%
\definecolor{currentstroke}{rgb}{0.000000,0.000000,0.000000}%
\pgfsetstrokecolor{currentstroke}%
\pgfsetstrokeopacity{0.700000}%
\pgfsetdash{}{0pt}%
\pgfpathmoveto{\pgfqpoint{2.840732in}{0.550000in}}%
\pgfpathlineto{\pgfqpoint{2.868253in}{0.550000in}}%
\pgfpathlineto{\pgfqpoint{2.868253in}{0.884085in}}%
\pgfpathlineto{\pgfqpoint{2.840732in}{0.884085in}}%
\pgfpathlineto{\pgfqpoint{2.840732in}{0.550000in}}%
\pgfpathclose%
\pgfusepath{fill}%
\end{pgfscope}%
\begin{pgfscope}%
\pgfpathrectangle{\pgfqpoint{1.250000in}{0.550000in}}{\pgfqpoint{7.750000in}{3.850000in}}%
\pgfusepath{clip}%
\pgfsetbuttcap%
\pgfsetmiterjoin%
\definecolor{currentfill}{rgb}{0.501961,0.501961,0.501961}%
\pgfsetfillcolor{currentfill}%
\pgfsetfillopacity{0.700000}%
\pgfsetlinewidth{0.000000pt}%
\definecolor{currentstroke}{rgb}{0.000000,0.000000,0.000000}%
\pgfsetstrokecolor{currentstroke}%
\pgfsetstrokeopacity{0.700000}%
\pgfsetdash{}{0pt}%
\pgfpathmoveto{\pgfqpoint{2.868253in}{0.550000in}}%
\pgfpathlineto{\pgfqpoint{2.895774in}{0.550000in}}%
\pgfpathlineto{\pgfqpoint{2.895774in}{0.881023in}}%
\pgfpathlineto{\pgfqpoint{2.868253in}{0.881023in}}%
\pgfpathlineto{\pgfqpoint{2.868253in}{0.550000in}}%
\pgfpathclose%
\pgfusepath{fill}%
\end{pgfscope}%
\begin{pgfscope}%
\pgfpathrectangle{\pgfqpoint{1.250000in}{0.550000in}}{\pgfqpoint{7.750000in}{3.850000in}}%
\pgfusepath{clip}%
\pgfsetbuttcap%
\pgfsetmiterjoin%
\definecolor{currentfill}{rgb}{0.501961,0.501961,0.501961}%
\pgfsetfillcolor{currentfill}%
\pgfsetfillopacity{0.700000}%
\pgfsetlinewidth{0.000000pt}%
\definecolor{currentstroke}{rgb}{0.000000,0.000000,0.000000}%
\pgfsetstrokecolor{currentstroke}%
\pgfsetstrokeopacity{0.700000}%
\pgfsetdash{}{0pt}%
\pgfpathmoveto{\pgfqpoint{2.895774in}{0.550000in}}%
\pgfpathlineto{\pgfqpoint{2.923295in}{0.550000in}}%
\pgfpathlineto{\pgfqpoint{2.923295in}{0.901846in}}%
\pgfpathlineto{\pgfqpoint{2.895774in}{0.901846in}}%
\pgfpathlineto{\pgfqpoint{2.895774in}{0.550000in}}%
\pgfpathclose%
\pgfusepath{fill}%
\end{pgfscope}%
\begin{pgfscope}%
\pgfpathrectangle{\pgfqpoint{1.250000in}{0.550000in}}{\pgfqpoint{7.750000in}{3.850000in}}%
\pgfusepath{clip}%
\pgfsetbuttcap%
\pgfsetmiterjoin%
\definecolor{currentfill}{rgb}{0.501961,0.501961,0.501961}%
\pgfsetfillcolor{currentfill}%
\pgfsetfillopacity{0.700000}%
\pgfsetlinewidth{0.000000pt}%
\definecolor{currentstroke}{rgb}{0.000000,0.000000,0.000000}%
\pgfsetstrokecolor{currentstroke}%
\pgfsetstrokeopacity{0.700000}%
\pgfsetdash{}{0pt}%
\pgfpathmoveto{\pgfqpoint{2.923295in}{0.550000in}}%
\pgfpathlineto{\pgfqpoint{2.950817in}{0.550000in}}%
\pgfpathlineto{\pgfqpoint{2.950817in}{0.780430in}}%
\pgfpathlineto{\pgfqpoint{2.923295in}{0.780430in}}%
\pgfpathlineto{\pgfqpoint{2.923295in}{0.550000in}}%
\pgfpathclose%
\pgfusepath{fill}%
\end{pgfscope}%
\begin{pgfscope}%
\pgfpathrectangle{\pgfqpoint{1.250000in}{0.550000in}}{\pgfqpoint{7.750000in}{3.850000in}}%
\pgfusepath{clip}%
\pgfsetbuttcap%
\pgfsetmiterjoin%
\definecolor{currentfill}{rgb}{0.501961,0.501961,0.501961}%
\pgfsetfillcolor{currentfill}%
\pgfsetfillopacity{0.700000}%
\pgfsetlinewidth{0.000000pt}%
\definecolor{currentstroke}{rgb}{0.000000,0.000000,0.000000}%
\pgfsetstrokecolor{currentstroke}%
\pgfsetstrokeopacity{0.700000}%
\pgfsetdash{}{0pt}%
\pgfpathmoveto{\pgfqpoint{2.950817in}{0.550000in}}%
\pgfpathlineto{\pgfqpoint{2.978338in}{0.550000in}}%
\pgfpathlineto{\pgfqpoint{2.978338in}{0.798803in}}%
\pgfpathlineto{\pgfqpoint{2.950817in}{0.798803in}}%
\pgfpathlineto{\pgfqpoint{2.950817in}{0.550000in}}%
\pgfpathclose%
\pgfusepath{fill}%
\end{pgfscope}%
\begin{pgfscope}%
\pgfpathrectangle{\pgfqpoint{1.250000in}{0.550000in}}{\pgfqpoint{7.750000in}{3.850000in}}%
\pgfusepath{clip}%
\pgfsetbuttcap%
\pgfsetmiterjoin%
\definecolor{currentfill}{rgb}{0.501961,0.501961,0.501961}%
\pgfsetfillcolor{currentfill}%
\pgfsetfillopacity{0.700000}%
\pgfsetlinewidth{0.000000pt}%
\definecolor{currentstroke}{rgb}{0.000000,0.000000,0.000000}%
\pgfsetstrokecolor{currentstroke}%
\pgfsetstrokeopacity{0.700000}%
\pgfsetdash{}{0pt}%
\pgfpathmoveto{\pgfqpoint{2.978338in}{0.550000in}}%
\pgfpathlineto{\pgfqpoint{3.005859in}{0.550000in}}%
\pgfpathlineto{\pgfqpoint{3.005859in}{0.760219in}}%
\pgfpathlineto{\pgfqpoint{2.978338in}{0.760219in}}%
\pgfpathlineto{\pgfqpoint{2.978338in}{0.550000in}}%
\pgfpathclose%
\pgfusepath{fill}%
\end{pgfscope}%
\begin{pgfscope}%
\pgfpathrectangle{\pgfqpoint{1.250000in}{0.550000in}}{\pgfqpoint{7.750000in}{3.850000in}}%
\pgfusepath{clip}%
\pgfsetbuttcap%
\pgfsetmiterjoin%
\definecolor{currentfill}{rgb}{0.501961,0.501961,0.501961}%
\pgfsetfillcolor{currentfill}%
\pgfsetfillopacity{0.700000}%
\pgfsetlinewidth{0.000000pt}%
\definecolor{currentstroke}{rgb}{0.000000,0.000000,0.000000}%
\pgfsetstrokecolor{currentstroke}%
\pgfsetstrokeopacity{0.700000}%
\pgfsetdash{}{0pt}%
\pgfpathmoveto{\pgfqpoint{3.005859in}{0.550000in}}%
\pgfpathlineto{\pgfqpoint{3.033381in}{0.550000in}}%
\pgfpathlineto{\pgfqpoint{3.033381in}{0.721330in}}%
\pgfpathlineto{\pgfqpoint{3.005859in}{0.721330in}}%
\pgfpathlineto{\pgfqpoint{3.005859in}{0.550000in}}%
\pgfpathclose%
\pgfusepath{fill}%
\end{pgfscope}%
\begin{pgfscope}%
\pgfpathrectangle{\pgfqpoint{1.250000in}{0.550000in}}{\pgfqpoint{7.750000in}{3.850000in}}%
\pgfusepath{clip}%
\pgfsetbuttcap%
\pgfsetmiterjoin%
\definecolor{currentfill}{rgb}{0.501961,0.501961,0.501961}%
\pgfsetfillcolor{currentfill}%
\pgfsetfillopacity{0.700000}%
\pgfsetlinewidth{0.000000pt}%
\definecolor{currentstroke}{rgb}{0.000000,0.000000,0.000000}%
\pgfsetstrokecolor{currentstroke}%
\pgfsetstrokeopacity{0.700000}%
\pgfsetdash{}{0pt}%
\pgfpathmoveto{\pgfqpoint{3.033381in}{0.550000in}}%
\pgfpathlineto{\pgfqpoint{3.060902in}{0.550000in}}%
\pgfpathlineto{\pgfqpoint{3.060902in}{0.761291in}}%
\pgfpathlineto{\pgfqpoint{3.033381in}{0.761291in}}%
\pgfpathlineto{\pgfqpoint{3.033381in}{0.550000in}}%
\pgfpathclose%
\pgfusepath{fill}%
\end{pgfscope}%
\begin{pgfscope}%
\pgfpathrectangle{\pgfqpoint{1.250000in}{0.550000in}}{\pgfqpoint{7.750000in}{3.850000in}}%
\pgfusepath{clip}%
\pgfsetbuttcap%
\pgfsetmiterjoin%
\definecolor{currentfill}{rgb}{0.501961,0.501961,0.501961}%
\pgfsetfillcolor{currentfill}%
\pgfsetfillopacity{0.700000}%
\pgfsetlinewidth{0.000000pt}%
\definecolor{currentstroke}{rgb}{0.000000,0.000000,0.000000}%
\pgfsetstrokecolor{currentstroke}%
\pgfsetstrokeopacity{0.700000}%
\pgfsetdash{}{0pt}%
\pgfpathmoveto{\pgfqpoint{3.060902in}{0.550000in}}%
\pgfpathlineto{\pgfqpoint{3.088423in}{0.550000in}}%
\pgfpathlineto{\pgfqpoint{3.088423in}{0.709693in}}%
\pgfpathlineto{\pgfqpoint{3.060902in}{0.709693in}}%
\pgfpathlineto{\pgfqpoint{3.060902in}{0.550000in}}%
\pgfpathclose%
\pgfusepath{fill}%
\end{pgfscope}%
\begin{pgfscope}%
\pgfpathrectangle{\pgfqpoint{1.250000in}{0.550000in}}{\pgfqpoint{7.750000in}{3.850000in}}%
\pgfusepath{clip}%
\pgfsetbuttcap%
\pgfsetmiterjoin%
\definecolor{currentfill}{rgb}{0.501961,0.501961,0.501961}%
\pgfsetfillcolor{currentfill}%
\pgfsetfillopacity{0.700000}%
\pgfsetlinewidth{0.000000pt}%
\definecolor{currentstroke}{rgb}{0.000000,0.000000,0.000000}%
\pgfsetstrokecolor{currentstroke}%
\pgfsetstrokeopacity{0.700000}%
\pgfsetdash{}{0pt}%
\pgfpathmoveto{\pgfqpoint{3.088423in}{0.550000in}}%
\pgfpathlineto{\pgfqpoint{3.115945in}{0.550000in}}%
\pgfpathlineto{\pgfqpoint{3.115945in}{0.738631in}}%
\pgfpathlineto{\pgfqpoint{3.088423in}{0.738631in}}%
\pgfpathlineto{\pgfqpoint{3.088423in}{0.550000in}}%
\pgfpathclose%
\pgfusepath{fill}%
\end{pgfscope}%
\begin{pgfscope}%
\pgfpathrectangle{\pgfqpoint{1.250000in}{0.550000in}}{\pgfqpoint{7.750000in}{3.850000in}}%
\pgfusepath{clip}%
\pgfsetbuttcap%
\pgfsetmiterjoin%
\definecolor{currentfill}{rgb}{0.501961,0.501961,0.501961}%
\pgfsetfillcolor{currentfill}%
\pgfsetfillopacity{0.700000}%
\pgfsetlinewidth{0.000000pt}%
\definecolor{currentstroke}{rgb}{0.000000,0.000000,0.000000}%
\pgfsetstrokecolor{currentstroke}%
\pgfsetstrokeopacity{0.700000}%
\pgfsetdash{}{0pt}%
\pgfpathmoveto{\pgfqpoint{3.115945in}{0.550000in}}%
\pgfpathlineto{\pgfqpoint{3.143466in}{0.550000in}}%
\pgfpathlineto{\pgfqpoint{3.143466in}{0.707090in}}%
\pgfpathlineto{\pgfqpoint{3.115945in}{0.707090in}}%
\pgfpathlineto{\pgfqpoint{3.115945in}{0.550000in}}%
\pgfpathclose%
\pgfusepath{fill}%
\end{pgfscope}%
\begin{pgfscope}%
\pgfpathrectangle{\pgfqpoint{1.250000in}{0.550000in}}{\pgfqpoint{7.750000in}{3.850000in}}%
\pgfusepath{clip}%
\pgfsetbuttcap%
\pgfsetmiterjoin%
\definecolor{currentfill}{rgb}{0.501961,0.501961,0.501961}%
\pgfsetfillcolor{currentfill}%
\pgfsetfillopacity{0.700000}%
\pgfsetlinewidth{0.000000pt}%
\definecolor{currentstroke}{rgb}{0.000000,0.000000,0.000000}%
\pgfsetstrokecolor{currentstroke}%
\pgfsetstrokeopacity{0.700000}%
\pgfsetdash{}{0pt}%
\pgfpathmoveto{\pgfqpoint{3.143466in}{0.550000in}}%
\pgfpathlineto{\pgfqpoint{3.170987in}{0.550000in}}%
\pgfpathlineto{\pgfqpoint{3.170987in}{0.702650in}}%
\pgfpathlineto{\pgfqpoint{3.143466in}{0.702650in}}%
\pgfpathlineto{\pgfqpoint{3.143466in}{0.550000in}}%
\pgfpathclose%
\pgfusepath{fill}%
\end{pgfscope}%
\begin{pgfscope}%
\pgfpathrectangle{\pgfqpoint{1.250000in}{0.550000in}}{\pgfqpoint{7.750000in}{3.850000in}}%
\pgfusepath{clip}%
\pgfsetbuttcap%
\pgfsetmiterjoin%
\definecolor{currentfill}{rgb}{0.501961,0.501961,0.501961}%
\pgfsetfillcolor{currentfill}%
\pgfsetfillopacity{0.700000}%
\pgfsetlinewidth{0.000000pt}%
\definecolor{currentstroke}{rgb}{0.000000,0.000000,0.000000}%
\pgfsetstrokecolor{currentstroke}%
\pgfsetstrokeopacity{0.700000}%
\pgfsetdash{}{0pt}%
\pgfpathmoveto{\pgfqpoint{3.170987in}{0.550000in}}%
\pgfpathlineto{\pgfqpoint{3.198509in}{0.550000in}}%
\pgfpathlineto{\pgfqpoint{3.198509in}{0.733731in}}%
\pgfpathlineto{\pgfqpoint{3.170987in}{0.733731in}}%
\pgfpathlineto{\pgfqpoint{3.170987in}{0.550000in}}%
\pgfpathclose%
\pgfusepath{fill}%
\end{pgfscope}%
\begin{pgfscope}%
\pgfpathrectangle{\pgfqpoint{1.250000in}{0.550000in}}{\pgfqpoint{7.750000in}{3.850000in}}%
\pgfusepath{clip}%
\pgfsetbuttcap%
\pgfsetmiterjoin%
\definecolor{currentfill}{rgb}{0.501961,0.501961,0.501961}%
\pgfsetfillcolor{currentfill}%
\pgfsetfillopacity{0.700000}%
\pgfsetlinewidth{0.000000pt}%
\definecolor{currentstroke}{rgb}{0.000000,0.000000,0.000000}%
\pgfsetstrokecolor{currentstroke}%
\pgfsetstrokeopacity{0.700000}%
\pgfsetdash{}{0pt}%
\pgfpathmoveto{\pgfqpoint{3.198509in}{0.550000in}}%
\pgfpathlineto{\pgfqpoint{3.226030in}{0.550000in}}%
\pgfpathlineto{\pgfqpoint{3.226030in}{0.702191in}}%
\pgfpathlineto{\pgfqpoint{3.198509in}{0.702191in}}%
\pgfpathlineto{\pgfqpoint{3.198509in}{0.550000in}}%
\pgfpathclose%
\pgfusepath{fill}%
\end{pgfscope}%
\begin{pgfscope}%
\pgfpathrectangle{\pgfqpoint{1.250000in}{0.550000in}}{\pgfqpoint{7.750000in}{3.850000in}}%
\pgfusepath{clip}%
\pgfsetbuttcap%
\pgfsetmiterjoin%
\definecolor{currentfill}{rgb}{0.501961,0.501961,0.501961}%
\pgfsetfillcolor{currentfill}%
\pgfsetfillopacity{0.700000}%
\pgfsetlinewidth{0.000000pt}%
\definecolor{currentstroke}{rgb}{0.000000,0.000000,0.000000}%
\pgfsetstrokecolor{currentstroke}%
\pgfsetstrokeopacity{0.700000}%
\pgfsetdash{}{0pt}%
\pgfpathmoveto{\pgfqpoint{3.226030in}{0.550000in}}%
\pgfpathlineto{\pgfqpoint{3.253551in}{0.550000in}}%
\pgfpathlineto{\pgfqpoint{3.253551in}{0.725923in}}%
\pgfpathlineto{\pgfqpoint{3.226030in}{0.725923in}}%
\pgfpathlineto{\pgfqpoint{3.226030in}{0.550000in}}%
\pgfpathclose%
\pgfusepath{fill}%
\end{pgfscope}%
\begin{pgfscope}%
\pgfpathrectangle{\pgfqpoint{1.250000in}{0.550000in}}{\pgfqpoint{7.750000in}{3.850000in}}%
\pgfusepath{clip}%
\pgfsetbuttcap%
\pgfsetmiterjoin%
\definecolor{currentfill}{rgb}{0.501961,0.501961,0.501961}%
\pgfsetfillcolor{currentfill}%
\pgfsetfillopacity{0.700000}%
\pgfsetlinewidth{0.000000pt}%
\definecolor{currentstroke}{rgb}{0.000000,0.000000,0.000000}%
\pgfsetstrokecolor{currentstroke}%
\pgfsetstrokeopacity{0.700000}%
\pgfsetdash{}{0pt}%
\pgfpathmoveto{\pgfqpoint{3.253551in}{0.550000in}}%
\pgfpathlineto{\pgfqpoint{3.281072in}{0.550000in}}%
\pgfpathlineto{\pgfqpoint{3.281072in}{0.718727in}}%
\pgfpathlineto{\pgfqpoint{3.253551in}{0.718727in}}%
\pgfpathlineto{\pgfqpoint{3.253551in}{0.550000in}}%
\pgfpathclose%
\pgfusepath{fill}%
\end{pgfscope}%
\begin{pgfscope}%
\pgfpathrectangle{\pgfqpoint{1.250000in}{0.550000in}}{\pgfqpoint{7.750000in}{3.850000in}}%
\pgfusepath{clip}%
\pgfsetbuttcap%
\pgfsetmiterjoin%
\definecolor{currentfill}{rgb}{0.501961,0.501961,0.501961}%
\pgfsetfillcolor{currentfill}%
\pgfsetfillopacity{0.700000}%
\pgfsetlinewidth{0.000000pt}%
\definecolor{currentstroke}{rgb}{0.000000,0.000000,0.000000}%
\pgfsetstrokecolor{currentstroke}%
\pgfsetstrokeopacity{0.700000}%
\pgfsetdash{}{0pt}%
\pgfpathmoveto{\pgfqpoint{3.281072in}{0.550000in}}%
\pgfpathlineto{\pgfqpoint{3.308594in}{0.550000in}}%
\pgfpathlineto{\pgfqpoint{3.308594in}{0.711531in}}%
\pgfpathlineto{\pgfqpoint{3.281072in}{0.711531in}}%
\pgfpathlineto{\pgfqpoint{3.281072in}{0.550000in}}%
\pgfpathclose%
\pgfusepath{fill}%
\end{pgfscope}%
\begin{pgfscope}%
\pgfpathrectangle{\pgfqpoint{1.250000in}{0.550000in}}{\pgfqpoint{7.750000in}{3.850000in}}%
\pgfusepath{clip}%
\pgfsetbuttcap%
\pgfsetmiterjoin%
\definecolor{currentfill}{rgb}{0.501961,0.501961,0.501961}%
\pgfsetfillcolor{currentfill}%
\pgfsetfillopacity{0.700000}%
\pgfsetlinewidth{0.000000pt}%
\definecolor{currentstroke}{rgb}{0.000000,0.000000,0.000000}%
\pgfsetstrokecolor{currentstroke}%
\pgfsetstrokeopacity{0.700000}%
\pgfsetdash{}{0pt}%
\pgfpathmoveto{\pgfqpoint{3.308594in}{0.550000in}}%
\pgfpathlineto{\pgfqpoint{3.336115in}{0.550000in}}%
\pgfpathlineto{\pgfqpoint{3.336115in}{0.681674in}}%
\pgfpathlineto{\pgfqpoint{3.308594in}{0.681674in}}%
\pgfpathlineto{\pgfqpoint{3.308594in}{0.550000in}}%
\pgfpathclose%
\pgfusepath{fill}%
\end{pgfscope}%
\begin{pgfscope}%
\pgfpathrectangle{\pgfqpoint{1.250000in}{0.550000in}}{\pgfqpoint{7.750000in}{3.850000in}}%
\pgfusepath{clip}%
\pgfsetbuttcap%
\pgfsetmiterjoin%
\definecolor{currentfill}{rgb}{0.501961,0.501961,0.501961}%
\pgfsetfillcolor{currentfill}%
\pgfsetfillopacity{0.700000}%
\pgfsetlinewidth{0.000000pt}%
\definecolor{currentstroke}{rgb}{0.000000,0.000000,0.000000}%
\pgfsetstrokecolor{currentstroke}%
\pgfsetstrokeopacity{0.700000}%
\pgfsetdash{}{0pt}%
\pgfpathmoveto{\pgfqpoint{3.336115in}{0.550000in}}%
\pgfpathlineto{\pgfqpoint{3.363636in}{0.550000in}}%
\pgfpathlineto{\pgfqpoint{3.363636in}{0.704641in}}%
\pgfpathlineto{\pgfqpoint{3.336115in}{0.704641in}}%
\pgfpathlineto{\pgfqpoint{3.336115in}{0.550000in}}%
\pgfpathclose%
\pgfusepath{fill}%
\end{pgfscope}%
\begin{pgfscope}%
\pgfpathrectangle{\pgfqpoint{1.250000in}{0.550000in}}{\pgfqpoint{7.750000in}{3.850000in}}%
\pgfusepath{clip}%
\pgfsetbuttcap%
\pgfsetmiterjoin%
\definecolor{currentfill}{rgb}{0.501961,0.501961,0.501961}%
\pgfsetfillcolor{currentfill}%
\pgfsetfillopacity{0.700000}%
\pgfsetlinewidth{0.000000pt}%
\definecolor{currentstroke}{rgb}{0.000000,0.000000,0.000000}%
\pgfsetstrokecolor{currentstroke}%
\pgfsetstrokeopacity{0.700000}%
\pgfsetdash{}{0pt}%
\pgfpathmoveto{\pgfqpoint{3.363636in}{0.550000in}}%
\pgfpathlineto{\pgfqpoint{3.391158in}{0.550000in}}%
\pgfpathlineto{\pgfqpoint{3.391158in}{0.689483in}}%
\pgfpathlineto{\pgfqpoint{3.363636in}{0.689483in}}%
\pgfpathlineto{\pgfqpoint{3.363636in}{0.550000in}}%
\pgfpathclose%
\pgfusepath{fill}%
\end{pgfscope}%
\begin{pgfscope}%
\pgfpathrectangle{\pgfqpoint{1.250000in}{0.550000in}}{\pgfqpoint{7.750000in}{3.850000in}}%
\pgfusepath{clip}%
\pgfsetbuttcap%
\pgfsetmiterjoin%
\definecolor{currentfill}{rgb}{0.501961,0.501961,0.501961}%
\pgfsetfillcolor{currentfill}%
\pgfsetfillopacity{0.700000}%
\pgfsetlinewidth{0.000000pt}%
\definecolor{currentstroke}{rgb}{0.000000,0.000000,0.000000}%
\pgfsetstrokecolor{currentstroke}%
\pgfsetstrokeopacity{0.700000}%
\pgfsetdash{}{0pt}%
\pgfpathmoveto{\pgfqpoint{3.391158in}{0.550000in}}%
\pgfpathlineto{\pgfqpoint{3.418679in}{0.550000in}}%
\pgfpathlineto{\pgfqpoint{3.418679in}{0.674937in}}%
\pgfpathlineto{\pgfqpoint{3.391158in}{0.674937in}}%
\pgfpathlineto{\pgfqpoint{3.391158in}{0.550000in}}%
\pgfpathclose%
\pgfusepath{fill}%
\end{pgfscope}%
\begin{pgfscope}%
\pgfpathrectangle{\pgfqpoint{1.250000in}{0.550000in}}{\pgfqpoint{7.750000in}{3.850000in}}%
\pgfusepath{clip}%
\pgfsetbuttcap%
\pgfsetmiterjoin%
\definecolor{currentfill}{rgb}{0.501961,0.501961,0.501961}%
\pgfsetfillcolor{currentfill}%
\pgfsetfillopacity{0.700000}%
\pgfsetlinewidth{0.000000pt}%
\definecolor{currentstroke}{rgb}{0.000000,0.000000,0.000000}%
\pgfsetstrokecolor{currentstroke}%
\pgfsetstrokeopacity{0.700000}%
\pgfsetdash{}{0pt}%
\pgfpathmoveto{\pgfqpoint{3.418679in}{0.550000in}}%
\pgfpathlineto{\pgfqpoint{3.446200in}{0.550000in}}%
\pgfpathlineto{\pgfqpoint{3.446200in}{0.669272in}}%
\pgfpathlineto{\pgfqpoint{3.418679in}{0.669272in}}%
\pgfpathlineto{\pgfqpoint{3.418679in}{0.550000in}}%
\pgfpathclose%
\pgfusepath{fill}%
\end{pgfscope}%
\begin{pgfscope}%
\pgfpathrectangle{\pgfqpoint{1.250000in}{0.550000in}}{\pgfqpoint{7.750000in}{3.850000in}}%
\pgfusepath{clip}%
\pgfsetbuttcap%
\pgfsetmiterjoin%
\definecolor{currentfill}{rgb}{0.501961,0.501961,0.501961}%
\pgfsetfillcolor{currentfill}%
\pgfsetfillopacity{0.700000}%
\pgfsetlinewidth{0.000000pt}%
\definecolor{currentstroke}{rgb}{0.000000,0.000000,0.000000}%
\pgfsetstrokecolor{currentstroke}%
\pgfsetstrokeopacity{0.700000}%
\pgfsetdash{}{0pt}%
\pgfpathmoveto{\pgfqpoint{3.446200in}{0.550000in}}%
\pgfpathlineto{\pgfqpoint{3.473722in}{0.550000in}}%
\pgfpathlineto{\pgfqpoint{3.473722in}{0.671110in}}%
\pgfpathlineto{\pgfqpoint{3.446200in}{0.671110in}}%
\pgfpathlineto{\pgfqpoint{3.446200in}{0.550000in}}%
\pgfpathclose%
\pgfusepath{fill}%
\end{pgfscope}%
\begin{pgfscope}%
\pgfpathrectangle{\pgfqpoint{1.250000in}{0.550000in}}{\pgfqpoint{7.750000in}{3.850000in}}%
\pgfusepath{clip}%
\pgfsetbuttcap%
\pgfsetmiterjoin%
\definecolor{currentfill}{rgb}{0.501961,0.501961,0.501961}%
\pgfsetfillcolor{currentfill}%
\pgfsetfillopacity{0.700000}%
\pgfsetlinewidth{0.000000pt}%
\definecolor{currentstroke}{rgb}{0.000000,0.000000,0.000000}%
\pgfsetstrokecolor{currentstroke}%
\pgfsetstrokeopacity{0.700000}%
\pgfsetdash{}{0pt}%
\pgfpathmoveto{\pgfqpoint{3.473722in}{0.550000in}}%
\pgfpathlineto{\pgfqpoint{3.501243in}{0.550000in}}%
\pgfpathlineto{\pgfqpoint{3.501243in}{0.673100in}}%
\pgfpathlineto{\pgfqpoint{3.473722in}{0.673100in}}%
\pgfpathlineto{\pgfqpoint{3.473722in}{0.550000in}}%
\pgfpathclose%
\pgfusepath{fill}%
\end{pgfscope}%
\begin{pgfscope}%
\pgfpathrectangle{\pgfqpoint{1.250000in}{0.550000in}}{\pgfqpoint{7.750000in}{3.850000in}}%
\pgfusepath{clip}%
\pgfsetbuttcap%
\pgfsetmiterjoin%
\definecolor{currentfill}{rgb}{0.501961,0.501961,0.501961}%
\pgfsetfillcolor{currentfill}%
\pgfsetfillopacity{0.700000}%
\pgfsetlinewidth{0.000000pt}%
\definecolor{currentstroke}{rgb}{0.000000,0.000000,0.000000}%
\pgfsetstrokecolor{currentstroke}%
\pgfsetstrokeopacity{0.700000}%
\pgfsetdash{}{0pt}%
\pgfpathmoveto{\pgfqpoint{3.501243in}{0.550000in}}%
\pgfpathlineto{\pgfqpoint{3.528764in}{0.550000in}}%
\pgfpathlineto{\pgfqpoint{3.528764in}{0.683358in}}%
\pgfpathlineto{\pgfqpoint{3.501243in}{0.683358in}}%
\pgfpathlineto{\pgfqpoint{3.501243in}{0.550000in}}%
\pgfpathclose%
\pgfusepath{fill}%
\end{pgfscope}%
\begin{pgfscope}%
\pgfpathrectangle{\pgfqpoint{1.250000in}{0.550000in}}{\pgfqpoint{7.750000in}{3.850000in}}%
\pgfusepath{clip}%
\pgfsetbuttcap%
\pgfsetmiterjoin%
\definecolor{currentfill}{rgb}{0.501961,0.501961,0.501961}%
\pgfsetfillcolor{currentfill}%
\pgfsetfillopacity{0.700000}%
\pgfsetlinewidth{0.000000pt}%
\definecolor{currentstroke}{rgb}{0.000000,0.000000,0.000000}%
\pgfsetstrokecolor{currentstroke}%
\pgfsetstrokeopacity{0.700000}%
\pgfsetdash{}{0pt}%
\pgfpathmoveto{\pgfqpoint{3.528764in}{0.550000in}}%
\pgfpathlineto{\pgfqpoint{3.556286in}{0.550000in}}%
\pgfpathlineto{\pgfqpoint{3.556286in}{0.668201in}}%
\pgfpathlineto{\pgfqpoint{3.528764in}{0.668201in}}%
\pgfpathlineto{\pgfqpoint{3.528764in}{0.550000in}}%
\pgfpathclose%
\pgfusepath{fill}%
\end{pgfscope}%
\begin{pgfscope}%
\pgfpathrectangle{\pgfqpoint{1.250000in}{0.550000in}}{\pgfqpoint{7.750000in}{3.850000in}}%
\pgfusepath{clip}%
\pgfsetbuttcap%
\pgfsetmiterjoin%
\definecolor{currentfill}{rgb}{0.501961,0.501961,0.501961}%
\pgfsetfillcolor{currentfill}%
\pgfsetfillopacity{0.700000}%
\pgfsetlinewidth{0.000000pt}%
\definecolor{currentstroke}{rgb}{0.000000,0.000000,0.000000}%
\pgfsetstrokecolor{currentstroke}%
\pgfsetstrokeopacity{0.700000}%
\pgfsetdash{}{0pt}%
\pgfpathmoveto{\pgfqpoint{3.556286in}{0.550000in}}%
\pgfpathlineto{\pgfqpoint{3.583807in}{0.550000in}}%
\pgfpathlineto{\pgfqpoint{3.583807in}{0.653655in}}%
\pgfpathlineto{\pgfqpoint{3.556286in}{0.653655in}}%
\pgfpathlineto{\pgfqpoint{3.556286in}{0.550000in}}%
\pgfpathclose%
\pgfusepath{fill}%
\end{pgfscope}%
\begin{pgfscope}%
\pgfpathrectangle{\pgfqpoint{1.250000in}{0.550000in}}{\pgfqpoint{7.750000in}{3.850000in}}%
\pgfusepath{clip}%
\pgfsetbuttcap%
\pgfsetmiterjoin%
\definecolor{currentfill}{rgb}{0.501961,0.501961,0.501961}%
\pgfsetfillcolor{currentfill}%
\pgfsetfillopacity{0.700000}%
\pgfsetlinewidth{0.000000pt}%
\definecolor{currentstroke}{rgb}{0.000000,0.000000,0.000000}%
\pgfsetstrokecolor{currentstroke}%
\pgfsetstrokeopacity{0.700000}%
\pgfsetdash{}{0pt}%
\pgfpathmoveto{\pgfqpoint{3.583807in}{0.550000in}}%
\pgfpathlineto{\pgfqpoint{3.611328in}{0.550000in}}%
\pgfpathlineto{\pgfqpoint{3.611328in}{0.664067in}}%
\pgfpathlineto{\pgfqpoint{3.583807in}{0.664067in}}%
\pgfpathlineto{\pgfqpoint{3.583807in}{0.550000in}}%
\pgfpathclose%
\pgfusepath{fill}%
\end{pgfscope}%
\begin{pgfscope}%
\pgfpathrectangle{\pgfqpoint{1.250000in}{0.550000in}}{\pgfqpoint{7.750000in}{3.850000in}}%
\pgfusepath{clip}%
\pgfsetbuttcap%
\pgfsetmiterjoin%
\definecolor{currentfill}{rgb}{0.501961,0.501961,0.501961}%
\pgfsetfillcolor{currentfill}%
\pgfsetfillopacity{0.700000}%
\pgfsetlinewidth{0.000000pt}%
\definecolor{currentstroke}{rgb}{0.000000,0.000000,0.000000}%
\pgfsetstrokecolor{currentstroke}%
\pgfsetstrokeopacity{0.700000}%
\pgfsetdash{}{0pt}%
\pgfpathmoveto{\pgfqpoint{3.611328in}{0.550000in}}%
\pgfpathlineto{\pgfqpoint{3.638849in}{0.550000in}}%
\pgfpathlineto{\pgfqpoint{3.638849in}{0.652124in}}%
\pgfpathlineto{\pgfqpoint{3.611328in}{0.652124in}}%
\pgfpathlineto{\pgfqpoint{3.611328in}{0.550000in}}%
\pgfpathclose%
\pgfusepath{fill}%
\end{pgfscope}%
\begin{pgfscope}%
\pgfpathrectangle{\pgfqpoint{1.250000in}{0.550000in}}{\pgfqpoint{7.750000in}{3.850000in}}%
\pgfusepath{clip}%
\pgfsetbuttcap%
\pgfsetmiterjoin%
\definecolor{currentfill}{rgb}{0.501961,0.501961,0.501961}%
\pgfsetfillcolor{currentfill}%
\pgfsetfillopacity{0.700000}%
\pgfsetlinewidth{0.000000pt}%
\definecolor{currentstroke}{rgb}{0.000000,0.000000,0.000000}%
\pgfsetstrokecolor{currentstroke}%
\pgfsetstrokeopacity{0.700000}%
\pgfsetdash{}{0pt}%
\pgfpathmoveto{\pgfqpoint{3.638849in}{0.550000in}}%
\pgfpathlineto{\pgfqpoint{3.666371in}{0.550000in}}%
\pgfpathlineto{\pgfqpoint{3.666371in}{0.657942in}}%
\pgfpathlineto{\pgfqpoint{3.638849in}{0.657942in}}%
\pgfpathlineto{\pgfqpoint{3.638849in}{0.550000in}}%
\pgfpathclose%
\pgfusepath{fill}%
\end{pgfscope}%
\begin{pgfscope}%
\pgfpathrectangle{\pgfqpoint{1.250000in}{0.550000in}}{\pgfqpoint{7.750000in}{3.850000in}}%
\pgfusepath{clip}%
\pgfsetbuttcap%
\pgfsetmiterjoin%
\definecolor{currentfill}{rgb}{0.501961,0.501961,0.501961}%
\pgfsetfillcolor{currentfill}%
\pgfsetfillopacity{0.700000}%
\pgfsetlinewidth{0.000000pt}%
\definecolor{currentstroke}{rgb}{0.000000,0.000000,0.000000}%
\pgfsetstrokecolor{currentstroke}%
\pgfsetstrokeopacity{0.700000}%
\pgfsetdash{}{0pt}%
\pgfpathmoveto{\pgfqpoint{3.666371in}{0.550000in}}%
\pgfpathlineto{\pgfqpoint{3.693892in}{0.550000in}}%
\pgfpathlineto{\pgfqpoint{3.693892in}{0.644009in}}%
\pgfpathlineto{\pgfqpoint{3.666371in}{0.644009in}}%
\pgfpathlineto{\pgfqpoint{3.666371in}{0.550000in}}%
\pgfpathclose%
\pgfusepath{fill}%
\end{pgfscope}%
\begin{pgfscope}%
\pgfpathrectangle{\pgfqpoint{1.250000in}{0.550000in}}{\pgfqpoint{7.750000in}{3.850000in}}%
\pgfusepath{clip}%
\pgfsetbuttcap%
\pgfsetmiterjoin%
\definecolor{currentfill}{rgb}{0.501961,0.501961,0.501961}%
\pgfsetfillcolor{currentfill}%
\pgfsetfillopacity{0.700000}%
\pgfsetlinewidth{0.000000pt}%
\definecolor{currentstroke}{rgb}{0.000000,0.000000,0.000000}%
\pgfsetstrokecolor{currentstroke}%
\pgfsetstrokeopacity{0.700000}%
\pgfsetdash{}{0pt}%
\pgfpathmoveto{\pgfqpoint{3.693892in}{0.550000in}}%
\pgfpathlineto{\pgfqpoint{3.721413in}{0.550000in}}%
\pgfpathlineto{\pgfqpoint{3.721413in}{0.650593in}}%
\pgfpathlineto{\pgfqpoint{3.693892in}{0.650593in}}%
\pgfpathlineto{\pgfqpoint{3.693892in}{0.550000in}}%
\pgfpathclose%
\pgfusepath{fill}%
\end{pgfscope}%
\begin{pgfscope}%
\pgfpathrectangle{\pgfqpoint{1.250000in}{0.550000in}}{\pgfqpoint{7.750000in}{3.850000in}}%
\pgfusepath{clip}%
\pgfsetbuttcap%
\pgfsetmiterjoin%
\definecolor{currentfill}{rgb}{0.501961,0.501961,0.501961}%
\pgfsetfillcolor{currentfill}%
\pgfsetfillopacity{0.700000}%
\pgfsetlinewidth{0.000000pt}%
\definecolor{currentstroke}{rgb}{0.000000,0.000000,0.000000}%
\pgfsetstrokecolor{currentstroke}%
\pgfsetstrokeopacity{0.700000}%
\pgfsetdash{}{0pt}%
\pgfpathmoveto{\pgfqpoint{3.721413in}{0.550000in}}%
\pgfpathlineto{\pgfqpoint{3.748935in}{0.550000in}}%
\pgfpathlineto{\pgfqpoint{3.748935in}{0.640641in}}%
\pgfpathlineto{\pgfqpoint{3.721413in}{0.640641in}}%
\pgfpathlineto{\pgfqpoint{3.721413in}{0.550000in}}%
\pgfpathclose%
\pgfusepath{fill}%
\end{pgfscope}%
\begin{pgfscope}%
\pgfpathrectangle{\pgfqpoint{1.250000in}{0.550000in}}{\pgfqpoint{7.750000in}{3.850000in}}%
\pgfusepath{clip}%
\pgfsetbuttcap%
\pgfsetmiterjoin%
\definecolor{currentfill}{rgb}{0.501961,0.501961,0.501961}%
\pgfsetfillcolor{currentfill}%
\pgfsetfillopacity{0.700000}%
\pgfsetlinewidth{0.000000pt}%
\definecolor{currentstroke}{rgb}{0.000000,0.000000,0.000000}%
\pgfsetstrokecolor{currentstroke}%
\pgfsetstrokeopacity{0.700000}%
\pgfsetdash{}{0pt}%
\pgfpathmoveto{\pgfqpoint{3.748935in}{0.550000in}}%
\pgfpathlineto{\pgfqpoint{3.776456in}{0.550000in}}%
\pgfpathlineto{\pgfqpoint{3.776456in}{0.645387in}}%
\pgfpathlineto{\pgfqpoint{3.748935in}{0.645387in}}%
\pgfpathlineto{\pgfqpoint{3.748935in}{0.550000in}}%
\pgfpathclose%
\pgfusepath{fill}%
\end{pgfscope}%
\begin{pgfscope}%
\pgfpathrectangle{\pgfqpoint{1.250000in}{0.550000in}}{\pgfqpoint{7.750000in}{3.850000in}}%
\pgfusepath{clip}%
\pgfsetbuttcap%
\pgfsetmiterjoin%
\definecolor{currentfill}{rgb}{0.501961,0.501961,0.501961}%
\pgfsetfillcolor{currentfill}%
\pgfsetfillopacity{0.700000}%
\pgfsetlinewidth{0.000000pt}%
\definecolor{currentstroke}{rgb}{0.000000,0.000000,0.000000}%
\pgfsetstrokecolor{currentstroke}%
\pgfsetstrokeopacity{0.700000}%
\pgfsetdash{}{0pt}%
\pgfpathmoveto{\pgfqpoint{3.776456in}{0.550000in}}%
\pgfpathlineto{\pgfqpoint{3.803977in}{0.550000in}}%
\pgfpathlineto{\pgfqpoint{3.803977in}{0.632373in}}%
\pgfpathlineto{\pgfqpoint{3.776456in}{0.632373in}}%
\pgfpathlineto{\pgfqpoint{3.776456in}{0.550000in}}%
\pgfpathclose%
\pgfusepath{fill}%
\end{pgfscope}%
\begin{pgfscope}%
\pgfpathrectangle{\pgfqpoint{1.250000in}{0.550000in}}{\pgfqpoint{7.750000in}{3.850000in}}%
\pgfusepath{clip}%
\pgfsetbuttcap%
\pgfsetmiterjoin%
\definecolor{currentfill}{rgb}{0.501961,0.501961,0.501961}%
\pgfsetfillcolor{currentfill}%
\pgfsetfillopacity{0.700000}%
\pgfsetlinewidth{0.000000pt}%
\definecolor{currentstroke}{rgb}{0.000000,0.000000,0.000000}%
\pgfsetstrokecolor{currentstroke}%
\pgfsetstrokeopacity{0.700000}%
\pgfsetdash{}{0pt}%
\pgfpathmoveto{\pgfqpoint{3.803977in}{0.550000in}}%
\pgfpathlineto{\pgfqpoint{3.831499in}{0.550000in}}%
\pgfpathlineto{\pgfqpoint{3.831499in}{0.642019in}}%
\pgfpathlineto{\pgfqpoint{3.803977in}{0.642019in}}%
\pgfpathlineto{\pgfqpoint{3.803977in}{0.550000in}}%
\pgfpathclose%
\pgfusepath{fill}%
\end{pgfscope}%
\begin{pgfscope}%
\pgfpathrectangle{\pgfqpoint{1.250000in}{0.550000in}}{\pgfqpoint{7.750000in}{3.850000in}}%
\pgfusepath{clip}%
\pgfsetbuttcap%
\pgfsetmiterjoin%
\definecolor{currentfill}{rgb}{0.501961,0.501961,0.501961}%
\pgfsetfillcolor{currentfill}%
\pgfsetfillopacity{0.700000}%
\pgfsetlinewidth{0.000000pt}%
\definecolor{currentstroke}{rgb}{0.000000,0.000000,0.000000}%
\pgfsetstrokecolor{currentstroke}%
\pgfsetstrokeopacity{0.700000}%
\pgfsetdash{}{0pt}%
\pgfpathmoveto{\pgfqpoint{3.831499in}{0.550000in}}%
\pgfpathlineto{\pgfqpoint{3.859020in}{0.550000in}}%
\pgfpathlineto{\pgfqpoint{3.859020in}{0.634976in}}%
\pgfpathlineto{\pgfqpoint{3.831499in}{0.634976in}}%
\pgfpathlineto{\pgfqpoint{3.831499in}{0.550000in}}%
\pgfpathclose%
\pgfusepath{fill}%
\end{pgfscope}%
\begin{pgfscope}%
\pgfpathrectangle{\pgfqpoint{1.250000in}{0.550000in}}{\pgfqpoint{7.750000in}{3.850000in}}%
\pgfusepath{clip}%
\pgfsetbuttcap%
\pgfsetmiterjoin%
\definecolor{currentfill}{rgb}{0.501961,0.501961,0.501961}%
\pgfsetfillcolor{currentfill}%
\pgfsetfillopacity{0.700000}%
\pgfsetlinewidth{0.000000pt}%
\definecolor{currentstroke}{rgb}{0.000000,0.000000,0.000000}%
\pgfsetstrokecolor{currentstroke}%
\pgfsetstrokeopacity{0.700000}%
\pgfsetdash{}{0pt}%
\pgfpathmoveto{\pgfqpoint{3.859020in}{0.550000in}}%
\pgfpathlineto{\pgfqpoint{3.886541in}{0.550000in}}%
\pgfpathlineto{\pgfqpoint{3.886541in}{0.641559in}}%
\pgfpathlineto{\pgfqpoint{3.859020in}{0.641559in}}%
\pgfpathlineto{\pgfqpoint{3.859020in}{0.550000in}}%
\pgfpathclose%
\pgfusepath{fill}%
\end{pgfscope}%
\begin{pgfscope}%
\pgfpathrectangle{\pgfqpoint{1.250000in}{0.550000in}}{\pgfqpoint{7.750000in}{3.850000in}}%
\pgfusepath{clip}%
\pgfsetbuttcap%
\pgfsetmiterjoin%
\definecolor{currentfill}{rgb}{0.501961,0.501961,0.501961}%
\pgfsetfillcolor{currentfill}%
\pgfsetfillopacity{0.700000}%
\pgfsetlinewidth{0.000000pt}%
\definecolor{currentstroke}{rgb}{0.000000,0.000000,0.000000}%
\pgfsetstrokecolor{currentstroke}%
\pgfsetstrokeopacity{0.700000}%
\pgfsetdash{}{0pt}%
\pgfpathmoveto{\pgfqpoint{3.886541in}{0.550000in}}%
\pgfpathlineto{\pgfqpoint{3.914062in}{0.550000in}}%
\pgfpathlineto{\pgfqpoint{3.914062in}{0.634363in}}%
\pgfpathlineto{\pgfqpoint{3.886541in}{0.634363in}}%
\pgfpathlineto{\pgfqpoint{3.886541in}{0.550000in}}%
\pgfpathclose%
\pgfusepath{fill}%
\end{pgfscope}%
\begin{pgfscope}%
\pgfpathrectangle{\pgfqpoint{1.250000in}{0.550000in}}{\pgfqpoint{7.750000in}{3.850000in}}%
\pgfusepath{clip}%
\pgfsetbuttcap%
\pgfsetmiterjoin%
\definecolor{currentfill}{rgb}{0.501961,0.501961,0.501961}%
\pgfsetfillcolor{currentfill}%
\pgfsetfillopacity{0.700000}%
\pgfsetlinewidth{0.000000pt}%
\definecolor{currentstroke}{rgb}{0.000000,0.000000,0.000000}%
\pgfsetstrokecolor{currentstroke}%
\pgfsetstrokeopacity{0.700000}%
\pgfsetdash{}{0pt}%
\pgfpathmoveto{\pgfqpoint{3.914062in}{0.550000in}}%
\pgfpathlineto{\pgfqpoint{3.941584in}{0.550000in}}%
\pgfpathlineto{\pgfqpoint{3.941584in}{0.632985in}}%
\pgfpathlineto{\pgfqpoint{3.914062in}{0.632985in}}%
\pgfpathlineto{\pgfqpoint{3.914062in}{0.550000in}}%
\pgfpathclose%
\pgfusepath{fill}%
\end{pgfscope}%
\begin{pgfscope}%
\pgfpathrectangle{\pgfqpoint{1.250000in}{0.550000in}}{\pgfqpoint{7.750000in}{3.850000in}}%
\pgfusepath{clip}%
\pgfsetbuttcap%
\pgfsetmiterjoin%
\definecolor{currentfill}{rgb}{0.501961,0.501961,0.501961}%
\pgfsetfillcolor{currentfill}%
\pgfsetfillopacity{0.700000}%
\pgfsetlinewidth{0.000000pt}%
\definecolor{currentstroke}{rgb}{0.000000,0.000000,0.000000}%
\pgfsetstrokecolor{currentstroke}%
\pgfsetstrokeopacity{0.700000}%
\pgfsetdash{}{0pt}%
\pgfpathmoveto{\pgfqpoint{3.941584in}{0.550000in}}%
\pgfpathlineto{\pgfqpoint{3.969105in}{0.550000in}}%
\pgfpathlineto{\pgfqpoint{3.969105in}{0.628545in}}%
\pgfpathlineto{\pgfqpoint{3.941584in}{0.628545in}}%
\pgfpathlineto{\pgfqpoint{3.941584in}{0.550000in}}%
\pgfpathclose%
\pgfusepath{fill}%
\end{pgfscope}%
\begin{pgfscope}%
\pgfpathrectangle{\pgfqpoint{1.250000in}{0.550000in}}{\pgfqpoint{7.750000in}{3.850000in}}%
\pgfusepath{clip}%
\pgfsetbuttcap%
\pgfsetmiterjoin%
\definecolor{currentfill}{rgb}{0.501961,0.501961,0.501961}%
\pgfsetfillcolor{currentfill}%
\pgfsetfillopacity{0.700000}%
\pgfsetlinewidth{0.000000pt}%
\definecolor{currentstroke}{rgb}{0.000000,0.000000,0.000000}%
\pgfsetstrokecolor{currentstroke}%
\pgfsetstrokeopacity{0.700000}%
\pgfsetdash{}{0pt}%
\pgfpathmoveto{\pgfqpoint{3.969105in}{0.550000in}}%
\pgfpathlineto{\pgfqpoint{3.996626in}{0.550000in}}%
\pgfpathlineto{\pgfqpoint{3.996626in}{0.637426in}}%
\pgfpathlineto{\pgfqpoint{3.969105in}{0.637426in}}%
\pgfpathlineto{\pgfqpoint{3.969105in}{0.550000in}}%
\pgfpathclose%
\pgfusepath{fill}%
\end{pgfscope}%
\begin{pgfscope}%
\pgfpathrectangle{\pgfqpoint{1.250000in}{0.550000in}}{\pgfqpoint{7.750000in}{3.850000in}}%
\pgfusepath{clip}%
\pgfsetbuttcap%
\pgfsetmiterjoin%
\definecolor{currentfill}{rgb}{0.501961,0.501961,0.501961}%
\pgfsetfillcolor{currentfill}%
\pgfsetfillopacity{0.700000}%
\pgfsetlinewidth{0.000000pt}%
\definecolor{currentstroke}{rgb}{0.000000,0.000000,0.000000}%
\pgfsetstrokecolor{currentstroke}%
\pgfsetstrokeopacity{0.700000}%
\pgfsetdash{}{0pt}%
\pgfpathmoveto{\pgfqpoint{3.996626in}{0.550000in}}%
\pgfpathlineto{\pgfqpoint{4.024148in}{0.550000in}}%
\pgfpathlineto{\pgfqpoint{4.024148in}{0.624717in}}%
\pgfpathlineto{\pgfqpoint{3.996626in}{0.624717in}}%
\pgfpathlineto{\pgfqpoint{3.996626in}{0.550000in}}%
\pgfpathclose%
\pgfusepath{fill}%
\end{pgfscope}%
\begin{pgfscope}%
\pgfpathrectangle{\pgfqpoint{1.250000in}{0.550000in}}{\pgfqpoint{7.750000in}{3.850000in}}%
\pgfusepath{clip}%
\pgfsetbuttcap%
\pgfsetmiterjoin%
\definecolor{currentfill}{rgb}{0.501961,0.501961,0.501961}%
\pgfsetfillcolor{currentfill}%
\pgfsetfillopacity{0.700000}%
\pgfsetlinewidth{0.000000pt}%
\definecolor{currentstroke}{rgb}{0.000000,0.000000,0.000000}%
\pgfsetstrokecolor{currentstroke}%
\pgfsetstrokeopacity{0.700000}%
\pgfsetdash{}{0pt}%
\pgfpathmoveto{\pgfqpoint{4.024148in}{0.550000in}}%
\pgfpathlineto{\pgfqpoint{4.051669in}{0.550000in}}%
\pgfpathlineto{\pgfqpoint{4.051669in}{0.641866in}}%
\pgfpathlineto{\pgfqpoint{4.024148in}{0.641866in}}%
\pgfpathlineto{\pgfqpoint{4.024148in}{0.550000in}}%
\pgfpathclose%
\pgfusepath{fill}%
\end{pgfscope}%
\begin{pgfscope}%
\pgfpathrectangle{\pgfqpoint{1.250000in}{0.550000in}}{\pgfqpoint{7.750000in}{3.850000in}}%
\pgfusepath{clip}%
\pgfsetbuttcap%
\pgfsetmiterjoin%
\definecolor{currentfill}{rgb}{0.501961,0.501961,0.501961}%
\pgfsetfillcolor{currentfill}%
\pgfsetfillopacity{0.700000}%
\pgfsetlinewidth{0.000000pt}%
\definecolor{currentstroke}{rgb}{0.000000,0.000000,0.000000}%
\pgfsetstrokecolor{currentstroke}%
\pgfsetstrokeopacity{0.700000}%
\pgfsetdash{}{0pt}%
\pgfpathmoveto{\pgfqpoint{4.051669in}{0.550000in}}%
\pgfpathlineto{\pgfqpoint{4.079190in}{0.550000in}}%
\pgfpathlineto{\pgfqpoint{4.079190in}{0.629464in}}%
\pgfpathlineto{\pgfqpoint{4.051669in}{0.629464in}}%
\pgfpathlineto{\pgfqpoint{4.051669in}{0.550000in}}%
\pgfpathclose%
\pgfusepath{fill}%
\end{pgfscope}%
\begin{pgfscope}%
\pgfpathrectangle{\pgfqpoint{1.250000in}{0.550000in}}{\pgfqpoint{7.750000in}{3.850000in}}%
\pgfusepath{clip}%
\pgfsetbuttcap%
\pgfsetmiterjoin%
\definecolor{currentfill}{rgb}{0.501961,0.501961,0.501961}%
\pgfsetfillcolor{currentfill}%
\pgfsetfillopacity{0.700000}%
\pgfsetlinewidth{0.000000pt}%
\definecolor{currentstroke}{rgb}{0.000000,0.000000,0.000000}%
\pgfsetstrokecolor{currentstroke}%
\pgfsetstrokeopacity{0.700000}%
\pgfsetdash{}{0pt}%
\pgfpathmoveto{\pgfqpoint{4.079190in}{0.550000in}}%
\pgfpathlineto{\pgfqpoint{4.106712in}{0.550000in}}%
\pgfpathlineto{\pgfqpoint{4.106712in}{0.630382in}}%
\pgfpathlineto{\pgfqpoint{4.079190in}{0.630382in}}%
\pgfpathlineto{\pgfqpoint{4.079190in}{0.550000in}}%
\pgfpathclose%
\pgfusepath{fill}%
\end{pgfscope}%
\begin{pgfscope}%
\pgfpathrectangle{\pgfqpoint{1.250000in}{0.550000in}}{\pgfqpoint{7.750000in}{3.850000in}}%
\pgfusepath{clip}%
\pgfsetbuttcap%
\pgfsetmiterjoin%
\definecolor{currentfill}{rgb}{0.501961,0.501961,0.501961}%
\pgfsetfillcolor{currentfill}%
\pgfsetfillopacity{0.700000}%
\pgfsetlinewidth{0.000000pt}%
\definecolor{currentstroke}{rgb}{0.000000,0.000000,0.000000}%
\pgfsetstrokecolor{currentstroke}%
\pgfsetstrokeopacity{0.700000}%
\pgfsetdash{}{0pt}%
\pgfpathmoveto{\pgfqpoint{4.106712in}{0.550000in}}%
\pgfpathlineto{\pgfqpoint{4.134233in}{0.550000in}}%
\pgfpathlineto{\pgfqpoint{4.134233in}{0.628239in}}%
\pgfpathlineto{\pgfqpoint{4.106712in}{0.628239in}}%
\pgfpathlineto{\pgfqpoint{4.106712in}{0.550000in}}%
\pgfpathclose%
\pgfusepath{fill}%
\end{pgfscope}%
\begin{pgfscope}%
\pgfpathrectangle{\pgfqpoint{1.250000in}{0.550000in}}{\pgfqpoint{7.750000in}{3.850000in}}%
\pgfusepath{clip}%
\pgfsetbuttcap%
\pgfsetmiterjoin%
\definecolor{currentfill}{rgb}{0.501961,0.501961,0.501961}%
\pgfsetfillcolor{currentfill}%
\pgfsetfillopacity{0.700000}%
\pgfsetlinewidth{0.000000pt}%
\definecolor{currentstroke}{rgb}{0.000000,0.000000,0.000000}%
\pgfsetstrokecolor{currentstroke}%
\pgfsetstrokeopacity{0.700000}%
\pgfsetdash{}{0pt}%
\pgfpathmoveto{\pgfqpoint{4.134233in}{0.550000in}}%
\pgfpathlineto{\pgfqpoint{4.161754in}{0.550000in}}%
\pgfpathlineto{\pgfqpoint{4.161754in}{0.632985in}}%
\pgfpathlineto{\pgfqpoint{4.134233in}{0.632985in}}%
\pgfpathlineto{\pgfqpoint{4.134233in}{0.550000in}}%
\pgfpathclose%
\pgfusepath{fill}%
\end{pgfscope}%
\begin{pgfscope}%
\pgfpathrectangle{\pgfqpoint{1.250000in}{0.550000in}}{\pgfqpoint{7.750000in}{3.850000in}}%
\pgfusepath{clip}%
\pgfsetbuttcap%
\pgfsetmiterjoin%
\definecolor{currentfill}{rgb}{0.501961,0.501961,0.501961}%
\pgfsetfillcolor{currentfill}%
\pgfsetfillopacity{0.700000}%
\pgfsetlinewidth{0.000000pt}%
\definecolor{currentstroke}{rgb}{0.000000,0.000000,0.000000}%
\pgfsetstrokecolor{currentstroke}%
\pgfsetstrokeopacity{0.700000}%
\pgfsetdash{}{0pt}%
\pgfpathmoveto{\pgfqpoint{4.161754in}{0.550000in}}%
\pgfpathlineto{\pgfqpoint{4.189276in}{0.550000in}}%
\pgfpathlineto{\pgfqpoint{4.189276in}{0.634516in}}%
\pgfpathlineto{\pgfqpoint{4.161754in}{0.634516in}}%
\pgfpathlineto{\pgfqpoint{4.161754in}{0.550000in}}%
\pgfpathclose%
\pgfusepath{fill}%
\end{pgfscope}%
\begin{pgfscope}%
\pgfpathrectangle{\pgfqpoint{1.250000in}{0.550000in}}{\pgfqpoint{7.750000in}{3.850000in}}%
\pgfusepath{clip}%
\pgfsetbuttcap%
\pgfsetmiterjoin%
\definecolor{currentfill}{rgb}{0.501961,0.501961,0.501961}%
\pgfsetfillcolor{currentfill}%
\pgfsetfillopacity{0.700000}%
\pgfsetlinewidth{0.000000pt}%
\definecolor{currentstroke}{rgb}{0.000000,0.000000,0.000000}%
\pgfsetstrokecolor{currentstroke}%
\pgfsetstrokeopacity{0.700000}%
\pgfsetdash{}{0pt}%
\pgfpathmoveto{\pgfqpoint{4.189276in}{0.550000in}}%
\pgfpathlineto{\pgfqpoint{4.216797in}{0.550000in}}%
\pgfpathlineto{\pgfqpoint{4.216797in}{0.638038in}}%
\pgfpathlineto{\pgfqpoint{4.189276in}{0.638038in}}%
\pgfpathlineto{\pgfqpoint{4.189276in}{0.550000in}}%
\pgfpathclose%
\pgfusepath{fill}%
\end{pgfscope}%
\begin{pgfscope}%
\pgfpathrectangle{\pgfqpoint{1.250000in}{0.550000in}}{\pgfqpoint{7.750000in}{3.850000in}}%
\pgfusepath{clip}%
\pgfsetbuttcap%
\pgfsetmiterjoin%
\definecolor{currentfill}{rgb}{0.501961,0.501961,0.501961}%
\pgfsetfillcolor{currentfill}%
\pgfsetfillopacity{0.700000}%
\pgfsetlinewidth{0.000000pt}%
\definecolor{currentstroke}{rgb}{0.000000,0.000000,0.000000}%
\pgfsetstrokecolor{currentstroke}%
\pgfsetstrokeopacity{0.700000}%
\pgfsetdash{}{0pt}%
\pgfpathmoveto{\pgfqpoint{4.216797in}{0.550000in}}%
\pgfpathlineto{\pgfqpoint{4.244318in}{0.550000in}}%
\pgfpathlineto{\pgfqpoint{4.244318in}{0.629923in}}%
\pgfpathlineto{\pgfqpoint{4.216797in}{0.629923in}}%
\pgfpathlineto{\pgfqpoint{4.216797in}{0.550000in}}%
\pgfpathclose%
\pgfusepath{fill}%
\end{pgfscope}%
\begin{pgfscope}%
\pgfpathrectangle{\pgfqpoint{1.250000in}{0.550000in}}{\pgfqpoint{7.750000in}{3.850000in}}%
\pgfusepath{clip}%
\pgfsetbuttcap%
\pgfsetmiterjoin%
\definecolor{currentfill}{rgb}{0.501961,0.501961,0.501961}%
\pgfsetfillcolor{currentfill}%
\pgfsetfillopacity{0.700000}%
\pgfsetlinewidth{0.000000pt}%
\definecolor{currentstroke}{rgb}{0.000000,0.000000,0.000000}%
\pgfsetstrokecolor{currentstroke}%
\pgfsetstrokeopacity{0.700000}%
\pgfsetdash{}{0pt}%
\pgfpathmoveto{\pgfqpoint{4.244318in}{0.550000in}}%
\pgfpathlineto{\pgfqpoint{4.271839in}{0.550000in}}%
\pgfpathlineto{\pgfqpoint{4.271839in}{0.643397in}}%
\pgfpathlineto{\pgfqpoint{4.244318in}{0.643397in}}%
\pgfpathlineto{\pgfqpoint{4.244318in}{0.550000in}}%
\pgfpathclose%
\pgfusepath{fill}%
\end{pgfscope}%
\begin{pgfscope}%
\pgfpathrectangle{\pgfqpoint{1.250000in}{0.550000in}}{\pgfqpoint{7.750000in}{3.850000in}}%
\pgfusepath{clip}%
\pgfsetbuttcap%
\pgfsetmiterjoin%
\definecolor{currentfill}{rgb}{0.501961,0.501961,0.501961}%
\pgfsetfillcolor{currentfill}%
\pgfsetfillopacity{0.700000}%
\pgfsetlinewidth{0.000000pt}%
\definecolor{currentstroke}{rgb}{0.000000,0.000000,0.000000}%
\pgfsetstrokecolor{currentstroke}%
\pgfsetstrokeopacity{0.700000}%
\pgfsetdash{}{0pt}%
\pgfpathmoveto{\pgfqpoint{4.271839in}{0.550000in}}%
\pgfpathlineto{\pgfqpoint{4.299361in}{0.550000in}}%
\pgfpathlineto{\pgfqpoint{4.299361in}{0.633598in}}%
\pgfpathlineto{\pgfqpoint{4.271839in}{0.633598in}}%
\pgfpathlineto{\pgfqpoint{4.271839in}{0.550000in}}%
\pgfpathclose%
\pgfusepath{fill}%
\end{pgfscope}%
\begin{pgfscope}%
\pgfpathrectangle{\pgfqpoint{1.250000in}{0.550000in}}{\pgfqpoint{7.750000in}{3.850000in}}%
\pgfusepath{clip}%
\pgfsetbuttcap%
\pgfsetmiterjoin%
\definecolor{currentfill}{rgb}{0.501961,0.501961,0.501961}%
\pgfsetfillcolor{currentfill}%
\pgfsetfillopacity{0.700000}%
\pgfsetlinewidth{0.000000pt}%
\definecolor{currentstroke}{rgb}{0.000000,0.000000,0.000000}%
\pgfsetstrokecolor{currentstroke}%
\pgfsetstrokeopacity{0.700000}%
\pgfsetdash{}{0pt}%
\pgfpathmoveto{\pgfqpoint{4.299361in}{0.550000in}}%
\pgfpathlineto{\pgfqpoint{4.326882in}{0.550000in}}%
\pgfpathlineto{\pgfqpoint{4.326882in}{0.627473in}}%
\pgfpathlineto{\pgfqpoint{4.299361in}{0.627473in}}%
\pgfpathlineto{\pgfqpoint{4.299361in}{0.550000in}}%
\pgfpathclose%
\pgfusepath{fill}%
\end{pgfscope}%
\begin{pgfscope}%
\pgfpathrectangle{\pgfqpoint{1.250000in}{0.550000in}}{\pgfqpoint{7.750000in}{3.850000in}}%
\pgfusepath{clip}%
\pgfsetbuttcap%
\pgfsetmiterjoin%
\definecolor{currentfill}{rgb}{0.501961,0.501961,0.501961}%
\pgfsetfillcolor{currentfill}%
\pgfsetfillopacity{0.700000}%
\pgfsetlinewidth{0.000000pt}%
\definecolor{currentstroke}{rgb}{0.000000,0.000000,0.000000}%
\pgfsetstrokecolor{currentstroke}%
\pgfsetstrokeopacity{0.700000}%
\pgfsetdash{}{0pt}%
\pgfpathmoveto{\pgfqpoint{4.326882in}{0.550000in}}%
\pgfpathlineto{\pgfqpoint{4.354403in}{0.550000in}}%
\pgfpathlineto{\pgfqpoint{4.354403in}{0.644622in}}%
\pgfpathlineto{\pgfqpoint{4.326882in}{0.644622in}}%
\pgfpathlineto{\pgfqpoint{4.326882in}{0.550000in}}%
\pgfpathclose%
\pgfusepath{fill}%
\end{pgfscope}%
\begin{pgfscope}%
\pgfpathrectangle{\pgfqpoint{1.250000in}{0.550000in}}{\pgfqpoint{7.750000in}{3.850000in}}%
\pgfusepath{clip}%
\pgfsetbuttcap%
\pgfsetmiterjoin%
\definecolor{currentfill}{rgb}{0.501961,0.501961,0.501961}%
\pgfsetfillcolor{currentfill}%
\pgfsetfillopacity{0.700000}%
\pgfsetlinewidth{0.000000pt}%
\definecolor{currentstroke}{rgb}{0.000000,0.000000,0.000000}%
\pgfsetstrokecolor{currentstroke}%
\pgfsetstrokeopacity{0.700000}%
\pgfsetdash{}{0pt}%
\pgfpathmoveto{\pgfqpoint{4.354403in}{0.550000in}}%
\pgfpathlineto{\pgfqpoint{4.381925in}{0.550000in}}%
\pgfpathlineto{\pgfqpoint{4.381925in}{0.637426in}}%
\pgfpathlineto{\pgfqpoint{4.354403in}{0.637426in}}%
\pgfpathlineto{\pgfqpoint{4.354403in}{0.550000in}}%
\pgfpathclose%
\pgfusepath{fill}%
\end{pgfscope}%
\begin{pgfscope}%
\pgfpathrectangle{\pgfqpoint{1.250000in}{0.550000in}}{\pgfqpoint{7.750000in}{3.850000in}}%
\pgfusepath{clip}%
\pgfsetbuttcap%
\pgfsetmiterjoin%
\definecolor{currentfill}{rgb}{0.501961,0.501961,0.501961}%
\pgfsetfillcolor{currentfill}%
\pgfsetfillopacity{0.700000}%
\pgfsetlinewidth{0.000000pt}%
\definecolor{currentstroke}{rgb}{0.000000,0.000000,0.000000}%
\pgfsetstrokecolor{currentstroke}%
\pgfsetstrokeopacity{0.700000}%
\pgfsetdash{}{0pt}%
\pgfpathmoveto{\pgfqpoint{4.381925in}{0.550000in}}%
\pgfpathlineto{\pgfqpoint{4.409446in}{0.550000in}}%
\pgfpathlineto{\pgfqpoint{4.409446in}{0.628851in}}%
\pgfpathlineto{\pgfqpoint{4.381925in}{0.628851in}}%
\pgfpathlineto{\pgfqpoint{4.381925in}{0.550000in}}%
\pgfpathclose%
\pgfusepath{fill}%
\end{pgfscope}%
\begin{pgfscope}%
\pgfpathrectangle{\pgfqpoint{1.250000in}{0.550000in}}{\pgfqpoint{7.750000in}{3.850000in}}%
\pgfusepath{clip}%
\pgfsetbuttcap%
\pgfsetmiterjoin%
\definecolor{currentfill}{rgb}{0.501961,0.501961,0.501961}%
\pgfsetfillcolor{currentfill}%
\pgfsetfillopacity{0.700000}%
\pgfsetlinewidth{0.000000pt}%
\definecolor{currentstroke}{rgb}{0.000000,0.000000,0.000000}%
\pgfsetstrokecolor{currentstroke}%
\pgfsetstrokeopacity{0.700000}%
\pgfsetdash{}{0pt}%
\pgfpathmoveto{\pgfqpoint{4.409446in}{0.550000in}}%
\pgfpathlineto{\pgfqpoint{4.436967in}{0.550000in}}%
\pgfpathlineto{\pgfqpoint{4.436967in}{0.638957in}}%
\pgfpathlineto{\pgfqpoint{4.409446in}{0.638957in}}%
\pgfpathlineto{\pgfqpoint{4.409446in}{0.550000in}}%
\pgfpathclose%
\pgfusepath{fill}%
\end{pgfscope}%
\begin{pgfscope}%
\pgfpathrectangle{\pgfqpoint{1.250000in}{0.550000in}}{\pgfqpoint{7.750000in}{3.850000in}}%
\pgfusepath{clip}%
\pgfsetbuttcap%
\pgfsetmiterjoin%
\definecolor{currentfill}{rgb}{0.501961,0.501961,0.501961}%
\pgfsetfillcolor{currentfill}%
\pgfsetfillopacity{0.700000}%
\pgfsetlinewidth{0.000000pt}%
\definecolor{currentstroke}{rgb}{0.000000,0.000000,0.000000}%
\pgfsetstrokecolor{currentstroke}%
\pgfsetstrokeopacity{0.700000}%
\pgfsetdash{}{0pt}%
\pgfpathmoveto{\pgfqpoint{4.436967in}{0.550000in}}%
\pgfpathlineto{\pgfqpoint{4.464489in}{0.550000in}}%
\pgfpathlineto{\pgfqpoint{4.464489in}{0.632067in}}%
\pgfpathlineto{\pgfqpoint{4.436967in}{0.632067in}}%
\pgfpathlineto{\pgfqpoint{4.436967in}{0.550000in}}%
\pgfpathclose%
\pgfusepath{fill}%
\end{pgfscope}%
\begin{pgfscope}%
\pgfpathrectangle{\pgfqpoint{1.250000in}{0.550000in}}{\pgfqpoint{7.750000in}{3.850000in}}%
\pgfusepath{clip}%
\pgfsetbuttcap%
\pgfsetmiterjoin%
\definecolor{currentfill}{rgb}{0.501961,0.501961,0.501961}%
\pgfsetfillcolor{currentfill}%
\pgfsetfillopacity{0.700000}%
\pgfsetlinewidth{0.000000pt}%
\definecolor{currentstroke}{rgb}{0.000000,0.000000,0.000000}%
\pgfsetstrokecolor{currentstroke}%
\pgfsetstrokeopacity{0.700000}%
\pgfsetdash{}{0pt}%
\pgfpathmoveto{\pgfqpoint{4.464489in}{0.550000in}}%
\pgfpathlineto{\pgfqpoint{4.492010in}{0.550000in}}%
\pgfpathlineto{\pgfqpoint{4.492010in}{0.622727in}}%
\pgfpathlineto{\pgfqpoint{4.464489in}{0.622727in}}%
\pgfpathlineto{\pgfqpoint{4.464489in}{0.550000in}}%
\pgfpathclose%
\pgfusepath{fill}%
\end{pgfscope}%
\begin{pgfscope}%
\pgfpathrectangle{\pgfqpoint{1.250000in}{0.550000in}}{\pgfqpoint{7.750000in}{3.850000in}}%
\pgfusepath{clip}%
\pgfsetbuttcap%
\pgfsetmiterjoin%
\definecolor{currentfill}{rgb}{0.501961,0.501961,0.501961}%
\pgfsetfillcolor{currentfill}%
\pgfsetfillopacity{0.700000}%
\pgfsetlinewidth{0.000000pt}%
\definecolor{currentstroke}{rgb}{0.000000,0.000000,0.000000}%
\pgfsetstrokecolor{currentstroke}%
\pgfsetstrokeopacity{0.700000}%
\pgfsetdash{}{0pt}%
\pgfpathmoveto{\pgfqpoint{4.492010in}{0.550000in}}%
\pgfpathlineto{\pgfqpoint{4.519531in}{0.550000in}}%
\pgfpathlineto{\pgfqpoint{4.519531in}{0.641713in}}%
\pgfpathlineto{\pgfqpoint{4.492010in}{0.641713in}}%
\pgfpathlineto{\pgfqpoint{4.492010in}{0.550000in}}%
\pgfpathclose%
\pgfusepath{fill}%
\end{pgfscope}%
\begin{pgfscope}%
\pgfpathrectangle{\pgfqpoint{1.250000in}{0.550000in}}{\pgfqpoint{7.750000in}{3.850000in}}%
\pgfusepath{clip}%
\pgfsetbuttcap%
\pgfsetmiterjoin%
\definecolor{currentfill}{rgb}{0.501961,0.501961,0.501961}%
\pgfsetfillcolor{currentfill}%
\pgfsetfillopacity{0.700000}%
\pgfsetlinewidth{0.000000pt}%
\definecolor{currentstroke}{rgb}{0.000000,0.000000,0.000000}%
\pgfsetstrokecolor{currentstroke}%
\pgfsetstrokeopacity{0.700000}%
\pgfsetdash{}{0pt}%
\pgfpathmoveto{\pgfqpoint{4.519531in}{0.550000in}}%
\pgfpathlineto{\pgfqpoint{4.547053in}{0.550000in}}%
\pgfpathlineto{\pgfqpoint{4.547053in}{0.624411in}}%
\pgfpathlineto{\pgfqpoint{4.519531in}{0.624411in}}%
\pgfpathlineto{\pgfqpoint{4.519531in}{0.550000in}}%
\pgfpathclose%
\pgfusepath{fill}%
\end{pgfscope}%
\begin{pgfscope}%
\pgfpathrectangle{\pgfqpoint{1.250000in}{0.550000in}}{\pgfqpoint{7.750000in}{3.850000in}}%
\pgfusepath{clip}%
\pgfsetbuttcap%
\pgfsetmiterjoin%
\definecolor{currentfill}{rgb}{0.501961,0.501961,0.501961}%
\pgfsetfillcolor{currentfill}%
\pgfsetfillopacity{0.700000}%
\pgfsetlinewidth{0.000000pt}%
\definecolor{currentstroke}{rgb}{0.000000,0.000000,0.000000}%
\pgfsetstrokecolor{currentstroke}%
\pgfsetstrokeopacity{0.700000}%
\pgfsetdash{}{0pt}%
\pgfpathmoveto{\pgfqpoint{4.547053in}{0.550000in}}%
\pgfpathlineto{\pgfqpoint{4.574574in}{0.550000in}}%
\pgfpathlineto{\pgfqpoint{4.574574in}{0.637579in}}%
\pgfpathlineto{\pgfqpoint{4.547053in}{0.637579in}}%
\pgfpathlineto{\pgfqpoint{4.547053in}{0.550000in}}%
\pgfpathclose%
\pgfusepath{fill}%
\end{pgfscope}%
\begin{pgfscope}%
\pgfpathrectangle{\pgfqpoint{1.250000in}{0.550000in}}{\pgfqpoint{7.750000in}{3.850000in}}%
\pgfusepath{clip}%
\pgfsetbuttcap%
\pgfsetmiterjoin%
\definecolor{currentfill}{rgb}{0.501961,0.501961,0.501961}%
\pgfsetfillcolor{currentfill}%
\pgfsetfillopacity{0.700000}%
\pgfsetlinewidth{0.000000pt}%
\definecolor{currentstroke}{rgb}{0.000000,0.000000,0.000000}%
\pgfsetstrokecolor{currentstroke}%
\pgfsetstrokeopacity{0.700000}%
\pgfsetdash{}{0pt}%
\pgfpathmoveto{\pgfqpoint{4.574574in}{0.550000in}}%
\pgfpathlineto{\pgfqpoint{4.602095in}{0.550000in}}%
\pgfpathlineto{\pgfqpoint{4.602095in}{0.631760in}}%
\pgfpathlineto{\pgfqpoint{4.574574in}{0.631760in}}%
\pgfpathlineto{\pgfqpoint{4.574574in}{0.550000in}}%
\pgfpathclose%
\pgfusepath{fill}%
\end{pgfscope}%
\begin{pgfscope}%
\pgfpathrectangle{\pgfqpoint{1.250000in}{0.550000in}}{\pgfqpoint{7.750000in}{3.850000in}}%
\pgfusepath{clip}%
\pgfsetbuttcap%
\pgfsetmiterjoin%
\definecolor{currentfill}{rgb}{0.501961,0.501961,0.501961}%
\pgfsetfillcolor{currentfill}%
\pgfsetfillopacity{0.700000}%
\pgfsetlinewidth{0.000000pt}%
\definecolor{currentstroke}{rgb}{0.000000,0.000000,0.000000}%
\pgfsetstrokecolor{currentstroke}%
\pgfsetstrokeopacity{0.700000}%
\pgfsetdash{}{0pt}%
\pgfpathmoveto{\pgfqpoint{4.602095in}{0.550000in}}%
\pgfpathlineto{\pgfqpoint{4.629616in}{0.550000in}}%
\pgfpathlineto{\pgfqpoint{4.629616in}{0.633598in}}%
\pgfpathlineto{\pgfqpoint{4.602095in}{0.633598in}}%
\pgfpathlineto{\pgfqpoint{4.602095in}{0.550000in}}%
\pgfpathclose%
\pgfusepath{fill}%
\end{pgfscope}%
\begin{pgfscope}%
\pgfpathrectangle{\pgfqpoint{1.250000in}{0.550000in}}{\pgfqpoint{7.750000in}{3.850000in}}%
\pgfusepath{clip}%
\pgfsetbuttcap%
\pgfsetmiterjoin%
\definecolor{currentfill}{rgb}{0.501961,0.501961,0.501961}%
\pgfsetfillcolor{currentfill}%
\pgfsetfillopacity{0.700000}%
\pgfsetlinewidth{0.000000pt}%
\definecolor{currentstroke}{rgb}{0.000000,0.000000,0.000000}%
\pgfsetstrokecolor{currentstroke}%
\pgfsetstrokeopacity{0.700000}%
\pgfsetdash{}{0pt}%
\pgfpathmoveto{\pgfqpoint{4.629616in}{0.550000in}}%
\pgfpathlineto{\pgfqpoint{4.657138in}{0.550000in}}%
\pgfpathlineto{\pgfqpoint{4.657138in}{0.642937in}}%
\pgfpathlineto{\pgfqpoint{4.629616in}{0.642937in}}%
\pgfpathlineto{\pgfqpoint{4.629616in}{0.550000in}}%
\pgfpathclose%
\pgfusepath{fill}%
\end{pgfscope}%
\begin{pgfscope}%
\pgfpathrectangle{\pgfqpoint{1.250000in}{0.550000in}}{\pgfqpoint{7.750000in}{3.850000in}}%
\pgfusepath{clip}%
\pgfsetbuttcap%
\pgfsetmiterjoin%
\definecolor{currentfill}{rgb}{0.501961,0.501961,0.501961}%
\pgfsetfillcolor{currentfill}%
\pgfsetfillopacity{0.700000}%
\pgfsetlinewidth{0.000000pt}%
\definecolor{currentstroke}{rgb}{0.000000,0.000000,0.000000}%
\pgfsetstrokecolor{currentstroke}%
\pgfsetstrokeopacity{0.700000}%
\pgfsetdash{}{0pt}%
\pgfpathmoveto{\pgfqpoint{4.657138in}{0.550000in}}%
\pgfpathlineto{\pgfqpoint{4.684659in}{0.550000in}}%
\pgfpathlineto{\pgfqpoint{4.684659in}{0.624411in}}%
\pgfpathlineto{\pgfqpoint{4.657138in}{0.624411in}}%
\pgfpathlineto{\pgfqpoint{4.657138in}{0.550000in}}%
\pgfpathclose%
\pgfusepath{fill}%
\end{pgfscope}%
\begin{pgfscope}%
\pgfpathrectangle{\pgfqpoint{1.250000in}{0.550000in}}{\pgfqpoint{7.750000in}{3.850000in}}%
\pgfusepath{clip}%
\pgfsetbuttcap%
\pgfsetmiterjoin%
\definecolor{currentfill}{rgb}{0.501961,0.501961,0.501961}%
\pgfsetfillcolor{currentfill}%
\pgfsetfillopacity{0.700000}%
\pgfsetlinewidth{0.000000pt}%
\definecolor{currentstroke}{rgb}{0.000000,0.000000,0.000000}%
\pgfsetstrokecolor{currentstroke}%
\pgfsetstrokeopacity{0.700000}%
\pgfsetdash{}{0pt}%
\pgfpathmoveto{\pgfqpoint{4.684659in}{0.550000in}}%
\pgfpathlineto{\pgfqpoint{4.712180in}{0.550000in}}%
\pgfpathlineto{\pgfqpoint{4.712180in}{0.627627in}}%
\pgfpathlineto{\pgfqpoint{4.684659in}{0.627627in}}%
\pgfpathlineto{\pgfqpoint{4.684659in}{0.550000in}}%
\pgfpathclose%
\pgfusepath{fill}%
\end{pgfscope}%
\begin{pgfscope}%
\pgfpathrectangle{\pgfqpoint{1.250000in}{0.550000in}}{\pgfqpoint{7.750000in}{3.850000in}}%
\pgfusepath{clip}%
\pgfsetbuttcap%
\pgfsetmiterjoin%
\definecolor{currentfill}{rgb}{0.501961,0.501961,0.501961}%
\pgfsetfillcolor{currentfill}%
\pgfsetfillopacity{0.700000}%
\pgfsetlinewidth{0.000000pt}%
\definecolor{currentstroke}{rgb}{0.000000,0.000000,0.000000}%
\pgfsetstrokecolor{currentstroke}%
\pgfsetstrokeopacity{0.700000}%
\pgfsetdash{}{0pt}%
\pgfpathmoveto{\pgfqpoint{4.712180in}{0.550000in}}%
\pgfpathlineto{\pgfqpoint{4.739702in}{0.550000in}}%
\pgfpathlineto{\pgfqpoint{4.739702in}{0.636048in}}%
\pgfpathlineto{\pgfqpoint{4.712180in}{0.636048in}}%
\pgfpathlineto{\pgfqpoint{4.712180in}{0.550000in}}%
\pgfpathclose%
\pgfusepath{fill}%
\end{pgfscope}%
\begin{pgfscope}%
\pgfpathrectangle{\pgfqpoint{1.250000in}{0.550000in}}{\pgfqpoint{7.750000in}{3.850000in}}%
\pgfusepath{clip}%
\pgfsetbuttcap%
\pgfsetmiterjoin%
\definecolor{currentfill}{rgb}{0.501961,0.501961,0.501961}%
\pgfsetfillcolor{currentfill}%
\pgfsetfillopacity{0.700000}%
\pgfsetlinewidth{0.000000pt}%
\definecolor{currentstroke}{rgb}{0.000000,0.000000,0.000000}%
\pgfsetstrokecolor{currentstroke}%
\pgfsetstrokeopacity{0.700000}%
\pgfsetdash{}{0pt}%
\pgfpathmoveto{\pgfqpoint{4.739702in}{0.550000in}}%
\pgfpathlineto{\pgfqpoint{4.767223in}{0.550000in}}%
\pgfpathlineto{\pgfqpoint{4.767223in}{0.619052in}}%
\pgfpathlineto{\pgfqpoint{4.739702in}{0.619052in}}%
\pgfpathlineto{\pgfqpoint{4.739702in}{0.550000in}}%
\pgfpathclose%
\pgfusepath{fill}%
\end{pgfscope}%
\begin{pgfscope}%
\pgfpathrectangle{\pgfqpoint{1.250000in}{0.550000in}}{\pgfqpoint{7.750000in}{3.850000in}}%
\pgfusepath{clip}%
\pgfsetbuttcap%
\pgfsetmiterjoin%
\definecolor{currentfill}{rgb}{0.501961,0.501961,0.501961}%
\pgfsetfillcolor{currentfill}%
\pgfsetfillopacity{0.700000}%
\pgfsetlinewidth{0.000000pt}%
\definecolor{currentstroke}{rgb}{0.000000,0.000000,0.000000}%
\pgfsetstrokecolor{currentstroke}%
\pgfsetstrokeopacity{0.700000}%
\pgfsetdash{}{0pt}%
\pgfpathmoveto{\pgfqpoint{4.767223in}{0.550000in}}%
\pgfpathlineto{\pgfqpoint{4.794744in}{0.550000in}}%
\pgfpathlineto{\pgfqpoint{4.794744in}{0.637426in}}%
\pgfpathlineto{\pgfqpoint{4.767223in}{0.637426in}}%
\pgfpathlineto{\pgfqpoint{4.767223in}{0.550000in}}%
\pgfpathclose%
\pgfusepath{fill}%
\end{pgfscope}%
\begin{pgfscope}%
\pgfpathrectangle{\pgfqpoint{1.250000in}{0.550000in}}{\pgfqpoint{7.750000in}{3.850000in}}%
\pgfusepath{clip}%
\pgfsetbuttcap%
\pgfsetmiterjoin%
\definecolor{currentfill}{rgb}{0.501961,0.501961,0.501961}%
\pgfsetfillcolor{currentfill}%
\pgfsetfillopacity{0.700000}%
\pgfsetlinewidth{0.000000pt}%
\definecolor{currentstroke}{rgb}{0.000000,0.000000,0.000000}%
\pgfsetstrokecolor{currentstroke}%
\pgfsetstrokeopacity{0.700000}%
\pgfsetdash{}{0pt}%
\pgfpathmoveto{\pgfqpoint{4.794744in}{0.550000in}}%
\pgfpathlineto{\pgfqpoint{4.822266in}{0.550000in}}%
\pgfpathlineto{\pgfqpoint{4.822266in}{0.622421in}}%
\pgfpathlineto{\pgfqpoint{4.794744in}{0.622421in}}%
\pgfpathlineto{\pgfqpoint{4.794744in}{0.550000in}}%
\pgfpathclose%
\pgfusepath{fill}%
\end{pgfscope}%
\begin{pgfscope}%
\pgfpathrectangle{\pgfqpoint{1.250000in}{0.550000in}}{\pgfqpoint{7.750000in}{3.850000in}}%
\pgfusepath{clip}%
\pgfsetbuttcap%
\pgfsetmiterjoin%
\definecolor{currentfill}{rgb}{0.501961,0.501961,0.501961}%
\pgfsetfillcolor{currentfill}%
\pgfsetfillopacity{0.700000}%
\pgfsetlinewidth{0.000000pt}%
\definecolor{currentstroke}{rgb}{0.000000,0.000000,0.000000}%
\pgfsetstrokecolor{currentstroke}%
\pgfsetstrokeopacity{0.700000}%
\pgfsetdash{}{0pt}%
\pgfpathmoveto{\pgfqpoint{4.822266in}{0.550000in}}%
\pgfpathlineto{\pgfqpoint{4.849787in}{0.550000in}}%
\pgfpathlineto{\pgfqpoint{4.849787in}{0.633904in}}%
\pgfpathlineto{\pgfqpoint{4.822266in}{0.633904in}}%
\pgfpathlineto{\pgfqpoint{4.822266in}{0.550000in}}%
\pgfpathclose%
\pgfusepath{fill}%
\end{pgfscope}%
\begin{pgfscope}%
\pgfpathrectangle{\pgfqpoint{1.250000in}{0.550000in}}{\pgfqpoint{7.750000in}{3.850000in}}%
\pgfusepath{clip}%
\pgfsetbuttcap%
\pgfsetmiterjoin%
\definecolor{currentfill}{rgb}{0.501961,0.501961,0.501961}%
\pgfsetfillcolor{currentfill}%
\pgfsetfillopacity{0.700000}%
\pgfsetlinewidth{0.000000pt}%
\definecolor{currentstroke}{rgb}{0.000000,0.000000,0.000000}%
\pgfsetstrokecolor{currentstroke}%
\pgfsetstrokeopacity{0.700000}%
\pgfsetdash{}{0pt}%
\pgfpathmoveto{\pgfqpoint{4.849787in}{0.550000in}}%
\pgfpathlineto{\pgfqpoint{4.877308in}{0.550000in}}%
\pgfpathlineto{\pgfqpoint{4.877308in}{0.619052in}}%
\pgfpathlineto{\pgfqpoint{4.849787in}{0.619052in}}%
\pgfpathlineto{\pgfqpoint{4.849787in}{0.550000in}}%
\pgfpathclose%
\pgfusepath{fill}%
\end{pgfscope}%
\begin{pgfscope}%
\pgfpathrectangle{\pgfqpoint{1.250000in}{0.550000in}}{\pgfqpoint{7.750000in}{3.850000in}}%
\pgfusepath{clip}%
\pgfsetbuttcap%
\pgfsetmiterjoin%
\definecolor{currentfill}{rgb}{0.501961,0.501961,0.501961}%
\pgfsetfillcolor{currentfill}%
\pgfsetfillopacity{0.700000}%
\pgfsetlinewidth{0.000000pt}%
\definecolor{currentstroke}{rgb}{0.000000,0.000000,0.000000}%
\pgfsetstrokecolor{currentstroke}%
\pgfsetstrokeopacity{0.700000}%
\pgfsetdash{}{0pt}%
\pgfpathmoveto{\pgfqpoint{4.877308in}{0.550000in}}%
\pgfpathlineto{\pgfqpoint{4.904830in}{0.550000in}}%
\pgfpathlineto{\pgfqpoint{4.904830in}{0.622574in}}%
\pgfpathlineto{\pgfqpoint{4.877308in}{0.622574in}}%
\pgfpathlineto{\pgfqpoint{4.877308in}{0.550000in}}%
\pgfpathclose%
\pgfusepath{fill}%
\end{pgfscope}%
\begin{pgfscope}%
\pgfpathrectangle{\pgfqpoint{1.250000in}{0.550000in}}{\pgfqpoint{7.750000in}{3.850000in}}%
\pgfusepath{clip}%
\pgfsetbuttcap%
\pgfsetmiterjoin%
\definecolor{currentfill}{rgb}{0.501961,0.501961,0.501961}%
\pgfsetfillcolor{currentfill}%
\pgfsetfillopacity{0.700000}%
\pgfsetlinewidth{0.000000pt}%
\definecolor{currentstroke}{rgb}{0.000000,0.000000,0.000000}%
\pgfsetstrokecolor{currentstroke}%
\pgfsetstrokeopacity{0.700000}%
\pgfsetdash{}{0pt}%
\pgfpathmoveto{\pgfqpoint{4.904830in}{0.550000in}}%
\pgfpathlineto{\pgfqpoint{4.932351in}{0.550000in}}%
\pgfpathlineto{\pgfqpoint{4.932351in}{0.628545in}}%
\pgfpathlineto{\pgfqpoint{4.904830in}{0.628545in}}%
\pgfpathlineto{\pgfqpoint{4.904830in}{0.550000in}}%
\pgfpathclose%
\pgfusepath{fill}%
\end{pgfscope}%
\begin{pgfscope}%
\pgfpathrectangle{\pgfqpoint{1.250000in}{0.550000in}}{\pgfqpoint{7.750000in}{3.850000in}}%
\pgfusepath{clip}%
\pgfsetbuttcap%
\pgfsetmiterjoin%
\definecolor{currentfill}{rgb}{0.501961,0.501961,0.501961}%
\pgfsetfillcolor{currentfill}%
\pgfsetfillopacity{0.700000}%
\pgfsetlinewidth{0.000000pt}%
\definecolor{currentstroke}{rgb}{0.000000,0.000000,0.000000}%
\pgfsetstrokecolor{currentstroke}%
\pgfsetstrokeopacity{0.700000}%
\pgfsetdash{}{0pt}%
\pgfpathmoveto{\pgfqpoint{4.932351in}{0.550000in}}%
\pgfpathlineto{\pgfqpoint{4.959872in}{0.550000in}}%
\pgfpathlineto{\pgfqpoint{4.959872in}{0.616143in}}%
\pgfpathlineto{\pgfqpoint{4.932351in}{0.616143in}}%
\pgfpathlineto{\pgfqpoint{4.932351in}{0.550000in}}%
\pgfpathclose%
\pgfusepath{fill}%
\end{pgfscope}%
\begin{pgfscope}%
\pgfpathrectangle{\pgfqpoint{1.250000in}{0.550000in}}{\pgfqpoint{7.750000in}{3.850000in}}%
\pgfusepath{clip}%
\pgfsetbuttcap%
\pgfsetmiterjoin%
\definecolor{currentfill}{rgb}{0.501961,0.501961,0.501961}%
\pgfsetfillcolor{currentfill}%
\pgfsetfillopacity{0.700000}%
\pgfsetlinewidth{0.000000pt}%
\definecolor{currentstroke}{rgb}{0.000000,0.000000,0.000000}%
\pgfsetstrokecolor{currentstroke}%
\pgfsetstrokeopacity{0.700000}%
\pgfsetdash{}{0pt}%
\pgfpathmoveto{\pgfqpoint{4.959872in}{0.550000in}}%
\pgfpathlineto{\pgfqpoint{4.987393in}{0.550000in}}%
\pgfpathlineto{\pgfqpoint{4.987393in}{0.620583in}}%
\pgfpathlineto{\pgfqpoint{4.959872in}{0.620583in}}%
\pgfpathlineto{\pgfqpoint{4.959872in}{0.550000in}}%
\pgfpathclose%
\pgfusepath{fill}%
\end{pgfscope}%
\begin{pgfscope}%
\pgfpathrectangle{\pgfqpoint{1.250000in}{0.550000in}}{\pgfqpoint{7.750000in}{3.850000in}}%
\pgfusepath{clip}%
\pgfsetbuttcap%
\pgfsetmiterjoin%
\definecolor{currentfill}{rgb}{0.501961,0.501961,0.501961}%
\pgfsetfillcolor{currentfill}%
\pgfsetfillopacity{0.700000}%
\pgfsetlinewidth{0.000000pt}%
\definecolor{currentstroke}{rgb}{0.000000,0.000000,0.000000}%
\pgfsetstrokecolor{currentstroke}%
\pgfsetstrokeopacity{0.700000}%
\pgfsetdash{}{0pt}%
\pgfpathmoveto{\pgfqpoint{4.987393in}{0.550000in}}%
\pgfpathlineto{\pgfqpoint{5.014915in}{0.550000in}}%
\pgfpathlineto{\pgfqpoint{5.014915in}{0.619971in}}%
\pgfpathlineto{\pgfqpoint{4.987393in}{0.619971in}}%
\pgfpathlineto{\pgfqpoint{4.987393in}{0.550000in}}%
\pgfpathclose%
\pgfusepath{fill}%
\end{pgfscope}%
\begin{pgfscope}%
\pgfpathrectangle{\pgfqpoint{1.250000in}{0.550000in}}{\pgfqpoint{7.750000in}{3.850000in}}%
\pgfusepath{clip}%
\pgfsetbuttcap%
\pgfsetmiterjoin%
\definecolor{currentfill}{rgb}{0.501961,0.501961,0.501961}%
\pgfsetfillcolor{currentfill}%
\pgfsetfillopacity{0.700000}%
\pgfsetlinewidth{0.000000pt}%
\definecolor{currentstroke}{rgb}{0.000000,0.000000,0.000000}%
\pgfsetstrokecolor{currentstroke}%
\pgfsetstrokeopacity{0.700000}%
\pgfsetdash{}{0pt}%
\pgfpathmoveto{\pgfqpoint{5.014915in}{0.550000in}}%
\pgfpathlineto{\pgfqpoint{5.042436in}{0.550000in}}%
\pgfpathlineto{\pgfqpoint{5.042436in}{0.610784in}}%
\pgfpathlineto{\pgfqpoint{5.014915in}{0.610784in}}%
\pgfpathlineto{\pgfqpoint{5.014915in}{0.550000in}}%
\pgfpathclose%
\pgfusepath{fill}%
\end{pgfscope}%
\begin{pgfscope}%
\pgfpathrectangle{\pgfqpoint{1.250000in}{0.550000in}}{\pgfqpoint{7.750000in}{3.850000in}}%
\pgfusepath{clip}%
\pgfsetbuttcap%
\pgfsetmiterjoin%
\definecolor{currentfill}{rgb}{0.501961,0.501961,0.501961}%
\pgfsetfillcolor{currentfill}%
\pgfsetfillopacity{0.700000}%
\pgfsetlinewidth{0.000000pt}%
\definecolor{currentstroke}{rgb}{0.000000,0.000000,0.000000}%
\pgfsetstrokecolor{currentstroke}%
\pgfsetstrokeopacity{0.700000}%
\pgfsetdash{}{0pt}%
\pgfpathmoveto{\pgfqpoint{5.042436in}{0.550000in}}%
\pgfpathlineto{\pgfqpoint{5.069957in}{0.550000in}}%
\pgfpathlineto{\pgfqpoint{5.069957in}{0.604201in}}%
\pgfpathlineto{\pgfqpoint{5.042436in}{0.604201in}}%
\pgfpathlineto{\pgfqpoint{5.042436in}{0.550000in}}%
\pgfpathclose%
\pgfusepath{fill}%
\end{pgfscope}%
\begin{pgfscope}%
\pgfpathrectangle{\pgfqpoint{1.250000in}{0.550000in}}{\pgfqpoint{7.750000in}{3.850000in}}%
\pgfusepath{clip}%
\pgfsetbuttcap%
\pgfsetmiterjoin%
\definecolor{currentfill}{rgb}{0.501961,0.501961,0.501961}%
\pgfsetfillcolor{currentfill}%
\pgfsetfillopacity{0.700000}%
\pgfsetlinewidth{0.000000pt}%
\definecolor{currentstroke}{rgb}{0.000000,0.000000,0.000000}%
\pgfsetstrokecolor{currentstroke}%
\pgfsetstrokeopacity{0.700000}%
\pgfsetdash{}{0pt}%
\pgfpathmoveto{\pgfqpoint{5.069957in}{0.550000in}}%
\pgfpathlineto{\pgfqpoint{5.097479in}{0.550000in}}%
\pgfpathlineto{\pgfqpoint{5.097479in}{0.601751in}}%
\pgfpathlineto{\pgfqpoint{5.069957in}{0.601751in}}%
\pgfpathlineto{\pgfqpoint{5.069957in}{0.550000in}}%
\pgfpathclose%
\pgfusepath{fill}%
\end{pgfscope}%
\begin{pgfscope}%
\pgfpathrectangle{\pgfqpoint{1.250000in}{0.550000in}}{\pgfqpoint{7.750000in}{3.850000in}}%
\pgfusepath{clip}%
\pgfsetbuttcap%
\pgfsetmiterjoin%
\definecolor{currentfill}{rgb}{0.501961,0.501961,0.501961}%
\pgfsetfillcolor{currentfill}%
\pgfsetfillopacity{0.700000}%
\pgfsetlinewidth{0.000000pt}%
\definecolor{currentstroke}{rgb}{0.000000,0.000000,0.000000}%
\pgfsetstrokecolor{currentstroke}%
\pgfsetstrokeopacity{0.700000}%
\pgfsetdash{}{0pt}%
\pgfpathmoveto{\pgfqpoint{5.097479in}{0.550000in}}%
\pgfpathlineto{\pgfqpoint{5.125000in}{0.550000in}}%
\pgfpathlineto{\pgfqpoint{5.125000in}{0.600679in}}%
\pgfpathlineto{\pgfqpoint{5.097479in}{0.600679in}}%
\pgfpathlineto{\pgfqpoint{5.097479in}{0.550000in}}%
\pgfpathclose%
\pgfusepath{fill}%
\end{pgfscope}%
\begin{pgfscope}%
\pgfpathrectangle{\pgfqpoint{1.250000in}{0.550000in}}{\pgfqpoint{7.750000in}{3.850000in}}%
\pgfusepath{clip}%
\pgfsetbuttcap%
\pgfsetmiterjoin%
\definecolor{currentfill}{rgb}{0.501961,0.501961,0.501961}%
\pgfsetfillcolor{currentfill}%
\pgfsetfillopacity{0.700000}%
\pgfsetlinewidth{0.000000pt}%
\definecolor{currentstroke}{rgb}{0.000000,0.000000,0.000000}%
\pgfsetstrokecolor{currentstroke}%
\pgfsetstrokeopacity{0.700000}%
\pgfsetdash{}{0pt}%
\pgfpathmoveto{\pgfqpoint{5.125000in}{0.550000in}}%
\pgfpathlineto{\pgfqpoint{5.152521in}{0.550000in}}%
\pgfpathlineto{\pgfqpoint{5.152521in}{0.600526in}}%
\pgfpathlineto{\pgfqpoint{5.125000in}{0.600526in}}%
\pgfpathlineto{\pgfqpoint{5.125000in}{0.550000in}}%
\pgfpathclose%
\pgfusepath{fill}%
\end{pgfscope}%
\begin{pgfscope}%
\pgfpathrectangle{\pgfqpoint{1.250000in}{0.550000in}}{\pgfqpoint{7.750000in}{3.850000in}}%
\pgfusepath{clip}%
\pgfsetbuttcap%
\pgfsetmiterjoin%
\definecolor{currentfill}{rgb}{0.501961,0.501961,0.501961}%
\pgfsetfillcolor{currentfill}%
\pgfsetfillopacity{0.700000}%
\pgfsetlinewidth{0.000000pt}%
\definecolor{currentstroke}{rgb}{0.000000,0.000000,0.000000}%
\pgfsetstrokecolor{currentstroke}%
\pgfsetstrokeopacity{0.700000}%
\pgfsetdash{}{0pt}%
\pgfpathmoveto{\pgfqpoint{5.152521in}{0.550000in}}%
\pgfpathlineto{\pgfqpoint{5.180043in}{0.550000in}}%
\pgfpathlineto{\pgfqpoint{5.180043in}{0.589196in}}%
\pgfpathlineto{\pgfqpoint{5.152521in}{0.589196in}}%
\pgfpathlineto{\pgfqpoint{5.152521in}{0.550000in}}%
\pgfpathclose%
\pgfusepath{fill}%
\end{pgfscope}%
\begin{pgfscope}%
\pgfpathrectangle{\pgfqpoint{1.250000in}{0.550000in}}{\pgfqpoint{7.750000in}{3.850000in}}%
\pgfusepath{clip}%
\pgfsetbuttcap%
\pgfsetmiterjoin%
\definecolor{currentfill}{rgb}{0.501961,0.501961,0.501961}%
\pgfsetfillcolor{currentfill}%
\pgfsetfillopacity{0.700000}%
\pgfsetlinewidth{0.000000pt}%
\definecolor{currentstroke}{rgb}{0.000000,0.000000,0.000000}%
\pgfsetstrokecolor{currentstroke}%
\pgfsetstrokeopacity{0.700000}%
\pgfsetdash{}{0pt}%
\pgfpathmoveto{\pgfqpoint{5.180043in}{0.550000in}}%
\pgfpathlineto{\pgfqpoint{5.207564in}{0.550000in}}%
\pgfpathlineto{\pgfqpoint{5.207564in}{0.597464in}}%
\pgfpathlineto{\pgfqpoint{5.180043in}{0.597464in}}%
\pgfpathlineto{\pgfqpoint{5.180043in}{0.550000in}}%
\pgfpathclose%
\pgfusepath{fill}%
\end{pgfscope}%
\begin{pgfscope}%
\pgfpathrectangle{\pgfqpoint{1.250000in}{0.550000in}}{\pgfqpoint{7.750000in}{3.850000in}}%
\pgfusepath{clip}%
\pgfsetbuttcap%
\pgfsetmiterjoin%
\definecolor{currentfill}{rgb}{0.501961,0.501961,0.501961}%
\pgfsetfillcolor{currentfill}%
\pgfsetfillopacity{0.700000}%
\pgfsetlinewidth{0.000000pt}%
\definecolor{currentstroke}{rgb}{0.000000,0.000000,0.000000}%
\pgfsetstrokecolor{currentstroke}%
\pgfsetstrokeopacity{0.700000}%
\pgfsetdash{}{0pt}%
\pgfpathmoveto{\pgfqpoint{5.207564in}{0.550000in}}%
\pgfpathlineto{\pgfqpoint{5.235085in}{0.550000in}}%
\pgfpathlineto{\pgfqpoint{5.235085in}{0.587818in}}%
\pgfpathlineto{\pgfqpoint{5.207564in}{0.587818in}}%
\pgfpathlineto{\pgfqpoint{5.207564in}{0.550000in}}%
\pgfpathclose%
\pgfusepath{fill}%
\end{pgfscope}%
\begin{pgfscope}%
\pgfpathrectangle{\pgfqpoint{1.250000in}{0.550000in}}{\pgfqpoint{7.750000in}{3.850000in}}%
\pgfusepath{clip}%
\pgfsetbuttcap%
\pgfsetmiterjoin%
\definecolor{currentfill}{rgb}{0.501961,0.501961,0.501961}%
\pgfsetfillcolor{currentfill}%
\pgfsetfillopacity{0.700000}%
\pgfsetlinewidth{0.000000pt}%
\definecolor{currentstroke}{rgb}{0.000000,0.000000,0.000000}%
\pgfsetstrokecolor{currentstroke}%
\pgfsetstrokeopacity{0.700000}%
\pgfsetdash{}{0pt}%
\pgfpathmoveto{\pgfqpoint{5.235085in}{0.550000in}}%
\pgfpathlineto{\pgfqpoint{5.262607in}{0.550000in}}%
\pgfpathlineto{\pgfqpoint{5.262607in}{0.592564in}}%
\pgfpathlineto{\pgfqpoint{5.235085in}{0.592564in}}%
\pgfpathlineto{\pgfqpoint{5.235085in}{0.550000in}}%
\pgfpathclose%
\pgfusepath{fill}%
\end{pgfscope}%
\begin{pgfscope}%
\pgfpathrectangle{\pgfqpoint{1.250000in}{0.550000in}}{\pgfqpoint{7.750000in}{3.850000in}}%
\pgfusepath{clip}%
\pgfsetbuttcap%
\pgfsetmiterjoin%
\definecolor{currentfill}{rgb}{0.501961,0.501961,0.501961}%
\pgfsetfillcolor{currentfill}%
\pgfsetfillopacity{0.700000}%
\pgfsetlinewidth{0.000000pt}%
\definecolor{currentstroke}{rgb}{0.000000,0.000000,0.000000}%
\pgfsetstrokecolor{currentstroke}%
\pgfsetstrokeopacity{0.700000}%
\pgfsetdash{}{0pt}%
\pgfpathmoveto{\pgfqpoint{5.262607in}{0.550000in}}%
\pgfpathlineto{\pgfqpoint{5.290128in}{0.550000in}}%
\pgfpathlineto{\pgfqpoint{5.290128in}{0.584909in}}%
\pgfpathlineto{\pgfqpoint{5.262607in}{0.584909in}}%
\pgfpathlineto{\pgfqpoint{5.262607in}{0.550000in}}%
\pgfpathclose%
\pgfusepath{fill}%
\end{pgfscope}%
\begin{pgfscope}%
\pgfpathrectangle{\pgfqpoint{1.250000in}{0.550000in}}{\pgfqpoint{7.750000in}{3.850000in}}%
\pgfusepath{clip}%
\pgfsetbuttcap%
\pgfsetmiterjoin%
\definecolor{currentfill}{rgb}{0.501961,0.501961,0.501961}%
\pgfsetfillcolor{currentfill}%
\pgfsetfillopacity{0.700000}%
\pgfsetlinewidth{0.000000pt}%
\definecolor{currentstroke}{rgb}{0.000000,0.000000,0.000000}%
\pgfsetstrokecolor{currentstroke}%
\pgfsetstrokeopacity{0.700000}%
\pgfsetdash{}{0pt}%
\pgfpathmoveto{\pgfqpoint{5.290128in}{0.550000in}}%
\pgfpathlineto{\pgfqpoint{5.317649in}{0.550000in}}%
\pgfpathlineto{\pgfqpoint{5.317649in}{0.579091in}}%
\pgfpathlineto{\pgfqpoint{5.290128in}{0.579091in}}%
\pgfpathlineto{\pgfqpoint{5.290128in}{0.550000in}}%
\pgfpathclose%
\pgfusepath{fill}%
\end{pgfscope}%
\begin{pgfscope}%
\pgfpathrectangle{\pgfqpoint{1.250000in}{0.550000in}}{\pgfqpoint{7.750000in}{3.850000in}}%
\pgfusepath{clip}%
\pgfsetbuttcap%
\pgfsetmiterjoin%
\definecolor{currentfill}{rgb}{0.501961,0.501961,0.501961}%
\pgfsetfillcolor{currentfill}%
\pgfsetfillopacity{0.700000}%
\pgfsetlinewidth{0.000000pt}%
\definecolor{currentstroke}{rgb}{0.000000,0.000000,0.000000}%
\pgfsetstrokecolor{currentstroke}%
\pgfsetstrokeopacity{0.700000}%
\pgfsetdash{}{0pt}%
\pgfpathmoveto{\pgfqpoint{5.317649in}{0.550000in}}%
\pgfpathlineto{\pgfqpoint{5.345170in}{0.550000in}}%
\pgfpathlineto{\pgfqpoint{5.345170in}{0.579550in}}%
\pgfpathlineto{\pgfqpoint{5.317649in}{0.579550in}}%
\pgfpathlineto{\pgfqpoint{5.317649in}{0.550000in}}%
\pgfpathclose%
\pgfusepath{fill}%
\end{pgfscope}%
\begin{pgfscope}%
\pgfpathrectangle{\pgfqpoint{1.250000in}{0.550000in}}{\pgfqpoint{7.750000in}{3.850000in}}%
\pgfusepath{clip}%
\pgfsetbuttcap%
\pgfsetmiterjoin%
\definecolor{currentfill}{rgb}{0.501961,0.501961,0.501961}%
\pgfsetfillcolor{currentfill}%
\pgfsetfillopacity{0.700000}%
\pgfsetlinewidth{0.000000pt}%
\definecolor{currentstroke}{rgb}{0.000000,0.000000,0.000000}%
\pgfsetstrokecolor{currentstroke}%
\pgfsetstrokeopacity{0.700000}%
\pgfsetdash{}{0pt}%
\pgfpathmoveto{\pgfqpoint{5.345170in}{0.550000in}}%
\pgfpathlineto{\pgfqpoint{5.372692in}{0.550000in}}%
\pgfpathlineto{\pgfqpoint{5.372692in}{0.575263in}}%
\pgfpathlineto{\pgfqpoint{5.345170in}{0.575263in}}%
\pgfpathlineto{\pgfqpoint{5.345170in}{0.550000in}}%
\pgfpathclose%
\pgfusepath{fill}%
\end{pgfscope}%
\begin{pgfscope}%
\pgfpathrectangle{\pgfqpoint{1.250000in}{0.550000in}}{\pgfqpoint{7.750000in}{3.850000in}}%
\pgfusepath{clip}%
\pgfsetbuttcap%
\pgfsetmiterjoin%
\definecolor{currentfill}{rgb}{0.501961,0.501961,0.501961}%
\pgfsetfillcolor{currentfill}%
\pgfsetfillopacity{0.700000}%
\pgfsetlinewidth{0.000000pt}%
\definecolor{currentstroke}{rgb}{0.000000,0.000000,0.000000}%
\pgfsetstrokecolor{currentstroke}%
\pgfsetstrokeopacity{0.700000}%
\pgfsetdash{}{0pt}%
\pgfpathmoveto{\pgfqpoint{5.372692in}{0.550000in}}%
\pgfpathlineto{\pgfqpoint{5.400213in}{0.550000in}}%
\pgfpathlineto{\pgfqpoint{5.400213in}{0.574498in}}%
\pgfpathlineto{\pgfqpoint{5.372692in}{0.574498in}}%
\pgfpathlineto{\pgfqpoint{5.372692in}{0.550000in}}%
\pgfpathclose%
\pgfusepath{fill}%
\end{pgfscope}%
\begin{pgfscope}%
\pgfpathrectangle{\pgfqpoint{1.250000in}{0.550000in}}{\pgfqpoint{7.750000in}{3.850000in}}%
\pgfusepath{clip}%
\pgfsetbuttcap%
\pgfsetmiterjoin%
\definecolor{currentfill}{rgb}{0.501961,0.501961,0.501961}%
\pgfsetfillcolor{currentfill}%
\pgfsetfillopacity{0.700000}%
\pgfsetlinewidth{0.000000pt}%
\definecolor{currentstroke}{rgb}{0.000000,0.000000,0.000000}%
\pgfsetstrokecolor{currentstroke}%
\pgfsetstrokeopacity{0.700000}%
\pgfsetdash{}{0pt}%
\pgfpathmoveto{\pgfqpoint{5.400213in}{0.550000in}}%
\pgfpathlineto{\pgfqpoint{5.427734in}{0.550000in}}%
\pgfpathlineto{\pgfqpoint{5.427734in}{0.574804in}}%
\pgfpathlineto{\pgfqpoint{5.400213in}{0.574804in}}%
\pgfpathlineto{\pgfqpoint{5.400213in}{0.550000in}}%
\pgfpathclose%
\pgfusepath{fill}%
\end{pgfscope}%
\begin{pgfscope}%
\pgfpathrectangle{\pgfqpoint{1.250000in}{0.550000in}}{\pgfqpoint{7.750000in}{3.850000in}}%
\pgfusepath{clip}%
\pgfsetbuttcap%
\pgfsetmiterjoin%
\definecolor{currentfill}{rgb}{0.501961,0.501961,0.501961}%
\pgfsetfillcolor{currentfill}%
\pgfsetfillopacity{0.700000}%
\pgfsetlinewidth{0.000000pt}%
\definecolor{currentstroke}{rgb}{0.000000,0.000000,0.000000}%
\pgfsetstrokecolor{currentstroke}%
\pgfsetstrokeopacity{0.700000}%
\pgfsetdash{}{0pt}%
\pgfpathmoveto{\pgfqpoint{5.427734in}{0.550000in}}%
\pgfpathlineto{\pgfqpoint{5.455256in}{0.550000in}}%
\pgfpathlineto{\pgfqpoint{5.455256in}{0.573426in}}%
\pgfpathlineto{\pgfqpoint{5.427734in}{0.573426in}}%
\pgfpathlineto{\pgfqpoint{5.427734in}{0.550000in}}%
\pgfpathclose%
\pgfusepath{fill}%
\end{pgfscope}%
\begin{pgfscope}%
\pgfpathrectangle{\pgfqpoint{1.250000in}{0.550000in}}{\pgfqpoint{7.750000in}{3.850000in}}%
\pgfusepath{clip}%
\pgfsetbuttcap%
\pgfsetmiterjoin%
\definecolor{currentfill}{rgb}{0.501961,0.501961,0.501961}%
\pgfsetfillcolor{currentfill}%
\pgfsetfillopacity{0.700000}%
\pgfsetlinewidth{0.000000pt}%
\definecolor{currentstroke}{rgb}{0.000000,0.000000,0.000000}%
\pgfsetstrokecolor{currentstroke}%
\pgfsetstrokeopacity{0.700000}%
\pgfsetdash{}{0pt}%
\pgfpathmoveto{\pgfqpoint{5.455256in}{0.550000in}}%
\pgfpathlineto{\pgfqpoint{5.482777in}{0.550000in}}%
\pgfpathlineto{\pgfqpoint{5.482777in}{0.568832in}}%
\pgfpathlineto{\pgfqpoint{5.455256in}{0.568832in}}%
\pgfpathlineto{\pgfqpoint{5.455256in}{0.550000in}}%
\pgfpathclose%
\pgfusepath{fill}%
\end{pgfscope}%
\begin{pgfscope}%
\pgfpathrectangle{\pgfqpoint{1.250000in}{0.550000in}}{\pgfqpoint{7.750000in}{3.850000in}}%
\pgfusepath{clip}%
\pgfsetbuttcap%
\pgfsetmiterjoin%
\definecolor{currentfill}{rgb}{0.501961,0.501961,0.501961}%
\pgfsetfillcolor{currentfill}%
\pgfsetfillopacity{0.700000}%
\pgfsetlinewidth{0.000000pt}%
\definecolor{currentstroke}{rgb}{0.000000,0.000000,0.000000}%
\pgfsetstrokecolor{currentstroke}%
\pgfsetstrokeopacity{0.700000}%
\pgfsetdash{}{0pt}%
\pgfpathmoveto{\pgfqpoint{5.482777in}{0.550000in}}%
\pgfpathlineto{\pgfqpoint{5.510298in}{0.550000in}}%
\pgfpathlineto{\pgfqpoint{5.510298in}{0.566995in}}%
\pgfpathlineto{\pgfqpoint{5.482777in}{0.566995in}}%
\pgfpathlineto{\pgfqpoint{5.482777in}{0.550000in}}%
\pgfpathclose%
\pgfusepath{fill}%
\end{pgfscope}%
\begin{pgfscope}%
\pgfpathrectangle{\pgfqpoint{1.250000in}{0.550000in}}{\pgfqpoint{7.750000in}{3.850000in}}%
\pgfusepath{clip}%
\pgfsetbuttcap%
\pgfsetmiterjoin%
\definecolor{currentfill}{rgb}{0.501961,0.501961,0.501961}%
\pgfsetfillcolor{currentfill}%
\pgfsetfillopacity{0.700000}%
\pgfsetlinewidth{0.000000pt}%
\definecolor{currentstroke}{rgb}{0.000000,0.000000,0.000000}%
\pgfsetstrokecolor{currentstroke}%
\pgfsetstrokeopacity{0.700000}%
\pgfsetdash{}{0pt}%
\pgfpathmoveto{\pgfqpoint{5.510298in}{0.550000in}}%
\pgfpathlineto{\pgfqpoint{5.537820in}{0.550000in}}%
\pgfpathlineto{\pgfqpoint{5.537820in}{0.566383in}}%
\pgfpathlineto{\pgfqpoint{5.510298in}{0.566383in}}%
\pgfpathlineto{\pgfqpoint{5.510298in}{0.550000in}}%
\pgfpathclose%
\pgfusepath{fill}%
\end{pgfscope}%
\begin{pgfscope}%
\pgfpathrectangle{\pgfqpoint{1.250000in}{0.550000in}}{\pgfqpoint{7.750000in}{3.850000in}}%
\pgfusepath{clip}%
\pgfsetbuttcap%
\pgfsetmiterjoin%
\definecolor{currentfill}{rgb}{0.501961,0.501961,0.501961}%
\pgfsetfillcolor{currentfill}%
\pgfsetfillopacity{0.700000}%
\pgfsetlinewidth{0.000000pt}%
\definecolor{currentstroke}{rgb}{0.000000,0.000000,0.000000}%
\pgfsetstrokecolor{currentstroke}%
\pgfsetstrokeopacity{0.700000}%
\pgfsetdash{}{0pt}%
\pgfpathmoveto{\pgfqpoint{5.537820in}{0.550000in}}%
\pgfpathlineto{\pgfqpoint{5.565341in}{0.550000in}}%
\pgfpathlineto{\pgfqpoint{5.565341in}{0.561483in}}%
\pgfpathlineto{\pgfqpoint{5.537820in}{0.561483in}}%
\pgfpathlineto{\pgfqpoint{5.537820in}{0.550000in}}%
\pgfpathclose%
\pgfusepath{fill}%
\end{pgfscope}%
\begin{pgfscope}%
\pgfpathrectangle{\pgfqpoint{1.250000in}{0.550000in}}{\pgfqpoint{7.750000in}{3.850000in}}%
\pgfusepath{clip}%
\pgfsetbuttcap%
\pgfsetmiterjoin%
\definecolor{currentfill}{rgb}{0.501961,0.501961,0.501961}%
\pgfsetfillcolor{currentfill}%
\pgfsetfillopacity{0.700000}%
\pgfsetlinewidth{0.000000pt}%
\definecolor{currentstroke}{rgb}{0.000000,0.000000,0.000000}%
\pgfsetstrokecolor{currentstroke}%
\pgfsetstrokeopacity{0.700000}%
\pgfsetdash{}{0pt}%
\pgfpathmoveto{\pgfqpoint{5.565341in}{0.550000in}}%
\pgfpathlineto{\pgfqpoint{5.592862in}{0.550000in}}%
\pgfpathlineto{\pgfqpoint{5.592862in}{0.563167in}}%
\pgfpathlineto{\pgfqpoint{5.565341in}{0.563167in}}%
\pgfpathlineto{\pgfqpoint{5.565341in}{0.550000in}}%
\pgfpathclose%
\pgfusepath{fill}%
\end{pgfscope}%
\begin{pgfscope}%
\pgfpathrectangle{\pgfqpoint{1.250000in}{0.550000in}}{\pgfqpoint{7.750000in}{3.850000in}}%
\pgfusepath{clip}%
\pgfsetbuttcap%
\pgfsetmiterjoin%
\definecolor{currentfill}{rgb}{0.501961,0.501961,0.501961}%
\pgfsetfillcolor{currentfill}%
\pgfsetfillopacity{0.700000}%
\pgfsetlinewidth{0.000000pt}%
\definecolor{currentstroke}{rgb}{0.000000,0.000000,0.000000}%
\pgfsetstrokecolor{currentstroke}%
\pgfsetstrokeopacity{0.700000}%
\pgfsetdash{}{0pt}%
\pgfpathmoveto{\pgfqpoint{5.592862in}{0.550000in}}%
\pgfpathlineto{\pgfqpoint{5.620384in}{0.550000in}}%
\pgfpathlineto{\pgfqpoint{5.620384in}{0.561636in}}%
\pgfpathlineto{\pgfqpoint{5.592862in}{0.561636in}}%
\pgfpathlineto{\pgfqpoint{5.592862in}{0.550000in}}%
\pgfpathclose%
\pgfusepath{fill}%
\end{pgfscope}%
\begin{pgfscope}%
\pgfpathrectangle{\pgfqpoint{1.250000in}{0.550000in}}{\pgfqpoint{7.750000in}{3.850000in}}%
\pgfusepath{clip}%
\pgfsetbuttcap%
\pgfsetmiterjoin%
\definecolor{currentfill}{rgb}{0.501961,0.501961,0.501961}%
\pgfsetfillcolor{currentfill}%
\pgfsetfillopacity{0.700000}%
\pgfsetlinewidth{0.000000pt}%
\definecolor{currentstroke}{rgb}{0.000000,0.000000,0.000000}%
\pgfsetstrokecolor{currentstroke}%
\pgfsetstrokeopacity{0.700000}%
\pgfsetdash{}{0pt}%
\pgfpathmoveto{\pgfqpoint{5.620384in}{0.550000in}}%
\pgfpathlineto{\pgfqpoint{5.647905in}{0.550000in}}%
\pgfpathlineto{\pgfqpoint{5.647905in}{0.560258in}}%
\pgfpathlineto{\pgfqpoint{5.620384in}{0.560258in}}%
\pgfpathlineto{\pgfqpoint{5.620384in}{0.550000in}}%
\pgfpathclose%
\pgfusepath{fill}%
\end{pgfscope}%
\begin{pgfscope}%
\pgfpathrectangle{\pgfqpoint{1.250000in}{0.550000in}}{\pgfqpoint{7.750000in}{3.850000in}}%
\pgfusepath{clip}%
\pgfsetbuttcap%
\pgfsetmiterjoin%
\definecolor{currentfill}{rgb}{0.501961,0.501961,0.501961}%
\pgfsetfillcolor{currentfill}%
\pgfsetfillopacity{0.700000}%
\pgfsetlinewidth{0.000000pt}%
\definecolor{currentstroke}{rgb}{0.000000,0.000000,0.000000}%
\pgfsetstrokecolor{currentstroke}%
\pgfsetstrokeopacity{0.700000}%
\pgfsetdash{}{0pt}%
\pgfpathmoveto{\pgfqpoint{5.647905in}{0.550000in}}%
\pgfpathlineto{\pgfqpoint{5.675426in}{0.550000in}}%
\pgfpathlineto{\pgfqpoint{5.675426in}{0.559799in}}%
\pgfpathlineto{\pgfqpoint{5.647905in}{0.559799in}}%
\pgfpathlineto{\pgfqpoint{5.647905in}{0.550000in}}%
\pgfpathclose%
\pgfusepath{fill}%
\end{pgfscope}%
\begin{pgfscope}%
\pgfpathrectangle{\pgfqpoint{1.250000in}{0.550000in}}{\pgfqpoint{7.750000in}{3.850000in}}%
\pgfusepath{clip}%
\pgfsetbuttcap%
\pgfsetmiterjoin%
\definecolor{currentfill}{rgb}{0.501961,0.501961,0.501961}%
\pgfsetfillcolor{currentfill}%
\pgfsetfillopacity{0.700000}%
\pgfsetlinewidth{0.000000pt}%
\definecolor{currentstroke}{rgb}{0.000000,0.000000,0.000000}%
\pgfsetstrokecolor{currentstroke}%
\pgfsetstrokeopacity{0.700000}%
\pgfsetdash{}{0pt}%
\pgfpathmoveto{\pgfqpoint{5.675426in}{0.550000in}}%
\pgfpathlineto{\pgfqpoint{5.702947in}{0.550000in}}%
\pgfpathlineto{\pgfqpoint{5.702947in}{0.559493in}}%
\pgfpathlineto{\pgfqpoint{5.675426in}{0.559493in}}%
\pgfpathlineto{\pgfqpoint{5.675426in}{0.550000in}}%
\pgfpathclose%
\pgfusepath{fill}%
\end{pgfscope}%
\begin{pgfscope}%
\pgfpathrectangle{\pgfqpoint{1.250000in}{0.550000in}}{\pgfqpoint{7.750000in}{3.850000in}}%
\pgfusepath{clip}%
\pgfsetbuttcap%
\pgfsetmiterjoin%
\definecolor{currentfill}{rgb}{0.501961,0.501961,0.501961}%
\pgfsetfillcolor{currentfill}%
\pgfsetfillopacity{0.700000}%
\pgfsetlinewidth{0.000000pt}%
\definecolor{currentstroke}{rgb}{0.000000,0.000000,0.000000}%
\pgfsetstrokecolor{currentstroke}%
\pgfsetstrokeopacity{0.700000}%
\pgfsetdash{}{0pt}%
\pgfpathmoveto{\pgfqpoint{5.702947in}{0.550000in}}%
\pgfpathlineto{\pgfqpoint{5.730469in}{0.550000in}}%
\pgfpathlineto{\pgfqpoint{5.730469in}{0.559033in}}%
\pgfpathlineto{\pgfqpoint{5.702947in}{0.559033in}}%
\pgfpathlineto{\pgfqpoint{5.702947in}{0.550000in}}%
\pgfpathclose%
\pgfusepath{fill}%
\end{pgfscope}%
\begin{pgfscope}%
\pgfpathrectangle{\pgfqpoint{1.250000in}{0.550000in}}{\pgfqpoint{7.750000in}{3.850000in}}%
\pgfusepath{clip}%
\pgfsetbuttcap%
\pgfsetmiterjoin%
\definecolor{currentfill}{rgb}{0.501961,0.501961,0.501961}%
\pgfsetfillcolor{currentfill}%
\pgfsetfillopacity{0.700000}%
\pgfsetlinewidth{0.000000pt}%
\definecolor{currentstroke}{rgb}{0.000000,0.000000,0.000000}%
\pgfsetstrokecolor{currentstroke}%
\pgfsetstrokeopacity{0.700000}%
\pgfsetdash{}{0pt}%
\pgfpathmoveto{\pgfqpoint{5.730469in}{0.550000in}}%
\pgfpathlineto{\pgfqpoint{5.757990in}{0.550000in}}%
\pgfpathlineto{\pgfqpoint{5.757990in}{0.556124in}}%
\pgfpathlineto{\pgfqpoint{5.730469in}{0.556124in}}%
\pgfpathlineto{\pgfqpoint{5.730469in}{0.550000in}}%
\pgfpathclose%
\pgfusepath{fill}%
\end{pgfscope}%
\begin{pgfscope}%
\pgfpathrectangle{\pgfqpoint{1.250000in}{0.550000in}}{\pgfqpoint{7.750000in}{3.850000in}}%
\pgfusepath{clip}%
\pgfsetbuttcap%
\pgfsetmiterjoin%
\definecolor{currentfill}{rgb}{0.501961,0.501961,0.501961}%
\pgfsetfillcolor{currentfill}%
\pgfsetfillopacity{0.700000}%
\pgfsetlinewidth{0.000000pt}%
\definecolor{currentstroke}{rgb}{0.000000,0.000000,0.000000}%
\pgfsetstrokecolor{currentstroke}%
\pgfsetstrokeopacity{0.700000}%
\pgfsetdash{}{0pt}%
\pgfpathmoveto{\pgfqpoint{5.757990in}{0.550000in}}%
\pgfpathlineto{\pgfqpoint{5.785511in}{0.550000in}}%
\pgfpathlineto{\pgfqpoint{5.785511in}{0.557502in}}%
\pgfpathlineto{\pgfqpoint{5.757990in}{0.557502in}}%
\pgfpathlineto{\pgfqpoint{5.757990in}{0.550000in}}%
\pgfpathclose%
\pgfusepath{fill}%
\end{pgfscope}%
\begin{pgfscope}%
\pgfpathrectangle{\pgfqpoint{1.250000in}{0.550000in}}{\pgfqpoint{7.750000in}{3.850000in}}%
\pgfusepath{clip}%
\pgfsetbuttcap%
\pgfsetmiterjoin%
\definecolor{currentfill}{rgb}{0.501961,0.501961,0.501961}%
\pgfsetfillcolor{currentfill}%
\pgfsetfillopacity{0.700000}%
\pgfsetlinewidth{0.000000pt}%
\definecolor{currentstroke}{rgb}{0.000000,0.000000,0.000000}%
\pgfsetstrokecolor{currentstroke}%
\pgfsetstrokeopacity{0.700000}%
\pgfsetdash{}{0pt}%
\pgfpathmoveto{\pgfqpoint{5.785511in}{0.550000in}}%
\pgfpathlineto{\pgfqpoint{5.813033in}{0.550000in}}%
\pgfpathlineto{\pgfqpoint{5.813033in}{0.556890in}}%
\pgfpathlineto{\pgfqpoint{5.785511in}{0.556890in}}%
\pgfpathlineto{\pgfqpoint{5.785511in}{0.550000in}}%
\pgfpathclose%
\pgfusepath{fill}%
\end{pgfscope}%
\begin{pgfscope}%
\pgfpathrectangle{\pgfqpoint{1.250000in}{0.550000in}}{\pgfqpoint{7.750000in}{3.850000in}}%
\pgfusepath{clip}%
\pgfsetbuttcap%
\pgfsetmiterjoin%
\definecolor{currentfill}{rgb}{0.501961,0.501961,0.501961}%
\pgfsetfillcolor{currentfill}%
\pgfsetfillopacity{0.700000}%
\pgfsetlinewidth{0.000000pt}%
\definecolor{currentstroke}{rgb}{0.000000,0.000000,0.000000}%
\pgfsetstrokecolor{currentstroke}%
\pgfsetstrokeopacity{0.700000}%
\pgfsetdash{}{0pt}%
\pgfpathmoveto{\pgfqpoint{5.813033in}{0.550000in}}%
\pgfpathlineto{\pgfqpoint{5.840554in}{0.550000in}}%
\pgfpathlineto{\pgfqpoint{5.840554in}{0.558574in}}%
\pgfpathlineto{\pgfqpoint{5.813033in}{0.558574in}}%
\pgfpathlineto{\pgfqpoint{5.813033in}{0.550000in}}%
\pgfpathclose%
\pgfusepath{fill}%
\end{pgfscope}%
\begin{pgfscope}%
\pgfpathrectangle{\pgfqpoint{1.250000in}{0.550000in}}{\pgfqpoint{7.750000in}{3.850000in}}%
\pgfusepath{clip}%
\pgfsetbuttcap%
\pgfsetmiterjoin%
\definecolor{currentfill}{rgb}{0.501961,0.501961,0.501961}%
\pgfsetfillcolor{currentfill}%
\pgfsetfillopacity{0.700000}%
\pgfsetlinewidth{0.000000pt}%
\definecolor{currentstroke}{rgb}{0.000000,0.000000,0.000000}%
\pgfsetstrokecolor{currentstroke}%
\pgfsetstrokeopacity{0.700000}%
\pgfsetdash{}{0pt}%
\pgfpathmoveto{\pgfqpoint{5.840554in}{0.550000in}}%
\pgfpathlineto{\pgfqpoint{5.868075in}{0.550000in}}%
\pgfpathlineto{\pgfqpoint{5.868075in}{0.556431in}}%
\pgfpathlineto{\pgfqpoint{5.840554in}{0.556431in}}%
\pgfpathlineto{\pgfqpoint{5.840554in}{0.550000in}}%
\pgfpathclose%
\pgfusepath{fill}%
\end{pgfscope}%
\begin{pgfscope}%
\pgfpathrectangle{\pgfqpoint{1.250000in}{0.550000in}}{\pgfqpoint{7.750000in}{3.850000in}}%
\pgfusepath{clip}%
\pgfsetbuttcap%
\pgfsetmiterjoin%
\definecolor{currentfill}{rgb}{0.501961,0.501961,0.501961}%
\pgfsetfillcolor{currentfill}%
\pgfsetfillopacity{0.700000}%
\pgfsetlinewidth{0.000000pt}%
\definecolor{currentstroke}{rgb}{0.000000,0.000000,0.000000}%
\pgfsetstrokecolor{currentstroke}%
\pgfsetstrokeopacity{0.700000}%
\pgfsetdash{}{0pt}%
\pgfpathmoveto{\pgfqpoint{5.868075in}{0.550000in}}%
\pgfpathlineto{\pgfqpoint{5.895597in}{0.550000in}}%
\pgfpathlineto{\pgfqpoint{5.895597in}{0.558115in}}%
\pgfpathlineto{\pgfqpoint{5.868075in}{0.558115in}}%
\pgfpathlineto{\pgfqpoint{5.868075in}{0.550000in}}%
\pgfpathclose%
\pgfusepath{fill}%
\end{pgfscope}%
\begin{pgfscope}%
\pgfpathrectangle{\pgfqpoint{1.250000in}{0.550000in}}{\pgfqpoint{7.750000in}{3.850000in}}%
\pgfusepath{clip}%
\pgfsetbuttcap%
\pgfsetmiterjoin%
\definecolor{currentfill}{rgb}{0.501961,0.501961,0.501961}%
\pgfsetfillcolor{currentfill}%
\pgfsetfillopacity{0.700000}%
\pgfsetlinewidth{0.000000pt}%
\definecolor{currentstroke}{rgb}{0.000000,0.000000,0.000000}%
\pgfsetstrokecolor{currentstroke}%
\pgfsetstrokeopacity{0.700000}%
\pgfsetdash{}{0pt}%
\pgfpathmoveto{\pgfqpoint{5.895597in}{0.550000in}}%
\pgfpathlineto{\pgfqpoint{5.923118in}{0.550000in}}%
\pgfpathlineto{\pgfqpoint{5.923118in}{0.557502in}}%
\pgfpathlineto{\pgfqpoint{5.895597in}{0.557502in}}%
\pgfpathlineto{\pgfqpoint{5.895597in}{0.550000in}}%
\pgfpathclose%
\pgfusepath{fill}%
\end{pgfscope}%
\begin{pgfscope}%
\pgfpathrectangle{\pgfqpoint{1.250000in}{0.550000in}}{\pgfqpoint{7.750000in}{3.850000in}}%
\pgfusepath{clip}%
\pgfsetbuttcap%
\pgfsetmiterjoin%
\definecolor{currentfill}{rgb}{0.501961,0.501961,0.501961}%
\pgfsetfillcolor{currentfill}%
\pgfsetfillopacity{0.700000}%
\pgfsetlinewidth{0.000000pt}%
\definecolor{currentstroke}{rgb}{0.000000,0.000000,0.000000}%
\pgfsetstrokecolor{currentstroke}%
\pgfsetstrokeopacity{0.700000}%
\pgfsetdash{}{0pt}%
\pgfpathmoveto{\pgfqpoint{5.923118in}{0.550000in}}%
\pgfpathlineto{\pgfqpoint{5.950639in}{0.550000in}}%
\pgfpathlineto{\pgfqpoint{5.950639in}{0.556584in}}%
\pgfpathlineto{\pgfqpoint{5.923118in}{0.556584in}}%
\pgfpathlineto{\pgfqpoint{5.923118in}{0.550000in}}%
\pgfpathclose%
\pgfusepath{fill}%
\end{pgfscope}%
\begin{pgfscope}%
\pgfpathrectangle{\pgfqpoint{1.250000in}{0.550000in}}{\pgfqpoint{7.750000in}{3.850000in}}%
\pgfusepath{clip}%
\pgfsetbuttcap%
\pgfsetmiterjoin%
\definecolor{currentfill}{rgb}{0.501961,0.501961,0.501961}%
\pgfsetfillcolor{currentfill}%
\pgfsetfillopacity{0.700000}%
\pgfsetlinewidth{0.000000pt}%
\definecolor{currentstroke}{rgb}{0.000000,0.000000,0.000000}%
\pgfsetstrokecolor{currentstroke}%
\pgfsetstrokeopacity{0.700000}%
\pgfsetdash{}{0pt}%
\pgfpathmoveto{\pgfqpoint{5.950639in}{0.550000in}}%
\pgfpathlineto{\pgfqpoint{5.978161in}{0.550000in}}%
\pgfpathlineto{\pgfqpoint{5.978161in}{0.555818in}}%
\pgfpathlineto{\pgfqpoint{5.950639in}{0.555818in}}%
\pgfpathlineto{\pgfqpoint{5.950639in}{0.550000in}}%
\pgfpathclose%
\pgfusepath{fill}%
\end{pgfscope}%
\begin{pgfscope}%
\pgfpathrectangle{\pgfqpoint{1.250000in}{0.550000in}}{\pgfqpoint{7.750000in}{3.850000in}}%
\pgfusepath{clip}%
\pgfsetbuttcap%
\pgfsetmiterjoin%
\definecolor{currentfill}{rgb}{0.501961,0.501961,0.501961}%
\pgfsetfillcolor{currentfill}%
\pgfsetfillopacity{0.700000}%
\pgfsetlinewidth{0.000000pt}%
\definecolor{currentstroke}{rgb}{0.000000,0.000000,0.000000}%
\pgfsetstrokecolor{currentstroke}%
\pgfsetstrokeopacity{0.700000}%
\pgfsetdash{}{0pt}%
\pgfpathmoveto{\pgfqpoint{5.978161in}{0.550000in}}%
\pgfpathlineto{\pgfqpoint{6.005682in}{0.550000in}}%
\pgfpathlineto{\pgfqpoint{6.005682in}{0.556890in}}%
\pgfpathlineto{\pgfqpoint{5.978161in}{0.556890in}}%
\pgfpathlineto{\pgfqpoint{5.978161in}{0.550000in}}%
\pgfpathclose%
\pgfusepath{fill}%
\end{pgfscope}%
\begin{pgfscope}%
\pgfpathrectangle{\pgfqpoint{1.250000in}{0.550000in}}{\pgfqpoint{7.750000in}{3.850000in}}%
\pgfusepath{clip}%
\pgfsetbuttcap%
\pgfsetmiterjoin%
\definecolor{currentfill}{rgb}{0.501961,0.501961,0.501961}%
\pgfsetfillcolor{currentfill}%
\pgfsetfillopacity{0.700000}%
\pgfsetlinewidth{0.000000pt}%
\definecolor{currentstroke}{rgb}{0.000000,0.000000,0.000000}%
\pgfsetstrokecolor{currentstroke}%
\pgfsetstrokeopacity{0.700000}%
\pgfsetdash{}{0pt}%
\pgfpathmoveto{\pgfqpoint{6.005682in}{0.550000in}}%
\pgfpathlineto{\pgfqpoint{6.033203in}{0.550000in}}%
\pgfpathlineto{\pgfqpoint{6.033203in}{0.555206in}}%
\pgfpathlineto{\pgfqpoint{6.005682in}{0.555206in}}%
\pgfpathlineto{\pgfqpoint{6.005682in}{0.550000in}}%
\pgfpathclose%
\pgfusepath{fill}%
\end{pgfscope}%
\begin{pgfscope}%
\pgfpathrectangle{\pgfqpoint{1.250000in}{0.550000in}}{\pgfqpoint{7.750000in}{3.850000in}}%
\pgfusepath{clip}%
\pgfsetbuttcap%
\pgfsetmiterjoin%
\definecolor{currentfill}{rgb}{0.501961,0.501961,0.501961}%
\pgfsetfillcolor{currentfill}%
\pgfsetfillopacity{0.700000}%
\pgfsetlinewidth{0.000000pt}%
\definecolor{currentstroke}{rgb}{0.000000,0.000000,0.000000}%
\pgfsetstrokecolor{currentstroke}%
\pgfsetstrokeopacity{0.700000}%
\pgfsetdash{}{0pt}%
\pgfpathmoveto{\pgfqpoint{6.033203in}{0.550000in}}%
\pgfpathlineto{\pgfqpoint{6.060724in}{0.550000in}}%
\pgfpathlineto{\pgfqpoint{6.060724in}{0.556584in}}%
\pgfpathlineto{\pgfqpoint{6.033203in}{0.556584in}}%
\pgfpathlineto{\pgfqpoint{6.033203in}{0.550000in}}%
\pgfpathclose%
\pgfusepath{fill}%
\end{pgfscope}%
\begin{pgfscope}%
\pgfpathrectangle{\pgfqpoint{1.250000in}{0.550000in}}{\pgfqpoint{7.750000in}{3.850000in}}%
\pgfusepath{clip}%
\pgfsetbuttcap%
\pgfsetmiterjoin%
\definecolor{currentfill}{rgb}{0.501961,0.501961,0.501961}%
\pgfsetfillcolor{currentfill}%
\pgfsetfillopacity{0.700000}%
\pgfsetlinewidth{0.000000pt}%
\definecolor{currentstroke}{rgb}{0.000000,0.000000,0.000000}%
\pgfsetstrokecolor{currentstroke}%
\pgfsetstrokeopacity{0.700000}%
\pgfsetdash{}{0pt}%
\pgfpathmoveto{\pgfqpoint{6.060724in}{0.550000in}}%
\pgfpathlineto{\pgfqpoint{6.088246in}{0.550000in}}%
\pgfpathlineto{\pgfqpoint{6.088246in}{0.557502in}}%
\pgfpathlineto{\pgfqpoint{6.060724in}{0.557502in}}%
\pgfpathlineto{\pgfqpoint{6.060724in}{0.550000in}}%
\pgfpathclose%
\pgfusepath{fill}%
\end{pgfscope}%
\begin{pgfscope}%
\pgfpathrectangle{\pgfqpoint{1.250000in}{0.550000in}}{\pgfqpoint{7.750000in}{3.850000in}}%
\pgfusepath{clip}%
\pgfsetbuttcap%
\pgfsetmiterjoin%
\definecolor{currentfill}{rgb}{0.501961,0.501961,0.501961}%
\pgfsetfillcolor{currentfill}%
\pgfsetfillopacity{0.700000}%
\pgfsetlinewidth{0.000000pt}%
\definecolor{currentstroke}{rgb}{0.000000,0.000000,0.000000}%
\pgfsetstrokecolor{currentstroke}%
\pgfsetstrokeopacity{0.700000}%
\pgfsetdash{}{0pt}%
\pgfpathmoveto{\pgfqpoint{6.088246in}{0.550000in}}%
\pgfpathlineto{\pgfqpoint{6.115767in}{0.550000in}}%
\pgfpathlineto{\pgfqpoint{6.115767in}{0.555053in}}%
\pgfpathlineto{\pgfqpoint{6.088246in}{0.555053in}}%
\pgfpathlineto{\pgfqpoint{6.088246in}{0.550000in}}%
\pgfpathclose%
\pgfusepath{fill}%
\end{pgfscope}%
\begin{pgfscope}%
\pgfpathrectangle{\pgfqpoint{1.250000in}{0.550000in}}{\pgfqpoint{7.750000in}{3.850000in}}%
\pgfusepath{clip}%
\pgfsetbuttcap%
\pgfsetmiterjoin%
\definecolor{currentfill}{rgb}{0.501961,0.501961,0.501961}%
\pgfsetfillcolor{currentfill}%
\pgfsetfillopacity{0.700000}%
\pgfsetlinewidth{0.000000pt}%
\definecolor{currentstroke}{rgb}{0.000000,0.000000,0.000000}%
\pgfsetstrokecolor{currentstroke}%
\pgfsetstrokeopacity{0.700000}%
\pgfsetdash{}{0pt}%
\pgfpathmoveto{\pgfqpoint{6.115767in}{0.550000in}}%
\pgfpathlineto{\pgfqpoint{6.143288in}{0.550000in}}%
\pgfpathlineto{\pgfqpoint{6.143288in}{0.556890in}}%
\pgfpathlineto{\pgfqpoint{6.115767in}{0.556890in}}%
\pgfpathlineto{\pgfqpoint{6.115767in}{0.550000in}}%
\pgfpathclose%
\pgfusepath{fill}%
\end{pgfscope}%
\begin{pgfscope}%
\pgfpathrectangle{\pgfqpoint{1.250000in}{0.550000in}}{\pgfqpoint{7.750000in}{3.850000in}}%
\pgfusepath{clip}%
\pgfsetbuttcap%
\pgfsetmiterjoin%
\definecolor{currentfill}{rgb}{0.501961,0.501961,0.501961}%
\pgfsetfillcolor{currentfill}%
\pgfsetfillopacity{0.700000}%
\pgfsetlinewidth{0.000000pt}%
\definecolor{currentstroke}{rgb}{0.000000,0.000000,0.000000}%
\pgfsetstrokecolor{currentstroke}%
\pgfsetstrokeopacity{0.700000}%
\pgfsetdash{}{0pt}%
\pgfpathmoveto{\pgfqpoint{6.143288in}{0.550000in}}%
\pgfpathlineto{\pgfqpoint{6.170810in}{0.550000in}}%
\pgfpathlineto{\pgfqpoint{6.170810in}{0.553981in}}%
\pgfpathlineto{\pgfqpoint{6.143288in}{0.553981in}}%
\pgfpathlineto{\pgfqpoint{6.143288in}{0.550000in}}%
\pgfpathclose%
\pgfusepath{fill}%
\end{pgfscope}%
\begin{pgfscope}%
\pgfpathrectangle{\pgfqpoint{1.250000in}{0.550000in}}{\pgfqpoint{7.750000in}{3.850000in}}%
\pgfusepath{clip}%
\pgfsetbuttcap%
\pgfsetmiterjoin%
\definecolor{currentfill}{rgb}{0.501961,0.501961,0.501961}%
\pgfsetfillcolor{currentfill}%
\pgfsetfillopacity{0.700000}%
\pgfsetlinewidth{0.000000pt}%
\definecolor{currentstroke}{rgb}{0.000000,0.000000,0.000000}%
\pgfsetstrokecolor{currentstroke}%
\pgfsetstrokeopacity{0.700000}%
\pgfsetdash{}{0pt}%
\pgfpathmoveto{\pgfqpoint{6.170810in}{0.550000in}}%
\pgfpathlineto{\pgfqpoint{6.198331in}{0.550000in}}%
\pgfpathlineto{\pgfqpoint{6.198331in}{0.554900in}}%
\pgfpathlineto{\pgfqpoint{6.170810in}{0.554900in}}%
\pgfpathlineto{\pgfqpoint{6.170810in}{0.550000in}}%
\pgfpathclose%
\pgfusepath{fill}%
\end{pgfscope}%
\begin{pgfscope}%
\pgfpathrectangle{\pgfqpoint{1.250000in}{0.550000in}}{\pgfqpoint{7.750000in}{3.850000in}}%
\pgfusepath{clip}%
\pgfsetbuttcap%
\pgfsetmiterjoin%
\definecolor{currentfill}{rgb}{0.501961,0.501961,0.501961}%
\pgfsetfillcolor{currentfill}%
\pgfsetfillopacity{0.700000}%
\pgfsetlinewidth{0.000000pt}%
\definecolor{currentstroke}{rgb}{0.000000,0.000000,0.000000}%
\pgfsetstrokecolor{currentstroke}%
\pgfsetstrokeopacity{0.700000}%
\pgfsetdash{}{0pt}%
\pgfpathmoveto{\pgfqpoint{6.198331in}{0.550000in}}%
\pgfpathlineto{\pgfqpoint{6.225852in}{0.550000in}}%
\pgfpathlineto{\pgfqpoint{6.225852in}{0.555359in}}%
\pgfpathlineto{\pgfqpoint{6.198331in}{0.555359in}}%
\pgfpathlineto{\pgfqpoint{6.198331in}{0.550000in}}%
\pgfpathclose%
\pgfusepath{fill}%
\end{pgfscope}%
\begin{pgfscope}%
\pgfpathrectangle{\pgfqpoint{1.250000in}{0.550000in}}{\pgfqpoint{7.750000in}{3.850000in}}%
\pgfusepath{clip}%
\pgfsetbuttcap%
\pgfsetmiterjoin%
\definecolor{currentfill}{rgb}{0.501961,0.501961,0.501961}%
\pgfsetfillcolor{currentfill}%
\pgfsetfillopacity{0.700000}%
\pgfsetlinewidth{0.000000pt}%
\definecolor{currentstroke}{rgb}{0.000000,0.000000,0.000000}%
\pgfsetstrokecolor{currentstroke}%
\pgfsetstrokeopacity{0.700000}%
\pgfsetdash{}{0pt}%
\pgfpathmoveto{\pgfqpoint{6.225852in}{0.550000in}}%
\pgfpathlineto{\pgfqpoint{6.253374in}{0.550000in}}%
\pgfpathlineto{\pgfqpoint{6.253374in}{0.557043in}}%
\pgfpathlineto{\pgfqpoint{6.225852in}{0.557043in}}%
\pgfpathlineto{\pgfqpoint{6.225852in}{0.550000in}}%
\pgfpathclose%
\pgfusepath{fill}%
\end{pgfscope}%
\begin{pgfscope}%
\pgfpathrectangle{\pgfqpoint{1.250000in}{0.550000in}}{\pgfqpoint{7.750000in}{3.850000in}}%
\pgfusepath{clip}%
\pgfsetbuttcap%
\pgfsetmiterjoin%
\definecolor{currentfill}{rgb}{0.501961,0.501961,0.501961}%
\pgfsetfillcolor{currentfill}%
\pgfsetfillopacity{0.700000}%
\pgfsetlinewidth{0.000000pt}%
\definecolor{currentstroke}{rgb}{0.000000,0.000000,0.000000}%
\pgfsetstrokecolor{currentstroke}%
\pgfsetstrokeopacity{0.700000}%
\pgfsetdash{}{0pt}%
\pgfpathmoveto{\pgfqpoint{6.253374in}{0.550000in}}%
\pgfpathlineto{\pgfqpoint{6.280895in}{0.550000in}}%
\pgfpathlineto{\pgfqpoint{6.280895in}{0.555512in}}%
\pgfpathlineto{\pgfqpoint{6.253374in}{0.555512in}}%
\pgfpathlineto{\pgfqpoint{6.253374in}{0.550000in}}%
\pgfpathclose%
\pgfusepath{fill}%
\end{pgfscope}%
\begin{pgfscope}%
\pgfpathrectangle{\pgfqpoint{1.250000in}{0.550000in}}{\pgfqpoint{7.750000in}{3.850000in}}%
\pgfusepath{clip}%
\pgfsetbuttcap%
\pgfsetmiterjoin%
\definecolor{currentfill}{rgb}{0.501961,0.501961,0.501961}%
\pgfsetfillcolor{currentfill}%
\pgfsetfillopacity{0.700000}%
\pgfsetlinewidth{0.000000pt}%
\definecolor{currentstroke}{rgb}{0.000000,0.000000,0.000000}%
\pgfsetstrokecolor{currentstroke}%
\pgfsetstrokeopacity{0.700000}%
\pgfsetdash{}{0pt}%
\pgfpathmoveto{\pgfqpoint{6.280895in}{0.550000in}}%
\pgfpathlineto{\pgfqpoint{6.308416in}{0.550000in}}%
\pgfpathlineto{\pgfqpoint{6.308416in}{0.554746in}}%
\pgfpathlineto{\pgfqpoint{6.280895in}{0.554746in}}%
\pgfpathlineto{\pgfqpoint{6.280895in}{0.550000in}}%
\pgfpathclose%
\pgfusepath{fill}%
\end{pgfscope}%
\begin{pgfscope}%
\pgfpathrectangle{\pgfqpoint{1.250000in}{0.550000in}}{\pgfqpoint{7.750000in}{3.850000in}}%
\pgfusepath{clip}%
\pgfsetbuttcap%
\pgfsetmiterjoin%
\definecolor{currentfill}{rgb}{0.501961,0.501961,0.501961}%
\pgfsetfillcolor{currentfill}%
\pgfsetfillopacity{0.700000}%
\pgfsetlinewidth{0.000000pt}%
\definecolor{currentstroke}{rgb}{0.000000,0.000000,0.000000}%
\pgfsetstrokecolor{currentstroke}%
\pgfsetstrokeopacity{0.700000}%
\pgfsetdash{}{0pt}%
\pgfpathmoveto{\pgfqpoint{6.308416in}{0.550000in}}%
\pgfpathlineto{\pgfqpoint{6.335937in}{0.550000in}}%
\pgfpathlineto{\pgfqpoint{6.335937in}{0.553675in}}%
\pgfpathlineto{\pgfqpoint{6.308416in}{0.553675in}}%
\pgfpathlineto{\pgfqpoint{6.308416in}{0.550000in}}%
\pgfpathclose%
\pgfusepath{fill}%
\end{pgfscope}%
\begin{pgfscope}%
\pgfpathrectangle{\pgfqpoint{1.250000in}{0.550000in}}{\pgfqpoint{7.750000in}{3.850000in}}%
\pgfusepath{clip}%
\pgfsetbuttcap%
\pgfsetmiterjoin%
\definecolor{currentfill}{rgb}{0.501961,0.501961,0.501961}%
\pgfsetfillcolor{currentfill}%
\pgfsetfillopacity{0.700000}%
\pgfsetlinewidth{0.000000pt}%
\definecolor{currentstroke}{rgb}{0.000000,0.000000,0.000000}%
\pgfsetstrokecolor{currentstroke}%
\pgfsetstrokeopacity{0.700000}%
\pgfsetdash{}{0pt}%
\pgfpathmoveto{\pgfqpoint{6.335938in}{0.550000in}}%
\pgfpathlineto{\pgfqpoint{6.363459in}{0.550000in}}%
\pgfpathlineto{\pgfqpoint{6.363459in}{0.553368in}}%
\pgfpathlineto{\pgfqpoint{6.335938in}{0.553368in}}%
\pgfpathlineto{\pgfqpoint{6.335938in}{0.550000in}}%
\pgfpathclose%
\pgfusepath{fill}%
\end{pgfscope}%
\begin{pgfscope}%
\pgfpathrectangle{\pgfqpoint{1.250000in}{0.550000in}}{\pgfqpoint{7.750000in}{3.850000in}}%
\pgfusepath{clip}%
\pgfsetbuttcap%
\pgfsetmiterjoin%
\definecolor{currentfill}{rgb}{0.501961,0.501961,0.501961}%
\pgfsetfillcolor{currentfill}%
\pgfsetfillopacity{0.700000}%
\pgfsetlinewidth{0.000000pt}%
\definecolor{currentstroke}{rgb}{0.000000,0.000000,0.000000}%
\pgfsetstrokecolor{currentstroke}%
\pgfsetstrokeopacity{0.700000}%
\pgfsetdash{}{0pt}%
\pgfpathmoveto{\pgfqpoint{6.363459in}{0.550000in}}%
\pgfpathlineto{\pgfqpoint{6.390980in}{0.550000in}}%
\pgfpathlineto{\pgfqpoint{6.390980in}{0.554746in}}%
\pgfpathlineto{\pgfqpoint{6.363459in}{0.554746in}}%
\pgfpathlineto{\pgfqpoint{6.363459in}{0.550000in}}%
\pgfpathclose%
\pgfusepath{fill}%
\end{pgfscope}%
\begin{pgfscope}%
\pgfpathrectangle{\pgfqpoint{1.250000in}{0.550000in}}{\pgfqpoint{7.750000in}{3.850000in}}%
\pgfusepath{clip}%
\pgfsetbuttcap%
\pgfsetmiterjoin%
\definecolor{currentfill}{rgb}{0.501961,0.501961,0.501961}%
\pgfsetfillcolor{currentfill}%
\pgfsetfillopacity{0.700000}%
\pgfsetlinewidth{0.000000pt}%
\definecolor{currentstroke}{rgb}{0.000000,0.000000,0.000000}%
\pgfsetstrokecolor{currentstroke}%
\pgfsetstrokeopacity{0.700000}%
\pgfsetdash{}{0pt}%
\pgfpathmoveto{\pgfqpoint{6.390980in}{0.550000in}}%
\pgfpathlineto{\pgfqpoint{6.418501in}{0.550000in}}%
\pgfpathlineto{\pgfqpoint{6.418501in}{0.554746in}}%
\pgfpathlineto{\pgfqpoint{6.390980in}{0.554746in}}%
\pgfpathlineto{\pgfqpoint{6.390980in}{0.550000in}}%
\pgfpathclose%
\pgfusepath{fill}%
\end{pgfscope}%
\begin{pgfscope}%
\pgfpathrectangle{\pgfqpoint{1.250000in}{0.550000in}}{\pgfqpoint{7.750000in}{3.850000in}}%
\pgfusepath{clip}%
\pgfsetbuttcap%
\pgfsetmiterjoin%
\definecolor{currentfill}{rgb}{0.501961,0.501961,0.501961}%
\pgfsetfillcolor{currentfill}%
\pgfsetfillopacity{0.700000}%
\pgfsetlinewidth{0.000000pt}%
\definecolor{currentstroke}{rgb}{0.000000,0.000000,0.000000}%
\pgfsetstrokecolor{currentstroke}%
\pgfsetstrokeopacity{0.700000}%
\pgfsetdash{}{0pt}%
\pgfpathmoveto{\pgfqpoint{6.418501in}{0.550000in}}%
\pgfpathlineto{\pgfqpoint{6.446023in}{0.550000in}}%
\pgfpathlineto{\pgfqpoint{6.446023in}{0.555818in}}%
\pgfpathlineto{\pgfqpoint{6.418501in}{0.555818in}}%
\pgfpathlineto{\pgfqpoint{6.418501in}{0.550000in}}%
\pgfpathclose%
\pgfusepath{fill}%
\end{pgfscope}%
\begin{pgfscope}%
\pgfpathrectangle{\pgfqpoint{1.250000in}{0.550000in}}{\pgfqpoint{7.750000in}{3.850000in}}%
\pgfusepath{clip}%
\pgfsetbuttcap%
\pgfsetmiterjoin%
\definecolor{currentfill}{rgb}{0.501961,0.501961,0.501961}%
\pgfsetfillcolor{currentfill}%
\pgfsetfillopacity{0.700000}%
\pgfsetlinewidth{0.000000pt}%
\definecolor{currentstroke}{rgb}{0.000000,0.000000,0.000000}%
\pgfsetstrokecolor{currentstroke}%
\pgfsetstrokeopacity{0.700000}%
\pgfsetdash{}{0pt}%
\pgfpathmoveto{\pgfqpoint{6.446023in}{0.550000in}}%
\pgfpathlineto{\pgfqpoint{6.473544in}{0.550000in}}%
\pgfpathlineto{\pgfqpoint{6.473544in}{0.552756in}}%
\pgfpathlineto{\pgfqpoint{6.446023in}{0.552756in}}%
\pgfpathlineto{\pgfqpoint{6.446023in}{0.550000in}}%
\pgfpathclose%
\pgfusepath{fill}%
\end{pgfscope}%
\begin{pgfscope}%
\pgfpathrectangle{\pgfqpoint{1.250000in}{0.550000in}}{\pgfqpoint{7.750000in}{3.850000in}}%
\pgfusepath{clip}%
\pgfsetbuttcap%
\pgfsetmiterjoin%
\definecolor{currentfill}{rgb}{0.501961,0.501961,0.501961}%
\pgfsetfillcolor{currentfill}%
\pgfsetfillopacity{0.700000}%
\pgfsetlinewidth{0.000000pt}%
\definecolor{currentstroke}{rgb}{0.000000,0.000000,0.000000}%
\pgfsetstrokecolor{currentstroke}%
\pgfsetstrokeopacity{0.700000}%
\pgfsetdash{}{0pt}%
\pgfpathmoveto{\pgfqpoint{6.473544in}{0.550000in}}%
\pgfpathlineto{\pgfqpoint{6.501065in}{0.550000in}}%
\pgfpathlineto{\pgfqpoint{6.501065in}{0.553675in}}%
\pgfpathlineto{\pgfqpoint{6.473544in}{0.553675in}}%
\pgfpathlineto{\pgfqpoint{6.473544in}{0.550000in}}%
\pgfpathclose%
\pgfusepath{fill}%
\end{pgfscope}%
\begin{pgfscope}%
\pgfpathrectangle{\pgfqpoint{1.250000in}{0.550000in}}{\pgfqpoint{7.750000in}{3.850000in}}%
\pgfusepath{clip}%
\pgfsetbuttcap%
\pgfsetmiterjoin%
\definecolor{currentfill}{rgb}{0.501961,0.501961,0.501961}%
\pgfsetfillcolor{currentfill}%
\pgfsetfillopacity{0.700000}%
\pgfsetlinewidth{0.000000pt}%
\definecolor{currentstroke}{rgb}{0.000000,0.000000,0.000000}%
\pgfsetstrokecolor{currentstroke}%
\pgfsetstrokeopacity{0.700000}%
\pgfsetdash{}{0pt}%
\pgfpathmoveto{\pgfqpoint{6.501065in}{0.550000in}}%
\pgfpathlineto{\pgfqpoint{6.528587in}{0.550000in}}%
\pgfpathlineto{\pgfqpoint{6.528587in}{0.554746in}}%
\pgfpathlineto{\pgfqpoint{6.501065in}{0.554746in}}%
\pgfpathlineto{\pgfqpoint{6.501065in}{0.550000in}}%
\pgfpathclose%
\pgfusepath{fill}%
\end{pgfscope}%
\begin{pgfscope}%
\pgfpathrectangle{\pgfqpoint{1.250000in}{0.550000in}}{\pgfqpoint{7.750000in}{3.850000in}}%
\pgfusepath{clip}%
\pgfsetbuttcap%
\pgfsetmiterjoin%
\definecolor{currentfill}{rgb}{0.501961,0.501961,0.501961}%
\pgfsetfillcolor{currentfill}%
\pgfsetfillopacity{0.700000}%
\pgfsetlinewidth{0.000000pt}%
\definecolor{currentstroke}{rgb}{0.000000,0.000000,0.000000}%
\pgfsetstrokecolor{currentstroke}%
\pgfsetstrokeopacity{0.700000}%
\pgfsetdash{}{0pt}%
\pgfpathmoveto{\pgfqpoint{6.528587in}{0.550000in}}%
\pgfpathlineto{\pgfqpoint{6.556108in}{0.550000in}}%
\pgfpathlineto{\pgfqpoint{6.556108in}{0.553828in}}%
\pgfpathlineto{\pgfqpoint{6.528587in}{0.553828in}}%
\pgfpathlineto{\pgfqpoint{6.528587in}{0.550000in}}%
\pgfpathclose%
\pgfusepath{fill}%
\end{pgfscope}%
\begin{pgfscope}%
\pgfpathrectangle{\pgfqpoint{1.250000in}{0.550000in}}{\pgfqpoint{7.750000in}{3.850000in}}%
\pgfusepath{clip}%
\pgfsetbuttcap%
\pgfsetmiterjoin%
\definecolor{currentfill}{rgb}{0.501961,0.501961,0.501961}%
\pgfsetfillcolor{currentfill}%
\pgfsetfillopacity{0.700000}%
\pgfsetlinewidth{0.000000pt}%
\definecolor{currentstroke}{rgb}{0.000000,0.000000,0.000000}%
\pgfsetstrokecolor{currentstroke}%
\pgfsetstrokeopacity{0.700000}%
\pgfsetdash{}{0pt}%
\pgfpathmoveto{\pgfqpoint{6.556108in}{0.550000in}}%
\pgfpathlineto{\pgfqpoint{6.583629in}{0.550000in}}%
\pgfpathlineto{\pgfqpoint{6.583629in}{0.553215in}}%
\pgfpathlineto{\pgfqpoint{6.556108in}{0.553215in}}%
\pgfpathlineto{\pgfqpoint{6.556108in}{0.550000in}}%
\pgfpathclose%
\pgfusepath{fill}%
\end{pgfscope}%
\begin{pgfscope}%
\pgfpathrectangle{\pgfqpoint{1.250000in}{0.550000in}}{\pgfqpoint{7.750000in}{3.850000in}}%
\pgfusepath{clip}%
\pgfsetbuttcap%
\pgfsetmiterjoin%
\definecolor{currentfill}{rgb}{0.501961,0.501961,0.501961}%
\pgfsetfillcolor{currentfill}%
\pgfsetfillopacity{0.700000}%
\pgfsetlinewidth{0.000000pt}%
\definecolor{currentstroke}{rgb}{0.000000,0.000000,0.000000}%
\pgfsetstrokecolor{currentstroke}%
\pgfsetstrokeopacity{0.700000}%
\pgfsetdash{}{0pt}%
\pgfpathmoveto{\pgfqpoint{6.583629in}{0.550000in}}%
\pgfpathlineto{\pgfqpoint{6.611151in}{0.550000in}}%
\pgfpathlineto{\pgfqpoint{6.611151in}{0.552603in}}%
\pgfpathlineto{\pgfqpoint{6.583629in}{0.552603in}}%
\pgfpathlineto{\pgfqpoint{6.583629in}{0.550000in}}%
\pgfpathclose%
\pgfusepath{fill}%
\end{pgfscope}%
\begin{pgfscope}%
\pgfpathrectangle{\pgfqpoint{1.250000in}{0.550000in}}{\pgfqpoint{7.750000in}{3.850000in}}%
\pgfusepath{clip}%
\pgfsetbuttcap%
\pgfsetmiterjoin%
\definecolor{currentfill}{rgb}{0.501961,0.501961,0.501961}%
\pgfsetfillcolor{currentfill}%
\pgfsetfillopacity{0.700000}%
\pgfsetlinewidth{0.000000pt}%
\definecolor{currentstroke}{rgb}{0.000000,0.000000,0.000000}%
\pgfsetstrokecolor{currentstroke}%
\pgfsetstrokeopacity{0.700000}%
\pgfsetdash{}{0pt}%
\pgfpathmoveto{\pgfqpoint{6.611151in}{0.550000in}}%
\pgfpathlineto{\pgfqpoint{6.638672in}{0.550000in}}%
\pgfpathlineto{\pgfqpoint{6.638672in}{0.552297in}}%
\pgfpathlineto{\pgfqpoint{6.611151in}{0.552297in}}%
\pgfpathlineto{\pgfqpoint{6.611151in}{0.550000in}}%
\pgfpathclose%
\pgfusepath{fill}%
\end{pgfscope}%
\begin{pgfscope}%
\pgfpathrectangle{\pgfqpoint{1.250000in}{0.550000in}}{\pgfqpoint{7.750000in}{3.850000in}}%
\pgfusepath{clip}%
\pgfsetbuttcap%
\pgfsetmiterjoin%
\definecolor{currentfill}{rgb}{0.501961,0.501961,0.501961}%
\pgfsetfillcolor{currentfill}%
\pgfsetfillopacity{0.700000}%
\pgfsetlinewidth{0.000000pt}%
\definecolor{currentstroke}{rgb}{0.000000,0.000000,0.000000}%
\pgfsetstrokecolor{currentstroke}%
\pgfsetstrokeopacity{0.700000}%
\pgfsetdash{}{0pt}%
\pgfpathmoveto{\pgfqpoint{6.638672in}{0.550000in}}%
\pgfpathlineto{\pgfqpoint{6.666193in}{0.550000in}}%
\pgfpathlineto{\pgfqpoint{6.666193in}{0.552603in}}%
\pgfpathlineto{\pgfqpoint{6.638672in}{0.552603in}}%
\pgfpathlineto{\pgfqpoint{6.638672in}{0.550000in}}%
\pgfpathclose%
\pgfusepath{fill}%
\end{pgfscope}%
\begin{pgfscope}%
\pgfpathrectangle{\pgfqpoint{1.250000in}{0.550000in}}{\pgfqpoint{7.750000in}{3.850000in}}%
\pgfusepath{clip}%
\pgfsetbuttcap%
\pgfsetmiterjoin%
\definecolor{currentfill}{rgb}{0.501961,0.501961,0.501961}%
\pgfsetfillcolor{currentfill}%
\pgfsetfillopacity{0.700000}%
\pgfsetlinewidth{0.000000pt}%
\definecolor{currentstroke}{rgb}{0.000000,0.000000,0.000000}%
\pgfsetstrokecolor{currentstroke}%
\pgfsetstrokeopacity{0.700000}%
\pgfsetdash{}{0pt}%
\pgfpathmoveto{\pgfqpoint{6.666193in}{0.550000in}}%
\pgfpathlineto{\pgfqpoint{6.693714in}{0.550000in}}%
\pgfpathlineto{\pgfqpoint{6.693714in}{0.552450in}}%
\pgfpathlineto{\pgfqpoint{6.666193in}{0.552450in}}%
\pgfpathlineto{\pgfqpoint{6.666193in}{0.550000in}}%
\pgfpathclose%
\pgfusepath{fill}%
\end{pgfscope}%
\begin{pgfscope}%
\pgfpathrectangle{\pgfqpoint{1.250000in}{0.550000in}}{\pgfqpoint{7.750000in}{3.850000in}}%
\pgfusepath{clip}%
\pgfsetbuttcap%
\pgfsetmiterjoin%
\definecolor{currentfill}{rgb}{0.501961,0.501961,0.501961}%
\pgfsetfillcolor{currentfill}%
\pgfsetfillopacity{0.700000}%
\pgfsetlinewidth{0.000000pt}%
\definecolor{currentstroke}{rgb}{0.000000,0.000000,0.000000}%
\pgfsetstrokecolor{currentstroke}%
\pgfsetstrokeopacity{0.700000}%
\pgfsetdash{}{0pt}%
\pgfpathmoveto{\pgfqpoint{6.693714in}{0.550000in}}%
\pgfpathlineto{\pgfqpoint{6.721236in}{0.550000in}}%
\pgfpathlineto{\pgfqpoint{6.721236in}{0.553981in}}%
\pgfpathlineto{\pgfqpoint{6.693714in}{0.553981in}}%
\pgfpathlineto{\pgfqpoint{6.693714in}{0.550000in}}%
\pgfpathclose%
\pgfusepath{fill}%
\end{pgfscope}%
\begin{pgfscope}%
\pgfpathrectangle{\pgfqpoint{1.250000in}{0.550000in}}{\pgfqpoint{7.750000in}{3.850000in}}%
\pgfusepath{clip}%
\pgfsetbuttcap%
\pgfsetmiterjoin%
\definecolor{currentfill}{rgb}{0.501961,0.501961,0.501961}%
\pgfsetfillcolor{currentfill}%
\pgfsetfillopacity{0.700000}%
\pgfsetlinewidth{0.000000pt}%
\definecolor{currentstroke}{rgb}{0.000000,0.000000,0.000000}%
\pgfsetstrokecolor{currentstroke}%
\pgfsetstrokeopacity{0.700000}%
\pgfsetdash{}{0pt}%
\pgfpathmoveto{\pgfqpoint{6.721236in}{0.550000in}}%
\pgfpathlineto{\pgfqpoint{6.748757in}{0.550000in}}%
\pgfpathlineto{\pgfqpoint{6.748757in}{0.553368in}}%
\pgfpathlineto{\pgfqpoint{6.721236in}{0.553368in}}%
\pgfpathlineto{\pgfqpoint{6.721236in}{0.550000in}}%
\pgfpathclose%
\pgfusepath{fill}%
\end{pgfscope}%
\begin{pgfscope}%
\pgfpathrectangle{\pgfqpoint{1.250000in}{0.550000in}}{\pgfqpoint{7.750000in}{3.850000in}}%
\pgfusepath{clip}%
\pgfsetbuttcap%
\pgfsetmiterjoin%
\definecolor{currentfill}{rgb}{0.501961,0.501961,0.501961}%
\pgfsetfillcolor{currentfill}%
\pgfsetfillopacity{0.700000}%
\pgfsetlinewidth{0.000000pt}%
\definecolor{currentstroke}{rgb}{0.000000,0.000000,0.000000}%
\pgfsetstrokecolor{currentstroke}%
\pgfsetstrokeopacity{0.700000}%
\pgfsetdash{}{0pt}%
\pgfpathmoveto{\pgfqpoint{6.748757in}{0.550000in}}%
\pgfpathlineto{\pgfqpoint{6.776278in}{0.550000in}}%
\pgfpathlineto{\pgfqpoint{6.776278in}{0.553062in}}%
\pgfpathlineto{\pgfqpoint{6.748757in}{0.553062in}}%
\pgfpathlineto{\pgfqpoint{6.748757in}{0.550000in}}%
\pgfpathclose%
\pgfusepath{fill}%
\end{pgfscope}%
\begin{pgfscope}%
\pgfpathrectangle{\pgfqpoint{1.250000in}{0.550000in}}{\pgfqpoint{7.750000in}{3.850000in}}%
\pgfusepath{clip}%
\pgfsetbuttcap%
\pgfsetmiterjoin%
\definecolor{currentfill}{rgb}{0.501961,0.501961,0.501961}%
\pgfsetfillcolor{currentfill}%
\pgfsetfillopacity{0.700000}%
\pgfsetlinewidth{0.000000pt}%
\definecolor{currentstroke}{rgb}{0.000000,0.000000,0.000000}%
\pgfsetstrokecolor{currentstroke}%
\pgfsetstrokeopacity{0.700000}%
\pgfsetdash{}{0pt}%
\pgfpathmoveto{\pgfqpoint{6.776278in}{0.550000in}}%
\pgfpathlineto{\pgfqpoint{6.803800in}{0.550000in}}%
\pgfpathlineto{\pgfqpoint{6.803800in}{0.551684in}}%
\pgfpathlineto{\pgfqpoint{6.776278in}{0.551684in}}%
\pgfpathlineto{\pgfqpoint{6.776278in}{0.550000in}}%
\pgfpathclose%
\pgfusepath{fill}%
\end{pgfscope}%
\begin{pgfscope}%
\pgfpathrectangle{\pgfqpoint{1.250000in}{0.550000in}}{\pgfqpoint{7.750000in}{3.850000in}}%
\pgfusepath{clip}%
\pgfsetbuttcap%
\pgfsetmiterjoin%
\definecolor{currentfill}{rgb}{0.501961,0.501961,0.501961}%
\pgfsetfillcolor{currentfill}%
\pgfsetfillopacity{0.700000}%
\pgfsetlinewidth{0.000000pt}%
\definecolor{currentstroke}{rgb}{0.000000,0.000000,0.000000}%
\pgfsetstrokecolor{currentstroke}%
\pgfsetstrokeopacity{0.700000}%
\pgfsetdash{}{0pt}%
\pgfpathmoveto{\pgfqpoint{6.803800in}{0.550000in}}%
\pgfpathlineto{\pgfqpoint{6.831321in}{0.550000in}}%
\pgfpathlineto{\pgfqpoint{6.831321in}{0.552450in}}%
\pgfpathlineto{\pgfqpoint{6.803800in}{0.552450in}}%
\pgfpathlineto{\pgfqpoint{6.803800in}{0.550000in}}%
\pgfpathclose%
\pgfusepath{fill}%
\end{pgfscope}%
\begin{pgfscope}%
\pgfpathrectangle{\pgfqpoint{1.250000in}{0.550000in}}{\pgfqpoint{7.750000in}{3.850000in}}%
\pgfusepath{clip}%
\pgfsetbuttcap%
\pgfsetmiterjoin%
\definecolor{currentfill}{rgb}{0.501961,0.501961,0.501961}%
\pgfsetfillcolor{currentfill}%
\pgfsetfillopacity{0.700000}%
\pgfsetlinewidth{0.000000pt}%
\definecolor{currentstroke}{rgb}{0.000000,0.000000,0.000000}%
\pgfsetstrokecolor{currentstroke}%
\pgfsetstrokeopacity{0.700000}%
\pgfsetdash{}{0pt}%
\pgfpathmoveto{\pgfqpoint{6.831321in}{0.550000in}}%
\pgfpathlineto{\pgfqpoint{6.858842in}{0.550000in}}%
\pgfpathlineto{\pgfqpoint{6.858842in}{0.551378in}}%
\pgfpathlineto{\pgfqpoint{6.831321in}{0.551378in}}%
\pgfpathlineto{\pgfqpoint{6.831321in}{0.550000in}}%
\pgfpathclose%
\pgfusepath{fill}%
\end{pgfscope}%
\begin{pgfscope}%
\pgfpathrectangle{\pgfqpoint{1.250000in}{0.550000in}}{\pgfqpoint{7.750000in}{3.850000in}}%
\pgfusepath{clip}%
\pgfsetbuttcap%
\pgfsetmiterjoin%
\definecolor{currentfill}{rgb}{0.501961,0.501961,0.501961}%
\pgfsetfillcolor{currentfill}%
\pgfsetfillopacity{0.700000}%
\pgfsetlinewidth{0.000000pt}%
\definecolor{currentstroke}{rgb}{0.000000,0.000000,0.000000}%
\pgfsetstrokecolor{currentstroke}%
\pgfsetstrokeopacity{0.700000}%
\pgfsetdash{}{0pt}%
\pgfpathmoveto{\pgfqpoint{6.858842in}{0.550000in}}%
\pgfpathlineto{\pgfqpoint{6.886364in}{0.550000in}}%
\pgfpathlineto{\pgfqpoint{6.886364in}{0.551684in}}%
\pgfpathlineto{\pgfqpoint{6.858842in}{0.551684in}}%
\pgfpathlineto{\pgfqpoint{6.858842in}{0.550000in}}%
\pgfpathclose%
\pgfusepath{fill}%
\end{pgfscope}%
\begin{pgfscope}%
\pgfpathrectangle{\pgfqpoint{1.250000in}{0.550000in}}{\pgfqpoint{7.750000in}{3.850000in}}%
\pgfusepath{clip}%
\pgfsetbuttcap%
\pgfsetmiterjoin%
\definecolor{currentfill}{rgb}{0.501961,0.501961,0.501961}%
\pgfsetfillcolor{currentfill}%
\pgfsetfillopacity{0.700000}%
\pgfsetlinewidth{0.000000pt}%
\definecolor{currentstroke}{rgb}{0.000000,0.000000,0.000000}%
\pgfsetstrokecolor{currentstroke}%
\pgfsetstrokeopacity{0.700000}%
\pgfsetdash{}{0pt}%
\pgfpathmoveto{\pgfqpoint{6.886364in}{0.550000in}}%
\pgfpathlineto{\pgfqpoint{6.913885in}{0.550000in}}%
\pgfpathlineto{\pgfqpoint{6.913885in}{0.551990in}}%
\pgfpathlineto{\pgfqpoint{6.886364in}{0.551990in}}%
\pgfpathlineto{\pgfqpoint{6.886364in}{0.550000in}}%
\pgfpathclose%
\pgfusepath{fill}%
\end{pgfscope}%
\begin{pgfscope}%
\pgfpathrectangle{\pgfqpoint{1.250000in}{0.550000in}}{\pgfqpoint{7.750000in}{3.850000in}}%
\pgfusepath{clip}%
\pgfsetbuttcap%
\pgfsetmiterjoin%
\definecolor{currentfill}{rgb}{0.501961,0.501961,0.501961}%
\pgfsetfillcolor{currentfill}%
\pgfsetfillopacity{0.700000}%
\pgfsetlinewidth{0.000000pt}%
\definecolor{currentstroke}{rgb}{0.000000,0.000000,0.000000}%
\pgfsetstrokecolor{currentstroke}%
\pgfsetstrokeopacity{0.700000}%
\pgfsetdash{}{0pt}%
\pgfpathmoveto{\pgfqpoint{6.913885in}{0.550000in}}%
\pgfpathlineto{\pgfqpoint{6.941406in}{0.550000in}}%
\pgfpathlineto{\pgfqpoint{6.941406in}{0.552909in}}%
\pgfpathlineto{\pgfqpoint{6.913885in}{0.552909in}}%
\pgfpathlineto{\pgfqpoint{6.913885in}{0.550000in}}%
\pgfpathclose%
\pgfusepath{fill}%
\end{pgfscope}%
\begin{pgfscope}%
\pgfpathrectangle{\pgfqpoint{1.250000in}{0.550000in}}{\pgfqpoint{7.750000in}{3.850000in}}%
\pgfusepath{clip}%
\pgfsetbuttcap%
\pgfsetmiterjoin%
\definecolor{currentfill}{rgb}{0.501961,0.501961,0.501961}%
\pgfsetfillcolor{currentfill}%
\pgfsetfillopacity{0.700000}%
\pgfsetlinewidth{0.000000pt}%
\definecolor{currentstroke}{rgb}{0.000000,0.000000,0.000000}%
\pgfsetstrokecolor{currentstroke}%
\pgfsetstrokeopacity{0.700000}%
\pgfsetdash{}{0pt}%
\pgfpathmoveto{\pgfqpoint{6.941406in}{0.550000in}}%
\pgfpathlineto{\pgfqpoint{6.968928in}{0.550000in}}%
\pgfpathlineto{\pgfqpoint{6.968928in}{0.552450in}}%
\pgfpathlineto{\pgfqpoint{6.941406in}{0.552450in}}%
\pgfpathlineto{\pgfqpoint{6.941406in}{0.550000in}}%
\pgfpathclose%
\pgfusepath{fill}%
\end{pgfscope}%
\begin{pgfscope}%
\pgfpathrectangle{\pgfqpoint{1.250000in}{0.550000in}}{\pgfqpoint{7.750000in}{3.850000in}}%
\pgfusepath{clip}%
\pgfsetbuttcap%
\pgfsetmiterjoin%
\definecolor{currentfill}{rgb}{0.501961,0.501961,0.501961}%
\pgfsetfillcolor{currentfill}%
\pgfsetfillopacity{0.700000}%
\pgfsetlinewidth{0.000000pt}%
\definecolor{currentstroke}{rgb}{0.000000,0.000000,0.000000}%
\pgfsetstrokecolor{currentstroke}%
\pgfsetstrokeopacity{0.700000}%
\pgfsetdash{}{0pt}%
\pgfpathmoveto{\pgfqpoint{6.968928in}{0.550000in}}%
\pgfpathlineto{\pgfqpoint{6.996449in}{0.550000in}}%
\pgfpathlineto{\pgfqpoint{6.996449in}{0.551072in}}%
\pgfpathlineto{\pgfqpoint{6.968928in}{0.551072in}}%
\pgfpathlineto{\pgfqpoint{6.968928in}{0.550000in}}%
\pgfpathclose%
\pgfusepath{fill}%
\end{pgfscope}%
\begin{pgfscope}%
\pgfpathrectangle{\pgfqpoint{1.250000in}{0.550000in}}{\pgfqpoint{7.750000in}{3.850000in}}%
\pgfusepath{clip}%
\pgfsetbuttcap%
\pgfsetmiterjoin%
\definecolor{currentfill}{rgb}{0.501961,0.501961,0.501961}%
\pgfsetfillcolor{currentfill}%
\pgfsetfillopacity{0.700000}%
\pgfsetlinewidth{0.000000pt}%
\definecolor{currentstroke}{rgb}{0.000000,0.000000,0.000000}%
\pgfsetstrokecolor{currentstroke}%
\pgfsetstrokeopacity{0.700000}%
\pgfsetdash{}{0pt}%
\pgfpathmoveto{\pgfqpoint{6.996449in}{0.550000in}}%
\pgfpathlineto{\pgfqpoint{7.023970in}{0.550000in}}%
\pgfpathlineto{\pgfqpoint{7.023970in}{0.552297in}}%
\pgfpathlineto{\pgfqpoint{6.996449in}{0.552297in}}%
\pgfpathlineto{\pgfqpoint{6.996449in}{0.550000in}}%
\pgfpathclose%
\pgfusepath{fill}%
\end{pgfscope}%
\begin{pgfscope}%
\pgfpathrectangle{\pgfqpoint{1.250000in}{0.550000in}}{\pgfqpoint{7.750000in}{3.850000in}}%
\pgfusepath{clip}%
\pgfsetbuttcap%
\pgfsetmiterjoin%
\definecolor{currentfill}{rgb}{0.501961,0.501961,0.501961}%
\pgfsetfillcolor{currentfill}%
\pgfsetfillopacity{0.700000}%
\pgfsetlinewidth{0.000000pt}%
\definecolor{currentstroke}{rgb}{0.000000,0.000000,0.000000}%
\pgfsetstrokecolor{currentstroke}%
\pgfsetstrokeopacity{0.700000}%
\pgfsetdash{}{0pt}%
\pgfpathmoveto{\pgfqpoint{7.023970in}{0.550000in}}%
\pgfpathlineto{\pgfqpoint{7.051491in}{0.550000in}}%
\pgfpathlineto{\pgfqpoint{7.051491in}{0.552144in}}%
\pgfpathlineto{\pgfqpoint{7.023970in}{0.552144in}}%
\pgfpathlineto{\pgfqpoint{7.023970in}{0.550000in}}%
\pgfpathclose%
\pgfusepath{fill}%
\end{pgfscope}%
\begin{pgfscope}%
\pgfpathrectangle{\pgfqpoint{1.250000in}{0.550000in}}{\pgfqpoint{7.750000in}{3.850000in}}%
\pgfusepath{clip}%
\pgfsetbuttcap%
\pgfsetmiterjoin%
\definecolor{currentfill}{rgb}{0.501961,0.501961,0.501961}%
\pgfsetfillcolor{currentfill}%
\pgfsetfillopacity{0.700000}%
\pgfsetlinewidth{0.000000pt}%
\definecolor{currentstroke}{rgb}{0.000000,0.000000,0.000000}%
\pgfsetstrokecolor{currentstroke}%
\pgfsetstrokeopacity{0.700000}%
\pgfsetdash{}{0pt}%
\pgfpathmoveto{\pgfqpoint{7.051491in}{0.550000in}}%
\pgfpathlineto{\pgfqpoint{7.079013in}{0.550000in}}%
\pgfpathlineto{\pgfqpoint{7.079013in}{0.552756in}}%
\pgfpathlineto{\pgfqpoint{7.051491in}{0.552756in}}%
\pgfpathlineto{\pgfqpoint{7.051491in}{0.550000in}}%
\pgfpathclose%
\pgfusepath{fill}%
\end{pgfscope}%
\begin{pgfscope}%
\pgfpathrectangle{\pgfqpoint{1.250000in}{0.550000in}}{\pgfqpoint{7.750000in}{3.850000in}}%
\pgfusepath{clip}%
\pgfsetbuttcap%
\pgfsetmiterjoin%
\definecolor{currentfill}{rgb}{0.501961,0.501961,0.501961}%
\pgfsetfillcolor{currentfill}%
\pgfsetfillopacity{0.700000}%
\pgfsetlinewidth{0.000000pt}%
\definecolor{currentstroke}{rgb}{0.000000,0.000000,0.000000}%
\pgfsetstrokecolor{currentstroke}%
\pgfsetstrokeopacity{0.700000}%
\pgfsetdash{}{0pt}%
\pgfpathmoveto{\pgfqpoint{7.079013in}{0.550000in}}%
\pgfpathlineto{\pgfqpoint{7.106534in}{0.550000in}}%
\pgfpathlineto{\pgfqpoint{7.106534in}{0.551684in}}%
\pgfpathlineto{\pgfqpoint{7.079013in}{0.551684in}}%
\pgfpathlineto{\pgfqpoint{7.079013in}{0.550000in}}%
\pgfpathclose%
\pgfusepath{fill}%
\end{pgfscope}%
\begin{pgfscope}%
\pgfpathrectangle{\pgfqpoint{1.250000in}{0.550000in}}{\pgfqpoint{7.750000in}{3.850000in}}%
\pgfusepath{clip}%
\pgfsetbuttcap%
\pgfsetmiterjoin%
\definecolor{currentfill}{rgb}{0.501961,0.501961,0.501961}%
\pgfsetfillcolor{currentfill}%
\pgfsetfillopacity{0.700000}%
\pgfsetlinewidth{0.000000pt}%
\definecolor{currentstroke}{rgb}{0.000000,0.000000,0.000000}%
\pgfsetstrokecolor{currentstroke}%
\pgfsetstrokeopacity{0.700000}%
\pgfsetdash{}{0pt}%
\pgfpathmoveto{\pgfqpoint{7.106534in}{0.550000in}}%
\pgfpathlineto{\pgfqpoint{7.134055in}{0.550000in}}%
\pgfpathlineto{\pgfqpoint{7.134055in}{0.551837in}}%
\pgfpathlineto{\pgfqpoint{7.106534in}{0.551837in}}%
\pgfpathlineto{\pgfqpoint{7.106534in}{0.550000in}}%
\pgfpathclose%
\pgfusepath{fill}%
\end{pgfscope}%
\begin{pgfscope}%
\pgfpathrectangle{\pgfqpoint{1.250000in}{0.550000in}}{\pgfqpoint{7.750000in}{3.850000in}}%
\pgfusepath{clip}%
\pgfsetbuttcap%
\pgfsetmiterjoin%
\definecolor{currentfill}{rgb}{0.501961,0.501961,0.501961}%
\pgfsetfillcolor{currentfill}%
\pgfsetfillopacity{0.700000}%
\pgfsetlinewidth{0.000000pt}%
\definecolor{currentstroke}{rgb}{0.000000,0.000000,0.000000}%
\pgfsetstrokecolor{currentstroke}%
\pgfsetstrokeopacity{0.700000}%
\pgfsetdash{}{0pt}%
\pgfpathmoveto{\pgfqpoint{7.134055in}{0.550000in}}%
\pgfpathlineto{\pgfqpoint{7.161577in}{0.550000in}}%
\pgfpathlineto{\pgfqpoint{7.161577in}{0.551072in}}%
\pgfpathlineto{\pgfqpoint{7.134055in}{0.551072in}}%
\pgfpathlineto{\pgfqpoint{7.134055in}{0.550000in}}%
\pgfpathclose%
\pgfusepath{fill}%
\end{pgfscope}%
\begin{pgfscope}%
\pgfpathrectangle{\pgfqpoint{1.250000in}{0.550000in}}{\pgfqpoint{7.750000in}{3.850000in}}%
\pgfusepath{clip}%
\pgfsetbuttcap%
\pgfsetmiterjoin%
\definecolor{currentfill}{rgb}{0.501961,0.501961,0.501961}%
\pgfsetfillcolor{currentfill}%
\pgfsetfillopacity{0.700000}%
\pgfsetlinewidth{0.000000pt}%
\definecolor{currentstroke}{rgb}{0.000000,0.000000,0.000000}%
\pgfsetstrokecolor{currentstroke}%
\pgfsetstrokeopacity{0.700000}%
\pgfsetdash{}{0pt}%
\pgfpathmoveto{\pgfqpoint{7.161577in}{0.550000in}}%
\pgfpathlineto{\pgfqpoint{7.189098in}{0.550000in}}%
\pgfpathlineto{\pgfqpoint{7.189098in}{0.551225in}}%
\pgfpathlineto{\pgfqpoint{7.161577in}{0.551225in}}%
\pgfpathlineto{\pgfqpoint{7.161577in}{0.550000in}}%
\pgfpathclose%
\pgfusepath{fill}%
\end{pgfscope}%
\begin{pgfscope}%
\pgfpathrectangle{\pgfqpoint{1.250000in}{0.550000in}}{\pgfqpoint{7.750000in}{3.850000in}}%
\pgfusepath{clip}%
\pgfsetbuttcap%
\pgfsetmiterjoin%
\definecolor{currentfill}{rgb}{0.501961,0.501961,0.501961}%
\pgfsetfillcolor{currentfill}%
\pgfsetfillopacity{0.700000}%
\pgfsetlinewidth{0.000000pt}%
\definecolor{currentstroke}{rgb}{0.000000,0.000000,0.000000}%
\pgfsetstrokecolor{currentstroke}%
\pgfsetstrokeopacity{0.700000}%
\pgfsetdash{}{0pt}%
\pgfpathmoveto{\pgfqpoint{7.189098in}{0.550000in}}%
\pgfpathlineto{\pgfqpoint{7.216619in}{0.550000in}}%
\pgfpathlineto{\pgfqpoint{7.216619in}{0.550459in}}%
\pgfpathlineto{\pgfqpoint{7.189098in}{0.550459in}}%
\pgfpathlineto{\pgfqpoint{7.189098in}{0.550000in}}%
\pgfpathclose%
\pgfusepath{fill}%
\end{pgfscope}%
\begin{pgfscope}%
\pgfpathrectangle{\pgfqpoint{1.250000in}{0.550000in}}{\pgfqpoint{7.750000in}{3.850000in}}%
\pgfusepath{clip}%
\pgfsetbuttcap%
\pgfsetmiterjoin%
\definecolor{currentfill}{rgb}{0.501961,0.501961,0.501961}%
\pgfsetfillcolor{currentfill}%
\pgfsetfillopacity{0.700000}%
\pgfsetlinewidth{0.000000pt}%
\definecolor{currentstroke}{rgb}{0.000000,0.000000,0.000000}%
\pgfsetstrokecolor{currentstroke}%
\pgfsetstrokeopacity{0.700000}%
\pgfsetdash{}{0pt}%
\pgfpathmoveto{\pgfqpoint{7.216619in}{0.550000in}}%
\pgfpathlineto{\pgfqpoint{7.244141in}{0.550000in}}%
\pgfpathlineto{\pgfqpoint{7.244141in}{0.550919in}}%
\pgfpathlineto{\pgfqpoint{7.216619in}{0.550919in}}%
\pgfpathlineto{\pgfqpoint{7.216619in}{0.550000in}}%
\pgfpathclose%
\pgfusepath{fill}%
\end{pgfscope}%
\begin{pgfscope}%
\pgfpathrectangle{\pgfqpoint{1.250000in}{0.550000in}}{\pgfqpoint{7.750000in}{3.850000in}}%
\pgfusepath{clip}%
\pgfsetbuttcap%
\pgfsetmiterjoin%
\definecolor{currentfill}{rgb}{0.501961,0.501961,0.501961}%
\pgfsetfillcolor{currentfill}%
\pgfsetfillopacity{0.700000}%
\pgfsetlinewidth{0.000000pt}%
\definecolor{currentstroke}{rgb}{0.000000,0.000000,0.000000}%
\pgfsetstrokecolor{currentstroke}%
\pgfsetstrokeopacity{0.700000}%
\pgfsetdash{}{0pt}%
\pgfpathmoveto{\pgfqpoint{7.244141in}{0.550000in}}%
\pgfpathlineto{\pgfqpoint{7.271662in}{0.550000in}}%
\pgfpathlineto{\pgfqpoint{7.271662in}{0.550000in}}%
\pgfpathlineto{\pgfqpoint{7.244141in}{0.550000in}}%
\pgfpathlineto{\pgfqpoint{7.244141in}{0.550000in}}%
\pgfpathclose%
\pgfusepath{fill}%
\end{pgfscope}%
\begin{pgfscope}%
\pgfpathrectangle{\pgfqpoint{1.250000in}{0.550000in}}{\pgfqpoint{7.750000in}{3.850000in}}%
\pgfusepath{clip}%
\pgfsetbuttcap%
\pgfsetmiterjoin%
\definecolor{currentfill}{rgb}{0.501961,0.501961,0.501961}%
\pgfsetfillcolor{currentfill}%
\pgfsetfillopacity{0.700000}%
\pgfsetlinewidth{0.000000pt}%
\definecolor{currentstroke}{rgb}{0.000000,0.000000,0.000000}%
\pgfsetstrokecolor{currentstroke}%
\pgfsetstrokeopacity{0.700000}%
\pgfsetdash{}{0pt}%
\pgfpathmoveto{\pgfqpoint{7.271662in}{0.550000in}}%
\pgfpathlineto{\pgfqpoint{7.299183in}{0.550000in}}%
\pgfpathlineto{\pgfqpoint{7.299183in}{0.550153in}}%
\pgfpathlineto{\pgfqpoint{7.271662in}{0.550153in}}%
\pgfpathlineto{\pgfqpoint{7.271662in}{0.550000in}}%
\pgfpathclose%
\pgfusepath{fill}%
\end{pgfscope}%
\begin{pgfscope}%
\pgfpathrectangle{\pgfqpoint{1.250000in}{0.550000in}}{\pgfqpoint{7.750000in}{3.850000in}}%
\pgfusepath{clip}%
\pgfsetbuttcap%
\pgfsetmiterjoin%
\definecolor{currentfill}{rgb}{0.501961,0.501961,0.501961}%
\pgfsetfillcolor{currentfill}%
\pgfsetfillopacity{0.700000}%
\pgfsetlinewidth{0.000000pt}%
\definecolor{currentstroke}{rgb}{0.000000,0.000000,0.000000}%
\pgfsetstrokecolor{currentstroke}%
\pgfsetstrokeopacity{0.700000}%
\pgfsetdash{}{0pt}%
\pgfpathmoveto{\pgfqpoint{7.299183in}{0.550000in}}%
\pgfpathlineto{\pgfqpoint{7.326705in}{0.550000in}}%
\pgfpathlineto{\pgfqpoint{7.326705in}{0.550306in}}%
\pgfpathlineto{\pgfqpoint{7.299183in}{0.550306in}}%
\pgfpathlineto{\pgfqpoint{7.299183in}{0.550000in}}%
\pgfpathclose%
\pgfusepath{fill}%
\end{pgfscope}%
\begin{pgfscope}%
\pgfpathrectangle{\pgfqpoint{1.250000in}{0.550000in}}{\pgfqpoint{7.750000in}{3.850000in}}%
\pgfusepath{clip}%
\pgfsetbuttcap%
\pgfsetmiterjoin%
\definecolor{currentfill}{rgb}{0.501961,0.501961,0.501961}%
\pgfsetfillcolor{currentfill}%
\pgfsetfillopacity{0.700000}%
\pgfsetlinewidth{0.000000pt}%
\definecolor{currentstroke}{rgb}{0.000000,0.000000,0.000000}%
\pgfsetstrokecolor{currentstroke}%
\pgfsetstrokeopacity{0.700000}%
\pgfsetdash{}{0pt}%
\pgfpathmoveto{\pgfqpoint{7.326705in}{0.550000in}}%
\pgfpathlineto{\pgfqpoint{7.354226in}{0.550000in}}%
\pgfpathlineto{\pgfqpoint{7.354226in}{0.550000in}}%
\pgfpathlineto{\pgfqpoint{7.326705in}{0.550000in}}%
\pgfpathlineto{\pgfqpoint{7.326705in}{0.550000in}}%
\pgfpathclose%
\pgfusepath{fill}%
\end{pgfscope}%
\begin{pgfscope}%
\pgfpathrectangle{\pgfqpoint{1.250000in}{0.550000in}}{\pgfqpoint{7.750000in}{3.850000in}}%
\pgfusepath{clip}%
\pgfsetbuttcap%
\pgfsetmiterjoin%
\definecolor{currentfill}{rgb}{0.501961,0.501961,0.501961}%
\pgfsetfillcolor{currentfill}%
\pgfsetfillopacity{0.700000}%
\pgfsetlinewidth{0.000000pt}%
\definecolor{currentstroke}{rgb}{0.000000,0.000000,0.000000}%
\pgfsetstrokecolor{currentstroke}%
\pgfsetstrokeopacity{0.700000}%
\pgfsetdash{}{0pt}%
\pgfpathmoveto{\pgfqpoint{7.354226in}{0.550000in}}%
\pgfpathlineto{\pgfqpoint{7.381747in}{0.550000in}}%
\pgfpathlineto{\pgfqpoint{7.381747in}{0.550000in}}%
\pgfpathlineto{\pgfqpoint{7.354226in}{0.550000in}}%
\pgfpathlineto{\pgfqpoint{7.354226in}{0.550000in}}%
\pgfpathclose%
\pgfusepath{fill}%
\end{pgfscope}%
\begin{pgfscope}%
\pgfpathrectangle{\pgfqpoint{1.250000in}{0.550000in}}{\pgfqpoint{7.750000in}{3.850000in}}%
\pgfusepath{clip}%
\pgfsetbuttcap%
\pgfsetmiterjoin%
\definecolor{currentfill}{rgb}{0.501961,0.501961,0.501961}%
\pgfsetfillcolor{currentfill}%
\pgfsetfillopacity{0.700000}%
\pgfsetlinewidth{0.000000pt}%
\definecolor{currentstroke}{rgb}{0.000000,0.000000,0.000000}%
\pgfsetstrokecolor{currentstroke}%
\pgfsetstrokeopacity{0.700000}%
\pgfsetdash{}{0pt}%
\pgfpathmoveto{\pgfqpoint{7.381747in}{0.550000in}}%
\pgfpathlineto{\pgfqpoint{7.409268in}{0.550000in}}%
\pgfpathlineto{\pgfqpoint{7.409268in}{0.550000in}}%
\pgfpathlineto{\pgfqpoint{7.381747in}{0.550000in}}%
\pgfpathlineto{\pgfqpoint{7.381747in}{0.550000in}}%
\pgfpathclose%
\pgfusepath{fill}%
\end{pgfscope}%
\begin{pgfscope}%
\pgfpathrectangle{\pgfqpoint{1.250000in}{0.550000in}}{\pgfqpoint{7.750000in}{3.850000in}}%
\pgfusepath{clip}%
\pgfsetbuttcap%
\pgfsetmiterjoin%
\definecolor{currentfill}{rgb}{0.501961,0.501961,0.501961}%
\pgfsetfillcolor{currentfill}%
\pgfsetfillopacity{0.700000}%
\pgfsetlinewidth{0.000000pt}%
\definecolor{currentstroke}{rgb}{0.000000,0.000000,0.000000}%
\pgfsetstrokecolor{currentstroke}%
\pgfsetstrokeopacity{0.700000}%
\pgfsetdash{}{0pt}%
\pgfpathmoveto{\pgfqpoint{7.409268in}{0.550000in}}%
\pgfpathlineto{\pgfqpoint{7.436790in}{0.550000in}}%
\pgfpathlineto{\pgfqpoint{7.436790in}{0.550153in}}%
\pgfpathlineto{\pgfqpoint{7.409268in}{0.550153in}}%
\pgfpathlineto{\pgfqpoint{7.409268in}{0.550000in}}%
\pgfpathclose%
\pgfusepath{fill}%
\end{pgfscope}%
\begin{pgfscope}%
\pgfpathrectangle{\pgfqpoint{1.250000in}{0.550000in}}{\pgfqpoint{7.750000in}{3.850000in}}%
\pgfusepath{clip}%
\pgfsetbuttcap%
\pgfsetmiterjoin%
\definecolor{currentfill}{rgb}{0.501961,0.501961,0.501961}%
\pgfsetfillcolor{currentfill}%
\pgfsetfillopacity{0.700000}%
\pgfsetlinewidth{0.000000pt}%
\definecolor{currentstroke}{rgb}{0.000000,0.000000,0.000000}%
\pgfsetstrokecolor{currentstroke}%
\pgfsetstrokeopacity{0.700000}%
\pgfsetdash{}{0pt}%
\pgfpathmoveto{\pgfqpoint{7.436790in}{0.550000in}}%
\pgfpathlineto{\pgfqpoint{7.464311in}{0.550000in}}%
\pgfpathlineto{\pgfqpoint{7.464311in}{0.550000in}}%
\pgfpathlineto{\pgfqpoint{7.436790in}{0.550000in}}%
\pgfpathlineto{\pgfqpoint{7.436790in}{0.550000in}}%
\pgfpathclose%
\pgfusepath{fill}%
\end{pgfscope}%
\begin{pgfscope}%
\pgfpathrectangle{\pgfqpoint{1.250000in}{0.550000in}}{\pgfqpoint{7.750000in}{3.850000in}}%
\pgfusepath{clip}%
\pgfsetbuttcap%
\pgfsetmiterjoin%
\definecolor{currentfill}{rgb}{0.501961,0.501961,0.501961}%
\pgfsetfillcolor{currentfill}%
\pgfsetfillopacity{0.700000}%
\pgfsetlinewidth{0.000000pt}%
\definecolor{currentstroke}{rgb}{0.000000,0.000000,0.000000}%
\pgfsetstrokecolor{currentstroke}%
\pgfsetstrokeopacity{0.700000}%
\pgfsetdash{}{0pt}%
\pgfpathmoveto{\pgfqpoint{7.464311in}{0.550000in}}%
\pgfpathlineto{\pgfqpoint{7.491832in}{0.550000in}}%
\pgfpathlineto{\pgfqpoint{7.491832in}{0.550153in}}%
\pgfpathlineto{\pgfqpoint{7.464311in}{0.550153in}}%
\pgfpathlineto{\pgfqpoint{7.464311in}{0.550000in}}%
\pgfpathclose%
\pgfusepath{fill}%
\end{pgfscope}%
\begin{pgfscope}%
\pgfpathrectangle{\pgfqpoint{1.250000in}{0.550000in}}{\pgfqpoint{7.750000in}{3.850000in}}%
\pgfusepath{clip}%
\pgfsetbuttcap%
\pgfsetmiterjoin%
\definecolor{currentfill}{rgb}{0.501961,0.501961,0.501961}%
\pgfsetfillcolor{currentfill}%
\pgfsetfillopacity{0.700000}%
\pgfsetlinewidth{0.000000pt}%
\definecolor{currentstroke}{rgb}{0.000000,0.000000,0.000000}%
\pgfsetstrokecolor{currentstroke}%
\pgfsetstrokeopacity{0.700000}%
\pgfsetdash{}{0pt}%
\pgfpathmoveto{\pgfqpoint{7.491832in}{0.550000in}}%
\pgfpathlineto{\pgfqpoint{7.519354in}{0.550000in}}%
\pgfpathlineto{\pgfqpoint{7.519354in}{0.550000in}}%
\pgfpathlineto{\pgfqpoint{7.491832in}{0.550000in}}%
\pgfpathlineto{\pgfqpoint{7.491832in}{0.550000in}}%
\pgfpathclose%
\pgfusepath{fill}%
\end{pgfscope}%
\begin{pgfscope}%
\pgfpathrectangle{\pgfqpoint{1.250000in}{0.550000in}}{\pgfqpoint{7.750000in}{3.850000in}}%
\pgfusepath{clip}%
\pgfsetbuttcap%
\pgfsetmiterjoin%
\definecolor{currentfill}{rgb}{0.501961,0.501961,0.501961}%
\pgfsetfillcolor{currentfill}%
\pgfsetfillopacity{0.700000}%
\pgfsetlinewidth{0.000000pt}%
\definecolor{currentstroke}{rgb}{0.000000,0.000000,0.000000}%
\pgfsetstrokecolor{currentstroke}%
\pgfsetstrokeopacity{0.700000}%
\pgfsetdash{}{0pt}%
\pgfpathmoveto{\pgfqpoint{7.519354in}{0.550000in}}%
\pgfpathlineto{\pgfqpoint{7.546875in}{0.550000in}}%
\pgfpathlineto{\pgfqpoint{7.546875in}{0.550000in}}%
\pgfpathlineto{\pgfqpoint{7.519354in}{0.550000in}}%
\pgfpathlineto{\pgfqpoint{7.519354in}{0.550000in}}%
\pgfpathclose%
\pgfusepath{fill}%
\end{pgfscope}%
\begin{pgfscope}%
\pgfpathrectangle{\pgfqpoint{1.250000in}{0.550000in}}{\pgfqpoint{7.750000in}{3.850000in}}%
\pgfusepath{clip}%
\pgfsetbuttcap%
\pgfsetmiterjoin%
\definecolor{currentfill}{rgb}{0.501961,0.501961,0.501961}%
\pgfsetfillcolor{currentfill}%
\pgfsetfillopacity{0.700000}%
\pgfsetlinewidth{0.000000pt}%
\definecolor{currentstroke}{rgb}{0.000000,0.000000,0.000000}%
\pgfsetstrokecolor{currentstroke}%
\pgfsetstrokeopacity{0.700000}%
\pgfsetdash{}{0pt}%
\pgfpathmoveto{\pgfqpoint{7.546875in}{0.550000in}}%
\pgfpathlineto{\pgfqpoint{7.574396in}{0.550000in}}%
\pgfpathlineto{\pgfqpoint{7.574396in}{0.550000in}}%
\pgfpathlineto{\pgfqpoint{7.546875in}{0.550000in}}%
\pgfpathlineto{\pgfqpoint{7.546875in}{0.550000in}}%
\pgfpathclose%
\pgfusepath{fill}%
\end{pgfscope}%
\begin{pgfscope}%
\pgfpathrectangle{\pgfqpoint{1.250000in}{0.550000in}}{\pgfqpoint{7.750000in}{3.850000in}}%
\pgfusepath{clip}%
\pgfsetbuttcap%
\pgfsetmiterjoin%
\definecolor{currentfill}{rgb}{0.501961,0.501961,0.501961}%
\pgfsetfillcolor{currentfill}%
\pgfsetfillopacity{0.700000}%
\pgfsetlinewidth{0.000000pt}%
\definecolor{currentstroke}{rgb}{0.000000,0.000000,0.000000}%
\pgfsetstrokecolor{currentstroke}%
\pgfsetstrokeopacity{0.700000}%
\pgfsetdash{}{0pt}%
\pgfpathmoveto{\pgfqpoint{7.574396in}{0.550000in}}%
\pgfpathlineto{\pgfqpoint{7.601918in}{0.550000in}}%
\pgfpathlineto{\pgfqpoint{7.601918in}{0.550000in}}%
\pgfpathlineto{\pgfqpoint{7.574396in}{0.550000in}}%
\pgfpathlineto{\pgfqpoint{7.574396in}{0.550000in}}%
\pgfpathclose%
\pgfusepath{fill}%
\end{pgfscope}%
\begin{pgfscope}%
\pgfpathrectangle{\pgfqpoint{1.250000in}{0.550000in}}{\pgfqpoint{7.750000in}{3.850000in}}%
\pgfusepath{clip}%
\pgfsetbuttcap%
\pgfsetmiterjoin%
\definecolor{currentfill}{rgb}{0.501961,0.501961,0.501961}%
\pgfsetfillcolor{currentfill}%
\pgfsetfillopacity{0.700000}%
\pgfsetlinewidth{0.000000pt}%
\definecolor{currentstroke}{rgb}{0.000000,0.000000,0.000000}%
\pgfsetstrokecolor{currentstroke}%
\pgfsetstrokeopacity{0.700000}%
\pgfsetdash{}{0pt}%
\pgfpathmoveto{\pgfqpoint{7.601918in}{0.550000in}}%
\pgfpathlineto{\pgfqpoint{7.629439in}{0.550000in}}%
\pgfpathlineto{\pgfqpoint{7.629439in}{0.550000in}}%
\pgfpathlineto{\pgfqpoint{7.601918in}{0.550000in}}%
\pgfpathlineto{\pgfqpoint{7.601918in}{0.550000in}}%
\pgfpathclose%
\pgfusepath{fill}%
\end{pgfscope}%
\begin{pgfscope}%
\pgfpathrectangle{\pgfqpoint{1.250000in}{0.550000in}}{\pgfqpoint{7.750000in}{3.850000in}}%
\pgfusepath{clip}%
\pgfsetbuttcap%
\pgfsetmiterjoin%
\definecolor{currentfill}{rgb}{0.501961,0.501961,0.501961}%
\pgfsetfillcolor{currentfill}%
\pgfsetfillopacity{0.700000}%
\pgfsetlinewidth{0.000000pt}%
\definecolor{currentstroke}{rgb}{0.000000,0.000000,0.000000}%
\pgfsetstrokecolor{currentstroke}%
\pgfsetstrokeopacity{0.700000}%
\pgfsetdash{}{0pt}%
\pgfpathmoveto{\pgfqpoint{7.629439in}{0.550000in}}%
\pgfpathlineto{\pgfqpoint{7.656960in}{0.550000in}}%
\pgfpathlineto{\pgfqpoint{7.656960in}{0.550000in}}%
\pgfpathlineto{\pgfqpoint{7.629439in}{0.550000in}}%
\pgfpathlineto{\pgfqpoint{7.629439in}{0.550000in}}%
\pgfpathclose%
\pgfusepath{fill}%
\end{pgfscope}%
\begin{pgfscope}%
\pgfpathrectangle{\pgfqpoint{1.250000in}{0.550000in}}{\pgfqpoint{7.750000in}{3.850000in}}%
\pgfusepath{clip}%
\pgfsetbuttcap%
\pgfsetmiterjoin%
\definecolor{currentfill}{rgb}{0.501961,0.501961,0.501961}%
\pgfsetfillcolor{currentfill}%
\pgfsetfillopacity{0.700000}%
\pgfsetlinewidth{0.000000pt}%
\definecolor{currentstroke}{rgb}{0.000000,0.000000,0.000000}%
\pgfsetstrokecolor{currentstroke}%
\pgfsetstrokeopacity{0.700000}%
\pgfsetdash{}{0pt}%
\pgfpathmoveto{\pgfqpoint{7.656960in}{0.550000in}}%
\pgfpathlineto{\pgfqpoint{7.684482in}{0.550000in}}%
\pgfpathlineto{\pgfqpoint{7.684482in}{0.550306in}}%
\pgfpathlineto{\pgfqpoint{7.656960in}{0.550306in}}%
\pgfpathlineto{\pgfqpoint{7.656960in}{0.550000in}}%
\pgfpathclose%
\pgfusepath{fill}%
\end{pgfscope}%
\begin{pgfscope}%
\pgfpathrectangle{\pgfqpoint{1.250000in}{0.550000in}}{\pgfqpoint{7.750000in}{3.850000in}}%
\pgfusepath{clip}%
\pgfsetbuttcap%
\pgfsetmiterjoin%
\definecolor{currentfill}{rgb}{0.501961,0.501961,0.501961}%
\pgfsetfillcolor{currentfill}%
\pgfsetfillopacity{0.700000}%
\pgfsetlinewidth{0.000000pt}%
\definecolor{currentstroke}{rgb}{0.000000,0.000000,0.000000}%
\pgfsetstrokecolor{currentstroke}%
\pgfsetstrokeopacity{0.700000}%
\pgfsetdash{}{0pt}%
\pgfpathmoveto{\pgfqpoint{7.684482in}{0.550000in}}%
\pgfpathlineto{\pgfqpoint{7.712003in}{0.550000in}}%
\pgfpathlineto{\pgfqpoint{7.712003in}{0.550000in}}%
\pgfpathlineto{\pgfqpoint{7.684482in}{0.550000in}}%
\pgfpathlineto{\pgfqpoint{7.684482in}{0.550000in}}%
\pgfpathclose%
\pgfusepath{fill}%
\end{pgfscope}%
\begin{pgfscope}%
\pgfpathrectangle{\pgfqpoint{1.250000in}{0.550000in}}{\pgfqpoint{7.750000in}{3.850000in}}%
\pgfusepath{clip}%
\pgfsetbuttcap%
\pgfsetmiterjoin%
\definecolor{currentfill}{rgb}{0.501961,0.501961,0.501961}%
\pgfsetfillcolor{currentfill}%
\pgfsetfillopacity{0.700000}%
\pgfsetlinewidth{0.000000pt}%
\definecolor{currentstroke}{rgb}{0.000000,0.000000,0.000000}%
\pgfsetstrokecolor{currentstroke}%
\pgfsetstrokeopacity{0.700000}%
\pgfsetdash{}{0pt}%
\pgfpathmoveto{\pgfqpoint{7.712003in}{0.550000in}}%
\pgfpathlineto{\pgfqpoint{7.739524in}{0.550000in}}%
\pgfpathlineto{\pgfqpoint{7.739524in}{0.550153in}}%
\pgfpathlineto{\pgfqpoint{7.712003in}{0.550153in}}%
\pgfpathlineto{\pgfqpoint{7.712003in}{0.550000in}}%
\pgfpathclose%
\pgfusepath{fill}%
\end{pgfscope}%
\begin{pgfscope}%
\pgfpathrectangle{\pgfqpoint{1.250000in}{0.550000in}}{\pgfqpoint{7.750000in}{3.850000in}}%
\pgfusepath{clip}%
\pgfsetbuttcap%
\pgfsetmiterjoin%
\definecolor{currentfill}{rgb}{0.501961,0.501961,0.501961}%
\pgfsetfillcolor{currentfill}%
\pgfsetfillopacity{0.700000}%
\pgfsetlinewidth{0.000000pt}%
\definecolor{currentstroke}{rgb}{0.000000,0.000000,0.000000}%
\pgfsetstrokecolor{currentstroke}%
\pgfsetstrokeopacity{0.700000}%
\pgfsetdash{}{0pt}%
\pgfpathmoveto{\pgfqpoint{7.739524in}{0.550000in}}%
\pgfpathlineto{\pgfqpoint{7.767045in}{0.550000in}}%
\pgfpathlineto{\pgfqpoint{7.767045in}{0.550000in}}%
\pgfpathlineto{\pgfqpoint{7.739524in}{0.550000in}}%
\pgfpathlineto{\pgfqpoint{7.739524in}{0.550000in}}%
\pgfpathclose%
\pgfusepath{fill}%
\end{pgfscope}%
\begin{pgfscope}%
\pgfpathrectangle{\pgfqpoint{1.250000in}{0.550000in}}{\pgfqpoint{7.750000in}{3.850000in}}%
\pgfusepath{clip}%
\pgfsetbuttcap%
\pgfsetmiterjoin%
\definecolor{currentfill}{rgb}{0.501961,0.501961,0.501961}%
\pgfsetfillcolor{currentfill}%
\pgfsetfillopacity{0.700000}%
\pgfsetlinewidth{0.000000pt}%
\definecolor{currentstroke}{rgb}{0.000000,0.000000,0.000000}%
\pgfsetstrokecolor{currentstroke}%
\pgfsetstrokeopacity{0.700000}%
\pgfsetdash{}{0pt}%
\pgfpathmoveto{\pgfqpoint{7.767045in}{0.550000in}}%
\pgfpathlineto{\pgfqpoint{7.794567in}{0.550000in}}%
\pgfpathlineto{\pgfqpoint{7.794567in}{0.550153in}}%
\pgfpathlineto{\pgfqpoint{7.767045in}{0.550153in}}%
\pgfpathlineto{\pgfqpoint{7.767045in}{0.550000in}}%
\pgfpathclose%
\pgfusepath{fill}%
\end{pgfscope}%
\begin{pgfscope}%
\pgfpathrectangle{\pgfqpoint{1.250000in}{0.550000in}}{\pgfqpoint{7.750000in}{3.850000in}}%
\pgfusepath{clip}%
\pgfsetbuttcap%
\pgfsetmiterjoin%
\definecolor{currentfill}{rgb}{0.501961,0.501961,0.501961}%
\pgfsetfillcolor{currentfill}%
\pgfsetfillopacity{0.700000}%
\pgfsetlinewidth{0.000000pt}%
\definecolor{currentstroke}{rgb}{0.000000,0.000000,0.000000}%
\pgfsetstrokecolor{currentstroke}%
\pgfsetstrokeopacity{0.700000}%
\pgfsetdash{}{0pt}%
\pgfpathmoveto{\pgfqpoint{7.794567in}{0.550000in}}%
\pgfpathlineto{\pgfqpoint{7.822088in}{0.550000in}}%
\pgfpathlineto{\pgfqpoint{7.822088in}{0.550153in}}%
\pgfpathlineto{\pgfqpoint{7.794567in}{0.550153in}}%
\pgfpathlineto{\pgfqpoint{7.794567in}{0.550000in}}%
\pgfpathclose%
\pgfusepath{fill}%
\end{pgfscope}%
\begin{pgfscope}%
\pgfpathrectangle{\pgfqpoint{1.250000in}{0.550000in}}{\pgfqpoint{7.750000in}{3.850000in}}%
\pgfusepath{clip}%
\pgfsetbuttcap%
\pgfsetmiterjoin%
\definecolor{currentfill}{rgb}{0.501961,0.501961,0.501961}%
\pgfsetfillcolor{currentfill}%
\pgfsetfillopacity{0.700000}%
\pgfsetlinewidth{0.000000pt}%
\definecolor{currentstroke}{rgb}{0.000000,0.000000,0.000000}%
\pgfsetstrokecolor{currentstroke}%
\pgfsetstrokeopacity{0.700000}%
\pgfsetdash{}{0pt}%
\pgfpathmoveto{\pgfqpoint{7.822088in}{0.550000in}}%
\pgfpathlineto{\pgfqpoint{7.849609in}{0.550000in}}%
\pgfpathlineto{\pgfqpoint{7.849609in}{0.550153in}}%
\pgfpathlineto{\pgfqpoint{7.822088in}{0.550153in}}%
\pgfpathlineto{\pgfqpoint{7.822088in}{0.550000in}}%
\pgfpathclose%
\pgfusepath{fill}%
\end{pgfscope}%
\begin{pgfscope}%
\pgfpathrectangle{\pgfqpoint{1.250000in}{0.550000in}}{\pgfqpoint{7.750000in}{3.850000in}}%
\pgfusepath{clip}%
\pgfsetbuttcap%
\pgfsetmiterjoin%
\definecolor{currentfill}{rgb}{0.501961,0.501961,0.501961}%
\pgfsetfillcolor{currentfill}%
\pgfsetfillopacity{0.700000}%
\pgfsetlinewidth{0.000000pt}%
\definecolor{currentstroke}{rgb}{0.000000,0.000000,0.000000}%
\pgfsetstrokecolor{currentstroke}%
\pgfsetstrokeopacity{0.700000}%
\pgfsetdash{}{0pt}%
\pgfpathmoveto{\pgfqpoint{7.849609in}{0.550000in}}%
\pgfpathlineto{\pgfqpoint{7.877131in}{0.550000in}}%
\pgfpathlineto{\pgfqpoint{7.877131in}{0.550000in}}%
\pgfpathlineto{\pgfqpoint{7.849609in}{0.550000in}}%
\pgfpathlineto{\pgfqpoint{7.849609in}{0.550000in}}%
\pgfpathclose%
\pgfusepath{fill}%
\end{pgfscope}%
\begin{pgfscope}%
\pgfpathrectangle{\pgfqpoint{1.250000in}{0.550000in}}{\pgfqpoint{7.750000in}{3.850000in}}%
\pgfusepath{clip}%
\pgfsetbuttcap%
\pgfsetmiterjoin%
\definecolor{currentfill}{rgb}{0.501961,0.501961,0.501961}%
\pgfsetfillcolor{currentfill}%
\pgfsetfillopacity{0.700000}%
\pgfsetlinewidth{0.000000pt}%
\definecolor{currentstroke}{rgb}{0.000000,0.000000,0.000000}%
\pgfsetstrokecolor{currentstroke}%
\pgfsetstrokeopacity{0.700000}%
\pgfsetdash{}{0pt}%
\pgfpathmoveto{\pgfqpoint{7.877131in}{0.550000in}}%
\pgfpathlineto{\pgfqpoint{7.904652in}{0.550000in}}%
\pgfpathlineto{\pgfqpoint{7.904652in}{0.550000in}}%
\pgfpathlineto{\pgfqpoint{7.877131in}{0.550000in}}%
\pgfpathlineto{\pgfqpoint{7.877131in}{0.550000in}}%
\pgfpathclose%
\pgfusepath{fill}%
\end{pgfscope}%
\begin{pgfscope}%
\pgfpathrectangle{\pgfqpoint{1.250000in}{0.550000in}}{\pgfqpoint{7.750000in}{3.850000in}}%
\pgfusepath{clip}%
\pgfsetbuttcap%
\pgfsetmiterjoin%
\definecolor{currentfill}{rgb}{0.501961,0.501961,0.501961}%
\pgfsetfillcolor{currentfill}%
\pgfsetfillopacity{0.700000}%
\pgfsetlinewidth{0.000000pt}%
\definecolor{currentstroke}{rgb}{0.000000,0.000000,0.000000}%
\pgfsetstrokecolor{currentstroke}%
\pgfsetstrokeopacity{0.700000}%
\pgfsetdash{}{0pt}%
\pgfpathmoveto{\pgfqpoint{7.904652in}{0.550000in}}%
\pgfpathlineto{\pgfqpoint{7.932173in}{0.550000in}}%
\pgfpathlineto{\pgfqpoint{7.932173in}{0.550153in}}%
\pgfpathlineto{\pgfqpoint{7.904652in}{0.550153in}}%
\pgfpathlineto{\pgfqpoint{7.904652in}{0.550000in}}%
\pgfpathclose%
\pgfusepath{fill}%
\end{pgfscope}%
\begin{pgfscope}%
\pgfpathrectangle{\pgfqpoint{1.250000in}{0.550000in}}{\pgfqpoint{7.750000in}{3.850000in}}%
\pgfusepath{clip}%
\pgfsetbuttcap%
\pgfsetmiterjoin%
\definecolor{currentfill}{rgb}{0.501961,0.501961,0.501961}%
\pgfsetfillcolor{currentfill}%
\pgfsetfillopacity{0.700000}%
\pgfsetlinewidth{0.000000pt}%
\definecolor{currentstroke}{rgb}{0.000000,0.000000,0.000000}%
\pgfsetstrokecolor{currentstroke}%
\pgfsetstrokeopacity{0.700000}%
\pgfsetdash{}{0pt}%
\pgfpathmoveto{\pgfqpoint{7.932173in}{0.550000in}}%
\pgfpathlineto{\pgfqpoint{7.959695in}{0.550000in}}%
\pgfpathlineto{\pgfqpoint{7.959695in}{0.550306in}}%
\pgfpathlineto{\pgfqpoint{7.932173in}{0.550306in}}%
\pgfpathlineto{\pgfqpoint{7.932173in}{0.550000in}}%
\pgfpathclose%
\pgfusepath{fill}%
\end{pgfscope}%
\begin{pgfscope}%
\pgfpathrectangle{\pgfqpoint{1.250000in}{0.550000in}}{\pgfqpoint{7.750000in}{3.850000in}}%
\pgfusepath{clip}%
\pgfsetbuttcap%
\pgfsetmiterjoin%
\definecolor{currentfill}{rgb}{0.501961,0.501961,0.501961}%
\pgfsetfillcolor{currentfill}%
\pgfsetfillopacity{0.700000}%
\pgfsetlinewidth{0.000000pt}%
\definecolor{currentstroke}{rgb}{0.000000,0.000000,0.000000}%
\pgfsetstrokecolor{currentstroke}%
\pgfsetstrokeopacity{0.700000}%
\pgfsetdash{}{0pt}%
\pgfpathmoveto{\pgfqpoint{7.959695in}{0.550000in}}%
\pgfpathlineto{\pgfqpoint{7.987216in}{0.550000in}}%
\pgfpathlineto{\pgfqpoint{7.987216in}{0.550000in}}%
\pgfpathlineto{\pgfqpoint{7.959695in}{0.550000in}}%
\pgfpathlineto{\pgfqpoint{7.959695in}{0.550000in}}%
\pgfpathclose%
\pgfusepath{fill}%
\end{pgfscope}%
\begin{pgfscope}%
\pgfpathrectangle{\pgfqpoint{1.250000in}{0.550000in}}{\pgfqpoint{7.750000in}{3.850000in}}%
\pgfusepath{clip}%
\pgfsetbuttcap%
\pgfsetmiterjoin%
\definecolor{currentfill}{rgb}{0.501961,0.501961,0.501961}%
\pgfsetfillcolor{currentfill}%
\pgfsetfillopacity{0.700000}%
\pgfsetlinewidth{0.000000pt}%
\definecolor{currentstroke}{rgb}{0.000000,0.000000,0.000000}%
\pgfsetstrokecolor{currentstroke}%
\pgfsetstrokeopacity{0.700000}%
\pgfsetdash{}{0pt}%
\pgfpathmoveto{\pgfqpoint{7.987216in}{0.550000in}}%
\pgfpathlineto{\pgfqpoint{8.014737in}{0.550000in}}%
\pgfpathlineto{\pgfqpoint{8.014737in}{0.550153in}}%
\pgfpathlineto{\pgfqpoint{7.987216in}{0.550153in}}%
\pgfpathlineto{\pgfqpoint{7.987216in}{0.550000in}}%
\pgfpathclose%
\pgfusepath{fill}%
\end{pgfscope}%
\begin{pgfscope}%
\pgfpathrectangle{\pgfqpoint{1.250000in}{0.550000in}}{\pgfqpoint{7.750000in}{3.850000in}}%
\pgfusepath{clip}%
\pgfsetbuttcap%
\pgfsetmiterjoin%
\definecolor{currentfill}{rgb}{0.501961,0.501961,0.501961}%
\pgfsetfillcolor{currentfill}%
\pgfsetfillopacity{0.700000}%
\pgfsetlinewidth{0.000000pt}%
\definecolor{currentstroke}{rgb}{0.000000,0.000000,0.000000}%
\pgfsetstrokecolor{currentstroke}%
\pgfsetstrokeopacity{0.700000}%
\pgfsetdash{}{0pt}%
\pgfpathmoveto{\pgfqpoint{8.014737in}{0.550000in}}%
\pgfpathlineto{\pgfqpoint{8.042259in}{0.550000in}}%
\pgfpathlineto{\pgfqpoint{8.042259in}{0.550153in}}%
\pgfpathlineto{\pgfqpoint{8.014737in}{0.550153in}}%
\pgfpathlineto{\pgfqpoint{8.014737in}{0.550000in}}%
\pgfpathclose%
\pgfusepath{fill}%
\end{pgfscope}%
\begin{pgfscope}%
\pgfpathrectangle{\pgfqpoint{1.250000in}{0.550000in}}{\pgfqpoint{7.750000in}{3.850000in}}%
\pgfusepath{clip}%
\pgfsetbuttcap%
\pgfsetmiterjoin%
\definecolor{currentfill}{rgb}{0.501961,0.501961,0.501961}%
\pgfsetfillcolor{currentfill}%
\pgfsetfillopacity{0.700000}%
\pgfsetlinewidth{0.000000pt}%
\definecolor{currentstroke}{rgb}{0.000000,0.000000,0.000000}%
\pgfsetstrokecolor{currentstroke}%
\pgfsetstrokeopacity{0.700000}%
\pgfsetdash{}{0pt}%
\pgfpathmoveto{\pgfqpoint{8.042259in}{0.550000in}}%
\pgfpathlineto{\pgfqpoint{8.069780in}{0.550000in}}%
\pgfpathlineto{\pgfqpoint{8.069780in}{0.550000in}}%
\pgfpathlineto{\pgfqpoint{8.042259in}{0.550000in}}%
\pgfpathlineto{\pgfqpoint{8.042259in}{0.550000in}}%
\pgfpathclose%
\pgfusepath{fill}%
\end{pgfscope}%
\begin{pgfscope}%
\pgfpathrectangle{\pgfqpoint{1.250000in}{0.550000in}}{\pgfqpoint{7.750000in}{3.850000in}}%
\pgfusepath{clip}%
\pgfsetbuttcap%
\pgfsetmiterjoin%
\definecolor{currentfill}{rgb}{0.501961,0.501961,0.501961}%
\pgfsetfillcolor{currentfill}%
\pgfsetfillopacity{0.700000}%
\pgfsetlinewidth{0.000000pt}%
\definecolor{currentstroke}{rgb}{0.000000,0.000000,0.000000}%
\pgfsetstrokecolor{currentstroke}%
\pgfsetstrokeopacity{0.700000}%
\pgfsetdash{}{0pt}%
\pgfpathmoveto{\pgfqpoint{8.069780in}{0.550000in}}%
\pgfpathlineto{\pgfqpoint{8.097301in}{0.550000in}}%
\pgfpathlineto{\pgfqpoint{8.097301in}{0.550000in}}%
\pgfpathlineto{\pgfqpoint{8.069780in}{0.550000in}}%
\pgfpathlineto{\pgfqpoint{8.069780in}{0.550000in}}%
\pgfpathclose%
\pgfusepath{fill}%
\end{pgfscope}%
\begin{pgfscope}%
\pgfpathrectangle{\pgfqpoint{1.250000in}{0.550000in}}{\pgfqpoint{7.750000in}{3.850000in}}%
\pgfusepath{clip}%
\pgfsetbuttcap%
\pgfsetmiterjoin%
\definecolor{currentfill}{rgb}{0.501961,0.501961,0.501961}%
\pgfsetfillcolor{currentfill}%
\pgfsetfillopacity{0.700000}%
\pgfsetlinewidth{0.000000pt}%
\definecolor{currentstroke}{rgb}{0.000000,0.000000,0.000000}%
\pgfsetstrokecolor{currentstroke}%
\pgfsetstrokeopacity{0.700000}%
\pgfsetdash{}{0pt}%
\pgfpathmoveto{\pgfqpoint{8.097301in}{0.550000in}}%
\pgfpathlineto{\pgfqpoint{8.124822in}{0.550000in}}%
\pgfpathlineto{\pgfqpoint{8.124822in}{0.550153in}}%
\pgfpathlineto{\pgfqpoint{8.097301in}{0.550153in}}%
\pgfpathlineto{\pgfqpoint{8.097301in}{0.550000in}}%
\pgfpathclose%
\pgfusepath{fill}%
\end{pgfscope}%
\begin{pgfscope}%
\pgfpathrectangle{\pgfqpoint{1.250000in}{0.550000in}}{\pgfqpoint{7.750000in}{3.850000in}}%
\pgfusepath{clip}%
\pgfsetbuttcap%
\pgfsetmiterjoin%
\definecolor{currentfill}{rgb}{0.501961,0.501961,0.501961}%
\pgfsetfillcolor{currentfill}%
\pgfsetfillopacity{0.700000}%
\pgfsetlinewidth{0.000000pt}%
\definecolor{currentstroke}{rgb}{0.000000,0.000000,0.000000}%
\pgfsetstrokecolor{currentstroke}%
\pgfsetstrokeopacity{0.700000}%
\pgfsetdash{}{0pt}%
\pgfpathmoveto{\pgfqpoint{8.124822in}{0.550000in}}%
\pgfpathlineto{\pgfqpoint{8.152344in}{0.550000in}}%
\pgfpathlineto{\pgfqpoint{8.152344in}{0.550306in}}%
\pgfpathlineto{\pgfqpoint{8.124822in}{0.550306in}}%
\pgfpathlineto{\pgfqpoint{8.124822in}{0.550000in}}%
\pgfpathclose%
\pgfusepath{fill}%
\end{pgfscope}%
\begin{pgfscope}%
\pgfpathrectangle{\pgfqpoint{1.250000in}{0.550000in}}{\pgfqpoint{7.750000in}{3.850000in}}%
\pgfusepath{clip}%
\pgfsetbuttcap%
\pgfsetmiterjoin%
\definecolor{currentfill}{rgb}{0.501961,0.501961,0.501961}%
\pgfsetfillcolor{currentfill}%
\pgfsetfillopacity{0.700000}%
\pgfsetlinewidth{0.000000pt}%
\definecolor{currentstroke}{rgb}{0.000000,0.000000,0.000000}%
\pgfsetstrokecolor{currentstroke}%
\pgfsetstrokeopacity{0.700000}%
\pgfsetdash{}{0pt}%
\pgfpathmoveto{\pgfqpoint{8.152344in}{0.550000in}}%
\pgfpathlineto{\pgfqpoint{8.179865in}{0.550000in}}%
\pgfpathlineto{\pgfqpoint{8.179865in}{0.550000in}}%
\pgfpathlineto{\pgfqpoint{8.152344in}{0.550000in}}%
\pgfpathlineto{\pgfqpoint{8.152344in}{0.550000in}}%
\pgfpathclose%
\pgfusepath{fill}%
\end{pgfscope}%
\begin{pgfscope}%
\pgfpathrectangle{\pgfqpoint{1.250000in}{0.550000in}}{\pgfqpoint{7.750000in}{3.850000in}}%
\pgfusepath{clip}%
\pgfsetbuttcap%
\pgfsetmiterjoin%
\definecolor{currentfill}{rgb}{0.501961,0.501961,0.501961}%
\pgfsetfillcolor{currentfill}%
\pgfsetfillopacity{0.700000}%
\pgfsetlinewidth{0.000000pt}%
\definecolor{currentstroke}{rgb}{0.000000,0.000000,0.000000}%
\pgfsetstrokecolor{currentstroke}%
\pgfsetstrokeopacity{0.700000}%
\pgfsetdash{}{0pt}%
\pgfpathmoveto{\pgfqpoint{8.179865in}{0.550000in}}%
\pgfpathlineto{\pgfqpoint{8.207386in}{0.550000in}}%
\pgfpathlineto{\pgfqpoint{8.207386in}{0.550000in}}%
\pgfpathlineto{\pgfqpoint{8.179865in}{0.550000in}}%
\pgfpathlineto{\pgfqpoint{8.179865in}{0.550000in}}%
\pgfpathclose%
\pgfusepath{fill}%
\end{pgfscope}%
\begin{pgfscope}%
\pgfpathrectangle{\pgfqpoint{1.250000in}{0.550000in}}{\pgfqpoint{7.750000in}{3.850000in}}%
\pgfusepath{clip}%
\pgfsetbuttcap%
\pgfsetmiterjoin%
\definecolor{currentfill}{rgb}{0.501961,0.501961,0.501961}%
\pgfsetfillcolor{currentfill}%
\pgfsetfillopacity{0.700000}%
\pgfsetlinewidth{0.000000pt}%
\definecolor{currentstroke}{rgb}{0.000000,0.000000,0.000000}%
\pgfsetstrokecolor{currentstroke}%
\pgfsetstrokeopacity{0.700000}%
\pgfsetdash{}{0pt}%
\pgfpathmoveto{\pgfqpoint{8.207386in}{0.550000in}}%
\pgfpathlineto{\pgfqpoint{8.234908in}{0.550000in}}%
\pgfpathlineto{\pgfqpoint{8.234908in}{0.550000in}}%
\pgfpathlineto{\pgfqpoint{8.207386in}{0.550000in}}%
\pgfpathlineto{\pgfqpoint{8.207386in}{0.550000in}}%
\pgfpathclose%
\pgfusepath{fill}%
\end{pgfscope}%
\begin{pgfscope}%
\pgfpathrectangle{\pgfqpoint{1.250000in}{0.550000in}}{\pgfqpoint{7.750000in}{3.850000in}}%
\pgfusepath{clip}%
\pgfsetbuttcap%
\pgfsetmiterjoin%
\definecolor{currentfill}{rgb}{0.501961,0.501961,0.501961}%
\pgfsetfillcolor{currentfill}%
\pgfsetfillopacity{0.700000}%
\pgfsetlinewidth{0.000000pt}%
\definecolor{currentstroke}{rgb}{0.000000,0.000000,0.000000}%
\pgfsetstrokecolor{currentstroke}%
\pgfsetstrokeopacity{0.700000}%
\pgfsetdash{}{0pt}%
\pgfpathmoveto{\pgfqpoint{8.234908in}{0.550000in}}%
\pgfpathlineto{\pgfqpoint{8.262429in}{0.550000in}}%
\pgfpathlineto{\pgfqpoint{8.262429in}{0.550000in}}%
\pgfpathlineto{\pgfqpoint{8.234908in}{0.550000in}}%
\pgfpathlineto{\pgfqpoint{8.234908in}{0.550000in}}%
\pgfpathclose%
\pgfusepath{fill}%
\end{pgfscope}%
\begin{pgfscope}%
\pgfpathrectangle{\pgfqpoint{1.250000in}{0.550000in}}{\pgfqpoint{7.750000in}{3.850000in}}%
\pgfusepath{clip}%
\pgfsetbuttcap%
\pgfsetmiterjoin%
\definecolor{currentfill}{rgb}{0.501961,0.501961,0.501961}%
\pgfsetfillcolor{currentfill}%
\pgfsetfillopacity{0.700000}%
\pgfsetlinewidth{0.000000pt}%
\definecolor{currentstroke}{rgb}{0.000000,0.000000,0.000000}%
\pgfsetstrokecolor{currentstroke}%
\pgfsetstrokeopacity{0.700000}%
\pgfsetdash{}{0pt}%
\pgfpathmoveto{\pgfqpoint{8.262429in}{0.550000in}}%
\pgfpathlineto{\pgfqpoint{8.289950in}{0.550000in}}%
\pgfpathlineto{\pgfqpoint{8.289950in}{0.550153in}}%
\pgfpathlineto{\pgfqpoint{8.262429in}{0.550153in}}%
\pgfpathlineto{\pgfqpoint{8.262429in}{0.550000in}}%
\pgfpathclose%
\pgfusepath{fill}%
\end{pgfscope}%
\begin{pgfscope}%
\pgfpathrectangle{\pgfqpoint{1.250000in}{0.550000in}}{\pgfqpoint{7.750000in}{3.850000in}}%
\pgfusepath{clip}%
\pgfsetbuttcap%
\pgfsetmiterjoin%
\definecolor{currentfill}{rgb}{0.501961,0.501961,0.501961}%
\pgfsetfillcolor{currentfill}%
\pgfsetfillopacity{0.700000}%
\pgfsetlinewidth{0.000000pt}%
\definecolor{currentstroke}{rgb}{0.000000,0.000000,0.000000}%
\pgfsetstrokecolor{currentstroke}%
\pgfsetstrokeopacity{0.700000}%
\pgfsetdash{}{0pt}%
\pgfpathmoveto{\pgfqpoint{8.289950in}{0.550000in}}%
\pgfpathlineto{\pgfqpoint{8.317472in}{0.550000in}}%
\pgfpathlineto{\pgfqpoint{8.317472in}{0.550000in}}%
\pgfpathlineto{\pgfqpoint{8.289950in}{0.550000in}}%
\pgfpathlineto{\pgfqpoint{8.289950in}{0.550000in}}%
\pgfpathclose%
\pgfusepath{fill}%
\end{pgfscope}%
\begin{pgfscope}%
\pgfpathrectangle{\pgfqpoint{1.250000in}{0.550000in}}{\pgfqpoint{7.750000in}{3.850000in}}%
\pgfusepath{clip}%
\pgfsetbuttcap%
\pgfsetmiterjoin%
\definecolor{currentfill}{rgb}{0.501961,0.501961,0.501961}%
\pgfsetfillcolor{currentfill}%
\pgfsetfillopacity{0.700000}%
\pgfsetlinewidth{0.000000pt}%
\definecolor{currentstroke}{rgb}{0.000000,0.000000,0.000000}%
\pgfsetstrokecolor{currentstroke}%
\pgfsetstrokeopacity{0.700000}%
\pgfsetdash{}{0pt}%
\pgfpathmoveto{\pgfqpoint{8.317472in}{0.550000in}}%
\pgfpathlineto{\pgfqpoint{8.344993in}{0.550000in}}%
\pgfpathlineto{\pgfqpoint{8.344993in}{0.550000in}}%
\pgfpathlineto{\pgfqpoint{8.317472in}{0.550000in}}%
\pgfpathlineto{\pgfqpoint{8.317472in}{0.550000in}}%
\pgfpathclose%
\pgfusepath{fill}%
\end{pgfscope}%
\begin{pgfscope}%
\pgfpathrectangle{\pgfqpoint{1.250000in}{0.550000in}}{\pgfqpoint{7.750000in}{3.850000in}}%
\pgfusepath{clip}%
\pgfsetbuttcap%
\pgfsetmiterjoin%
\definecolor{currentfill}{rgb}{0.501961,0.501961,0.501961}%
\pgfsetfillcolor{currentfill}%
\pgfsetfillopacity{0.700000}%
\pgfsetlinewidth{0.000000pt}%
\definecolor{currentstroke}{rgb}{0.000000,0.000000,0.000000}%
\pgfsetstrokecolor{currentstroke}%
\pgfsetstrokeopacity{0.700000}%
\pgfsetdash{}{0pt}%
\pgfpathmoveto{\pgfqpoint{8.344993in}{0.550000in}}%
\pgfpathlineto{\pgfqpoint{8.372514in}{0.550000in}}%
\pgfpathlineto{\pgfqpoint{8.372514in}{0.550000in}}%
\pgfpathlineto{\pgfqpoint{8.344993in}{0.550000in}}%
\pgfpathlineto{\pgfqpoint{8.344993in}{0.550000in}}%
\pgfpathclose%
\pgfusepath{fill}%
\end{pgfscope}%
\begin{pgfscope}%
\pgfpathrectangle{\pgfqpoint{1.250000in}{0.550000in}}{\pgfqpoint{7.750000in}{3.850000in}}%
\pgfusepath{clip}%
\pgfsetbuttcap%
\pgfsetmiterjoin%
\definecolor{currentfill}{rgb}{0.501961,0.501961,0.501961}%
\pgfsetfillcolor{currentfill}%
\pgfsetfillopacity{0.700000}%
\pgfsetlinewidth{0.000000pt}%
\definecolor{currentstroke}{rgb}{0.000000,0.000000,0.000000}%
\pgfsetstrokecolor{currentstroke}%
\pgfsetstrokeopacity{0.700000}%
\pgfsetdash{}{0pt}%
\pgfpathmoveto{\pgfqpoint{8.372514in}{0.550000in}}%
\pgfpathlineto{\pgfqpoint{8.400036in}{0.550000in}}%
\pgfpathlineto{\pgfqpoint{8.400036in}{0.550000in}}%
\pgfpathlineto{\pgfqpoint{8.372514in}{0.550000in}}%
\pgfpathlineto{\pgfqpoint{8.372514in}{0.550000in}}%
\pgfpathclose%
\pgfusepath{fill}%
\end{pgfscope}%
\begin{pgfscope}%
\pgfpathrectangle{\pgfqpoint{1.250000in}{0.550000in}}{\pgfqpoint{7.750000in}{3.850000in}}%
\pgfusepath{clip}%
\pgfsetbuttcap%
\pgfsetmiterjoin%
\definecolor{currentfill}{rgb}{0.501961,0.501961,0.501961}%
\pgfsetfillcolor{currentfill}%
\pgfsetfillopacity{0.700000}%
\pgfsetlinewidth{0.000000pt}%
\definecolor{currentstroke}{rgb}{0.000000,0.000000,0.000000}%
\pgfsetstrokecolor{currentstroke}%
\pgfsetstrokeopacity{0.700000}%
\pgfsetdash{}{0pt}%
\pgfpathmoveto{\pgfqpoint{8.400036in}{0.550000in}}%
\pgfpathlineto{\pgfqpoint{8.427557in}{0.550000in}}%
\pgfpathlineto{\pgfqpoint{8.427557in}{0.550000in}}%
\pgfpathlineto{\pgfqpoint{8.400036in}{0.550000in}}%
\pgfpathlineto{\pgfqpoint{8.400036in}{0.550000in}}%
\pgfpathclose%
\pgfusepath{fill}%
\end{pgfscope}%
\begin{pgfscope}%
\pgfpathrectangle{\pgfqpoint{1.250000in}{0.550000in}}{\pgfqpoint{7.750000in}{3.850000in}}%
\pgfusepath{clip}%
\pgfsetbuttcap%
\pgfsetmiterjoin%
\definecolor{currentfill}{rgb}{0.501961,0.501961,0.501961}%
\pgfsetfillcolor{currentfill}%
\pgfsetfillopacity{0.700000}%
\pgfsetlinewidth{0.000000pt}%
\definecolor{currentstroke}{rgb}{0.000000,0.000000,0.000000}%
\pgfsetstrokecolor{currentstroke}%
\pgfsetstrokeopacity{0.700000}%
\pgfsetdash{}{0pt}%
\pgfpathmoveto{\pgfqpoint{8.427557in}{0.550000in}}%
\pgfpathlineto{\pgfqpoint{8.455078in}{0.550000in}}%
\pgfpathlineto{\pgfqpoint{8.455078in}{0.550000in}}%
\pgfpathlineto{\pgfqpoint{8.427557in}{0.550000in}}%
\pgfpathlineto{\pgfqpoint{8.427557in}{0.550000in}}%
\pgfpathclose%
\pgfusepath{fill}%
\end{pgfscope}%
\begin{pgfscope}%
\pgfpathrectangle{\pgfqpoint{1.250000in}{0.550000in}}{\pgfqpoint{7.750000in}{3.850000in}}%
\pgfusepath{clip}%
\pgfsetbuttcap%
\pgfsetmiterjoin%
\definecolor{currentfill}{rgb}{0.501961,0.501961,0.501961}%
\pgfsetfillcolor{currentfill}%
\pgfsetfillopacity{0.700000}%
\pgfsetlinewidth{0.000000pt}%
\definecolor{currentstroke}{rgb}{0.000000,0.000000,0.000000}%
\pgfsetstrokecolor{currentstroke}%
\pgfsetstrokeopacity{0.700000}%
\pgfsetdash{}{0pt}%
\pgfpathmoveto{\pgfqpoint{8.455078in}{0.550000in}}%
\pgfpathlineto{\pgfqpoint{8.482599in}{0.550000in}}%
\pgfpathlineto{\pgfqpoint{8.482599in}{0.550000in}}%
\pgfpathlineto{\pgfqpoint{8.455078in}{0.550000in}}%
\pgfpathlineto{\pgfqpoint{8.455078in}{0.550000in}}%
\pgfpathclose%
\pgfusepath{fill}%
\end{pgfscope}%
\begin{pgfscope}%
\pgfpathrectangle{\pgfqpoint{1.250000in}{0.550000in}}{\pgfqpoint{7.750000in}{3.850000in}}%
\pgfusepath{clip}%
\pgfsetbuttcap%
\pgfsetmiterjoin%
\definecolor{currentfill}{rgb}{0.501961,0.501961,0.501961}%
\pgfsetfillcolor{currentfill}%
\pgfsetfillopacity{0.700000}%
\pgfsetlinewidth{0.000000pt}%
\definecolor{currentstroke}{rgb}{0.000000,0.000000,0.000000}%
\pgfsetstrokecolor{currentstroke}%
\pgfsetstrokeopacity{0.700000}%
\pgfsetdash{}{0pt}%
\pgfpathmoveto{\pgfqpoint{8.482599in}{0.550000in}}%
\pgfpathlineto{\pgfqpoint{8.510121in}{0.550000in}}%
\pgfpathlineto{\pgfqpoint{8.510121in}{0.550000in}}%
\pgfpathlineto{\pgfqpoint{8.482599in}{0.550000in}}%
\pgfpathlineto{\pgfqpoint{8.482599in}{0.550000in}}%
\pgfpathclose%
\pgfusepath{fill}%
\end{pgfscope}%
\begin{pgfscope}%
\pgfpathrectangle{\pgfqpoint{1.250000in}{0.550000in}}{\pgfqpoint{7.750000in}{3.850000in}}%
\pgfusepath{clip}%
\pgfsetbuttcap%
\pgfsetmiterjoin%
\definecolor{currentfill}{rgb}{0.501961,0.501961,0.501961}%
\pgfsetfillcolor{currentfill}%
\pgfsetfillopacity{0.700000}%
\pgfsetlinewidth{0.000000pt}%
\definecolor{currentstroke}{rgb}{0.000000,0.000000,0.000000}%
\pgfsetstrokecolor{currentstroke}%
\pgfsetstrokeopacity{0.700000}%
\pgfsetdash{}{0pt}%
\pgfpathmoveto{\pgfqpoint{8.510121in}{0.550000in}}%
\pgfpathlineto{\pgfqpoint{8.537642in}{0.550000in}}%
\pgfpathlineto{\pgfqpoint{8.537642in}{0.550000in}}%
\pgfpathlineto{\pgfqpoint{8.510121in}{0.550000in}}%
\pgfpathlineto{\pgfqpoint{8.510121in}{0.550000in}}%
\pgfpathclose%
\pgfusepath{fill}%
\end{pgfscope}%
\begin{pgfscope}%
\pgfpathrectangle{\pgfqpoint{1.250000in}{0.550000in}}{\pgfqpoint{7.750000in}{3.850000in}}%
\pgfusepath{clip}%
\pgfsetbuttcap%
\pgfsetmiterjoin%
\definecolor{currentfill}{rgb}{0.501961,0.501961,0.501961}%
\pgfsetfillcolor{currentfill}%
\pgfsetfillopacity{0.700000}%
\pgfsetlinewidth{0.000000pt}%
\definecolor{currentstroke}{rgb}{0.000000,0.000000,0.000000}%
\pgfsetstrokecolor{currentstroke}%
\pgfsetstrokeopacity{0.700000}%
\pgfsetdash{}{0pt}%
\pgfpathmoveto{\pgfqpoint{8.537642in}{0.550000in}}%
\pgfpathlineto{\pgfqpoint{8.565163in}{0.550000in}}%
\pgfpathlineto{\pgfqpoint{8.565163in}{0.550000in}}%
\pgfpathlineto{\pgfqpoint{8.537642in}{0.550000in}}%
\pgfpathlineto{\pgfqpoint{8.537642in}{0.550000in}}%
\pgfpathclose%
\pgfusepath{fill}%
\end{pgfscope}%
\begin{pgfscope}%
\pgfpathrectangle{\pgfqpoint{1.250000in}{0.550000in}}{\pgfqpoint{7.750000in}{3.850000in}}%
\pgfusepath{clip}%
\pgfsetbuttcap%
\pgfsetmiterjoin%
\definecolor{currentfill}{rgb}{0.501961,0.501961,0.501961}%
\pgfsetfillcolor{currentfill}%
\pgfsetfillopacity{0.700000}%
\pgfsetlinewidth{0.000000pt}%
\definecolor{currentstroke}{rgb}{0.000000,0.000000,0.000000}%
\pgfsetstrokecolor{currentstroke}%
\pgfsetstrokeopacity{0.700000}%
\pgfsetdash{}{0pt}%
\pgfpathmoveto{\pgfqpoint{8.565163in}{0.550000in}}%
\pgfpathlineto{\pgfqpoint{8.592685in}{0.550000in}}%
\pgfpathlineto{\pgfqpoint{8.592685in}{0.550306in}}%
\pgfpathlineto{\pgfqpoint{8.565163in}{0.550306in}}%
\pgfpathlineto{\pgfqpoint{8.565163in}{0.550000in}}%
\pgfpathclose%
\pgfusepath{fill}%
\end{pgfscope}%
\begin{pgfscope}%
\pgfpathrectangle{\pgfqpoint{1.250000in}{0.550000in}}{\pgfqpoint{7.750000in}{3.850000in}}%
\pgfusepath{clip}%
\pgfsetbuttcap%
\pgfsetmiterjoin%
\definecolor{currentfill}{rgb}{0.501961,0.501961,0.501961}%
\pgfsetfillcolor{currentfill}%
\pgfsetfillopacity{0.700000}%
\pgfsetlinewidth{0.000000pt}%
\definecolor{currentstroke}{rgb}{0.000000,0.000000,0.000000}%
\pgfsetstrokecolor{currentstroke}%
\pgfsetstrokeopacity{0.700000}%
\pgfsetdash{}{0pt}%
\pgfpathmoveto{\pgfqpoint{8.592685in}{0.550000in}}%
\pgfpathlineto{\pgfqpoint{8.620206in}{0.550000in}}%
\pgfpathlineto{\pgfqpoint{8.620206in}{0.550153in}}%
\pgfpathlineto{\pgfqpoint{8.592685in}{0.550153in}}%
\pgfpathlineto{\pgfqpoint{8.592685in}{0.550000in}}%
\pgfpathclose%
\pgfusepath{fill}%
\end{pgfscope}%
\begin{pgfscope}%
\pgfpathrectangle{\pgfqpoint{1.250000in}{0.550000in}}{\pgfqpoint{7.750000in}{3.850000in}}%
\pgfusepath{clip}%
\pgfsetbuttcap%
\pgfsetmiterjoin%
\definecolor{currentfill}{rgb}{0.501961,0.501961,0.501961}%
\pgfsetfillcolor{currentfill}%
\pgfsetfillopacity{0.700000}%
\pgfsetlinewidth{0.000000pt}%
\definecolor{currentstroke}{rgb}{0.000000,0.000000,0.000000}%
\pgfsetstrokecolor{currentstroke}%
\pgfsetstrokeopacity{0.700000}%
\pgfsetdash{}{0pt}%
\pgfpathmoveto{\pgfqpoint{8.620206in}{0.550000in}}%
\pgfpathlineto{\pgfqpoint{8.647727in}{0.550000in}}%
\pgfpathlineto{\pgfqpoint{8.647727in}{0.550612in}}%
\pgfpathlineto{\pgfqpoint{8.620206in}{0.550612in}}%
\pgfpathlineto{\pgfqpoint{8.620206in}{0.550000in}}%
\pgfpathclose%
\pgfusepath{fill}%
\end{pgfscope}%
\begin{pgfscope}%
\pgfsetbuttcap%
\pgfsetroundjoin%
\definecolor{currentfill}{rgb}{0.000000,0.000000,0.000000}%
\pgfsetfillcolor{currentfill}%
\pgfsetlinewidth{0.803000pt}%
\definecolor{currentstroke}{rgb}{0.000000,0.000000,0.000000}%
\pgfsetstrokecolor{currentstroke}%
\pgfsetdash{}{0pt}%
\pgfsys@defobject{currentmarker}{\pgfqpoint{0.000000in}{-0.048611in}}{\pgfqpoint{0.000000in}{0.000000in}}{%
\pgfpathmoveto{\pgfqpoint{0.000000in}{0.000000in}}%
\pgfpathlineto{\pgfqpoint{0.000000in}{-0.048611in}}%
\pgfusepath{stroke,fill}%
}%
\begin{pgfscope}%
\pgfsys@transformshift{1.602273in}{0.550000in}%
\pgfsys@useobject{currentmarker}{}%
\end{pgfscope}%
\end{pgfscope}%
\begin{pgfscope}%
\definecolor{textcolor}{rgb}{0.000000,0.000000,0.000000}%
\pgfsetstrokecolor{textcolor}%
\pgfsetfillcolor{textcolor}%
\pgftext[x=1.602273in,y=0.452778in,,top]{\color{textcolor}{\sffamily\fontsize{10.000000}{12.000000}\selectfont\catcode`\^=\active\def^{\ifmmode\sp\else\^{}\fi}\catcode`\%=\active\def%{\%}0}}%
\end{pgfscope}%
\begin{pgfscope}%
\pgfsetbuttcap%
\pgfsetroundjoin%
\definecolor{currentfill}{rgb}{0.000000,0.000000,0.000000}%
\pgfsetfillcolor{currentfill}%
\pgfsetlinewidth{0.803000pt}%
\definecolor{currentstroke}{rgb}{0.000000,0.000000,0.000000}%
\pgfsetstrokecolor{currentstroke}%
\pgfsetdash{}{0pt}%
\pgfsys@defobject{currentmarker}{\pgfqpoint{0.000000in}{-0.048611in}}{\pgfqpoint{0.000000in}{0.000000in}}{%
\pgfpathmoveto{\pgfqpoint{0.000000in}{0.000000in}}%
\pgfpathlineto{\pgfqpoint{0.000000in}{-0.048611in}}%
\pgfusepath{stroke,fill}%
}%
\begin{pgfscope}%
\pgfsys@transformshift{2.978338in}{0.550000in}%
\pgfsys@useobject{currentmarker}{}%
\end{pgfscope}%
\end{pgfscope}%
\begin{pgfscope}%
\definecolor{textcolor}{rgb}{0.000000,0.000000,0.000000}%
\pgfsetstrokecolor{textcolor}%
\pgfsetfillcolor{textcolor}%
\pgftext[x=2.978338in,y=0.452778in,,top]{\color{textcolor}{\sffamily\fontsize{10.000000}{12.000000}\selectfont\catcode`\^=\active\def^{\ifmmode\sp\else\^{}\fi}\catcode`\%=\active\def%{\%}50}}%
\end{pgfscope}%
\begin{pgfscope}%
\pgfsetbuttcap%
\pgfsetroundjoin%
\definecolor{currentfill}{rgb}{0.000000,0.000000,0.000000}%
\pgfsetfillcolor{currentfill}%
\pgfsetlinewidth{0.803000pt}%
\definecolor{currentstroke}{rgb}{0.000000,0.000000,0.000000}%
\pgfsetstrokecolor{currentstroke}%
\pgfsetdash{}{0pt}%
\pgfsys@defobject{currentmarker}{\pgfqpoint{0.000000in}{-0.048611in}}{\pgfqpoint{0.000000in}{0.000000in}}{%
\pgfpathmoveto{\pgfqpoint{0.000000in}{0.000000in}}%
\pgfpathlineto{\pgfqpoint{0.000000in}{-0.048611in}}%
\pgfusepath{stroke,fill}%
}%
\begin{pgfscope}%
\pgfsys@transformshift{4.354403in}{0.550000in}%
\pgfsys@useobject{currentmarker}{}%
\end{pgfscope}%
\end{pgfscope}%
\begin{pgfscope}%
\definecolor{textcolor}{rgb}{0.000000,0.000000,0.000000}%
\pgfsetstrokecolor{textcolor}%
\pgfsetfillcolor{textcolor}%
\pgftext[x=4.354403in,y=0.452778in,,top]{\color{textcolor}{\sffamily\fontsize{10.000000}{12.000000}\selectfont\catcode`\^=\active\def^{\ifmmode\sp\else\^{}\fi}\catcode`\%=\active\def%{\%}100}}%
\end{pgfscope}%
\begin{pgfscope}%
\pgfsetbuttcap%
\pgfsetroundjoin%
\definecolor{currentfill}{rgb}{0.000000,0.000000,0.000000}%
\pgfsetfillcolor{currentfill}%
\pgfsetlinewidth{0.803000pt}%
\definecolor{currentstroke}{rgb}{0.000000,0.000000,0.000000}%
\pgfsetstrokecolor{currentstroke}%
\pgfsetdash{}{0pt}%
\pgfsys@defobject{currentmarker}{\pgfqpoint{0.000000in}{-0.048611in}}{\pgfqpoint{0.000000in}{0.000000in}}{%
\pgfpathmoveto{\pgfqpoint{0.000000in}{0.000000in}}%
\pgfpathlineto{\pgfqpoint{0.000000in}{-0.048611in}}%
\pgfusepath{stroke,fill}%
}%
\begin{pgfscope}%
\pgfsys@transformshift{5.730469in}{0.550000in}%
\pgfsys@useobject{currentmarker}{}%
\end{pgfscope}%
\end{pgfscope}%
\begin{pgfscope}%
\definecolor{textcolor}{rgb}{0.000000,0.000000,0.000000}%
\pgfsetstrokecolor{textcolor}%
\pgfsetfillcolor{textcolor}%
\pgftext[x=5.730469in,y=0.452778in,,top]{\color{textcolor}{\sffamily\fontsize{10.000000}{12.000000}\selectfont\catcode`\^=\active\def^{\ifmmode\sp\else\^{}\fi}\catcode`\%=\active\def%{\%}150}}%
\end{pgfscope}%
\begin{pgfscope}%
\pgfsetbuttcap%
\pgfsetroundjoin%
\definecolor{currentfill}{rgb}{0.000000,0.000000,0.000000}%
\pgfsetfillcolor{currentfill}%
\pgfsetlinewidth{0.803000pt}%
\definecolor{currentstroke}{rgb}{0.000000,0.000000,0.000000}%
\pgfsetstrokecolor{currentstroke}%
\pgfsetdash{}{0pt}%
\pgfsys@defobject{currentmarker}{\pgfqpoint{0.000000in}{-0.048611in}}{\pgfqpoint{0.000000in}{0.000000in}}{%
\pgfpathmoveto{\pgfqpoint{0.000000in}{0.000000in}}%
\pgfpathlineto{\pgfqpoint{0.000000in}{-0.048611in}}%
\pgfusepath{stroke,fill}%
}%
\begin{pgfscope}%
\pgfsys@transformshift{7.106534in}{0.550000in}%
\pgfsys@useobject{currentmarker}{}%
\end{pgfscope}%
\end{pgfscope}%
\begin{pgfscope}%
\definecolor{textcolor}{rgb}{0.000000,0.000000,0.000000}%
\pgfsetstrokecolor{textcolor}%
\pgfsetfillcolor{textcolor}%
\pgftext[x=7.106534in,y=0.452778in,,top]{\color{textcolor}{\sffamily\fontsize{10.000000}{12.000000}\selectfont\catcode`\^=\active\def^{\ifmmode\sp\else\^{}\fi}\catcode`\%=\active\def%{\%}200}}%
\end{pgfscope}%
\begin{pgfscope}%
\pgfsetbuttcap%
\pgfsetroundjoin%
\definecolor{currentfill}{rgb}{0.000000,0.000000,0.000000}%
\pgfsetfillcolor{currentfill}%
\pgfsetlinewidth{0.803000pt}%
\definecolor{currentstroke}{rgb}{0.000000,0.000000,0.000000}%
\pgfsetstrokecolor{currentstroke}%
\pgfsetdash{}{0pt}%
\pgfsys@defobject{currentmarker}{\pgfqpoint{0.000000in}{-0.048611in}}{\pgfqpoint{0.000000in}{0.000000in}}{%
\pgfpathmoveto{\pgfqpoint{0.000000in}{0.000000in}}%
\pgfpathlineto{\pgfqpoint{0.000000in}{-0.048611in}}%
\pgfusepath{stroke,fill}%
}%
\begin{pgfscope}%
\pgfsys@transformshift{8.482599in}{0.550000in}%
\pgfsys@useobject{currentmarker}{}%
\end{pgfscope}%
\end{pgfscope}%
\begin{pgfscope}%
\definecolor{textcolor}{rgb}{0.000000,0.000000,0.000000}%
\pgfsetstrokecolor{textcolor}%
\pgfsetfillcolor{textcolor}%
\pgftext[x=8.482599in,y=0.452778in,,top]{\color{textcolor}{\sffamily\fontsize{10.000000}{12.000000}\selectfont\catcode`\^=\active\def^{\ifmmode\sp\else\^{}\fi}\catcode`\%=\active\def%{\%}250}}%
\end{pgfscope}%
\begin{pgfscope}%
\definecolor{textcolor}{rgb}{0.000000,0.000000,0.000000}%
\pgfsetstrokecolor{textcolor}%
\pgfsetfillcolor{textcolor}%
\pgftext[x=5.125000in,y=0.262809in,,top]{\color{textcolor}{\sffamily\fontsize{10.000000}{12.000000}\selectfont\catcode`\^=\active\def^{\ifmmode\sp\else\^{}\fi}\catcode`\%=\active\def%{\%}Intensidad}}%
\end{pgfscope}%
\begin{pgfscope}%
\pgfsetbuttcap%
\pgfsetroundjoin%
\definecolor{currentfill}{rgb}{0.000000,0.000000,0.000000}%
\pgfsetfillcolor{currentfill}%
\pgfsetlinewidth{0.803000pt}%
\definecolor{currentstroke}{rgb}{0.000000,0.000000,0.000000}%
\pgfsetstrokecolor{currentstroke}%
\pgfsetdash{}{0pt}%
\pgfsys@defobject{currentmarker}{\pgfqpoint{-0.048611in}{0.000000in}}{\pgfqpoint{-0.000000in}{0.000000in}}{%
\pgfpathmoveto{\pgfqpoint{-0.000000in}{0.000000in}}%
\pgfpathlineto{\pgfqpoint{-0.048611in}{0.000000in}}%
\pgfusepath{stroke,fill}%
}%
\begin{pgfscope}%
\pgfsys@transformshift{1.250000in}{0.550000in}%
\pgfsys@useobject{currentmarker}{}%
\end{pgfscope}%
\end{pgfscope}%
\begin{pgfscope}%
\definecolor{textcolor}{rgb}{0.000000,0.000000,0.000000}%
\pgfsetstrokecolor{textcolor}%
\pgfsetfillcolor{textcolor}%
\pgftext[x=1.064412in, y=0.497238in, left, base]{\color{textcolor}{\sffamily\fontsize{10.000000}{12.000000}\selectfont\catcode`\^=\active\def^{\ifmmode\sp\else\^{}\fi}\catcode`\%=\active\def%{\%}0}}%
\end{pgfscope}%
\begin{pgfscope}%
\pgfsetbuttcap%
\pgfsetroundjoin%
\definecolor{currentfill}{rgb}{0.000000,0.000000,0.000000}%
\pgfsetfillcolor{currentfill}%
\pgfsetlinewidth{0.803000pt}%
\definecolor{currentstroke}{rgb}{0.000000,0.000000,0.000000}%
\pgfsetstrokecolor{currentstroke}%
\pgfsetdash{}{0pt}%
\pgfsys@defobject{currentmarker}{\pgfqpoint{-0.048611in}{0.000000in}}{\pgfqpoint{-0.000000in}{0.000000in}}{%
\pgfpathmoveto{\pgfqpoint{-0.000000in}{0.000000in}}%
\pgfpathlineto{\pgfqpoint{-0.048611in}{0.000000in}}%
\pgfusepath{stroke,fill}%
}%
\begin{pgfscope}%
\pgfsys@transformshift{1.250000in}{1.315548in}%
\pgfsys@useobject{currentmarker}{}%
\end{pgfscope}%
\end{pgfscope}%
\begin{pgfscope}%
\definecolor{textcolor}{rgb}{0.000000,0.000000,0.000000}%
\pgfsetstrokecolor{textcolor}%
\pgfsetfillcolor{textcolor}%
\pgftext[x=0.799316in, y=1.262786in, left, base]{\color{textcolor}{\sffamily\fontsize{10.000000}{12.000000}\selectfont\catcode`\^=\active\def^{\ifmmode\sp\else\^{}\fi}\catcode`\%=\active\def%{\%}5000}}%
\end{pgfscope}%
\begin{pgfscope}%
\pgfsetbuttcap%
\pgfsetroundjoin%
\definecolor{currentfill}{rgb}{0.000000,0.000000,0.000000}%
\pgfsetfillcolor{currentfill}%
\pgfsetlinewidth{0.803000pt}%
\definecolor{currentstroke}{rgb}{0.000000,0.000000,0.000000}%
\pgfsetstrokecolor{currentstroke}%
\pgfsetdash{}{0pt}%
\pgfsys@defobject{currentmarker}{\pgfqpoint{-0.048611in}{0.000000in}}{\pgfqpoint{-0.000000in}{0.000000in}}{%
\pgfpathmoveto{\pgfqpoint{-0.000000in}{0.000000in}}%
\pgfpathlineto{\pgfqpoint{-0.048611in}{0.000000in}}%
\pgfusepath{stroke,fill}%
}%
\begin{pgfscope}%
\pgfsys@transformshift{1.250000in}{2.081095in}%
\pgfsys@useobject{currentmarker}{}%
\end{pgfscope}%
\end{pgfscope}%
\begin{pgfscope}%
\definecolor{textcolor}{rgb}{0.000000,0.000000,0.000000}%
\pgfsetstrokecolor{textcolor}%
\pgfsetfillcolor{textcolor}%
\pgftext[x=0.710951in, y=2.028334in, left, base]{\color{textcolor}{\sffamily\fontsize{10.000000}{12.000000}\selectfont\catcode`\^=\active\def^{\ifmmode\sp\else\^{}\fi}\catcode`\%=\active\def%{\%}10000}}%
\end{pgfscope}%
\begin{pgfscope}%
\pgfsetbuttcap%
\pgfsetroundjoin%
\definecolor{currentfill}{rgb}{0.000000,0.000000,0.000000}%
\pgfsetfillcolor{currentfill}%
\pgfsetlinewidth{0.803000pt}%
\definecolor{currentstroke}{rgb}{0.000000,0.000000,0.000000}%
\pgfsetstrokecolor{currentstroke}%
\pgfsetdash{}{0pt}%
\pgfsys@defobject{currentmarker}{\pgfqpoint{-0.048611in}{0.000000in}}{\pgfqpoint{-0.000000in}{0.000000in}}{%
\pgfpathmoveto{\pgfqpoint{-0.000000in}{0.000000in}}%
\pgfpathlineto{\pgfqpoint{-0.048611in}{0.000000in}}%
\pgfusepath{stroke,fill}%
}%
\begin{pgfscope}%
\pgfsys@transformshift{1.250000in}{2.846643in}%
\pgfsys@useobject{currentmarker}{}%
\end{pgfscope}%
\end{pgfscope}%
\begin{pgfscope}%
\definecolor{textcolor}{rgb}{0.000000,0.000000,0.000000}%
\pgfsetstrokecolor{textcolor}%
\pgfsetfillcolor{textcolor}%
\pgftext[x=0.710951in, y=2.793881in, left, base]{\color{textcolor}{\sffamily\fontsize{10.000000}{12.000000}\selectfont\catcode`\^=\active\def^{\ifmmode\sp\else\^{}\fi}\catcode`\%=\active\def%{\%}15000}}%
\end{pgfscope}%
\begin{pgfscope}%
\pgfsetbuttcap%
\pgfsetroundjoin%
\definecolor{currentfill}{rgb}{0.000000,0.000000,0.000000}%
\pgfsetfillcolor{currentfill}%
\pgfsetlinewidth{0.803000pt}%
\definecolor{currentstroke}{rgb}{0.000000,0.000000,0.000000}%
\pgfsetstrokecolor{currentstroke}%
\pgfsetdash{}{0pt}%
\pgfsys@defobject{currentmarker}{\pgfqpoint{-0.048611in}{0.000000in}}{\pgfqpoint{-0.000000in}{0.000000in}}{%
\pgfpathmoveto{\pgfqpoint{-0.000000in}{0.000000in}}%
\pgfpathlineto{\pgfqpoint{-0.048611in}{0.000000in}}%
\pgfusepath{stroke,fill}%
}%
\begin{pgfscope}%
\pgfsys@transformshift{1.250000in}{3.612190in}%
\pgfsys@useobject{currentmarker}{}%
\end{pgfscope}%
\end{pgfscope}%
\begin{pgfscope}%
\definecolor{textcolor}{rgb}{0.000000,0.000000,0.000000}%
\pgfsetstrokecolor{textcolor}%
\pgfsetfillcolor{textcolor}%
\pgftext[x=0.710951in, y=3.559429in, left, base]{\color{textcolor}{\sffamily\fontsize{10.000000}{12.000000}\selectfont\catcode`\^=\active\def^{\ifmmode\sp\else\^{}\fi}\catcode`\%=\active\def%{\%}20000}}%
\end{pgfscope}%
\begin{pgfscope}%
\pgfsetbuttcap%
\pgfsetroundjoin%
\definecolor{currentfill}{rgb}{0.000000,0.000000,0.000000}%
\pgfsetfillcolor{currentfill}%
\pgfsetlinewidth{0.803000pt}%
\definecolor{currentstroke}{rgb}{0.000000,0.000000,0.000000}%
\pgfsetstrokecolor{currentstroke}%
\pgfsetdash{}{0pt}%
\pgfsys@defobject{currentmarker}{\pgfqpoint{-0.048611in}{0.000000in}}{\pgfqpoint{-0.000000in}{0.000000in}}{%
\pgfpathmoveto{\pgfqpoint{-0.000000in}{0.000000in}}%
\pgfpathlineto{\pgfqpoint{-0.048611in}{0.000000in}}%
\pgfusepath{stroke,fill}%
}%
\begin{pgfscope}%
\pgfsys@transformshift{1.250000in}{4.377738in}%
\pgfsys@useobject{currentmarker}{}%
\end{pgfscope}%
\end{pgfscope}%
\begin{pgfscope}%
\definecolor{textcolor}{rgb}{0.000000,0.000000,0.000000}%
\pgfsetstrokecolor{textcolor}%
\pgfsetfillcolor{textcolor}%
\pgftext[x=0.710951in, y=4.324976in, left, base]{\color{textcolor}{\sffamily\fontsize{10.000000}{12.000000}\selectfont\catcode`\^=\active\def^{\ifmmode\sp\else\^{}\fi}\catcode`\%=\active\def%{\%}25000}}%
\end{pgfscope}%
\begin{pgfscope}%
\definecolor{textcolor}{rgb}{0.000000,0.000000,0.000000}%
\pgfsetstrokecolor{textcolor}%
\pgfsetfillcolor{textcolor}%
\pgftext[x=0.655396in,y=2.475000in,,bottom,rotate=90.000000]{\color{textcolor}{\sffamily\fontsize{10.000000}{12.000000}\selectfont\catcode`\^=\active\def^{\ifmmode\sp\else\^{}\fi}\catcode`\%=\active\def%{\%}Frecuencia}}%
\end{pgfscope}%
\begin{pgfscope}%
\pgfsetrectcap%
\pgfsetmiterjoin%
\pgfsetlinewidth{0.803000pt}%
\definecolor{currentstroke}{rgb}{0.000000,0.000000,0.000000}%
\pgfsetstrokecolor{currentstroke}%
\pgfsetdash{}{0pt}%
\pgfpathmoveto{\pgfqpoint{1.250000in}{0.550000in}}%
\pgfpathlineto{\pgfqpoint{1.250000in}{4.400000in}}%
\pgfusepath{stroke}%
\end{pgfscope}%
\begin{pgfscope}%
\pgfsetrectcap%
\pgfsetmiterjoin%
\pgfsetlinewidth{0.803000pt}%
\definecolor{currentstroke}{rgb}{0.000000,0.000000,0.000000}%
\pgfsetstrokecolor{currentstroke}%
\pgfsetdash{}{0pt}%
\pgfpathmoveto{\pgfqpoint{9.000000in}{0.550000in}}%
\pgfpathlineto{\pgfqpoint{9.000000in}{4.400000in}}%
\pgfusepath{stroke}%
\end{pgfscope}%
\begin{pgfscope}%
\pgfsetrectcap%
\pgfsetmiterjoin%
\pgfsetlinewidth{0.803000pt}%
\definecolor{currentstroke}{rgb}{0.000000,0.000000,0.000000}%
\pgfsetstrokecolor{currentstroke}%
\pgfsetdash{}{0pt}%
\pgfpathmoveto{\pgfqpoint{1.250000in}{0.550000in}}%
\pgfpathlineto{\pgfqpoint{9.000000in}{0.550000in}}%
\pgfusepath{stroke}%
\end{pgfscope}%
\begin{pgfscope}%
\pgfsetrectcap%
\pgfsetmiterjoin%
\pgfsetlinewidth{0.803000pt}%
\definecolor{currentstroke}{rgb}{0.000000,0.000000,0.000000}%
\pgfsetstrokecolor{currentstroke}%
\pgfsetdash{}{0pt}%
\pgfpathmoveto{\pgfqpoint{1.250000in}{4.400000in}}%
\pgfpathlineto{\pgfqpoint{9.000000in}{4.400000in}}%
\pgfusepath{stroke}%
\end{pgfscope}%
\begin{pgfscope}%
\definecolor{textcolor}{rgb}{0.000000,0.000000,0.000000}%
\pgfsetstrokecolor{textcolor}%
\pgfsetfillcolor{textcolor}%
\pgftext[x=5.125000in,y=4.483333in,,base]{\color{textcolor}{\sffamily\fontsize{12.000000}{14.400000}\selectfont\catcode`\^=\active\def^{\ifmmode\sp\else\^{}\fi}\catcode`\%=\active\def%{\%}Histograma de CLAHE}}%
\end{pgfscope}%
\end{pgfpicture}%
\makeatother%
\endgroup%
}
		\caption{CLAHE}
		\label{fig:clahe_1}
	\end{subfigure}

	% Tercera fila (1 histograma centrado)
	\vspace{5mm}
	\centering
	\begin{subfigure}[b]{0.45\textwidth}
		\resizebox{\linewidth}{!}{%% Creator: Matplotlib, PGF backend
%%
%% To include the figure in your LaTeX document, write
%%   \input{<filename>.pgf}
%%
%% Make sure the required packages are loaded in your preamble
%%   \usepackage{pgf}
%%
%% Also ensure that all the required font packages are loaded; for instance,
%% the lmodern package is sometimes necessary when using math font.
%%   \usepackage{lmodern}
%%
%% Figures using additional raster images can only be included by \input if
%% they are in the same directory as the main LaTeX file. For loading figures
%% from other directories you can use the `import` package
%%   \usepackage{import}
%%
%% and then include the figures with
%%   \import{<path to file>}{<filename>.pgf}
%%
%% Matplotlib used the following preamble
%%   \def\mathdefault#1{#1}
%%   \everymath=\expandafter{\the\everymath\displaystyle}
%%   \IfFileExists{scrextend.sty}{
%%     \usepackage[fontsize=10.000000pt]{scrextend}
%%   }{
%%     \renewcommand{\normalsize}{\fontsize{10.000000}{12.000000}\selectfont}
%%     \normalsize
%%   }
%%   
%%   \ifdefined\pdftexversion\else  % non-pdftex case.
%%     \usepackage{fontspec}
%%     \setmainfont{DejaVuSerif.ttf}[Path=\detokenize{/home/elias/.virtualenvs/src-pway/lib/python3.13/site-packages/matplotlib/mpl-data/fonts/ttf/}]
%%     \setsansfont{DejaVuSans.ttf}[Path=\detokenize{/home/elias/.virtualenvs/src-pway/lib/python3.13/site-packages/matplotlib/mpl-data/fonts/ttf/}]
%%     \setmonofont{DejaVuSansMono.ttf}[Path=\detokenize{/home/elias/.virtualenvs/src-pway/lib/python3.13/site-packages/matplotlib/mpl-data/fonts/ttf/}]
%%   \fi
%%   \makeatletter\@ifpackageloaded{underscore}{}{\usepackage[strings]{underscore}}\makeatother
%%
\begingroup%
\makeatletter%
\begin{pgfpicture}%
\pgfpathrectangle{\pgfpointorigin}{\pgfqpoint{10.000000in}{5.000000in}}%
\pgfusepath{use as bounding box, clip}%
\begin{pgfscope}%
\pgfsetbuttcap%
\pgfsetmiterjoin%
\definecolor{currentfill}{rgb}{1.000000,1.000000,1.000000}%
\pgfsetfillcolor{currentfill}%
\pgfsetlinewidth{0.000000pt}%
\definecolor{currentstroke}{rgb}{1.000000,1.000000,1.000000}%
\pgfsetstrokecolor{currentstroke}%
\pgfsetdash{}{0pt}%
\pgfpathmoveto{\pgfqpoint{0.000000in}{0.000000in}}%
\pgfpathlineto{\pgfqpoint{10.000000in}{0.000000in}}%
\pgfpathlineto{\pgfqpoint{10.000000in}{5.000000in}}%
\pgfpathlineto{\pgfqpoint{0.000000in}{5.000000in}}%
\pgfpathlineto{\pgfqpoint{0.000000in}{0.000000in}}%
\pgfpathclose%
\pgfusepath{fill}%
\end{pgfscope}%
\begin{pgfscope}%
\pgfsetbuttcap%
\pgfsetmiterjoin%
\definecolor{currentfill}{rgb}{1.000000,1.000000,1.000000}%
\pgfsetfillcolor{currentfill}%
\pgfsetlinewidth{0.000000pt}%
\definecolor{currentstroke}{rgb}{0.000000,0.000000,0.000000}%
\pgfsetstrokecolor{currentstroke}%
\pgfsetstrokeopacity{0.000000}%
\pgfsetdash{}{0pt}%
\pgfpathmoveto{\pgfqpoint{1.250000in}{0.550000in}}%
\pgfpathlineto{\pgfqpoint{9.000000in}{0.550000in}}%
\pgfpathlineto{\pgfqpoint{9.000000in}{4.400000in}}%
\pgfpathlineto{\pgfqpoint{1.250000in}{4.400000in}}%
\pgfpathlineto{\pgfqpoint{1.250000in}{0.550000in}}%
\pgfpathclose%
\pgfusepath{fill}%
\end{pgfscope}%
\begin{pgfscope}%
\pgfpathrectangle{\pgfqpoint{1.250000in}{0.550000in}}{\pgfqpoint{7.750000in}{3.850000in}}%
\pgfusepath{clip}%
\pgfsetbuttcap%
\pgfsetmiterjoin%
\definecolor{currentfill}{rgb}{0.501961,0.501961,0.501961}%
\pgfsetfillcolor{currentfill}%
\pgfsetfillopacity{0.700000}%
\pgfsetlinewidth{0.000000pt}%
\definecolor{currentstroke}{rgb}{0.000000,0.000000,0.000000}%
\pgfsetstrokecolor{currentstroke}%
\pgfsetstrokeopacity{0.700000}%
\pgfsetdash{}{0pt}%
\pgfpathmoveto{\pgfqpoint{1.602273in}{0.550000in}}%
\pgfpathlineto{\pgfqpoint{1.629794in}{0.550000in}}%
\pgfpathlineto{\pgfqpoint{1.629794in}{0.550000in}}%
\pgfpathlineto{\pgfqpoint{1.602273in}{0.550000in}}%
\pgfpathlineto{\pgfqpoint{1.602273in}{0.550000in}}%
\pgfpathclose%
\pgfusepath{fill}%
\end{pgfscope}%
\begin{pgfscope}%
\pgfpathrectangle{\pgfqpoint{1.250000in}{0.550000in}}{\pgfqpoint{7.750000in}{3.850000in}}%
\pgfusepath{clip}%
\pgfsetbuttcap%
\pgfsetmiterjoin%
\definecolor{currentfill}{rgb}{0.501961,0.501961,0.501961}%
\pgfsetfillcolor{currentfill}%
\pgfsetfillopacity{0.700000}%
\pgfsetlinewidth{0.000000pt}%
\definecolor{currentstroke}{rgb}{0.000000,0.000000,0.000000}%
\pgfsetstrokecolor{currentstroke}%
\pgfsetstrokeopacity{0.700000}%
\pgfsetdash{}{0pt}%
\pgfpathmoveto{\pgfqpoint{1.629794in}{0.550000in}}%
\pgfpathlineto{\pgfqpoint{1.657315in}{0.550000in}}%
\pgfpathlineto{\pgfqpoint{1.657315in}{1.579420in}}%
\pgfpathlineto{\pgfqpoint{1.629794in}{1.579420in}}%
\pgfpathlineto{\pgfqpoint{1.629794in}{0.550000in}}%
\pgfpathclose%
\pgfusepath{fill}%
\end{pgfscope}%
\begin{pgfscope}%
\pgfpathrectangle{\pgfqpoint{1.250000in}{0.550000in}}{\pgfqpoint{7.750000in}{3.850000in}}%
\pgfusepath{clip}%
\pgfsetbuttcap%
\pgfsetmiterjoin%
\definecolor{currentfill}{rgb}{0.501961,0.501961,0.501961}%
\pgfsetfillcolor{currentfill}%
\pgfsetfillopacity{0.700000}%
\pgfsetlinewidth{0.000000pt}%
\definecolor{currentstroke}{rgb}{0.000000,0.000000,0.000000}%
\pgfsetstrokecolor{currentstroke}%
\pgfsetstrokeopacity{0.700000}%
\pgfsetdash{}{0pt}%
\pgfpathmoveto{\pgfqpoint{1.657315in}{0.550000in}}%
\pgfpathlineto{\pgfqpoint{1.684837in}{0.550000in}}%
\pgfpathlineto{\pgfqpoint{1.684837in}{4.216667in}}%
\pgfpathlineto{\pgfqpoint{1.657315in}{4.216667in}}%
\pgfpathlineto{\pgfqpoint{1.657315in}{0.550000in}}%
\pgfpathclose%
\pgfusepath{fill}%
\end{pgfscope}%
\begin{pgfscope}%
\pgfpathrectangle{\pgfqpoint{1.250000in}{0.550000in}}{\pgfqpoint{7.750000in}{3.850000in}}%
\pgfusepath{clip}%
\pgfsetbuttcap%
\pgfsetmiterjoin%
\definecolor{currentfill}{rgb}{0.501961,0.501961,0.501961}%
\pgfsetfillcolor{currentfill}%
\pgfsetfillopacity{0.700000}%
\pgfsetlinewidth{0.000000pt}%
\definecolor{currentstroke}{rgb}{0.000000,0.000000,0.000000}%
\pgfsetstrokecolor{currentstroke}%
\pgfsetstrokeopacity{0.700000}%
\pgfsetdash{}{0pt}%
\pgfpathmoveto{\pgfqpoint{1.684837in}{0.550000in}}%
\pgfpathlineto{\pgfqpoint{1.712358in}{0.550000in}}%
\pgfpathlineto{\pgfqpoint{1.712358in}{2.976377in}}%
\pgfpathlineto{\pgfqpoint{1.684837in}{2.976377in}}%
\pgfpathlineto{\pgfqpoint{1.684837in}{0.550000in}}%
\pgfpathclose%
\pgfusepath{fill}%
\end{pgfscope}%
\begin{pgfscope}%
\pgfpathrectangle{\pgfqpoint{1.250000in}{0.550000in}}{\pgfqpoint{7.750000in}{3.850000in}}%
\pgfusepath{clip}%
\pgfsetbuttcap%
\pgfsetmiterjoin%
\definecolor{currentfill}{rgb}{0.501961,0.501961,0.501961}%
\pgfsetfillcolor{currentfill}%
\pgfsetfillopacity{0.700000}%
\pgfsetlinewidth{0.000000pt}%
\definecolor{currentstroke}{rgb}{0.000000,0.000000,0.000000}%
\pgfsetstrokecolor{currentstroke}%
\pgfsetstrokeopacity{0.700000}%
\pgfsetdash{}{0pt}%
\pgfpathmoveto{\pgfqpoint{1.712358in}{0.550000in}}%
\pgfpathlineto{\pgfqpoint{1.739879in}{0.550000in}}%
\pgfpathlineto{\pgfqpoint{1.739879in}{1.805323in}}%
\pgfpathlineto{\pgfqpoint{1.712358in}{1.805323in}}%
\pgfpathlineto{\pgfqpoint{1.712358in}{0.550000in}}%
\pgfpathclose%
\pgfusepath{fill}%
\end{pgfscope}%
\begin{pgfscope}%
\pgfpathrectangle{\pgfqpoint{1.250000in}{0.550000in}}{\pgfqpoint{7.750000in}{3.850000in}}%
\pgfusepath{clip}%
\pgfsetbuttcap%
\pgfsetmiterjoin%
\definecolor{currentfill}{rgb}{0.501961,0.501961,0.501961}%
\pgfsetfillcolor{currentfill}%
\pgfsetfillopacity{0.700000}%
\pgfsetlinewidth{0.000000pt}%
\definecolor{currentstroke}{rgb}{0.000000,0.000000,0.000000}%
\pgfsetstrokecolor{currentstroke}%
\pgfsetstrokeopacity{0.700000}%
\pgfsetdash{}{0pt}%
\pgfpathmoveto{\pgfqpoint{1.739879in}{0.550000in}}%
\pgfpathlineto{\pgfqpoint{1.767401in}{0.550000in}}%
\pgfpathlineto{\pgfqpoint{1.767401in}{1.526011in}}%
\pgfpathlineto{\pgfqpoint{1.739879in}{1.526011in}}%
\pgfpathlineto{\pgfqpoint{1.739879in}{0.550000in}}%
\pgfpathclose%
\pgfusepath{fill}%
\end{pgfscope}%
\begin{pgfscope}%
\pgfpathrectangle{\pgfqpoint{1.250000in}{0.550000in}}{\pgfqpoint{7.750000in}{3.850000in}}%
\pgfusepath{clip}%
\pgfsetbuttcap%
\pgfsetmiterjoin%
\definecolor{currentfill}{rgb}{0.501961,0.501961,0.501961}%
\pgfsetfillcolor{currentfill}%
\pgfsetfillopacity{0.700000}%
\pgfsetlinewidth{0.000000pt}%
\definecolor{currentstroke}{rgb}{0.000000,0.000000,0.000000}%
\pgfsetstrokecolor{currentstroke}%
\pgfsetstrokeopacity{0.700000}%
\pgfsetdash{}{0pt}%
\pgfpathmoveto{\pgfqpoint{1.767401in}{0.550000in}}%
\pgfpathlineto{\pgfqpoint{1.794922in}{0.550000in}}%
\pgfpathlineto{\pgfqpoint{1.794922in}{1.324637in}}%
\pgfpathlineto{\pgfqpoint{1.767401in}{1.324637in}}%
\pgfpathlineto{\pgfqpoint{1.767401in}{0.550000in}}%
\pgfpathclose%
\pgfusepath{fill}%
\end{pgfscope}%
\begin{pgfscope}%
\pgfpathrectangle{\pgfqpoint{1.250000in}{0.550000in}}{\pgfqpoint{7.750000in}{3.850000in}}%
\pgfusepath{clip}%
\pgfsetbuttcap%
\pgfsetmiterjoin%
\definecolor{currentfill}{rgb}{0.501961,0.501961,0.501961}%
\pgfsetfillcolor{currentfill}%
\pgfsetfillopacity{0.700000}%
\pgfsetlinewidth{0.000000pt}%
\definecolor{currentstroke}{rgb}{0.000000,0.000000,0.000000}%
\pgfsetstrokecolor{currentstroke}%
\pgfsetstrokeopacity{0.700000}%
\pgfsetdash{}{0pt}%
\pgfpathmoveto{\pgfqpoint{1.794922in}{0.550000in}}%
\pgfpathlineto{\pgfqpoint{1.822443in}{0.550000in}}%
\pgfpathlineto{\pgfqpoint{1.822443in}{1.264106in}}%
\pgfpathlineto{\pgfqpoint{1.794922in}{1.264106in}}%
\pgfpathlineto{\pgfqpoint{1.794922in}{0.550000in}}%
\pgfpathclose%
\pgfusepath{fill}%
\end{pgfscope}%
\begin{pgfscope}%
\pgfpathrectangle{\pgfqpoint{1.250000in}{0.550000in}}{\pgfqpoint{7.750000in}{3.850000in}}%
\pgfusepath{clip}%
\pgfsetbuttcap%
\pgfsetmiterjoin%
\definecolor{currentfill}{rgb}{0.501961,0.501961,0.501961}%
\pgfsetfillcolor{currentfill}%
\pgfsetfillopacity{0.700000}%
\pgfsetlinewidth{0.000000pt}%
\definecolor{currentstroke}{rgb}{0.000000,0.000000,0.000000}%
\pgfsetstrokecolor{currentstroke}%
\pgfsetstrokeopacity{0.700000}%
\pgfsetdash{}{0pt}%
\pgfpathmoveto{\pgfqpoint{1.822443in}{0.550000in}}%
\pgfpathlineto{\pgfqpoint{1.849964in}{0.550000in}}%
\pgfpathlineto{\pgfqpoint{1.849964in}{1.248281in}}%
\pgfpathlineto{\pgfqpoint{1.822443in}{1.248281in}}%
\pgfpathlineto{\pgfqpoint{1.822443in}{0.550000in}}%
\pgfpathclose%
\pgfusepath{fill}%
\end{pgfscope}%
\begin{pgfscope}%
\pgfpathrectangle{\pgfqpoint{1.250000in}{0.550000in}}{\pgfqpoint{7.750000in}{3.850000in}}%
\pgfusepath{clip}%
\pgfsetbuttcap%
\pgfsetmiterjoin%
\definecolor{currentfill}{rgb}{0.501961,0.501961,0.501961}%
\pgfsetfillcolor{currentfill}%
\pgfsetfillopacity{0.700000}%
\pgfsetlinewidth{0.000000pt}%
\definecolor{currentstroke}{rgb}{0.000000,0.000000,0.000000}%
\pgfsetstrokecolor{currentstroke}%
\pgfsetstrokeopacity{0.700000}%
\pgfsetdash{}{0pt}%
\pgfpathmoveto{\pgfqpoint{1.849964in}{0.550000in}}%
\pgfpathlineto{\pgfqpoint{1.877486in}{0.550000in}}%
\pgfpathlineto{\pgfqpoint{1.877486in}{1.290217in}}%
\pgfpathlineto{\pgfqpoint{1.849964in}{1.290217in}}%
\pgfpathlineto{\pgfqpoint{1.849964in}{0.550000in}}%
\pgfpathclose%
\pgfusepath{fill}%
\end{pgfscope}%
\begin{pgfscope}%
\pgfpathrectangle{\pgfqpoint{1.250000in}{0.550000in}}{\pgfqpoint{7.750000in}{3.850000in}}%
\pgfusepath{clip}%
\pgfsetbuttcap%
\pgfsetmiterjoin%
\definecolor{currentfill}{rgb}{0.501961,0.501961,0.501961}%
\pgfsetfillcolor{currentfill}%
\pgfsetfillopacity{0.700000}%
\pgfsetlinewidth{0.000000pt}%
\definecolor{currentstroke}{rgb}{0.000000,0.000000,0.000000}%
\pgfsetstrokecolor{currentstroke}%
\pgfsetstrokeopacity{0.700000}%
\pgfsetdash{}{0pt}%
\pgfpathmoveto{\pgfqpoint{1.877486in}{0.550000in}}%
\pgfpathlineto{\pgfqpoint{1.905007in}{0.550000in}}%
\pgfpathlineto{\pgfqpoint{1.905007in}{0.550000in}}%
\pgfpathlineto{\pgfqpoint{1.877486in}{0.550000in}}%
\pgfpathlineto{\pgfqpoint{1.877486in}{0.550000in}}%
\pgfpathclose%
\pgfusepath{fill}%
\end{pgfscope}%
\begin{pgfscope}%
\pgfpathrectangle{\pgfqpoint{1.250000in}{0.550000in}}{\pgfqpoint{7.750000in}{3.850000in}}%
\pgfusepath{clip}%
\pgfsetbuttcap%
\pgfsetmiterjoin%
\definecolor{currentfill}{rgb}{0.501961,0.501961,0.501961}%
\pgfsetfillcolor{currentfill}%
\pgfsetfillopacity{0.700000}%
\pgfsetlinewidth{0.000000pt}%
\definecolor{currentstroke}{rgb}{0.000000,0.000000,0.000000}%
\pgfsetstrokecolor{currentstroke}%
\pgfsetstrokeopacity{0.700000}%
\pgfsetdash{}{0pt}%
\pgfpathmoveto{\pgfqpoint{1.905007in}{0.550000in}}%
\pgfpathlineto{\pgfqpoint{1.932528in}{0.550000in}}%
\pgfpathlineto{\pgfqpoint{1.932528in}{1.256193in}}%
\pgfpathlineto{\pgfqpoint{1.905007in}{1.256193in}}%
\pgfpathlineto{\pgfqpoint{1.905007in}{0.550000in}}%
\pgfpathclose%
\pgfusepath{fill}%
\end{pgfscope}%
\begin{pgfscope}%
\pgfpathrectangle{\pgfqpoint{1.250000in}{0.550000in}}{\pgfqpoint{7.750000in}{3.850000in}}%
\pgfusepath{clip}%
\pgfsetbuttcap%
\pgfsetmiterjoin%
\definecolor{currentfill}{rgb}{0.501961,0.501961,0.501961}%
\pgfsetfillcolor{currentfill}%
\pgfsetfillopacity{0.700000}%
\pgfsetlinewidth{0.000000pt}%
\definecolor{currentstroke}{rgb}{0.000000,0.000000,0.000000}%
\pgfsetstrokecolor{currentstroke}%
\pgfsetstrokeopacity{0.700000}%
\pgfsetdash{}{0pt}%
\pgfpathmoveto{\pgfqpoint{1.932528in}{0.550000in}}%
\pgfpathlineto{\pgfqpoint{1.960050in}{0.550000in}}%
\pgfpathlineto{\pgfqpoint{1.960050in}{1.258963in}}%
\pgfpathlineto{\pgfqpoint{1.932528in}{1.258963in}}%
\pgfpathlineto{\pgfqpoint{1.932528in}{0.550000in}}%
\pgfpathclose%
\pgfusepath{fill}%
\end{pgfscope}%
\begin{pgfscope}%
\pgfpathrectangle{\pgfqpoint{1.250000in}{0.550000in}}{\pgfqpoint{7.750000in}{3.850000in}}%
\pgfusepath{clip}%
\pgfsetbuttcap%
\pgfsetmiterjoin%
\definecolor{currentfill}{rgb}{0.501961,0.501961,0.501961}%
\pgfsetfillcolor{currentfill}%
\pgfsetfillopacity{0.700000}%
\pgfsetlinewidth{0.000000pt}%
\definecolor{currentstroke}{rgb}{0.000000,0.000000,0.000000}%
\pgfsetstrokecolor{currentstroke}%
\pgfsetstrokeopacity{0.700000}%
\pgfsetdash{}{0pt}%
\pgfpathmoveto{\pgfqpoint{1.960050in}{0.550000in}}%
\pgfpathlineto{\pgfqpoint{1.987571in}{0.550000in}}%
\pgfpathlineto{\pgfqpoint{1.987571in}{1.819961in}}%
\pgfpathlineto{\pgfqpoint{1.960050in}{1.819961in}}%
\pgfpathlineto{\pgfqpoint{1.960050in}{0.550000in}}%
\pgfpathclose%
\pgfusepath{fill}%
\end{pgfscope}%
\begin{pgfscope}%
\pgfpathrectangle{\pgfqpoint{1.250000in}{0.550000in}}{\pgfqpoint{7.750000in}{3.850000in}}%
\pgfusepath{clip}%
\pgfsetbuttcap%
\pgfsetmiterjoin%
\definecolor{currentfill}{rgb}{0.501961,0.501961,0.501961}%
\pgfsetfillcolor{currentfill}%
\pgfsetfillopacity{0.700000}%
\pgfsetlinewidth{0.000000pt}%
\definecolor{currentstroke}{rgb}{0.000000,0.000000,0.000000}%
\pgfsetstrokecolor{currentstroke}%
\pgfsetstrokeopacity{0.700000}%
\pgfsetdash{}{0pt}%
\pgfpathmoveto{\pgfqpoint{1.987571in}{0.550000in}}%
\pgfpathlineto{\pgfqpoint{2.015092in}{0.550000in}}%
\pgfpathlineto{\pgfqpoint{2.015092in}{1.143044in}}%
\pgfpathlineto{\pgfqpoint{1.987571in}{1.143044in}}%
\pgfpathlineto{\pgfqpoint{1.987571in}{0.550000in}}%
\pgfpathclose%
\pgfusepath{fill}%
\end{pgfscope}%
\begin{pgfscope}%
\pgfpathrectangle{\pgfqpoint{1.250000in}{0.550000in}}{\pgfqpoint{7.750000in}{3.850000in}}%
\pgfusepath{clip}%
\pgfsetbuttcap%
\pgfsetmiterjoin%
\definecolor{currentfill}{rgb}{0.501961,0.501961,0.501961}%
\pgfsetfillcolor{currentfill}%
\pgfsetfillopacity{0.700000}%
\pgfsetlinewidth{0.000000pt}%
\definecolor{currentstroke}{rgb}{0.000000,0.000000,0.000000}%
\pgfsetstrokecolor{currentstroke}%
\pgfsetstrokeopacity{0.700000}%
\pgfsetdash{}{0pt}%
\pgfpathmoveto{\pgfqpoint{2.015092in}{0.550000in}}%
\pgfpathlineto{\pgfqpoint{2.042614in}{0.550000in}}%
\pgfpathlineto{\pgfqpoint{2.042614in}{1.129593in}}%
\pgfpathlineto{\pgfqpoint{2.015092in}{1.129593in}}%
\pgfpathlineto{\pgfqpoint{2.015092in}{0.550000in}}%
\pgfpathclose%
\pgfusepath{fill}%
\end{pgfscope}%
\begin{pgfscope}%
\pgfpathrectangle{\pgfqpoint{1.250000in}{0.550000in}}{\pgfqpoint{7.750000in}{3.850000in}}%
\pgfusepath{clip}%
\pgfsetbuttcap%
\pgfsetmiterjoin%
\definecolor{currentfill}{rgb}{0.501961,0.501961,0.501961}%
\pgfsetfillcolor{currentfill}%
\pgfsetfillopacity{0.700000}%
\pgfsetlinewidth{0.000000pt}%
\definecolor{currentstroke}{rgb}{0.000000,0.000000,0.000000}%
\pgfsetstrokecolor{currentstroke}%
\pgfsetstrokeopacity{0.700000}%
\pgfsetdash{}{0pt}%
\pgfpathmoveto{\pgfqpoint{2.042614in}{0.550000in}}%
\pgfpathlineto{\pgfqpoint{2.070135in}{0.550000in}}%
\pgfpathlineto{\pgfqpoint{2.070135in}{1.086074in}}%
\pgfpathlineto{\pgfqpoint{2.042614in}{1.086074in}}%
\pgfpathlineto{\pgfqpoint{2.042614in}{0.550000in}}%
\pgfpathclose%
\pgfusepath{fill}%
\end{pgfscope}%
\begin{pgfscope}%
\pgfpathrectangle{\pgfqpoint{1.250000in}{0.550000in}}{\pgfqpoint{7.750000in}{3.850000in}}%
\pgfusepath{clip}%
\pgfsetbuttcap%
\pgfsetmiterjoin%
\definecolor{currentfill}{rgb}{0.501961,0.501961,0.501961}%
\pgfsetfillcolor{currentfill}%
\pgfsetfillopacity{0.700000}%
\pgfsetlinewidth{0.000000pt}%
\definecolor{currentstroke}{rgb}{0.000000,0.000000,0.000000}%
\pgfsetstrokecolor{currentstroke}%
\pgfsetstrokeopacity{0.700000}%
\pgfsetdash{}{0pt}%
\pgfpathmoveto{\pgfqpoint{2.070135in}{0.550000in}}%
\pgfpathlineto{\pgfqpoint{2.097656in}{0.550000in}}%
\pgfpathlineto{\pgfqpoint{2.097656in}{1.580211in}}%
\pgfpathlineto{\pgfqpoint{2.070135in}{1.580211in}}%
\pgfpathlineto{\pgfqpoint{2.070135in}{0.550000in}}%
\pgfpathclose%
\pgfusepath{fill}%
\end{pgfscope}%
\begin{pgfscope}%
\pgfpathrectangle{\pgfqpoint{1.250000in}{0.550000in}}{\pgfqpoint{7.750000in}{3.850000in}}%
\pgfusepath{clip}%
\pgfsetbuttcap%
\pgfsetmiterjoin%
\definecolor{currentfill}{rgb}{0.501961,0.501961,0.501961}%
\pgfsetfillcolor{currentfill}%
\pgfsetfillopacity{0.700000}%
\pgfsetlinewidth{0.000000pt}%
\definecolor{currentstroke}{rgb}{0.000000,0.000000,0.000000}%
\pgfsetstrokecolor{currentstroke}%
\pgfsetstrokeopacity{0.700000}%
\pgfsetdash{}{0pt}%
\pgfpathmoveto{\pgfqpoint{2.097656in}{0.550000in}}%
\pgfpathlineto{\pgfqpoint{2.125178in}{0.550000in}}%
\pgfpathlineto{\pgfqpoint{2.125178in}{1.048885in}}%
\pgfpathlineto{\pgfqpoint{2.097656in}{1.048885in}}%
\pgfpathlineto{\pgfqpoint{2.097656in}{0.550000in}}%
\pgfpathclose%
\pgfusepath{fill}%
\end{pgfscope}%
\begin{pgfscope}%
\pgfpathrectangle{\pgfqpoint{1.250000in}{0.550000in}}{\pgfqpoint{7.750000in}{3.850000in}}%
\pgfusepath{clip}%
\pgfsetbuttcap%
\pgfsetmiterjoin%
\definecolor{currentfill}{rgb}{0.501961,0.501961,0.501961}%
\pgfsetfillcolor{currentfill}%
\pgfsetfillopacity{0.700000}%
\pgfsetlinewidth{0.000000pt}%
\definecolor{currentstroke}{rgb}{0.000000,0.000000,0.000000}%
\pgfsetstrokecolor{currentstroke}%
\pgfsetstrokeopacity{0.700000}%
\pgfsetdash{}{0pt}%
\pgfpathmoveto{\pgfqpoint{2.125178in}{0.550000in}}%
\pgfpathlineto{\pgfqpoint{2.152699in}{0.550000in}}%
\pgfpathlineto{\pgfqpoint{2.152699in}{1.068666in}}%
\pgfpathlineto{\pgfqpoint{2.125178in}{1.068666in}}%
\pgfpathlineto{\pgfqpoint{2.125178in}{0.550000in}}%
\pgfpathclose%
\pgfusepath{fill}%
\end{pgfscope}%
\begin{pgfscope}%
\pgfpathrectangle{\pgfqpoint{1.250000in}{0.550000in}}{\pgfqpoint{7.750000in}{3.850000in}}%
\pgfusepath{clip}%
\pgfsetbuttcap%
\pgfsetmiterjoin%
\definecolor{currentfill}{rgb}{0.501961,0.501961,0.501961}%
\pgfsetfillcolor{currentfill}%
\pgfsetfillopacity{0.700000}%
\pgfsetlinewidth{0.000000pt}%
\definecolor{currentstroke}{rgb}{0.000000,0.000000,0.000000}%
\pgfsetstrokecolor{currentstroke}%
\pgfsetstrokeopacity{0.700000}%
\pgfsetdash{}{0pt}%
\pgfpathmoveto{\pgfqpoint{2.152699in}{0.550000in}}%
\pgfpathlineto{\pgfqpoint{2.180220in}{0.550000in}}%
\pgfpathlineto{\pgfqpoint{2.180220in}{1.102690in}}%
\pgfpathlineto{\pgfqpoint{2.152699in}{1.102690in}}%
\pgfpathlineto{\pgfqpoint{2.152699in}{0.550000in}}%
\pgfpathclose%
\pgfusepath{fill}%
\end{pgfscope}%
\begin{pgfscope}%
\pgfpathrectangle{\pgfqpoint{1.250000in}{0.550000in}}{\pgfqpoint{7.750000in}{3.850000in}}%
\pgfusepath{clip}%
\pgfsetbuttcap%
\pgfsetmiterjoin%
\definecolor{currentfill}{rgb}{0.501961,0.501961,0.501961}%
\pgfsetfillcolor{currentfill}%
\pgfsetfillopacity{0.700000}%
\pgfsetlinewidth{0.000000pt}%
\definecolor{currentstroke}{rgb}{0.000000,0.000000,0.000000}%
\pgfsetstrokecolor{currentstroke}%
\pgfsetstrokeopacity{0.700000}%
\pgfsetdash{}{0pt}%
\pgfpathmoveto{\pgfqpoint{2.180220in}{0.550000in}}%
\pgfpathlineto{\pgfqpoint{2.207741in}{0.550000in}}%
\pgfpathlineto{\pgfqpoint{2.207741in}{1.743210in}}%
\pgfpathlineto{\pgfqpoint{2.180220in}{1.743210in}}%
\pgfpathlineto{\pgfqpoint{2.180220in}{0.550000in}}%
\pgfpathclose%
\pgfusepath{fill}%
\end{pgfscope}%
\begin{pgfscope}%
\pgfpathrectangle{\pgfqpoint{1.250000in}{0.550000in}}{\pgfqpoint{7.750000in}{3.850000in}}%
\pgfusepath{clip}%
\pgfsetbuttcap%
\pgfsetmiterjoin%
\definecolor{currentfill}{rgb}{0.501961,0.501961,0.501961}%
\pgfsetfillcolor{currentfill}%
\pgfsetfillopacity{0.700000}%
\pgfsetlinewidth{0.000000pt}%
\definecolor{currentstroke}{rgb}{0.000000,0.000000,0.000000}%
\pgfsetstrokecolor{currentstroke}%
\pgfsetstrokeopacity{0.700000}%
\pgfsetdash{}{0pt}%
\pgfpathmoveto{\pgfqpoint{2.207741in}{0.550000in}}%
\pgfpathlineto{\pgfqpoint{2.235263in}{0.550000in}}%
\pgfpathlineto{\pgfqpoint{2.235263in}{1.135132in}}%
\pgfpathlineto{\pgfqpoint{2.207741in}{1.135132in}}%
\pgfpathlineto{\pgfqpoint{2.207741in}{0.550000in}}%
\pgfpathclose%
\pgfusepath{fill}%
\end{pgfscope}%
\begin{pgfscope}%
\pgfpathrectangle{\pgfqpoint{1.250000in}{0.550000in}}{\pgfqpoint{7.750000in}{3.850000in}}%
\pgfusepath{clip}%
\pgfsetbuttcap%
\pgfsetmiterjoin%
\definecolor{currentfill}{rgb}{0.501961,0.501961,0.501961}%
\pgfsetfillcolor{currentfill}%
\pgfsetfillopacity{0.700000}%
\pgfsetlinewidth{0.000000pt}%
\definecolor{currentstroke}{rgb}{0.000000,0.000000,0.000000}%
\pgfsetstrokecolor{currentstroke}%
\pgfsetstrokeopacity{0.700000}%
\pgfsetdash{}{0pt}%
\pgfpathmoveto{\pgfqpoint{2.235263in}{0.550000in}}%
\pgfpathlineto{\pgfqpoint{2.262784in}{0.550000in}}%
\pgfpathlineto{\pgfqpoint{2.262784in}{1.118120in}}%
\pgfpathlineto{\pgfqpoint{2.235263in}{1.118120in}}%
\pgfpathlineto{\pgfqpoint{2.235263in}{0.550000in}}%
\pgfpathclose%
\pgfusepath{fill}%
\end{pgfscope}%
\begin{pgfscope}%
\pgfpathrectangle{\pgfqpoint{1.250000in}{0.550000in}}{\pgfqpoint{7.750000in}{3.850000in}}%
\pgfusepath{clip}%
\pgfsetbuttcap%
\pgfsetmiterjoin%
\definecolor{currentfill}{rgb}{0.501961,0.501961,0.501961}%
\pgfsetfillcolor{currentfill}%
\pgfsetfillopacity{0.700000}%
\pgfsetlinewidth{0.000000pt}%
\definecolor{currentstroke}{rgb}{0.000000,0.000000,0.000000}%
\pgfsetstrokecolor{currentstroke}%
\pgfsetstrokeopacity{0.700000}%
\pgfsetdash{}{0pt}%
\pgfpathmoveto{\pgfqpoint{2.262784in}{0.550000in}}%
\pgfpathlineto{\pgfqpoint{2.290305in}{0.550000in}}%
\pgfpathlineto{\pgfqpoint{2.290305in}{1.145418in}}%
\pgfpathlineto{\pgfqpoint{2.262784in}{1.145418in}}%
\pgfpathlineto{\pgfqpoint{2.262784in}{0.550000in}}%
\pgfpathclose%
\pgfusepath{fill}%
\end{pgfscope}%
\begin{pgfscope}%
\pgfpathrectangle{\pgfqpoint{1.250000in}{0.550000in}}{\pgfqpoint{7.750000in}{3.850000in}}%
\pgfusepath{clip}%
\pgfsetbuttcap%
\pgfsetmiterjoin%
\definecolor{currentfill}{rgb}{0.501961,0.501961,0.501961}%
\pgfsetfillcolor{currentfill}%
\pgfsetfillopacity{0.700000}%
\pgfsetlinewidth{0.000000pt}%
\definecolor{currentstroke}{rgb}{0.000000,0.000000,0.000000}%
\pgfsetstrokecolor{currentstroke}%
\pgfsetstrokeopacity{0.700000}%
\pgfsetdash{}{0pt}%
\pgfpathmoveto{\pgfqpoint{2.290305in}{0.550000in}}%
\pgfpathlineto{\pgfqpoint{2.317827in}{0.550000in}}%
\pgfpathlineto{\pgfqpoint{2.317827in}{1.164012in}}%
\pgfpathlineto{\pgfqpoint{2.290305in}{1.164012in}}%
\pgfpathlineto{\pgfqpoint{2.290305in}{0.550000in}}%
\pgfpathclose%
\pgfusepath{fill}%
\end{pgfscope}%
\begin{pgfscope}%
\pgfpathrectangle{\pgfqpoint{1.250000in}{0.550000in}}{\pgfqpoint{7.750000in}{3.850000in}}%
\pgfusepath{clip}%
\pgfsetbuttcap%
\pgfsetmiterjoin%
\definecolor{currentfill}{rgb}{0.501961,0.501961,0.501961}%
\pgfsetfillcolor{currentfill}%
\pgfsetfillopacity{0.700000}%
\pgfsetlinewidth{0.000000pt}%
\definecolor{currentstroke}{rgb}{0.000000,0.000000,0.000000}%
\pgfsetstrokecolor{currentstroke}%
\pgfsetstrokeopacity{0.700000}%
\pgfsetdash{}{0pt}%
\pgfpathmoveto{\pgfqpoint{2.317827in}{0.550000in}}%
\pgfpathlineto{\pgfqpoint{2.345348in}{0.550000in}}%
\pgfpathlineto{\pgfqpoint{2.345348in}{1.191311in}}%
\pgfpathlineto{\pgfqpoint{2.317827in}{1.191311in}}%
\pgfpathlineto{\pgfqpoint{2.317827in}{0.550000in}}%
\pgfpathclose%
\pgfusepath{fill}%
\end{pgfscope}%
\begin{pgfscope}%
\pgfpathrectangle{\pgfqpoint{1.250000in}{0.550000in}}{\pgfqpoint{7.750000in}{3.850000in}}%
\pgfusepath{clip}%
\pgfsetbuttcap%
\pgfsetmiterjoin%
\definecolor{currentfill}{rgb}{0.501961,0.501961,0.501961}%
\pgfsetfillcolor{currentfill}%
\pgfsetfillopacity{0.700000}%
\pgfsetlinewidth{0.000000pt}%
\definecolor{currentstroke}{rgb}{0.000000,0.000000,0.000000}%
\pgfsetstrokecolor{currentstroke}%
\pgfsetstrokeopacity{0.700000}%
\pgfsetdash{}{0pt}%
\pgfpathmoveto{\pgfqpoint{2.345348in}{0.550000in}}%
\pgfpathlineto{\pgfqpoint{2.372869in}{0.550000in}}%
\pgfpathlineto{\pgfqpoint{2.372869in}{1.294569in}}%
\pgfpathlineto{\pgfqpoint{2.345348in}{1.294569in}}%
\pgfpathlineto{\pgfqpoint{2.345348in}{0.550000in}}%
\pgfpathclose%
\pgfusepath{fill}%
\end{pgfscope}%
\begin{pgfscope}%
\pgfpathrectangle{\pgfqpoint{1.250000in}{0.550000in}}{\pgfqpoint{7.750000in}{3.850000in}}%
\pgfusepath{clip}%
\pgfsetbuttcap%
\pgfsetmiterjoin%
\definecolor{currentfill}{rgb}{0.501961,0.501961,0.501961}%
\pgfsetfillcolor{currentfill}%
\pgfsetfillopacity{0.700000}%
\pgfsetlinewidth{0.000000pt}%
\definecolor{currentstroke}{rgb}{0.000000,0.000000,0.000000}%
\pgfsetstrokecolor{currentstroke}%
\pgfsetstrokeopacity{0.700000}%
\pgfsetdash{}{0pt}%
\pgfpathmoveto{\pgfqpoint{2.372869in}{0.550000in}}%
\pgfpathlineto{\pgfqpoint{2.400391in}{0.550000in}}%
\pgfpathlineto{\pgfqpoint{2.400391in}{1.368156in}}%
\pgfpathlineto{\pgfqpoint{2.372869in}{1.368156in}}%
\pgfpathlineto{\pgfqpoint{2.372869in}{0.550000in}}%
\pgfpathclose%
\pgfusepath{fill}%
\end{pgfscope}%
\begin{pgfscope}%
\pgfpathrectangle{\pgfqpoint{1.250000in}{0.550000in}}{\pgfqpoint{7.750000in}{3.850000in}}%
\pgfusepath{clip}%
\pgfsetbuttcap%
\pgfsetmiterjoin%
\definecolor{currentfill}{rgb}{0.501961,0.501961,0.501961}%
\pgfsetfillcolor{currentfill}%
\pgfsetfillopacity{0.700000}%
\pgfsetlinewidth{0.000000pt}%
\definecolor{currentstroke}{rgb}{0.000000,0.000000,0.000000}%
\pgfsetstrokecolor{currentstroke}%
\pgfsetstrokeopacity{0.700000}%
\pgfsetdash{}{0pt}%
\pgfpathmoveto{\pgfqpoint{2.400391in}{0.550000in}}%
\pgfpathlineto{\pgfqpoint{2.427912in}{0.550000in}}%
\pgfpathlineto{\pgfqpoint{2.427912in}{1.476557in}}%
\pgfpathlineto{\pgfqpoint{2.400391in}{1.476557in}}%
\pgfpathlineto{\pgfqpoint{2.400391in}{0.550000in}}%
\pgfpathclose%
\pgfusepath{fill}%
\end{pgfscope}%
\begin{pgfscope}%
\pgfpathrectangle{\pgfqpoint{1.250000in}{0.550000in}}{\pgfqpoint{7.750000in}{3.850000in}}%
\pgfusepath{clip}%
\pgfsetbuttcap%
\pgfsetmiterjoin%
\definecolor{currentfill}{rgb}{0.501961,0.501961,0.501961}%
\pgfsetfillcolor{currentfill}%
\pgfsetfillopacity{0.700000}%
\pgfsetlinewidth{0.000000pt}%
\definecolor{currentstroke}{rgb}{0.000000,0.000000,0.000000}%
\pgfsetstrokecolor{currentstroke}%
\pgfsetstrokeopacity{0.700000}%
\pgfsetdash{}{0pt}%
\pgfpathmoveto{\pgfqpoint{2.427912in}{0.550000in}}%
\pgfpathlineto{\pgfqpoint{2.455433in}{0.550000in}}%
\pgfpathlineto{\pgfqpoint{2.455433in}{1.641929in}}%
\pgfpathlineto{\pgfqpoint{2.427912in}{1.641929in}}%
\pgfpathlineto{\pgfqpoint{2.427912in}{0.550000in}}%
\pgfpathclose%
\pgfusepath{fill}%
\end{pgfscope}%
\begin{pgfscope}%
\pgfpathrectangle{\pgfqpoint{1.250000in}{0.550000in}}{\pgfqpoint{7.750000in}{3.850000in}}%
\pgfusepath{clip}%
\pgfsetbuttcap%
\pgfsetmiterjoin%
\definecolor{currentfill}{rgb}{0.501961,0.501961,0.501961}%
\pgfsetfillcolor{currentfill}%
\pgfsetfillopacity{0.700000}%
\pgfsetlinewidth{0.000000pt}%
\definecolor{currentstroke}{rgb}{0.000000,0.000000,0.000000}%
\pgfsetstrokecolor{currentstroke}%
\pgfsetstrokeopacity{0.700000}%
\pgfsetdash{}{0pt}%
\pgfpathmoveto{\pgfqpoint{2.455433in}{0.550000in}}%
\pgfpathlineto{\pgfqpoint{2.482955in}{0.550000in}}%
\pgfpathlineto{\pgfqpoint{2.482955in}{1.565573in}}%
\pgfpathlineto{\pgfqpoint{2.455433in}{1.565573in}}%
\pgfpathlineto{\pgfqpoint{2.455433in}{0.550000in}}%
\pgfpathclose%
\pgfusepath{fill}%
\end{pgfscope}%
\begin{pgfscope}%
\pgfpathrectangle{\pgfqpoint{1.250000in}{0.550000in}}{\pgfqpoint{7.750000in}{3.850000in}}%
\pgfusepath{clip}%
\pgfsetbuttcap%
\pgfsetmiterjoin%
\definecolor{currentfill}{rgb}{0.501961,0.501961,0.501961}%
\pgfsetfillcolor{currentfill}%
\pgfsetfillopacity{0.700000}%
\pgfsetlinewidth{0.000000pt}%
\definecolor{currentstroke}{rgb}{0.000000,0.000000,0.000000}%
\pgfsetstrokecolor{currentstroke}%
\pgfsetstrokeopacity{0.700000}%
\pgfsetdash{}{0pt}%
\pgfpathmoveto{\pgfqpoint{2.482955in}{0.550000in}}%
\pgfpathlineto{\pgfqpoint{2.510476in}{0.550000in}}%
\pgfpathlineto{\pgfqpoint{2.510476in}{1.613048in}}%
\pgfpathlineto{\pgfqpoint{2.482955in}{1.613048in}}%
\pgfpathlineto{\pgfqpoint{2.482955in}{0.550000in}}%
\pgfpathclose%
\pgfusepath{fill}%
\end{pgfscope}%
\begin{pgfscope}%
\pgfpathrectangle{\pgfqpoint{1.250000in}{0.550000in}}{\pgfqpoint{7.750000in}{3.850000in}}%
\pgfusepath{clip}%
\pgfsetbuttcap%
\pgfsetmiterjoin%
\definecolor{currentfill}{rgb}{0.501961,0.501961,0.501961}%
\pgfsetfillcolor{currentfill}%
\pgfsetfillopacity{0.700000}%
\pgfsetlinewidth{0.000000pt}%
\definecolor{currentstroke}{rgb}{0.000000,0.000000,0.000000}%
\pgfsetstrokecolor{currentstroke}%
\pgfsetstrokeopacity{0.700000}%
\pgfsetdash{}{0pt}%
\pgfpathmoveto{\pgfqpoint{2.510476in}{0.550000in}}%
\pgfpathlineto{\pgfqpoint{2.537997in}{0.550000in}}%
\pgfpathlineto{\pgfqpoint{2.537997in}{1.790289in}}%
\pgfpathlineto{\pgfqpoint{2.510476in}{1.790289in}}%
\pgfpathlineto{\pgfqpoint{2.510476in}{0.550000in}}%
\pgfpathclose%
\pgfusepath{fill}%
\end{pgfscope}%
\begin{pgfscope}%
\pgfpathrectangle{\pgfqpoint{1.250000in}{0.550000in}}{\pgfqpoint{7.750000in}{3.850000in}}%
\pgfusepath{clip}%
\pgfsetbuttcap%
\pgfsetmiterjoin%
\definecolor{currentfill}{rgb}{0.501961,0.501961,0.501961}%
\pgfsetfillcolor{currentfill}%
\pgfsetfillopacity{0.700000}%
\pgfsetlinewidth{0.000000pt}%
\definecolor{currentstroke}{rgb}{0.000000,0.000000,0.000000}%
\pgfsetstrokecolor{currentstroke}%
\pgfsetstrokeopacity{0.700000}%
\pgfsetdash{}{0pt}%
\pgfpathmoveto{\pgfqpoint{2.537997in}{0.550000in}}%
\pgfpathlineto{\pgfqpoint{2.565518in}{0.550000in}}%
\pgfpathlineto{\pgfqpoint{2.565518in}{1.673975in}}%
\pgfpathlineto{\pgfqpoint{2.537997in}{1.673975in}}%
\pgfpathlineto{\pgfqpoint{2.537997in}{0.550000in}}%
\pgfpathclose%
\pgfusepath{fill}%
\end{pgfscope}%
\begin{pgfscope}%
\pgfpathrectangle{\pgfqpoint{1.250000in}{0.550000in}}{\pgfqpoint{7.750000in}{3.850000in}}%
\pgfusepath{clip}%
\pgfsetbuttcap%
\pgfsetmiterjoin%
\definecolor{currentfill}{rgb}{0.501961,0.501961,0.501961}%
\pgfsetfillcolor{currentfill}%
\pgfsetfillopacity{0.700000}%
\pgfsetlinewidth{0.000000pt}%
\definecolor{currentstroke}{rgb}{0.000000,0.000000,0.000000}%
\pgfsetstrokecolor{currentstroke}%
\pgfsetstrokeopacity{0.700000}%
\pgfsetdash{}{0pt}%
\pgfpathmoveto{\pgfqpoint{2.565518in}{0.550000in}}%
\pgfpathlineto{\pgfqpoint{2.593040in}{0.550000in}}%
\pgfpathlineto{\pgfqpoint{2.593040in}{1.753892in}}%
\pgfpathlineto{\pgfqpoint{2.565518in}{1.753892in}}%
\pgfpathlineto{\pgfqpoint{2.565518in}{0.550000in}}%
\pgfpathclose%
\pgfusepath{fill}%
\end{pgfscope}%
\begin{pgfscope}%
\pgfpathrectangle{\pgfqpoint{1.250000in}{0.550000in}}{\pgfqpoint{7.750000in}{3.850000in}}%
\pgfusepath{clip}%
\pgfsetbuttcap%
\pgfsetmiterjoin%
\definecolor{currentfill}{rgb}{0.501961,0.501961,0.501961}%
\pgfsetfillcolor{currentfill}%
\pgfsetfillopacity{0.700000}%
\pgfsetlinewidth{0.000000pt}%
\definecolor{currentstroke}{rgb}{0.000000,0.000000,0.000000}%
\pgfsetstrokecolor{currentstroke}%
\pgfsetstrokeopacity{0.700000}%
\pgfsetdash{}{0pt}%
\pgfpathmoveto{\pgfqpoint{2.593040in}{0.550000in}}%
\pgfpathlineto{\pgfqpoint{2.620561in}{0.550000in}}%
\pgfpathlineto{\pgfqpoint{2.620561in}{1.768134in}}%
\pgfpathlineto{\pgfqpoint{2.593040in}{1.768134in}}%
\pgfpathlineto{\pgfqpoint{2.593040in}{0.550000in}}%
\pgfpathclose%
\pgfusepath{fill}%
\end{pgfscope}%
\begin{pgfscope}%
\pgfpathrectangle{\pgfqpoint{1.250000in}{0.550000in}}{\pgfqpoint{7.750000in}{3.850000in}}%
\pgfusepath{clip}%
\pgfsetbuttcap%
\pgfsetmiterjoin%
\definecolor{currentfill}{rgb}{0.501961,0.501961,0.501961}%
\pgfsetfillcolor{currentfill}%
\pgfsetfillopacity{0.700000}%
\pgfsetlinewidth{0.000000pt}%
\definecolor{currentstroke}{rgb}{0.000000,0.000000,0.000000}%
\pgfsetstrokecolor{currentstroke}%
\pgfsetstrokeopacity{0.700000}%
\pgfsetdash{}{0pt}%
\pgfpathmoveto{\pgfqpoint{2.620561in}{0.550000in}}%
\pgfpathlineto{\pgfqpoint{2.648082in}{0.550000in}}%
\pgfpathlineto{\pgfqpoint{2.648082in}{0.550000in}}%
\pgfpathlineto{\pgfqpoint{2.620561in}{0.550000in}}%
\pgfpathlineto{\pgfqpoint{2.620561in}{0.550000in}}%
\pgfpathclose%
\pgfusepath{fill}%
\end{pgfscope}%
\begin{pgfscope}%
\pgfpathrectangle{\pgfqpoint{1.250000in}{0.550000in}}{\pgfqpoint{7.750000in}{3.850000in}}%
\pgfusepath{clip}%
\pgfsetbuttcap%
\pgfsetmiterjoin%
\definecolor{currentfill}{rgb}{0.501961,0.501961,0.501961}%
\pgfsetfillcolor{currentfill}%
\pgfsetfillopacity{0.700000}%
\pgfsetlinewidth{0.000000pt}%
\definecolor{currentstroke}{rgb}{0.000000,0.000000,0.000000}%
\pgfsetstrokecolor{currentstroke}%
\pgfsetstrokeopacity{0.700000}%
\pgfsetdash{}{0pt}%
\pgfpathmoveto{\pgfqpoint{2.648082in}{0.550000in}}%
\pgfpathlineto{\pgfqpoint{2.675604in}{0.550000in}}%
\pgfpathlineto{\pgfqpoint{2.675604in}{1.801367in}}%
\pgfpathlineto{\pgfqpoint{2.648082in}{1.801367in}}%
\pgfpathlineto{\pgfqpoint{2.648082in}{0.550000in}}%
\pgfpathclose%
\pgfusepath{fill}%
\end{pgfscope}%
\begin{pgfscope}%
\pgfpathrectangle{\pgfqpoint{1.250000in}{0.550000in}}{\pgfqpoint{7.750000in}{3.850000in}}%
\pgfusepath{clip}%
\pgfsetbuttcap%
\pgfsetmiterjoin%
\definecolor{currentfill}{rgb}{0.501961,0.501961,0.501961}%
\pgfsetfillcolor{currentfill}%
\pgfsetfillopacity{0.700000}%
\pgfsetlinewidth{0.000000pt}%
\definecolor{currentstroke}{rgb}{0.000000,0.000000,0.000000}%
\pgfsetstrokecolor{currentstroke}%
\pgfsetstrokeopacity{0.700000}%
\pgfsetdash{}{0pt}%
\pgfpathmoveto{\pgfqpoint{2.675604in}{0.550000in}}%
\pgfpathlineto{\pgfqpoint{2.703125in}{0.550000in}}%
\pgfpathlineto{\pgfqpoint{2.703125in}{1.788707in}}%
\pgfpathlineto{\pgfqpoint{2.675604in}{1.788707in}}%
\pgfpathlineto{\pgfqpoint{2.675604in}{0.550000in}}%
\pgfpathclose%
\pgfusepath{fill}%
\end{pgfscope}%
\begin{pgfscope}%
\pgfpathrectangle{\pgfqpoint{1.250000in}{0.550000in}}{\pgfqpoint{7.750000in}{3.850000in}}%
\pgfusepath{clip}%
\pgfsetbuttcap%
\pgfsetmiterjoin%
\definecolor{currentfill}{rgb}{0.501961,0.501961,0.501961}%
\pgfsetfillcolor{currentfill}%
\pgfsetfillopacity{0.700000}%
\pgfsetlinewidth{0.000000pt}%
\definecolor{currentstroke}{rgb}{0.000000,0.000000,0.000000}%
\pgfsetstrokecolor{currentstroke}%
\pgfsetstrokeopacity{0.700000}%
\pgfsetdash{}{0pt}%
\pgfpathmoveto{\pgfqpoint{2.703125in}{0.550000in}}%
\pgfpathlineto{\pgfqpoint{2.730646in}{0.550000in}}%
\pgfpathlineto{\pgfqpoint{2.730646in}{1.716703in}}%
\pgfpathlineto{\pgfqpoint{2.703125in}{1.716703in}}%
\pgfpathlineto{\pgfqpoint{2.703125in}{0.550000in}}%
\pgfpathclose%
\pgfusepath{fill}%
\end{pgfscope}%
\begin{pgfscope}%
\pgfpathrectangle{\pgfqpoint{1.250000in}{0.550000in}}{\pgfqpoint{7.750000in}{3.850000in}}%
\pgfusepath{clip}%
\pgfsetbuttcap%
\pgfsetmiterjoin%
\definecolor{currentfill}{rgb}{0.501961,0.501961,0.501961}%
\pgfsetfillcolor{currentfill}%
\pgfsetfillopacity{0.700000}%
\pgfsetlinewidth{0.000000pt}%
\definecolor{currentstroke}{rgb}{0.000000,0.000000,0.000000}%
\pgfsetstrokecolor{currentstroke}%
\pgfsetstrokeopacity{0.700000}%
\pgfsetdash{}{0pt}%
\pgfpathmoveto{\pgfqpoint{2.730646in}{0.550000in}}%
\pgfpathlineto{\pgfqpoint{2.758168in}{0.550000in}}%
\pgfpathlineto{\pgfqpoint{2.758168in}{1.641534in}}%
\pgfpathlineto{\pgfqpoint{2.730646in}{1.641534in}}%
\pgfpathlineto{\pgfqpoint{2.730646in}{0.550000in}}%
\pgfpathclose%
\pgfusepath{fill}%
\end{pgfscope}%
\begin{pgfscope}%
\pgfpathrectangle{\pgfqpoint{1.250000in}{0.550000in}}{\pgfqpoint{7.750000in}{3.850000in}}%
\pgfusepath{clip}%
\pgfsetbuttcap%
\pgfsetmiterjoin%
\definecolor{currentfill}{rgb}{0.501961,0.501961,0.501961}%
\pgfsetfillcolor{currentfill}%
\pgfsetfillopacity{0.700000}%
\pgfsetlinewidth{0.000000pt}%
\definecolor{currentstroke}{rgb}{0.000000,0.000000,0.000000}%
\pgfsetstrokecolor{currentstroke}%
\pgfsetstrokeopacity{0.700000}%
\pgfsetdash{}{0pt}%
\pgfpathmoveto{\pgfqpoint{2.758168in}{0.550000in}}%
\pgfpathlineto{\pgfqpoint{2.785689in}{0.550000in}}%
\pgfpathlineto{\pgfqpoint{2.785689in}{1.477349in}}%
\pgfpathlineto{\pgfqpoint{2.758168in}{1.477349in}}%
\pgfpathlineto{\pgfqpoint{2.758168in}{0.550000in}}%
\pgfpathclose%
\pgfusepath{fill}%
\end{pgfscope}%
\begin{pgfscope}%
\pgfpathrectangle{\pgfqpoint{1.250000in}{0.550000in}}{\pgfqpoint{7.750000in}{3.850000in}}%
\pgfusepath{clip}%
\pgfsetbuttcap%
\pgfsetmiterjoin%
\definecolor{currentfill}{rgb}{0.501961,0.501961,0.501961}%
\pgfsetfillcolor{currentfill}%
\pgfsetfillopacity{0.700000}%
\pgfsetlinewidth{0.000000pt}%
\definecolor{currentstroke}{rgb}{0.000000,0.000000,0.000000}%
\pgfsetstrokecolor{currentstroke}%
\pgfsetstrokeopacity{0.700000}%
\pgfsetdash{}{0pt}%
\pgfpathmoveto{\pgfqpoint{2.785689in}{0.550000in}}%
\pgfpathlineto{\pgfqpoint{2.813210in}{0.550000in}}%
\pgfpathlineto{\pgfqpoint{2.813210in}{1.431060in}}%
\pgfpathlineto{\pgfqpoint{2.785689in}{1.431060in}}%
\pgfpathlineto{\pgfqpoint{2.785689in}{0.550000in}}%
\pgfpathclose%
\pgfusepath{fill}%
\end{pgfscope}%
\begin{pgfscope}%
\pgfpathrectangle{\pgfqpoint{1.250000in}{0.550000in}}{\pgfqpoint{7.750000in}{3.850000in}}%
\pgfusepath{clip}%
\pgfsetbuttcap%
\pgfsetmiterjoin%
\definecolor{currentfill}{rgb}{0.501961,0.501961,0.501961}%
\pgfsetfillcolor{currentfill}%
\pgfsetfillopacity{0.700000}%
\pgfsetlinewidth{0.000000pt}%
\definecolor{currentstroke}{rgb}{0.000000,0.000000,0.000000}%
\pgfsetstrokecolor{currentstroke}%
\pgfsetstrokeopacity{0.700000}%
\pgfsetdash{}{0pt}%
\pgfpathmoveto{\pgfqpoint{2.813210in}{0.550000in}}%
\pgfpathlineto{\pgfqpoint{2.840732in}{0.550000in}}%
\pgfpathlineto{\pgfqpoint{2.840732in}{1.289426in}}%
\pgfpathlineto{\pgfqpoint{2.813210in}{1.289426in}}%
\pgfpathlineto{\pgfqpoint{2.813210in}{0.550000in}}%
\pgfpathclose%
\pgfusepath{fill}%
\end{pgfscope}%
\begin{pgfscope}%
\pgfpathrectangle{\pgfqpoint{1.250000in}{0.550000in}}{\pgfqpoint{7.750000in}{3.850000in}}%
\pgfusepath{clip}%
\pgfsetbuttcap%
\pgfsetmiterjoin%
\definecolor{currentfill}{rgb}{0.501961,0.501961,0.501961}%
\pgfsetfillcolor{currentfill}%
\pgfsetfillopacity{0.700000}%
\pgfsetlinewidth{0.000000pt}%
\definecolor{currentstroke}{rgb}{0.000000,0.000000,0.000000}%
\pgfsetstrokecolor{currentstroke}%
\pgfsetstrokeopacity{0.700000}%
\pgfsetdash{}{0pt}%
\pgfpathmoveto{\pgfqpoint{2.840732in}{0.550000in}}%
\pgfpathlineto{\pgfqpoint{2.868253in}{0.550000in}}%
\pgfpathlineto{\pgfqpoint{2.868253in}{1.277162in}}%
\pgfpathlineto{\pgfqpoint{2.840732in}{1.277162in}}%
\pgfpathlineto{\pgfqpoint{2.840732in}{0.550000in}}%
\pgfpathclose%
\pgfusepath{fill}%
\end{pgfscope}%
\begin{pgfscope}%
\pgfpathrectangle{\pgfqpoint{1.250000in}{0.550000in}}{\pgfqpoint{7.750000in}{3.850000in}}%
\pgfusepath{clip}%
\pgfsetbuttcap%
\pgfsetmiterjoin%
\definecolor{currentfill}{rgb}{0.501961,0.501961,0.501961}%
\pgfsetfillcolor{currentfill}%
\pgfsetfillopacity{0.700000}%
\pgfsetlinewidth{0.000000pt}%
\definecolor{currentstroke}{rgb}{0.000000,0.000000,0.000000}%
\pgfsetstrokecolor{currentstroke}%
\pgfsetstrokeopacity{0.700000}%
\pgfsetdash{}{0pt}%
\pgfpathmoveto{\pgfqpoint{2.868253in}{0.550000in}}%
\pgfpathlineto{\pgfqpoint{2.895774in}{0.550000in}}%
\pgfpathlineto{\pgfqpoint{2.895774in}{1.270040in}}%
\pgfpathlineto{\pgfqpoint{2.868253in}{1.270040in}}%
\pgfpathlineto{\pgfqpoint{2.868253in}{0.550000in}}%
\pgfpathclose%
\pgfusepath{fill}%
\end{pgfscope}%
\begin{pgfscope}%
\pgfpathrectangle{\pgfqpoint{1.250000in}{0.550000in}}{\pgfqpoint{7.750000in}{3.850000in}}%
\pgfusepath{clip}%
\pgfsetbuttcap%
\pgfsetmiterjoin%
\definecolor{currentfill}{rgb}{0.501961,0.501961,0.501961}%
\pgfsetfillcolor{currentfill}%
\pgfsetfillopacity{0.700000}%
\pgfsetlinewidth{0.000000pt}%
\definecolor{currentstroke}{rgb}{0.000000,0.000000,0.000000}%
\pgfsetstrokecolor{currentstroke}%
\pgfsetstrokeopacity{0.700000}%
\pgfsetdash{}{0pt}%
\pgfpathmoveto{\pgfqpoint{2.895774in}{0.550000in}}%
\pgfpathlineto{\pgfqpoint{2.923295in}{0.550000in}}%
\pgfpathlineto{\pgfqpoint{2.923295in}{1.246303in}}%
\pgfpathlineto{\pgfqpoint{2.895774in}{1.246303in}}%
\pgfpathlineto{\pgfqpoint{2.895774in}{0.550000in}}%
\pgfpathclose%
\pgfusepath{fill}%
\end{pgfscope}%
\begin{pgfscope}%
\pgfpathrectangle{\pgfqpoint{1.250000in}{0.550000in}}{\pgfqpoint{7.750000in}{3.850000in}}%
\pgfusepath{clip}%
\pgfsetbuttcap%
\pgfsetmiterjoin%
\definecolor{currentfill}{rgb}{0.501961,0.501961,0.501961}%
\pgfsetfillcolor{currentfill}%
\pgfsetfillopacity{0.700000}%
\pgfsetlinewidth{0.000000pt}%
\definecolor{currentstroke}{rgb}{0.000000,0.000000,0.000000}%
\pgfsetstrokecolor{currentstroke}%
\pgfsetstrokeopacity{0.700000}%
\pgfsetdash{}{0pt}%
\pgfpathmoveto{\pgfqpoint{2.923295in}{0.550000in}}%
\pgfpathlineto{\pgfqpoint{2.950817in}{0.550000in}}%
\pgfpathlineto{\pgfqpoint{2.950817in}{1.208718in}}%
\pgfpathlineto{\pgfqpoint{2.923295in}{1.208718in}}%
\pgfpathlineto{\pgfqpoint{2.923295in}{0.550000in}}%
\pgfpathclose%
\pgfusepath{fill}%
\end{pgfscope}%
\begin{pgfscope}%
\pgfpathrectangle{\pgfqpoint{1.250000in}{0.550000in}}{\pgfqpoint{7.750000in}{3.850000in}}%
\pgfusepath{clip}%
\pgfsetbuttcap%
\pgfsetmiterjoin%
\definecolor{currentfill}{rgb}{0.501961,0.501961,0.501961}%
\pgfsetfillcolor{currentfill}%
\pgfsetfillopacity{0.700000}%
\pgfsetlinewidth{0.000000pt}%
\definecolor{currentstroke}{rgb}{0.000000,0.000000,0.000000}%
\pgfsetstrokecolor{currentstroke}%
\pgfsetstrokeopacity{0.700000}%
\pgfsetdash{}{0pt}%
\pgfpathmoveto{\pgfqpoint{2.950817in}{0.550000in}}%
\pgfpathlineto{\pgfqpoint{2.978338in}{0.550000in}}%
\pgfpathlineto{\pgfqpoint{2.978338in}{0.550000in}}%
\pgfpathlineto{\pgfqpoint{2.950817in}{0.550000in}}%
\pgfpathlineto{\pgfqpoint{2.950817in}{0.550000in}}%
\pgfpathclose%
\pgfusepath{fill}%
\end{pgfscope}%
\begin{pgfscope}%
\pgfpathrectangle{\pgfqpoint{1.250000in}{0.550000in}}{\pgfqpoint{7.750000in}{3.850000in}}%
\pgfusepath{clip}%
\pgfsetbuttcap%
\pgfsetmiterjoin%
\definecolor{currentfill}{rgb}{0.501961,0.501961,0.501961}%
\pgfsetfillcolor{currentfill}%
\pgfsetfillopacity{0.700000}%
\pgfsetlinewidth{0.000000pt}%
\definecolor{currentstroke}{rgb}{0.000000,0.000000,0.000000}%
\pgfsetstrokecolor{currentstroke}%
\pgfsetstrokeopacity{0.700000}%
\pgfsetdash{}{0pt}%
\pgfpathmoveto{\pgfqpoint{2.978338in}{0.550000in}}%
\pgfpathlineto{\pgfqpoint{3.005859in}{0.550000in}}%
\pgfpathlineto{\pgfqpoint{3.005859in}{1.212674in}}%
\pgfpathlineto{\pgfqpoint{2.978338in}{1.212674in}}%
\pgfpathlineto{\pgfqpoint{2.978338in}{0.550000in}}%
\pgfpathclose%
\pgfusepath{fill}%
\end{pgfscope}%
\begin{pgfscope}%
\pgfpathrectangle{\pgfqpoint{1.250000in}{0.550000in}}{\pgfqpoint{7.750000in}{3.850000in}}%
\pgfusepath{clip}%
\pgfsetbuttcap%
\pgfsetmiterjoin%
\definecolor{currentfill}{rgb}{0.501961,0.501961,0.501961}%
\pgfsetfillcolor{currentfill}%
\pgfsetfillopacity{0.700000}%
\pgfsetlinewidth{0.000000pt}%
\definecolor{currentstroke}{rgb}{0.000000,0.000000,0.000000}%
\pgfsetstrokecolor{currentstroke}%
\pgfsetstrokeopacity{0.700000}%
\pgfsetdash{}{0pt}%
\pgfpathmoveto{\pgfqpoint{3.005859in}{0.550000in}}%
\pgfpathlineto{\pgfqpoint{3.033381in}{0.550000in}}%
\pgfpathlineto{\pgfqpoint{3.033381in}{0.550000in}}%
\pgfpathlineto{\pgfqpoint{3.005859in}{0.550000in}}%
\pgfpathlineto{\pgfqpoint{3.005859in}{0.550000in}}%
\pgfpathclose%
\pgfusepath{fill}%
\end{pgfscope}%
\begin{pgfscope}%
\pgfpathrectangle{\pgfqpoint{1.250000in}{0.550000in}}{\pgfqpoint{7.750000in}{3.850000in}}%
\pgfusepath{clip}%
\pgfsetbuttcap%
\pgfsetmiterjoin%
\definecolor{currentfill}{rgb}{0.501961,0.501961,0.501961}%
\pgfsetfillcolor{currentfill}%
\pgfsetfillopacity{0.700000}%
\pgfsetlinewidth{0.000000pt}%
\definecolor{currentstroke}{rgb}{0.000000,0.000000,0.000000}%
\pgfsetstrokecolor{currentstroke}%
\pgfsetstrokeopacity{0.700000}%
\pgfsetdash{}{0pt}%
\pgfpathmoveto{\pgfqpoint{3.033381in}{0.550000in}}%
\pgfpathlineto{\pgfqpoint{3.060902in}{0.550000in}}%
\pgfpathlineto{\pgfqpoint{3.060902in}{1.162034in}}%
\pgfpathlineto{\pgfqpoint{3.033381in}{1.162034in}}%
\pgfpathlineto{\pgfqpoint{3.033381in}{0.550000in}}%
\pgfpathclose%
\pgfusepath{fill}%
\end{pgfscope}%
\begin{pgfscope}%
\pgfpathrectangle{\pgfqpoint{1.250000in}{0.550000in}}{\pgfqpoint{7.750000in}{3.850000in}}%
\pgfusepath{clip}%
\pgfsetbuttcap%
\pgfsetmiterjoin%
\definecolor{currentfill}{rgb}{0.501961,0.501961,0.501961}%
\pgfsetfillcolor{currentfill}%
\pgfsetfillopacity{0.700000}%
\pgfsetlinewidth{0.000000pt}%
\definecolor{currentstroke}{rgb}{0.000000,0.000000,0.000000}%
\pgfsetstrokecolor{currentstroke}%
\pgfsetstrokeopacity{0.700000}%
\pgfsetdash{}{0pt}%
\pgfpathmoveto{\pgfqpoint{3.060902in}{0.550000in}}%
\pgfpathlineto{\pgfqpoint{3.088423in}{0.550000in}}%
\pgfpathlineto{\pgfqpoint{3.088423in}{1.161243in}}%
\pgfpathlineto{\pgfqpoint{3.060902in}{1.161243in}}%
\pgfpathlineto{\pgfqpoint{3.060902in}{0.550000in}}%
\pgfpathclose%
\pgfusepath{fill}%
\end{pgfscope}%
\begin{pgfscope}%
\pgfpathrectangle{\pgfqpoint{1.250000in}{0.550000in}}{\pgfqpoint{7.750000in}{3.850000in}}%
\pgfusepath{clip}%
\pgfsetbuttcap%
\pgfsetmiterjoin%
\definecolor{currentfill}{rgb}{0.501961,0.501961,0.501961}%
\pgfsetfillcolor{currentfill}%
\pgfsetfillopacity{0.700000}%
\pgfsetlinewidth{0.000000pt}%
\definecolor{currentstroke}{rgb}{0.000000,0.000000,0.000000}%
\pgfsetstrokecolor{currentstroke}%
\pgfsetstrokeopacity{0.700000}%
\pgfsetdash{}{0pt}%
\pgfpathmoveto{\pgfqpoint{3.088423in}{0.550000in}}%
\pgfpathlineto{\pgfqpoint{3.115945in}{0.550000in}}%
\pgfpathlineto{\pgfqpoint{3.115945in}{1.144231in}}%
\pgfpathlineto{\pgfqpoint{3.088423in}{1.144231in}}%
\pgfpathlineto{\pgfqpoint{3.088423in}{0.550000in}}%
\pgfpathclose%
\pgfusepath{fill}%
\end{pgfscope}%
\begin{pgfscope}%
\pgfpathrectangle{\pgfqpoint{1.250000in}{0.550000in}}{\pgfqpoint{7.750000in}{3.850000in}}%
\pgfusepath{clip}%
\pgfsetbuttcap%
\pgfsetmiterjoin%
\definecolor{currentfill}{rgb}{0.501961,0.501961,0.501961}%
\pgfsetfillcolor{currentfill}%
\pgfsetfillopacity{0.700000}%
\pgfsetlinewidth{0.000000pt}%
\definecolor{currentstroke}{rgb}{0.000000,0.000000,0.000000}%
\pgfsetstrokecolor{currentstroke}%
\pgfsetstrokeopacity{0.700000}%
\pgfsetdash{}{0pt}%
\pgfpathmoveto{\pgfqpoint{3.115945in}{0.550000in}}%
\pgfpathlineto{\pgfqpoint{3.143466in}{0.550000in}}%
\pgfpathlineto{\pgfqpoint{3.143466in}{1.059171in}}%
\pgfpathlineto{\pgfqpoint{3.115945in}{1.059171in}}%
\pgfpathlineto{\pgfqpoint{3.115945in}{0.550000in}}%
\pgfpathclose%
\pgfusepath{fill}%
\end{pgfscope}%
\begin{pgfscope}%
\pgfpathrectangle{\pgfqpoint{1.250000in}{0.550000in}}{\pgfqpoint{7.750000in}{3.850000in}}%
\pgfusepath{clip}%
\pgfsetbuttcap%
\pgfsetmiterjoin%
\definecolor{currentfill}{rgb}{0.501961,0.501961,0.501961}%
\pgfsetfillcolor{currentfill}%
\pgfsetfillopacity{0.700000}%
\pgfsetlinewidth{0.000000pt}%
\definecolor{currentstroke}{rgb}{0.000000,0.000000,0.000000}%
\pgfsetstrokecolor{currentstroke}%
\pgfsetstrokeopacity{0.700000}%
\pgfsetdash{}{0pt}%
\pgfpathmoveto{\pgfqpoint{3.143466in}{0.550000in}}%
\pgfpathlineto{\pgfqpoint{3.170987in}{0.550000in}}%
\pgfpathlineto{\pgfqpoint{3.170987in}{0.550000in}}%
\pgfpathlineto{\pgfqpoint{3.143466in}{0.550000in}}%
\pgfpathlineto{\pgfqpoint{3.143466in}{0.550000in}}%
\pgfpathclose%
\pgfusepath{fill}%
\end{pgfscope}%
\begin{pgfscope}%
\pgfpathrectangle{\pgfqpoint{1.250000in}{0.550000in}}{\pgfqpoint{7.750000in}{3.850000in}}%
\pgfusepath{clip}%
\pgfsetbuttcap%
\pgfsetmiterjoin%
\definecolor{currentfill}{rgb}{0.501961,0.501961,0.501961}%
\pgfsetfillcolor{currentfill}%
\pgfsetfillopacity{0.700000}%
\pgfsetlinewidth{0.000000pt}%
\definecolor{currentstroke}{rgb}{0.000000,0.000000,0.000000}%
\pgfsetstrokecolor{currentstroke}%
\pgfsetstrokeopacity{0.700000}%
\pgfsetdash{}{0pt}%
\pgfpathmoveto{\pgfqpoint{3.170987in}{0.550000in}}%
\pgfpathlineto{\pgfqpoint{3.198509in}{0.550000in}}%
\pgfpathlineto{\pgfqpoint{3.198509in}{1.025147in}}%
\pgfpathlineto{\pgfqpoint{3.170987in}{1.025147in}}%
\pgfpathlineto{\pgfqpoint{3.170987in}{0.550000in}}%
\pgfpathclose%
\pgfusepath{fill}%
\end{pgfscope}%
\begin{pgfscope}%
\pgfpathrectangle{\pgfqpoint{1.250000in}{0.550000in}}{\pgfqpoint{7.750000in}{3.850000in}}%
\pgfusepath{clip}%
\pgfsetbuttcap%
\pgfsetmiterjoin%
\definecolor{currentfill}{rgb}{0.501961,0.501961,0.501961}%
\pgfsetfillcolor{currentfill}%
\pgfsetfillopacity{0.700000}%
\pgfsetlinewidth{0.000000pt}%
\definecolor{currentstroke}{rgb}{0.000000,0.000000,0.000000}%
\pgfsetstrokecolor{currentstroke}%
\pgfsetstrokeopacity{0.700000}%
\pgfsetdash{}{0pt}%
\pgfpathmoveto{\pgfqpoint{3.198509in}{0.550000in}}%
\pgfpathlineto{\pgfqpoint{3.226030in}{0.550000in}}%
\pgfpathlineto{\pgfqpoint{3.226030in}{1.054819in}}%
\pgfpathlineto{\pgfqpoint{3.198509in}{1.054819in}}%
\pgfpathlineto{\pgfqpoint{3.198509in}{0.550000in}}%
\pgfpathclose%
\pgfusepath{fill}%
\end{pgfscope}%
\begin{pgfscope}%
\pgfpathrectangle{\pgfqpoint{1.250000in}{0.550000in}}{\pgfqpoint{7.750000in}{3.850000in}}%
\pgfusepath{clip}%
\pgfsetbuttcap%
\pgfsetmiterjoin%
\definecolor{currentfill}{rgb}{0.501961,0.501961,0.501961}%
\pgfsetfillcolor{currentfill}%
\pgfsetfillopacity{0.700000}%
\pgfsetlinewidth{0.000000pt}%
\definecolor{currentstroke}{rgb}{0.000000,0.000000,0.000000}%
\pgfsetstrokecolor{currentstroke}%
\pgfsetstrokeopacity{0.700000}%
\pgfsetdash{}{0pt}%
\pgfpathmoveto{\pgfqpoint{3.226030in}{0.550000in}}%
\pgfpathlineto{\pgfqpoint{3.253551in}{0.550000in}}%
\pgfpathlineto{\pgfqpoint{3.253551in}{1.013279in}}%
\pgfpathlineto{\pgfqpoint{3.226030in}{1.013279in}}%
\pgfpathlineto{\pgfqpoint{3.226030in}{0.550000in}}%
\pgfpathclose%
\pgfusepath{fill}%
\end{pgfscope}%
\begin{pgfscope}%
\pgfpathrectangle{\pgfqpoint{1.250000in}{0.550000in}}{\pgfqpoint{7.750000in}{3.850000in}}%
\pgfusepath{clip}%
\pgfsetbuttcap%
\pgfsetmiterjoin%
\definecolor{currentfill}{rgb}{0.501961,0.501961,0.501961}%
\pgfsetfillcolor{currentfill}%
\pgfsetfillopacity{0.700000}%
\pgfsetlinewidth{0.000000pt}%
\definecolor{currentstroke}{rgb}{0.000000,0.000000,0.000000}%
\pgfsetstrokecolor{currentstroke}%
\pgfsetstrokeopacity{0.700000}%
\pgfsetdash{}{0pt}%
\pgfpathmoveto{\pgfqpoint{3.253551in}{0.550000in}}%
\pgfpathlineto{\pgfqpoint{3.281072in}{0.550000in}}%
\pgfpathlineto{\pgfqpoint{3.281072in}{0.986772in}}%
\pgfpathlineto{\pgfqpoint{3.253551in}{0.986772in}}%
\pgfpathlineto{\pgfqpoint{3.253551in}{0.550000in}}%
\pgfpathclose%
\pgfusepath{fill}%
\end{pgfscope}%
\begin{pgfscope}%
\pgfpathrectangle{\pgfqpoint{1.250000in}{0.550000in}}{\pgfqpoint{7.750000in}{3.850000in}}%
\pgfusepath{clip}%
\pgfsetbuttcap%
\pgfsetmiterjoin%
\definecolor{currentfill}{rgb}{0.501961,0.501961,0.501961}%
\pgfsetfillcolor{currentfill}%
\pgfsetfillopacity{0.700000}%
\pgfsetlinewidth{0.000000pt}%
\definecolor{currentstroke}{rgb}{0.000000,0.000000,0.000000}%
\pgfsetstrokecolor{currentstroke}%
\pgfsetstrokeopacity{0.700000}%
\pgfsetdash{}{0pt}%
\pgfpathmoveto{\pgfqpoint{3.281072in}{0.550000in}}%
\pgfpathlineto{\pgfqpoint{3.308594in}{0.550000in}}%
\pgfpathlineto{\pgfqpoint{3.308594in}{0.550000in}}%
\pgfpathlineto{\pgfqpoint{3.281072in}{0.550000in}}%
\pgfpathlineto{\pgfqpoint{3.281072in}{0.550000in}}%
\pgfpathclose%
\pgfusepath{fill}%
\end{pgfscope}%
\begin{pgfscope}%
\pgfpathrectangle{\pgfqpoint{1.250000in}{0.550000in}}{\pgfqpoint{7.750000in}{3.850000in}}%
\pgfusepath{clip}%
\pgfsetbuttcap%
\pgfsetmiterjoin%
\definecolor{currentfill}{rgb}{0.501961,0.501961,0.501961}%
\pgfsetfillcolor{currentfill}%
\pgfsetfillopacity{0.700000}%
\pgfsetlinewidth{0.000000pt}%
\definecolor{currentstroke}{rgb}{0.000000,0.000000,0.000000}%
\pgfsetstrokecolor{currentstroke}%
\pgfsetstrokeopacity{0.700000}%
\pgfsetdash{}{0pt}%
\pgfpathmoveto{\pgfqpoint{3.308594in}{0.550000in}}%
\pgfpathlineto{\pgfqpoint{3.336115in}{0.550000in}}%
\pgfpathlineto{\pgfqpoint{3.336115in}{0.974903in}}%
\pgfpathlineto{\pgfqpoint{3.308594in}{0.974903in}}%
\pgfpathlineto{\pgfqpoint{3.308594in}{0.550000in}}%
\pgfpathclose%
\pgfusepath{fill}%
\end{pgfscope}%
\begin{pgfscope}%
\pgfpathrectangle{\pgfqpoint{1.250000in}{0.550000in}}{\pgfqpoint{7.750000in}{3.850000in}}%
\pgfusepath{clip}%
\pgfsetbuttcap%
\pgfsetmiterjoin%
\definecolor{currentfill}{rgb}{0.501961,0.501961,0.501961}%
\pgfsetfillcolor{currentfill}%
\pgfsetfillopacity{0.700000}%
\pgfsetlinewidth{0.000000pt}%
\definecolor{currentstroke}{rgb}{0.000000,0.000000,0.000000}%
\pgfsetstrokecolor{currentstroke}%
\pgfsetstrokeopacity{0.700000}%
\pgfsetdash{}{0pt}%
\pgfpathmoveto{\pgfqpoint{3.336115in}{0.550000in}}%
\pgfpathlineto{\pgfqpoint{3.363636in}{0.550000in}}%
\pgfpathlineto{\pgfqpoint{3.363636in}{0.948396in}}%
\pgfpathlineto{\pgfqpoint{3.336115in}{0.948396in}}%
\pgfpathlineto{\pgfqpoint{3.336115in}{0.550000in}}%
\pgfpathclose%
\pgfusepath{fill}%
\end{pgfscope}%
\begin{pgfscope}%
\pgfpathrectangle{\pgfqpoint{1.250000in}{0.550000in}}{\pgfqpoint{7.750000in}{3.850000in}}%
\pgfusepath{clip}%
\pgfsetbuttcap%
\pgfsetmiterjoin%
\definecolor{currentfill}{rgb}{0.501961,0.501961,0.501961}%
\pgfsetfillcolor{currentfill}%
\pgfsetfillopacity{0.700000}%
\pgfsetlinewidth{0.000000pt}%
\definecolor{currentstroke}{rgb}{0.000000,0.000000,0.000000}%
\pgfsetstrokecolor{currentstroke}%
\pgfsetstrokeopacity{0.700000}%
\pgfsetdash{}{0pt}%
\pgfpathmoveto{\pgfqpoint{3.363636in}{0.550000in}}%
\pgfpathlineto{\pgfqpoint{3.391158in}{0.550000in}}%
\pgfpathlineto{\pgfqpoint{3.391158in}{0.914768in}}%
\pgfpathlineto{\pgfqpoint{3.363636in}{0.914768in}}%
\pgfpathlineto{\pgfqpoint{3.363636in}{0.550000in}}%
\pgfpathclose%
\pgfusepath{fill}%
\end{pgfscope}%
\begin{pgfscope}%
\pgfpathrectangle{\pgfqpoint{1.250000in}{0.550000in}}{\pgfqpoint{7.750000in}{3.850000in}}%
\pgfusepath{clip}%
\pgfsetbuttcap%
\pgfsetmiterjoin%
\definecolor{currentfill}{rgb}{0.501961,0.501961,0.501961}%
\pgfsetfillcolor{currentfill}%
\pgfsetfillopacity{0.700000}%
\pgfsetlinewidth{0.000000pt}%
\definecolor{currentstroke}{rgb}{0.000000,0.000000,0.000000}%
\pgfsetstrokecolor{currentstroke}%
\pgfsetstrokeopacity{0.700000}%
\pgfsetdash{}{0pt}%
\pgfpathmoveto{\pgfqpoint{3.391158in}{0.550000in}}%
\pgfpathlineto{\pgfqpoint{3.418679in}{0.550000in}}%
\pgfpathlineto{\pgfqpoint{3.418679in}{0.925845in}}%
\pgfpathlineto{\pgfqpoint{3.391158in}{0.925845in}}%
\pgfpathlineto{\pgfqpoint{3.391158in}{0.550000in}}%
\pgfpathclose%
\pgfusepath{fill}%
\end{pgfscope}%
\begin{pgfscope}%
\pgfpathrectangle{\pgfqpoint{1.250000in}{0.550000in}}{\pgfqpoint{7.750000in}{3.850000in}}%
\pgfusepath{clip}%
\pgfsetbuttcap%
\pgfsetmiterjoin%
\definecolor{currentfill}{rgb}{0.501961,0.501961,0.501961}%
\pgfsetfillcolor{currentfill}%
\pgfsetfillopacity{0.700000}%
\pgfsetlinewidth{0.000000pt}%
\definecolor{currentstroke}{rgb}{0.000000,0.000000,0.000000}%
\pgfsetstrokecolor{currentstroke}%
\pgfsetstrokeopacity{0.700000}%
\pgfsetdash{}{0pt}%
\pgfpathmoveto{\pgfqpoint{3.418679in}{0.550000in}}%
\pgfpathlineto{\pgfqpoint{3.446200in}{0.550000in}}%
\pgfpathlineto{\pgfqpoint{3.446200in}{0.550000in}}%
\pgfpathlineto{\pgfqpoint{3.418679in}{0.550000in}}%
\pgfpathlineto{\pgfqpoint{3.418679in}{0.550000in}}%
\pgfpathclose%
\pgfusepath{fill}%
\end{pgfscope}%
\begin{pgfscope}%
\pgfpathrectangle{\pgfqpoint{1.250000in}{0.550000in}}{\pgfqpoint{7.750000in}{3.850000in}}%
\pgfusepath{clip}%
\pgfsetbuttcap%
\pgfsetmiterjoin%
\definecolor{currentfill}{rgb}{0.501961,0.501961,0.501961}%
\pgfsetfillcolor{currentfill}%
\pgfsetfillopacity{0.700000}%
\pgfsetlinewidth{0.000000pt}%
\definecolor{currentstroke}{rgb}{0.000000,0.000000,0.000000}%
\pgfsetstrokecolor{currentstroke}%
\pgfsetstrokeopacity{0.700000}%
\pgfsetdash{}{0pt}%
\pgfpathmoveto{\pgfqpoint{3.446200in}{0.550000in}}%
\pgfpathlineto{\pgfqpoint{3.473722in}{0.550000in}}%
\pgfpathlineto{\pgfqpoint{3.473722in}{0.927032in}}%
\pgfpathlineto{\pgfqpoint{3.446200in}{0.927032in}}%
\pgfpathlineto{\pgfqpoint{3.446200in}{0.550000in}}%
\pgfpathclose%
\pgfusepath{fill}%
\end{pgfscope}%
\begin{pgfscope}%
\pgfpathrectangle{\pgfqpoint{1.250000in}{0.550000in}}{\pgfqpoint{7.750000in}{3.850000in}}%
\pgfusepath{clip}%
\pgfsetbuttcap%
\pgfsetmiterjoin%
\definecolor{currentfill}{rgb}{0.501961,0.501961,0.501961}%
\pgfsetfillcolor{currentfill}%
\pgfsetfillopacity{0.700000}%
\pgfsetlinewidth{0.000000pt}%
\definecolor{currentstroke}{rgb}{0.000000,0.000000,0.000000}%
\pgfsetstrokecolor{currentstroke}%
\pgfsetstrokeopacity{0.700000}%
\pgfsetdash{}{0pt}%
\pgfpathmoveto{\pgfqpoint{3.473722in}{0.550000in}}%
\pgfpathlineto{\pgfqpoint{3.501243in}{0.550000in}}%
\pgfpathlineto{\pgfqpoint{3.501243in}{0.900921in}}%
\pgfpathlineto{\pgfqpoint{3.473722in}{0.900921in}}%
\pgfpathlineto{\pgfqpoint{3.473722in}{0.550000in}}%
\pgfpathclose%
\pgfusepath{fill}%
\end{pgfscope}%
\begin{pgfscope}%
\pgfpathrectangle{\pgfqpoint{1.250000in}{0.550000in}}{\pgfqpoint{7.750000in}{3.850000in}}%
\pgfusepath{clip}%
\pgfsetbuttcap%
\pgfsetmiterjoin%
\definecolor{currentfill}{rgb}{0.501961,0.501961,0.501961}%
\pgfsetfillcolor{currentfill}%
\pgfsetfillopacity{0.700000}%
\pgfsetlinewidth{0.000000pt}%
\definecolor{currentstroke}{rgb}{0.000000,0.000000,0.000000}%
\pgfsetstrokecolor{currentstroke}%
\pgfsetstrokeopacity{0.700000}%
\pgfsetdash{}{0pt}%
\pgfpathmoveto{\pgfqpoint{3.501243in}{0.550000in}}%
\pgfpathlineto{\pgfqpoint{3.528764in}{0.550000in}}%
\pgfpathlineto{\pgfqpoint{3.528764in}{0.904086in}}%
\pgfpathlineto{\pgfqpoint{3.501243in}{0.904086in}}%
\pgfpathlineto{\pgfqpoint{3.501243in}{0.550000in}}%
\pgfpathclose%
\pgfusepath{fill}%
\end{pgfscope}%
\begin{pgfscope}%
\pgfpathrectangle{\pgfqpoint{1.250000in}{0.550000in}}{\pgfqpoint{7.750000in}{3.850000in}}%
\pgfusepath{clip}%
\pgfsetbuttcap%
\pgfsetmiterjoin%
\definecolor{currentfill}{rgb}{0.501961,0.501961,0.501961}%
\pgfsetfillcolor{currentfill}%
\pgfsetfillopacity{0.700000}%
\pgfsetlinewidth{0.000000pt}%
\definecolor{currentstroke}{rgb}{0.000000,0.000000,0.000000}%
\pgfsetstrokecolor{currentstroke}%
\pgfsetstrokeopacity{0.700000}%
\pgfsetdash{}{0pt}%
\pgfpathmoveto{\pgfqpoint{3.528764in}{0.550000in}}%
\pgfpathlineto{\pgfqpoint{3.556286in}{0.550000in}}%
\pgfpathlineto{\pgfqpoint{3.556286in}{0.911207in}}%
\pgfpathlineto{\pgfqpoint{3.528764in}{0.911207in}}%
\pgfpathlineto{\pgfqpoint{3.528764in}{0.550000in}}%
\pgfpathclose%
\pgfusepath{fill}%
\end{pgfscope}%
\begin{pgfscope}%
\pgfpathrectangle{\pgfqpoint{1.250000in}{0.550000in}}{\pgfqpoint{7.750000in}{3.850000in}}%
\pgfusepath{clip}%
\pgfsetbuttcap%
\pgfsetmiterjoin%
\definecolor{currentfill}{rgb}{0.501961,0.501961,0.501961}%
\pgfsetfillcolor{currentfill}%
\pgfsetfillopacity{0.700000}%
\pgfsetlinewidth{0.000000pt}%
\definecolor{currentstroke}{rgb}{0.000000,0.000000,0.000000}%
\pgfsetstrokecolor{currentstroke}%
\pgfsetstrokeopacity{0.700000}%
\pgfsetdash{}{0pt}%
\pgfpathmoveto{\pgfqpoint{3.556286in}{0.550000in}}%
\pgfpathlineto{\pgfqpoint{3.583807in}{0.550000in}}%
\pgfpathlineto{\pgfqpoint{3.583807in}{0.550000in}}%
\pgfpathlineto{\pgfqpoint{3.556286in}{0.550000in}}%
\pgfpathlineto{\pgfqpoint{3.556286in}{0.550000in}}%
\pgfpathclose%
\pgfusepath{fill}%
\end{pgfscope}%
\begin{pgfscope}%
\pgfpathrectangle{\pgfqpoint{1.250000in}{0.550000in}}{\pgfqpoint{7.750000in}{3.850000in}}%
\pgfusepath{clip}%
\pgfsetbuttcap%
\pgfsetmiterjoin%
\definecolor{currentfill}{rgb}{0.501961,0.501961,0.501961}%
\pgfsetfillcolor{currentfill}%
\pgfsetfillopacity{0.700000}%
\pgfsetlinewidth{0.000000pt}%
\definecolor{currentstroke}{rgb}{0.000000,0.000000,0.000000}%
\pgfsetstrokecolor{currentstroke}%
\pgfsetstrokeopacity{0.700000}%
\pgfsetdash{}{0pt}%
\pgfpathmoveto{\pgfqpoint{3.583807in}{0.550000in}}%
\pgfpathlineto{\pgfqpoint{3.611328in}{0.550000in}}%
\pgfpathlineto{\pgfqpoint{3.611328in}{0.910416in}}%
\pgfpathlineto{\pgfqpoint{3.583807in}{0.910416in}}%
\pgfpathlineto{\pgfqpoint{3.583807in}{0.550000in}}%
\pgfpathclose%
\pgfusepath{fill}%
\end{pgfscope}%
\begin{pgfscope}%
\pgfpathrectangle{\pgfqpoint{1.250000in}{0.550000in}}{\pgfqpoint{7.750000in}{3.850000in}}%
\pgfusepath{clip}%
\pgfsetbuttcap%
\pgfsetmiterjoin%
\definecolor{currentfill}{rgb}{0.501961,0.501961,0.501961}%
\pgfsetfillcolor{currentfill}%
\pgfsetfillopacity{0.700000}%
\pgfsetlinewidth{0.000000pt}%
\definecolor{currentstroke}{rgb}{0.000000,0.000000,0.000000}%
\pgfsetstrokecolor{currentstroke}%
\pgfsetstrokeopacity{0.700000}%
\pgfsetdash{}{0pt}%
\pgfpathmoveto{\pgfqpoint{3.611328in}{0.550000in}}%
\pgfpathlineto{\pgfqpoint{3.638849in}{0.550000in}}%
\pgfpathlineto{\pgfqpoint{3.638849in}{0.893008in}}%
\pgfpathlineto{\pgfqpoint{3.611328in}{0.893008in}}%
\pgfpathlineto{\pgfqpoint{3.611328in}{0.550000in}}%
\pgfpathclose%
\pgfusepath{fill}%
\end{pgfscope}%
\begin{pgfscope}%
\pgfpathrectangle{\pgfqpoint{1.250000in}{0.550000in}}{\pgfqpoint{7.750000in}{3.850000in}}%
\pgfusepath{clip}%
\pgfsetbuttcap%
\pgfsetmiterjoin%
\definecolor{currentfill}{rgb}{0.501961,0.501961,0.501961}%
\pgfsetfillcolor{currentfill}%
\pgfsetfillopacity{0.700000}%
\pgfsetlinewidth{0.000000pt}%
\definecolor{currentstroke}{rgb}{0.000000,0.000000,0.000000}%
\pgfsetstrokecolor{currentstroke}%
\pgfsetstrokeopacity{0.700000}%
\pgfsetdash{}{0pt}%
\pgfpathmoveto{\pgfqpoint{3.638849in}{0.550000in}}%
\pgfpathlineto{\pgfqpoint{3.666371in}{0.550000in}}%
\pgfpathlineto{\pgfqpoint{3.666371in}{0.896964in}}%
\pgfpathlineto{\pgfqpoint{3.638849in}{0.896964in}}%
\pgfpathlineto{\pgfqpoint{3.638849in}{0.550000in}}%
\pgfpathclose%
\pgfusepath{fill}%
\end{pgfscope}%
\begin{pgfscope}%
\pgfpathrectangle{\pgfqpoint{1.250000in}{0.550000in}}{\pgfqpoint{7.750000in}{3.850000in}}%
\pgfusepath{clip}%
\pgfsetbuttcap%
\pgfsetmiterjoin%
\definecolor{currentfill}{rgb}{0.501961,0.501961,0.501961}%
\pgfsetfillcolor{currentfill}%
\pgfsetfillopacity{0.700000}%
\pgfsetlinewidth{0.000000pt}%
\definecolor{currentstroke}{rgb}{0.000000,0.000000,0.000000}%
\pgfsetstrokecolor{currentstroke}%
\pgfsetstrokeopacity{0.700000}%
\pgfsetdash{}{0pt}%
\pgfpathmoveto{\pgfqpoint{3.666371in}{0.550000in}}%
\pgfpathlineto{\pgfqpoint{3.693892in}{0.550000in}}%
\pgfpathlineto{\pgfqpoint{3.693892in}{0.893799in}}%
\pgfpathlineto{\pgfqpoint{3.666371in}{0.893799in}}%
\pgfpathlineto{\pgfqpoint{3.666371in}{0.550000in}}%
\pgfpathclose%
\pgfusepath{fill}%
\end{pgfscope}%
\begin{pgfscope}%
\pgfpathrectangle{\pgfqpoint{1.250000in}{0.550000in}}{\pgfqpoint{7.750000in}{3.850000in}}%
\pgfusepath{clip}%
\pgfsetbuttcap%
\pgfsetmiterjoin%
\definecolor{currentfill}{rgb}{0.501961,0.501961,0.501961}%
\pgfsetfillcolor{currentfill}%
\pgfsetfillopacity{0.700000}%
\pgfsetlinewidth{0.000000pt}%
\definecolor{currentstroke}{rgb}{0.000000,0.000000,0.000000}%
\pgfsetstrokecolor{currentstroke}%
\pgfsetstrokeopacity{0.700000}%
\pgfsetdash{}{0pt}%
\pgfpathmoveto{\pgfqpoint{3.693892in}{0.550000in}}%
\pgfpathlineto{\pgfqpoint{3.721413in}{0.550000in}}%
\pgfpathlineto{\pgfqpoint{3.721413in}{0.550000in}}%
\pgfpathlineto{\pgfqpoint{3.693892in}{0.550000in}}%
\pgfpathlineto{\pgfqpoint{3.693892in}{0.550000in}}%
\pgfpathclose%
\pgfusepath{fill}%
\end{pgfscope}%
\begin{pgfscope}%
\pgfpathrectangle{\pgfqpoint{1.250000in}{0.550000in}}{\pgfqpoint{7.750000in}{3.850000in}}%
\pgfusepath{clip}%
\pgfsetbuttcap%
\pgfsetmiterjoin%
\definecolor{currentfill}{rgb}{0.501961,0.501961,0.501961}%
\pgfsetfillcolor{currentfill}%
\pgfsetfillopacity{0.700000}%
\pgfsetlinewidth{0.000000pt}%
\definecolor{currentstroke}{rgb}{0.000000,0.000000,0.000000}%
\pgfsetstrokecolor{currentstroke}%
\pgfsetstrokeopacity{0.700000}%
\pgfsetdash{}{0pt}%
\pgfpathmoveto{\pgfqpoint{3.721413in}{0.550000in}}%
\pgfpathlineto{\pgfqpoint{3.748935in}{0.550000in}}%
\pgfpathlineto{\pgfqpoint{3.748935in}{0.875996in}}%
\pgfpathlineto{\pgfqpoint{3.721413in}{0.875996in}}%
\pgfpathlineto{\pgfqpoint{3.721413in}{0.550000in}}%
\pgfpathclose%
\pgfusepath{fill}%
\end{pgfscope}%
\begin{pgfscope}%
\pgfpathrectangle{\pgfqpoint{1.250000in}{0.550000in}}{\pgfqpoint{7.750000in}{3.850000in}}%
\pgfusepath{clip}%
\pgfsetbuttcap%
\pgfsetmiterjoin%
\definecolor{currentfill}{rgb}{0.501961,0.501961,0.501961}%
\pgfsetfillcolor{currentfill}%
\pgfsetfillopacity{0.700000}%
\pgfsetlinewidth{0.000000pt}%
\definecolor{currentstroke}{rgb}{0.000000,0.000000,0.000000}%
\pgfsetstrokecolor{currentstroke}%
\pgfsetstrokeopacity{0.700000}%
\pgfsetdash{}{0pt}%
\pgfpathmoveto{\pgfqpoint{3.748935in}{0.550000in}}%
\pgfpathlineto{\pgfqpoint{3.776456in}{0.550000in}}%
\pgfpathlineto{\pgfqpoint{3.776456in}{0.877579in}}%
\pgfpathlineto{\pgfqpoint{3.748935in}{0.877579in}}%
\pgfpathlineto{\pgfqpoint{3.748935in}{0.550000in}}%
\pgfpathclose%
\pgfusepath{fill}%
\end{pgfscope}%
\begin{pgfscope}%
\pgfpathrectangle{\pgfqpoint{1.250000in}{0.550000in}}{\pgfqpoint{7.750000in}{3.850000in}}%
\pgfusepath{clip}%
\pgfsetbuttcap%
\pgfsetmiterjoin%
\definecolor{currentfill}{rgb}{0.501961,0.501961,0.501961}%
\pgfsetfillcolor{currentfill}%
\pgfsetfillopacity{0.700000}%
\pgfsetlinewidth{0.000000pt}%
\definecolor{currentstroke}{rgb}{0.000000,0.000000,0.000000}%
\pgfsetstrokecolor{currentstroke}%
\pgfsetstrokeopacity{0.700000}%
\pgfsetdash{}{0pt}%
\pgfpathmoveto{\pgfqpoint{3.776456in}{0.550000in}}%
\pgfpathlineto{\pgfqpoint{3.803977in}{0.550000in}}%
\pgfpathlineto{\pgfqpoint{3.803977in}{0.868084in}}%
\pgfpathlineto{\pgfqpoint{3.776456in}{0.868084in}}%
\pgfpathlineto{\pgfqpoint{3.776456in}{0.550000in}}%
\pgfpathclose%
\pgfusepath{fill}%
\end{pgfscope}%
\begin{pgfscope}%
\pgfpathrectangle{\pgfqpoint{1.250000in}{0.550000in}}{\pgfqpoint{7.750000in}{3.850000in}}%
\pgfusepath{clip}%
\pgfsetbuttcap%
\pgfsetmiterjoin%
\definecolor{currentfill}{rgb}{0.501961,0.501961,0.501961}%
\pgfsetfillcolor{currentfill}%
\pgfsetfillopacity{0.700000}%
\pgfsetlinewidth{0.000000pt}%
\definecolor{currentstroke}{rgb}{0.000000,0.000000,0.000000}%
\pgfsetstrokecolor{currentstroke}%
\pgfsetstrokeopacity{0.700000}%
\pgfsetdash{}{0pt}%
\pgfpathmoveto{\pgfqpoint{3.803977in}{0.550000in}}%
\pgfpathlineto{\pgfqpoint{3.831499in}{0.550000in}}%
\pgfpathlineto{\pgfqpoint{3.831499in}{0.866501in}}%
\pgfpathlineto{\pgfqpoint{3.803977in}{0.866501in}}%
\pgfpathlineto{\pgfqpoint{3.803977in}{0.550000in}}%
\pgfpathclose%
\pgfusepath{fill}%
\end{pgfscope}%
\begin{pgfscope}%
\pgfpathrectangle{\pgfqpoint{1.250000in}{0.550000in}}{\pgfqpoint{7.750000in}{3.850000in}}%
\pgfusepath{clip}%
\pgfsetbuttcap%
\pgfsetmiterjoin%
\definecolor{currentfill}{rgb}{0.501961,0.501961,0.501961}%
\pgfsetfillcolor{currentfill}%
\pgfsetfillopacity{0.700000}%
\pgfsetlinewidth{0.000000pt}%
\definecolor{currentstroke}{rgb}{0.000000,0.000000,0.000000}%
\pgfsetstrokecolor{currentstroke}%
\pgfsetstrokeopacity{0.700000}%
\pgfsetdash{}{0pt}%
\pgfpathmoveto{\pgfqpoint{3.831499in}{0.550000in}}%
\pgfpathlineto{\pgfqpoint{3.859020in}{0.550000in}}%
\pgfpathlineto{\pgfqpoint{3.859020in}{0.550000in}}%
\pgfpathlineto{\pgfqpoint{3.831499in}{0.550000in}}%
\pgfpathlineto{\pgfqpoint{3.831499in}{0.550000in}}%
\pgfpathclose%
\pgfusepath{fill}%
\end{pgfscope}%
\begin{pgfscope}%
\pgfpathrectangle{\pgfqpoint{1.250000in}{0.550000in}}{\pgfqpoint{7.750000in}{3.850000in}}%
\pgfusepath{clip}%
\pgfsetbuttcap%
\pgfsetmiterjoin%
\definecolor{currentfill}{rgb}{0.501961,0.501961,0.501961}%
\pgfsetfillcolor{currentfill}%
\pgfsetfillopacity{0.700000}%
\pgfsetlinewidth{0.000000pt}%
\definecolor{currentstroke}{rgb}{0.000000,0.000000,0.000000}%
\pgfsetstrokecolor{currentstroke}%
\pgfsetstrokeopacity{0.700000}%
\pgfsetdash{}{0pt}%
\pgfpathmoveto{\pgfqpoint{3.859020in}{0.550000in}}%
\pgfpathlineto{\pgfqpoint{3.886541in}{0.550000in}}%
\pgfpathlineto{\pgfqpoint{3.886541in}{0.841577in}}%
\pgfpathlineto{\pgfqpoint{3.859020in}{0.841577in}}%
\pgfpathlineto{\pgfqpoint{3.859020in}{0.550000in}}%
\pgfpathclose%
\pgfusepath{fill}%
\end{pgfscope}%
\begin{pgfscope}%
\pgfpathrectangle{\pgfqpoint{1.250000in}{0.550000in}}{\pgfqpoint{7.750000in}{3.850000in}}%
\pgfusepath{clip}%
\pgfsetbuttcap%
\pgfsetmiterjoin%
\definecolor{currentfill}{rgb}{0.501961,0.501961,0.501961}%
\pgfsetfillcolor{currentfill}%
\pgfsetfillopacity{0.700000}%
\pgfsetlinewidth{0.000000pt}%
\definecolor{currentstroke}{rgb}{0.000000,0.000000,0.000000}%
\pgfsetstrokecolor{currentstroke}%
\pgfsetstrokeopacity{0.700000}%
\pgfsetdash{}{0pt}%
\pgfpathmoveto{\pgfqpoint{3.886541in}{0.550000in}}%
\pgfpathlineto{\pgfqpoint{3.914062in}{0.550000in}}%
\pgfpathlineto{\pgfqpoint{3.914062in}{0.837620in}}%
\pgfpathlineto{\pgfqpoint{3.886541in}{0.837620in}}%
\pgfpathlineto{\pgfqpoint{3.886541in}{0.550000in}}%
\pgfpathclose%
\pgfusepath{fill}%
\end{pgfscope}%
\begin{pgfscope}%
\pgfpathrectangle{\pgfqpoint{1.250000in}{0.550000in}}{\pgfqpoint{7.750000in}{3.850000in}}%
\pgfusepath{clip}%
\pgfsetbuttcap%
\pgfsetmiterjoin%
\definecolor{currentfill}{rgb}{0.501961,0.501961,0.501961}%
\pgfsetfillcolor{currentfill}%
\pgfsetfillopacity{0.700000}%
\pgfsetlinewidth{0.000000pt}%
\definecolor{currentstroke}{rgb}{0.000000,0.000000,0.000000}%
\pgfsetstrokecolor{currentstroke}%
\pgfsetstrokeopacity{0.700000}%
\pgfsetdash{}{0pt}%
\pgfpathmoveto{\pgfqpoint{3.914062in}{0.550000in}}%
\pgfpathlineto{\pgfqpoint{3.941584in}{0.550000in}}%
\pgfpathlineto{\pgfqpoint{3.941584in}{0.827730in}}%
\pgfpathlineto{\pgfqpoint{3.914062in}{0.827730in}}%
\pgfpathlineto{\pgfqpoint{3.914062in}{0.550000in}}%
\pgfpathclose%
\pgfusepath{fill}%
\end{pgfscope}%
\begin{pgfscope}%
\pgfpathrectangle{\pgfqpoint{1.250000in}{0.550000in}}{\pgfqpoint{7.750000in}{3.850000in}}%
\pgfusepath{clip}%
\pgfsetbuttcap%
\pgfsetmiterjoin%
\definecolor{currentfill}{rgb}{0.501961,0.501961,0.501961}%
\pgfsetfillcolor{currentfill}%
\pgfsetfillopacity{0.700000}%
\pgfsetlinewidth{0.000000pt}%
\definecolor{currentstroke}{rgb}{0.000000,0.000000,0.000000}%
\pgfsetstrokecolor{currentstroke}%
\pgfsetstrokeopacity{0.700000}%
\pgfsetdash{}{0pt}%
\pgfpathmoveto{\pgfqpoint{3.941584in}{0.550000in}}%
\pgfpathlineto{\pgfqpoint{3.969105in}{0.550000in}}%
\pgfpathlineto{\pgfqpoint{3.969105in}{0.550000in}}%
\pgfpathlineto{\pgfqpoint{3.941584in}{0.550000in}}%
\pgfpathlineto{\pgfqpoint{3.941584in}{0.550000in}}%
\pgfpathclose%
\pgfusepath{fill}%
\end{pgfscope}%
\begin{pgfscope}%
\pgfpathrectangle{\pgfqpoint{1.250000in}{0.550000in}}{\pgfqpoint{7.750000in}{3.850000in}}%
\pgfusepath{clip}%
\pgfsetbuttcap%
\pgfsetmiterjoin%
\definecolor{currentfill}{rgb}{0.501961,0.501961,0.501961}%
\pgfsetfillcolor{currentfill}%
\pgfsetfillopacity{0.700000}%
\pgfsetlinewidth{0.000000pt}%
\definecolor{currentstroke}{rgb}{0.000000,0.000000,0.000000}%
\pgfsetstrokecolor{currentstroke}%
\pgfsetstrokeopacity{0.700000}%
\pgfsetdash{}{0pt}%
\pgfpathmoveto{\pgfqpoint{3.969105in}{0.550000in}}%
\pgfpathlineto{\pgfqpoint{3.996626in}{0.550000in}}%
\pgfpathlineto{\pgfqpoint{3.996626in}{0.813092in}}%
\pgfpathlineto{\pgfqpoint{3.969105in}{0.813092in}}%
\pgfpathlineto{\pgfqpoint{3.969105in}{0.550000in}}%
\pgfpathclose%
\pgfusepath{fill}%
\end{pgfscope}%
\begin{pgfscope}%
\pgfpathrectangle{\pgfqpoint{1.250000in}{0.550000in}}{\pgfqpoint{7.750000in}{3.850000in}}%
\pgfusepath{clip}%
\pgfsetbuttcap%
\pgfsetmiterjoin%
\definecolor{currentfill}{rgb}{0.501961,0.501961,0.501961}%
\pgfsetfillcolor{currentfill}%
\pgfsetfillopacity{0.700000}%
\pgfsetlinewidth{0.000000pt}%
\definecolor{currentstroke}{rgb}{0.000000,0.000000,0.000000}%
\pgfsetstrokecolor{currentstroke}%
\pgfsetstrokeopacity{0.700000}%
\pgfsetdash{}{0pt}%
\pgfpathmoveto{\pgfqpoint{3.996626in}{0.550000in}}%
\pgfpathlineto{\pgfqpoint{4.024148in}{0.550000in}}%
\pgfpathlineto{\pgfqpoint{4.024148in}{0.815070in}}%
\pgfpathlineto{\pgfqpoint{3.996626in}{0.815070in}}%
\pgfpathlineto{\pgfqpoint{3.996626in}{0.550000in}}%
\pgfpathclose%
\pgfusepath{fill}%
\end{pgfscope}%
\begin{pgfscope}%
\pgfpathrectangle{\pgfqpoint{1.250000in}{0.550000in}}{\pgfqpoint{7.750000in}{3.850000in}}%
\pgfusepath{clip}%
\pgfsetbuttcap%
\pgfsetmiterjoin%
\definecolor{currentfill}{rgb}{0.501961,0.501961,0.501961}%
\pgfsetfillcolor{currentfill}%
\pgfsetfillopacity{0.700000}%
\pgfsetlinewidth{0.000000pt}%
\definecolor{currentstroke}{rgb}{0.000000,0.000000,0.000000}%
\pgfsetstrokecolor{currentstroke}%
\pgfsetstrokeopacity{0.700000}%
\pgfsetdash{}{0pt}%
\pgfpathmoveto{\pgfqpoint{4.024148in}{0.550000in}}%
\pgfpathlineto{\pgfqpoint{4.051669in}{0.550000in}}%
\pgfpathlineto{\pgfqpoint{4.051669in}{0.804388in}}%
\pgfpathlineto{\pgfqpoint{4.024148in}{0.804388in}}%
\pgfpathlineto{\pgfqpoint{4.024148in}{0.550000in}}%
\pgfpathclose%
\pgfusepath{fill}%
\end{pgfscope}%
\begin{pgfscope}%
\pgfpathrectangle{\pgfqpoint{1.250000in}{0.550000in}}{\pgfqpoint{7.750000in}{3.850000in}}%
\pgfusepath{clip}%
\pgfsetbuttcap%
\pgfsetmiterjoin%
\definecolor{currentfill}{rgb}{0.501961,0.501961,0.501961}%
\pgfsetfillcolor{currentfill}%
\pgfsetfillopacity{0.700000}%
\pgfsetlinewidth{0.000000pt}%
\definecolor{currentstroke}{rgb}{0.000000,0.000000,0.000000}%
\pgfsetstrokecolor{currentstroke}%
\pgfsetstrokeopacity{0.700000}%
\pgfsetdash{}{0pt}%
\pgfpathmoveto{\pgfqpoint{4.051669in}{0.550000in}}%
\pgfpathlineto{\pgfqpoint{4.079190in}{0.550000in}}%
\pgfpathlineto{\pgfqpoint{4.079190in}{0.794893in}}%
\pgfpathlineto{\pgfqpoint{4.051669in}{0.794893in}}%
\pgfpathlineto{\pgfqpoint{4.051669in}{0.550000in}}%
\pgfpathclose%
\pgfusepath{fill}%
\end{pgfscope}%
\begin{pgfscope}%
\pgfpathrectangle{\pgfqpoint{1.250000in}{0.550000in}}{\pgfqpoint{7.750000in}{3.850000in}}%
\pgfusepath{clip}%
\pgfsetbuttcap%
\pgfsetmiterjoin%
\definecolor{currentfill}{rgb}{0.501961,0.501961,0.501961}%
\pgfsetfillcolor{currentfill}%
\pgfsetfillopacity{0.700000}%
\pgfsetlinewidth{0.000000pt}%
\definecolor{currentstroke}{rgb}{0.000000,0.000000,0.000000}%
\pgfsetstrokecolor{currentstroke}%
\pgfsetstrokeopacity{0.700000}%
\pgfsetdash{}{0pt}%
\pgfpathmoveto{\pgfqpoint{4.079190in}{0.550000in}}%
\pgfpathlineto{\pgfqpoint{4.106712in}{0.550000in}}%
\pgfpathlineto{\pgfqpoint{4.106712in}{0.550000in}}%
\pgfpathlineto{\pgfqpoint{4.079190in}{0.550000in}}%
\pgfpathlineto{\pgfqpoint{4.079190in}{0.550000in}}%
\pgfpathclose%
\pgfusepath{fill}%
\end{pgfscope}%
\begin{pgfscope}%
\pgfpathrectangle{\pgfqpoint{1.250000in}{0.550000in}}{\pgfqpoint{7.750000in}{3.850000in}}%
\pgfusepath{clip}%
\pgfsetbuttcap%
\pgfsetmiterjoin%
\definecolor{currentfill}{rgb}{0.501961,0.501961,0.501961}%
\pgfsetfillcolor{currentfill}%
\pgfsetfillopacity{0.700000}%
\pgfsetlinewidth{0.000000pt}%
\definecolor{currentstroke}{rgb}{0.000000,0.000000,0.000000}%
\pgfsetstrokecolor{currentstroke}%
\pgfsetstrokeopacity{0.700000}%
\pgfsetdash{}{0pt}%
\pgfpathmoveto{\pgfqpoint{4.106712in}{0.550000in}}%
\pgfpathlineto{\pgfqpoint{4.134233in}{0.550000in}}%
\pgfpathlineto{\pgfqpoint{4.134233in}{0.799640in}}%
\pgfpathlineto{\pgfqpoint{4.106712in}{0.799640in}}%
\pgfpathlineto{\pgfqpoint{4.106712in}{0.550000in}}%
\pgfpathclose%
\pgfusepath{fill}%
\end{pgfscope}%
\begin{pgfscope}%
\pgfpathrectangle{\pgfqpoint{1.250000in}{0.550000in}}{\pgfqpoint{7.750000in}{3.850000in}}%
\pgfusepath{clip}%
\pgfsetbuttcap%
\pgfsetmiterjoin%
\definecolor{currentfill}{rgb}{0.501961,0.501961,0.501961}%
\pgfsetfillcolor{currentfill}%
\pgfsetfillopacity{0.700000}%
\pgfsetlinewidth{0.000000pt}%
\definecolor{currentstroke}{rgb}{0.000000,0.000000,0.000000}%
\pgfsetstrokecolor{currentstroke}%
\pgfsetstrokeopacity{0.700000}%
\pgfsetdash{}{0pt}%
\pgfpathmoveto{\pgfqpoint{4.134233in}{0.550000in}}%
\pgfpathlineto{\pgfqpoint{4.161754in}{0.550000in}}%
\pgfpathlineto{\pgfqpoint{4.161754in}{0.813487in}}%
\pgfpathlineto{\pgfqpoint{4.134233in}{0.813487in}}%
\pgfpathlineto{\pgfqpoint{4.134233in}{0.550000in}}%
\pgfpathclose%
\pgfusepath{fill}%
\end{pgfscope}%
\begin{pgfscope}%
\pgfpathrectangle{\pgfqpoint{1.250000in}{0.550000in}}{\pgfqpoint{7.750000in}{3.850000in}}%
\pgfusepath{clip}%
\pgfsetbuttcap%
\pgfsetmiterjoin%
\definecolor{currentfill}{rgb}{0.501961,0.501961,0.501961}%
\pgfsetfillcolor{currentfill}%
\pgfsetfillopacity{0.700000}%
\pgfsetlinewidth{0.000000pt}%
\definecolor{currentstroke}{rgb}{0.000000,0.000000,0.000000}%
\pgfsetstrokecolor{currentstroke}%
\pgfsetstrokeopacity{0.700000}%
\pgfsetdash{}{0pt}%
\pgfpathmoveto{\pgfqpoint{4.161754in}{0.550000in}}%
\pgfpathlineto{\pgfqpoint{4.189276in}{0.550000in}}%
\pgfpathlineto{\pgfqpoint{4.189276in}{0.819422in}}%
\pgfpathlineto{\pgfqpoint{4.161754in}{0.819422in}}%
\pgfpathlineto{\pgfqpoint{4.161754in}{0.550000in}}%
\pgfpathclose%
\pgfusepath{fill}%
\end{pgfscope}%
\begin{pgfscope}%
\pgfpathrectangle{\pgfqpoint{1.250000in}{0.550000in}}{\pgfqpoint{7.750000in}{3.850000in}}%
\pgfusepath{clip}%
\pgfsetbuttcap%
\pgfsetmiterjoin%
\definecolor{currentfill}{rgb}{0.501961,0.501961,0.501961}%
\pgfsetfillcolor{currentfill}%
\pgfsetfillopacity{0.700000}%
\pgfsetlinewidth{0.000000pt}%
\definecolor{currentstroke}{rgb}{0.000000,0.000000,0.000000}%
\pgfsetstrokecolor{currentstroke}%
\pgfsetstrokeopacity{0.700000}%
\pgfsetdash{}{0pt}%
\pgfpathmoveto{\pgfqpoint{4.189276in}{0.550000in}}%
\pgfpathlineto{\pgfqpoint{4.216797in}{0.550000in}}%
\pgfpathlineto{\pgfqpoint{4.216797in}{0.790937in}}%
\pgfpathlineto{\pgfqpoint{4.189276in}{0.790937in}}%
\pgfpathlineto{\pgfqpoint{4.189276in}{0.550000in}}%
\pgfpathclose%
\pgfusepath{fill}%
\end{pgfscope}%
\begin{pgfscope}%
\pgfpathrectangle{\pgfqpoint{1.250000in}{0.550000in}}{\pgfqpoint{7.750000in}{3.850000in}}%
\pgfusepath{clip}%
\pgfsetbuttcap%
\pgfsetmiterjoin%
\definecolor{currentfill}{rgb}{0.501961,0.501961,0.501961}%
\pgfsetfillcolor{currentfill}%
\pgfsetfillopacity{0.700000}%
\pgfsetlinewidth{0.000000pt}%
\definecolor{currentstroke}{rgb}{0.000000,0.000000,0.000000}%
\pgfsetstrokecolor{currentstroke}%
\pgfsetstrokeopacity{0.700000}%
\pgfsetdash{}{0pt}%
\pgfpathmoveto{\pgfqpoint{4.216797in}{0.550000in}}%
\pgfpathlineto{\pgfqpoint{4.244318in}{0.550000in}}%
\pgfpathlineto{\pgfqpoint{4.244318in}{0.550000in}}%
\pgfpathlineto{\pgfqpoint{4.216797in}{0.550000in}}%
\pgfpathlineto{\pgfqpoint{4.216797in}{0.550000in}}%
\pgfpathclose%
\pgfusepath{fill}%
\end{pgfscope}%
\begin{pgfscope}%
\pgfpathrectangle{\pgfqpoint{1.250000in}{0.550000in}}{\pgfqpoint{7.750000in}{3.850000in}}%
\pgfusepath{clip}%
\pgfsetbuttcap%
\pgfsetmiterjoin%
\definecolor{currentfill}{rgb}{0.501961,0.501961,0.501961}%
\pgfsetfillcolor{currentfill}%
\pgfsetfillopacity{0.700000}%
\pgfsetlinewidth{0.000000pt}%
\definecolor{currentstroke}{rgb}{0.000000,0.000000,0.000000}%
\pgfsetstrokecolor{currentstroke}%
\pgfsetstrokeopacity{0.700000}%
\pgfsetdash{}{0pt}%
\pgfpathmoveto{\pgfqpoint{4.244318in}{0.550000in}}%
\pgfpathlineto{\pgfqpoint{4.271839in}{0.550000in}}%
\pgfpathlineto{\pgfqpoint{4.271839in}{0.788958in}}%
\pgfpathlineto{\pgfqpoint{4.244318in}{0.788958in}}%
\pgfpathlineto{\pgfqpoint{4.244318in}{0.550000in}}%
\pgfpathclose%
\pgfusepath{fill}%
\end{pgfscope}%
\begin{pgfscope}%
\pgfpathrectangle{\pgfqpoint{1.250000in}{0.550000in}}{\pgfqpoint{7.750000in}{3.850000in}}%
\pgfusepath{clip}%
\pgfsetbuttcap%
\pgfsetmiterjoin%
\definecolor{currentfill}{rgb}{0.501961,0.501961,0.501961}%
\pgfsetfillcolor{currentfill}%
\pgfsetfillopacity{0.700000}%
\pgfsetlinewidth{0.000000pt}%
\definecolor{currentstroke}{rgb}{0.000000,0.000000,0.000000}%
\pgfsetstrokecolor{currentstroke}%
\pgfsetstrokeopacity{0.700000}%
\pgfsetdash{}{0pt}%
\pgfpathmoveto{\pgfqpoint{4.271839in}{0.550000in}}%
\pgfpathlineto{\pgfqpoint{4.299361in}{0.550000in}}%
\pgfpathlineto{\pgfqpoint{4.299361in}{0.779859in}}%
\pgfpathlineto{\pgfqpoint{4.271839in}{0.779859in}}%
\pgfpathlineto{\pgfqpoint{4.271839in}{0.550000in}}%
\pgfpathclose%
\pgfusepath{fill}%
\end{pgfscope}%
\begin{pgfscope}%
\pgfpathrectangle{\pgfqpoint{1.250000in}{0.550000in}}{\pgfqpoint{7.750000in}{3.850000in}}%
\pgfusepath{clip}%
\pgfsetbuttcap%
\pgfsetmiterjoin%
\definecolor{currentfill}{rgb}{0.501961,0.501961,0.501961}%
\pgfsetfillcolor{currentfill}%
\pgfsetfillopacity{0.700000}%
\pgfsetlinewidth{0.000000pt}%
\definecolor{currentstroke}{rgb}{0.000000,0.000000,0.000000}%
\pgfsetstrokecolor{currentstroke}%
\pgfsetstrokeopacity{0.700000}%
\pgfsetdash{}{0pt}%
\pgfpathmoveto{\pgfqpoint{4.299361in}{0.550000in}}%
\pgfpathlineto{\pgfqpoint{4.326882in}{0.550000in}}%
\pgfpathlineto{\pgfqpoint{4.326882in}{0.763243in}}%
\pgfpathlineto{\pgfqpoint{4.299361in}{0.763243in}}%
\pgfpathlineto{\pgfqpoint{4.299361in}{0.550000in}}%
\pgfpathclose%
\pgfusepath{fill}%
\end{pgfscope}%
\begin{pgfscope}%
\pgfpathrectangle{\pgfqpoint{1.250000in}{0.550000in}}{\pgfqpoint{7.750000in}{3.850000in}}%
\pgfusepath{clip}%
\pgfsetbuttcap%
\pgfsetmiterjoin%
\definecolor{currentfill}{rgb}{0.501961,0.501961,0.501961}%
\pgfsetfillcolor{currentfill}%
\pgfsetfillopacity{0.700000}%
\pgfsetlinewidth{0.000000pt}%
\definecolor{currentstroke}{rgb}{0.000000,0.000000,0.000000}%
\pgfsetstrokecolor{currentstroke}%
\pgfsetstrokeopacity{0.700000}%
\pgfsetdash{}{0pt}%
\pgfpathmoveto{\pgfqpoint{4.326882in}{0.550000in}}%
\pgfpathlineto{\pgfqpoint{4.354403in}{0.550000in}}%
\pgfpathlineto{\pgfqpoint{4.354403in}{0.756517in}}%
\pgfpathlineto{\pgfqpoint{4.326882in}{0.756517in}}%
\pgfpathlineto{\pgfqpoint{4.326882in}{0.550000in}}%
\pgfpathclose%
\pgfusepath{fill}%
\end{pgfscope}%
\begin{pgfscope}%
\pgfpathrectangle{\pgfqpoint{1.250000in}{0.550000in}}{\pgfqpoint{7.750000in}{3.850000in}}%
\pgfusepath{clip}%
\pgfsetbuttcap%
\pgfsetmiterjoin%
\definecolor{currentfill}{rgb}{0.501961,0.501961,0.501961}%
\pgfsetfillcolor{currentfill}%
\pgfsetfillopacity{0.700000}%
\pgfsetlinewidth{0.000000pt}%
\definecolor{currentstroke}{rgb}{0.000000,0.000000,0.000000}%
\pgfsetstrokecolor{currentstroke}%
\pgfsetstrokeopacity{0.700000}%
\pgfsetdash{}{0pt}%
\pgfpathmoveto{\pgfqpoint{4.354403in}{0.550000in}}%
\pgfpathlineto{\pgfqpoint{4.381925in}{0.550000in}}%
\pgfpathlineto{\pgfqpoint{4.381925in}{0.550000in}}%
\pgfpathlineto{\pgfqpoint{4.354403in}{0.550000in}}%
\pgfpathlineto{\pgfqpoint{4.354403in}{0.550000in}}%
\pgfpathclose%
\pgfusepath{fill}%
\end{pgfscope}%
\begin{pgfscope}%
\pgfpathrectangle{\pgfqpoint{1.250000in}{0.550000in}}{\pgfqpoint{7.750000in}{3.850000in}}%
\pgfusepath{clip}%
\pgfsetbuttcap%
\pgfsetmiterjoin%
\definecolor{currentfill}{rgb}{0.501961,0.501961,0.501961}%
\pgfsetfillcolor{currentfill}%
\pgfsetfillopacity{0.700000}%
\pgfsetlinewidth{0.000000pt}%
\definecolor{currentstroke}{rgb}{0.000000,0.000000,0.000000}%
\pgfsetstrokecolor{currentstroke}%
\pgfsetstrokeopacity{0.700000}%
\pgfsetdash{}{0pt}%
\pgfpathmoveto{\pgfqpoint{4.381925in}{0.550000in}}%
\pgfpathlineto{\pgfqpoint{4.409446in}{0.550000in}}%
\pgfpathlineto{\pgfqpoint{4.409446in}{0.744253in}}%
\pgfpathlineto{\pgfqpoint{4.381925in}{0.744253in}}%
\pgfpathlineto{\pgfqpoint{4.381925in}{0.550000in}}%
\pgfpathclose%
\pgfusepath{fill}%
\end{pgfscope}%
\begin{pgfscope}%
\pgfpathrectangle{\pgfqpoint{1.250000in}{0.550000in}}{\pgfqpoint{7.750000in}{3.850000in}}%
\pgfusepath{clip}%
\pgfsetbuttcap%
\pgfsetmiterjoin%
\definecolor{currentfill}{rgb}{0.501961,0.501961,0.501961}%
\pgfsetfillcolor{currentfill}%
\pgfsetfillopacity{0.700000}%
\pgfsetlinewidth{0.000000pt}%
\definecolor{currentstroke}{rgb}{0.000000,0.000000,0.000000}%
\pgfsetstrokecolor{currentstroke}%
\pgfsetstrokeopacity{0.700000}%
\pgfsetdash{}{0pt}%
\pgfpathmoveto{\pgfqpoint{4.409446in}{0.550000in}}%
\pgfpathlineto{\pgfqpoint{4.436967in}{0.550000in}}%
\pgfpathlineto{\pgfqpoint{4.436967in}{0.737923in}}%
\pgfpathlineto{\pgfqpoint{4.409446in}{0.737923in}}%
\pgfpathlineto{\pgfqpoint{4.409446in}{0.550000in}}%
\pgfpathclose%
\pgfusepath{fill}%
\end{pgfscope}%
\begin{pgfscope}%
\pgfpathrectangle{\pgfqpoint{1.250000in}{0.550000in}}{\pgfqpoint{7.750000in}{3.850000in}}%
\pgfusepath{clip}%
\pgfsetbuttcap%
\pgfsetmiterjoin%
\definecolor{currentfill}{rgb}{0.501961,0.501961,0.501961}%
\pgfsetfillcolor{currentfill}%
\pgfsetfillopacity{0.700000}%
\pgfsetlinewidth{0.000000pt}%
\definecolor{currentstroke}{rgb}{0.000000,0.000000,0.000000}%
\pgfsetstrokecolor{currentstroke}%
\pgfsetstrokeopacity{0.700000}%
\pgfsetdash{}{0pt}%
\pgfpathmoveto{\pgfqpoint{4.436967in}{0.550000in}}%
\pgfpathlineto{\pgfqpoint{4.464489in}{0.550000in}}%
\pgfpathlineto{\pgfqpoint{4.464489in}{0.731593in}}%
\pgfpathlineto{\pgfqpoint{4.436967in}{0.731593in}}%
\pgfpathlineto{\pgfqpoint{4.436967in}{0.550000in}}%
\pgfpathclose%
\pgfusepath{fill}%
\end{pgfscope}%
\begin{pgfscope}%
\pgfpathrectangle{\pgfqpoint{1.250000in}{0.550000in}}{\pgfqpoint{7.750000in}{3.850000in}}%
\pgfusepath{clip}%
\pgfsetbuttcap%
\pgfsetmiterjoin%
\definecolor{currentfill}{rgb}{0.501961,0.501961,0.501961}%
\pgfsetfillcolor{currentfill}%
\pgfsetfillopacity{0.700000}%
\pgfsetlinewidth{0.000000pt}%
\definecolor{currentstroke}{rgb}{0.000000,0.000000,0.000000}%
\pgfsetstrokecolor{currentstroke}%
\pgfsetstrokeopacity{0.700000}%
\pgfsetdash{}{0pt}%
\pgfpathmoveto{\pgfqpoint{4.464489in}{0.550000in}}%
\pgfpathlineto{\pgfqpoint{4.492010in}{0.550000in}}%
\pgfpathlineto{\pgfqpoint{4.492010in}{0.723284in}}%
\pgfpathlineto{\pgfqpoint{4.464489in}{0.723284in}}%
\pgfpathlineto{\pgfqpoint{4.464489in}{0.550000in}}%
\pgfpathclose%
\pgfusepath{fill}%
\end{pgfscope}%
\begin{pgfscope}%
\pgfpathrectangle{\pgfqpoint{1.250000in}{0.550000in}}{\pgfqpoint{7.750000in}{3.850000in}}%
\pgfusepath{clip}%
\pgfsetbuttcap%
\pgfsetmiterjoin%
\definecolor{currentfill}{rgb}{0.501961,0.501961,0.501961}%
\pgfsetfillcolor{currentfill}%
\pgfsetfillopacity{0.700000}%
\pgfsetlinewidth{0.000000pt}%
\definecolor{currentstroke}{rgb}{0.000000,0.000000,0.000000}%
\pgfsetstrokecolor{currentstroke}%
\pgfsetstrokeopacity{0.700000}%
\pgfsetdash{}{0pt}%
\pgfpathmoveto{\pgfqpoint{4.492010in}{0.550000in}}%
\pgfpathlineto{\pgfqpoint{4.519531in}{0.550000in}}%
\pgfpathlineto{\pgfqpoint{4.519531in}{0.550000in}}%
\pgfpathlineto{\pgfqpoint{4.492010in}{0.550000in}}%
\pgfpathlineto{\pgfqpoint{4.492010in}{0.550000in}}%
\pgfpathclose%
\pgfusepath{fill}%
\end{pgfscope}%
\begin{pgfscope}%
\pgfpathrectangle{\pgfqpoint{1.250000in}{0.550000in}}{\pgfqpoint{7.750000in}{3.850000in}}%
\pgfusepath{clip}%
\pgfsetbuttcap%
\pgfsetmiterjoin%
\definecolor{currentfill}{rgb}{0.501961,0.501961,0.501961}%
\pgfsetfillcolor{currentfill}%
\pgfsetfillopacity{0.700000}%
\pgfsetlinewidth{0.000000pt}%
\definecolor{currentstroke}{rgb}{0.000000,0.000000,0.000000}%
\pgfsetstrokecolor{currentstroke}%
\pgfsetstrokeopacity{0.700000}%
\pgfsetdash{}{0pt}%
\pgfpathmoveto{\pgfqpoint{4.519531in}{0.550000in}}%
\pgfpathlineto{\pgfqpoint{4.547053in}{0.550000in}}%
\pgfpathlineto{\pgfqpoint{4.547053in}{0.726449in}}%
\pgfpathlineto{\pgfqpoint{4.519531in}{0.726449in}}%
\pgfpathlineto{\pgfqpoint{4.519531in}{0.550000in}}%
\pgfpathclose%
\pgfusepath{fill}%
\end{pgfscope}%
\begin{pgfscope}%
\pgfpathrectangle{\pgfqpoint{1.250000in}{0.550000in}}{\pgfqpoint{7.750000in}{3.850000in}}%
\pgfusepath{clip}%
\pgfsetbuttcap%
\pgfsetmiterjoin%
\definecolor{currentfill}{rgb}{0.501961,0.501961,0.501961}%
\pgfsetfillcolor{currentfill}%
\pgfsetfillopacity{0.700000}%
\pgfsetlinewidth{0.000000pt}%
\definecolor{currentstroke}{rgb}{0.000000,0.000000,0.000000}%
\pgfsetstrokecolor{currentstroke}%
\pgfsetstrokeopacity{0.700000}%
\pgfsetdash{}{0pt}%
\pgfpathmoveto{\pgfqpoint{4.547053in}{0.550000in}}%
\pgfpathlineto{\pgfqpoint{4.574574in}{0.550000in}}%
\pgfpathlineto{\pgfqpoint{4.574574in}{0.727636in}}%
\pgfpathlineto{\pgfqpoint{4.547053in}{0.727636in}}%
\pgfpathlineto{\pgfqpoint{4.547053in}{0.550000in}}%
\pgfpathclose%
\pgfusepath{fill}%
\end{pgfscope}%
\begin{pgfscope}%
\pgfpathrectangle{\pgfqpoint{1.250000in}{0.550000in}}{\pgfqpoint{7.750000in}{3.850000in}}%
\pgfusepath{clip}%
\pgfsetbuttcap%
\pgfsetmiterjoin%
\definecolor{currentfill}{rgb}{0.501961,0.501961,0.501961}%
\pgfsetfillcolor{currentfill}%
\pgfsetfillopacity{0.700000}%
\pgfsetlinewidth{0.000000pt}%
\definecolor{currentstroke}{rgb}{0.000000,0.000000,0.000000}%
\pgfsetstrokecolor{currentstroke}%
\pgfsetstrokeopacity{0.700000}%
\pgfsetdash{}{0pt}%
\pgfpathmoveto{\pgfqpoint{4.574574in}{0.550000in}}%
\pgfpathlineto{\pgfqpoint{4.602095in}{0.550000in}}%
\pgfpathlineto{\pgfqpoint{4.602095in}{0.726054in}}%
\pgfpathlineto{\pgfqpoint{4.574574in}{0.726054in}}%
\pgfpathlineto{\pgfqpoint{4.574574in}{0.550000in}}%
\pgfpathclose%
\pgfusepath{fill}%
\end{pgfscope}%
\begin{pgfscope}%
\pgfpathrectangle{\pgfqpoint{1.250000in}{0.550000in}}{\pgfqpoint{7.750000in}{3.850000in}}%
\pgfusepath{clip}%
\pgfsetbuttcap%
\pgfsetmiterjoin%
\definecolor{currentfill}{rgb}{0.501961,0.501961,0.501961}%
\pgfsetfillcolor{currentfill}%
\pgfsetfillopacity{0.700000}%
\pgfsetlinewidth{0.000000pt}%
\definecolor{currentstroke}{rgb}{0.000000,0.000000,0.000000}%
\pgfsetstrokecolor{currentstroke}%
\pgfsetstrokeopacity{0.700000}%
\pgfsetdash{}{0pt}%
\pgfpathmoveto{\pgfqpoint{4.602095in}{0.550000in}}%
\pgfpathlineto{\pgfqpoint{4.629616in}{0.550000in}}%
\pgfpathlineto{\pgfqpoint{4.629616in}{0.729219in}}%
\pgfpathlineto{\pgfqpoint{4.602095in}{0.729219in}}%
\pgfpathlineto{\pgfqpoint{4.602095in}{0.550000in}}%
\pgfpathclose%
\pgfusepath{fill}%
\end{pgfscope}%
\begin{pgfscope}%
\pgfpathrectangle{\pgfqpoint{1.250000in}{0.550000in}}{\pgfqpoint{7.750000in}{3.850000in}}%
\pgfusepath{clip}%
\pgfsetbuttcap%
\pgfsetmiterjoin%
\definecolor{currentfill}{rgb}{0.501961,0.501961,0.501961}%
\pgfsetfillcolor{currentfill}%
\pgfsetfillopacity{0.700000}%
\pgfsetlinewidth{0.000000pt}%
\definecolor{currentstroke}{rgb}{0.000000,0.000000,0.000000}%
\pgfsetstrokecolor{currentstroke}%
\pgfsetstrokeopacity{0.700000}%
\pgfsetdash{}{0pt}%
\pgfpathmoveto{\pgfqpoint{4.629616in}{0.550000in}}%
\pgfpathlineto{\pgfqpoint{4.657138in}{0.550000in}}%
\pgfpathlineto{\pgfqpoint{4.657138in}{0.550000in}}%
\pgfpathlineto{\pgfqpoint{4.629616in}{0.550000in}}%
\pgfpathlineto{\pgfqpoint{4.629616in}{0.550000in}}%
\pgfpathclose%
\pgfusepath{fill}%
\end{pgfscope}%
\begin{pgfscope}%
\pgfpathrectangle{\pgfqpoint{1.250000in}{0.550000in}}{\pgfqpoint{7.750000in}{3.850000in}}%
\pgfusepath{clip}%
\pgfsetbuttcap%
\pgfsetmiterjoin%
\definecolor{currentfill}{rgb}{0.501961,0.501961,0.501961}%
\pgfsetfillcolor{currentfill}%
\pgfsetfillopacity{0.700000}%
\pgfsetlinewidth{0.000000pt}%
\definecolor{currentstroke}{rgb}{0.000000,0.000000,0.000000}%
\pgfsetstrokecolor{currentstroke}%
\pgfsetstrokeopacity{0.700000}%
\pgfsetdash{}{0pt}%
\pgfpathmoveto{\pgfqpoint{4.657138in}{0.550000in}}%
\pgfpathlineto{\pgfqpoint{4.684659in}{0.550000in}}%
\pgfpathlineto{\pgfqpoint{4.684659in}{0.726845in}}%
\pgfpathlineto{\pgfqpoint{4.657138in}{0.726845in}}%
\pgfpathlineto{\pgfqpoint{4.657138in}{0.550000in}}%
\pgfpathclose%
\pgfusepath{fill}%
\end{pgfscope}%
\begin{pgfscope}%
\pgfpathrectangle{\pgfqpoint{1.250000in}{0.550000in}}{\pgfqpoint{7.750000in}{3.850000in}}%
\pgfusepath{clip}%
\pgfsetbuttcap%
\pgfsetmiterjoin%
\definecolor{currentfill}{rgb}{0.501961,0.501961,0.501961}%
\pgfsetfillcolor{currentfill}%
\pgfsetfillopacity{0.700000}%
\pgfsetlinewidth{0.000000pt}%
\definecolor{currentstroke}{rgb}{0.000000,0.000000,0.000000}%
\pgfsetstrokecolor{currentstroke}%
\pgfsetstrokeopacity{0.700000}%
\pgfsetdash{}{0pt}%
\pgfpathmoveto{\pgfqpoint{4.684659in}{0.550000in}}%
\pgfpathlineto{\pgfqpoint{4.712180in}{0.550000in}}%
\pgfpathlineto{\pgfqpoint{4.712180in}{0.707855in}}%
\pgfpathlineto{\pgfqpoint{4.684659in}{0.707855in}}%
\pgfpathlineto{\pgfqpoint{4.684659in}{0.550000in}}%
\pgfpathclose%
\pgfusepath{fill}%
\end{pgfscope}%
\begin{pgfscope}%
\pgfpathrectangle{\pgfqpoint{1.250000in}{0.550000in}}{\pgfqpoint{7.750000in}{3.850000in}}%
\pgfusepath{clip}%
\pgfsetbuttcap%
\pgfsetmiterjoin%
\definecolor{currentfill}{rgb}{0.501961,0.501961,0.501961}%
\pgfsetfillcolor{currentfill}%
\pgfsetfillopacity{0.700000}%
\pgfsetlinewidth{0.000000pt}%
\definecolor{currentstroke}{rgb}{0.000000,0.000000,0.000000}%
\pgfsetstrokecolor{currentstroke}%
\pgfsetstrokeopacity{0.700000}%
\pgfsetdash{}{0pt}%
\pgfpathmoveto{\pgfqpoint{4.712180in}{0.550000in}}%
\pgfpathlineto{\pgfqpoint{4.739702in}{0.550000in}}%
\pgfpathlineto{\pgfqpoint{4.739702in}{0.705877in}}%
\pgfpathlineto{\pgfqpoint{4.712180in}{0.705877in}}%
\pgfpathlineto{\pgfqpoint{4.712180in}{0.550000in}}%
\pgfpathclose%
\pgfusepath{fill}%
\end{pgfscope}%
\begin{pgfscope}%
\pgfpathrectangle{\pgfqpoint{1.250000in}{0.550000in}}{\pgfqpoint{7.750000in}{3.850000in}}%
\pgfusepath{clip}%
\pgfsetbuttcap%
\pgfsetmiterjoin%
\definecolor{currentfill}{rgb}{0.501961,0.501961,0.501961}%
\pgfsetfillcolor{currentfill}%
\pgfsetfillopacity{0.700000}%
\pgfsetlinewidth{0.000000pt}%
\definecolor{currentstroke}{rgb}{0.000000,0.000000,0.000000}%
\pgfsetstrokecolor{currentstroke}%
\pgfsetstrokeopacity{0.700000}%
\pgfsetdash{}{0pt}%
\pgfpathmoveto{\pgfqpoint{4.739702in}{0.550000in}}%
\pgfpathlineto{\pgfqpoint{4.767223in}{0.550000in}}%
\pgfpathlineto{\pgfqpoint{4.767223in}{0.709833in}}%
\pgfpathlineto{\pgfqpoint{4.739702in}{0.709833in}}%
\pgfpathlineto{\pgfqpoint{4.739702in}{0.550000in}}%
\pgfpathclose%
\pgfusepath{fill}%
\end{pgfscope}%
\begin{pgfscope}%
\pgfpathrectangle{\pgfqpoint{1.250000in}{0.550000in}}{\pgfqpoint{7.750000in}{3.850000in}}%
\pgfusepath{clip}%
\pgfsetbuttcap%
\pgfsetmiterjoin%
\definecolor{currentfill}{rgb}{0.501961,0.501961,0.501961}%
\pgfsetfillcolor{currentfill}%
\pgfsetfillopacity{0.700000}%
\pgfsetlinewidth{0.000000pt}%
\definecolor{currentstroke}{rgb}{0.000000,0.000000,0.000000}%
\pgfsetstrokecolor{currentstroke}%
\pgfsetstrokeopacity{0.700000}%
\pgfsetdash{}{0pt}%
\pgfpathmoveto{\pgfqpoint{4.767223in}{0.550000in}}%
\pgfpathlineto{\pgfqpoint{4.794744in}{0.550000in}}%
\pgfpathlineto{\pgfqpoint{4.794744in}{0.550000in}}%
\pgfpathlineto{\pgfqpoint{4.767223in}{0.550000in}}%
\pgfpathlineto{\pgfqpoint{4.767223in}{0.550000in}}%
\pgfpathclose%
\pgfusepath{fill}%
\end{pgfscope}%
\begin{pgfscope}%
\pgfpathrectangle{\pgfqpoint{1.250000in}{0.550000in}}{\pgfqpoint{7.750000in}{3.850000in}}%
\pgfusepath{clip}%
\pgfsetbuttcap%
\pgfsetmiterjoin%
\definecolor{currentfill}{rgb}{0.501961,0.501961,0.501961}%
\pgfsetfillcolor{currentfill}%
\pgfsetfillopacity{0.700000}%
\pgfsetlinewidth{0.000000pt}%
\definecolor{currentstroke}{rgb}{0.000000,0.000000,0.000000}%
\pgfsetstrokecolor{currentstroke}%
\pgfsetstrokeopacity{0.700000}%
\pgfsetdash{}{0pt}%
\pgfpathmoveto{\pgfqpoint{4.794744in}{0.550000in}}%
\pgfpathlineto{\pgfqpoint{4.822266in}{0.550000in}}%
\pgfpathlineto{\pgfqpoint{4.822266in}{0.699942in}}%
\pgfpathlineto{\pgfqpoint{4.794744in}{0.699942in}}%
\pgfpathlineto{\pgfqpoint{4.794744in}{0.550000in}}%
\pgfpathclose%
\pgfusepath{fill}%
\end{pgfscope}%
\begin{pgfscope}%
\pgfpathrectangle{\pgfqpoint{1.250000in}{0.550000in}}{\pgfqpoint{7.750000in}{3.850000in}}%
\pgfusepath{clip}%
\pgfsetbuttcap%
\pgfsetmiterjoin%
\definecolor{currentfill}{rgb}{0.501961,0.501961,0.501961}%
\pgfsetfillcolor{currentfill}%
\pgfsetfillopacity{0.700000}%
\pgfsetlinewidth{0.000000pt}%
\definecolor{currentstroke}{rgb}{0.000000,0.000000,0.000000}%
\pgfsetstrokecolor{currentstroke}%
\pgfsetstrokeopacity{0.700000}%
\pgfsetdash{}{0pt}%
\pgfpathmoveto{\pgfqpoint{4.822266in}{0.550000in}}%
\pgfpathlineto{\pgfqpoint{4.849787in}{0.550000in}}%
\pgfpathlineto{\pgfqpoint{4.849787in}{0.689261in}}%
\pgfpathlineto{\pgfqpoint{4.822266in}{0.689261in}}%
\pgfpathlineto{\pgfqpoint{4.822266in}{0.550000in}}%
\pgfpathclose%
\pgfusepath{fill}%
\end{pgfscope}%
\begin{pgfscope}%
\pgfpathrectangle{\pgfqpoint{1.250000in}{0.550000in}}{\pgfqpoint{7.750000in}{3.850000in}}%
\pgfusepath{clip}%
\pgfsetbuttcap%
\pgfsetmiterjoin%
\definecolor{currentfill}{rgb}{0.501961,0.501961,0.501961}%
\pgfsetfillcolor{currentfill}%
\pgfsetfillopacity{0.700000}%
\pgfsetlinewidth{0.000000pt}%
\definecolor{currentstroke}{rgb}{0.000000,0.000000,0.000000}%
\pgfsetstrokecolor{currentstroke}%
\pgfsetstrokeopacity{0.700000}%
\pgfsetdash{}{0pt}%
\pgfpathmoveto{\pgfqpoint{4.849787in}{0.550000in}}%
\pgfpathlineto{\pgfqpoint{4.877308in}{0.550000in}}%
\pgfpathlineto{\pgfqpoint{4.877308in}{0.690843in}}%
\pgfpathlineto{\pgfqpoint{4.849787in}{0.690843in}}%
\pgfpathlineto{\pgfqpoint{4.849787in}{0.550000in}}%
\pgfpathclose%
\pgfusepath{fill}%
\end{pgfscope}%
\begin{pgfscope}%
\pgfpathrectangle{\pgfqpoint{1.250000in}{0.550000in}}{\pgfqpoint{7.750000in}{3.850000in}}%
\pgfusepath{clip}%
\pgfsetbuttcap%
\pgfsetmiterjoin%
\definecolor{currentfill}{rgb}{0.501961,0.501961,0.501961}%
\pgfsetfillcolor{currentfill}%
\pgfsetfillopacity{0.700000}%
\pgfsetlinewidth{0.000000pt}%
\definecolor{currentstroke}{rgb}{0.000000,0.000000,0.000000}%
\pgfsetstrokecolor{currentstroke}%
\pgfsetstrokeopacity{0.700000}%
\pgfsetdash{}{0pt}%
\pgfpathmoveto{\pgfqpoint{4.877308in}{0.550000in}}%
\pgfpathlineto{\pgfqpoint{4.904830in}{0.550000in}}%
\pgfpathlineto{\pgfqpoint{4.904830in}{0.550000in}}%
\pgfpathlineto{\pgfqpoint{4.877308in}{0.550000in}}%
\pgfpathlineto{\pgfqpoint{4.877308in}{0.550000in}}%
\pgfpathclose%
\pgfusepath{fill}%
\end{pgfscope}%
\begin{pgfscope}%
\pgfpathrectangle{\pgfqpoint{1.250000in}{0.550000in}}{\pgfqpoint{7.750000in}{3.850000in}}%
\pgfusepath{clip}%
\pgfsetbuttcap%
\pgfsetmiterjoin%
\definecolor{currentfill}{rgb}{0.501961,0.501961,0.501961}%
\pgfsetfillcolor{currentfill}%
\pgfsetfillopacity{0.700000}%
\pgfsetlinewidth{0.000000pt}%
\definecolor{currentstroke}{rgb}{0.000000,0.000000,0.000000}%
\pgfsetstrokecolor{currentstroke}%
\pgfsetstrokeopacity{0.700000}%
\pgfsetdash{}{0pt}%
\pgfpathmoveto{\pgfqpoint{4.904830in}{0.550000in}}%
\pgfpathlineto{\pgfqpoint{4.932351in}{0.550000in}}%
\pgfpathlineto{\pgfqpoint{4.932351in}{0.663149in}}%
\pgfpathlineto{\pgfqpoint{4.904830in}{0.663149in}}%
\pgfpathlineto{\pgfqpoint{4.904830in}{0.550000in}}%
\pgfpathclose%
\pgfusepath{fill}%
\end{pgfscope}%
\begin{pgfscope}%
\pgfpathrectangle{\pgfqpoint{1.250000in}{0.550000in}}{\pgfqpoint{7.750000in}{3.850000in}}%
\pgfusepath{clip}%
\pgfsetbuttcap%
\pgfsetmiterjoin%
\definecolor{currentfill}{rgb}{0.501961,0.501961,0.501961}%
\pgfsetfillcolor{currentfill}%
\pgfsetfillopacity{0.700000}%
\pgfsetlinewidth{0.000000pt}%
\definecolor{currentstroke}{rgb}{0.000000,0.000000,0.000000}%
\pgfsetstrokecolor{currentstroke}%
\pgfsetstrokeopacity{0.700000}%
\pgfsetdash{}{0pt}%
\pgfpathmoveto{\pgfqpoint{4.932351in}{0.550000in}}%
\pgfpathlineto{\pgfqpoint{4.959872in}{0.550000in}}%
\pgfpathlineto{\pgfqpoint{4.959872in}{0.674622in}}%
\pgfpathlineto{\pgfqpoint{4.932351in}{0.674622in}}%
\pgfpathlineto{\pgfqpoint{4.932351in}{0.550000in}}%
\pgfpathclose%
\pgfusepath{fill}%
\end{pgfscope}%
\begin{pgfscope}%
\pgfpathrectangle{\pgfqpoint{1.250000in}{0.550000in}}{\pgfqpoint{7.750000in}{3.850000in}}%
\pgfusepath{clip}%
\pgfsetbuttcap%
\pgfsetmiterjoin%
\definecolor{currentfill}{rgb}{0.501961,0.501961,0.501961}%
\pgfsetfillcolor{currentfill}%
\pgfsetfillopacity{0.700000}%
\pgfsetlinewidth{0.000000pt}%
\definecolor{currentstroke}{rgb}{0.000000,0.000000,0.000000}%
\pgfsetstrokecolor{currentstroke}%
\pgfsetstrokeopacity{0.700000}%
\pgfsetdash{}{0pt}%
\pgfpathmoveto{\pgfqpoint{4.959872in}{0.550000in}}%
\pgfpathlineto{\pgfqpoint{4.987393in}{0.550000in}}%
\pgfpathlineto{\pgfqpoint{4.987393in}{0.670270in}}%
\pgfpathlineto{\pgfqpoint{4.959872in}{0.670270in}}%
\pgfpathlineto{\pgfqpoint{4.959872in}{0.550000in}}%
\pgfpathclose%
\pgfusepath{fill}%
\end{pgfscope}%
\begin{pgfscope}%
\pgfpathrectangle{\pgfqpoint{1.250000in}{0.550000in}}{\pgfqpoint{7.750000in}{3.850000in}}%
\pgfusepath{clip}%
\pgfsetbuttcap%
\pgfsetmiterjoin%
\definecolor{currentfill}{rgb}{0.501961,0.501961,0.501961}%
\pgfsetfillcolor{currentfill}%
\pgfsetfillopacity{0.700000}%
\pgfsetlinewidth{0.000000pt}%
\definecolor{currentstroke}{rgb}{0.000000,0.000000,0.000000}%
\pgfsetstrokecolor{currentstroke}%
\pgfsetstrokeopacity{0.700000}%
\pgfsetdash{}{0pt}%
\pgfpathmoveto{\pgfqpoint{4.987393in}{0.550000in}}%
\pgfpathlineto{\pgfqpoint{5.014915in}{0.550000in}}%
\pgfpathlineto{\pgfqpoint{5.014915in}{0.688865in}}%
\pgfpathlineto{\pgfqpoint{4.987393in}{0.688865in}}%
\pgfpathlineto{\pgfqpoint{4.987393in}{0.550000in}}%
\pgfpathclose%
\pgfusepath{fill}%
\end{pgfscope}%
\begin{pgfscope}%
\pgfpathrectangle{\pgfqpoint{1.250000in}{0.550000in}}{\pgfqpoint{7.750000in}{3.850000in}}%
\pgfusepath{clip}%
\pgfsetbuttcap%
\pgfsetmiterjoin%
\definecolor{currentfill}{rgb}{0.501961,0.501961,0.501961}%
\pgfsetfillcolor{currentfill}%
\pgfsetfillopacity{0.700000}%
\pgfsetlinewidth{0.000000pt}%
\definecolor{currentstroke}{rgb}{0.000000,0.000000,0.000000}%
\pgfsetstrokecolor{currentstroke}%
\pgfsetstrokeopacity{0.700000}%
\pgfsetdash{}{0pt}%
\pgfpathmoveto{\pgfqpoint{5.014915in}{0.550000in}}%
\pgfpathlineto{\pgfqpoint{5.042436in}{0.550000in}}%
\pgfpathlineto{\pgfqpoint{5.042436in}{0.550000in}}%
\pgfpathlineto{\pgfqpoint{5.014915in}{0.550000in}}%
\pgfpathlineto{\pgfqpoint{5.014915in}{0.550000in}}%
\pgfpathclose%
\pgfusepath{fill}%
\end{pgfscope}%
\begin{pgfscope}%
\pgfpathrectangle{\pgfqpoint{1.250000in}{0.550000in}}{\pgfqpoint{7.750000in}{3.850000in}}%
\pgfusepath{clip}%
\pgfsetbuttcap%
\pgfsetmiterjoin%
\definecolor{currentfill}{rgb}{0.501961,0.501961,0.501961}%
\pgfsetfillcolor{currentfill}%
\pgfsetfillopacity{0.700000}%
\pgfsetlinewidth{0.000000pt}%
\definecolor{currentstroke}{rgb}{0.000000,0.000000,0.000000}%
\pgfsetstrokecolor{currentstroke}%
\pgfsetstrokeopacity{0.700000}%
\pgfsetdash{}{0pt}%
\pgfpathmoveto{\pgfqpoint{5.042436in}{0.550000in}}%
\pgfpathlineto{\pgfqpoint{5.069957in}{0.550000in}}%
\pgfpathlineto{\pgfqpoint{5.069957in}{0.677787in}}%
\pgfpathlineto{\pgfqpoint{5.042436in}{0.677787in}}%
\pgfpathlineto{\pgfqpoint{5.042436in}{0.550000in}}%
\pgfpathclose%
\pgfusepath{fill}%
\end{pgfscope}%
\begin{pgfscope}%
\pgfpathrectangle{\pgfqpoint{1.250000in}{0.550000in}}{\pgfqpoint{7.750000in}{3.850000in}}%
\pgfusepath{clip}%
\pgfsetbuttcap%
\pgfsetmiterjoin%
\definecolor{currentfill}{rgb}{0.501961,0.501961,0.501961}%
\pgfsetfillcolor{currentfill}%
\pgfsetfillopacity{0.700000}%
\pgfsetlinewidth{0.000000pt}%
\definecolor{currentstroke}{rgb}{0.000000,0.000000,0.000000}%
\pgfsetstrokecolor{currentstroke}%
\pgfsetstrokeopacity{0.700000}%
\pgfsetdash{}{0pt}%
\pgfpathmoveto{\pgfqpoint{5.069957in}{0.550000in}}%
\pgfpathlineto{\pgfqpoint{5.097479in}{0.550000in}}%
\pgfpathlineto{\pgfqpoint{5.097479in}{0.670270in}}%
\pgfpathlineto{\pgfqpoint{5.069957in}{0.670270in}}%
\pgfpathlineto{\pgfqpoint{5.069957in}{0.550000in}}%
\pgfpathclose%
\pgfusepath{fill}%
\end{pgfscope}%
\begin{pgfscope}%
\pgfpathrectangle{\pgfqpoint{1.250000in}{0.550000in}}{\pgfqpoint{7.750000in}{3.850000in}}%
\pgfusepath{clip}%
\pgfsetbuttcap%
\pgfsetmiterjoin%
\definecolor{currentfill}{rgb}{0.501961,0.501961,0.501961}%
\pgfsetfillcolor{currentfill}%
\pgfsetfillopacity{0.700000}%
\pgfsetlinewidth{0.000000pt}%
\definecolor{currentstroke}{rgb}{0.000000,0.000000,0.000000}%
\pgfsetstrokecolor{currentstroke}%
\pgfsetstrokeopacity{0.700000}%
\pgfsetdash{}{0pt}%
\pgfpathmoveto{\pgfqpoint{5.097479in}{0.550000in}}%
\pgfpathlineto{\pgfqpoint{5.125000in}{0.550000in}}%
\pgfpathlineto{\pgfqpoint{5.125000in}{0.658402in}}%
\pgfpathlineto{\pgfqpoint{5.097479in}{0.658402in}}%
\pgfpathlineto{\pgfqpoint{5.097479in}{0.550000in}}%
\pgfpathclose%
\pgfusepath{fill}%
\end{pgfscope}%
\begin{pgfscope}%
\pgfpathrectangle{\pgfqpoint{1.250000in}{0.550000in}}{\pgfqpoint{7.750000in}{3.850000in}}%
\pgfusepath{clip}%
\pgfsetbuttcap%
\pgfsetmiterjoin%
\definecolor{currentfill}{rgb}{0.501961,0.501961,0.501961}%
\pgfsetfillcolor{currentfill}%
\pgfsetfillopacity{0.700000}%
\pgfsetlinewidth{0.000000pt}%
\definecolor{currentstroke}{rgb}{0.000000,0.000000,0.000000}%
\pgfsetstrokecolor{currentstroke}%
\pgfsetstrokeopacity{0.700000}%
\pgfsetdash{}{0pt}%
\pgfpathmoveto{\pgfqpoint{5.125000in}{0.550000in}}%
\pgfpathlineto{\pgfqpoint{5.152521in}{0.550000in}}%
\pgfpathlineto{\pgfqpoint{5.152521in}{0.647324in}}%
\pgfpathlineto{\pgfqpoint{5.125000in}{0.647324in}}%
\pgfpathlineto{\pgfqpoint{5.125000in}{0.550000in}}%
\pgfpathclose%
\pgfusepath{fill}%
\end{pgfscope}%
\begin{pgfscope}%
\pgfpathrectangle{\pgfqpoint{1.250000in}{0.550000in}}{\pgfqpoint{7.750000in}{3.850000in}}%
\pgfusepath{clip}%
\pgfsetbuttcap%
\pgfsetmiterjoin%
\definecolor{currentfill}{rgb}{0.501961,0.501961,0.501961}%
\pgfsetfillcolor{currentfill}%
\pgfsetfillopacity{0.700000}%
\pgfsetlinewidth{0.000000pt}%
\definecolor{currentstroke}{rgb}{0.000000,0.000000,0.000000}%
\pgfsetstrokecolor{currentstroke}%
\pgfsetstrokeopacity{0.700000}%
\pgfsetdash{}{0pt}%
\pgfpathmoveto{\pgfqpoint{5.152521in}{0.550000in}}%
\pgfpathlineto{\pgfqpoint{5.180043in}{0.550000in}}%
\pgfpathlineto{\pgfqpoint{5.180043in}{0.550000in}}%
\pgfpathlineto{\pgfqpoint{5.152521in}{0.550000in}}%
\pgfpathlineto{\pgfqpoint{5.152521in}{0.550000in}}%
\pgfpathclose%
\pgfusepath{fill}%
\end{pgfscope}%
\begin{pgfscope}%
\pgfpathrectangle{\pgfqpoint{1.250000in}{0.550000in}}{\pgfqpoint{7.750000in}{3.850000in}}%
\pgfusepath{clip}%
\pgfsetbuttcap%
\pgfsetmiterjoin%
\definecolor{currentfill}{rgb}{0.501961,0.501961,0.501961}%
\pgfsetfillcolor{currentfill}%
\pgfsetfillopacity{0.700000}%
\pgfsetlinewidth{0.000000pt}%
\definecolor{currentstroke}{rgb}{0.000000,0.000000,0.000000}%
\pgfsetstrokecolor{currentstroke}%
\pgfsetstrokeopacity{0.700000}%
\pgfsetdash{}{0pt}%
\pgfpathmoveto{\pgfqpoint{5.180043in}{0.550000in}}%
\pgfpathlineto{\pgfqpoint{5.207564in}{0.550000in}}%
\pgfpathlineto{\pgfqpoint{5.207564in}{0.658006in}}%
\pgfpathlineto{\pgfqpoint{5.180043in}{0.658006in}}%
\pgfpathlineto{\pgfqpoint{5.180043in}{0.550000in}}%
\pgfpathclose%
\pgfusepath{fill}%
\end{pgfscope}%
\begin{pgfscope}%
\pgfpathrectangle{\pgfqpoint{1.250000in}{0.550000in}}{\pgfqpoint{7.750000in}{3.850000in}}%
\pgfusepath{clip}%
\pgfsetbuttcap%
\pgfsetmiterjoin%
\definecolor{currentfill}{rgb}{0.501961,0.501961,0.501961}%
\pgfsetfillcolor{currentfill}%
\pgfsetfillopacity{0.700000}%
\pgfsetlinewidth{0.000000pt}%
\definecolor{currentstroke}{rgb}{0.000000,0.000000,0.000000}%
\pgfsetstrokecolor{currentstroke}%
\pgfsetstrokeopacity{0.700000}%
\pgfsetdash{}{0pt}%
\pgfpathmoveto{\pgfqpoint{5.207564in}{0.550000in}}%
\pgfpathlineto{\pgfqpoint{5.235085in}{0.550000in}}%
\pgfpathlineto{\pgfqpoint{5.235085in}{0.681744in}}%
\pgfpathlineto{\pgfqpoint{5.207564in}{0.681744in}}%
\pgfpathlineto{\pgfqpoint{5.207564in}{0.550000in}}%
\pgfpathclose%
\pgfusepath{fill}%
\end{pgfscope}%
\begin{pgfscope}%
\pgfpathrectangle{\pgfqpoint{1.250000in}{0.550000in}}{\pgfqpoint{7.750000in}{3.850000in}}%
\pgfusepath{clip}%
\pgfsetbuttcap%
\pgfsetmiterjoin%
\definecolor{currentfill}{rgb}{0.501961,0.501961,0.501961}%
\pgfsetfillcolor{currentfill}%
\pgfsetfillopacity{0.700000}%
\pgfsetlinewidth{0.000000pt}%
\definecolor{currentstroke}{rgb}{0.000000,0.000000,0.000000}%
\pgfsetstrokecolor{currentstroke}%
\pgfsetstrokeopacity{0.700000}%
\pgfsetdash{}{0pt}%
\pgfpathmoveto{\pgfqpoint{5.235085in}{0.550000in}}%
\pgfpathlineto{\pgfqpoint{5.262607in}{0.550000in}}%
\pgfpathlineto{\pgfqpoint{5.262607in}{0.665523in}}%
\pgfpathlineto{\pgfqpoint{5.235085in}{0.665523in}}%
\pgfpathlineto{\pgfqpoint{5.235085in}{0.550000in}}%
\pgfpathclose%
\pgfusepath{fill}%
\end{pgfscope}%
\begin{pgfscope}%
\pgfpathrectangle{\pgfqpoint{1.250000in}{0.550000in}}{\pgfqpoint{7.750000in}{3.850000in}}%
\pgfusepath{clip}%
\pgfsetbuttcap%
\pgfsetmiterjoin%
\definecolor{currentfill}{rgb}{0.501961,0.501961,0.501961}%
\pgfsetfillcolor{currentfill}%
\pgfsetfillopacity{0.700000}%
\pgfsetlinewidth{0.000000pt}%
\definecolor{currentstroke}{rgb}{0.000000,0.000000,0.000000}%
\pgfsetstrokecolor{currentstroke}%
\pgfsetstrokeopacity{0.700000}%
\pgfsetdash{}{0pt}%
\pgfpathmoveto{\pgfqpoint{5.262607in}{0.550000in}}%
\pgfpathlineto{\pgfqpoint{5.290128in}{0.550000in}}%
\pgfpathlineto{\pgfqpoint{5.290128in}{0.654050in}}%
\pgfpathlineto{\pgfqpoint{5.262607in}{0.654050in}}%
\pgfpathlineto{\pgfqpoint{5.262607in}{0.550000in}}%
\pgfpathclose%
\pgfusepath{fill}%
\end{pgfscope}%
\begin{pgfscope}%
\pgfpathrectangle{\pgfqpoint{1.250000in}{0.550000in}}{\pgfqpoint{7.750000in}{3.850000in}}%
\pgfusepath{clip}%
\pgfsetbuttcap%
\pgfsetmiterjoin%
\definecolor{currentfill}{rgb}{0.501961,0.501961,0.501961}%
\pgfsetfillcolor{currentfill}%
\pgfsetfillopacity{0.700000}%
\pgfsetlinewidth{0.000000pt}%
\definecolor{currentstroke}{rgb}{0.000000,0.000000,0.000000}%
\pgfsetstrokecolor{currentstroke}%
\pgfsetstrokeopacity{0.700000}%
\pgfsetdash{}{0pt}%
\pgfpathmoveto{\pgfqpoint{5.290128in}{0.550000in}}%
\pgfpathlineto{\pgfqpoint{5.317649in}{0.550000in}}%
\pgfpathlineto{\pgfqpoint{5.317649in}{0.550000in}}%
\pgfpathlineto{\pgfqpoint{5.290128in}{0.550000in}}%
\pgfpathlineto{\pgfqpoint{5.290128in}{0.550000in}}%
\pgfpathclose%
\pgfusepath{fill}%
\end{pgfscope}%
\begin{pgfscope}%
\pgfpathrectangle{\pgfqpoint{1.250000in}{0.550000in}}{\pgfqpoint{7.750000in}{3.850000in}}%
\pgfusepath{clip}%
\pgfsetbuttcap%
\pgfsetmiterjoin%
\definecolor{currentfill}{rgb}{0.501961,0.501961,0.501961}%
\pgfsetfillcolor{currentfill}%
\pgfsetfillopacity{0.700000}%
\pgfsetlinewidth{0.000000pt}%
\definecolor{currentstroke}{rgb}{0.000000,0.000000,0.000000}%
\pgfsetstrokecolor{currentstroke}%
\pgfsetstrokeopacity{0.700000}%
\pgfsetdash{}{0pt}%
\pgfpathmoveto{\pgfqpoint{5.317649in}{0.550000in}}%
\pgfpathlineto{\pgfqpoint{5.345170in}{0.550000in}}%
\pgfpathlineto{\pgfqpoint{5.345170in}{0.648511in}}%
\pgfpathlineto{\pgfqpoint{5.317649in}{0.648511in}}%
\pgfpathlineto{\pgfqpoint{5.317649in}{0.550000in}}%
\pgfpathclose%
\pgfusepath{fill}%
\end{pgfscope}%
\begin{pgfscope}%
\pgfpathrectangle{\pgfqpoint{1.250000in}{0.550000in}}{\pgfqpoint{7.750000in}{3.850000in}}%
\pgfusepath{clip}%
\pgfsetbuttcap%
\pgfsetmiterjoin%
\definecolor{currentfill}{rgb}{0.501961,0.501961,0.501961}%
\pgfsetfillcolor{currentfill}%
\pgfsetfillopacity{0.700000}%
\pgfsetlinewidth{0.000000pt}%
\definecolor{currentstroke}{rgb}{0.000000,0.000000,0.000000}%
\pgfsetstrokecolor{currentstroke}%
\pgfsetstrokeopacity{0.700000}%
\pgfsetdash{}{0pt}%
\pgfpathmoveto{\pgfqpoint{5.345170in}{0.550000in}}%
\pgfpathlineto{\pgfqpoint{5.372692in}{0.550000in}}%
\pgfpathlineto{\pgfqpoint{5.372692in}{0.645346in}}%
\pgfpathlineto{\pgfqpoint{5.345170in}{0.645346in}}%
\pgfpathlineto{\pgfqpoint{5.345170in}{0.550000in}}%
\pgfpathclose%
\pgfusepath{fill}%
\end{pgfscope}%
\begin{pgfscope}%
\pgfpathrectangle{\pgfqpoint{1.250000in}{0.550000in}}{\pgfqpoint{7.750000in}{3.850000in}}%
\pgfusepath{clip}%
\pgfsetbuttcap%
\pgfsetmiterjoin%
\definecolor{currentfill}{rgb}{0.501961,0.501961,0.501961}%
\pgfsetfillcolor{currentfill}%
\pgfsetfillopacity{0.700000}%
\pgfsetlinewidth{0.000000pt}%
\definecolor{currentstroke}{rgb}{0.000000,0.000000,0.000000}%
\pgfsetstrokecolor{currentstroke}%
\pgfsetstrokeopacity{0.700000}%
\pgfsetdash{}{0pt}%
\pgfpathmoveto{\pgfqpoint{5.372692in}{0.550000in}}%
\pgfpathlineto{\pgfqpoint{5.400213in}{0.550000in}}%
\pgfpathlineto{\pgfqpoint{5.400213in}{0.639807in}}%
\pgfpathlineto{\pgfqpoint{5.372692in}{0.639807in}}%
\pgfpathlineto{\pgfqpoint{5.372692in}{0.550000in}}%
\pgfpathclose%
\pgfusepath{fill}%
\end{pgfscope}%
\begin{pgfscope}%
\pgfpathrectangle{\pgfqpoint{1.250000in}{0.550000in}}{\pgfqpoint{7.750000in}{3.850000in}}%
\pgfusepath{clip}%
\pgfsetbuttcap%
\pgfsetmiterjoin%
\definecolor{currentfill}{rgb}{0.501961,0.501961,0.501961}%
\pgfsetfillcolor{currentfill}%
\pgfsetfillopacity{0.700000}%
\pgfsetlinewidth{0.000000pt}%
\definecolor{currentstroke}{rgb}{0.000000,0.000000,0.000000}%
\pgfsetstrokecolor{currentstroke}%
\pgfsetstrokeopacity{0.700000}%
\pgfsetdash{}{0pt}%
\pgfpathmoveto{\pgfqpoint{5.400213in}{0.550000in}}%
\pgfpathlineto{\pgfqpoint{5.427734in}{0.550000in}}%
\pgfpathlineto{\pgfqpoint{5.427734in}{0.639016in}}%
\pgfpathlineto{\pgfqpoint{5.400213in}{0.639016in}}%
\pgfpathlineto{\pgfqpoint{5.400213in}{0.550000in}}%
\pgfpathclose%
\pgfusepath{fill}%
\end{pgfscope}%
\begin{pgfscope}%
\pgfpathrectangle{\pgfqpoint{1.250000in}{0.550000in}}{\pgfqpoint{7.750000in}{3.850000in}}%
\pgfusepath{clip}%
\pgfsetbuttcap%
\pgfsetmiterjoin%
\definecolor{currentfill}{rgb}{0.501961,0.501961,0.501961}%
\pgfsetfillcolor{currentfill}%
\pgfsetfillopacity{0.700000}%
\pgfsetlinewidth{0.000000pt}%
\definecolor{currentstroke}{rgb}{0.000000,0.000000,0.000000}%
\pgfsetstrokecolor{currentstroke}%
\pgfsetstrokeopacity{0.700000}%
\pgfsetdash{}{0pt}%
\pgfpathmoveto{\pgfqpoint{5.427734in}{0.550000in}}%
\pgfpathlineto{\pgfqpoint{5.455256in}{0.550000in}}%
\pgfpathlineto{\pgfqpoint{5.455256in}{0.550000in}}%
\pgfpathlineto{\pgfqpoint{5.427734in}{0.550000in}}%
\pgfpathlineto{\pgfqpoint{5.427734in}{0.550000in}}%
\pgfpathclose%
\pgfusepath{fill}%
\end{pgfscope}%
\begin{pgfscope}%
\pgfpathrectangle{\pgfqpoint{1.250000in}{0.550000in}}{\pgfqpoint{7.750000in}{3.850000in}}%
\pgfusepath{clip}%
\pgfsetbuttcap%
\pgfsetmiterjoin%
\definecolor{currentfill}{rgb}{0.501961,0.501961,0.501961}%
\pgfsetfillcolor{currentfill}%
\pgfsetfillopacity{0.700000}%
\pgfsetlinewidth{0.000000pt}%
\definecolor{currentstroke}{rgb}{0.000000,0.000000,0.000000}%
\pgfsetstrokecolor{currentstroke}%
\pgfsetstrokeopacity{0.700000}%
\pgfsetdash{}{0pt}%
\pgfpathmoveto{\pgfqpoint{5.455256in}{0.550000in}}%
\pgfpathlineto{\pgfqpoint{5.482777in}{0.550000in}}%
\pgfpathlineto{\pgfqpoint{5.482777in}{0.640598in}}%
\pgfpathlineto{\pgfqpoint{5.455256in}{0.640598in}}%
\pgfpathlineto{\pgfqpoint{5.455256in}{0.550000in}}%
\pgfpathclose%
\pgfusepath{fill}%
\end{pgfscope}%
\begin{pgfscope}%
\pgfpathrectangle{\pgfqpoint{1.250000in}{0.550000in}}{\pgfqpoint{7.750000in}{3.850000in}}%
\pgfusepath{clip}%
\pgfsetbuttcap%
\pgfsetmiterjoin%
\definecolor{currentfill}{rgb}{0.501961,0.501961,0.501961}%
\pgfsetfillcolor{currentfill}%
\pgfsetfillopacity{0.700000}%
\pgfsetlinewidth{0.000000pt}%
\definecolor{currentstroke}{rgb}{0.000000,0.000000,0.000000}%
\pgfsetstrokecolor{currentstroke}%
\pgfsetstrokeopacity{0.700000}%
\pgfsetdash{}{0pt}%
\pgfpathmoveto{\pgfqpoint{5.482777in}{0.550000in}}%
\pgfpathlineto{\pgfqpoint{5.510298in}{0.550000in}}%
\pgfpathlineto{\pgfqpoint{5.510298in}{0.631499in}}%
\pgfpathlineto{\pgfqpoint{5.482777in}{0.631499in}}%
\pgfpathlineto{\pgfqpoint{5.482777in}{0.550000in}}%
\pgfpathclose%
\pgfusepath{fill}%
\end{pgfscope}%
\begin{pgfscope}%
\pgfpathrectangle{\pgfqpoint{1.250000in}{0.550000in}}{\pgfqpoint{7.750000in}{3.850000in}}%
\pgfusepath{clip}%
\pgfsetbuttcap%
\pgfsetmiterjoin%
\definecolor{currentfill}{rgb}{0.501961,0.501961,0.501961}%
\pgfsetfillcolor{currentfill}%
\pgfsetfillopacity{0.700000}%
\pgfsetlinewidth{0.000000pt}%
\definecolor{currentstroke}{rgb}{0.000000,0.000000,0.000000}%
\pgfsetstrokecolor{currentstroke}%
\pgfsetstrokeopacity{0.700000}%
\pgfsetdash{}{0pt}%
\pgfpathmoveto{\pgfqpoint{5.510298in}{0.550000in}}%
\pgfpathlineto{\pgfqpoint{5.537820in}{0.550000in}}%
\pgfpathlineto{\pgfqpoint{5.537820in}{0.647720in}}%
\pgfpathlineto{\pgfqpoint{5.510298in}{0.647720in}}%
\pgfpathlineto{\pgfqpoint{5.510298in}{0.550000in}}%
\pgfpathclose%
\pgfusepath{fill}%
\end{pgfscope}%
\begin{pgfscope}%
\pgfpathrectangle{\pgfqpoint{1.250000in}{0.550000in}}{\pgfqpoint{7.750000in}{3.850000in}}%
\pgfusepath{clip}%
\pgfsetbuttcap%
\pgfsetmiterjoin%
\definecolor{currentfill}{rgb}{0.501961,0.501961,0.501961}%
\pgfsetfillcolor{currentfill}%
\pgfsetfillopacity{0.700000}%
\pgfsetlinewidth{0.000000pt}%
\definecolor{currentstroke}{rgb}{0.000000,0.000000,0.000000}%
\pgfsetstrokecolor{currentstroke}%
\pgfsetstrokeopacity{0.700000}%
\pgfsetdash{}{0pt}%
\pgfpathmoveto{\pgfqpoint{5.537820in}{0.550000in}}%
\pgfpathlineto{\pgfqpoint{5.565341in}{0.550000in}}%
\pgfpathlineto{\pgfqpoint{5.565341in}{0.635851in}}%
\pgfpathlineto{\pgfqpoint{5.537820in}{0.635851in}}%
\pgfpathlineto{\pgfqpoint{5.537820in}{0.550000in}}%
\pgfpathclose%
\pgfusepath{fill}%
\end{pgfscope}%
\begin{pgfscope}%
\pgfpathrectangle{\pgfqpoint{1.250000in}{0.550000in}}{\pgfqpoint{7.750000in}{3.850000in}}%
\pgfusepath{clip}%
\pgfsetbuttcap%
\pgfsetmiterjoin%
\definecolor{currentfill}{rgb}{0.501961,0.501961,0.501961}%
\pgfsetfillcolor{currentfill}%
\pgfsetfillopacity{0.700000}%
\pgfsetlinewidth{0.000000pt}%
\definecolor{currentstroke}{rgb}{0.000000,0.000000,0.000000}%
\pgfsetstrokecolor{currentstroke}%
\pgfsetstrokeopacity{0.700000}%
\pgfsetdash{}{0pt}%
\pgfpathmoveto{\pgfqpoint{5.565341in}{0.550000in}}%
\pgfpathlineto{\pgfqpoint{5.592862in}{0.550000in}}%
\pgfpathlineto{\pgfqpoint{5.592862in}{0.550000in}}%
\pgfpathlineto{\pgfqpoint{5.565341in}{0.550000in}}%
\pgfpathlineto{\pgfqpoint{5.565341in}{0.550000in}}%
\pgfpathclose%
\pgfusepath{fill}%
\end{pgfscope}%
\begin{pgfscope}%
\pgfpathrectangle{\pgfqpoint{1.250000in}{0.550000in}}{\pgfqpoint{7.750000in}{3.850000in}}%
\pgfusepath{clip}%
\pgfsetbuttcap%
\pgfsetmiterjoin%
\definecolor{currentfill}{rgb}{0.501961,0.501961,0.501961}%
\pgfsetfillcolor{currentfill}%
\pgfsetfillopacity{0.700000}%
\pgfsetlinewidth{0.000000pt}%
\definecolor{currentstroke}{rgb}{0.000000,0.000000,0.000000}%
\pgfsetstrokecolor{currentstroke}%
\pgfsetstrokeopacity{0.700000}%
\pgfsetdash{}{0pt}%
\pgfpathmoveto{\pgfqpoint{5.592862in}{0.550000in}}%
\pgfpathlineto{\pgfqpoint{5.620384in}{0.550000in}}%
\pgfpathlineto{\pgfqpoint{5.620384in}{0.633082in}}%
\pgfpathlineto{\pgfqpoint{5.592862in}{0.633082in}}%
\pgfpathlineto{\pgfqpoint{5.592862in}{0.550000in}}%
\pgfpathclose%
\pgfusepath{fill}%
\end{pgfscope}%
\begin{pgfscope}%
\pgfpathrectangle{\pgfqpoint{1.250000in}{0.550000in}}{\pgfqpoint{7.750000in}{3.850000in}}%
\pgfusepath{clip}%
\pgfsetbuttcap%
\pgfsetmiterjoin%
\definecolor{currentfill}{rgb}{0.501961,0.501961,0.501961}%
\pgfsetfillcolor{currentfill}%
\pgfsetfillopacity{0.700000}%
\pgfsetlinewidth{0.000000pt}%
\definecolor{currentstroke}{rgb}{0.000000,0.000000,0.000000}%
\pgfsetstrokecolor{currentstroke}%
\pgfsetstrokeopacity{0.700000}%
\pgfsetdash{}{0pt}%
\pgfpathmoveto{\pgfqpoint{5.620384in}{0.550000in}}%
\pgfpathlineto{\pgfqpoint{5.647905in}{0.550000in}}%
\pgfpathlineto{\pgfqpoint{5.647905in}{0.631103in}}%
\pgfpathlineto{\pgfqpoint{5.620384in}{0.631103in}}%
\pgfpathlineto{\pgfqpoint{5.620384in}{0.550000in}}%
\pgfpathclose%
\pgfusepath{fill}%
\end{pgfscope}%
\begin{pgfscope}%
\pgfpathrectangle{\pgfqpoint{1.250000in}{0.550000in}}{\pgfqpoint{7.750000in}{3.850000in}}%
\pgfusepath{clip}%
\pgfsetbuttcap%
\pgfsetmiterjoin%
\definecolor{currentfill}{rgb}{0.501961,0.501961,0.501961}%
\pgfsetfillcolor{currentfill}%
\pgfsetfillopacity{0.700000}%
\pgfsetlinewidth{0.000000pt}%
\definecolor{currentstroke}{rgb}{0.000000,0.000000,0.000000}%
\pgfsetstrokecolor{currentstroke}%
\pgfsetstrokeopacity{0.700000}%
\pgfsetdash{}{0pt}%
\pgfpathmoveto{\pgfqpoint{5.647905in}{0.550000in}}%
\pgfpathlineto{\pgfqpoint{5.675426in}{0.550000in}}%
\pgfpathlineto{\pgfqpoint{5.675426in}{0.626752in}}%
\pgfpathlineto{\pgfqpoint{5.647905in}{0.626752in}}%
\pgfpathlineto{\pgfqpoint{5.647905in}{0.550000in}}%
\pgfpathclose%
\pgfusepath{fill}%
\end{pgfscope}%
\begin{pgfscope}%
\pgfpathrectangle{\pgfqpoint{1.250000in}{0.550000in}}{\pgfqpoint{7.750000in}{3.850000in}}%
\pgfusepath{clip}%
\pgfsetbuttcap%
\pgfsetmiterjoin%
\definecolor{currentfill}{rgb}{0.501961,0.501961,0.501961}%
\pgfsetfillcolor{currentfill}%
\pgfsetfillopacity{0.700000}%
\pgfsetlinewidth{0.000000pt}%
\definecolor{currentstroke}{rgb}{0.000000,0.000000,0.000000}%
\pgfsetstrokecolor{currentstroke}%
\pgfsetstrokeopacity{0.700000}%
\pgfsetdash{}{0pt}%
\pgfpathmoveto{\pgfqpoint{5.675426in}{0.550000in}}%
\pgfpathlineto{\pgfqpoint{5.702947in}{0.550000in}}%
\pgfpathlineto{\pgfqpoint{5.702947in}{0.621608in}}%
\pgfpathlineto{\pgfqpoint{5.675426in}{0.621608in}}%
\pgfpathlineto{\pgfqpoint{5.675426in}{0.550000in}}%
\pgfpathclose%
\pgfusepath{fill}%
\end{pgfscope}%
\begin{pgfscope}%
\pgfpathrectangle{\pgfqpoint{1.250000in}{0.550000in}}{\pgfqpoint{7.750000in}{3.850000in}}%
\pgfusepath{clip}%
\pgfsetbuttcap%
\pgfsetmiterjoin%
\definecolor{currentfill}{rgb}{0.501961,0.501961,0.501961}%
\pgfsetfillcolor{currentfill}%
\pgfsetfillopacity{0.700000}%
\pgfsetlinewidth{0.000000pt}%
\definecolor{currentstroke}{rgb}{0.000000,0.000000,0.000000}%
\pgfsetstrokecolor{currentstroke}%
\pgfsetstrokeopacity{0.700000}%
\pgfsetdash{}{0pt}%
\pgfpathmoveto{\pgfqpoint{5.702947in}{0.550000in}}%
\pgfpathlineto{\pgfqpoint{5.730469in}{0.550000in}}%
\pgfpathlineto{\pgfqpoint{5.730469in}{0.550000in}}%
\pgfpathlineto{\pgfqpoint{5.702947in}{0.550000in}}%
\pgfpathlineto{\pgfqpoint{5.702947in}{0.550000in}}%
\pgfpathclose%
\pgfusepath{fill}%
\end{pgfscope}%
\begin{pgfscope}%
\pgfpathrectangle{\pgfqpoint{1.250000in}{0.550000in}}{\pgfqpoint{7.750000in}{3.850000in}}%
\pgfusepath{clip}%
\pgfsetbuttcap%
\pgfsetmiterjoin%
\definecolor{currentfill}{rgb}{0.501961,0.501961,0.501961}%
\pgfsetfillcolor{currentfill}%
\pgfsetfillopacity{0.700000}%
\pgfsetlinewidth{0.000000pt}%
\definecolor{currentstroke}{rgb}{0.000000,0.000000,0.000000}%
\pgfsetstrokecolor{currentstroke}%
\pgfsetstrokeopacity{0.700000}%
\pgfsetdash{}{0pt}%
\pgfpathmoveto{\pgfqpoint{5.730469in}{0.550000in}}%
\pgfpathlineto{\pgfqpoint{5.757990in}{0.550000in}}%
\pgfpathlineto{\pgfqpoint{5.757990in}{0.627543in}}%
\pgfpathlineto{\pgfqpoint{5.730469in}{0.627543in}}%
\pgfpathlineto{\pgfqpoint{5.730469in}{0.550000in}}%
\pgfpathclose%
\pgfusepath{fill}%
\end{pgfscope}%
\begin{pgfscope}%
\pgfpathrectangle{\pgfqpoint{1.250000in}{0.550000in}}{\pgfqpoint{7.750000in}{3.850000in}}%
\pgfusepath{clip}%
\pgfsetbuttcap%
\pgfsetmiterjoin%
\definecolor{currentfill}{rgb}{0.501961,0.501961,0.501961}%
\pgfsetfillcolor{currentfill}%
\pgfsetfillopacity{0.700000}%
\pgfsetlinewidth{0.000000pt}%
\definecolor{currentstroke}{rgb}{0.000000,0.000000,0.000000}%
\pgfsetstrokecolor{currentstroke}%
\pgfsetstrokeopacity{0.700000}%
\pgfsetdash{}{0pt}%
\pgfpathmoveto{\pgfqpoint{5.757990in}{0.550000in}}%
\pgfpathlineto{\pgfqpoint{5.785511in}{0.550000in}}%
\pgfpathlineto{\pgfqpoint{5.785511in}{0.619235in}}%
\pgfpathlineto{\pgfqpoint{5.757990in}{0.619235in}}%
\pgfpathlineto{\pgfqpoint{5.757990in}{0.550000in}}%
\pgfpathclose%
\pgfusepath{fill}%
\end{pgfscope}%
\begin{pgfscope}%
\pgfpathrectangle{\pgfqpoint{1.250000in}{0.550000in}}{\pgfqpoint{7.750000in}{3.850000in}}%
\pgfusepath{clip}%
\pgfsetbuttcap%
\pgfsetmiterjoin%
\definecolor{currentfill}{rgb}{0.501961,0.501961,0.501961}%
\pgfsetfillcolor{currentfill}%
\pgfsetfillopacity{0.700000}%
\pgfsetlinewidth{0.000000pt}%
\definecolor{currentstroke}{rgb}{0.000000,0.000000,0.000000}%
\pgfsetstrokecolor{currentstroke}%
\pgfsetstrokeopacity{0.700000}%
\pgfsetdash{}{0pt}%
\pgfpathmoveto{\pgfqpoint{5.785511in}{0.550000in}}%
\pgfpathlineto{\pgfqpoint{5.813033in}{0.550000in}}%
\pgfpathlineto{\pgfqpoint{5.813033in}{0.620422in}}%
\pgfpathlineto{\pgfqpoint{5.785511in}{0.620422in}}%
\pgfpathlineto{\pgfqpoint{5.785511in}{0.550000in}}%
\pgfpathclose%
\pgfusepath{fill}%
\end{pgfscope}%
\begin{pgfscope}%
\pgfpathrectangle{\pgfqpoint{1.250000in}{0.550000in}}{\pgfqpoint{7.750000in}{3.850000in}}%
\pgfusepath{clip}%
\pgfsetbuttcap%
\pgfsetmiterjoin%
\definecolor{currentfill}{rgb}{0.501961,0.501961,0.501961}%
\pgfsetfillcolor{currentfill}%
\pgfsetfillopacity{0.700000}%
\pgfsetlinewidth{0.000000pt}%
\definecolor{currentstroke}{rgb}{0.000000,0.000000,0.000000}%
\pgfsetstrokecolor{currentstroke}%
\pgfsetstrokeopacity{0.700000}%
\pgfsetdash{}{0pt}%
\pgfpathmoveto{\pgfqpoint{5.813033in}{0.550000in}}%
\pgfpathlineto{\pgfqpoint{5.840554in}{0.550000in}}%
\pgfpathlineto{\pgfqpoint{5.840554in}{0.550000in}}%
\pgfpathlineto{\pgfqpoint{5.813033in}{0.550000in}}%
\pgfpathlineto{\pgfqpoint{5.813033in}{0.550000in}}%
\pgfpathclose%
\pgfusepath{fill}%
\end{pgfscope}%
\begin{pgfscope}%
\pgfpathrectangle{\pgfqpoint{1.250000in}{0.550000in}}{\pgfqpoint{7.750000in}{3.850000in}}%
\pgfusepath{clip}%
\pgfsetbuttcap%
\pgfsetmiterjoin%
\definecolor{currentfill}{rgb}{0.501961,0.501961,0.501961}%
\pgfsetfillcolor{currentfill}%
\pgfsetfillopacity{0.700000}%
\pgfsetlinewidth{0.000000pt}%
\definecolor{currentstroke}{rgb}{0.000000,0.000000,0.000000}%
\pgfsetstrokecolor{currentstroke}%
\pgfsetstrokeopacity{0.700000}%
\pgfsetdash{}{0pt}%
\pgfpathmoveto{\pgfqpoint{5.840554in}{0.550000in}}%
\pgfpathlineto{\pgfqpoint{5.868075in}{0.550000in}}%
\pgfpathlineto{\pgfqpoint{5.868075in}{0.610135in}}%
\pgfpathlineto{\pgfqpoint{5.840554in}{0.610135in}}%
\pgfpathlineto{\pgfqpoint{5.840554in}{0.550000in}}%
\pgfpathclose%
\pgfusepath{fill}%
\end{pgfscope}%
\begin{pgfscope}%
\pgfpathrectangle{\pgfqpoint{1.250000in}{0.550000in}}{\pgfqpoint{7.750000in}{3.850000in}}%
\pgfusepath{clip}%
\pgfsetbuttcap%
\pgfsetmiterjoin%
\definecolor{currentfill}{rgb}{0.501961,0.501961,0.501961}%
\pgfsetfillcolor{currentfill}%
\pgfsetfillopacity{0.700000}%
\pgfsetlinewidth{0.000000pt}%
\definecolor{currentstroke}{rgb}{0.000000,0.000000,0.000000}%
\pgfsetstrokecolor{currentstroke}%
\pgfsetstrokeopacity{0.700000}%
\pgfsetdash{}{0pt}%
\pgfpathmoveto{\pgfqpoint{5.868075in}{0.550000in}}%
\pgfpathlineto{\pgfqpoint{5.895597in}{0.550000in}}%
\pgfpathlineto{\pgfqpoint{5.895597in}{0.635060in}}%
\pgfpathlineto{\pgfqpoint{5.868075in}{0.635060in}}%
\pgfpathlineto{\pgfqpoint{5.868075in}{0.550000in}}%
\pgfpathclose%
\pgfusepath{fill}%
\end{pgfscope}%
\begin{pgfscope}%
\pgfpathrectangle{\pgfqpoint{1.250000in}{0.550000in}}{\pgfqpoint{7.750000in}{3.850000in}}%
\pgfusepath{clip}%
\pgfsetbuttcap%
\pgfsetmiterjoin%
\definecolor{currentfill}{rgb}{0.501961,0.501961,0.501961}%
\pgfsetfillcolor{currentfill}%
\pgfsetfillopacity{0.700000}%
\pgfsetlinewidth{0.000000pt}%
\definecolor{currentstroke}{rgb}{0.000000,0.000000,0.000000}%
\pgfsetstrokecolor{currentstroke}%
\pgfsetstrokeopacity{0.700000}%
\pgfsetdash{}{0pt}%
\pgfpathmoveto{\pgfqpoint{5.895597in}{0.550000in}}%
\pgfpathlineto{\pgfqpoint{5.923118in}{0.550000in}}%
\pgfpathlineto{\pgfqpoint{5.923118in}{0.621213in}}%
\pgfpathlineto{\pgfqpoint{5.895597in}{0.621213in}}%
\pgfpathlineto{\pgfqpoint{5.895597in}{0.550000in}}%
\pgfpathclose%
\pgfusepath{fill}%
\end{pgfscope}%
\begin{pgfscope}%
\pgfpathrectangle{\pgfqpoint{1.250000in}{0.550000in}}{\pgfqpoint{7.750000in}{3.850000in}}%
\pgfusepath{clip}%
\pgfsetbuttcap%
\pgfsetmiterjoin%
\definecolor{currentfill}{rgb}{0.501961,0.501961,0.501961}%
\pgfsetfillcolor{currentfill}%
\pgfsetfillopacity{0.700000}%
\pgfsetlinewidth{0.000000pt}%
\definecolor{currentstroke}{rgb}{0.000000,0.000000,0.000000}%
\pgfsetstrokecolor{currentstroke}%
\pgfsetstrokeopacity{0.700000}%
\pgfsetdash{}{0pt}%
\pgfpathmoveto{\pgfqpoint{5.923118in}{0.550000in}}%
\pgfpathlineto{\pgfqpoint{5.950639in}{0.550000in}}%
\pgfpathlineto{\pgfqpoint{5.950639in}{0.607366in}}%
\pgfpathlineto{\pgfqpoint{5.923118in}{0.607366in}}%
\pgfpathlineto{\pgfqpoint{5.923118in}{0.550000in}}%
\pgfpathclose%
\pgfusepath{fill}%
\end{pgfscope}%
\begin{pgfscope}%
\pgfpathrectangle{\pgfqpoint{1.250000in}{0.550000in}}{\pgfqpoint{7.750000in}{3.850000in}}%
\pgfusepath{clip}%
\pgfsetbuttcap%
\pgfsetmiterjoin%
\definecolor{currentfill}{rgb}{0.501961,0.501961,0.501961}%
\pgfsetfillcolor{currentfill}%
\pgfsetfillopacity{0.700000}%
\pgfsetlinewidth{0.000000pt}%
\definecolor{currentstroke}{rgb}{0.000000,0.000000,0.000000}%
\pgfsetstrokecolor{currentstroke}%
\pgfsetstrokeopacity{0.700000}%
\pgfsetdash{}{0pt}%
\pgfpathmoveto{\pgfqpoint{5.950639in}{0.550000in}}%
\pgfpathlineto{\pgfqpoint{5.978161in}{0.550000in}}%
\pgfpathlineto{\pgfqpoint{5.978161in}{0.550000in}}%
\pgfpathlineto{\pgfqpoint{5.950639in}{0.550000in}}%
\pgfpathlineto{\pgfqpoint{5.950639in}{0.550000in}}%
\pgfpathclose%
\pgfusepath{fill}%
\end{pgfscope}%
\begin{pgfscope}%
\pgfpathrectangle{\pgfqpoint{1.250000in}{0.550000in}}{\pgfqpoint{7.750000in}{3.850000in}}%
\pgfusepath{clip}%
\pgfsetbuttcap%
\pgfsetmiterjoin%
\definecolor{currentfill}{rgb}{0.501961,0.501961,0.501961}%
\pgfsetfillcolor{currentfill}%
\pgfsetfillopacity{0.700000}%
\pgfsetlinewidth{0.000000pt}%
\definecolor{currentstroke}{rgb}{0.000000,0.000000,0.000000}%
\pgfsetstrokecolor{currentstroke}%
\pgfsetstrokeopacity{0.700000}%
\pgfsetdash{}{0pt}%
\pgfpathmoveto{\pgfqpoint{5.978161in}{0.550000in}}%
\pgfpathlineto{\pgfqpoint{6.005682in}{0.550000in}}%
\pgfpathlineto{\pgfqpoint{6.005682in}{0.616070in}}%
\pgfpathlineto{\pgfqpoint{5.978161in}{0.616070in}}%
\pgfpathlineto{\pgfqpoint{5.978161in}{0.550000in}}%
\pgfpathclose%
\pgfusepath{fill}%
\end{pgfscope}%
\begin{pgfscope}%
\pgfpathrectangle{\pgfqpoint{1.250000in}{0.550000in}}{\pgfqpoint{7.750000in}{3.850000in}}%
\pgfusepath{clip}%
\pgfsetbuttcap%
\pgfsetmiterjoin%
\definecolor{currentfill}{rgb}{0.501961,0.501961,0.501961}%
\pgfsetfillcolor{currentfill}%
\pgfsetfillopacity{0.700000}%
\pgfsetlinewidth{0.000000pt}%
\definecolor{currentstroke}{rgb}{0.000000,0.000000,0.000000}%
\pgfsetstrokecolor{currentstroke}%
\pgfsetstrokeopacity{0.700000}%
\pgfsetdash{}{0pt}%
\pgfpathmoveto{\pgfqpoint{6.005682in}{0.550000in}}%
\pgfpathlineto{\pgfqpoint{6.033203in}{0.550000in}}%
\pgfpathlineto{\pgfqpoint{6.033203in}{0.615674in}}%
\pgfpathlineto{\pgfqpoint{6.005682in}{0.615674in}}%
\pgfpathlineto{\pgfqpoint{6.005682in}{0.550000in}}%
\pgfpathclose%
\pgfusepath{fill}%
\end{pgfscope}%
\begin{pgfscope}%
\pgfpathrectangle{\pgfqpoint{1.250000in}{0.550000in}}{\pgfqpoint{7.750000in}{3.850000in}}%
\pgfusepath{clip}%
\pgfsetbuttcap%
\pgfsetmiterjoin%
\definecolor{currentfill}{rgb}{0.501961,0.501961,0.501961}%
\pgfsetfillcolor{currentfill}%
\pgfsetfillopacity{0.700000}%
\pgfsetlinewidth{0.000000pt}%
\definecolor{currentstroke}{rgb}{0.000000,0.000000,0.000000}%
\pgfsetstrokecolor{currentstroke}%
\pgfsetstrokeopacity{0.700000}%
\pgfsetdash{}{0pt}%
\pgfpathmoveto{\pgfqpoint{6.033203in}{0.550000in}}%
\pgfpathlineto{\pgfqpoint{6.060724in}{0.550000in}}%
\pgfpathlineto{\pgfqpoint{6.060724in}{0.606179in}}%
\pgfpathlineto{\pgfqpoint{6.033203in}{0.606179in}}%
\pgfpathlineto{\pgfqpoint{6.033203in}{0.550000in}}%
\pgfpathclose%
\pgfusepath{fill}%
\end{pgfscope}%
\begin{pgfscope}%
\pgfpathrectangle{\pgfqpoint{1.250000in}{0.550000in}}{\pgfqpoint{7.750000in}{3.850000in}}%
\pgfusepath{clip}%
\pgfsetbuttcap%
\pgfsetmiterjoin%
\definecolor{currentfill}{rgb}{0.501961,0.501961,0.501961}%
\pgfsetfillcolor{currentfill}%
\pgfsetfillopacity{0.700000}%
\pgfsetlinewidth{0.000000pt}%
\definecolor{currentstroke}{rgb}{0.000000,0.000000,0.000000}%
\pgfsetstrokecolor{currentstroke}%
\pgfsetstrokeopacity{0.700000}%
\pgfsetdash{}{0pt}%
\pgfpathmoveto{\pgfqpoint{6.060724in}{0.550000in}}%
\pgfpathlineto{\pgfqpoint{6.088246in}{0.550000in}}%
\pgfpathlineto{\pgfqpoint{6.088246in}{0.602618in}}%
\pgfpathlineto{\pgfqpoint{6.060724in}{0.602618in}}%
\pgfpathlineto{\pgfqpoint{6.060724in}{0.550000in}}%
\pgfpathclose%
\pgfusepath{fill}%
\end{pgfscope}%
\begin{pgfscope}%
\pgfpathrectangle{\pgfqpoint{1.250000in}{0.550000in}}{\pgfqpoint{7.750000in}{3.850000in}}%
\pgfusepath{clip}%
\pgfsetbuttcap%
\pgfsetmiterjoin%
\definecolor{currentfill}{rgb}{0.501961,0.501961,0.501961}%
\pgfsetfillcolor{currentfill}%
\pgfsetfillopacity{0.700000}%
\pgfsetlinewidth{0.000000pt}%
\definecolor{currentstroke}{rgb}{0.000000,0.000000,0.000000}%
\pgfsetstrokecolor{currentstroke}%
\pgfsetstrokeopacity{0.700000}%
\pgfsetdash{}{0pt}%
\pgfpathmoveto{\pgfqpoint{6.088246in}{0.550000in}}%
\pgfpathlineto{\pgfqpoint{6.115767in}{0.550000in}}%
\pgfpathlineto{\pgfqpoint{6.115767in}{0.550000in}}%
\pgfpathlineto{\pgfqpoint{6.088246in}{0.550000in}}%
\pgfpathlineto{\pgfqpoint{6.088246in}{0.550000in}}%
\pgfpathclose%
\pgfusepath{fill}%
\end{pgfscope}%
\begin{pgfscope}%
\pgfpathrectangle{\pgfqpoint{1.250000in}{0.550000in}}{\pgfqpoint{7.750000in}{3.850000in}}%
\pgfusepath{clip}%
\pgfsetbuttcap%
\pgfsetmiterjoin%
\definecolor{currentfill}{rgb}{0.501961,0.501961,0.501961}%
\pgfsetfillcolor{currentfill}%
\pgfsetfillopacity{0.700000}%
\pgfsetlinewidth{0.000000pt}%
\definecolor{currentstroke}{rgb}{0.000000,0.000000,0.000000}%
\pgfsetstrokecolor{currentstroke}%
\pgfsetstrokeopacity{0.700000}%
\pgfsetdash{}{0pt}%
\pgfpathmoveto{\pgfqpoint{6.115767in}{0.550000in}}%
\pgfpathlineto{\pgfqpoint{6.143288in}{0.550000in}}%
\pgfpathlineto{\pgfqpoint{6.143288in}{0.606970in}}%
\pgfpathlineto{\pgfqpoint{6.115767in}{0.606970in}}%
\pgfpathlineto{\pgfqpoint{6.115767in}{0.550000in}}%
\pgfpathclose%
\pgfusepath{fill}%
\end{pgfscope}%
\begin{pgfscope}%
\pgfpathrectangle{\pgfqpoint{1.250000in}{0.550000in}}{\pgfqpoint{7.750000in}{3.850000in}}%
\pgfusepath{clip}%
\pgfsetbuttcap%
\pgfsetmiterjoin%
\definecolor{currentfill}{rgb}{0.501961,0.501961,0.501961}%
\pgfsetfillcolor{currentfill}%
\pgfsetfillopacity{0.700000}%
\pgfsetlinewidth{0.000000pt}%
\definecolor{currentstroke}{rgb}{0.000000,0.000000,0.000000}%
\pgfsetstrokecolor{currentstroke}%
\pgfsetstrokeopacity{0.700000}%
\pgfsetdash{}{0pt}%
\pgfpathmoveto{\pgfqpoint{6.143288in}{0.550000in}}%
\pgfpathlineto{\pgfqpoint{6.170810in}{0.550000in}}%
\pgfpathlineto{\pgfqpoint{6.170810in}{0.617652in}}%
\pgfpathlineto{\pgfqpoint{6.143288in}{0.617652in}}%
\pgfpathlineto{\pgfqpoint{6.143288in}{0.550000in}}%
\pgfpathclose%
\pgfusepath{fill}%
\end{pgfscope}%
\begin{pgfscope}%
\pgfpathrectangle{\pgfqpoint{1.250000in}{0.550000in}}{\pgfqpoint{7.750000in}{3.850000in}}%
\pgfusepath{clip}%
\pgfsetbuttcap%
\pgfsetmiterjoin%
\definecolor{currentfill}{rgb}{0.501961,0.501961,0.501961}%
\pgfsetfillcolor{currentfill}%
\pgfsetfillopacity{0.700000}%
\pgfsetlinewidth{0.000000pt}%
\definecolor{currentstroke}{rgb}{0.000000,0.000000,0.000000}%
\pgfsetstrokecolor{currentstroke}%
\pgfsetstrokeopacity{0.700000}%
\pgfsetdash{}{0pt}%
\pgfpathmoveto{\pgfqpoint{6.170810in}{0.550000in}}%
\pgfpathlineto{\pgfqpoint{6.198331in}{0.550000in}}%
\pgfpathlineto{\pgfqpoint{6.198331in}{0.610135in}}%
\pgfpathlineto{\pgfqpoint{6.170810in}{0.610135in}}%
\pgfpathlineto{\pgfqpoint{6.170810in}{0.550000in}}%
\pgfpathclose%
\pgfusepath{fill}%
\end{pgfscope}%
\begin{pgfscope}%
\pgfpathrectangle{\pgfqpoint{1.250000in}{0.550000in}}{\pgfqpoint{7.750000in}{3.850000in}}%
\pgfusepath{clip}%
\pgfsetbuttcap%
\pgfsetmiterjoin%
\definecolor{currentfill}{rgb}{0.501961,0.501961,0.501961}%
\pgfsetfillcolor{currentfill}%
\pgfsetfillopacity{0.700000}%
\pgfsetlinewidth{0.000000pt}%
\definecolor{currentstroke}{rgb}{0.000000,0.000000,0.000000}%
\pgfsetstrokecolor{currentstroke}%
\pgfsetstrokeopacity{0.700000}%
\pgfsetdash{}{0pt}%
\pgfpathmoveto{\pgfqpoint{6.198331in}{0.550000in}}%
\pgfpathlineto{\pgfqpoint{6.225852in}{0.550000in}}%
\pgfpathlineto{\pgfqpoint{6.225852in}{0.617257in}}%
\pgfpathlineto{\pgfqpoint{6.198331in}{0.617257in}}%
\pgfpathlineto{\pgfqpoint{6.198331in}{0.550000in}}%
\pgfpathclose%
\pgfusepath{fill}%
\end{pgfscope}%
\begin{pgfscope}%
\pgfpathrectangle{\pgfqpoint{1.250000in}{0.550000in}}{\pgfqpoint{7.750000in}{3.850000in}}%
\pgfusepath{clip}%
\pgfsetbuttcap%
\pgfsetmiterjoin%
\definecolor{currentfill}{rgb}{0.501961,0.501961,0.501961}%
\pgfsetfillcolor{currentfill}%
\pgfsetfillopacity{0.700000}%
\pgfsetlinewidth{0.000000pt}%
\definecolor{currentstroke}{rgb}{0.000000,0.000000,0.000000}%
\pgfsetstrokecolor{currentstroke}%
\pgfsetstrokeopacity{0.700000}%
\pgfsetdash{}{0pt}%
\pgfpathmoveto{\pgfqpoint{6.225852in}{0.550000in}}%
\pgfpathlineto{\pgfqpoint{6.253374in}{0.550000in}}%
\pgfpathlineto{\pgfqpoint{6.253374in}{0.550000in}}%
\pgfpathlineto{\pgfqpoint{6.225852in}{0.550000in}}%
\pgfpathlineto{\pgfqpoint{6.225852in}{0.550000in}}%
\pgfpathclose%
\pgfusepath{fill}%
\end{pgfscope}%
\begin{pgfscope}%
\pgfpathrectangle{\pgfqpoint{1.250000in}{0.550000in}}{\pgfqpoint{7.750000in}{3.850000in}}%
\pgfusepath{clip}%
\pgfsetbuttcap%
\pgfsetmiterjoin%
\definecolor{currentfill}{rgb}{0.501961,0.501961,0.501961}%
\pgfsetfillcolor{currentfill}%
\pgfsetfillopacity{0.700000}%
\pgfsetlinewidth{0.000000pt}%
\definecolor{currentstroke}{rgb}{0.000000,0.000000,0.000000}%
\pgfsetstrokecolor{currentstroke}%
\pgfsetstrokeopacity{0.700000}%
\pgfsetdash{}{0pt}%
\pgfpathmoveto{\pgfqpoint{6.253374in}{0.550000in}}%
\pgfpathlineto{\pgfqpoint{6.280895in}{0.550000in}}%
\pgfpathlineto{\pgfqpoint{6.280895in}{0.608948in}}%
\pgfpathlineto{\pgfqpoint{6.253374in}{0.608948in}}%
\pgfpathlineto{\pgfqpoint{6.253374in}{0.550000in}}%
\pgfpathclose%
\pgfusepath{fill}%
\end{pgfscope}%
\begin{pgfscope}%
\pgfpathrectangle{\pgfqpoint{1.250000in}{0.550000in}}{\pgfqpoint{7.750000in}{3.850000in}}%
\pgfusepath{clip}%
\pgfsetbuttcap%
\pgfsetmiterjoin%
\definecolor{currentfill}{rgb}{0.501961,0.501961,0.501961}%
\pgfsetfillcolor{currentfill}%
\pgfsetfillopacity{0.700000}%
\pgfsetlinewidth{0.000000pt}%
\definecolor{currentstroke}{rgb}{0.000000,0.000000,0.000000}%
\pgfsetstrokecolor{currentstroke}%
\pgfsetstrokeopacity{0.700000}%
\pgfsetdash{}{0pt}%
\pgfpathmoveto{\pgfqpoint{6.280895in}{0.550000in}}%
\pgfpathlineto{\pgfqpoint{6.308416in}{0.550000in}}%
\pgfpathlineto{\pgfqpoint{6.308416in}{0.603410in}}%
\pgfpathlineto{\pgfqpoint{6.280895in}{0.603410in}}%
\pgfpathlineto{\pgfqpoint{6.280895in}{0.550000in}}%
\pgfpathclose%
\pgfusepath{fill}%
\end{pgfscope}%
\begin{pgfscope}%
\pgfpathrectangle{\pgfqpoint{1.250000in}{0.550000in}}{\pgfqpoint{7.750000in}{3.850000in}}%
\pgfusepath{clip}%
\pgfsetbuttcap%
\pgfsetmiterjoin%
\definecolor{currentfill}{rgb}{0.501961,0.501961,0.501961}%
\pgfsetfillcolor{currentfill}%
\pgfsetfillopacity{0.700000}%
\pgfsetlinewidth{0.000000pt}%
\definecolor{currentstroke}{rgb}{0.000000,0.000000,0.000000}%
\pgfsetstrokecolor{currentstroke}%
\pgfsetstrokeopacity{0.700000}%
\pgfsetdash{}{0pt}%
\pgfpathmoveto{\pgfqpoint{6.308416in}{0.550000in}}%
\pgfpathlineto{\pgfqpoint{6.335937in}{0.550000in}}%
\pgfpathlineto{\pgfqpoint{6.335937in}{0.597080in}}%
\pgfpathlineto{\pgfqpoint{6.308416in}{0.597080in}}%
\pgfpathlineto{\pgfqpoint{6.308416in}{0.550000in}}%
\pgfpathclose%
\pgfusepath{fill}%
\end{pgfscope}%
\begin{pgfscope}%
\pgfpathrectangle{\pgfqpoint{1.250000in}{0.550000in}}{\pgfqpoint{7.750000in}{3.850000in}}%
\pgfusepath{clip}%
\pgfsetbuttcap%
\pgfsetmiterjoin%
\definecolor{currentfill}{rgb}{0.501961,0.501961,0.501961}%
\pgfsetfillcolor{currentfill}%
\pgfsetfillopacity{0.700000}%
\pgfsetlinewidth{0.000000pt}%
\definecolor{currentstroke}{rgb}{0.000000,0.000000,0.000000}%
\pgfsetstrokecolor{currentstroke}%
\pgfsetstrokeopacity{0.700000}%
\pgfsetdash{}{0pt}%
\pgfpathmoveto{\pgfqpoint{6.335938in}{0.550000in}}%
\pgfpathlineto{\pgfqpoint{6.363459in}{0.550000in}}%
\pgfpathlineto{\pgfqpoint{6.363459in}{0.593519in}}%
\pgfpathlineto{\pgfqpoint{6.335938in}{0.593519in}}%
\pgfpathlineto{\pgfqpoint{6.335938in}{0.550000in}}%
\pgfpathclose%
\pgfusepath{fill}%
\end{pgfscope}%
\begin{pgfscope}%
\pgfpathrectangle{\pgfqpoint{1.250000in}{0.550000in}}{\pgfqpoint{7.750000in}{3.850000in}}%
\pgfusepath{clip}%
\pgfsetbuttcap%
\pgfsetmiterjoin%
\definecolor{currentfill}{rgb}{0.501961,0.501961,0.501961}%
\pgfsetfillcolor{currentfill}%
\pgfsetfillopacity{0.700000}%
\pgfsetlinewidth{0.000000pt}%
\definecolor{currentstroke}{rgb}{0.000000,0.000000,0.000000}%
\pgfsetstrokecolor{currentstroke}%
\pgfsetstrokeopacity{0.700000}%
\pgfsetdash{}{0pt}%
\pgfpathmoveto{\pgfqpoint{6.363459in}{0.550000in}}%
\pgfpathlineto{\pgfqpoint{6.390980in}{0.550000in}}%
\pgfpathlineto{\pgfqpoint{6.390980in}{0.597080in}}%
\pgfpathlineto{\pgfqpoint{6.363459in}{0.597080in}}%
\pgfpathlineto{\pgfqpoint{6.363459in}{0.550000in}}%
\pgfpathclose%
\pgfusepath{fill}%
\end{pgfscope}%
\begin{pgfscope}%
\pgfpathrectangle{\pgfqpoint{1.250000in}{0.550000in}}{\pgfqpoint{7.750000in}{3.850000in}}%
\pgfusepath{clip}%
\pgfsetbuttcap%
\pgfsetmiterjoin%
\definecolor{currentfill}{rgb}{0.501961,0.501961,0.501961}%
\pgfsetfillcolor{currentfill}%
\pgfsetfillopacity{0.700000}%
\pgfsetlinewidth{0.000000pt}%
\definecolor{currentstroke}{rgb}{0.000000,0.000000,0.000000}%
\pgfsetstrokecolor{currentstroke}%
\pgfsetstrokeopacity{0.700000}%
\pgfsetdash{}{0pt}%
\pgfpathmoveto{\pgfqpoint{6.390980in}{0.550000in}}%
\pgfpathlineto{\pgfqpoint{6.418501in}{0.550000in}}%
\pgfpathlineto{\pgfqpoint{6.418501in}{0.590750in}}%
\pgfpathlineto{\pgfqpoint{6.390980in}{0.590750in}}%
\pgfpathlineto{\pgfqpoint{6.390980in}{0.550000in}}%
\pgfpathclose%
\pgfusepath{fill}%
\end{pgfscope}%
\begin{pgfscope}%
\pgfpathrectangle{\pgfqpoint{1.250000in}{0.550000in}}{\pgfqpoint{7.750000in}{3.850000in}}%
\pgfusepath{clip}%
\pgfsetbuttcap%
\pgfsetmiterjoin%
\definecolor{currentfill}{rgb}{0.501961,0.501961,0.501961}%
\pgfsetfillcolor{currentfill}%
\pgfsetfillopacity{0.700000}%
\pgfsetlinewidth{0.000000pt}%
\definecolor{currentstroke}{rgb}{0.000000,0.000000,0.000000}%
\pgfsetstrokecolor{currentstroke}%
\pgfsetstrokeopacity{0.700000}%
\pgfsetdash{}{0pt}%
\pgfpathmoveto{\pgfqpoint{6.418501in}{0.550000in}}%
\pgfpathlineto{\pgfqpoint{6.446023in}{0.550000in}}%
\pgfpathlineto{\pgfqpoint{6.446023in}{0.550000in}}%
\pgfpathlineto{\pgfqpoint{6.418501in}{0.550000in}}%
\pgfpathlineto{\pgfqpoint{6.418501in}{0.550000in}}%
\pgfpathclose%
\pgfusepath{fill}%
\end{pgfscope}%
\begin{pgfscope}%
\pgfpathrectangle{\pgfqpoint{1.250000in}{0.550000in}}{\pgfqpoint{7.750000in}{3.850000in}}%
\pgfusepath{clip}%
\pgfsetbuttcap%
\pgfsetmiterjoin%
\definecolor{currentfill}{rgb}{0.501961,0.501961,0.501961}%
\pgfsetfillcolor{currentfill}%
\pgfsetfillopacity{0.700000}%
\pgfsetlinewidth{0.000000pt}%
\definecolor{currentstroke}{rgb}{0.000000,0.000000,0.000000}%
\pgfsetstrokecolor{currentstroke}%
\pgfsetstrokeopacity{0.700000}%
\pgfsetdash{}{0pt}%
\pgfpathmoveto{\pgfqpoint{6.446023in}{0.550000in}}%
\pgfpathlineto{\pgfqpoint{6.473544in}{0.550000in}}%
\pgfpathlineto{\pgfqpoint{6.473544in}{0.596288in}}%
\pgfpathlineto{\pgfqpoint{6.446023in}{0.596288in}}%
\pgfpathlineto{\pgfqpoint{6.446023in}{0.550000in}}%
\pgfpathclose%
\pgfusepath{fill}%
\end{pgfscope}%
\begin{pgfscope}%
\pgfpathrectangle{\pgfqpoint{1.250000in}{0.550000in}}{\pgfqpoint{7.750000in}{3.850000in}}%
\pgfusepath{clip}%
\pgfsetbuttcap%
\pgfsetmiterjoin%
\definecolor{currentfill}{rgb}{0.501961,0.501961,0.501961}%
\pgfsetfillcolor{currentfill}%
\pgfsetfillopacity{0.700000}%
\pgfsetlinewidth{0.000000pt}%
\definecolor{currentstroke}{rgb}{0.000000,0.000000,0.000000}%
\pgfsetstrokecolor{currentstroke}%
\pgfsetstrokeopacity{0.700000}%
\pgfsetdash{}{0pt}%
\pgfpathmoveto{\pgfqpoint{6.473544in}{0.550000in}}%
\pgfpathlineto{\pgfqpoint{6.501065in}{0.550000in}}%
\pgfpathlineto{\pgfqpoint{6.501065in}{0.594310in}}%
\pgfpathlineto{\pgfqpoint{6.473544in}{0.594310in}}%
\pgfpathlineto{\pgfqpoint{6.473544in}{0.550000in}}%
\pgfpathclose%
\pgfusepath{fill}%
\end{pgfscope}%
\begin{pgfscope}%
\pgfpathrectangle{\pgfqpoint{1.250000in}{0.550000in}}{\pgfqpoint{7.750000in}{3.850000in}}%
\pgfusepath{clip}%
\pgfsetbuttcap%
\pgfsetmiterjoin%
\definecolor{currentfill}{rgb}{0.501961,0.501961,0.501961}%
\pgfsetfillcolor{currentfill}%
\pgfsetfillopacity{0.700000}%
\pgfsetlinewidth{0.000000pt}%
\definecolor{currentstroke}{rgb}{0.000000,0.000000,0.000000}%
\pgfsetstrokecolor{currentstroke}%
\pgfsetstrokeopacity{0.700000}%
\pgfsetdash{}{0pt}%
\pgfpathmoveto{\pgfqpoint{6.501065in}{0.550000in}}%
\pgfpathlineto{\pgfqpoint{6.528587in}{0.550000in}}%
\pgfpathlineto{\pgfqpoint{6.528587in}{0.597080in}}%
\pgfpathlineto{\pgfqpoint{6.501065in}{0.597080in}}%
\pgfpathlineto{\pgfqpoint{6.501065in}{0.550000in}}%
\pgfpathclose%
\pgfusepath{fill}%
\end{pgfscope}%
\begin{pgfscope}%
\pgfpathrectangle{\pgfqpoint{1.250000in}{0.550000in}}{\pgfqpoint{7.750000in}{3.850000in}}%
\pgfusepath{clip}%
\pgfsetbuttcap%
\pgfsetmiterjoin%
\definecolor{currentfill}{rgb}{0.501961,0.501961,0.501961}%
\pgfsetfillcolor{currentfill}%
\pgfsetfillopacity{0.700000}%
\pgfsetlinewidth{0.000000pt}%
\definecolor{currentstroke}{rgb}{0.000000,0.000000,0.000000}%
\pgfsetstrokecolor{currentstroke}%
\pgfsetstrokeopacity{0.700000}%
\pgfsetdash{}{0pt}%
\pgfpathmoveto{\pgfqpoint{6.528587in}{0.550000in}}%
\pgfpathlineto{\pgfqpoint{6.556108in}{0.550000in}}%
\pgfpathlineto{\pgfqpoint{6.556108in}{0.591936in}}%
\pgfpathlineto{\pgfqpoint{6.528587in}{0.591936in}}%
\pgfpathlineto{\pgfqpoint{6.528587in}{0.550000in}}%
\pgfpathclose%
\pgfusepath{fill}%
\end{pgfscope}%
\begin{pgfscope}%
\pgfpathrectangle{\pgfqpoint{1.250000in}{0.550000in}}{\pgfqpoint{7.750000in}{3.850000in}}%
\pgfusepath{clip}%
\pgfsetbuttcap%
\pgfsetmiterjoin%
\definecolor{currentfill}{rgb}{0.501961,0.501961,0.501961}%
\pgfsetfillcolor{currentfill}%
\pgfsetfillopacity{0.700000}%
\pgfsetlinewidth{0.000000pt}%
\definecolor{currentstroke}{rgb}{0.000000,0.000000,0.000000}%
\pgfsetstrokecolor{currentstroke}%
\pgfsetstrokeopacity{0.700000}%
\pgfsetdash{}{0pt}%
\pgfpathmoveto{\pgfqpoint{6.556108in}{0.550000in}}%
\pgfpathlineto{\pgfqpoint{6.583629in}{0.550000in}}%
\pgfpathlineto{\pgfqpoint{6.583629in}{0.592332in}}%
\pgfpathlineto{\pgfqpoint{6.556108in}{0.592332in}}%
\pgfpathlineto{\pgfqpoint{6.556108in}{0.550000in}}%
\pgfpathclose%
\pgfusepath{fill}%
\end{pgfscope}%
\begin{pgfscope}%
\pgfpathrectangle{\pgfqpoint{1.250000in}{0.550000in}}{\pgfqpoint{7.750000in}{3.850000in}}%
\pgfusepath{clip}%
\pgfsetbuttcap%
\pgfsetmiterjoin%
\definecolor{currentfill}{rgb}{0.501961,0.501961,0.501961}%
\pgfsetfillcolor{currentfill}%
\pgfsetfillopacity{0.700000}%
\pgfsetlinewidth{0.000000pt}%
\definecolor{currentstroke}{rgb}{0.000000,0.000000,0.000000}%
\pgfsetstrokecolor{currentstroke}%
\pgfsetstrokeopacity{0.700000}%
\pgfsetdash{}{0pt}%
\pgfpathmoveto{\pgfqpoint{6.583629in}{0.550000in}}%
\pgfpathlineto{\pgfqpoint{6.611151in}{0.550000in}}%
\pgfpathlineto{\pgfqpoint{6.611151in}{0.595497in}}%
\pgfpathlineto{\pgfqpoint{6.583629in}{0.595497in}}%
\pgfpathlineto{\pgfqpoint{6.583629in}{0.550000in}}%
\pgfpathclose%
\pgfusepath{fill}%
\end{pgfscope}%
\begin{pgfscope}%
\pgfpathrectangle{\pgfqpoint{1.250000in}{0.550000in}}{\pgfqpoint{7.750000in}{3.850000in}}%
\pgfusepath{clip}%
\pgfsetbuttcap%
\pgfsetmiterjoin%
\definecolor{currentfill}{rgb}{0.501961,0.501961,0.501961}%
\pgfsetfillcolor{currentfill}%
\pgfsetfillopacity{0.700000}%
\pgfsetlinewidth{0.000000pt}%
\definecolor{currentstroke}{rgb}{0.000000,0.000000,0.000000}%
\pgfsetstrokecolor{currentstroke}%
\pgfsetstrokeopacity{0.700000}%
\pgfsetdash{}{0pt}%
\pgfpathmoveto{\pgfqpoint{6.611151in}{0.550000in}}%
\pgfpathlineto{\pgfqpoint{6.638672in}{0.550000in}}%
\pgfpathlineto{\pgfqpoint{6.638672in}{0.593915in}}%
\pgfpathlineto{\pgfqpoint{6.611151in}{0.593915in}}%
\pgfpathlineto{\pgfqpoint{6.611151in}{0.550000in}}%
\pgfpathclose%
\pgfusepath{fill}%
\end{pgfscope}%
\begin{pgfscope}%
\pgfpathrectangle{\pgfqpoint{1.250000in}{0.550000in}}{\pgfqpoint{7.750000in}{3.850000in}}%
\pgfusepath{clip}%
\pgfsetbuttcap%
\pgfsetmiterjoin%
\definecolor{currentfill}{rgb}{0.501961,0.501961,0.501961}%
\pgfsetfillcolor{currentfill}%
\pgfsetfillopacity{0.700000}%
\pgfsetlinewidth{0.000000pt}%
\definecolor{currentstroke}{rgb}{0.000000,0.000000,0.000000}%
\pgfsetstrokecolor{currentstroke}%
\pgfsetstrokeopacity{0.700000}%
\pgfsetdash{}{0pt}%
\pgfpathmoveto{\pgfqpoint{6.638672in}{0.550000in}}%
\pgfpathlineto{\pgfqpoint{6.666193in}{0.550000in}}%
\pgfpathlineto{\pgfqpoint{6.666193in}{0.586398in}}%
\pgfpathlineto{\pgfqpoint{6.638672in}{0.586398in}}%
\pgfpathlineto{\pgfqpoint{6.638672in}{0.550000in}}%
\pgfpathclose%
\pgfusepath{fill}%
\end{pgfscope}%
\begin{pgfscope}%
\pgfpathrectangle{\pgfqpoint{1.250000in}{0.550000in}}{\pgfqpoint{7.750000in}{3.850000in}}%
\pgfusepath{clip}%
\pgfsetbuttcap%
\pgfsetmiterjoin%
\definecolor{currentfill}{rgb}{0.501961,0.501961,0.501961}%
\pgfsetfillcolor{currentfill}%
\pgfsetfillopacity{0.700000}%
\pgfsetlinewidth{0.000000pt}%
\definecolor{currentstroke}{rgb}{0.000000,0.000000,0.000000}%
\pgfsetstrokecolor{currentstroke}%
\pgfsetstrokeopacity{0.700000}%
\pgfsetdash{}{0pt}%
\pgfpathmoveto{\pgfqpoint{6.666193in}{0.550000in}}%
\pgfpathlineto{\pgfqpoint{6.693714in}{0.550000in}}%
\pgfpathlineto{\pgfqpoint{6.693714in}{0.585606in}}%
\pgfpathlineto{\pgfqpoint{6.666193in}{0.585606in}}%
\pgfpathlineto{\pgfqpoint{6.666193in}{0.550000in}}%
\pgfpathclose%
\pgfusepath{fill}%
\end{pgfscope}%
\begin{pgfscope}%
\pgfpathrectangle{\pgfqpoint{1.250000in}{0.550000in}}{\pgfqpoint{7.750000in}{3.850000in}}%
\pgfusepath{clip}%
\pgfsetbuttcap%
\pgfsetmiterjoin%
\definecolor{currentfill}{rgb}{0.501961,0.501961,0.501961}%
\pgfsetfillcolor{currentfill}%
\pgfsetfillopacity{0.700000}%
\pgfsetlinewidth{0.000000pt}%
\definecolor{currentstroke}{rgb}{0.000000,0.000000,0.000000}%
\pgfsetstrokecolor{currentstroke}%
\pgfsetstrokeopacity{0.700000}%
\pgfsetdash{}{0pt}%
\pgfpathmoveto{\pgfqpoint{6.693714in}{0.550000in}}%
\pgfpathlineto{\pgfqpoint{6.721236in}{0.550000in}}%
\pgfpathlineto{\pgfqpoint{6.721236in}{0.590354in}}%
\pgfpathlineto{\pgfqpoint{6.693714in}{0.590354in}}%
\pgfpathlineto{\pgfqpoint{6.693714in}{0.550000in}}%
\pgfpathclose%
\pgfusepath{fill}%
\end{pgfscope}%
\begin{pgfscope}%
\pgfpathrectangle{\pgfqpoint{1.250000in}{0.550000in}}{\pgfqpoint{7.750000in}{3.850000in}}%
\pgfusepath{clip}%
\pgfsetbuttcap%
\pgfsetmiterjoin%
\definecolor{currentfill}{rgb}{0.501961,0.501961,0.501961}%
\pgfsetfillcolor{currentfill}%
\pgfsetfillopacity{0.700000}%
\pgfsetlinewidth{0.000000pt}%
\definecolor{currentstroke}{rgb}{0.000000,0.000000,0.000000}%
\pgfsetstrokecolor{currentstroke}%
\pgfsetstrokeopacity{0.700000}%
\pgfsetdash{}{0pt}%
\pgfpathmoveto{\pgfqpoint{6.721236in}{0.550000in}}%
\pgfpathlineto{\pgfqpoint{6.748757in}{0.550000in}}%
\pgfpathlineto{\pgfqpoint{6.748757in}{0.591936in}}%
\pgfpathlineto{\pgfqpoint{6.721236in}{0.591936in}}%
\pgfpathlineto{\pgfqpoint{6.721236in}{0.550000in}}%
\pgfpathclose%
\pgfusepath{fill}%
\end{pgfscope}%
\begin{pgfscope}%
\pgfpathrectangle{\pgfqpoint{1.250000in}{0.550000in}}{\pgfqpoint{7.750000in}{3.850000in}}%
\pgfusepath{clip}%
\pgfsetbuttcap%
\pgfsetmiterjoin%
\definecolor{currentfill}{rgb}{0.501961,0.501961,0.501961}%
\pgfsetfillcolor{currentfill}%
\pgfsetfillopacity{0.700000}%
\pgfsetlinewidth{0.000000pt}%
\definecolor{currentstroke}{rgb}{0.000000,0.000000,0.000000}%
\pgfsetstrokecolor{currentstroke}%
\pgfsetstrokeopacity{0.700000}%
\pgfsetdash{}{0pt}%
\pgfpathmoveto{\pgfqpoint{6.748757in}{0.550000in}}%
\pgfpathlineto{\pgfqpoint{6.776278in}{0.550000in}}%
\pgfpathlineto{\pgfqpoint{6.776278in}{0.550000in}}%
\pgfpathlineto{\pgfqpoint{6.748757in}{0.550000in}}%
\pgfpathlineto{\pgfqpoint{6.748757in}{0.550000in}}%
\pgfpathclose%
\pgfusepath{fill}%
\end{pgfscope}%
\begin{pgfscope}%
\pgfpathrectangle{\pgfqpoint{1.250000in}{0.550000in}}{\pgfqpoint{7.750000in}{3.850000in}}%
\pgfusepath{clip}%
\pgfsetbuttcap%
\pgfsetmiterjoin%
\definecolor{currentfill}{rgb}{0.501961,0.501961,0.501961}%
\pgfsetfillcolor{currentfill}%
\pgfsetfillopacity{0.700000}%
\pgfsetlinewidth{0.000000pt}%
\definecolor{currentstroke}{rgb}{0.000000,0.000000,0.000000}%
\pgfsetstrokecolor{currentstroke}%
\pgfsetstrokeopacity{0.700000}%
\pgfsetdash{}{0pt}%
\pgfpathmoveto{\pgfqpoint{6.776278in}{0.550000in}}%
\pgfpathlineto{\pgfqpoint{6.803800in}{0.550000in}}%
\pgfpathlineto{\pgfqpoint{6.803800in}{0.595101in}}%
\pgfpathlineto{\pgfqpoint{6.776278in}{0.595101in}}%
\pgfpathlineto{\pgfqpoint{6.776278in}{0.550000in}}%
\pgfpathclose%
\pgfusepath{fill}%
\end{pgfscope}%
\begin{pgfscope}%
\pgfpathrectangle{\pgfqpoint{1.250000in}{0.550000in}}{\pgfqpoint{7.750000in}{3.850000in}}%
\pgfusepath{clip}%
\pgfsetbuttcap%
\pgfsetmiterjoin%
\definecolor{currentfill}{rgb}{0.501961,0.501961,0.501961}%
\pgfsetfillcolor{currentfill}%
\pgfsetfillopacity{0.700000}%
\pgfsetlinewidth{0.000000pt}%
\definecolor{currentstroke}{rgb}{0.000000,0.000000,0.000000}%
\pgfsetstrokecolor{currentstroke}%
\pgfsetstrokeopacity{0.700000}%
\pgfsetdash{}{0pt}%
\pgfpathmoveto{\pgfqpoint{6.803800in}{0.550000in}}%
\pgfpathlineto{\pgfqpoint{6.831321in}{0.550000in}}%
\pgfpathlineto{\pgfqpoint{6.831321in}{0.592728in}}%
\pgfpathlineto{\pgfqpoint{6.803800in}{0.592728in}}%
\pgfpathlineto{\pgfqpoint{6.803800in}{0.550000in}}%
\pgfpathclose%
\pgfusepath{fill}%
\end{pgfscope}%
\begin{pgfscope}%
\pgfpathrectangle{\pgfqpoint{1.250000in}{0.550000in}}{\pgfqpoint{7.750000in}{3.850000in}}%
\pgfusepath{clip}%
\pgfsetbuttcap%
\pgfsetmiterjoin%
\definecolor{currentfill}{rgb}{0.501961,0.501961,0.501961}%
\pgfsetfillcolor{currentfill}%
\pgfsetfillopacity{0.700000}%
\pgfsetlinewidth{0.000000pt}%
\definecolor{currentstroke}{rgb}{0.000000,0.000000,0.000000}%
\pgfsetstrokecolor{currentstroke}%
\pgfsetstrokeopacity{0.700000}%
\pgfsetdash{}{0pt}%
\pgfpathmoveto{\pgfqpoint{6.831321in}{0.550000in}}%
\pgfpathlineto{\pgfqpoint{6.858842in}{0.550000in}}%
\pgfpathlineto{\pgfqpoint{6.858842in}{0.595101in}}%
\pgfpathlineto{\pgfqpoint{6.831321in}{0.595101in}}%
\pgfpathlineto{\pgfqpoint{6.831321in}{0.550000in}}%
\pgfpathclose%
\pgfusepath{fill}%
\end{pgfscope}%
\begin{pgfscope}%
\pgfpathrectangle{\pgfqpoint{1.250000in}{0.550000in}}{\pgfqpoint{7.750000in}{3.850000in}}%
\pgfusepath{clip}%
\pgfsetbuttcap%
\pgfsetmiterjoin%
\definecolor{currentfill}{rgb}{0.501961,0.501961,0.501961}%
\pgfsetfillcolor{currentfill}%
\pgfsetfillopacity{0.700000}%
\pgfsetlinewidth{0.000000pt}%
\definecolor{currentstroke}{rgb}{0.000000,0.000000,0.000000}%
\pgfsetstrokecolor{currentstroke}%
\pgfsetstrokeopacity{0.700000}%
\pgfsetdash{}{0pt}%
\pgfpathmoveto{\pgfqpoint{6.858842in}{0.550000in}}%
\pgfpathlineto{\pgfqpoint{6.886364in}{0.550000in}}%
\pgfpathlineto{\pgfqpoint{6.886364in}{0.591145in}}%
\pgfpathlineto{\pgfqpoint{6.858842in}{0.591145in}}%
\pgfpathlineto{\pgfqpoint{6.858842in}{0.550000in}}%
\pgfpathclose%
\pgfusepath{fill}%
\end{pgfscope}%
\begin{pgfscope}%
\pgfpathrectangle{\pgfqpoint{1.250000in}{0.550000in}}{\pgfqpoint{7.750000in}{3.850000in}}%
\pgfusepath{clip}%
\pgfsetbuttcap%
\pgfsetmiterjoin%
\definecolor{currentfill}{rgb}{0.501961,0.501961,0.501961}%
\pgfsetfillcolor{currentfill}%
\pgfsetfillopacity{0.700000}%
\pgfsetlinewidth{0.000000pt}%
\definecolor{currentstroke}{rgb}{0.000000,0.000000,0.000000}%
\pgfsetstrokecolor{currentstroke}%
\pgfsetstrokeopacity{0.700000}%
\pgfsetdash{}{0pt}%
\pgfpathmoveto{\pgfqpoint{6.886364in}{0.550000in}}%
\pgfpathlineto{\pgfqpoint{6.913885in}{0.550000in}}%
\pgfpathlineto{\pgfqpoint{6.913885in}{0.590354in}}%
\pgfpathlineto{\pgfqpoint{6.886364in}{0.590354in}}%
\pgfpathlineto{\pgfqpoint{6.886364in}{0.550000in}}%
\pgfpathclose%
\pgfusepath{fill}%
\end{pgfscope}%
\begin{pgfscope}%
\pgfpathrectangle{\pgfqpoint{1.250000in}{0.550000in}}{\pgfqpoint{7.750000in}{3.850000in}}%
\pgfusepath{clip}%
\pgfsetbuttcap%
\pgfsetmiterjoin%
\definecolor{currentfill}{rgb}{0.501961,0.501961,0.501961}%
\pgfsetfillcolor{currentfill}%
\pgfsetfillopacity{0.700000}%
\pgfsetlinewidth{0.000000pt}%
\definecolor{currentstroke}{rgb}{0.000000,0.000000,0.000000}%
\pgfsetstrokecolor{currentstroke}%
\pgfsetstrokeopacity{0.700000}%
\pgfsetdash{}{0pt}%
\pgfpathmoveto{\pgfqpoint{6.913885in}{0.550000in}}%
\pgfpathlineto{\pgfqpoint{6.941406in}{0.550000in}}%
\pgfpathlineto{\pgfqpoint{6.941406in}{0.587980in}}%
\pgfpathlineto{\pgfqpoint{6.913885in}{0.587980in}}%
\pgfpathlineto{\pgfqpoint{6.913885in}{0.550000in}}%
\pgfpathclose%
\pgfusepath{fill}%
\end{pgfscope}%
\begin{pgfscope}%
\pgfpathrectangle{\pgfqpoint{1.250000in}{0.550000in}}{\pgfqpoint{7.750000in}{3.850000in}}%
\pgfusepath{clip}%
\pgfsetbuttcap%
\pgfsetmiterjoin%
\definecolor{currentfill}{rgb}{0.501961,0.501961,0.501961}%
\pgfsetfillcolor{currentfill}%
\pgfsetfillopacity{0.700000}%
\pgfsetlinewidth{0.000000pt}%
\definecolor{currentstroke}{rgb}{0.000000,0.000000,0.000000}%
\pgfsetstrokecolor{currentstroke}%
\pgfsetstrokeopacity{0.700000}%
\pgfsetdash{}{0pt}%
\pgfpathmoveto{\pgfqpoint{6.941406in}{0.550000in}}%
\pgfpathlineto{\pgfqpoint{6.968928in}{0.550000in}}%
\pgfpathlineto{\pgfqpoint{6.968928in}{0.585211in}}%
\pgfpathlineto{\pgfqpoint{6.941406in}{0.585211in}}%
\pgfpathlineto{\pgfqpoint{6.941406in}{0.550000in}}%
\pgfpathclose%
\pgfusepath{fill}%
\end{pgfscope}%
\begin{pgfscope}%
\pgfpathrectangle{\pgfqpoint{1.250000in}{0.550000in}}{\pgfqpoint{7.750000in}{3.850000in}}%
\pgfusepath{clip}%
\pgfsetbuttcap%
\pgfsetmiterjoin%
\definecolor{currentfill}{rgb}{0.501961,0.501961,0.501961}%
\pgfsetfillcolor{currentfill}%
\pgfsetfillopacity{0.700000}%
\pgfsetlinewidth{0.000000pt}%
\definecolor{currentstroke}{rgb}{0.000000,0.000000,0.000000}%
\pgfsetstrokecolor{currentstroke}%
\pgfsetstrokeopacity{0.700000}%
\pgfsetdash{}{0pt}%
\pgfpathmoveto{\pgfqpoint{6.968928in}{0.550000in}}%
\pgfpathlineto{\pgfqpoint{6.996449in}{0.550000in}}%
\pgfpathlineto{\pgfqpoint{6.996449in}{0.587189in}}%
\pgfpathlineto{\pgfqpoint{6.968928in}{0.587189in}}%
\pgfpathlineto{\pgfqpoint{6.968928in}{0.550000in}}%
\pgfpathclose%
\pgfusepath{fill}%
\end{pgfscope}%
\begin{pgfscope}%
\pgfpathrectangle{\pgfqpoint{1.250000in}{0.550000in}}{\pgfqpoint{7.750000in}{3.850000in}}%
\pgfusepath{clip}%
\pgfsetbuttcap%
\pgfsetmiterjoin%
\definecolor{currentfill}{rgb}{0.501961,0.501961,0.501961}%
\pgfsetfillcolor{currentfill}%
\pgfsetfillopacity{0.700000}%
\pgfsetlinewidth{0.000000pt}%
\definecolor{currentstroke}{rgb}{0.000000,0.000000,0.000000}%
\pgfsetstrokecolor{currentstroke}%
\pgfsetstrokeopacity{0.700000}%
\pgfsetdash{}{0pt}%
\pgfpathmoveto{\pgfqpoint{6.996449in}{0.550000in}}%
\pgfpathlineto{\pgfqpoint{7.023970in}{0.550000in}}%
\pgfpathlineto{\pgfqpoint{7.023970in}{0.591936in}}%
\pgfpathlineto{\pgfqpoint{6.996449in}{0.591936in}}%
\pgfpathlineto{\pgfqpoint{6.996449in}{0.550000in}}%
\pgfpathclose%
\pgfusepath{fill}%
\end{pgfscope}%
\begin{pgfscope}%
\pgfpathrectangle{\pgfqpoint{1.250000in}{0.550000in}}{\pgfqpoint{7.750000in}{3.850000in}}%
\pgfusepath{clip}%
\pgfsetbuttcap%
\pgfsetmiterjoin%
\definecolor{currentfill}{rgb}{0.501961,0.501961,0.501961}%
\pgfsetfillcolor{currentfill}%
\pgfsetfillopacity{0.700000}%
\pgfsetlinewidth{0.000000pt}%
\definecolor{currentstroke}{rgb}{0.000000,0.000000,0.000000}%
\pgfsetstrokecolor{currentstroke}%
\pgfsetstrokeopacity{0.700000}%
\pgfsetdash{}{0pt}%
\pgfpathmoveto{\pgfqpoint{7.023970in}{0.550000in}}%
\pgfpathlineto{\pgfqpoint{7.051491in}{0.550000in}}%
\pgfpathlineto{\pgfqpoint{7.051491in}{0.586398in}}%
\pgfpathlineto{\pgfqpoint{7.023970in}{0.586398in}}%
\pgfpathlineto{\pgfqpoint{7.023970in}{0.550000in}}%
\pgfpathclose%
\pgfusepath{fill}%
\end{pgfscope}%
\begin{pgfscope}%
\pgfpathrectangle{\pgfqpoint{1.250000in}{0.550000in}}{\pgfqpoint{7.750000in}{3.850000in}}%
\pgfusepath{clip}%
\pgfsetbuttcap%
\pgfsetmiterjoin%
\definecolor{currentfill}{rgb}{0.501961,0.501961,0.501961}%
\pgfsetfillcolor{currentfill}%
\pgfsetfillopacity{0.700000}%
\pgfsetlinewidth{0.000000pt}%
\definecolor{currentstroke}{rgb}{0.000000,0.000000,0.000000}%
\pgfsetstrokecolor{currentstroke}%
\pgfsetstrokeopacity{0.700000}%
\pgfsetdash{}{0pt}%
\pgfpathmoveto{\pgfqpoint{7.051491in}{0.550000in}}%
\pgfpathlineto{\pgfqpoint{7.079013in}{0.550000in}}%
\pgfpathlineto{\pgfqpoint{7.079013in}{0.593915in}}%
\pgfpathlineto{\pgfqpoint{7.051491in}{0.593915in}}%
\pgfpathlineto{\pgfqpoint{7.051491in}{0.550000in}}%
\pgfpathclose%
\pgfusepath{fill}%
\end{pgfscope}%
\begin{pgfscope}%
\pgfpathrectangle{\pgfqpoint{1.250000in}{0.550000in}}{\pgfqpoint{7.750000in}{3.850000in}}%
\pgfusepath{clip}%
\pgfsetbuttcap%
\pgfsetmiterjoin%
\definecolor{currentfill}{rgb}{0.501961,0.501961,0.501961}%
\pgfsetfillcolor{currentfill}%
\pgfsetfillopacity{0.700000}%
\pgfsetlinewidth{0.000000pt}%
\definecolor{currentstroke}{rgb}{0.000000,0.000000,0.000000}%
\pgfsetstrokecolor{currentstroke}%
\pgfsetstrokeopacity{0.700000}%
\pgfsetdash{}{0pt}%
\pgfpathmoveto{\pgfqpoint{7.079013in}{0.550000in}}%
\pgfpathlineto{\pgfqpoint{7.106534in}{0.550000in}}%
\pgfpathlineto{\pgfqpoint{7.106534in}{0.601431in}}%
\pgfpathlineto{\pgfqpoint{7.079013in}{0.601431in}}%
\pgfpathlineto{\pgfqpoint{7.079013in}{0.550000in}}%
\pgfpathclose%
\pgfusepath{fill}%
\end{pgfscope}%
\begin{pgfscope}%
\pgfpathrectangle{\pgfqpoint{1.250000in}{0.550000in}}{\pgfqpoint{7.750000in}{3.850000in}}%
\pgfusepath{clip}%
\pgfsetbuttcap%
\pgfsetmiterjoin%
\definecolor{currentfill}{rgb}{0.501961,0.501961,0.501961}%
\pgfsetfillcolor{currentfill}%
\pgfsetfillopacity{0.700000}%
\pgfsetlinewidth{0.000000pt}%
\definecolor{currentstroke}{rgb}{0.000000,0.000000,0.000000}%
\pgfsetstrokecolor{currentstroke}%
\pgfsetstrokeopacity{0.700000}%
\pgfsetdash{}{0pt}%
\pgfpathmoveto{\pgfqpoint{7.106534in}{0.550000in}}%
\pgfpathlineto{\pgfqpoint{7.134055in}{0.550000in}}%
\pgfpathlineto{\pgfqpoint{7.134055in}{0.594310in}}%
\pgfpathlineto{\pgfqpoint{7.106534in}{0.594310in}}%
\pgfpathlineto{\pgfqpoint{7.106534in}{0.550000in}}%
\pgfpathclose%
\pgfusepath{fill}%
\end{pgfscope}%
\begin{pgfscope}%
\pgfpathrectangle{\pgfqpoint{1.250000in}{0.550000in}}{\pgfqpoint{7.750000in}{3.850000in}}%
\pgfusepath{clip}%
\pgfsetbuttcap%
\pgfsetmiterjoin%
\definecolor{currentfill}{rgb}{0.501961,0.501961,0.501961}%
\pgfsetfillcolor{currentfill}%
\pgfsetfillopacity{0.700000}%
\pgfsetlinewidth{0.000000pt}%
\definecolor{currentstroke}{rgb}{0.000000,0.000000,0.000000}%
\pgfsetstrokecolor{currentstroke}%
\pgfsetstrokeopacity{0.700000}%
\pgfsetdash{}{0pt}%
\pgfpathmoveto{\pgfqpoint{7.134055in}{0.550000in}}%
\pgfpathlineto{\pgfqpoint{7.161577in}{0.550000in}}%
\pgfpathlineto{\pgfqpoint{7.161577in}{0.550000in}}%
\pgfpathlineto{\pgfqpoint{7.134055in}{0.550000in}}%
\pgfpathlineto{\pgfqpoint{7.134055in}{0.550000in}}%
\pgfpathclose%
\pgfusepath{fill}%
\end{pgfscope}%
\begin{pgfscope}%
\pgfpathrectangle{\pgfqpoint{1.250000in}{0.550000in}}{\pgfqpoint{7.750000in}{3.850000in}}%
\pgfusepath{clip}%
\pgfsetbuttcap%
\pgfsetmiterjoin%
\definecolor{currentfill}{rgb}{0.501961,0.501961,0.501961}%
\pgfsetfillcolor{currentfill}%
\pgfsetfillopacity{0.700000}%
\pgfsetlinewidth{0.000000pt}%
\definecolor{currentstroke}{rgb}{0.000000,0.000000,0.000000}%
\pgfsetstrokecolor{currentstroke}%
\pgfsetstrokeopacity{0.700000}%
\pgfsetdash{}{0pt}%
\pgfpathmoveto{\pgfqpoint{7.161577in}{0.550000in}}%
\pgfpathlineto{\pgfqpoint{7.189098in}{0.550000in}}%
\pgfpathlineto{\pgfqpoint{7.189098in}{0.600245in}}%
\pgfpathlineto{\pgfqpoint{7.161577in}{0.600245in}}%
\pgfpathlineto{\pgfqpoint{7.161577in}{0.550000in}}%
\pgfpathclose%
\pgfusepath{fill}%
\end{pgfscope}%
\begin{pgfscope}%
\pgfpathrectangle{\pgfqpoint{1.250000in}{0.550000in}}{\pgfqpoint{7.750000in}{3.850000in}}%
\pgfusepath{clip}%
\pgfsetbuttcap%
\pgfsetmiterjoin%
\definecolor{currentfill}{rgb}{0.501961,0.501961,0.501961}%
\pgfsetfillcolor{currentfill}%
\pgfsetfillopacity{0.700000}%
\pgfsetlinewidth{0.000000pt}%
\definecolor{currentstroke}{rgb}{0.000000,0.000000,0.000000}%
\pgfsetstrokecolor{currentstroke}%
\pgfsetstrokeopacity{0.700000}%
\pgfsetdash{}{0pt}%
\pgfpathmoveto{\pgfqpoint{7.189098in}{0.550000in}}%
\pgfpathlineto{\pgfqpoint{7.216619in}{0.550000in}}%
\pgfpathlineto{\pgfqpoint{7.216619in}{0.596288in}}%
\pgfpathlineto{\pgfqpoint{7.189098in}{0.596288in}}%
\pgfpathlineto{\pgfqpoint{7.189098in}{0.550000in}}%
\pgfpathclose%
\pgfusepath{fill}%
\end{pgfscope}%
\begin{pgfscope}%
\pgfpathrectangle{\pgfqpoint{1.250000in}{0.550000in}}{\pgfqpoint{7.750000in}{3.850000in}}%
\pgfusepath{clip}%
\pgfsetbuttcap%
\pgfsetmiterjoin%
\definecolor{currentfill}{rgb}{0.501961,0.501961,0.501961}%
\pgfsetfillcolor{currentfill}%
\pgfsetfillopacity{0.700000}%
\pgfsetlinewidth{0.000000pt}%
\definecolor{currentstroke}{rgb}{0.000000,0.000000,0.000000}%
\pgfsetstrokecolor{currentstroke}%
\pgfsetstrokeopacity{0.700000}%
\pgfsetdash{}{0pt}%
\pgfpathmoveto{\pgfqpoint{7.216619in}{0.550000in}}%
\pgfpathlineto{\pgfqpoint{7.244141in}{0.550000in}}%
\pgfpathlineto{\pgfqpoint{7.244141in}{0.591936in}}%
\pgfpathlineto{\pgfqpoint{7.216619in}{0.591936in}}%
\pgfpathlineto{\pgfqpoint{7.216619in}{0.550000in}}%
\pgfpathclose%
\pgfusepath{fill}%
\end{pgfscope}%
\begin{pgfscope}%
\pgfpathrectangle{\pgfqpoint{1.250000in}{0.550000in}}{\pgfqpoint{7.750000in}{3.850000in}}%
\pgfusepath{clip}%
\pgfsetbuttcap%
\pgfsetmiterjoin%
\definecolor{currentfill}{rgb}{0.501961,0.501961,0.501961}%
\pgfsetfillcolor{currentfill}%
\pgfsetfillopacity{0.700000}%
\pgfsetlinewidth{0.000000pt}%
\definecolor{currentstroke}{rgb}{0.000000,0.000000,0.000000}%
\pgfsetstrokecolor{currentstroke}%
\pgfsetstrokeopacity{0.700000}%
\pgfsetdash{}{0pt}%
\pgfpathmoveto{\pgfqpoint{7.244141in}{0.550000in}}%
\pgfpathlineto{\pgfqpoint{7.271662in}{0.550000in}}%
\pgfpathlineto{\pgfqpoint{7.271662in}{0.593519in}}%
\pgfpathlineto{\pgfqpoint{7.244141in}{0.593519in}}%
\pgfpathlineto{\pgfqpoint{7.244141in}{0.550000in}}%
\pgfpathclose%
\pgfusepath{fill}%
\end{pgfscope}%
\begin{pgfscope}%
\pgfpathrectangle{\pgfqpoint{1.250000in}{0.550000in}}{\pgfqpoint{7.750000in}{3.850000in}}%
\pgfusepath{clip}%
\pgfsetbuttcap%
\pgfsetmiterjoin%
\definecolor{currentfill}{rgb}{0.501961,0.501961,0.501961}%
\pgfsetfillcolor{currentfill}%
\pgfsetfillopacity{0.700000}%
\pgfsetlinewidth{0.000000pt}%
\definecolor{currentstroke}{rgb}{0.000000,0.000000,0.000000}%
\pgfsetstrokecolor{currentstroke}%
\pgfsetstrokeopacity{0.700000}%
\pgfsetdash{}{0pt}%
\pgfpathmoveto{\pgfqpoint{7.271662in}{0.550000in}}%
\pgfpathlineto{\pgfqpoint{7.299183in}{0.550000in}}%
\pgfpathlineto{\pgfqpoint{7.299183in}{0.590750in}}%
\pgfpathlineto{\pgfqpoint{7.271662in}{0.590750in}}%
\pgfpathlineto{\pgfqpoint{7.271662in}{0.550000in}}%
\pgfpathclose%
\pgfusepath{fill}%
\end{pgfscope}%
\begin{pgfscope}%
\pgfpathrectangle{\pgfqpoint{1.250000in}{0.550000in}}{\pgfqpoint{7.750000in}{3.850000in}}%
\pgfusepath{clip}%
\pgfsetbuttcap%
\pgfsetmiterjoin%
\definecolor{currentfill}{rgb}{0.501961,0.501961,0.501961}%
\pgfsetfillcolor{currentfill}%
\pgfsetfillopacity{0.700000}%
\pgfsetlinewidth{0.000000pt}%
\definecolor{currentstroke}{rgb}{0.000000,0.000000,0.000000}%
\pgfsetstrokecolor{currentstroke}%
\pgfsetstrokeopacity{0.700000}%
\pgfsetdash{}{0pt}%
\pgfpathmoveto{\pgfqpoint{7.299183in}{0.550000in}}%
\pgfpathlineto{\pgfqpoint{7.326705in}{0.550000in}}%
\pgfpathlineto{\pgfqpoint{7.326705in}{0.586793in}}%
\pgfpathlineto{\pgfqpoint{7.299183in}{0.586793in}}%
\pgfpathlineto{\pgfqpoint{7.299183in}{0.550000in}}%
\pgfpathclose%
\pgfusepath{fill}%
\end{pgfscope}%
\begin{pgfscope}%
\pgfpathrectangle{\pgfqpoint{1.250000in}{0.550000in}}{\pgfqpoint{7.750000in}{3.850000in}}%
\pgfusepath{clip}%
\pgfsetbuttcap%
\pgfsetmiterjoin%
\definecolor{currentfill}{rgb}{0.501961,0.501961,0.501961}%
\pgfsetfillcolor{currentfill}%
\pgfsetfillopacity{0.700000}%
\pgfsetlinewidth{0.000000pt}%
\definecolor{currentstroke}{rgb}{0.000000,0.000000,0.000000}%
\pgfsetstrokecolor{currentstroke}%
\pgfsetstrokeopacity{0.700000}%
\pgfsetdash{}{0pt}%
\pgfpathmoveto{\pgfqpoint{7.326705in}{0.550000in}}%
\pgfpathlineto{\pgfqpoint{7.354226in}{0.550000in}}%
\pgfpathlineto{\pgfqpoint{7.354226in}{0.592332in}}%
\pgfpathlineto{\pgfqpoint{7.326705in}{0.592332in}}%
\pgfpathlineto{\pgfqpoint{7.326705in}{0.550000in}}%
\pgfpathclose%
\pgfusepath{fill}%
\end{pgfscope}%
\begin{pgfscope}%
\pgfpathrectangle{\pgfqpoint{1.250000in}{0.550000in}}{\pgfqpoint{7.750000in}{3.850000in}}%
\pgfusepath{clip}%
\pgfsetbuttcap%
\pgfsetmiterjoin%
\definecolor{currentfill}{rgb}{0.501961,0.501961,0.501961}%
\pgfsetfillcolor{currentfill}%
\pgfsetfillopacity{0.700000}%
\pgfsetlinewidth{0.000000pt}%
\definecolor{currentstroke}{rgb}{0.000000,0.000000,0.000000}%
\pgfsetstrokecolor{currentstroke}%
\pgfsetstrokeopacity{0.700000}%
\pgfsetdash{}{0pt}%
\pgfpathmoveto{\pgfqpoint{7.354226in}{0.550000in}}%
\pgfpathlineto{\pgfqpoint{7.381747in}{0.550000in}}%
\pgfpathlineto{\pgfqpoint{7.381747in}{0.592728in}}%
\pgfpathlineto{\pgfqpoint{7.354226in}{0.592728in}}%
\pgfpathlineto{\pgfqpoint{7.354226in}{0.550000in}}%
\pgfpathclose%
\pgfusepath{fill}%
\end{pgfscope}%
\begin{pgfscope}%
\pgfpathrectangle{\pgfqpoint{1.250000in}{0.550000in}}{\pgfqpoint{7.750000in}{3.850000in}}%
\pgfusepath{clip}%
\pgfsetbuttcap%
\pgfsetmiterjoin%
\definecolor{currentfill}{rgb}{0.501961,0.501961,0.501961}%
\pgfsetfillcolor{currentfill}%
\pgfsetfillopacity{0.700000}%
\pgfsetlinewidth{0.000000pt}%
\definecolor{currentstroke}{rgb}{0.000000,0.000000,0.000000}%
\pgfsetstrokecolor{currentstroke}%
\pgfsetstrokeopacity{0.700000}%
\pgfsetdash{}{0pt}%
\pgfpathmoveto{\pgfqpoint{7.381747in}{0.550000in}}%
\pgfpathlineto{\pgfqpoint{7.409268in}{0.550000in}}%
\pgfpathlineto{\pgfqpoint{7.409268in}{0.593915in}}%
\pgfpathlineto{\pgfqpoint{7.381747in}{0.593915in}}%
\pgfpathlineto{\pgfqpoint{7.381747in}{0.550000in}}%
\pgfpathclose%
\pgfusepath{fill}%
\end{pgfscope}%
\begin{pgfscope}%
\pgfpathrectangle{\pgfqpoint{1.250000in}{0.550000in}}{\pgfqpoint{7.750000in}{3.850000in}}%
\pgfusepath{clip}%
\pgfsetbuttcap%
\pgfsetmiterjoin%
\definecolor{currentfill}{rgb}{0.501961,0.501961,0.501961}%
\pgfsetfillcolor{currentfill}%
\pgfsetfillopacity{0.700000}%
\pgfsetlinewidth{0.000000pt}%
\definecolor{currentstroke}{rgb}{0.000000,0.000000,0.000000}%
\pgfsetstrokecolor{currentstroke}%
\pgfsetstrokeopacity{0.700000}%
\pgfsetdash{}{0pt}%
\pgfpathmoveto{\pgfqpoint{7.409268in}{0.550000in}}%
\pgfpathlineto{\pgfqpoint{7.436790in}{0.550000in}}%
\pgfpathlineto{\pgfqpoint{7.436790in}{0.598662in}}%
\pgfpathlineto{\pgfqpoint{7.409268in}{0.598662in}}%
\pgfpathlineto{\pgfqpoint{7.409268in}{0.550000in}}%
\pgfpathclose%
\pgfusepath{fill}%
\end{pgfscope}%
\begin{pgfscope}%
\pgfpathrectangle{\pgfqpoint{1.250000in}{0.550000in}}{\pgfqpoint{7.750000in}{3.850000in}}%
\pgfusepath{clip}%
\pgfsetbuttcap%
\pgfsetmiterjoin%
\definecolor{currentfill}{rgb}{0.501961,0.501961,0.501961}%
\pgfsetfillcolor{currentfill}%
\pgfsetfillopacity{0.700000}%
\pgfsetlinewidth{0.000000pt}%
\definecolor{currentstroke}{rgb}{0.000000,0.000000,0.000000}%
\pgfsetstrokecolor{currentstroke}%
\pgfsetstrokeopacity{0.700000}%
\pgfsetdash{}{0pt}%
\pgfpathmoveto{\pgfqpoint{7.436790in}{0.550000in}}%
\pgfpathlineto{\pgfqpoint{7.464311in}{0.550000in}}%
\pgfpathlineto{\pgfqpoint{7.464311in}{0.550000in}}%
\pgfpathlineto{\pgfqpoint{7.436790in}{0.550000in}}%
\pgfpathlineto{\pgfqpoint{7.436790in}{0.550000in}}%
\pgfpathclose%
\pgfusepath{fill}%
\end{pgfscope}%
\begin{pgfscope}%
\pgfpathrectangle{\pgfqpoint{1.250000in}{0.550000in}}{\pgfqpoint{7.750000in}{3.850000in}}%
\pgfusepath{clip}%
\pgfsetbuttcap%
\pgfsetmiterjoin%
\definecolor{currentfill}{rgb}{0.501961,0.501961,0.501961}%
\pgfsetfillcolor{currentfill}%
\pgfsetfillopacity{0.700000}%
\pgfsetlinewidth{0.000000pt}%
\definecolor{currentstroke}{rgb}{0.000000,0.000000,0.000000}%
\pgfsetstrokecolor{currentstroke}%
\pgfsetstrokeopacity{0.700000}%
\pgfsetdash{}{0pt}%
\pgfpathmoveto{\pgfqpoint{7.464311in}{0.550000in}}%
\pgfpathlineto{\pgfqpoint{7.491832in}{0.550000in}}%
\pgfpathlineto{\pgfqpoint{7.491832in}{0.598266in}}%
\pgfpathlineto{\pgfqpoint{7.464311in}{0.598266in}}%
\pgfpathlineto{\pgfqpoint{7.464311in}{0.550000in}}%
\pgfpathclose%
\pgfusepath{fill}%
\end{pgfscope}%
\begin{pgfscope}%
\pgfpathrectangle{\pgfqpoint{1.250000in}{0.550000in}}{\pgfqpoint{7.750000in}{3.850000in}}%
\pgfusepath{clip}%
\pgfsetbuttcap%
\pgfsetmiterjoin%
\definecolor{currentfill}{rgb}{0.501961,0.501961,0.501961}%
\pgfsetfillcolor{currentfill}%
\pgfsetfillopacity{0.700000}%
\pgfsetlinewidth{0.000000pt}%
\definecolor{currentstroke}{rgb}{0.000000,0.000000,0.000000}%
\pgfsetstrokecolor{currentstroke}%
\pgfsetstrokeopacity{0.700000}%
\pgfsetdash{}{0pt}%
\pgfpathmoveto{\pgfqpoint{7.491832in}{0.550000in}}%
\pgfpathlineto{\pgfqpoint{7.519354in}{0.550000in}}%
\pgfpathlineto{\pgfqpoint{7.519354in}{0.596684in}}%
\pgfpathlineto{\pgfqpoint{7.491832in}{0.596684in}}%
\pgfpathlineto{\pgfqpoint{7.491832in}{0.550000in}}%
\pgfpathclose%
\pgfusepath{fill}%
\end{pgfscope}%
\begin{pgfscope}%
\pgfpathrectangle{\pgfqpoint{1.250000in}{0.550000in}}{\pgfqpoint{7.750000in}{3.850000in}}%
\pgfusepath{clip}%
\pgfsetbuttcap%
\pgfsetmiterjoin%
\definecolor{currentfill}{rgb}{0.501961,0.501961,0.501961}%
\pgfsetfillcolor{currentfill}%
\pgfsetfillopacity{0.700000}%
\pgfsetlinewidth{0.000000pt}%
\definecolor{currentstroke}{rgb}{0.000000,0.000000,0.000000}%
\pgfsetstrokecolor{currentstroke}%
\pgfsetstrokeopacity{0.700000}%
\pgfsetdash{}{0pt}%
\pgfpathmoveto{\pgfqpoint{7.519354in}{0.550000in}}%
\pgfpathlineto{\pgfqpoint{7.546875in}{0.550000in}}%
\pgfpathlineto{\pgfqpoint{7.546875in}{0.589563in}}%
\pgfpathlineto{\pgfqpoint{7.519354in}{0.589563in}}%
\pgfpathlineto{\pgfqpoint{7.519354in}{0.550000in}}%
\pgfpathclose%
\pgfusepath{fill}%
\end{pgfscope}%
\begin{pgfscope}%
\pgfpathrectangle{\pgfqpoint{1.250000in}{0.550000in}}{\pgfqpoint{7.750000in}{3.850000in}}%
\pgfusepath{clip}%
\pgfsetbuttcap%
\pgfsetmiterjoin%
\definecolor{currentfill}{rgb}{0.501961,0.501961,0.501961}%
\pgfsetfillcolor{currentfill}%
\pgfsetfillopacity{0.700000}%
\pgfsetlinewidth{0.000000pt}%
\definecolor{currentstroke}{rgb}{0.000000,0.000000,0.000000}%
\pgfsetstrokecolor{currentstroke}%
\pgfsetstrokeopacity{0.700000}%
\pgfsetdash{}{0pt}%
\pgfpathmoveto{\pgfqpoint{7.546875in}{0.550000in}}%
\pgfpathlineto{\pgfqpoint{7.574396in}{0.550000in}}%
\pgfpathlineto{\pgfqpoint{7.574396in}{0.594706in}}%
\pgfpathlineto{\pgfqpoint{7.546875in}{0.594706in}}%
\pgfpathlineto{\pgfqpoint{7.546875in}{0.550000in}}%
\pgfpathclose%
\pgfusepath{fill}%
\end{pgfscope}%
\begin{pgfscope}%
\pgfpathrectangle{\pgfqpoint{1.250000in}{0.550000in}}{\pgfqpoint{7.750000in}{3.850000in}}%
\pgfusepath{clip}%
\pgfsetbuttcap%
\pgfsetmiterjoin%
\definecolor{currentfill}{rgb}{0.501961,0.501961,0.501961}%
\pgfsetfillcolor{currentfill}%
\pgfsetfillopacity{0.700000}%
\pgfsetlinewidth{0.000000pt}%
\definecolor{currentstroke}{rgb}{0.000000,0.000000,0.000000}%
\pgfsetstrokecolor{currentstroke}%
\pgfsetstrokeopacity{0.700000}%
\pgfsetdash{}{0pt}%
\pgfpathmoveto{\pgfqpoint{7.574396in}{0.550000in}}%
\pgfpathlineto{\pgfqpoint{7.601918in}{0.550000in}}%
\pgfpathlineto{\pgfqpoint{7.601918in}{0.595893in}}%
\pgfpathlineto{\pgfqpoint{7.574396in}{0.595893in}}%
\pgfpathlineto{\pgfqpoint{7.574396in}{0.550000in}}%
\pgfpathclose%
\pgfusepath{fill}%
\end{pgfscope}%
\begin{pgfscope}%
\pgfpathrectangle{\pgfqpoint{1.250000in}{0.550000in}}{\pgfqpoint{7.750000in}{3.850000in}}%
\pgfusepath{clip}%
\pgfsetbuttcap%
\pgfsetmiterjoin%
\definecolor{currentfill}{rgb}{0.501961,0.501961,0.501961}%
\pgfsetfillcolor{currentfill}%
\pgfsetfillopacity{0.700000}%
\pgfsetlinewidth{0.000000pt}%
\definecolor{currentstroke}{rgb}{0.000000,0.000000,0.000000}%
\pgfsetstrokecolor{currentstroke}%
\pgfsetstrokeopacity{0.700000}%
\pgfsetdash{}{0pt}%
\pgfpathmoveto{\pgfqpoint{7.601918in}{0.550000in}}%
\pgfpathlineto{\pgfqpoint{7.629439in}{0.550000in}}%
\pgfpathlineto{\pgfqpoint{7.629439in}{0.602223in}}%
\pgfpathlineto{\pgfqpoint{7.601918in}{0.602223in}}%
\pgfpathlineto{\pgfqpoint{7.601918in}{0.550000in}}%
\pgfpathclose%
\pgfusepath{fill}%
\end{pgfscope}%
\begin{pgfscope}%
\pgfpathrectangle{\pgfqpoint{1.250000in}{0.550000in}}{\pgfqpoint{7.750000in}{3.850000in}}%
\pgfusepath{clip}%
\pgfsetbuttcap%
\pgfsetmiterjoin%
\definecolor{currentfill}{rgb}{0.501961,0.501961,0.501961}%
\pgfsetfillcolor{currentfill}%
\pgfsetfillopacity{0.700000}%
\pgfsetlinewidth{0.000000pt}%
\definecolor{currentstroke}{rgb}{0.000000,0.000000,0.000000}%
\pgfsetstrokecolor{currentstroke}%
\pgfsetstrokeopacity{0.700000}%
\pgfsetdash{}{0pt}%
\pgfpathmoveto{\pgfqpoint{7.629439in}{0.550000in}}%
\pgfpathlineto{\pgfqpoint{7.656960in}{0.550000in}}%
\pgfpathlineto{\pgfqpoint{7.656960in}{0.550000in}}%
\pgfpathlineto{\pgfqpoint{7.629439in}{0.550000in}}%
\pgfpathlineto{\pgfqpoint{7.629439in}{0.550000in}}%
\pgfpathclose%
\pgfusepath{fill}%
\end{pgfscope}%
\begin{pgfscope}%
\pgfpathrectangle{\pgfqpoint{1.250000in}{0.550000in}}{\pgfqpoint{7.750000in}{3.850000in}}%
\pgfusepath{clip}%
\pgfsetbuttcap%
\pgfsetmiterjoin%
\definecolor{currentfill}{rgb}{0.501961,0.501961,0.501961}%
\pgfsetfillcolor{currentfill}%
\pgfsetfillopacity{0.700000}%
\pgfsetlinewidth{0.000000pt}%
\definecolor{currentstroke}{rgb}{0.000000,0.000000,0.000000}%
\pgfsetstrokecolor{currentstroke}%
\pgfsetstrokeopacity{0.700000}%
\pgfsetdash{}{0pt}%
\pgfpathmoveto{\pgfqpoint{7.656960in}{0.550000in}}%
\pgfpathlineto{\pgfqpoint{7.684482in}{0.550000in}}%
\pgfpathlineto{\pgfqpoint{7.684482in}{0.599453in}}%
\pgfpathlineto{\pgfqpoint{7.656960in}{0.599453in}}%
\pgfpathlineto{\pgfqpoint{7.656960in}{0.550000in}}%
\pgfpathclose%
\pgfusepath{fill}%
\end{pgfscope}%
\begin{pgfscope}%
\pgfpathrectangle{\pgfqpoint{1.250000in}{0.550000in}}{\pgfqpoint{7.750000in}{3.850000in}}%
\pgfusepath{clip}%
\pgfsetbuttcap%
\pgfsetmiterjoin%
\definecolor{currentfill}{rgb}{0.501961,0.501961,0.501961}%
\pgfsetfillcolor{currentfill}%
\pgfsetfillopacity{0.700000}%
\pgfsetlinewidth{0.000000pt}%
\definecolor{currentstroke}{rgb}{0.000000,0.000000,0.000000}%
\pgfsetstrokecolor{currentstroke}%
\pgfsetstrokeopacity{0.700000}%
\pgfsetdash{}{0pt}%
\pgfpathmoveto{\pgfqpoint{7.684482in}{0.550000in}}%
\pgfpathlineto{\pgfqpoint{7.712003in}{0.550000in}}%
\pgfpathlineto{\pgfqpoint{7.712003in}{0.602223in}}%
\pgfpathlineto{\pgfqpoint{7.684482in}{0.602223in}}%
\pgfpathlineto{\pgfqpoint{7.684482in}{0.550000in}}%
\pgfpathclose%
\pgfusepath{fill}%
\end{pgfscope}%
\begin{pgfscope}%
\pgfpathrectangle{\pgfqpoint{1.250000in}{0.550000in}}{\pgfqpoint{7.750000in}{3.850000in}}%
\pgfusepath{clip}%
\pgfsetbuttcap%
\pgfsetmiterjoin%
\definecolor{currentfill}{rgb}{0.501961,0.501961,0.501961}%
\pgfsetfillcolor{currentfill}%
\pgfsetfillopacity{0.700000}%
\pgfsetlinewidth{0.000000pt}%
\definecolor{currentstroke}{rgb}{0.000000,0.000000,0.000000}%
\pgfsetstrokecolor{currentstroke}%
\pgfsetstrokeopacity{0.700000}%
\pgfsetdash{}{0pt}%
\pgfpathmoveto{\pgfqpoint{7.712003in}{0.550000in}}%
\pgfpathlineto{\pgfqpoint{7.739524in}{0.550000in}}%
\pgfpathlineto{\pgfqpoint{7.739524in}{0.614883in}}%
\pgfpathlineto{\pgfqpoint{7.712003in}{0.614883in}}%
\pgfpathlineto{\pgfqpoint{7.712003in}{0.550000in}}%
\pgfpathclose%
\pgfusepath{fill}%
\end{pgfscope}%
\begin{pgfscope}%
\pgfpathrectangle{\pgfqpoint{1.250000in}{0.550000in}}{\pgfqpoint{7.750000in}{3.850000in}}%
\pgfusepath{clip}%
\pgfsetbuttcap%
\pgfsetmiterjoin%
\definecolor{currentfill}{rgb}{0.501961,0.501961,0.501961}%
\pgfsetfillcolor{currentfill}%
\pgfsetfillopacity{0.700000}%
\pgfsetlinewidth{0.000000pt}%
\definecolor{currentstroke}{rgb}{0.000000,0.000000,0.000000}%
\pgfsetstrokecolor{currentstroke}%
\pgfsetstrokeopacity{0.700000}%
\pgfsetdash{}{0pt}%
\pgfpathmoveto{\pgfqpoint{7.739524in}{0.550000in}}%
\pgfpathlineto{\pgfqpoint{7.767045in}{0.550000in}}%
\pgfpathlineto{\pgfqpoint{7.767045in}{0.550000in}}%
\pgfpathlineto{\pgfqpoint{7.739524in}{0.550000in}}%
\pgfpathlineto{\pgfqpoint{7.739524in}{0.550000in}}%
\pgfpathclose%
\pgfusepath{fill}%
\end{pgfscope}%
\begin{pgfscope}%
\pgfpathrectangle{\pgfqpoint{1.250000in}{0.550000in}}{\pgfqpoint{7.750000in}{3.850000in}}%
\pgfusepath{clip}%
\pgfsetbuttcap%
\pgfsetmiterjoin%
\definecolor{currentfill}{rgb}{0.501961,0.501961,0.501961}%
\pgfsetfillcolor{currentfill}%
\pgfsetfillopacity{0.700000}%
\pgfsetlinewidth{0.000000pt}%
\definecolor{currentstroke}{rgb}{0.000000,0.000000,0.000000}%
\pgfsetstrokecolor{currentstroke}%
\pgfsetstrokeopacity{0.700000}%
\pgfsetdash{}{0pt}%
\pgfpathmoveto{\pgfqpoint{7.767045in}{0.550000in}}%
\pgfpathlineto{\pgfqpoint{7.794567in}{0.550000in}}%
\pgfpathlineto{\pgfqpoint{7.794567in}{0.606970in}}%
\pgfpathlineto{\pgfqpoint{7.767045in}{0.606970in}}%
\pgfpathlineto{\pgfqpoint{7.767045in}{0.550000in}}%
\pgfpathclose%
\pgfusepath{fill}%
\end{pgfscope}%
\begin{pgfscope}%
\pgfpathrectangle{\pgfqpoint{1.250000in}{0.550000in}}{\pgfqpoint{7.750000in}{3.850000in}}%
\pgfusepath{clip}%
\pgfsetbuttcap%
\pgfsetmiterjoin%
\definecolor{currentfill}{rgb}{0.501961,0.501961,0.501961}%
\pgfsetfillcolor{currentfill}%
\pgfsetfillopacity{0.700000}%
\pgfsetlinewidth{0.000000pt}%
\definecolor{currentstroke}{rgb}{0.000000,0.000000,0.000000}%
\pgfsetstrokecolor{currentstroke}%
\pgfsetstrokeopacity{0.700000}%
\pgfsetdash{}{0pt}%
\pgfpathmoveto{\pgfqpoint{7.794567in}{0.550000in}}%
\pgfpathlineto{\pgfqpoint{7.822088in}{0.550000in}}%
\pgfpathlineto{\pgfqpoint{7.822088in}{0.599058in}}%
\pgfpathlineto{\pgfqpoint{7.794567in}{0.599058in}}%
\pgfpathlineto{\pgfqpoint{7.794567in}{0.550000in}}%
\pgfpathclose%
\pgfusepath{fill}%
\end{pgfscope}%
\begin{pgfscope}%
\pgfpathrectangle{\pgfqpoint{1.250000in}{0.550000in}}{\pgfqpoint{7.750000in}{3.850000in}}%
\pgfusepath{clip}%
\pgfsetbuttcap%
\pgfsetmiterjoin%
\definecolor{currentfill}{rgb}{0.501961,0.501961,0.501961}%
\pgfsetfillcolor{currentfill}%
\pgfsetfillopacity{0.700000}%
\pgfsetlinewidth{0.000000pt}%
\definecolor{currentstroke}{rgb}{0.000000,0.000000,0.000000}%
\pgfsetstrokecolor{currentstroke}%
\pgfsetstrokeopacity{0.700000}%
\pgfsetdash{}{0pt}%
\pgfpathmoveto{\pgfqpoint{7.822088in}{0.550000in}}%
\pgfpathlineto{\pgfqpoint{7.849609in}{0.550000in}}%
\pgfpathlineto{\pgfqpoint{7.849609in}{0.595893in}}%
\pgfpathlineto{\pgfqpoint{7.822088in}{0.595893in}}%
\pgfpathlineto{\pgfqpoint{7.822088in}{0.550000in}}%
\pgfpathclose%
\pgfusepath{fill}%
\end{pgfscope}%
\begin{pgfscope}%
\pgfpathrectangle{\pgfqpoint{1.250000in}{0.550000in}}{\pgfqpoint{7.750000in}{3.850000in}}%
\pgfusepath{clip}%
\pgfsetbuttcap%
\pgfsetmiterjoin%
\definecolor{currentfill}{rgb}{0.501961,0.501961,0.501961}%
\pgfsetfillcolor{currentfill}%
\pgfsetfillopacity{0.700000}%
\pgfsetlinewidth{0.000000pt}%
\definecolor{currentstroke}{rgb}{0.000000,0.000000,0.000000}%
\pgfsetstrokecolor{currentstroke}%
\pgfsetstrokeopacity{0.700000}%
\pgfsetdash{}{0pt}%
\pgfpathmoveto{\pgfqpoint{7.849609in}{0.550000in}}%
\pgfpathlineto{\pgfqpoint{7.877131in}{0.550000in}}%
\pgfpathlineto{\pgfqpoint{7.877131in}{0.592332in}}%
\pgfpathlineto{\pgfqpoint{7.849609in}{0.592332in}}%
\pgfpathlineto{\pgfqpoint{7.849609in}{0.550000in}}%
\pgfpathclose%
\pgfusepath{fill}%
\end{pgfscope}%
\begin{pgfscope}%
\pgfpathrectangle{\pgfqpoint{1.250000in}{0.550000in}}{\pgfqpoint{7.750000in}{3.850000in}}%
\pgfusepath{clip}%
\pgfsetbuttcap%
\pgfsetmiterjoin%
\definecolor{currentfill}{rgb}{0.501961,0.501961,0.501961}%
\pgfsetfillcolor{currentfill}%
\pgfsetfillopacity{0.700000}%
\pgfsetlinewidth{0.000000pt}%
\definecolor{currentstroke}{rgb}{0.000000,0.000000,0.000000}%
\pgfsetstrokecolor{currentstroke}%
\pgfsetstrokeopacity{0.700000}%
\pgfsetdash{}{0pt}%
\pgfpathmoveto{\pgfqpoint{7.877131in}{0.550000in}}%
\pgfpathlineto{\pgfqpoint{7.904652in}{0.550000in}}%
\pgfpathlineto{\pgfqpoint{7.904652in}{0.593519in}}%
\pgfpathlineto{\pgfqpoint{7.877131in}{0.593519in}}%
\pgfpathlineto{\pgfqpoint{7.877131in}{0.550000in}}%
\pgfpathclose%
\pgfusepath{fill}%
\end{pgfscope}%
\begin{pgfscope}%
\pgfpathrectangle{\pgfqpoint{1.250000in}{0.550000in}}{\pgfqpoint{7.750000in}{3.850000in}}%
\pgfusepath{clip}%
\pgfsetbuttcap%
\pgfsetmiterjoin%
\definecolor{currentfill}{rgb}{0.501961,0.501961,0.501961}%
\pgfsetfillcolor{currentfill}%
\pgfsetfillopacity{0.700000}%
\pgfsetlinewidth{0.000000pt}%
\definecolor{currentstroke}{rgb}{0.000000,0.000000,0.000000}%
\pgfsetstrokecolor{currentstroke}%
\pgfsetstrokeopacity{0.700000}%
\pgfsetdash{}{0pt}%
\pgfpathmoveto{\pgfqpoint{7.904652in}{0.550000in}}%
\pgfpathlineto{\pgfqpoint{7.932173in}{0.550000in}}%
\pgfpathlineto{\pgfqpoint{7.932173in}{0.588376in}}%
\pgfpathlineto{\pgfqpoint{7.904652in}{0.588376in}}%
\pgfpathlineto{\pgfqpoint{7.904652in}{0.550000in}}%
\pgfpathclose%
\pgfusepath{fill}%
\end{pgfscope}%
\begin{pgfscope}%
\pgfpathrectangle{\pgfqpoint{1.250000in}{0.550000in}}{\pgfqpoint{7.750000in}{3.850000in}}%
\pgfusepath{clip}%
\pgfsetbuttcap%
\pgfsetmiterjoin%
\definecolor{currentfill}{rgb}{0.501961,0.501961,0.501961}%
\pgfsetfillcolor{currentfill}%
\pgfsetfillopacity{0.700000}%
\pgfsetlinewidth{0.000000pt}%
\definecolor{currentstroke}{rgb}{0.000000,0.000000,0.000000}%
\pgfsetstrokecolor{currentstroke}%
\pgfsetstrokeopacity{0.700000}%
\pgfsetdash{}{0pt}%
\pgfpathmoveto{\pgfqpoint{7.932173in}{0.550000in}}%
\pgfpathlineto{\pgfqpoint{7.959695in}{0.550000in}}%
\pgfpathlineto{\pgfqpoint{7.959695in}{0.593519in}}%
\pgfpathlineto{\pgfqpoint{7.932173in}{0.593519in}}%
\pgfpathlineto{\pgfqpoint{7.932173in}{0.550000in}}%
\pgfpathclose%
\pgfusepath{fill}%
\end{pgfscope}%
\begin{pgfscope}%
\pgfpathrectangle{\pgfqpoint{1.250000in}{0.550000in}}{\pgfqpoint{7.750000in}{3.850000in}}%
\pgfusepath{clip}%
\pgfsetbuttcap%
\pgfsetmiterjoin%
\definecolor{currentfill}{rgb}{0.501961,0.501961,0.501961}%
\pgfsetfillcolor{currentfill}%
\pgfsetfillopacity{0.700000}%
\pgfsetlinewidth{0.000000pt}%
\definecolor{currentstroke}{rgb}{0.000000,0.000000,0.000000}%
\pgfsetstrokecolor{currentstroke}%
\pgfsetstrokeopacity{0.700000}%
\pgfsetdash{}{0pt}%
\pgfpathmoveto{\pgfqpoint{7.959695in}{0.550000in}}%
\pgfpathlineto{\pgfqpoint{7.987216in}{0.550000in}}%
\pgfpathlineto{\pgfqpoint{7.987216in}{0.597080in}}%
\pgfpathlineto{\pgfqpoint{7.959695in}{0.597080in}}%
\pgfpathlineto{\pgfqpoint{7.959695in}{0.550000in}}%
\pgfpathclose%
\pgfusepath{fill}%
\end{pgfscope}%
\begin{pgfscope}%
\pgfpathrectangle{\pgfqpoint{1.250000in}{0.550000in}}{\pgfqpoint{7.750000in}{3.850000in}}%
\pgfusepath{clip}%
\pgfsetbuttcap%
\pgfsetmiterjoin%
\definecolor{currentfill}{rgb}{0.501961,0.501961,0.501961}%
\pgfsetfillcolor{currentfill}%
\pgfsetfillopacity{0.700000}%
\pgfsetlinewidth{0.000000pt}%
\definecolor{currentstroke}{rgb}{0.000000,0.000000,0.000000}%
\pgfsetstrokecolor{currentstroke}%
\pgfsetstrokeopacity{0.700000}%
\pgfsetdash{}{0pt}%
\pgfpathmoveto{\pgfqpoint{7.987216in}{0.550000in}}%
\pgfpathlineto{\pgfqpoint{8.014737in}{0.550000in}}%
\pgfpathlineto{\pgfqpoint{8.014737in}{0.550000in}}%
\pgfpathlineto{\pgfqpoint{7.987216in}{0.550000in}}%
\pgfpathlineto{\pgfqpoint{7.987216in}{0.550000in}}%
\pgfpathclose%
\pgfusepath{fill}%
\end{pgfscope}%
\begin{pgfscope}%
\pgfpathrectangle{\pgfqpoint{1.250000in}{0.550000in}}{\pgfqpoint{7.750000in}{3.850000in}}%
\pgfusepath{clip}%
\pgfsetbuttcap%
\pgfsetmiterjoin%
\definecolor{currentfill}{rgb}{0.501961,0.501961,0.501961}%
\pgfsetfillcolor{currentfill}%
\pgfsetfillopacity{0.700000}%
\pgfsetlinewidth{0.000000pt}%
\definecolor{currentstroke}{rgb}{0.000000,0.000000,0.000000}%
\pgfsetstrokecolor{currentstroke}%
\pgfsetstrokeopacity{0.700000}%
\pgfsetdash{}{0pt}%
\pgfpathmoveto{\pgfqpoint{8.014737in}{0.550000in}}%
\pgfpathlineto{\pgfqpoint{8.042259in}{0.550000in}}%
\pgfpathlineto{\pgfqpoint{8.042259in}{0.597871in}}%
\pgfpathlineto{\pgfqpoint{8.014737in}{0.597871in}}%
\pgfpathlineto{\pgfqpoint{8.014737in}{0.550000in}}%
\pgfpathclose%
\pgfusepath{fill}%
\end{pgfscope}%
\begin{pgfscope}%
\pgfpathrectangle{\pgfqpoint{1.250000in}{0.550000in}}{\pgfqpoint{7.750000in}{3.850000in}}%
\pgfusepath{clip}%
\pgfsetbuttcap%
\pgfsetmiterjoin%
\definecolor{currentfill}{rgb}{0.501961,0.501961,0.501961}%
\pgfsetfillcolor{currentfill}%
\pgfsetfillopacity{0.700000}%
\pgfsetlinewidth{0.000000pt}%
\definecolor{currentstroke}{rgb}{0.000000,0.000000,0.000000}%
\pgfsetstrokecolor{currentstroke}%
\pgfsetstrokeopacity{0.700000}%
\pgfsetdash{}{0pt}%
\pgfpathmoveto{\pgfqpoint{8.042259in}{0.550000in}}%
\pgfpathlineto{\pgfqpoint{8.069780in}{0.550000in}}%
\pgfpathlineto{\pgfqpoint{8.069780in}{0.594706in}}%
\pgfpathlineto{\pgfqpoint{8.042259in}{0.594706in}}%
\pgfpathlineto{\pgfqpoint{8.042259in}{0.550000in}}%
\pgfpathclose%
\pgfusepath{fill}%
\end{pgfscope}%
\begin{pgfscope}%
\pgfpathrectangle{\pgfqpoint{1.250000in}{0.550000in}}{\pgfqpoint{7.750000in}{3.850000in}}%
\pgfusepath{clip}%
\pgfsetbuttcap%
\pgfsetmiterjoin%
\definecolor{currentfill}{rgb}{0.501961,0.501961,0.501961}%
\pgfsetfillcolor{currentfill}%
\pgfsetfillopacity{0.700000}%
\pgfsetlinewidth{0.000000pt}%
\definecolor{currentstroke}{rgb}{0.000000,0.000000,0.000000}%
\pgfsetstrokecolor{currentstroke}%
\pgfsetstrokeopacity{0.700000}%
\pgfsetdash{}{0pt}%
\pgfpathmoveto{\pgfqpoint{8.069780in}{0.550000in}}%
\pgfpathlineto{\pgfqpoint{8.097301in}{0.550000in}}%
\pgfpathlineto{\pgfqpoint{8.097301in}{0.589958in}}%
\pgfpathlineto{\pgfqpoint{8.069780in}{0.589958in}}%
\pgfpathlineto{\pgfqpoint{8.069780in}{0.550000in}}%
\pgfpathclose%
\pgfusepath{fill}%
\end{pgfscope}%
\begin{pgfscope}%
\pgfpathrectangle{\pgfqpoint{1.250000in}{0.550000in}}{\pgfqpoint{7.750000in}{3.850000in}}%
\pgfusepath{clip}%
\pgfsetbuttcap%
\pgfsetmiterjoin%
\definecolor{currentfill}{rgb}{0.501961,0.501961,0.501961}%
\pgfsetfillcolor{currentfill}%
\pgfsetfillopacity{0.700000}%
\pgfsetlinewidth{0.000000pt}%
\definecolor{currentstroke}{rgb}{0.000000,0.000000,0.000000}%
\pgfsetstrokecolor{currentstroke}%
\pgfsetstrokeopacity{0.700000}%
\pgfsetdash{}{0pt}%
\pgfpathmoveto{\pgfqpoint{8.097301in}{0.550000in}}%
\pgfpathlineto{\pgfqpoint{8.124822in}{0.550000in}}%
\pgfpathlineto{\pgfqpoint{8.124822in}{0.597475in}}%
\pgfpathlineto{\pgfqpoint{8.097301in}{0.597475in}}%
\pgfpathlineto{\pgfqpoint{8.097301in}{0.550000in}}%
\pgfpathclose%
\pgfusepath{fill}%
\end{pgfscope}%
\begin{pgfscope}%
\pgfpathrectangle{\pgfqpoint{1.250000in}{0.550000in}}{\pgfqpoint{7.750000in}{3.850000in}}%
\pgfusepath{clip}%
\pgfsetbuttcap%
\pgfsetmiterjoin%
\definecolor{currentfill}{rgb}{0.501961,0.501961,0.501961}%
\pgfsetfillcolor{currentfill}%
\pgfsetfillopacity{0.700000}%
\pgfsetlinewidth{0.000000pt}%
\definecolor{currentstroke}{rgb}{0.000000,0.000000,0.000000}%
\pgfsetstrokecolor{currentstroke}%
\pgfsetstrokeopacity{0.700000}%
\pgfsetdash{}{0pt}%
\pgfpathmoveto{\pgfqpoint{8.124822in}{0.550000in}}%
\pgfpathlineto{\pgfqpoint{8.152344in}{0.550000in}}%
\pgfpathlineto{\pgfqpoint{8.152344in}{0.589563in}}%
\pgfpathlineto{\pgfqpoint{8.124822in}{0.589563in}}%
\pgfpathlineto{\pgfqpoint{8.124822in}{0.550000in}}%
\pgfpathclose%
\pgfusepath{fill}%
\end{pgfscope}%
\begin{pgfscope}%
\pgfpathrectangle{\pgfqpoint{1.250000in}{0.550000in}}{\pgfqpoint{7.750000in}{3.850000in}}%
\pgfusepath{clip}%
\pgfsetbuttcap%
\pgfsetmiterjoin%
\definecolor{currentfill}{rgb}{0.501961,0.501961,0.501961}%
\pgfsetfillcolor{currentfill}%
\pgfsetfillopacity{0.700000}%
\pgfsetlinewidth{0.000000pt}%
\definecolor{currentstroke}{rgb}{0.000000,0.000000,0.000000}%
\pgfsetstrokecolor{currentstroke}%
\pgfsetstrokeopacity{0.700000}%
\pgfsetdash{}{0pt}%
\pgfpathmoveto{\pgfqpoint{8.152344in}{0.550000in}}%
\pgfpathlineto{\pgfqpoint{8.179865in}{0.550000in}}%
\pgfpathlineto{\pgfqpoint{8.179865in}{0.590750in}}%
\pgfpathlineto{\pgfqpoint{8.152344in}{0.590750in}}%
\pgfpathlineto{\pgfqpoint{8.152344in}{0.550000in}}%
\pgfpathclose%
\pgfusepath{fill}%
\end{pgfscope}%
\begin{pgfscope}%
\pgfpathrectangle{\pgfqpoint{1.250000in}{0.550000in}}{\pgfqpoint{7.750000in}{3.850000in}}%
\pgfusepath{clip}%
\pgfsetbuttcap%
\pgfsetmiterjoin%
\definecolor{currentfill}{rgb}{0.501961,0.501961,0.501961}%
\pgfsetfillcolor{currentfill}%
\pgfsetfillopacity{0.700000}%
\pgfsetlinewidth{0.000000pt}%
\definecolor{currentstroke}{rgb}{0.000000,0.000000,0.000000}%
\pgfsetstrokecolor{currentstroke}%
\pgfsetstrokeopacity{0.700000}%
\pgfsetdash{}{0pt}%
\pgfpathmoveto{\pgfqpoint{8.179865in}{0.550000in}}%
\pgfpathlineto{\pgfqpoint{8.207386in}{0.550000in}}%
\pgfpathlineto{\pgfqpoint{8.207386in}{0.582046in}}%
\pgfpathlineto{\pgfqpoint{8.179865in}{0.582046in}}%
\pgfpathlineto{\pgfqpoint{8.179865in}{0.550000in}}%
\pgfpathclose%
\pgfusepath{fill}%
\end{pgfscope}%
\begin{pgfscope}%
\pgfpathrectangle{\pgfqpoint{1.250000in}{0.550000in}}{\pgfqpoint{7.750000in}{3.850000in}}%
\pgfusepath{clip}%
\pgfsetbuttcap%
\pgfsetmiterjoin%
\definecolor{currentfill}{rgb}{0.501961,0.501961,0.501961}%
\pgfsetfillcolor{currentfill}%
\pgfsetfillopacity{0.700000}%
\pgfsetlinewidth{0.000000pt}%
\definecolor{currentstroke}{rgb}{0.000000,0.000000,0.000000}%
\pgfsetstrokecolor{currentstroke}%
\pgfsetstrokeopacity{0.700000}%
\pgfsetdash{}{0pt}%
\pgfpathmoveto{\pgfqpoint{8.207386in}{0.550000in}}%
\pgfpathlineto{\pgfqpoint{8.234908in}{0.550000in}}%
\pgfpathlineto{\pgfqpoint{8.234908in}{0.585606in}}%
\pgfpathlineto{\pgfqpoint{8.207386in}{0.585606in}}%
\pgfpathlineto{\pgfqpoint{8.207386in}{0.550000in}}%
\pgfpathclose%
\pgfusepath{fill}%
\end{pgfscope}%
\begin{pgfscope}%
\pgfpathrectangle{\pgfqpoint{1.250000in}{0.550000in}}{\pgfqpoint{7.750000in}{3.850000in}}%
\pgfusepath{clip}%
\pgfsetbuttcap%
\pgfsetmiterjoin%
\definecolor{currentfill}{rgb}{0.501961,0.501961,0.501961}%
\pgfsetfillcolor{currentfill}%
\pgfsetfillopacity{0.700000}%
\pgfsetlinewidth{0.000000pt}%
\definecolor{currentstroke}{rgb}{0.000000,0.000000,0.000000}%
\pgfsetstrokecolor{currentstroke}%
\pgfsetstrokeopacity{0.700000}%
\pgfsetdash{}{0pt}%
\pgfpathmoveto{\pgfqpoint{8.234908in}{0.550000in}}%
\pgfpathlineto{\pgfqpoint{8.262429in}{0.550000in}}%
\pgfpathlineto{\pgfqpoint{8.262429in}{0.605388in}}%
\pgfpathlineto{\pgfqpoint{8.234908in}{0.605388in}}%
\pgfpathlineto{\pgfqpoint{8.234908in}{0.550000in}}%
\pgfpathclose%
\pgfusepath{fill}%
\end{pgfscope}%
\begin{pgfscope}%
\pgfpathrectangle{\pgfqpoint{1.250000in}{0.550000in}}{\pgfqpoint{7.750000in}{3.850000in}}%
\pgfusepath{clip}%
\pgfsetbuttcap%
\pgfsetmiterjoin%
\definecolor{currentfill}{rgb}{0.501961,0.501961,0.501961}%
\pgfsetfillcolor{currentfill}%
\pgfsetfillopacity{0.700000}%
\pgfsetlinewidth{0.000000pt}%
\definecolor{currentstroke}{rgb}{0.000000,0.000000,0.000000}%
\pgfsetstrokecolor{currentstroke}%
\pgfsetstrokeopacity{0.700000}%
\pgfsetdash{}{0pt}%
\pgfpathmoveto{\pgfqpoint{8.262429in}{0.550000in}}%
\pgfpathlineto{\pgfqpoint{8.289950in}{0.550000in}}%
\pgfpathlineto{\pgfqpoint{8.289950in}{0.581650in}}%
\pgfpathlineto{\pgfqpoint{8.262429in}{0.581650in}}%
\pgfpathlineto{\pgfqpoint{8.262429in}{0.550000in}}%
\pgfpathclose%
\pgfusepath{fill}%
\end{pgfscope}%
\begin{pgfscope}%
\pgfpathrectangle{\pgfqpoint{1.250000in}{0.550000in}}{\pgfqpoint{7.750000in}{3.850000in}}%
\pgfusepath{clip}%
\pgfsetbuttcap%
\pgfsetmiterjoin%
\definecolor{currentfill}{rgb}{0.501961,0.501961,0.501961}%
\pgfsetfillcolor{currentfill}%
\pgfsetfillopacity{0.700000}%
\pgfsetlinewidth{0.000000pt}%
\definecolor{currentstroke}{rgb}{0.000000,0.000000,0.000000}%
\pgfsetstrokecolor{currentstroke}%
\pgfsetstrokeopacity{0.700000}%
\pgfsetdash{}{0pt}%
\pgfpathmoveto{\pgfqpoint{8.289950in}{0.550000in}}%
\pgfpathlineto{\pgfqpoint{8.317472in}{0.550000in}}%
\pgfpathlineto{\pgfqpoint{8.317472in}{0.579276in}}%
\pgfpathlineto{\pgfqpoint{8.289950in}{0.579276in}}%
\pgfpathlineto{\pgfqpoint{8.289950in}{0.550000in}}%
\pgfpathclose%
\pgfusepath{fill}%
\end{pgfscope}%
\begin{pgfscope}%
\pgfpathrectangle{\pgfqpoint{1.250000in}{0.550000in}}{\pgfqpoint{7.750000in}{3.850000in}}%
\pgfusepath{clip}%
\pgfsetbuttcap%
\pgfsetmiterjoin%
\definecolor{currentfill}{rgb}{0.501961,0.501961,0.501961}%
\pgfsetfillcolor{currentfill}%
\pgfsetfillopacity{0.700000}%
\pgfsetlinewidth{0.000000pt}%
\definecolor{currentstroke}{rgb}{0.000000,0.000000,0.000000}%
\pgfsetstrokecolor{currentstroke}%
\pgfsetstrokeopacity{0.700000}%
\pgfsetdash{}{0pt}%
\pgfpathmoveto{\pgfqpoint{8.317472in}{0.550000in}}%
\pgfpathlineto{\pgfqpoint{8.344993in}{0.550000in}}%
\pgfpathlineto{\pgfqpoint{8.344993in}{0.597871in}}%
\pgfpathlineto{\pgfqpoint{8.317472in}{0.597871in}}%
\pgfpathlineto{\pgfqpoint{8.317472in}{0.550000in}}%
\pgfpathclose%
\pgfusepath{fill}%
\end{pgfscope}%
\begin{pgfscope}%
\pgfpathrectangle{\pgfqpoint{1.250000in}{0.550000in}}{\pgfqpoint{7.750000in}{3.850000in}}%
\pgfusepath{clip}%
\pgfsetbuttcap%
\pgfsetmiterjoin%
\definecolor{currentfill}{rgb}{0.501961,0.501961,0.501961}%
\pgfsetfillcolor{currentfill}%
\pgfsetfillopacity{0.700000}%
\pgfsetlinewidth{0.000000pt}%
\definecolor{currentstroke}{rgb}{0.000000,0.000000,0.000000}%
\pgfsetstrokecolor{currentstroke}%
\pgfsetstrokeopacity{0.700000}%
\pgfsetdash{}{0pt}%
\pgfpathmoveto{\pgfqpoint{8.344993in}{0.550000in}}%
\pgfpathlineto{\pgfqpoint{8.372514in}{0.550000in}}%
\pgfpathlineto{\pgfqpoint{8.372514in}{0.590354in}}%
\pgfpathlineto{\pgfqpoint{8.344993in}{0.590354in}}%
\pgfpathlineto{\pgfqpoint{8.344993in}{0.550000in}}%
\pgfpathclose%
\pgfusepath{fill}%
\end{pgfscope}%
\begin{pgfscope}%
\pgfpathrectangle{\pgfqpoint{1.250000in}{0.550000in}}{\pgfqpoint{7.750000in}{3.850000in}}%
\pgfusepath{clip}%
\pgfsetbuttcap%
\pgfsetmiterjoin%
\definecolor{currentfill}{rgb}{0.501961,0.501961,0.501961}%
\pgfsetfillcolor{currentfill}%
\pgfsetfillopacity{0.700000}%
\pgfsetlinewidth{0.000000pt}%
\definecolor{currentstroke}{rgb}{0.000000,0.000000,0.000000}%
\pgfsetstrokecolor{currentstroke}%
\pgfsetstrokeopacity{0.700000}%
\pgfsetdash{}{0pt}%
\pgfpathmoveto{\pgfqpoint{8.372514in}{0.550000in}}%
\pgfpathlineto{\pgfqpoint{8.400036in}{0.550000in}}%
\pgfpathlineto{\pgfqpoint{8.400036in}{0.586398in}}%
\pgfpathlineto{\pgfqpoint{8.372514in}{0.586398in}}%
\pgfpathlineto{\pgfqpoint{8.372514in}{0.550000in}}%
\pgfpathclose%
\pgfusepath{fill}%
\end{pgfscope}%
\begin{pgfscope}%
\pgfpathrectangle{\pgfqpoint{1.250000in}{0.550000in}}{\pgfqpoint{7.750000in}{3.850000in}}%
\pgfusepath{clip}%
\pgfsetbuttcap%
\pgfsetmiterjoin%
\definecolor{currentfill}{rgb}{0.501961,0.501961,0.501961}%
\pgfsetfillcolor{currentfill}%
\pgfsetfillopacity{0.700000}%
\pgfsetlinewidth{0.000000pt}%
\definecolor{currentstroke}{rgb}{0.000000,0.000000,0.000000}%
\pgfsetstrokecolor{currentstroke}%
\pgfsetstrokeopacity{0.700000}%
\pgfsetdash{}{0pt}%
\pgfpathmoveto{\pgfqpoint{8.400036in}{0.550000in}}%
\pgfpathlineto{\pgfqpoint{8.427557in}{0.550000in}}%
\pgfpathlineto{\pgfqpoint{8.427557in}{0.595497in}}%
\pgfpathlineto{\pgfqpoint{8.400036in}{0.595497in}}%
\pgfpathlineto{\pgfqpoint{8.400036in}{0.550000in}}%
\pgfpathclose%
\pgfusepath{fill}%
\end{pgfscope}%
\begin{pgfscope}%
\pgfpathrectangle{\pgfqpoint{1.250000in}{0.550000in}}{\pgfqpoint{7.750000in}{3.850000in}}%
\pgfusepath{clip}%
\pgfsetbuttcap%
\pgfsetmiterjoin%
\definecolor{currentfill}{rgb}{0.501961,0.501961,0.501961}%
\pgfsetfillcolor{currentfill}%
\pgfsetfillopacity{0.700000}%
\pgfsetlinewidth{0.000000pt}%
\definecolor{currentstroke}{rgb}{0.000000,0.000000,0.000000}%
\pgfsetstrokecolor{currentstroke}%
\pgfsetstrokeopacity{0.700000}%
\pgfsetdash{}{0pt}%
\pgfpathmoveto{\pgfqpoint{8.427557in}{0.550000in}}%
\pgfpathlineto{\pgfqpoint{8.455078in}{0.550000in}}%
\pgfpathlineto{\pgfqpoint{8.455078in}{0.586398in}}%
\pgfpathlineto{\pgfqpoint{8.427557in}{0.586398in}}%
\pgfpathlineto{\pgfqpoint{8.427557in}{0.550000in}}%
\pgfpathclose%
\pgfusepath{fill}%
\end{pgfscope}%
\begin{pgfscope}%
\pgfpathrectangle{\pgfqpoint{1.250000in}{0.550000in}}{\pgfqpoint{7.750000in}{3.850000in}}%
\pgfusepath{clip}%
\pgfsetbuttcap%
\pgfsetmiterjoin%
\definecolor{currentfill}{rgb}{0.501961,0.501961,0.501961}%
\pgfsetfillcolor{currentfill}%
\pgfsetfillopacity{0.700000}%
\pgfsetlinewidth{0.000000pt}%
\definecolor{currentstroke}{rgb}{0.000000,0.000000,0.000000}%
\pgfsetstrokecolor{currentstroke}%
\pgfsetstrokeopacity{0.700000}%
\pgfsetdash{}{0pt}%
\pgfpathmoveto{\pgfqpoint{8.455078in}{0.550000in}}%
\pgfpathlineto{\pgfqpoint{8.482599in}{0.550000in}}%
\pgfpathlineto{\pgfqpoint{8.482599in}{0.593519in}}%
\pgfpathlineto{\pgfqpoint{8.455078in}{0.593519in}}%
\pgfpathlineto{\pgfqpoint{8.455078in}{0.550000in}}%
\pgfpathclose%
\pgfusepath{fill}%
\end{pgfscope}%
\begin{pgfscope}%
\pgfpathrectangle{\pgfqpoint{1.250000in}{0.550000in}}{\pgfqpoint{7.750000in}{3.850000in}}%
\pgfusepath{clip}%
\pgfsetbuttcap%
\pgfsetmiterjoin%
\definecolor{currentfill}{rgb}{0.501961,0.501961,0.501961}%
\pgfsetfillcolor{currentfill}%
\pgfsetfillopacity{0.700000}%
\pgfsetlinewidth{0.000000pt}%
\definecolor{currentstroke}{rgb}{0.000000,0.000000,0.000000}%
\pgfsetstrokecolor{currentstroke}%
\pgfsetstrokeopacity{0.700000}%
\pgfsetdash{}{0pt}%
\pgfpathmoveto{\pgfqpoint{8.482599in}{0.550000in}}%
\pgfpathlineto{\pgfqpoint{8.510121in}{0.550000in}}%
\pgfpathlineto{\pgfqpoint{8.510121in}{0.595497in}}%
\pgfpathlineto{\pgfqpoint{8.482599in}{0.595497in}}%
\pgfpathlineto{\pgfqpoint{8.482599in}{0.550000in}}%
\pgfpathclose%
\pgfusepath{fill}%
\end{pgfscope}%
\begin{pgfscope}%
\pgfpathrectangle{\pgfqpoint{1.250000in}{0.550000in}}{\pgfqpoint{7.750000in}{3.850000in}}%
\pgfusepath{clip}%
\pgfsetbuttcap%
\pgfsetmiterjoin%
\definecolor{currentfill}{rgb}{0.501961,0.501961,0.501961}%
\pgfsetfillcolor{currentfill}%
\pgfsetfillopacity{0.700000}%
\pgfsetlinewidth{0.000000pt}%
\definecolor{currentstroke}{rgb}{0.000000,0.000000,0.000000}%
\pgfsetstrokecolor{currentstroke}%
\pgfsetstrokeopacity{0.700000}%
\pgfsetdash{}{0pt}%
\pgfpathmoveto{\pgfqpoint{8.510121in}{0.550000in}}%
\pgfpathlineto{\pgfqpoint{8.537642in}{0.550000in}}%
\pgfpathlineto{\pgfqpoint{8.537642in}{0.584420in}}%
\pgfpathlineto{\pgfqpoint{8.510121in}{0.584420in}}%
\pgfpathlineto{\pgfqpoint{8.510121in}{0.550000in}}%
\pgfpathclose%
\pgfusepath{fill}%
\end{pgfscope}%
\begin{pgfscope}%
\pgfpathrectangle{\pgfqpoint{1.250000in}{0.550000in}}{\pgfqpoint{7.750000in}{3.850000in}}%
\pgfusepath{clip}%
\pgfsetbuttcap%
\pgfsetmiterjoin%
\definecolor{currentfill}{rgb}{0.501961,0.501961,0.501961}%
\pgfsetfillcolor{currentfill}%
\pgfsetfillopacity{0.700000}%
\pgfsetlinewidth{0.000000pt}%
\definecolor{currentstroke}{rgb}{0.000000,0.000000,0.000000}%
\pgfsetstrokecolor{currentstroke}%
\pgfsetstrokeopacity{0.700000}%
\pgfsetdash{}{0pt}%
\pgfpathmoveto{\pgfqpoint{8.537642in}{0.550000in}}%
\pgfpathlineto{\pgfqpoint{8.565163in}{0.550000in}}%
\pgfpathlineto{\pgfqpoint{8.565163in}{0.586398in}}%
\pgfpathlineto{\pgfqpoint{8.537642in}{0.586398in}}%
\pgfpathlineto{\pgfqpoint{8.537642in}{0.550000in}}%
\pgfpathclose%
\pgfusepath{fill}%
\end{pgfscope}%
\begin{pgfscope}%
\pgfpathrectangle{\pgfqpoint{1.250000in}{0.550000in}}{\pgfqpoint{7.750000in}{3.850000in}}%
\pgfusepath{clip}%
\pgfsetbuttcap%
\pgfsetmiterjoin%
\definecolor{currentfill}{rgb}{0.501961,0.501961,0.501961}%
\pgfsetfillcolor{currentfill}%
\pgfsetfillopacity{0.700000}%
\pgfsetlinewidth{0.000000pt}%
\definecolor{currentstroke}{rgb}{0.000000,0.000000,0.000000}%
\pgfsetstrokecolor{currentstroke}%
\pgfsetstrokeopacity{0.700000}%
\pgfsetdash{}{0pt}%
\pgfpathmoveto{\pgfqpoint{8.565163in}{0.550000in}}%
\pgfpathlineto{\pgfqpoint{8.592685in}{0.550000in}}%
\pgfpathlineto{\pgfqpoint{8.592685in}{0.589958in}}%
\pgfpathlineto{\pgfqpoint{8.565163in}{0.589958in}}%
\pgfpathlineto{\pgfqpoint{8.565163in}{0.550000in}}%
\pgfpathclose%
\pgfusepath{fill}%
\end{pgfscope}%
\begin{pgfscope}%
\pgfpathrectangle{\pgfqpoint{1.250000in}{0.550000in}}{\pgfqpoint{7.750000in}{3.850000in}}%
\pgfusepath{clip}%
\pgfsetbuttcap%
\pgfsetmiterjoin%
\definecolor{currentfill}{rgb}{0.501961,0.501961,0.501961}%
\pgfsetfillcolor{currentfill}%
\pgfsetfillopacity{0.700000}%
\pgfsetlinewidth{0.000000pt}%
\definecolor{currentstroke}{rgb}{0.000000,0.000000,0.000000}%
\pgfsetstrokecolor{currentstroke}%
\pgfsetstrokeopacity{0.700000}%
\pgfsetdash{}{0pt}%
\pgfpathmoveto{\pgfqpoint{8.592685in}{0.550000in}}%
\pgfpathlineto{\pgfqpoint{8.620206in}{0.550000in}}%
\pgfpathlineto{\pgfqpoint{8.620206in}{0.592332in}}%
\pgfpathlineto{\pgfqpoint{8.592685in}{0.592332in}}%
\pgfpathlineto{\pgfqpoint{8.592685in}{0.550000in}}%
\pgfpathclose%
\pgfusepath{fill}%
\end{pgfscope}%
\begin{pgfscope}%
\pgfpathrectangle{\pgfqpoint{1.250000in}{0.550000in}}{\pgfqpoint{7.750000in}{3.850000in}}%
\pgfusepath{clip}%
\pgfsetbuttcap%
\pgfsetmiterjoin%
\definecolor{currentfill}{rgb}{0.501961,0.501961,0.501961}%
\pgfsetfillcolor{currentfill}%
\pgfsetfillopacity{0.700000}%
\pgfsetlinewidth{0.000000pt}%
\definecolor{currentstroke}{rgb}{0.000000,0.000000,0.000000}%
\pgfsetstrokecolor{currentstroke}%
\pgfsetstrokeopacity{0.700000}%
\pgfsetdash{}{0pt}%
\pgfpathmoveto{\pgfqpoint{8.620206in}{0.550000in}}%
\pgfpathlineto{\pgfqpoint{8.647727in}{0.550000in}}%
\pgfpathlineto{\pgfqpoint{8.647727in}{0.573342in}}%
\pgfpathlineto{\pgfqpoint{8.620206in}{0.573342in}}%
\pgfpathlineto{\pgfqpoint{8.620206in}{0.550000in}}%
\pgfpathclose%
\pgfusepath{fill}%
\end{pgfscope}%
\begin{pgfscope}%
\pgfsetbuttcap%
\pgfsetroundjoin%
\definecolor{currentfill}{rgb}{0.000000,0.000000,0.000000}%
\pgfsetfillcolor{currentfill}%
\pgfsetlinewidth{0.803000pt}%
\definecolor{currentstroke}{rgb}{0.000000,0.000000,0.000000}%
\pgfsetstrokecolor{currentstroke}%
\pgfsetdash{}{0pt}%
\pgfsys@defobject{currentmarker}{\pgfqpoint{0.000000in}{-0.048611in}}{\pgfqpoint{0.000000in}{0.000000in}}{%
\pgfpathmoveto{\pgfqpoint{0.000000in}{0.000000in}}%
\pgfpathlineto{\pgfqpoint{0.000000in}{-0.048611in}}%
\pgfusepath{stroke,fill}%
}%
\begin{pgfscope}%
\pgfsys@transformshift{1.602273in}{0.550000in}%
\pgfsys@useobject{currentmarker}{}%
\end{pgfscope}%
\end{pgfscope}%
\begin{pgfscope}%
\definecolor{textcolor}{rgb}{0.000000,0.000000,0.000000}%
\pgfsetstrokecolor{textcolor}%
\pgfsetfillcolor{textcolor}%
\pgftext[x=1.602273in,y=0.452778in,,top]{\color{textcolor}{\sffamily\fontsize{10.000000}{12.000000}\selectfont\catcode`\^=\active\def^{\ifmmode\sp\else\^{}\fi}\catcode`\%=\active\def%{\%}0}}%
\end{pgfscope}%
\begin{pgfscope}%
\pgfsetbuttcap%
\pgfsetroundjoin%
\definecolor{currentfill}{rgb}{0.000000,0.000000,0.000000}%
\pgfsetfillcolor{currentfill}%
\pgfsetlinewidth{0.803000pt}%
\definecolor{currentstroke}{rgb}{0.000000,0.000000,0.000000}%
\pgfsetstrokecolor{currentstroke}%
\pgfsetdash{}{0pt}%
\pgfsys@defobject{currentmarker}{\pgfqpoint{0.000000in}{-0.048611in}}{\pgfqpoint{0.000000in}{0.000000in}}{%
\pgfpathmoveto{\pgfqpoint{0.000000in}{0.000000in}}%
\pgfpathlineto{\pgfqpoint{0.000000in}{-0.048611in}}%
\pgfusepath{stroke,fill}%
}%
\begin{pgfscope}%
\pgfsys@transformshift{2.978338in}{0.550000in}%
\pgfsys@useobject{currentmarker}{}%
\end{pgfscope}%
\end{pgfscope}%
\begin{pgfscope}%
\definecolor{textcolor}{rgb}{0.000000,0.000000,0.000000}%
\pgfsetstrokecolor{textcolor}%
\pgfsetfillcolor{textcolor}%
\pgftext[x=2.978338in,y=0.452778in,,top]{\color{textcolor}{\sffamily\fontsize{10.000000}{12.000000}\selectfont\catcode`\^=\active\def^{\ifmmode\sp\else\^{}\fi}\catcode`\%=\active\def%{\%}50}}%
\end{pgfscope}%
\begin{pgfscope}%
\pgfsetbuttcap%
\pgfsetroundjoin%
\definecolor{currentfill}{rgb}{0.000000,0.000000,0.000000}%
\pgfsetfillcolor{currentfill}%
\pgfsetlinewidth{0.803000pt}%
\definecolor{currentstroke}{rgb}{0.000000,0.000000,0.000000}%
\pgfsetstrokecolor{currentstroke}%
\pgfsetdash{}{0pt}%
\pgfsys@defobject{currentmarker}{\pgfqpoint{0.000000in}{-0.048611in}}{\pgfqpoint{0.000000in}{0.000000in}}{%
\pgfpathmoveto{\pgfqpoint{0.000000in}{0.000000in}}%
\pgfpathlineto{\pgfqpoint{0.000000in}{-0.048611in}}%
\pgfusepath{stroke,fill}%
}%
\begin{pgfscope}%
\pgfsys@transformshift{4.354403in}{0.550000in}%
\pgfsys@useobject{currentmarker}{}%
\end{pgfscope}%
\end{pgfscope}%
\begin{pgfscope}%
\definecolor{textcolor}{rgb}{0.000000,0.000000,0.000000}%
\pgfsetstrokecolor{textcolor}%
\pgfsetfillcolor{textcolor}%
\pgftext[x=4.354403in,y=0.452778in,,top]{\color{textcolor}{\sffamily\fontsize{10.000000}{12.000000}\selectfont\catcode`\^=\active\def^{\ifmmode\sp\else\^{}\fi}\catcode`\%=\active\def%{\%}100}}%
\end{pgfscope}%
\begin{pgfscope}%
\pgfsetbuttcap%
\pgfsetroundjoin%
\definecolor{currentfill}{rgb}{0.000000,0.000000,0.000000}%
\pgfsetfillcolor{currentfill}%
\pgfsetlinewidth{0.803000pt}%
\definecolor{currentstroke}{rgb}{0.000000,0.000000,0.000000}%
\pgfsetstrokecolor{currentstroke}%
\pgfsetdash{}{0pt}%
\pgfsys@defobject{currentmarker}{\pgfqpoint{0.000000in}{-0.048611in}}{\pgfqpoint{0.000000in}{0.000000in}}{%
\pgfpathmoveto{\pgfqpoint{0.000000in}{0.000000in}}%
\pgfpathlineto{\pgfqpoint{0.000000in}{-0.048611in}}%
\pgfusepath{stroke,fill}%
}%
\begin{pgfscope}%
\pgfsys@transformshift{5.730469in}{0.550000in}%
\pgfsys@useobject{currentmarker}{}%
\end{pgfscope}%
\end{pgfscope}%
\begin{pgfscope}%
\definecolor{textcolor}{rgb}{0.000000,0.000000,0.000000}%
\pgfsetstrokecolor{textcolor}%
\pgfsetfillcolor{textcolor}%
\pgftext[x=5.730469in,y=0.452778in,,top]{\color{textcolor}{\sffamily\fontsize{10.000000}{12.000000}\selectfont\catcode`\^=\active\def^{\ifmmode\sp\else\^{}\fi}\catcode`\%=\active\def%{\%}150}}%
\end{pgfscope}%
\begin{pgfscope}%
\pgfsetbuttcap%
\pgfsetroundjoin%
\definecolor{currentfill}{rgb}{0.000000,0.000000,0.000000}%
\pgfsetfillcolor{currentfill}%
\pgfsetlinewidth{0.803000pt}%
\definecolor{currentstroke}{rgb}{0.000000,0.000000,0.000000}%
\pgfsetstrokecolor{currentstroke}%
\pgfsetdash{}{0pt}%
\pgfsys@defobject{currentmarker}{\pgfqpoint{0.000000in}{-0.048611in}}{\pgfqpoint{0.000000in}{0.000000in}}{%
\pgfpathmoveto{\pgfqpoint{0.000000in}{0.000000in}}%
\pgfpathlineto{\pgfqpoint{0.000000in}{-0.048611in}}%
\pgfusepath{stroke,fill}%
}%
\begin{pgfscope}%
\pgfsys@transformshift{7.106534in}{0.550000in}%
\pgfsys@useobject{currentmarker}{}%
\end{pgfscope}%
\end{pgfscope}%
\begin{pgfscope}%
\definecolor{textcolor}{rgb}{0.000000,0.000000,0.000000}%
\pgfsetstrokecolor{textcolor}%
\pgfsetfillcolor{textcolor}%
\pgftext[x=7.106534in,y=0.452778in,,top]{\color{textcolor}{\sffamily\fontsize{10.000000}{12.000000}\selectfont\catcode`\^=\active\def^{\ifmmode\sp\else\^{}\fi}\catcode`\%=\active\def%{\%}200}}%
\end{pgfscope}%
\begin{pgfscope}%
\pgfsetbuttcap%
\pgfsetroundjoin%
\definecolor{currentfill}{rgb}{0.000000,0.000000,0.000000}%
\pgfsetfillcolor{currentfill}%
\pgfsetlinewidth{0.803000pt}%
\definecolor{currentstroke}{rgb}{0.000000,0.000000,0.000000}%
\pgfsetstrokecolor{currentstroke}%
\pgfsetdash{}{0pt}%
\pgfsys@defobject{currentmarker}{\pgfqpoint{0.000000in}{-0.048611in}}{\pgfqpoint{0.000000in}{0.000000in}}{%
\pgfpathmoveto{\pgfqpoint{0.000000in}{0.000000in}}%
\pgfpathlineto{\pgfqpoint{0.000000in}{-0.048611in}}%
\pgfusepath{stroke,fill}%
}%
\begin{pgfscope}%
\pgfsys@transformshift{8.482599in}{0.550000in}%
\pgfsys@useobject{currentmarker}{}%
\end{pgfscope}%
\end{pgfscope}%
\begin{pgfscope}%
\definecolor{textcolor}{rgb}{0.000000,0.000000,0.000000}%
\pgfsetstrokecolor{textcolor}%
\pgfsetfillcolor{textcolor}%
\pgftext[x=8.482599in,y=0.452778in,,top]{\color{textcolor}{\sffamily\fontsize{10.000000}{12.000000}\selectfont\catcode`\^=\active\def^{\ifmmode\sp\else\^{}\fi}\catcode`\%=\active\def%{\%}250}}%
\end{pgfscope}%
\begin{pgfscope}%
\definecolor{textcolor}{rgb}{0.000000,0.000000,0.000000}%
\pgfsetstrokecolor{textcolor}%
\pgfsetfillcolor{textcolor}%
\pgftext[x=5.125000in,y=0.262809in,,top]{\color{textcolor}{\sffamily\fontsize{10.000000}{12.000000}\selectfont\catcode`\^=\active\def^{\ifmmode\sp\else\^{}\fi}\catcode`\%=\active\def%{\%}Intensidad}}%
\end{pgfscope}%
\begin{pgfscope}%
\pgfsetbuttcap%
\pgfsetroundjoin%
\definecolor{currentfill}{rgb}{0.000000,0.000000,0.000000}%
\pgfsetfillcolor{currentfill}%
\pgfsetlinewidth{0.803000pt}%
\definecolor{currentstroke}{rgb}{0.000000,0.000000,0.000000}%
\pgfsetstrokecolor{currentstroke}%
\pgfsetdash{}{0pt}%
\pgfsys@defobject{currentmarker}{\pgfqpoint{-0.048611in}{0.000000in}}{\pgfqpoint{-0.000000in}{0.000000in}}{%
\pgfpathmoveto{\pgfqpoint{-0.000000in}{0.000000in}}%
\pgfpathlineto{\pgfqpoint{-0.048611in}{0.000000in}}%
\pgfusepath{stroke,fill}%
}%
\begin{pgfscope}%
\pgfsys@transformshift{1.250000in}{0.550000in}%
\pgfsys@useobject{currentmarker}{}%
\end{pgfscope}%
\end{pgfscope}%
\begin{pgfscope}%
\definecolor{textcolor}{rgb}{0.000000,0.000000,0.000000}%
\pgfsetstrokecolor{textcolor}%
\pgfsetfillcolor{textcolor}%
\pgftext[x=1.064412in, y=0.497238in, left, base]{\color{textcolor}{\sffamily\fontsize{10.000000}{12.000000}\selectfont\catcode`\^=\active\def^{\ifmmode\sp\else\^{}\fi}\catcode`\%=\active\def%{\%}0}}%
\end{pgfscope}%
\begin{pgfscope}%
\pgfsetbuttcap%
\pgfsetroundjoin%
\definecolor{currentfill}{rgb}{0.000000,0.000000,0.000000}%
\pgfsetfillcolor{currentfill}%
\pgfsetlinewidth{0.803000pt}%
\definecolor{currentstroke}{rgb}{0.000000,0.000000,0.000000}%
\pgfsetstrokecolor{currentstroke}%
\pgfsetdash{}{0pt}%
\pgfsys@defobject{currentmarker}{\pgfqpoint{-0.048611in}{0.000000in}}{\pgfqpoint{-0.000000in}{0.000000in}}{%
\pgfpathmoveto{\pgfqpoint{-0.000000in}{0.000000in}}%
\pgfpathlineto{\pgfqpoint{-0.048611in}{0.000000in}}%
\pgfusepath{stroke,fill}%
}%
\begin{pgfscope}%
\pgfsys@transformshift{1.250000in}{1.341253in}%
\pgfsys@useobject{currentmarker}{}%
\end{pgfscope}%
\end{pgfscope}%
\begin{pgfscope}%
\definecolor{textcolor}{rgb}{0.000000,0.000000,0.000000}%
\pgfsetstrokecolor{textcolor}%
\pgfsetfillcolor{textcolor}%
\pgftext[x=0.799316in, y=1.288492in, left, base]{\color{textcolor}{\sffamily\fontsize{10.000000}{12.000000}\selectfont\catcode`\^=\active\def^{\ifmmode\sp\else\^{}\fi}\catcode`\%=\active\def%{\%}2000}}%
\end{pgfscope}%
\begin{pgfscope}%
\pgfsetbuttcap%
\pgfsetroundjoin%
\definecolor{currentfill}{rgb}{0.000000,0.000000,0.000000}%
\pgfsetfillcolor{currentfill}%
\pgfsetlinewidth{0.803000pt}%
\definecolor{currentstroke}{rgb}{0.000000,0.000000,0.000000}%
\pgfsetstrokecolor{currentstroke}%
\pgfsetdash{}{0pt}%
\pgfsys@defobject{currentmarker}{\pgfqpoint{-0.048611in}{0.000000in}}{\pgfqpoint{-0.000000in}{0.000000in}}{%
\pgfpathmoveto{\pgfqpoint{-0.000000in}{0.000000in}}%
\pgfpathlineto{\pgfqpoint{-0.048611in}{0.000000in}}%
\pgfusepath{stroke,fill}%
}%
\begin{pgfscope}%
\pgfsys@transformshift{1.250000in}{2.132506in}%
\pgfsys@useobject{currentmarker}{}%
\end{pgfscope}%
\end{pgfscope}%
\begin{pgfscope}%
\definecolor{textcolor}{rgb}{0.000000,0.000000,0.000000}%
\pgfsetstrokecolor{textcolor}%
\pgfsetfillcolor{textcolor}%
\pgftext[x=0.799316in, y=2.079745in, left, base]{\color{textcolor}{\sffamily\fontsize{10.000000}{12.000000}\selectfont\catcode`\^=\active\def^{\ifmmode\sp\else\^{}\fi}\catcode`\%=\active\def%{\%}4000}}%
\end{pgfscope}%
\begin{pgfscope}%
\pgfsetbuttcap%
\pgfsetroundjoin%
\definecolor{currentfill}{rgb}{0.000000,0.000000,0.000000}%
\pgfsetfillcolor{currentfill}%
\pgfsetlinewidth{0.803000pt}%
\definecolor{currentstroke}{rgb}{0.000000,0.000000,0.000000}%
\pgfsetstrokecolor{currentstroke}%
\pgfsetdash{}{0pt}%
\pgfsys@defobject{currentmarker}{\pgfqpoint{-0.048611in}{0.000000in}}{\pgfqpoint{-0.000000in}{0.000000in}}{%
\pgfpathmoveto{\pgfqpoint{-0.000000in}{0.000000in}}%
\pgfpathlineto{\pgfqpoint{-0.048611in}{0.000000in}}%
\pgfusepath{stroke,fill}%
}%
\begin{pgfscope}%
\pgfsys@transformshift{1.250000in}{2.923759in}%
\pgfsys@useobject{currentmarker}{}%
\end{pgfscope}%
\end{pgfscope}%
\begin{pgfscope}%
\definecolor{textcolor}{rgb}{0.000000,0.000000,0.000000}%
\pgfsetstrokecolor{textcolor}%
\pgfsetfillcolor{textcolor}%
\pgftext[x=0.799316in, y=2.870998in, left, base]{\color{textcolor}{\sffamily\fontsize{10.000000}{12.000000}\selectfont\catcode`\^=\active\def^{\ifmmode\sp\else\^{}\fi}\catcode`\%=\active\def%{\%}6000}}%
\end{pgfscope}%
\begin{pgfscope}%
\pgfsetbuttcap%
\pgfsetroundjoin%
\definecolor{currentfill}{rgb}{0.000000,0.000000,0.000000}%
\pgfsetfillcolor{currentfill}%
\pgfsetlinewidth{0.803000pt}%
\definecolor{currentstroke}{rgb}{0.000000,0.000000,0.000000}%
\pgfsetstrokecolor{currentstroke}%
\pgfsetdash{}{0pt}%
\pgfsys@defobject{currentmarker}{\pgfqpoint{-0.048611in}{0.000000in}}{\pgfqpoint{-0.000000in}{0.000000in}}{%
\pgfpathmoveto{\pgfqpoint{-0.000000in}{0.000000in}}%
\pgfpathlineto{\pgfqpoint{-0.048611in}{0.000000in}}%
\pgfusepath{stroke,fill}%
}%
\begin{pgfscope}%
\pgfsys@transformshift{1.250000in}{3.715012in}%
\pgfsys@useobject{currentmarker}{}%
\end{pgfscope}%
\end{pgfscope}%
\begin{pgfscope}%
\definecolor{textcolor}{rgb}{0.000000,0.000000,0.000000}%
\pgfsetstrokecolor{textcolor}%
\pgfsetfillcolor{textcolor}%
\pgftext[x=0.799316in, y=3.662251in, left, base]{\color{textcolor}{\sffamily\fontsize{10.000000}{12.000000}\selectfont\catcode`\^=\active\def^{\ifmmode\sp\else\^{}\fi}\catcode`\%=\active\def%{\%}8000}}%
\end{pgfscope}%
\begin{pgfscope}%
\definecolor{textcolor}{rgb}{0.000000,0.000000,0.000000}%
\pgfsetstrokecolor{textcolor}%
\pgfsetfillcolor{textcolor}%
\pgftext[x=0.743761in,y=2.475000in,,bottom,rotate=90.000000]{\color{textcolor}{\sffamily\fontsize{10.000000}{12.000000}\selectfont\catcode`\^=\active\def^{\ifmmode\sp\else\^{}\fi}\catcode`\%=\active\def%{\%}Frecuencia}}%
\end{pgfscope}%
\begin{pgfscope}%
\pgfsetrectcap%
\pgfsetmiterjoin%
\pgfsetlinewidth{0.803000pt}%
\definecolor{currentstroke}{rgb}{0.000000,0.000000,0.000000}%
\pgfsetstrokecolor{currentstroke}%
\pgfsetdash{}{0pt}%
\pgfpathmoveto{\pgfqpoint{1.250000in}{0.550000in}}%
\pgfpathlineto{\pgfqpoint{1.250000in}{4.400000in}}%
\pgfusepath{stroke}%
\end{pgfscope}%
\begin{pgfscope}%
\pgfsetrectcap%
\pgfsetmiterjoin%
\pgfsetlinewidth{0.803000pt}%
\definecolor{currentstroke}{rgb}{0.000000,0.000000,0.000000}%
\pgfsetstrokecolor{currentstroke}%
\pgfsetdash{}{0pt}%
\pgfpathmoveto{\pgfqpoint{9.000000in}{0.550000in}}%
\pgfpathlineto{\pgfqpoint{9.000000in}{4.400000in}}%
\pgfusepath{stroke}%
\end{pgfscope}%
\begin{pgfscope}%
\pgfsetrectcap%
\pgfsetmiterjoin%
\pgfsetlinewidth{0.803000pt}%
\definecolor{currentstroke}{rgb}{0.000000,0.000000,0.000000}%
\pgfsetstrokecolor{currentstroke}%
\pgfsetdash{}{0pt}%
\pgfpathmoveto{\pgfqpoint{1.250000in}{0.550000in}}%
\pgfpathlineto{\pgfqpoint{9.000000in}{0.550000in}}%
\pgfusepath{stroke}%
\end{pgfscope}%
\begin{pgfscope}%
\pgfsetrectcap%
\pgfsetmiterjoin%
\pgfsetlinewidth{0.803000pt}%
\definecolor{currentstroke}{rgb}{0.000000,0.000000,0.000000}%
\pgfsetstrokecolor{currentstroke}%
\pgfsetdash{}{0pt}%
\pgfpathmoveto{\pgfqpoint{1.250000in}{4.400000in}}%
\pgfpathlineto{\pgfqpoint{9.000000in}{4.400000in}}%
\pgfusepath{stroke}%
\end{pgfscope}%
\begin{pgfscope}%
\definecolor{textcolor}{rgb}{0.000000,0.000000,0.000000}%
\pgfsetstrokecolor{textcolor}%
\pgfsetfillcolor{textcolor}%
\pgftext[x=5.125000in,y=4.483333in,,base]{\color{textcolor}{\sffamily\fontsize{12.000000}{14.400000}\selectfont\catcode`\^=\active\def^{\ifmmode\sp\else\^{}\fi}\catcode`\%=\active\def%{\%}Histograma de BHEPL-D}}%
\end{pgfscope}%
\end{pgfpicture}%
\makeatother%
\endgroup%
}
		\caption{BHEPL-D}
		\label{fig:bhepl_1}
	\end{subfigure}

	\caption{Comparación de histogramas para la imagen 2015\_00002}
	\label{fig:histogramas_1}
\end{figure}

\subsubsection{Análisis de imagen 2015\_00026}

Lorem ipsum dolor sit amet, officia excepteur ex fugiat reprehenderit enim labore culpa sint ad
nisi Lorem pariatur mollit ex esse exercitation amet. Nisi anim cupidatat excepteur officia.
Reprehenderit nostrud nostrud ipsum Lorem est aliquip amet voluptate voluptate dolor minim
nulla est proident. Nostrud officia pariatur ut officia. Sit irure elit esse ea nulla sunt ex
occaecat reprehenderit commodo officia dolor Lorem duis laboris cupidatat officia voluptate.
Culpa proident adipisicing id nulla nisi laboris ex in Lorem sunt duis officia eiusmod. Aliqua
reprehenderit commodo ex non excepteur duis sunt velit enim. Voluptate laboris sint cupidatat
ullamco ut ea consectetur et est culpa et culpa duis.

\begin{figure}[H]
	\centering
	\begin{subfigure}[b]{0.45\textwidth}
		\resizebox{\linewidth}{!}{%% Creator: Matplotlib, PGF backend
%%
%% To include the figure in your LaTeX document, write
%%   \input{<filename>.pgf}
%%
%% Make sure the required packages are loaded in your preamble
%%   \usepackage{pgf}
%%
%% Also ensure that all the required font packages are loaded; for instance,
%% the lmodern package is sometimes necessary when using math font.
%%   \usepackage{lmodern}
%%
%% Figures using additional raster images can only be included by \input if
%% they are in the same directory as the main LaTeX file. For loading figures
%% from other directories you can use the `import` package
%%   \usepackage{import}
%%
%% and then include the figures with
%%   \import{<path to file>}{<filename>.pgf}
%%
%% Matplotlib used the following preamble
%%   \def\mathdefault#1{#1}
%%   \everymath=\expandafter{\the\everymath\displaystyle}
%%   \IfFileExists{scrextend.sty}{
%%     \usepackage[fontsize=10.000000pt]{scrextend}
%%   }{
%%     \renewcommand{\normalsize}{\fontsize{10.000000}{12.000000}\selectfont}
%%     \normalsize
%%   }
%%   
%%   \ifdefined\pdftexversion\else  % non-pdftex case.
%%     \usepackage{fontspec}
%%     \setmainfont{DejaVuSerif.ttf}[Path=\detokenize{/home/elias/.virtualenvs/src-pway/lib/python3.13/site-packages/matplotlib/mpl-data/fonts/ttf/}]
%%     \setsansfont{DejaVuSans.ttf}[Path=\detokenize{/home/elias/.virtualenvs/src-pway/lib/python3.13/site-packages/matplotlib/mpl-data/fonts/ttf/}]
%%     \setmonofont{DejaVuSansMono.ttf}[Path=\detokenize{/home/elias/.virtualenvs/src-pway/lib/python3.13/site-packages/matplotlib/mpl-data/fonts/ttf/}]
%%   \fi
%%   \makeatletter\@ifpackageloaded{underscore}{}{\usepackage[strings]{underscore}}\makeatother
%%
\begingroup%
\makeatletter%
\begin{pgfpicture}%
\pgfpathrectangle{\pgfpointorigin}{\pgfqpoint{10.000000in}{5.000000in}}%
\pgfusepath{use as bounding box, clip}%
\begin{pgfscope}%
\pgfsetbuttcap%
\pgfsetmiterjoin%
\definecolor{currentfill}{rgb}{1.000000,1.000000,1.000000}%
\pgfsetfillcolor{currentfill}%
\pgfsetlinewidth{0.000000pt}%
\definecolor{currentstroke}{rgb}{1.000000,1.000000,1.000000}%
\pgfsetstrokecolor{currentstroke}%
\pgfsetdash{}{0pt}%
\pgfpathmoveto{\pgfqpoint{0.000000in}{0.000000in}}%
\pgfpathlineto{\pgfqpoint{10.000000in}{0.000000in}}%
\pgfpathlineto{\pgfqpoint{10.000000in}{5.000000in}}%
\pgfpathlineto{\pgfqpoint{0.000000in}{5.000000in}}%
\pgfpathlineto{\pgfqpoint{0.000000in}{0.000000in}}%
\pgfpathclose%
\pgfusepath{fill}%
\end{pgfscope}%
\begin{pgfscope}%
\pgfsetbuttcap%
\pgfsetmiterjoin%
\definecolor{currentfill}{rgb}{1.000000,1.000000,1.000000}%
\pgfsetfillcolor{currentfill}%
\pgfsetlinewidth{0.000000pt}%
\definecolor{currentstroke}{rgb}{0.000000,0.000000,0.000000}%
\pgfsetstrokecolor{currentstroke}%
\pgfsetstrokeopacity{0.000000}%
\pgfsetdash{}{0pt}%
\pgfpathmoveto{\pgfqpoint{1.250000in}{0.550000in}}%
\pgfpathlineto{\pgfqpoint{9.000000in}{0.550000in}}%
\pgfpathlineto{\pgfqpoint{9.000000in}{4.400000in}}%
\pgfpathlineto{\pgfqpoint{1.250000in}{4.400000in}}%
\pgfpathlineto{\pgfqpoint{1.250000in}{0.550000in}}%
\pgfpathclose%
\pgfusepath{fill}%
\end{pgfscope}%
\begin{pgfscope}%
\pgfpathrectangle{\pgfqpoint{1.250000in}{0.550000in}}{\pgfqpoint{7.750000in}{3.850000in}}%
\pgfusepath{clip}%
\pgfsetbuttcap%
\pgfsetmiterjoin%
\definecolor{currentfill}{rgb}{0.501961,0.501961,0.501961}%
\pgfsetfillcolor{currentfill}%
\pgfsetfillopacity{0.700000}%
\pgfsetlinewidth{0.000000pt}%
\definecolor{currentstroke}{rgb}{0.000000,0.000000,0.000000}%
\pgfsetstrokecolor{currentstroke}%
\pgfsetstrokeopacity{0.700000}%
\pgfsetdash{}{0pt}%
\pgfpathmoveto{\pgfqpoint{1.602273in}{0.550000in}}%
\pgfpathlineto{\pgfqpoint{1.629794in}{0.550000in}}%
\pgfpathlineto{\pgfqpoint{1.629794in}{2.398200in}}%
\pgfpathlineto{\pgfqpoint{1.602273in}{2.398200in}}%
\pgfpathlineto{\pgfqpoint{1.602273in}{0.550000in}}%
\pgfpathclose%
\pgfusepath{fill}%
\end{pgfscope}%
\begin{pgfscope}%
\pgfpathrectangle{\pgfqpoint{1.250000in}{0.550000in}}{\pgfqpoint{7.750000in}{3.850000in}}%
\pgfusepath{clip}%
\pgfsetbuttcap%
\pgfsetmiterjoin%
\definecolor{currentfill}{rgb}{0.501961,0.501961,0.501961}%
\pgfsetfillcolor{currentfill}%
\pgfsetfillopacity{0.700000}%
\pgfsetlinewidth{0.000000pt}%
\definecolor{currentstroke}{rgb}{0.000000,0.000000,0.000000}%
\pgfsetstrokecolor{currentstroke}%
\pgfsetstrokeopacity{0.700000}%
\pgfsetdash{}{0pt}%
\pgfpathmoveto{\pgfqpoint{1.629794in}{0.550000in}}%
\pgfpathlineto{\pgfqpoint{1.657315in}{0.550000in}}%
\pgfpathlineto{\pgfqpoint{1.657315in}{4.216667in}}%
\pgfpathlineto{\pgfqpoint{1.629794in}{4.216667in}}%
\pgfpathlineto{\pgfqpoint{1.629794in}{0.550000in}}%
\pgfpathclose%
\pgfusepath{fill}%
\end{pgfscope}%
\begin{pgfscope}%
\pgfpathrectangle{\pgfqpoint{1.250000in}{0.550000in}}{\pgfqpoint{7.750000in}{3.850000in}}%
\pgfusepath{clip}%
\pgfsetbuttcap%
\pgfsetmiterjoin%
\definecolor{currentfill}{rgb}{0.501961,0.501961,0.501961}%
\pgfsetfillcolor{currentfill}%
\pgfsetfillopacity{0.700000}%
\pgfsetlinewidth{0.000000pt}%
\definecolor{currentstroke}{rgb}{0.000000,0.000000,0.000000}%
\pgfsetstrokecolor{currentstroke}%
\pgfsetstrokeopacity{0.700000}%
\pgfsetdash{}{0pt}%
\pgfpathmoveto{\pgfqpoint{1.657315in}{0.550000in}}%
\pgfpathlineto{\pgfqpoint{1.684837in}{0.550000in}}%
\pgfpathlineto{\pgfqpoint{1.684837in}{3.586192in}}%
\pgfpathlineto{\pgfqpoint{1.657315in}{3.586192in}}%
\pgfpathlineto{\pgfqpoint{1.657315in}{0.550000in}}%
\pgfpathclose%
\pgfusepath{fill}%
\end{pgfscope}%
\begin{pgfscope}%
\pgfpathrectangle{\pgfqpoint{1.250000in}{0.550000in}}{\pgfqpoint{7.750000in}{3.850000in}}%
\pgfusepath{clip}%
\pgfsetbuttcap%
\pgfsetmiterjoin%
\definecolor{currentfill}{rgb}{0.501961,0.501961,0.501961}%
\pgfsetfillcolor{currentfill}%
\pgfsetfillopacity{0.700000}%
\pgfsetlinewidth{0.000000pt}%
\definecolor{currentstroke}{rgb}{0.000000,0.000000,0.000000}%
\pgfsetstrokecolor{currentstroke}%
\pgfsetstrokeopacity{0.700000}%
\pgfsetdash{}{0pt}%
\pgfpathmoveto{\pgfqpoint{1.684837in}{0.550000in}}%
\pgfpathlineto{\pgfqpoint{1.712358in}{0.550000in}}%
\pgfpathlineto{\pgfqpoint{1.712358in}{2.872296in}}%
\pgfpathlineto{\pgfqpoint{1.684837in}{2.872296in}}%
\pgfpathlineto{\pgfqpoint{1.684837in}{0.550000in}}%
\pgfpathclose%
\pgfusepath{fill}%
\end{pgfscope}%
\begin{pgfscope}%
\pgfpathrectangle{\pgfqpoint{1.250000in}{0.550000in}}{\pgfqpoint{7.750000in}{3.850000in}}%
\pgfusepath{clip}%
\pgfsetbuttcap%
\pgfsetmiterjoin%
\definecolor{currentfill}{rgb}{0.501961,0.501961,0.501961}%
\pgfsetfillcolor{currentfill}%
\pgfsetfillopacity{0.700000}%
\pgfsetlinewidth{0.000000pt}%
\definecolor{currentstroke}{rgb}{0.000000,0.000000,0.000000}%
\pgfsetstrokecolor{currentstroke}%
\pgfsetstrokeopacity{0.700000}%
\pgfsetdash{}{0pt}%
\pgfpathmoveto{\pgfqpoint{1.712358in}{0.550000in}}%
\pgfpathlineto{\pgfqpoint{1.739879in}{0.550000in}}%
\pgfpathlineto{\pgfqpoint{1.739879in}{2.257238in}}%
\pgfpathlineto{\pgfqpoint{1.712358in}{2.257238in}}%
\pgfpathlineto{\pgfqpoint{1.712358in}{0.550000in}}%
\pgfpathclose%
\pgfusepath{fill}%
\end{pgfscope}%
\begin{pgfscope}%
\pgfpathrectangle{\pgfqpoint{1.250000in}{0.550000in}}{\pgfqpoint{7.750000in}{3.850000in}}%
\pgfusepath{clip}%
\pgfsetbuttcap%
\pgfsetmiterjoin%
\definecolor{currentfill}{rgb}{0.501961,0.501961,0.501961}%
\pgfsetfillcolor{currentfill}%
\pgfsetfillopacity{0.700000}%
\pgfsetlinewidth{0.000000pt}%
\definecolor{currentstroke}{rgb}{0.000000,0.000000,0.000000}%
\pgfsetstrokecolor{currentstroke}%
\pgfsetstrokeopacity{0.700000}%
\pgfsetdash{}{0pt}%
\pgfpathmoveto{\pgfqpoint{1.739879in}{0.550000in}}%
\pgfpathlineto{\pgfqpoint{1.767401in}{0.550000in}}%
\pgfpathlineto{\pgfqpoint{1.767401in}{1.879229in}}%
\pgfpathlineto{\pgfqpoint{1.739879in}{1.879229in}}%
\pgfpathlineto{\pgfqpoint{1.739879in}{0.550000in}}%
\pgfpathclose%
\pgfusepath{fill}%
\end{pgfscope}%
\begin{pgfscope}%
\pgfpathrectangle{\pgfqpoint{1.250000in}{0.550000in}}{\pgfqpoint{7.750000in}{3.850000in}}%
\pgfusepath{clip}%
\pgfsetbuttcap%
\pgfsetmiterjoin%
\definecolor{currentfill}{rgb}{0.501961,0.501961,0.501961}%
\pgfsetfillcolor{currentfill}%
\pgfsetfillopacity{0.700000}%
\pgfsetlinewidth{0.000000pt}%
\definecolor{currentstroke}{rgb}{0.000000,0.000000,0.000000}%
\pgfsetstrokecolor{currentstroke}%
\pgfsetstrokeopacity{0.700000}%
\pgfsetdash{}{0pt}%
\pgfpathmoveto{\pgfqpoint{1.767401in}{0.550000in}}%
\pgfpathlineto{\pgfqpoint{1.794922in}{0.550000in}}%
\pgfpathlineto{\pgfqpoint{1.794922in}{1.607766in}}%
\pgfpathlineto{\pgfqpoint{1.767401in}{1.607766in}}%
\pgfpathlineto{\pgfqpoint{1.767401in}{0.550000in}}%
\pgfpathclose%
\pgfusepath{fill}%
\end{pgfscope}%
\begin{pgfscope}%
\pgfpathrectangle{\pgfqpoint{1.250000in}{0.550000in}}{\pgfqpoint{7.750000in}{3.850000in}}%
\pgfusepath{clip}%
\pgfsetbuttcap%
\pgfsetmiterjoin%
\definecolor{currentfill}{rgb}{0.501961,0.501961,0.501961}%
\pgfsetfillcolor{currentfill}%
\pgfsetfillopacity{0.700000}%
\pgfsetlinewidth{0.000000pt}%
\definecolor{currentstroke}{rgb}{0.000000,0.000000,0.000000}%
\pgfsetstrokecolor{currentstroke}%
\pgfsetstrokeopacity{0.700000}%
\pgfsetdash{}{0pt}%
\pgfpathmoveto{\pgfqpoint{1.794922in}{0.550000in}}%
\pgfpathlineto{\pgfqpoint{1.822443in}{0.550000in}}%
\pgfpathlineto{\pgfqpoint{1.822443in}{1.512232in}}%
\pgfpathlineto{\pgfqpoint{1.794922in}{1.512232in}}%
\pgfpathlineto{\pgfqpoint{1.794922in}{0.550000in}}%
\pgfpathclose%
\pgfusepath{fill}%
\end{pgfscope}%
\begin{pgfscope}%
\pgfpathrectangle{\pgfqpoint{1.250000in}{0.550000in}}{\pgfqpoint{7.750000in}{3.850000in}}%
\pgfusepath{clip}%
\pgfsetbuttcap%
\pgfsetmiterjoin%
\definecolor{currentfill}{rgb}{0.501961,0.501961,0.501961}%
\pgfsetfillcolor{currentfill}%
\pgfsetfillopacity{0.700000}%
\pgfsetlinewidth{0.000000pt}%
\definecolor{currentstroke}{rgb}{0.000000,0.000000,0.000000}%
\pgfsetstrokecolor{currentstroke}%
\pgfsetstrokeopacity{0.700000}%
\pgfsetdash{}{0pt}%
\pgfpathmoveto{\pgfqpoint{1.822443in}{0.550000in}}%
\pgfpathlineto{\pgfqpoint{1.849964in}{0.550000in}}%
\pgfpathlineto{\pgfqpoint{1.849964in}{1.347317in}}%
\pgfpathlineto{\pgfqpoint{1.822443in}{1.347317in}}%
\pgfpathlineto{\pgfqpoint{1.822443in}{0.550000in}}%
\pgfpathclose%
\pgfusepath{fill}%
\end{pgfscope}%
\begin{pgfscope}%
\pgfpathrectangle{\pgfqpoint{1.250000in}{0.550000in}}{\pgfqpoint{7.750000in}{3.850000in}}%
\pgfusepath{clip}%
\pgfsetbuttcap%
\pgfsetmiterjoin%
\definecolor{currentfill}{rgb}{0.501961,0.501961,0.501961}%
\pgfsetfillcolor{currentfill}%
\pgfsetfillopacity{0.700000}%
\pgfsetlinewidth{0.000000pt}%
\definecolor{currentstroke}{rgb}{0.000000,0.000000,0.000000}%
\pgfsetstrokecolor{currentstroke}%
\pgfsetstrokeopacity{0.700000}%
\pgfsetdash{}{0pt}%
\pgfpathmoveto{\pgfqpoint{1.849964in}{0.550000in}}%
\pgfpathlineto{\pgfqpoint{1.877486in}{0.550000in}}%
\pgfpathlineto{\pgfqpoint{1.877486in}{1.249855in}}%
\pgfpathlineto{\pgfqpoint{1.849964in}{1.249855in}}%
\pgfpathlineto{\pgfqpoint{1.849964in}{0.550000in}}%
\pgfpathclose%
\pgfusepath{fill}%
\end{pgfscope}%
\begin{pgfscope}%
\pgfpathrectangle{\pgfqpoint{1.250000in}{0.550000in}}{\pgfqpoint{7.750000in}{3.850000in}}%
\pgfusepath{clip}%
\pgfsetbuttcap%
\pgfsetmiterjoin%
\definecolor{currentfill}{rgb}{0.501961,0.501961,0.501961}%
\pgfsetfillcolor{currentfill}%
\pgfsetfillopacity{0.700000}%
\pgfsetlinewidth{0.000000pt}%
\definecolor{currentstroke}{rgb}{0.000000,0.000000,0.000000}%
\pgfsetstrokecolor{currentstroke}%
\pgfsetstrokeopacity{0.700000}%
\pgfsetdash{}{0pt}%
\pgfpathmoveto{\pgfqpoint{1.877486in}{0.550000in}}%
\pgfpathlineto{\pgfqpoint{1.905007in}{0.550000in}}%
\pgfpathlineto{\pgfqpoint{1.905007in}{1.151291in}}%
\pgfpathlineto{\pgfqpoint{1.877486in}{1.151291in}}%
\pgfpathlineto{\pgfqpoint{1.877486in}{0.550000in}}%
\pgfpathclose%
\pgfusepath{fill}%
\end{pgfscope}%
\begin{pgfscope}%
\pgfpathrectangle{\pgfqpoint{1.250000in}{0.550000in}}{\pgfqpoint{7.750000in}{3.850000in}}%
\pgfusepath{clip}%
\pgfsetbuttcap%
\pgfsetmiterjoin%
\definecolor{currentfill}{rgb}{0.501961,0.501961,0.501961}%
\pgfsetfillcolor{currentfill}%
\pgfsetfillopacity{0.700000}%
\pgfsetlinewidth{0.000000pt}%
\definecolor{currentstroke}{rgb}{0.000000,0.000000,0.000000}%
\pgfsetstrokecolor{currentstroke}%
\pgfsetstrokeopacity{0.700000}%
\pgfsetdash{}{0pt}%
\pgfpathmoveto{\pgfqpoint{1.905007in}{0.550000in}}%
\pgfpathlineto{\pgfqpoint{1.932528in}{0.550000in}}%
\pgfpathlineto{\pgfqpoint{1.932528in}{1.082187in}}%
\pgfpathlineto{\pgfqpoint{1.905007in}{1.082187in}}%
\pgfpathlineto{\pgfqpoint{1.905007in}{0.550000in}}%
\pgfpathclose%
\pgfusepath{fill}%
\end{pgfscope}%
\begin{pgfscope}%
\pgfpathrectangle{\pgfqpoint{1.250000in}{0.550000in}}{\pgfqpoint{7.750000in}{3.850000in}}%
\pgfusepath{clip}%
\pgfsetbuttcap%
\pgfsetmiterjoin%
\definecolor{currentfill}{rgb}{0.501961,0.501961,0.501961}%
\pgfsetfillcolor{currentfill}%
\pgfsetfillopacity{0.700000}%
\pgfsetlinewidth{0.000000pt}%
\definecolor{currentstroke}{rgb}{0.000000,0.000000,0.000000}%
\pgfsetstrokecolor{currentstroke}%
\pgfsetstrokeopacity{0.700000}%
\pgfsetdash{}{0pt}%
\pgfpathmoveto{\pgfqpoint{1.932528in}{0.550000in}}%
\pgfpathlineto{\pgfqpoint{1.960050in}{0.550000in}}%
\pgfpathlineto{\pgfqpoint{1.960050in}{1.053554in}}%
\pgfpathlineto{\pgfqpoint{1.932528in}{1.053554in}}%
\pgfpathlineto{\pgfqpoint{1.932528in}{0.550000in}}%
\pgfpathclose%
\pgfusepath{fill}%
\end{pgfscope}%
\begin{pgfscope}%
\pgfpathrectangle{\pgfqpoint{1.250000in}{0.550000in}}{\pgfqpoint{7.750000in}{3.850000in}}%
\pgfusepath{clip}%
\pgfsetbuttcap%
\pgfsetmiterjoin%
\definecolor{currentfill}{rgb}{0.501961,0.501961,0.501961}%
\pgfsetfillcolor{currentfill}%
\pgfsetfillopacity{0.700000}%
\pgfsetlinewidth{0.000000pt}%
\definecolor{currentstroke}{rgb}{0.000000,0.000000,0.000000}%
\pgfsetstrokecolor{currentstroke}%
\pgfsetstrokeopacity{0.700000}%
\pgfsetdash{}{0pt}%
\pgfpathmoveto{\pgfqpoint{1.960050in}{0.550000in}}%
\pgfpathlineto{\pgfqpoint{1.987571in}{0.550000in}}%
\pgfpathlineto{\pgfqpoint{1.987571in}{1.011431in}}%
\pgfpathlineto{\pgfqpoint{1.960050in}{1.011431in}}%
\pgfpathlineto{\pgfqpoint{1.960050in}{0.550000in}}%
\pgfpathclose%
\pgfusepath{fill}%
\end{pgfscope}%
\begin{pgfscope}%
\pgfpathrectangle{\pgfqpoint{1.250000in}{0.550000in}}{\pgfqpoint{7.750000in}{3.850000in}}%
\pgfusepath{clip}%
\pgfsetbuttcap%
\pgfsetmiterjoin%
\definecolor{currentfill}{rgb}{0.501961,0.501961,0.501961}%
\pgfsetfillcolor{currentfill}%
\pgfsetfillopacity{0.700000}%
\pgfsetlinewidth{0.000000pt}%
\definecolor{currentstroke}{rgb}{0.000000,0.000000,0.000000}%
\pgfsetstrokecolor{currentstroke}%
\pgfsetstrokeopacity{0.700000}%
\pgfsetdash{}{0pt}%
\pgfpathmoveto{\pgfqpoint{1.987571in}{0.550000in}}%
\pgfpathlineto{\pgfqpoint{2.015092in}{0.550000in}}%
\pgfpathlineto{\pgfqpoint{2.015092in}{0.968757in}}%
\pgfpathlineto{\pgfqpoint{1.987571in}{0.968757in}}%
\pgfpathlineto{\pgfqpoint{1.987571in}{0.550000in}}%
\pgfpathclose%
\pgfusepath{fill}%
\end{pgfscope}%
\begin{pgfscope}%
\pgfpathrectangle{\pgfqpoint{1.250000in}{0.550000in}}{\pgfqpoint{7.750000in}{3.850000in}}%
\pgfusepath{clip}%
\pgfsetbuttcap%
\pgfsetmiterjoin%
\definecolor{currentfill}{rgb}{0.501961,0.501961,0.501961}%
\pgfsetfillcolor{currentfill}%
\pgfsetfillopacity{0.700000}%
\pgfsetlinewidth{0.000000pt}%
\definecolor{currentstroke}{rgb}{0.000000,0.000000,0.000000}%
\pgfsetstrokecolor{currentstroke}%
\pgfsetstrokeopacity{0.700000}%
\pgfsetdash{}{0pt}%
\pgfpathmoveto{\pgfqpoint{2.015092in}{0.550000in}}%
\pgfpathlineto{\pgfqpoint{2.042614in}{0.550000in}}%
\pgfpathlineto{\pgfqpoint{2.042614in}{0.912592in}}%
\pgfpathlineto{\pgfqpoint{2.015092in}{0.912592in}}%
\pgfpathlineto{\pgfqpoint{2.015092in}{0.550000in}}%
\pgfpathclose%
\pgfusepath{fill}%
\end{pgfscope}%
\begin{pgfscope}%
\pgfpathrectangle{\pgfqpoint{1.250000in}{0.550000in}}{\pgfqpoint{7.750000in}{3.850000in}}%
\pgfusepath{clip}%
\pgfsetbuttcap%
\pgfsetmiterjoin%
\definecolor{currentfill}{rgb}{0.501961,0.501961,0.501961}%
\pgfsetfillcolor{currentfill}%
\pgfsetfillopacity{0.700000}%
\pgfsetlinewidth{0.000000pt}%
\definecolor{currentstroke}{rgb}{0.000000,0.000000,0.000000}%
\pgfsetstrokecolor{currentstroke}%
\pgfsetstrokeopacity{0.700000}%
\pgfsetdash{}{0pt}%
\pgfpathmoveto{\pgfqpoint{2.042614in}{0.550000in}}%
\pgfpathlineto{\pgfqpoint{2.070135in}{0.550000in}}%
\pgfpathlineto{\pgfqpoint{2.070135in}{0.906260in}}%
\pgfpathlineto{\pgfqpoint{2.042614in}{0.906260in}}%
\pgfpathlineto{\pgfqpoint{2.042614in}{0.550000in}}%
\pgfpathclose%
\pgfusepath{fill}%
\end{pgfscope}%
\begin{pgfscope}%
\pgfpathrectangle{\pgfqpoint{1.250000in}{0.550000in}}{\pgfqpoint{7.750000in}{3.850000in}}%
\pgfusepath{clip}%
\pgfsetbuttcap%
\pgfsetmiterjoin%
\definecolor{currentfill}{rgb}{0.501961,0.501961,0.501961}%
\pgfsetfillcolor{currentfill}%
\pgfsetfillopacity{0.700000}%
\pgfsetlinewidth{0.000000pt}%
\definecolor{currentstroke}{rgb}{0.000000,0.000000,0.000000}%
\pgfsetstrokecolor{currentstroke}%
\pgfsetstrokeopacity{0.700000}%
\pgfsetdash{}{0pt}%
\pgfpathmoveto{\pgfqpoint{2.070135in}{0.550000in}}%
\pgfpathlineto{\pgfqpoint{2.097656in}{0.550000in}}%
\pgfpathlineto{\pgfqpoint{2.097656in}{0.871019in}}%
\pgfpathlineto{\pgfqpoint{2.070135in}{0.871019in}}%
\pgfpathlineto{\pgfqpoint{2.070135in}{0.550000in}}%
\pgfpathclose%
\pgfusepath{fill}%
\end{pgfscope}%
\begin{pgfscope}%
\pgfpathrectangle{\pgfqpoint{1.250000in}{0.550000in}}{\pgfqpoint{7.750000in}{3.850000in}}%
\pgfusepath{clip}%
\pgfsetbuttcap%
\pgfsetmiterjoin%
\definecolor{currentfill}{rgb}{0.501961,0.501961,0.501961}%
\pgfsetfillcolor{currentfill}%
\pgfsetfillopacity{0.700000}%
\pgfsetlinewidth{0.000000pt}%
\definecolor{currentstroke}{rgb}{0.000000,0.000000,0.000000}%
\pgfsetstrokecolor{currentstroke}%
\pgfsetstrokeopacity{0.700000}%
\pgfsetdash{}{0pt}%
\pgfpathmoveto{\pgfqpoint{2.097656in}{0.550000in}}%
\pgfpathlineto{\pgfqpoint{2.125178in}{0.550000in}}%
\pgfpathlineto{\pgfqpoint{2.125178in}{0.875975in}}%
\pgfpathlineto{\pgfqpoint{2.097656in}{0.875975in}}%
\pgfpathlineto{\pgfqpoint{2.097656in}{0.550000in}}%
\pgfpathclose%
\pgfusepath{fill}%
\end{pgfscope}%
\begin{pgfscope}%
\pgfpathrectangle{\pgfqpoint{1.250000in}{0.550000in}}{\pgfqpoint{7.750000in}{3.850000in}}%
\pgfusepath{clip}%
\pgfsetbuttcap%
\pgfsetmiterjoin%
\definecolor{currentfill}{rgb}{0.501961,0.501961,0.501961}%
\pgfsetfillcolor{currentfill}%
\pgfsetfillopacity{0.700000}%
\pgfsetlinewidth{0.000000pt}%
\definecolor{currentstroke}{rgb}{0.000000,0.000000,0.000000}%
\pgfsetstrokecolor{currentstroke}%
\pgfsetstrokeopacity{0.700000}%
\pgfsetdash{}{0pt}%
\pgfpathmoveto{\pgfqpoint{2.125178in}{0.550000in}}%
\pgfpathlineto{\pgfqpoint{2.152699in}{0.550000in}}%
\pgfpathlineto{\pgfqpoint{2.152699in}{0.870193in}}%
\pgfpathlineto{\pgfqpoint{2.125178in}{0.870193in}}%
\pgfpathlineto{\pgfqpoint{2.125178in}{0.550000in}}%
\pgfpathclose%
\pgfusepath{fill}%
\end{pgfscope}%
\begin{pgfscope}%
\pgfpathrectangle{\pgfqpoint{1.250000in}{0.550000in}}{\pgfqpoint{7.750000in}{3.850000in}}%
\pgfusepath{clip}%
\pgfsetbuttcap%
\pgfsetmiterjoin%
\definecolor{currentfill}{rgb}{0.501961,0.501961,0.501961}%
\pgfsetfillcolor{currentfill}%
\pgfsetfillopacity{0.700000}%
\pgfsetlinewidth{0.000000pt}%
\definecolor{currentstroke}{rgb}{0.000000,0.000000,0.000000}%
\pgfsetstrokecolor{currentstroke}%
\pgfsetstrokeopacity{0.700000}%
\pgfsetdash{}{0pt}%
\pgfpathmoveto{\pgfqpoint{2.152699in}{0.550000in}}%
\pgfpathlineto{\pgfqpoint{2.180220in}{0.550000in}}%
\pgfpathlineto{\pgfqpoint{2.180220in}{0.874598in}}%
\pgfpathlineto{\pgfqpoint{2.152699in}{0.874598in}}%
\pgfpathlineto{\pgfqpoint{2.152699in}{0.550000in}}%
\pgfpathclose%
\pgfusepath{fill}%
\end{pgfscope}%
\begin{pgfscope}%
\pgfpathrectangle{\pgfqpoint{1.250000in}{0.550000in}}{\pgfqpoint{7.750000in}{3.850000in}}%
\pgfusepath{clip}%
\pgfsetbuttcap%
\pgfsetmiterjoin%
\definecolor{currentfill}{rgb}{0.501961,0.501961,0.501961}%
\pgfsetfillcolor{currentfill}%
\pgfsetfillopacity{0.700000}%
\pgfsetlinewidth{0.000000pt}%
\definecolor{currentstroke}{rgb}{0.000000,0.000000,0.000000}%
\pgfsetstrokecolor{currentstroke}%
\pgfsetstrokeopacity{0.700000}%
\pgfsetdash{}{0pt}%
\pgfpathmoveto{\pgfqpoint{2.180220in}{0.550000in}}%
\pgfpathlineto{\pgfqpoint{2.207741in}{0.550000in}}%
\pgfpathlineto{\pgfqpoint{2.207741in}{0.885060in}}%
\pgfpathlineto{\pgfqpoint{2.180220in}{0.885060in}}%
\pgfpathlineto{\pgfqpoint{2.180220in}{0.550000in}}%
\pgfpathclose%
\pgfusepath{fill}%
\end{pgfscope}%
\begin{pgfscope}%
\pgfpathrectangle{\pgfqpoint{1.250000in}{0.550000in}}{\pgfqpoint{7.750000in}{3.850000in}}%
\pgfusepath{clip}%
\pgfsetbuttcap%
\pgfsetmiterjoin%
\definecolor{currentfill}{rgb}{0.501961,0.501961,0.501961}%
\pgfsetfillcolor{currentfill}%
\pgfsetfillopacity{0.700000}%
\pgfsetlinewidth{0.000000pt}%
\definecolor{currentstroke}{rgb}{0.000000,0.000000,0.000000}%
\pgfsetstrokecolor{currentstroke}%
\pgfsetstrokeopacity{0.700000}%
\pgfsetdash{}{0pt}%
\pgfpathmoveto{\pgfqpoint{2.207741in}{0.550000in}}%
\pgfpathlineto{\pgfqpoint{2.235263in}{0.550000in}}%
\pgfpathlineto{\pgfqpoint{2.235263in}{0.871845in}}%
\pgfpathlineto{\pgfqpoint{2.207741in}{0.871845in}}%
\pgfpathlineto{\pgfqpoint{2.207741in}{0.550000in}}%
\pgfpathclose%
\pgfusepath{fill}%
\end{pgfscope}%
\begin{pgfscope}%
\pgfpathrectangle{\pgfqpoint{1.250000in}{0.550000in}}{\pgfqpoint{7.750000in}{3.850000in}}%
\pgfusepath{clip}%
\pgfsetbuttcap%
\pgfsetmiterjoin%
\definecolor{currentfill}{rgb}{0.501961,0.501961,0.501961}%
\pgfsetfillcolor{currentfill}%
\pgfsetfillopacity{0.700000}%
\pgfsetlinewidth{0.000000pt}%
\definecolor{currentstroke}{rgb}{0.000000,0.000000,0.000000}%
\pgfsetstrokecolor{currentstroke}%
\pgfsetstrokeopacity{0.700000}%
\pgfsetdash{}{0pt}%
\pgfpathmoveto{\pgfqpoint{2.235263in}{0.550000in}}%
\pgfpathlineto{\pgfqpoint{2.262784in}{0.550000in}}%
\pgfpathlineto{\pgfqpoint{2.262784in}{0.822839in}}%
\pgfpathlineto{\pgfqpoint{2.235263in}{0.822839in}}%
\pgfpathlineto{\pgfqpoint{2.235263in}{0.550000in}}%
\pgfpathclose%
\pgfusepath{fill}%
\end{pgfscope}%
\begin{pgfscope}%
\pgfpathrectangle{\pgfqpoint{1.250000in}{0.550000in}}{\pgfqpoint{7.750000in}{3.850000in}}%
\pgfusepath{clip}%
\pgfsetbuttcap%
\pgfsetmiterjoin%
\definecolor{currentfill}{rgb}{0.501961,0.501961,0.501961}%
\pgfsetfillcolor{currentfill}%
\pgfsetfillopacity{0.700000}%
\pgfsetlinewidth{0.000000pt}%
\definecolor{currentstroke}{rgb}{0.000000,0.000000,0.000000}%
\pgfsetstrokecolor{currentstroke}%
\pgfsetstrokeopacity{0.700000}%
\pgfsetdash{}{0pt}%
\pgfpathmoveto{\pgfqpoint{2.262784in}{0.550000in}}%
\pgfpathlineto{\pgfqpoint{2.290305in}{0.550000in}}%
\pgfpathlineto{\pgfqpoint{2.290305in}{0.820912in}}%
\pgfpathlineto{\pgfqpoint{2.262784in}{0.820912in}}%
\pgfpathlineto{\pgfqpoint{2.262784in}{0.550000in}}%
\pgfpathclose%
\pgfusepath{fill}%
\end{pgfscope}%
\begin{pgfscope}%
\pgfpathrectangle{\pgfqpoint{1.250000in}{0.550000in}}{\pgfqpoint{7.750000in}{3.850000in}}%
\pgfusepath{clip}%
\pgfsetbuttcap%
\pgfsetmiterjoin%
\definecolor{currentfill}{rgb}{0.501961,0.501961,0.501961}%
\pgfsetfillcolor{currentfill}%
\pgfsetfillopacity{0.700000}%
\pgfsetlinewidth{0.000000pt}%
\definecolor{currentstroke}{rgb}{0.000000,0.000000,0.000000}%
\pgfsetstrokecolor{currentstroke}%
\pgfsetstrokeopacity{0.700000}%
\pgfsetdash{}{0pt}%
\pgfpathmoveto{\pgfqpoint{2.290305in}{0.550000in}}%
\pgfpathlineto{\pgfqpoint{2.317827in}{0.550000in}}%
\pgfpathlineto{\pgfqpoint{2.317827in}{0.812652in}}%
\pgfpathlineto{\pgfqpoint{2.290305in}{0.812652in}}%
\pgfpathlineto{\pgfqpoint{2.290305in}{0.550000in}}%
\pgfpathclose%
\pgfusepath{fill}%
\end{pgfscope}%
\begin{pgfscope}%
\pgfpathrectangle{\pgfqpoint{1.250000in}{0.550000in}}{\pgfqpoint{7.750000in}{3.850000in}}%
\pgfusepath{clip}%
\pgfsetbuttcap%
\pgfsetmiterjoin%
\definecolor{currentfill}{rgb}{0.501961,0.501961,0.501961}%
\pgfsetfillcolor{currentfill}%
\pgfsetfillopacity{0.700000}%
\pgfsetlinewidth{0.000000pt}%
\definecolor{currentstroke}{rgb}{0.000000,0.000000,0.000000}%
\pgfsetstrokecolor{currentstroke}%
\pgfsetstrokeopacity{0.700000}%
\pgfsetdash{}{0pt}%
\pgfpathmoveto{\pgfqpoint{2.317827in}{0.550000in}}%
\pgfpathlineto{\pgfqpoint{2.345348in}{0.550000in}}%
\pgfpathlineto{\pgfqpoint{2.345348in}{0.811000in}}%
\pgfpathlineto{\pgfqpoint{2.317827in}{0.811000in}}%
\pgfpathlineto{\pgfqpoint{2.317827in}{0.550000in}}%
\pgfpathclose%
\pgfusepath{fill}%
\end{pgfscope}%
\begin{pgfscope}%
\pgfpathrectangle{\pgfqpoint{1.250000in}{0.550000in}}{\pgfqpoint{7.750000in}{3.850000in}}%
\pgfusepath{clip}%
\pgfsetbuttcap%
\pgfsetmiterjoin%
\definecolor{currentfill}{rgb}{0.501961,0.501961,0.501961}%
\pgfsetfillcolor{currentfill}%
\pgfsetfillopacity{0.700000}%
\pgfsetlinewidth{0.000000pt}%
\definecolor{currentstroke}{rgb}{0.000000,0.000000,0.000000}%
\pgfsetstrokecolor{currentstroke}%
\pgfsetstrokeopacity{0.700000}%
\pgfsetdash{}{0pt}%
\pgfpathmoveto{\pgfqpoint{2.345348in}{0.550000in}}%
\pgfpathlineto{\pgfqpoint{2.372869in}{0.550000in}}%
\pgfpathlineto{\pgfqpoint{2.372869in}{0.829722in}}%
\pgfpathlineto{\pgfqpoint{2.345348in}{0.829722in}}%
\pgfpathlineto{\pgfqpoint{2.345348in}{0.550000in}}%
\pgfpathclose%
\pgfusepath{fill}%
\end{pgfscope}%
\begin{pgfscope}%
\pgfpathrectangle{\pgfqpoint{1.250000in}{0.550000in}}{\pgfqpoint{7.750000in}{3.850000in}}%
\pgfusepath{clip}%
\pgfsetbuttcap%
\pgfsetmiterjoin%
\definecolor{currentfill}{rgb}{0.501961,0.501961,0.501961}%
\pgfsetfillcolor{currentfill}%
\pgfsetfillopacity{0.700000}%
\pgfsetlinewidth{0.000000pt}%
\definecolor{currentstroke}{rgb}{0.000000,0.000000,0.000000}%
\pgfsetstrokecolor{currentstroke}%
\pgfsetstrokeopacity{0.700000}%
\pgfsetdash{}{0pt}%
\pgfpathmoveto{\pgfqpoint{2.372869in}{0.550000in}}%
\pgfpathlineto{\pgfqpoint{2.400391in}{0.550000in}}%
\pgfpathlineto{\pgfqpoint{2.400391in}{0.808247in}}%
\pgfpathlineto{\pgfqpoint{2.372869in}{0.808247in}}%
\pgfpathlineto{\pgfqpoint{2.372869in}{0.550000in}}%
\pgfpathclose%
\pgfusepath{fill}%
\end{pgfscope}%
\begin{pgfscope}%
\pgfpathrectangle{\pgfqpoint{1.250000in}{0.550000in}}{\pgfqpoint{7.750000in}{3.850000in}}%
\pgfusepath{clip}%
\pgfsetbuttcap%
\pgfsetmiterjoin%
\definecolor{currentfill}{rgb}{0.501961,0.501961,0.501961}%
\pgfsetfillcolor{currentfill}%
\pgfsetfillopacity{0.700000}%
\pgfsetlinewidth{0.000000pt}%
\definecolor{currentstroke}{rgb}{0.000000,0.000000,0.000000}%
\pgfsetstrokecolor{currentstroke}%
\pgfsetstrokeopacity{0.700000}%
\pgfsetdash{}{0pt}%
\pgfpathmoveto{\pgfqpoint{2.400391in}{0.550000in}}%
\pgfpathlineto{\pgfqpoint{2.427912in}{0.550000in}}%
\pgfpathlineto{\pgfqpoint{2.427912in}{0.814304in}}%
\pgfpathlineto{\pgfqpoint{2.400391in}{0.814304in}}%
\pgfpathlineto{\pgfqpoint{2.400391in}{0.550000in}}%
\pgfpathclose%
\pgfusepath{fill}%
\end{pgfscope}%
\begin{pgfscope}%
\pgfpathrectangle{\pgfqpoint{1.250000in}{0.550000in}}{\pgfqpoint{7.750000in}{3.850000in}}%
\pgfusepath{clip}%
\pgfsetbuttcap%
\pgfsetmiterjoin%
\definecolor{currentfill}{rgb}{0.501961,0.501961,0.501961}%
\pgfsetfillcolor{currentfill}%
\pgfsetfillopacity{0.700000}%
\pgfsetlinewidth{0.000000pt}%
\definecolor{currentstroke}{rgb}{0.000000,0.000000,0.000000}%
\pgfsetstrokecolor{currentstroke}%
\pgfsetstrokeopacity{0.700000}%
\pgfsetdash{}{0pt}%
\pgfpathmoveto{\pgfqpoint{2.427912in}{0.550000in}}%
\pgfpathlineto{\pgfqpoint{2.455433in}{0.550000in}}%
\pgfpathlineto{\pgfqpoint{2.455433in}{0.811000in}}%
\pgfpathlineto{\pgfqpoint{2.427912in}{0.811000in}}%
\pgfpathlineto{\pgfqpoint{2.427912in}{0.550000in}}%
\pgfpathclose%
\pgfusepath{fill}%
\end{pgfscope}%
\begin{pgfscope}%
\pgfpathrectangle{\pgfqpoint{1.250000in}{0.550000in}}{\pgfqpoint{7.750000in}{3.850000in}}%
\pgfusepath{clip}%
\pgfsetbuttcap%
\pgfsetmiterjoin%
\definecolor{currentfill}{rgb}{0.501961,0.501961,0.501961}%
\pgfsetfillcolor{currentfill}%
\pgfsetfillopacity{0.700000}%
\pgfsetlinewidth{0.000000pt}%
\definecolor{currentstroke}{rgb}{0.000000,0.000000,0.000000}%
\pgfsetstrokecolor{currentstroke}%
\pgfsetstrokeopacity{0.700000}%
\pgfsetdash{}{0pt}%
\pgfpathmoveto{\pgfqpoint{2.455433in}{0.550000in}}%
\pgfpathlineto{\pgfqpoint{2.482955in}{0.550000in}}%
\pgfpathlineto{\pgfqpoint{2.482955in}{0.792829in}}%
\pgfpathlineto{\pgfqpoint{2.455433in}{0.792829in}}%
\pgfpathlineto{\pgfqpoint{2.455433in}{0.550000in}}%
\pgfpathclose%
\pgfusepath{fill}%
\end{pgfscope}%
\begin{pgfscope}%
\pgfpathrectangle{\pgfqpoint{1.250000in}{0.550000in}}{\pgfqpoint{7.750000in}{3.850000in}}%
\pgfusepath{clip}%
\pgfsetbuttcap%
\pgfsetmiterjoin%
\definecolor{currentfill}{rgb}{0.501961,0.501961,0.501961}%
\pgfsetfillcolor{currentfill}%
\pgfsetfillopacity{0.700000}%
\pgfsetlinewidth{0.000000pt}%
\definecolor{currentstroke}{rgb}{0.000000,0.000000,0.000000}%
\pgfsetstrokecolor{currentstroke}%
\pgfsetstrokeopacity{0.700000}%
\pgfsetdash{}{0pt}%
\pgfpathmoveto{\pgfqpoint{2.482955in}{0.550000in}}%
\pgfpathlineto{\pgfqpoint{2.510476in}{0.550000in}}%
\pgfpathlineto{\pgfqpoint{2.510476in}{0.764747in}}%
\pgfpathlineto{\pgfqpoint{2.482955in}{0.764747in}}%
\pgfpathlineto{\pgfqpoint{2.482955in}{0.550000in}}%
\pgfpathclose%
\pgfusepath{fill}%
\end{pgfscope}%
\begin{pgfscope}%
\pgfpathrectangle{\pgfqpoint{1.250000in}{0.550000in}}{\pgfqpoint{7.750000in}{3.850000in}}%
\pgfusepath{clip}%
\pgfsetbuttcap%
\pgfsetmiterjoin%
\definecolor{currentfill}{rgb}{0.501961,0.501961,0.501961}%
\pgfsetfillcolor{currentfill}%
\pgfsetfillopacity{0.700000}%
\pgfsetlinewidth{0.000000pt}%
\definecolor{currentstroke}{rgb}{0.000000,0.000000,0.000000}%
\pgfsetstrokecolor{currentstroke}%
\pgfsetstrokeopacity{0.700000}%
\pgfsetdash{}{0pt}%
\pgfpathmoveto{\pgfqpoint{2.510476in}{0.550000in}}%
\pgfpathlineto{\pgfqpoint{2.537997in}{0.550000in}}%
\pgfpathlineto{\pgfqpoint{2.537997in}{0.770529in}}%
\pgfpathlineto{\pgfqpoint{2.510476in}{0.770529in}}%
\pgfpathlineto{\pgfqpoint{2.510476in}{0.550000in}}%
\pgfpathclose%
\pgfusepath{fill}%
\end{pgfscope}%
\begin{pgfscope}%
\pgfpathrectangle{\pgfqpoint{1.250000in}{0.550000in}}{\pgfqpoint{7.750000in}{3.850000in}}%
\pgfusepath{clip}%
\pgfsetbuttcap%
\pgfsetmiterjoin%
\definecolor{currentfill}{rgb}{0.501961,0.501961,0.501961}%
\pgfsetfillcolor{currentfill}%
\pgfsetfillopacity{0.700000}%
\pgfsetlinewidth{0.000000pt}%
\definecolor{currentstroke}{rgb}{0.000000,0.000000,0.000000}%
\pgfsetstrokecolor{currentstroke}%
\pgfsetstrokeopacity{0.700000}%
\pgfsetdash{}{0pt}%
\pgfpathmoveto{\pgfqpoint{2.537997in}{0.550000in}}%
\pgfpathlineto{\pgfqpoint{2.565518in}{0.550000in}}%
\pgfpathlineto{\pgfqpoint{2.565518in}{0.755386in}}%
\pgfpathlineto{\pgfqpoint{2.537997in}{0.755386in}}%
\pgfpathlineto{\pgfqpoint{2.537997in}{0.550000in}}%
\pgfpathclose%
\pgfusepath{fill}%
\end{pgfscope}%
\begin{pgfscope}%
\pgfpathrectangle{\pgfqpoint{1.250000in}{0.550000in}}{\pgfqpoint{7.750000in}{3.850000in}}%
\pgfusepath{clip}%
\pgfsetbuttcap%
\pgfsetmiterjoin%
\definecolor{currentfill}{rgb}{0.501961,0.501961,0.501961}%
\pgfsetfillcolor{currentfill}%
\pgfsetfillopacity{0.700000}%
\pgfsetlinewidth{0.000000pt}%
\definecolor{currentstroke}{rgb}{0.000000,0.000000,0.000000}%
\pgfsetstrokecolor{currentstroke}%
\pgfsetstrokeopacity{0.700000}%
\pgfsetdash{}{0pt}%
\pgfpathmoveto{\pgfqpoint{2.565518in}{0.550000in}}%
\pgfpathlineto{\pgfqpoint{2.593040in}{0.550000in}}%
\pgfpathlineto{\pgfqpoint{2.593040in}{0.779339in}}%
\pgfpathlineto{\pgfqpoint{2.565518in}{0.779339in}}%
\pgfpathlineto{\pgfqpoint{2.565518in}{0.550000in}}%
\pgfpathclose%
\pgfusepath{fill}%
\end{pgfscope}%
\begin{pgfscope}%
\pgfpathrectangle{\pgfqpoint{1.250000in}{0.550000in}}{\pgfqpoint{7.750000in}{3.850000in}}%
\pgfusepath{clip}%
\pgfsetbuttcap%
\pgfsetmiterjoin%
\definecolor{currentfill}{rgb}{0.501961,0.501961,0.501961}%
\pgfsetfillcolor{currentfill}%
\pgfsetfillopacity{0.700000}%
\pgfsetlinewidth{0.000000pt}%
\definecolor{currentstroke}{rgb}{0.000000,0.000000,0.000000}%
\pgfsetstrokecolor{currentstroke}%
\pgfsetstrokeopacity{0.700000}%
\pgfsetdash{}{0pt}%
\pgfpathmoveto{\pgfqpoint{2.593040in}{0.550000in}}%
\pgfpathlineto{\pgfqpoint{2.620561in}{0.550000in}}%
\pgfpathlineto{\pgfqpoint{2.620561in}{0.763646in}}%
\pgfpathlineto{\pgfqpoint{2.593040in}{0.763646in}}%
\pgfpathlineto{\pgfqpoint{2.593040in}{0.550000in}}%
\pgfpathclose%
\pgfusepath{fill}%
\end{pgfscope}%
\begin{pgfscope}%
\pgfpathrectangle{\pgfqpoint{1.250000in}{0.550000in}}{\pgfqpoint{7.750000in}{3.850000in}}%
\pgfusepath{clip}%
\pgfsetbuttcap%
\pgfsetmiterjoin%
\definecolor{currentfill}{rgb}{0.501961,0.501961,0.501961}%
\pgfsetfillcolor{currentfill}%
\pgfsetfillopacity{0.700000}%
\pgfsetlinewidth{0.000000pt}%
\definecolor{currentstroke}{rgb}{0.000000,0.000000,0.000000}%
\pgfsetstrokecolor{currentstroke}%
\pgfsetstrokeopacity{0.700000}%
\pgfsetdash{}{0pt}%
\pgfpathmoveto{\pgfqpoint{2.620561in}{0.550000in}}%
\pgfpathlineto{\pgfqpoint{2.648082in}{0.550000in}}%
\pgfpathlineto{\pgfqpoint{2.648082in}{0.786222in}}%
\pgfpathlineto{\pgfqpoint{2.620561in}{0.786222in}}%
\pgfpathlineto{\pgfqpoint{2.620561in}{0.550000in}}%
\pgfpathclose%
\pgfusepath{fill}%
\end{pgfscope}%
\begin{pgfscope}%
\pgfpathrectangle{\pgfqpoint{1.250000in}{0.550000in}}{\pgfqpoint{7.750000in}{3.850000in}}%
\pgfusepath{clip}%
\pgfsetbuttcap%
\pgfsetmiterjoin%
\definecolor{currentfill}{rgb}{0.501961,0.501961,0.501961}%
\pgfsetfillcolor{currentfill}%
\pgfsetfillopacity{0.700000}%
\pgfsetlinewidth{0.000000pt}%
\definecolor{currentstroke}{rgb}{0.000000,0.000000,0.000000}%
\pgfsetstrokecolor{currentstroke}%
\pgfsetstrokeopacity{0.700000}%
\pgfsetdash{}{0pt}%
\pgfpathmoveto{\pgfqpoint{2.648082in}{0.550000in}}%
\pgfpathlineto{\pgfqpoint{2.675604in}{0.550000in}}%
\pgfpathlineto{\pgfqpoint{2.675604in}{0.762269in}}%
\pgfpathlineto{\pgfqpoint{2.648082in}{0.762269in}}%
\pgfpathlineto{\pgfqpoint{2.648082in}{0.550000in}}%
\pgfpathclose%
\pgfusepath{fill}%
\end{pgfscope}%
\begin{pgfscope}%
\pgfpathrectangle{\pgfqpoint{1.250000in}{0.550000in}}{\pgfqpoint{7.750000in}{3.850000in}}%
\pgfusepath{clip}%
\pgfsetbuttcap%
\pgfsetmiterjoin%
\definecolor{currentfill}{rgb}{0.501961,0.501961,0.501961}%
\pgfsetfillcolor{currentfill}%
\pgfsetfillopacity{0.700000}%
\pgfsetlinewidth{0.000000pt}%
\definecolor{currentstroke}{rgb}{0.000000,0.000000,0.000000}%
\pgfsetstrokecolor{currentstroke}%
\pgfsetstrokeopacity{0.700000}%
\pgfsetdash{}{0pt}%
\pgfpathmoveto{\pgfqpoint{2.675604in}{0.550000in}}%
\pgfpathlineto{\pgfqpoint{2.703125in}{0.550000in}}%
\pgfpathlineto{\pgfqpoint{2.703125in}{0.749329in}}%
\pgfpathlineto{\pgfqpoint{2.675604in}{0.749329in}}%
\pgfpathlineto{\pgfqpoint{2.675604in}{0.550000in}}%
\pgfpathclose%
\pgfusepath{fill}%
\end{pgfscope}%
\begin{pgfscope}%
\pgfpathrectangle{\pgfqpoint{1.250000in}{0.550000in}}{\pgfqpoint{7.750000in}{3.850000in}}%
\pgfusepath{clip}%
\pgfsetbuttcap%
\pgfsetmiterjoin%
\definecolor{currentfill}{rgb}{0.501961,0.501961,0.501961}%
\pgfsetfillcolor{currentfill}%
\pgfsetfillopacity{0.700000}%
\pgfsetlinewidth{0.000000pt}%
\definecolor{currentstroke}{rgb}{0.000000,0.000000,0.000000}%
\pgfsetstrokecolor{currentstroke}%
\pgfsetstrokeopacity{0.700000}%
\pgfsetdash{}{0pt}%
\pgfpathmoveto{\pgfqpoint{2.703125in}{0.550000in}}%
\pgfpathlineto{\pgfqpoint{2.730646in}{0.550000in}}%
\pgfpathlineto{\pgfqpoint{2.730646in}{0.746851in}}%
\pgfpathlineto{\pgfqpoint{2.703125in}{0.746851in}}%
\pgfpathlineto{\pgfqpoint{2.703125in}{0.550000in}}%
\pgfpathclose%
\pgfusepath{fill}%
\end{pgfscope}%
\begin{pgfscope}%
\pgfpathrectangle{\pgfqpoint{1.250000in}{0.550000in}}{\pgfqpoint{7.750000in}{3.850000in}}%
\pgfusepath{clip}%
\pgfsetbuttcap%
\pgfsetmiterjoin%
\definecolor{currentfill}{rgb}{0.501961,0.501961,0.501961}%
\pgfsetfillcolor{currentfill}%
\pgfsetfillopacity{0.700000}%
\pgfsetlinewidth{0.000000pt}%
\definecolor{currentstroke}{rgb}{0.000000,0.000000,0.000000}%
\pgfsetstrokecolor{currentstroke}%
\pgfsetstrokeopacity{0.700000}%
\pgfsetdash{}{0pt}%
\pgfpathmoveto{\pgfqpoint{2.730646in}{0.550000in}}%
\pgfpathlineto{\pgfqpoint{2.758168in}{0.550000in}}%
\pgfpathlineto{\pgfqpoint{2.758168in}{0.733911in}}%
\pgfpathlineto{\pgfqpoint{2.730646in}{0.733911in}}%
\pgfpathlineto{\pgfqpoint{2.730646in}{0.550000in}}%
\pgfpathclose%
\pgfusepath{fill}%
\end{pgfscope}%
\begin{pgfscope}%
\pgfpathrectangle{\pgfqpoint{1.250000in}{0.550000in}}{\pgfqpoint{7.750000in}{3.850000in}}%
\pgfusepath{clip}%
\pgfsetbuttcap%
\pgfsetmiterjoin%
\definecolor{currentfill}{rgb}{0.501961,0.501961,0.501961}%
\pgfsetfillcolor{currentfill}%
\pgfsetfillopacity{0.700000}%
\pgfsetlinewidth{0.000000pt}%
\definecolor{currentstroke}{rgb}{0.000000,0.000000,0.000000}%
\pgfsetstrokecolor{currentstroke}%
\pgfsetstrokeopacity{0.700000}%
\pgfsetdash{}{0pt}%
\pgfpathmoveto{\pgfqpoint{2.758168in}{0.550000in}}%
\pgfpathlineto{\pgfqpoint{2.785689in}{0.550000in}}%
\pgfpathlineto{\pgfqpoint{2.785689in}{0.735839in}}%
\pgfpathlineto{\pgfqpoint{2.758168in}{0.735839in}}%
\pgfpathlineto{\pgfqpoint{2.758168in}{0.550000in}}%
\pgfpathclose%
\pgfusepath{fill}%
\end{pgfscope}%
\begin{pgfscope}%
\pgfpathrectangle{\pgfqpoint{1.250000in}{0.550000in}}{\pgfqpoint{7.750000in}{3.850000in}}%
\pgfusepath{clip}%
\pgfsetbuttcap%
\pgfsetmiterjoin%
\definecolor{currentfill}{rgb}{0.501961,0.501961,0.501961}%
\pgfsetfillcolor{currentfill}%
\pgfsetfillopacity{0.700000}%
\pgfsetlinewidth{0.000000pt}%
\definecolor{currentstroke}{rgb}{0.000000,0.000000,0.000000}%
\pgfsetstrokecolor{currentstroke}%
\pgfsetstrokeopacity{0.700000}%
\pgfsetdash{}{0pt}%
\pgfpathmoveto{\pgfqpoint{2.785689in}{0.550000in}}%
\pgfpathlineto{\pgfqpoint{2.813210in}{0.550000in}}%
\pgfpathlineto{\pgfqpoint{2.813210in}{0.730883in}}%
\pgfpathlineto{\pgfqpoint{2.785689in}{0.730883in}}%
\pgfpathlineto{\pgfqpoint{2.785689in}{0.550000in}}%
\pgfpathclose%
\pgfusepath{fill}%
\end{pgfscope}%
\begin{pgfscope}%
\pgfpathrectangle{\pgfqpoint{1.250000in}{0.550000in}}{\pgfqpoint{7.750000in}{3.850000in}}%
\pgfusepath{clip}%
\pgfsetbuttcap%
\pgfsetmiterjoin%
\definecolor{currentfill}{rgb}{0.501961,0.501961,0.501961}%
\pgfsetfillcolor{currentfill}%
\pgfsetfillopacity{0.700000}%
\pgfsetlinewidth{0.000000pt}%
\definecolor{currentstroke}{rgb}{0.000000,0.000000,0.000000}%
\pgfsetstrokecolor{currentstroke}%
\pgfsetstrokeopacity{0.700000}%
\pgfsetdash{}{0pt}%
\pgfpathmoveto{\pgfqpoint{2.813210in}{0.550000in}}%
\pgfpathlineto{\pgfqpoint{2.840732in}{0.550000in}}%
\pgfpathlineto{\pgfqpoint{2.840732in}{0.725377in}}%
\pgfpathlineto{\pgfqpoint{2.813210in}{0.725377in}}%
\pgfpathlineto{\pgfqpoint{2.813210in}{0.550000in}}%
\pgfpathclose%
\pgfusepath{fill}%
\end{pgfscope}%
\begin{pgfscope}%
\pgfpathrectangle{\pgfqpoint{1.250000in}{0.550000in}}{\pgfqpoint{7.750000in}{3.850000in}}%
\pgfusepath{clip}%
\pgfsetbuttcap%
\pgfsetmiterjoin%
\definecolor{currentfill}{rgb}{0.501961,0.501961,0.501961}%
\pgfsetfillcolor{currentfill}%
\pgfsetfillopacity{0.700000}%
\pgfsetlinewidth{0.000000pt}%
\definecolor{currentstroke}{rgb}{0.000000,0.000000,0.000000}%
\pgfsetstrokecolor{currentstroke}%
\pgfsetstrokeopacity{0.700000}%
\pgfsetdash{}{0pt}%
\pgfpathmoveto{\pgfqpoint{2.840732in}{0.550000in}}%
\pgfpathlineto{\pgfqpoint{2.868253in}{0.550000in}}%
\pgfpathlineto{\pgfqpoint{2.868253in}{0.726753in}}%
\pgfpathlineto{\pgfqpoint{2.840732in}{0.726753in}}%
\pgfpathlineto{\pgfqpoint{2.840732in}{0.550000in}}%
\pgfpathclose%
\pgfusepath{fill}%
\end{pgfscope}%
\begin{pgfscope}%
\pgfpathrectangle{\pgfqpoint{1.250000in}{0.550000in}}{\pgfqpoint{7.750000in}{3.850000in}}%
\pgfusepath{clip}%
\pgfsetbuttcap%
\pgfsetmiterjoin%
\definecolor{currentfill}{rgb}{0.501961,0.501961,0.501961}%
\pgfsetfillcolor{currentfill}%
\pgfsetfillopacity{0.700000}%
\pgfsetlinewidth{0.000000pt}%
\definecolor{currentstroke}{rgb}{0.000000,0.000000,0.000000}%
\pgfsetstrokecolor{currentstroke}%
\pgfsetstrokeopacity{0.700000}%
\pgfsetdash{}{0pt}%
\pgfpathmoveto{\pgfqpoint{2.868253in}{0.550000in}}%
\pgfpathlineto{\pgfqpoint{2.895774in}{0.550000in}}%
\pgfpathlineto{\pgfqpoint{2.895774in}{0.734187in}}%
\pgfpathlineto{\pgfqpoint{2.868253in}{0.734187in}}%
\pgfpathlineto{\pgfqpoint{2.868253in}{0.550000in}}%
\pgfpathclose%
\pgfusepath{fill}%
\end{pgfscope}%
\begin{pgfscope}%
\pgfpathrectangle{\pgfqpoint{1.250000in}{0.550000in}}{\pgfqpoint{7.750000in}{3.850000in}}%
\pgfusepath{clip}%
\pgfsetbuttcap%
\pgfsetmiterjoin%
\definecolor{currentfill}{rgb}{0.501961,0.501961,0.501961}%
\pgfsetfillcolor{currentfill}%
\pgfsetfillopacity{0.700000}%
\pgfsetlinewidth{0.000000pt}%
\definecolor{currentstroke}{rgb}{0.000000,0.000000,0.000000}%
\pgfsetstrokecolor{currentstroke}%
\pgfsetstrokeopacity{0.700000}%
\pgfsetdash{}{0pt}%
\pgfpathmoveto{\pgfqpoint{2.895774in}{0.550000in}}%
\pgfpathlineto{\pgfqpoint{2.923295in}{0.550000in}}%
\pgfpathlineto{\pgfqpoint{2.923295in}{0.750430in}}%
\pgfpathlineto{\pgfqpoint{2.895774in}{0.750430in}}%
\pgfpathlineto{\pgfqpoint{2.895774in}{0.550000in}}%
\pgfpathclose%
\pgfusepath{fill}%
\end{pgfscope}%
\begin{pgfscope}%
\pgfpathrectangle{\pgfqpoint{1.250000in}{0.550000in}}{\pgfqpoint{7.750000in}{3.850000in}}%
\pgfusepath{clip}%
\pgfsetbuttcap%
\pgfsetmiterjoin%
\definecolor{currentfill}{rgb}{0.501961,0.501961,0.501961}%
\pgfsetfillcolor{currentfill}%
\pgfsetfillopacity{0.700000}%
\pgfsetlinewidth{0.000000pt}%
\definecolor{currentstroke}{rgb}{0.000000,0.000000,0.000000}%
\pgfsetstrokecolor{currentstroke}%
\pgfsetstrokeopacity{0.700000}%
\pgfsetdash{}{0pt}%
\pgfpathmoveto{\pgfqpoint{2.923295in}{0.550000in}}%
\pgfpathlineto{\pgfqpoint{2.950817in}{0.550000in}}%
\pgfpathlineto{\pgfqpoint{2.950817in}{0.741896in}}%
\pgfpathlineto{\pgfqpoint{2.923295in}{0.741896in}}%
\pgfpathlineto{\pgfqpoint{2.923295in}{0.550000in}}%
\pgfpathclose%
\pgfusepath{fill}%
\end{pgfscope}%
\begin{pgfscope}%
\pgfpathrectangle{\pgfqpoint{1.250000in}{0.550000in}}{\pgfqpoint{7.750000in}{3.850000in}}%
\pgfusepath{clip}%
\pgfsetbuttcap%
\pgfsetmiterjoin%
\definecolor{currentfill}{rgb}{0.501961,0.501961,0.501961}%
\pgfsetfillcolor{currentfill}%
\pgfsetfillopacity{0.700000}%
\pgfsetlinewidth{0.000000pt}%
\definecolor{currentstroke}{rgb}{0.000000,0.000000,0.000000}%
\pgfsetstrokecolor{currentstroke}%
\pgfsetstrokeopacity{0.700000}%
\pgfsetdash{}{0pt}%
\pgfpathmoveto{\pgfqpoint{2.950817in}{0.550000in}}%
\pgfpathlineto{\pgfqpoint{2.978338in}{0.550000in}}%
\pgfpathlineto{\pgfqpoint{2.978338in}{0.738592in}}%
\pgfpathlineto{\pgfqpoint{2.950817in}{0.738592in}}%
\pgfpathlineto{\pgfqpoint{2.950817in}{0.550000in}}%
\pgfpathclose%
\pgfusepath{fill}%
\end{pgfscope}%
\begin{pgfscope}%
\pgfpathrectangle{\pgfqpoint{1.250000in}{0.550000in}}{\pgfqpoint{7.750000in}{3.850000in}}%
\pgfusepath{clip}%
\pgfsetbuttcap%
\pgfsetmiterjoin%
\definecolor{currentfill}{rgb}{0.501961,0.501961,0.501961}%
\pgfsetfillcolor{currentfill}%
\pgfsetfillopacity{0.700000}%
\pgfsetlinewidth{0.000000pt}%
\definecolor{currentstroke}{rgb}{0.000000,0.000000,0.000000}%
\pgfsetstrokecolor{currentstroke}%
\pgfsetstrokeopacity{0.700000}%
\pgfsetdash{}{0pt}%
\pgfpathmoveto{\pgfqpoint{2.978338in}{0.550000in}}%
\pgfpathlineto{\pgfqpoint{3.005859in}{0.550000in}}%
\pgfpathlineto{\pgfqpoint{3.005859in}{0.724551in}}%
\pgfpathlineto{\pgfqpoint{2.978338in}{0.724551in}}%
\pgfpathlineto{\pgfqpoint{2.978338in}{0.550000in}}%
\pgfpathclose%
\pgfusepath{fill}%
\end{pgfscope}%
\begin{pgfscope}%
\pgfpathrectangle{\pgfqpoint{1.250000in}{0.550000in}}{\pgfqpoint{7.750000in}{3.850000in}}%
\pgfusepath{clip}%
\pgfsetbuttcap%
\pgfsetmiterjoin%
\definecolor{currentfill}{rgb}{0.501961,0.501961,0.501961}%
\pgfsetfillcolor{currentfill}%
\pgfsetfillopacity{0.700000}%
\pgfsetlinewidth{0.000000pt}%
\definecolor{currentstroke}{rgb}{0.000000,0.000000,0.000000}%
\pgfsetstrokecolor{currentstroke}%
\pgfsetstrokeopacity{0.700000}%
\pgfsetdash{}{0pt}%
\pgfpathmoveto{\pgfqpoint{3.005859in}{0.550000in}}%
\pgfpathlineto{\pgfqpoint{3.033381in}{0.550000in}}%
\pgfpathlineto{\pgfqpoint{3.033381in}{0.736940in}}%
\pgfpathlineto{\pgfqpoint{3.005859in}{0.736940in}}%
\pgfpathlineto{\pgfqpoint{3.005859in}{0.550000in}}%
\pgfpathclose%
\pgfusepath{fill}%
\end{pgfscope}%
\begin{pgfscope}%
\pgfpathrectangle{\pgfqpoint{1.250000in}{0.550000in}}{\pgfqpoint{7.750000in}{3.850000in}}%
\pgfusepath{clip}%
\pgfsetbuttcap%
\pgfsetmiterjoin%
\definecolor{currentfill}{rgb}{0.501961,0.501961,0.501961}%
\pgfsetfillcolor{currentfill}%
\pgfsetfillopacity{0.700000}%
\pgfsetlinewidth{0.000000pt}%
\definecolor{currentstroke}{rgb}{0.000000,0.000000,0.000000}%
\pgfsetstrokecolor{currentstroke}%
\pgfsetstrokeopacity{0.700000}%
\pgfsetdash{}{0pt}%
\pgfpathmoveto{\pgfqpoint{3.033381in}{0.550000in}}%
\pgfpathlineto{\pgfqpoint{3.060902in}{0.550000in}}%
\pgfpathlineto{\pgfqpoint{3.060902in}{0.730883in}}%
\pgfpathlineto{\pgfqpoint{3.033381in}{0.730883in}}%
\pgfpathlineto{\pgfqpoint{3.033381in}{0.550000in}}%
\pgfpathclose%
\pgfusepath{fill}%
\end{pgfscope}%
\begin{pgfscope}%
\pgfpathrectangle{\pgfqpoint{1.250000in}{0.550000in}}{\pgfqpoint{7.750000in}{3.850000in}}%
\pgfusepath{clip}%
\pgfsetbuttcap%
\pgfsetmiterjoin%
\definecolor{currentfill}{rgb}{0.501961,0.501961,0.501961}%
\pgfsetfillcolor{currentfill}%
\pgfsetfillopacity{0.700000}%
\pgfsetlinewidth{0.000000pt}%
\definecolor{currentstroke}{rgb}{0.000000,0.000000,0.000000}%
\pgfsetstrokecolor{currentstroke}%
\pgfsetstrokeopacity{0.700000}%
\pgfsetdash{}{0pt}%
\pgfpathmoveto{\pgfqpoint{3.060902in}{0.550000in}}%
\pgfpathlineto{\pgfqpoint{3.088423in}{0.550000in}}%
\pgfpathlineto{\pgfqpoint{3.088423in}{0.710785in}}%
\pgfpathlineto{\pgfqpoint{3.060902in}{0.710785in}}%
\pgfpathlineto{\pgfqpoint{3.060902in}{0.550000in}}%
\pgfpathclose%
\pgfusepath{fill}%
\end{pgfscope}%
\begin{pgfscope}%
\pgfpathrectangle{\pgfqpoint{1.250000in}{0.550000in}}{\pgfqpoint{7.750000in}{3.850000in}}%
\pgfusepath{clip}%
\pgfsetbuttcap%
\pgfsetmiterjoin%
\definecolor{currentfill}{rgb}{0.501961,0.501961,0.501961}%
\pgfsetfillcolor{currentfill}%
\pgfsetfillopacity{0.700000}%
\pgfsetlinewidth{0.000000pt}%
\definecolor{currentstroke}{rgb}{0.000000,0.000000,0.000000}%
\pgfsetstrokecolor{currentstroke}%
\pgfsetstrokeopacity{0.700000}%
\pgfsetdash{}{0pt}%
\pgfpathmoveto{\pgfqpoint{3.088423in}{0.550000in}}%
\pgfpathlineto{\pgfqpoint{3.115945in}{0.550000in}}%
\pgfpathlineto{\pgfqpoint{3.115945in}{0.706655in}}%
\pgfpathlineto{\pgfqpoint{3.088423in}{0.706655in}}%
\pgfpathlineto{\pgfqpoint{3.088423in}{0.550000in}}%
\pgfpathclose%
\pgfusepath{fill}%
\end{pgfscope}%
\begin{pgfscope}%
\pgfpathrectangle{\pgfqpoint{1.250000in}{0.550000in}}{\pgfqpoint{7.750000in}{3.850000in}}%
\pgfusepath{clip}%
\pgfsetbuttcap%
\pgfsetmiterjoin%
\definecolor{currentfill}{rgb}{0.501961,0.501961,0.501961}%
\pgfsetfillcolor{currentfill}%
\pgfsetfillopacity{0.700000}%
\pgfsetlinewidth{0.000000pt}%
\definecolor{currentstroke}{rgb}{0.000000,0.000000,0.000000}%
\pgfsetstrokecolor{currentstroke}%
\pgfsetstrokeopacity{0.700000}%
\pgfsetdash{}{0pt}%
\pgfpathmoveto{\pgfqpoint{3.115945in}{0.550000in}}%
\pgfpathlineto{\pgfqpoint{3.143466in}{0.550000in}}%
\pgfpathlineto{\pgfqpoint{3.143466in}{0.704728in}}%
\pgfpathlineto{\pgfqpoint{3.115945in}{0.704728in}}%
\pgfpathlineto{\pgfqpoint{3.115945in}{0.550000in}}%
\pgfpathclose%
\pgfusepath{fill}%
\end{pgfscope}%
\begin{pgfscope}%
\pgfpathrectangle{\pgfqpoint{1.250000in}{0.550000in}}{\pgfqpoint{7.750000in}{3.850000in}}%
\pgfusepath{clip}%
\pgfsetbuttcap%
\pgfsetmiterjoin%
\definecolor{currentfill}{rgb}{0.501961,0.501961,0.501961}%
\pgfsetfillcolor{currentfill}%
\pgfsetfillopacity{0.700000}%
\pgfsetlinewidth{0.000000pt}%
\definecolor{currentstroke}{rgb}{0.000000,0.000000,0.000000}%
\pgfsetstrokecolor{currentstroke}%
\pgfsetstrokeopacity{0.700000}%
\pgfsetdash{}{0pt}%
\pgfpathmoveto{\pgfqpoint{3.143466in}{0.550000in}}%
\pgfpathlineto{\pgfqpoint{3.170987in}{0.550000in}}%
\pgfpathlineto{\pgfqpoint{3.170987in}{0.722073in}}%
\pgfpathlineto{\pgfqpoint{3.143466in}{0.722073in}}%
\pgfpathlineto{\pgfqpoint{3.143466in}{0.550000in}}%
\pgfpathclose%
\pgfusepath{fill}%
\end{pgfscope}%
\begin{pgfscope}%
\pgfpathrectangle{\pgfqpoint{1.250000in}{0.550000in}}{\pgfqpoint{7.750000in}{3.850000in}}%
\pgfusepath{clip}%
\pgfsetbuttcap%
\pgfsetmiterjoin%
\definecolor{currentfill}{rgb}{0.501961,0.501961,0.501961}%
\pgfsetfillcolor{currentfill}%
\pgfsetfillopacity{0.700000}%
\pgfsetlinewidth{0.000000pt}%
\definecolor{currentstroke}{rgb}{0.000000,0.000000,0.000000}%
\pgfsetstrokecolor{currentstroke}%
\pgfsetstrokeopacity{0.700000}%
\pgfsetdash{}{0pt}%
\pgfpathmoveto{\pgfqpoint{3.170987in}{0.550000in}}%
\pgfpathlineto{\pgfqpoint{3.198509in}{0.550000in}}%
\pgfpathlineto{\pgfqpoint{3.198509in}{0.723725in}}%
\pgfpathlineto{\pgfqpoint{3.170987in}{0.723725in}}%
\pgfpathlineto{\pgfqpoint{3.170987in}{0.550000in}}%
\pgfpathclose%
\pgfusepath{fill}%
\end{pgfscope}%
\begin{pgfscope}%
\pgfpathrectangle{\pgfqpoint{1.250000in}{0.550000in}}{\pgfqpoint{7.750000in}{3.850000in}}%
\pgfusepath{clip}%
\pgfsetbuttcap%
\pgfsetmiterjoin%
\definecolor{currentfill}{rgb}{0.501961,0.501961,0.501961}%
\pgfsetfillcolor{currentfill}%
\pgfsetfillopacity{0.700000}%
\pgfsetlinewidth{0.000000pt}%
\definecolor{currentstroke}{rgb}{0.000000,0.000000,0.000000}%
\pgfsetstrokecolor{currentstroke}%
\pgfsetstrokeopacity{0.700000}%
\pgfsetdash{}{0pt}%
\pgfpathmoveto{\pgfqpoint{3.198509in}{0.550000in}}%
\pgfpathlineto{\pgfqpoint{3.226030in}{0.550000in}}%
\pgfpathlineto{\pgfqpoint{3.226030in}{0.724826in}}%
\pgfpathlineto{\pgfqpoint{3.198509in}{0.724826in}}%
\pgfpathlineto{\pgfqpoint{3.198509in}{0.550000in}}%
\pgfpathclose%
\pgfusepath{fill}%
\end{pgfscope}%
\begin{pgfscope}%
\pgfpathrectangle{\pgfqpoint{1.250000in}{0.550000in}}{\pgfqpoint{7.750000in}{3.850000in}}%
\pgfusepath{clip}%
\pgfsetbuttcap%
\pgfsetmiterjoin%
\definecolor{currentfill}{rgb}{0.501961,0.501961,0.501961}%
\pgfsetfillcolor{currentfill}%
\pgfsetfillopacity{0.700000}%
\pgfsetlinewidth{0.000000pt}%
\definecolor{currentstroke}{rgb}{0.000000,0.000000,0.000000}%
\pgfsetstrokecolor{currentstroke}%
\pgfsetstrokeopacity{0.700000}%
\pgfsetdash{}{0pt}%
\pgfpathmoveto{\pgfqpoint{3.226030in}{0.550000in}}%
\pgfpathlineto{\pgfqpoint{3.253551in}{0.550000in}}%
\pgfpathlineto{\pgfqpoint{3.253551in}{0.718218in}}%
\pgfpathlineto{\pgfqpoint{3.226030in}{0.718218in}}%
\pgfpathlineto{\pgfqpoint{3.226030in}{0.550000in}}%
\pgfpathclose%
\pgfusepath{fill}%
\end{pgfscope}%
\begin{pgfscope}%
\pgfpathrectangle{\pgfqpoint{1.250000in}{0.550000in}}{\pgfqpoint{7.750000in}{3.850000in}}%
\pgfusepath{clip}%
\pgfsetbuttcap%
\pgfsetmiterjoin%
\definecolor{currentfill}{rgb}{0.501961,0.501961,0.501961}%
\pgfsetfillcolor{currentfill}%
\pgfsetfillopacity{0.700000}%
\pgfsetlinewidth{0.000000pt}%
\definecolor{currentstroke}{rgb}{0.000000,0.000000,0.000000}%
\pgfsetstrokecolor{currentstroke}%
\pgfsetstrokeopacity{0.700000}%
\pgfsetdash{}{0pt}%
\pgfpathmoveto{\pgfqpoint{3.253551in}{0.550000in}}%
\pgfpathlineto{\pgfqpoint{3.281072in}{0.550000in}}%
\pgfpathlineto{\pgfqpoint{3.281072in}{0.714639in}}%
\pgfpathlineto{\pgfqpoint{3.253551in}{0.714639in}}%
\pgfpathlineto{\pgfqpoint{3.253551in}{0.550000in}}%
\pgfpathclose%
\pgfusepath{fill}%
\end{pgfscope}%
\begin{pgfscope}%
\pgfpathrectangle{\pgfqpoint{1.250000in}{0.550000in}}{\pgfqpoint{7.750000in}{3.850000in}}%
\pgfusepath{clip}%
\pgfsetbuttcap%
\pgfsetmiterjoin%
\definecolor{currentfill}{rgb}{0.501961,0.501961,0.501961}%
\pgfsetfillcolor{currentfill}%
\pgfsetfillopacity{0.700000}%
\pgfsetlinewidth{0.000000pt}%
\definecolor{currentstroke}{rgb}{0.000000,0.000000,0.000000}%
\pgfsetstrokecolor{currentstroke}%
\pgfsetstrokeopacity{0.700000}%
\pgfsetdash{}{0pt}%
\pgfpathmoveto{\pgfqpoint{3.281072in}{0.550000in}}%
\pgfpathlineto{\pgfqpoint{3.308594in}{0.550000in}}%
\pgfpathlineto{\pgfqpoint{3.308594in}{0.710234in}}%
\pgfpathlineto{\pgfqpoint{3.281072in}{0.710234in}}%
\pgfpathlineto{\pgfqpoint{3.281072in}{0.550000in}}%
\pgfpathclose%
\pgfusepath{fill}%
\end{pgfscope}%
\begin{pgfscope}%
\pgfpathrectangle{\pgfqpoint{1.250000in}{0.550000in}}{\pgfqpoint{7.750000in}{3.850000in}}%
\pgfusepath{clip}%
\pgfsetbuttcap%
\pgfsetmiterjoin%
\definecolor{currentfill}{rgb}{0.501961,0.501961,0.501961}%
\pgfsetfillcolor{currentfill}%
\pgfsetfillopacity{0.700000}%
\pgfsetlinewidth{0.000000pt}%
\definecolor{currentstroke}{rgb}{0.000000,0.000000,0.000000}%
\pgfsetstrokecolor{currentstroke}%
\pgfsetstrokeopacity{0.700000}%
\pgfsetdash{}{0pt}%
\pgfpathmoveto{\pgfqpoint{3.308594in}{0.550000in}}%
\pgfpathlineto{\pgfqpoint{3.336115in}{0.550000in}}%
\pgfpathlineto{\pgfqpoint{3.336115in}{0.709408in}}%
\pgfpathlineto{\pgfqpoint{3.308594in}{0.709408in}}%
\pgfpathlineto{\pgfqpoint{3.308594in}{0.550000in}}%
\pgfpathclose%
\pgfusepath{fill}%
\end{pgfscope}%
\begin{pgfscope}%
\pgfpathrectangle{\pgfqpoint{1.250000in}{0.550000in}}{\pgfqpoint{7.750000in}{3.850000in}}%
\pgfusepath{clip}%
\pgfsetbuttcap%
\pgfsetmiterjoin%
\definecolor{currentfill}{rgb}{0.501961,0.501961,0.501961}%
\pgfsetfillcolor{currentfill}%
\pgfsetfillopacity{0.700000}%
\pgfsetlinewidth{0.000000pt}%
\definecolor{currentstroke}{rgb}{0.000000,0.000000,0.000000}%
\pgfsetstrokecolor{currentstroke}%
\pgfsetstrokeopacity{0.700000}%
\pgfsetdash{}{0pt}%
\pgfpathmoveto{\pgfqpoint{3.336115in}{0.550000in}}%
\pgfpathlineto{\pgfqpoint{3.363636in}{0.550000in}}%
\pgfpathlineto{\pgfqpoint{3.363636in}{0.709408in}}%
\pgfpathlineto{\pgfqpoint{3.336115in}{0.709408in}}%
\pgfpathlineto{\pgfqpoint{3.336115in}{0.550000in}}%
\pgfpathclose%
\pgfusepath{fill}%
\end{pgfscope}%
\begin{pgfscope}%
\pgfpathrectangle{\pgfqpoint{1.250000in}{0.550000in}}{\pgfqpoint{7.750000in}{3.850000in}}%
\pgfusepath{clip}%
\pgfsetbuttcap%
\pgfsetmiterjoin%
\definecolor{currentfill}{rgb}{0.501961,0.501961,0.501961}%
\pgfsetfillcolor{currentfill}%
\pgfsetfillopacity{0.700000}%
\pgfsetlinewidth{0.000000pt}%
\definecolor{currentstroke}{rgb}{0.000000,0.000000,0.000000}%
\pgfsetstrokecolor{currentstroke}%
\pgfsetstrokeopacity{0.700000}%
\pgfsetdash{}{0pt}%
\pgfpathmoveto{\pgfqpoint{3.363636in}{0.550000in}}%
\pgfpathlineto{\pgfqpoint{3.391158in}{0.550000in}}%
\pgfpathlineto{\pgfqpoint{3.391158in}{0.692063in}}%
\pgfpathlineto{\pgfqpoint{3.363636in}{0.692063in}}%
\pgfpathlineto{\pgfqpoint{3.363636in}{0.550000in}}%
\pgfpathclose%
\pgfusepath{fill}%
\end{pgfscope}%
\begin{pgfscope}%
\pgfpathrectangle{\pgfqpoint{1.250000in}{0.550000in}}{\pgfqpoint{7.750000in}{3.850000in}}%
\pgfusepath{clip}%
\pgfsetbuttcap%
\pgfsetmiterjoin%
\definecolor{currentfill}{rgb}{0.501961,0.501961,0.501961}%
\pgfsetfillcolor{currentfill}%
\pgfsetfillopacity{0.700000}%
\pgfsetlinewidth{0.000000pt}%
\definecolor{currentstroke}{rgb}{0.000000,0.000000,0.000000}%
\pgfsetstrokecolor{currentstroke}%
\pgfsetstrokeopacity{0.700000}%
\pgfsetdash{}{0pt}%
\pgfpathmoveto{\pgfqpoint{3.391158in}{0.550000in}}%
\pgfpathlineto{\pgfqpoint{3.418679in}{0.550000in}}%
\pgfpathlineto{\pgfqpoint{3.418679in}{0.692063in}}%
\pgfpathlineto{\pgfqpoint{3.391158in}{0.692063in}}%
\pgfpathlineto{\pgfqpoint{3.391158in}{0.550000in}}%
\pgfpathclose%
\pgfusepath{fill}%
\end{pgfscope}%
\begin{pgfscope}%
\pgfpathrectangle{\pgfqpoint{1.250000in}{0.550000in}}{\pgfqpoint{7.750000in}{3.850000in}}%
\pgfusepath{clip}%
\pgfsetbuttcap%
\pgfsetmiterjoin%
\definecolor{currentfill}{rgb}{0.501961,0.501961,0.501961}%
\pgfsetfillcolor{currentfill}%
\pgfsetfillopacity{0.700000}%
\pgfsetlinewidth{0.000000pt}%
\definecolor{currentstroke}{rgb}{0.000000,0.000000,0.000000}%
\pgfsetstrokecolor{currentstroke}%
\pgfsetstrokeopacity{0.700000}%
\pgfsetdash{}{0pt}%
\pgfpathmoveto{\pgfqpoint{3.418679in}{0.550000in}}%
\pgfpathlineto{\pgfqpoint{3.446200in}{0.550000in}}%
\pgfpathlineto{\pgfqpoint{3.446200in}{0.684905in}}%
\pgfpathlineto{\pgfqpoint{3.418679in}{0.684905in}}%
\pgfpathlineto{\pgfqpoint{3.418679in}{0.550000in}}%
\pgfpathclose%
\pgfusepath{fill}%
\end{pgfscope}%
\begin{pgfscope}%
\pgfpathrectangle{\pgfqpoint{1.250000in}{0.550000in}}{\pgfqpoint{7.750000in}{3.850000in}}%
\pgfusepath{clip}%
\pgfsetbuttcap%
\pgfsetmiterjoin%
\definecolor{currentfill}{rgb}{0.501961,0.501961,0.501961}%
\pgfsetfillcolor{currentfill}%
\pgfsetfillopacity{0.700000}%
\pgfsetlinewidth{0.000000pt}%
\definecolor{currentstroke}{rgb}{0.000000,0.000000,0.000000}%
\pgfsetstrokecolor{currentstroke}%
\pgfsetstrokeopacity{0.700000}%
\pgfsetdash{}{0pt}%
\pgfpathmoveto{\pgfqpoint{3.446200in}{0.550000in}}%
\pgfpathlineto{\pgfqpoint{3.473722in}{0.550000in}}%
\pgfpathlineto{\pgfqpoint{3.473722in}{0.688209in}}%
\pgfpathlineto{\pgfqpoint{3.446200in}{0.688209in}}%
\pgfpathlineto{\pgfqpoint{3.446200in}{0.550000in}}%
\pgfpathclose%
\pgfusepath{fill}%
\end{pgfscope}%
\begin{pgfscope}%
\pgfpathrectangle{\pgfqpoint{1.250000in}{0.550000in}}{\pgfqpoint{7.750000in}{3.850000in}}%
\pgfusepath{clip}%
\pgfsetbuttcap%
\pgfsetmiterjoin%
\definecolor{currentfill}{rgb}{0.501961,0.501961,0.501961}%
\pgfsetfillcolor{currentfill}%
\pgfsetfillopacity{0.700000}%
\pgfsetlinewidth{0.000000pt}%
\definecolor{currentstroke}{rgb}{0.000000,0.000000,0.000000}%
\pgfsetstrokecolor{currentstroke}%
\pgfsetstrokeopacity{0.700000}%
\pgfsetdash{}{0pt}%
\pgfpathmoveto{\pgfqpoint{3.473722in}{0.550000in}}%
\pgfpathlineto{\pgfqpoint{3.501243in}{0.550000in}}%
\pgfpathlineto{\pgfqpoint{3.501243in}{0.700048in}}%
\pgfpathlineto{\pgfqpoint{3.473722in}{0.700048in}}%
\pgfpathlineto{\pgfqpoint{3.473722in}{0.550000in}}%
\pgfpathclose%
\pgfusepath{fill}%
\end{pgfscope}%
\begin{pgfscope}%
\pgfpathrectangle{\pgfqpoint{1.250000in}{0.550000in}}{\pgfqpoint{7.750000in}{3.850000in}}%
\pgfusepath{clip}%
\pgfsetbuttcap%
\pgfsetmiterjoin%
\definecolor{currentfill}{rgb}{0.501961,0.501961,0.501961}%
\pgfsetfillcolor{currentfill}%
\pgfsetfillopacity{0.700000}%
\pgfsetlinewidth{0.000000pt}%
\definecolor{currentstroke}{rgb}{0.000000,0.000000,0.000000}%
\pgfsetstrokecolor{currentstroke}%
\pgfsetstrokeopacity{0.700000}%
\pgfsetdash{}{0pt}%
\pgfpathmoveto{\pgfqpoint{3.501243in}{0.550000in}}%
\pgfpathlineto{\pgfqpoint{3.528764in}{0.550000in}}%
\pgfpathlineto{\pgfqpoint{3.528764in}{0.694817in}}%
\pgfpathlineto{\pgfqpoint{3.501243in}{0.694817in}}%
\pgfpathlineto{\pgfqpoint{3.501243in}{0.550000in}}%
\pgfpathclose%
\pgfusepath{fill}%
\end{pgfscope}%
\begin{pgfscope}%
\pgfpathrectangle{\pgfqpoint{1.250000in}{0.550000in}}{\pgfqpoint{7.750000in}{3.850000in}}%
\pgfusepath{clip}%
\pgfsetbuttcap%
\pgfsetmiterjoin%
\definecolor{currentfill}{rgb}{0.501961,0.501961,0.501961}%
\pgfsetfillcolor{currentfill}%
\pgfsetfillopacity{0.700000}%
\pgfsetlinewidth{0.000000pt}%
\definecolor{currentstroke}{rgb}{0.000000,0.000000,0.000000}%
\pgfsetstrokecolor{currentstroke}%
\pgfsetstrokeopacity{0.700000}%
\pgfsetdash{}{0pt}%
\pgfpathmoveto{\pgfqpoint{3.528764in}{0.550000in}}%
\pgfpathlineto{\pgfqpoint{3.556286in}{0.550000in}}%
\pgfpathlineto{\pgfqpoint{3.556286in}{0.700048in}}%
\pgfpathlineto{\pgfqpoint{3.528764in}{0.700048in}}%
\pgfpathlineto{\pgfqpoint{3.528764in}{0.550000in}}%
\pgfpathclose%
\pgfusepath{fill}%
\end{pgfscope}%
\begin{pgfscope}%
\pgfpathrectangle{\pgfqpoint{1.250000in}{0.550000in}}{\pgfqpoint{7.750000in}{3.850000in}}%
\pgfusepath{clip}%
\pgfsetbuttcap%
\pgfsetmiterjoin%
\definecolor{currentfill}{rgb}{0.501961,0.501961,0.501961}%
\pgfsetfillcolor{currentfill}%
\pgfsetfillopacity{0.700000}%
\pgfsetlinewidth{0.000000pt}%
\definecolor{currentstroke}{rgb}{0.000000,0.000000,0.000000}%
\pgfsetstrokecolor{currentstroke}%
\pgfsetstrokeopacity{0.700000}%
\pgfsetdash{}{0pt}%
\pgfpathmoveto{\pgfqpoint{3.556286in}{0.550000in}}%
\pgfpathlineto{\pgfqpoint{3.583807in}{0.550000in}}%
\pgfpathlineto{\pgfqpoint{3.583807in}{0.697294in}}%
\pgfpathlineto{\pgfqpoint{3.556286in}{0.697294in}}%
\pgfpathlineto{\pgfqpoint{3.556286in}{0.550000in}}%
\pgfpathclose%
\pgfusepath{fill}%
\end{pgfscope}%
\begin{pgfscope}%
\pgfpathrectangle{\pgfqpoint{1.250000in}{0.550000in}}{\pgfqpoint{7.750000in}{3.850000in}}%
\pgfusepath{clip}%
\pgfsetbuttcap%
\pgfsetmiterjoin%
\definecolor{currentfill}{rgb}{0.501961,0.501961,0.501961}%
\pgfsetfillcolor{currentfill}%
\pgfsetfillopacity{0.700000}%
\pgfsetlinewidth{0.000000pt}%
\definecolor{currentstroke}{rgb}{0.000000,0.000000,0.000000}%
\pgfsetstrokecolor{currentstroke}%
\pgfsetstrokeopacity{0.700000}%
\pgfsetdash{}{0pt}%
\pgfpathmoveto{\pgfqpoint{3.583807in}{0.550000in}}%
\pgfpathlineto{\pgfqpoint{3.611328in}{0.550000in}}%
\pgfpathlineto{\pgfqpoint{3.611328in}{0.687108in}}%
\pgfpathlineto{\pgfqpoint{3.583807in}{0.687108in}}%
\pgfpathlineto{\pgfqpoint{3.583807in}{0.550000in}}%
\pgfpathclose%
\pgfusepath{fill}%
\end{pgfscope}%
\begin{pgfscope}%
\pgfpathrectangle{\pgfqpoint{1.250000in}{0.550000in}}{\pgfqpoint{7.750000in}{3.850000in}}%
\pgfusepath{clip}%
\pgfsetbuttcap%
\pgfsetmiterjoin%
\definecolor{currentfill}{rgb}{0.501961,0.501961,0.501961}%
\pgfsetfillcolor{currentfill}%
\pgfsetfillopacity{0.700000}%
\pgfsetlinewidth{0.000000pt}%
\definecolor{currentstroke}{rgb}{0.000000,0.000000,0.000000}%
\pgfsetstrokecolor{currentstroke}%
\pgfsetstrokeopacity{0.700000}%
\pgfsetdash{}{0pt}%
\pgfpathmoveto{\pgfqpoint{3.611328in}{0.550000in}}%
\pgfpathlineto{\pgfqpoint{3.638849in}{0.550000in}}%
\pgfpathlineto{\pgfqpoint{3.638849in}{0.684355in}}%
\pgfpathlineto{\pgfqpoint{3.611328in}{0.684355in}}%
\pgfpathlineto{\pgfqpoint{3.611328in}{0.550000in}}%
\pgfpathclose%
\pgfusepath{fill}%
\end{pgfscope}%
\begin{pgfscope}%
\pgfpathrectangle{\pgfqpoint{1.250000in}{0.550000in}}{\pgfqpoint{7.750000in}{3.850000in}}%
\pgfusepath{clip}%
\pgfsetbuttcap%
\pgfsetmiterjoin%
\definecolor{currentfill}{rgb}{0.501961,0.501961,0.501961}%
\pgfsetfillcolor{currentfill}%
\pgfsetfillopacity{0.700000}%
\pgfsetlinewidth{0.000000pt}%
\definecolor{currentstroke}{rgb}{0.000000,0.000000,0.000000}%
\pgfsetstrokecolor{currentstroke}%
\pgfsetstrokeopacity{0.700000}%
\pgfsetdash{}{0pt}%
\pgfpathmoveto{\pgfqpoint{3.638849in}{0.550000in}}%
\pgfpathlineto{\pgfqpoint{3.666371in}{0.550000in}}%
\pgfpathlineto{\pgfqpoint{3.666371in}{0.683804in}}%
\pgfpathlineto{\pgfqpoint{3.638849in}{0.683804in}}%
\pgfpathlineto{\pgfqpoint{3.638849in}{0.550000in}}%
\pgfpathclose%
\pgfusepath{fill}%
\end{pgfscope}%
\begin{pgfscope}%
\pgfpathrectangle{\pgfqpoint{1.250000in}{0.550000in}}{\pgfqpoint{7.750000in}{3.850000in}}%
\pgfusepath{clip}%
\pgfsetbuttcap%
\pgfsetmiterjoin%
\definecolor{currentfill}{rgb}{0.501961,0.501961,0.501961}%
\pgfsetfillcolor{currentfill}%
\pgfsetfillopacity{0.700000}%
\pgfsetlinewidth{0.000000pt}%
\definecolor{currentstroke}{rgb}{0.000000,0.000000,0.000000}%
\pgfsetstrokecolor{currentstroke}%
\pgfsetstrokeopacity{0.700000}%
\pgfsetdash{}{0pt}%
\pgfpathmoveto{\pgfqpoint{3.666371in}{0.550000in}}%
\pgfpathlineto{\pgfqpoint{3.693892in}{0.550000in}}%
\pgfpathlineto{\pgfqpoint{3.693892in}{0.672241in}}%
\pgfpathlineto{\pgfqpoint{3.666371in}{0.672241in}}%
\pgfpathlineto{\pgfqpoint{3.666371in}{0.550000in}}%
\pgfpathclose%
\pgfusepath{fill}%
\end{pgfscope}%
\begin{pgfscope}%
\pgfpathrectangle{\pgfqpoint{1.250000in}{0.550000in}}{\pgfqpoint{7.750000in}{3.850000in}}%
\pgfusepath{clip}%
\pgfsetbuttcap%
\pgfsetmiterjoin%
\definecolor{currentfill}{rgb}{0.501961,0.501961,0.501961}%
\pgfsetfillcolor{currentfill}%
\pgfsetfillopacity{0.700000}%
\pgfsetlinewidth{0.000000pt}%
\definecolor{currentstroke}{rgb}{0.000000,0.000000,0.000000}%
\pgfsetstrokecolor{currentstroke}%
\pgfsetstrokeopacity{0.700000}%
\pgfsetdash{}{0pt}%
\pgfpathmoveto{\pgfqpoint{3.693892in}{0.550000in}}%
\pgfpathlineto{\pgfqpoint{3.721413in}{0.550000in}}%
\pgfpathlineto{\pgfqpoint{3.721413in}{0.679399in}}%
\pgfpathlineto{\pgfqpoint{3.693892in}{0.679399in}}%
\pgfpathlineto{\pgfqpoint{3.693892in}{0.550000in}}%
\pgfpathclose%
\pgfusepath{fill}%
\end{pgfscope}%
\begin{pgfscope}%
\pgfpathrectangle{\pgfqpoint{1.250000in}{0.550000in}}{\pgfqpoint{7.750000in}{3.850000in}}%
\pgfusepath{clip}%
\pgfsetbuttcap%
\pgfsetmiterjoin%
\definecolor{currentfill}{rgb}{0.501961,0.501961,0.501961}%
\pgfsetfillcolor{currentfill}%
\pgfsetfillopacity{0.700000}%
\pgfsetlinewidth{0.000000pt}%
\definecolor{currentstroke}{rgb}{0.000000,0.000000,0.000000}%
\pgfsetstrokecolor{currentstroke}%
\pgfsetstrokeopacity{0.700000}%
\pgfsetdash{}{0pt}%
\pgfpathmoveto{\pgfqpoint{3.721413in}{0.550000in}}%
\pgfpathlineto{\pgfqpoint{3.748935in}{0.550000in}}%
\pgfpathlineto{\pgfqpoint{3.748935in}{0.665082in}}%
\pgfpathlineto{\pgfqpoint{3.721413in}{0.665082in}}%
\pgfpathlineto{\pgfqpoint{3.721413in}{0.550000in}}%
\pgfpathclose%
\pgfusepath{fill}%
\end{pgfscope}%
\begin{pgfscope}%
\pgfpathrectangle{\pgfqpoint{1.250000in}{0.550000in}}{\pgfqpoint{7.750000in}{3.850000in}}%
\pgfusepath{clip}%
\pgfsetbuttcap%
\pgfsetmiterjoin%
\definecolor{currentfill}{rgb}{0.501961,0.501961,0.501961}%
\pgfsetfillcolor{currentfill}%
\pgfsetfillopacity{0.700000}%
\pgfsetlinewidth{0.000000pt}%
\definecolor{currentstroke}{rgb}{0.000000,0.000000,0.000000}%
\pgfsetstrokecolor{currentstroke}%
\pgfsetstrokeopacity{0.700000}%
\pgfsetdash{}{0pt}%
\pgfpathmoveto{\pgfqpoint{3.748935in}{0.550000in}}%
\pgfpathlineto{\pgfqpoint{3.776456in}{0.550000in}}%
\pgfpathlineto{\pgfqpoint{3.776456in}{0.667010in}}%
\pgfpathlineto{\pgfqpoint{3.748935in}{0.667010in}}%
\pgfpathlineto{\pgfqpoint{3.748935in}{0.550000in}}%
\pgfpathclose%
\pgfusepath{fill}%
\end{pgfscope}%
\begin{pgfscope}%
\pgfpathrectangle{\pgfqpoint{1.250000in}{0.550000in}}{\pgfqpoint{7.750000in}{3.850000in}}%
\pgfusepath{clip}%
\pgfsetbuttcap%
\pgfsetmiterjoin%
\definecolor{currentfill}{rgb}{0.501961,0.501961,0.501961}%
\pgfsetfillcolor{currentfill}%
\pgfsetfillopacity{0.700000}%
\pgfsetlinewidth{0.000000pt}%
\definecolor{currentstroke}{rgb}{0.000000,0.000000,0.000000}%
\pgfsetstrokecolor{currentstroke}%
\pgfsetstrokeopacity{0.700000}%
\pgfsetdash{}{0pt}%
\pgfpathmoveto{\pgfqpoint{3.776456in}{0.550000in}}%
\pgfpathlineto{\pgfqpoint{3.803977in}{0.550000in}}%
\pgfpathlineto{\pgfqpoint{3.803977in}{0.660402in}}%
\pgfpathlineto{\pgfqpoint{3.776456in}{0.660402in}}%
\pgfpathlineto{\pgfqpoint{3.776456in}{0.550000in}}%
\pgfpathclose%
\pgfusepath{fill}%
\end{pgfscope}%
\begin{pgfscope}%
\pgfpathrectangle{\pgfqpoint{1.250000in}{0.550000in}}{\pgfqpoint{7.750000in}{3.850000in}}%
\pgfusepath{clip}%
\pgfsetbuttcap%
\pgfsetmiterjoin%
\definecolor{currentfill}{rgb}{0.501961,0.501961,0.501961}%
\pgfsetfillcolor{currentfill}%
\pgfsetfillopacity{0.700000}%
\pgfsetlinewidth{0.000000pt}%
\definecolor{currentstroke}{rgb}{0.000000,0.000000,0.000000}%
\pgfsetstrokecolor{currentstroke}%
\pgfsetstrokeopacity{0.700000}%
\pgfsetdash{}{0pt}%
\pgfpathmoveto{\pgfqpoint{3.803977in}{0.550000in}}%
\pgfpathlineto{\pgfqpoint{3.831499in}{0.550000in}}%
\pgfpathlineto{\pgfqpoint{3.831499in}{0.651867in}}%
\pgfpathlineto{\pgfqpoint{3.803977in}{0.651867in}}%
\pgfpathlineto{\pgfqpoint{3.803977in}{0.550000in}}%
\pgfpathclose%
\pgfusepath{fill}%
\end{pgfscope}%
\begin{pgfscope}%
\pgfpathrectangle{\pgfqpoint{1.250000in}{0.550000in}}{\pgfqpoint{7.750000in}{3.850000in}}%
\pgfusepath{clip}%
\pgfsetbuttcap%
\pgfsetmiterjoin%
\definecolor{currentfill}{rgb}{0.501961,0.501961,0.501961}%
\pgfsetfillcolor{currentfill}%
\pgfsetfillopacity{0.700000}%
\pgfsetlinewidth{0.000000pt}%
\definecolor{currentstroke}{rgb}{0.000000,0.000000,0.000000}%
\pgfsetstrokecolor{currentstroke}%
\pgfsetstrokeopacity{0.700000}%
\pgfsetdash{}{0pt}%
\pgfpathmoveto{\pgfqpoint{3.831499in}{0.550000in}}%
\pgfpathlineto{\pgfqpoint{3.859020in}{0.550000in}}%
\pgfpathlineto{\pgfqpoint{3.859020in}{0.648288in}}%
\pgfpathlineto{\pgfqpoint{3.831499in}{0.648288in}}%
\pgfpathlineto{\pgfqpoint{3.831499in}{0.550000in}}%
\pgfpathclose%
\pgfusepath{fill}%
\end{pgfscope}%
\begin{pgfscope}%
\pgfpathrectangle{\pgfqpoint{1.250000in}{0.550000in}}{\pgfqpoint{7.750000in}{3.850000in}}%
\pgfusepath{clip}%
\pgfsetbuttcap%
\pgfsetmiterjoin%
\definecolor{currentfill}{rgb}{0.501961,0.501961,0.501961}%
\pgfsetfillcolor{currentfill}%
\pgfsetfillopacity{0.700000}%
\pgfsetlinewidth{0.000000pt}%
\definecolor{currentstroke}{rgb}{0.000000,0.000000,0.000000}%
\pgfsetstrokecolor{currentstroke}%
\pgfsetstrokeopacity{0.700000}%
\pgfsetdash{}{0pt}%
\pgfpathmoveto{\pgfqpoint{3.859020in}{0.550000in}}%
\pgfpathlineto{\pgfqpoint{3.886541in}{0.550000in}}%
\pgfpathlineto{\pgfqpoint{3.886541in}{0.647462in}}%
\pgfpathlineto{\pgfqpoint{3.859020in}{0.647462in}}%
\pgfpathlineto{\pgfqpoint{3.859020in}{0.550000in}}%
\pgfpathclose%
\pgfusepath{fill}%
\end{pgfscope}%
\begin{pgfscope}%
\pgfpathrectangle{\pgfqpoint{1.250000in}{0.550000in}}{\pgfqpoint{7.750000in}{3.850000in}}%
\pgfusepath{clip}%
\pgfsetbuttcap%
\pgfsetmiterjoin%
\definecolor{currentfill}{rgb}{0.501961,0.501961,0.501961}%
\pgfsetfillcolor{currentfill}%
\pgfsetfillopacity{0.700000}%
\pgfsetlinewidth{0.000000pt}%
\definecolor{currentstroke}{rgb}{0.000000,0.000000,0.000000}%
\pgfsetstrokecolor{currentstroke}%
\pgfsetstrokeopacity{0.700000}%
\pgfsetdash{}{0pt}%
\pgfpathmoveto{\pgfqpoint{3.886541in}{0.550000in}}%
\pgfpathlineto{\pgfqpoint{3.914062in}{0.550000in}}%
\pgfpathlineto{\pgfqpoint{3.914062in}{0.654620in}}%
\pgfpathlineto{\pgfqpoint{3.886541in}{0.654620in}}%
\pgfpathlineto{\pgfqpoint{3.886541in}{0.550000in}}%
\pgfpathclose%
\pgfusepath{fill}%
\end{pgfscope}%
\begin{pgfscope}%
\pgfpathrectangle{\pgfqpoint{1.250000in}{0.550000in}}{\pgfqpoint{7.750000in}{3.850000in}}%
\pgfusepath{clip}%
\pgfsetbuttcap%
\pgfsetmiterjoin%
\definecolor{currentfill}{rgb}{0.501961,0.501961,0.501961}%
\pgfsetfillcolor{currentfill}%
\pgfsetfillopacity{0.700000}%
\pgfsetlinewidth{0.000000pt}%
\definecolor{currentstroke}{rgb}{0.000000,0.000000,0.000000}%
\pgfsetstrokecolor{currentstroke}%
\pgfsetstrokeopacity{0.700000}%
\pgfsetdash{}{0pt}%
\pgfpathmoveto{\pgfqpoint{3.914062in}{0.550000in}}%
\pgfpathlineto{\pgfqpoint{3.941584in}{0.550000in}}%
\pgfpathlineto{\pgfqpoint{3.941584in}{0.651041in}}%
\pgfpathlineto{\pgfqpoint{3.914062in}{0.651041in}}%
\pgfpathlineto{\pgfqpoint{3.914062in}{0.550000in}}%
\pgfpathclose%
\pgfusepath{fill}%
\end{pgfscope}%
\begin{pgfscope}%
\pgfpathrectangle{\pgfqpoint{1.250000in}{0.550000in}}{\pgfqpoint{7.750000in}{3.850000in}}%
\pgfusepath{clip}%
\pgfsetbuttcap%
\pgfsetmiterjoin%
\definecolor{currentfill}{rgb}{0.501961,0.501961,0.501961}%
\pgfsetfillcolor{currentfill}%
\pgfsetfillopacity{0.700000}%
\pgfsetlinewidth{0.000000pt}%
\definecolor{currentstroke}{rgb}{0.000000,0.000000,0.000000}%
\pgfsetstrokecolor{currentstroke}%
\pgfsetstrokeopacity{0.700000}%
\pgfsetdash{}{0pt}%
\pgfpathmoveto{\pgfqpoint{3.941584in}{0.550000in}}%
\pgfpathlineto{\pgfqpoint{3.969105in}{0.550000in}}%
\pgfpathlineto{\pgfqpoint{3.969105in}{0.655722in}}%
\pgfpathlineto{\pgfqpoint{3.941584in}{0.655722in}}%
\pgfpathlineto{\pgfqpoint{3.941584in}{0.550000in}}%
\pgfpathclose%
\pgfusepath{fill}%
\end{pgfscope}%
\begin{pgfscope}%
\pgfpathrectangle{\pgfqpoint{1.250000in}{0.550000in}}{\pgfqpoint{7.750000in}{3.850000in}}%
\pgfusepath{clip}%
\pgfsetbuttcap%
\pgfsetmiterjoin%
\definecolor{currentfill}{rgb}{0.501961,0.501961,0.501961}%
\pgfsetfillcolor{currentfill}%
\pgfsetfillopacity{0.700000}%
\pgfsetlinewidth{0.000000pt}%
\definecolor{currentstroke}{rgb}{0.000000,0.000000,0.000000}%
\pgfsetstrokecolor{currentstroke}%
\pgfsetstrokeopacity{0.700000}%
\pgfsetdash{}{0pt}%
\pgfpathmoveto{\pgfqpoint{3.969105in}{0.550000in}}%
\pgfpathlineto{\pgfqpoint{3.996626in}{0.550000in}}%
\pgfpathlineto{\pgfqpoint{3.996626in}{0.654070in}}%
\pgfpathlineto{\pgfqpoint{3.969105in}{0.654070in}}%
\pgfpathlineto{\pgfqpoint{3.969105in}{0.550000in}}%
\pgfpathclose%
\pgfusepath{fill}%
\end{pgfscope}%
\begin{pgfscope}%
\pgfpathrectangle{\pgfqpoint{1.250000in}{0.550000in}}{\pgfqpoint{7.750000in}{3.850000in}}%
\pgfusepath{clip}%
\pgfsetbuttcap%
\pgfsetmiterjoin%
\definecolor{currentfill}{rgb}{0.501961,0.501961,0.501961}%
\pgfsetfillcolor{currentfill}%
\pgfsetfillopacity{0.700000}%
\pgfsetlinewidth{0.000000pt}%
\definecolor{currentstroke}{rgb}{0.000000,0.000000,0.000000}%
\pgfsetstrokecolor{currentstroke}%
\pgfsetstrokeopacity{0.700000}%
\pgfsetdash{}{0pt}%
\pgfpathmoveto{\pgfqpoint{3.996626in}{0.550000in}}%
\pgfpathlineto{\pgfqpoint{4.024148in}{0.550000in}}%
\pgfpathlineto{\pgfqpoint{4.024148in}{0.642782in}}%
\pgfpathlineto{\pgfqpoint{3.996626in}{0.642782in}}%
\pgfpathlineto{\pgfqpoint{3.996626in}{0.550000in}}%
\pgfpathclose%
\pgfusepath{fill}%
\end{pgfscope}%
\begin{pgfscope}%
\pgfpathrectangle{\pgfqpoint{1.250000in}{0.550000in}}{\pgfqpoint{7.750000in}{3.850000in}}%
\pgfusepath{clip}%
\pgfsetbuttcap%
\pgfsetmiterjoin%
\definecolor{currentfill}{rgb}{0.501961,0.501961,0.501961}%
\pgfsetfillcolor{currentfill}%
\pgfsetfillopacity{0.700000}%
\pgfsetlinewidth{0.000000pt}%
\definecolor{currentstroke}{rgb}{0.000000,0.000000,0.000000}%
\pgfsetstrokecolor{currentstroke}%
\pgfsetstrokeopacity{0.700000}%
\pgfsetdash{}{0pt}%
\pgfpathmoveto{\pgfqpoint{4.024148in}{0.550000in}}%
\pgfpathlineto{\pgfqpoint{4.051669in}{0.550000in}}%
\pgfpathlineto{\pgfqpoint{4.051669in}{0.647462in}}%
\pgfpathlineto{\pgfqpoint{4.024148in}{0.647462in}}%
\pgfpathlineto{\pgfqpoint{4.024148in}{0.550000in}}%
\pgfpathclose%
\pgfusepath{fill}%
\end{pgfscope}%
\begin{pgfscope}%
\pgfpathrectangle{\pgfqpoint{1.250000in}{0.550000in}}{\pgfqpoint{7.750000in}{3.850000in}}%
\pgfusepath{clip}%
\pgfsetbuttcap%
\pgfsetmiterjoin%
\definecolor{currentfill}{rgb}{0.501961,0.501961,0.501961}%
\pgfsetfillcolor{currentfill}%
\pgfsetfillopacity{0.700000}%
\pgfsetlinewidth{0.000000pt}%
\definecolor{currentstroke}{rgb}{0.000000,0.000000,0.000000}%
\pgfsetstrokecolor{currentstroke}%
\pgfsetstrokeopacity{0.700000}%
\pgfsetdash{}{0pt}%
\pgfpathmoveto{\pgfqpoint{4.051669in}{0.550000in}}%
\pgfpathlineto{\pgfqpoint{4.079190in}{0.550000in}}%
\pgfpathlineto{\pgfqpoint{4.079190in}{0.633696in}}%
\pgfpathlineto{\pgfqpoint{4.051669in}{0.633696in}}%
\pgfpathlineto{\pgfqpoint{4.051669in}{0.550000in}}%
\pgfpathclose%
\pgfusepath{fill}%
\end{pgfscope}%
\begin{pgfscope}%
\pgfpathrectangle{\pgfqpoint{1.250000in}{0.550000in}}{\pgfqpoint{7.750000in}{3.850000in}}%
\pgfusepath{clip}%
\pgfsetbuttcap%
\pgfsetmiterjoin%
\definecolor{currentfill}{rgb}{0.501961,0.501961,0.501961}%
\pgfsetfillcolor{currentfill}%
\pgfsetfillopacity{0.700000}%
\pgfsetlinewidth{0.000000pt}%
\definecolor{currentstroke}{rgb}{0.000000,0.000000,0.000000}%
\pgfsetstrokecolor{currentstroke}%
\pgfsetstrokeopacity{0.700000}%
\pgfsetdash{}{0pt}%
\pgfpathmoveto{\pgfqpoint{4.079190in}{0.550000in}}%
\pgfpathlineto{\pgfqpoint{4.106712in}{0.550000in}}%
\pgfpathlineto{\pgfqpoint{4.106712in}{0.654620in}}%
\pgfpathlineto{\pgfqpoint{4.079190in}{0.654620in}}%
\pgfpathlineto{\pgfqpoint{4.079190in}{0.550000in}}%
\pgfpathclose%
\pgfusepath{fill}%
\end{pgfscope}%
\begin{pgfscope}%
\pgfpathrectangle{\pgfqpoint{1.250000in}{0.550000in}}{\pgfqpoint{7.750000in}{3.850000in}}%
\pgfusepath{clip}%
\pgfsetbuttcap%
\pgfsetmiterjoin%
\definecolor{currentfill}{rgb}{0.501961,0.501961,0.501961}%
\pgfsetfillcolor{currentfill}%
\pgfsetfillopacity{0.700000}%
\pgfsetlinewidth{0.000000pt}%
\definecolor{currentstroke}{rgb}{0.000000,0.000000,0.000000}%
\pgfsetstrokecolor{currentstroke}%
\pgfsetstrokeopacity{0.700000}%
\pgfsetdash{}{0pt}%
\pgfpathmoveto{\pgfqpoint{4.106712in}{0.550000in}}%
\pgfpathlineto{\pgfqpoint{4.134233in}{0.550000in}}%
\pgfpathlineto{\pgfqpoint{4.134233in}{0.652142in}}%
\pgfpathlineto{\pgfqpoint{4.106712in}{0.652142in}}%
\pgfpathlineto{\pgfqpoint{4.106712in}{0.550000in}}%
\pgfpathclose%
\pgfusepath{fill}%
\end{pgfscope}%
\begin{pgfscope}%
\pgfpathrectangle{\pgfqpoint{1.250000in}{0.550000in}}{\pgfqpoint{7.750000in}{3.850000in}}%
\pgfusepath{clip}%
\pgfsetbuttcap%
\pgfsetmiterjoin%
\definecolor{currentfill}{rgb}{0.501961,0.501961,0.501961}%
\pgfsetfillcolor{currentfill}%
\pgfsetfillopacity{0.700000}%
\pgfsetlinewidth{0.000000pt}%
\definecolor{currentstroke}{rgb}{0.000000,0.000000,0.000000}%
\pgfsetstrokecolor{currentstroke}%
\pgfsetstrokeopacity{0.700000}%
\pgfsetdash{}{0pt}%
\pgfpathmoveto{\pgfqpoint{4.134233in}{0.550000in}}%
\pgfpathlineto{\pgfqpoint{4.161754in}{0.550000in}}%
\pgfpathlineto{\pgfqpoint{4.161754in}{0.646361in}}%
\pgfpathlineto{\pgfqpoint{4.134233in}{0.646361in}}%
\pgfpathlineto{\pgfqpoint{4.134233in}{0.550000in}}%
\pgfpathclose%
\pgfusepath{fill}%
\end{pgfscope}%
\begin{pgfscope}%
\pgfpathrectangle{\pgfqpoint{1.250000in}{0.550000in}}{\pgfqpoint{7.750000in}{3.850000in}}%
\pgfusepath{clip}%
\pgfsetbuttcap%
\pgfsetmiterjoin%
\definecolor{currentfill}{rgb}{0.501961,0.501961,0.501961}%
\pgfsetfillcolor{currentfill}%
\pgfsetfillopacity{0.700000}%
\pgfsetlinewidth{0.000000pt}%
\definecolor{currentstroke}{rgb}{0.000000,0.000000,0.000000}%
\pgfsetstrokecolor{currentstroke}%
\pgfsetstrokeopacity{0.700000}%
\pgfsetdash{}{0pt}%
\pgfpathmoveto{\pgfqpoint{4.161754in}{0.550000in}}%
\pgfpathlineto{\pgfqpoint{4.189276in}{0.550000in}}%
\pgfpathlineto{\pgfqpoint{4.189276in}{0.651592in}}%
\pgfpathlineto{\pgfqpoint{4.161754in}{0.651592in}}%
\pgfpathlineto{\pgfqpoint{4.161754in}{0.550000in}}%
\pgfpathclose%
\pgfusepath{fill}%
\end{pgfscope}%
\begin{pgfscope}%
\pgfpathrectangle{\pgfqpoint{1.250000in}{0.550000in}}{\pgfqpoint{7.750000in}{3.850000in}}%
\pgfusepath{clip}%
\pgfsetbuttcap%
\pgfsetmiterjoin%
\definecolor{currentfill}{rgb}{0.501961,0.501961,0.501961}%
\pgfsetfillcolor{currentfill}%
\pgfsetfillopacity{0.700000}%
\pgfsetlinewidth{0.000000pt}%
\definecolor{currentstroke}{rgb}{0.000000,0.000000,0.000000}%
\pgfsetstrokecolor{currentstroke}%
\pgfsetstrokeopacity{0.700000}%
\pgfsetdash{}{0pt}%
\pgfpathmoveto{\pgfqpoint{4.189276in}{0.550000in}}%
\pgfpathlineto{\pgfqpoint{4.216797in}{0.550000in}}%
\pgfpathlineto{\pgfqpoint{4.216797in}{0.653794in}}%
\pgfpathlineto{\pgfqpoint{4.189276in}{0.653794in}}%
\pgfpathlineto{\pgfqpoint{4.189276in}{0.550000in}}%
\pgfpathclose%
\pgfusepath{fill}%
\end{pgfscope}%
\begin{pgfscope}%
\pgfpathrectangle{\pgfqpoint{1.250000in}{0.550000in}}{\pgfqpoint{7.750000in}{3.850000in}}%
\pgfusepath{clip}%
\pgfsetbuttcap%
\pgfsetmiterjoin%
\definecolor{currentfill}{rgb}{0.501961,0.501961,0.501961}%
\pgfsetfillcolor{currentfill}%
\pgfsetfillopacity{0.700000}%
\pgfsetlinewidth{0.000000pt}%
\definecolor{currentstroke}{rgb}{0.000000,0.000000,0.000000}%
\pgfsetstrokecolor{currentstroke}%
\pgfsetstrokeopacity{0.700000}%
\pgfsetdash{}{0pt}%
\pgfpathmoveto{\pgfqpoint{4.216797in}{0.550000in}}%
\pgfpathlineto{\pgfqpoint{4.244318in}{0.550000in}}%
\pgfpathlineto{\pgfqpoint{4.244318in}{0.641130in}}%
\pgfpathlineto{\pgfqpoint{4.216797in}{0.641130in}}%
\pgfpathlineto{\pgfqpoint{4.216797in}{0.550000in}}%
\pgfpathclose%
\pgfusepath{fill}%
\end{pgfscope}%
\begin{pgfscope}%
\pgfpathrectangle{\pgfqpoint{1.250000in}{0.550000in}}{\pgfqpoint{7.750000in}{3.850000in}}%
\pgfusepath{clip}%
\pgfsetbuttcap%
\pgfsetmiterjoin%
\definecolor{currentfill}{rgb}{0.501961,0.501961,0.501961}%
\pgfsetfillcolor{currentfill}%
\pgfsetfillopacity{0.700000}%
\pgfsetlinewidth{0.000000pt}%
\definecolor{currentstroke}{rgb}{0.000000,0.000000,0.000000}%
\pgfsetstrokecolor{currentstroke}%
\pgfsetstrokeopacity{0.700000}%
\pgfsetdash{}{0pt}%
\pgfpathmoveto{\pgfqpoint{4.244318in}{0.550000in}}%
\pgfpathlineto{\pgfqpoint{4.271839in}{0.550000in}}%
\pgfpathlineto{\pgfqpoint{4.271839in}{0.650491in}}%
\pgfpathlineto{\pgfqpoint{4.244318in}{0.650491in}}%
\pgfpathlineto{\pgfqpoint{4.244318in}{0.550000in}}%
\pgfpathclose%
\pgfusepath{fill}%
\end{pgfscope}%
\begin{pgfscope}%
\pgfpathrectangle{\pgfqpoint{1.250000in}{0.550000in}}{\pgfqpoint{7.750000in}{3.850000in}}%
\pgfusepath{clip}%
\pgfsetbuttcap%
\pgfsetmiterjoin%
\definecolor{currentfill}{rgb}{0.501961,0.501961,0.501961}%
\pgfsetfillcolor{currentfill}%
\pgfsetfillopacity{0.700000}%
\pgfsetlinewidth{0.000000pt}%
\definecolor{currentstroke}{rgb}{0.000000,0.000000,0.000000}%
\pgfsetstrokecolor{currentstroke}%
\pgfsetstrokeopacity{0.700000}%
\pgfsetdash{}{0pt}%
\pgfpathmoveto{\pgfqpoint{4.271839in}{0.550000in}}%
\pgfpathlineto{\pgfqpoint{4.299361in}{0.550000in}}%
\pgfpathlineto{\pgfqpoint{4.299361in}{0.636174in}}%
\pgfpathlineto{\pgfqpoint{4.271839in}{0.636174in}}%
\pgfpathlineto{\pgfqpoint{4.271839in}{0.550000in}}%
\pgfpathclose%
\pgfusepath{fill}%
\end{pgfscope}%
\begin{pgfscope}%
\pgfpathrectangle{\pgfqpoint{1.250000in}{0.550000in}}{\pgfqpoint{7.750000in}{3.850000in}}%
\pgfusepath{clip}%
\pgfsetbuttcap%
\pgfsetmiterjoin%
\definecolor{currentfill}{rgb}{0.501961,0.501961,0.501961}%
\pgfsetfillcolor{currentfill}%
\pgfsetfillopacity{0.700000}%
\pgfsetlinewidth{0.000000pt}%
\definecolor{currentstroke}{rgb}{0.000000,0.000000,0.000000}%
\pgfsetstrokecolor{currentstroke}%
\pgfsetstrokeopacity{0.700000}%
\pgfsetdash{}{0pt}%
\pgfpathmoveto{\pgfqpoint{4.299361in}{0.550000in}}%
\pgfpathlineto{\pgfqpoint{4.326882in}{0.550000in}}%
\pgfpathlineto{\pgfqpoint{4.326882in}{0.643332in}}%
\pgfpathlineto{\pgfqpoint{4.299361in}{0.643332in}}%
\pgfpathlineto{\pgfqpoint{4.299361in}{0.550000in}}%
\pgfpathclose%
\pgfusepath{fill}%
\end{pgfscope}%
\begin{pgfscope}%
\pgfpathrectangle{\pgfqpoint{1.250000in}{0.550000in}}{\pgfqpoint{7.750000in}{3.850000in}}%
\pgfusepath{clip}%
\pgfsetbuttcap%
\pgfsetmiterjoin%
\definecolor{currentfill}{rgb}{0.501961,0.501961,0.501961}%
\pgfsetfillcolor{currentfill}%
\pgfsetfillopacity{0.700000}%
\pgfsetlinewidth{0.000000pt}%
\definecolor{currentstroke}{rgb}{0.000000,0.000000,0.000000}%
\pgfsetstrokecolor{currentstroke}%
\pgfsetstrokeopacity{0.700000}%
\pgfsetdash{}{0pt}%
\pgfpathmoveto{\pgfqpoint{4.326882in}{0.550000in}}%
\pgfpathlineto{\pgfqpoint{4.354403in}{0.550000in}}%
\pgfpathlineto{\pgfqpoint{4.354403in}{0.651317in}}%
\pgfpathlineto{\pgfqpoint{4.326882in}{0.651317in}}%
\pgfpathlineto{\pgfqpoint{4.326882in}{0.550000in}}%
\pgfpathclose%
\pgfusepath{fill}%
\end{pgfscope}%
\begin{pgfscope}%
\pgfpathrectangle{\pgfqpoint{1.250000in}{0.550000in}}{\pgfqpoint{7.750000in}{3.850000in}}%
\pgfusepath{clip}%
\pgfsetbuttcap%
\pgfsetmiterjoin%
\definecolor{currentfill}{rgb}{0.501961,0.501961,0.501961}%
\pgfsetfillcolor{currentfill}%
\pgfsetfillopacity{0.700000}%
\pgfsetlinewidth{0.000000pt}%
\definecolor{currentstroke}{rgb}{0.000000,0.000000,0.000000}%
\pgfsetstrokecolor{currentstroke}%
\pgfsetstrokeopacity{0.700000}%
\pgfsetdash{}{0pt}%
\pgfpathmoveto{\pgfqpoint{4.354403in}{0.550000in}}%
\pgfpathlineto{\pgfqpoint{4.381925in}{0.550000in}}%
\pgfpathlineto{\pgfqpoint{4.381925in}{0.645810in}}%
\pgfpathlineto{\pgfqpoint{4.354403in}{0.645810in}}%
\pgfpathlineto{\pgfqpoint{4.354403in}{0.550000in}}%
\pgfpathclose%
\pgfusepath{fill}%
\end{pgfscope}%
\begin{pgfscope}%
\pgfpathrectangle{\pgfqpoint{1.250000in}{0.550000in}}{\pgfqpoint{7.750000in}{3.850000in}}%
\pgfusepath{clip}%
\pgfsetbuttcap%
\pgfsetmiterjoin%
\definecolor{currentfill}{rgb}{0.501961,0.501961,0.501961}%
\pgfsetfillcolor{currentfill}%
\pgfsetfillopacity{0.700000}%
\pgfsetlinewidth{0.000000pt}%
\definecolor{currentstroke}{rgb}{0.000000,0.000000,0.000000}%
\pgfsetstrokecolor{currentstroke}%
\pgfsetstrokeopacity{0.700000}%
\pgfsetdash{}{0pt}%
\pgfpathmoveto{\pgfqpoint{4.381925in}{0.550000in}}%
\pgfpathlineto{\pgfqpoint{4.409446in}{0.550000in}}%
\pgfpathlineto{\pgfqpoint{4.409446in}{0.646636in}}%
\pgfpathlineto{\pgfqpoint{4.381925in}{0.646636in}}%
\pgfpathlineto{\pgfqpoint{4.381925in}{0.550000in}}%
\pgfpathclose%
\pgfusepath{fill}%
\end{pgfscope}%
\begin{pgfscope}%
\pgfpathrectangle{\pgfqpoint{1.250000in}{0.550000in}}{\pgfqpoint{7.750000in}{3.850000in}}%
\pgfusepath{clip}%
\pgfsetbuttcap%
\pgfsetmiterjoin%
\definecolor{currentfill}{rgb}{0.501961,0.501961,0.501961}%
\pgfsetfillcolor{currentfill}%
\pgfsetfillopacity{0.700000}%
\pgfsetlinewidth{0.000000pt}%
\definecolor{currentstroke}{rgb}{0.000000,0.000000,0.000000}%
\pgfsetstrokecolor{currentstroke}%
\pgfsetstrokeopacity{0.700000}%
\pgfsetdash{}{0pt}%
\pgfpathmoveto{\pgfqpoint{4.409446in}{0.550000in}}%
\pgfpathlineto{\pgfqpoint{4.436967in}{0.550000in}}%
\pgfpathlineto{\pgfqpoint{4.436967in}{0.654620in}}%
\pgfpathlineto{\pgfqpoint{4.409446in}{0.654620in}}%
\pgfpathlineto{\pgfqpoint{4.409446in}{0.550000in}}%
\pgfpathclose%
\pgfusepath{fill}%
\end{pgfscope}%
\begin{pgfscope}%
\pgfpathrectangle{\pgfqpoint{1.250000in}{0.550000in}}{\pgfqpoint{7.750000in}{3.850000in}}%
\pgfusepath{clip}%
\pgfsetbuttcap%
\pgfsetmiterjoin%
\definecolor{currentfill}{rgb}{0.501961,0.501961,0.501961}%
\pgfsetfillcolor{currentfill}%
\pgfsetfillopacity{0.700000}%
\pgfsetlinewidth{0.000000pt}%
\definecolor{currentstroke}{rgb}{0.000000,0.000000,0.000000}%
\pgfsetstrokecolor{currentstroke}%
\pgfsetstrokeopacity{0.700000}%
\pgfsetdash{}{0pt}%
\pgfpathmoveto{\pgfqpoint{4.436967in}{0.550000in}}%
\pgfpathlineto{\pgfqpoint{4.464489in}{0.550000in}}%
\pgfpathlineto{\pgfqpoint{4.464489in}{0.646085in}}%
\pgfpathlineto{\pgfqpoint{4.436967in}{0.646085in}}%
\pgfpathlineto{\pgfqpoint{4.436967in}{0.550000in}}%
\pgfpathclose%
\pgfusepath{fill}%
\end{pgfscope}%
\begin{pgfscope}%
\pgfpathrectangle{\pgfqpoint{1.250000in}{0.550000in}}{\pgfqpoint{7.750000in}{3.850000in}}%
\pgfusepath{clip}%
\pgfsetbuttcap%
\pgfsetmiterjoin%
\definecolor{currentfill}{rgb}{0.501961,0.501961,0.501961}%
\pgfsetfillcolor{currentfill}%
\pgfsetfillopacity{0.700000}%
\pgfsetlinewidth{0.000000pt}%
\definecolor{currentstroke}{rgb}{0.000000,0.000000,0.000000}%
\pgfsetstrokecolor{currentstroke}%
\pgfsetstrokeopacity{0.700000}%
\pgfsetdash{}{0pt}%
\pgfpathmoveto{\pgfqpoint{4.464489in}{0.550000in}}%
\pgfpathlineto{\pgfqpoint{4.492010in}{0.550000in}}%
\pgfpathlineto{\pgfqpoint{4.492010in}{0.642231in}}%
\pgfpathlineto{\pgfqpoint{4.464489in}{0.642231in}}%
\pgfpathlineto{\pgfqpoint{4.464489in}{0.550000in}}%
\pgfpathclose%
\pgfusepath{fill}%
\end{pgfscope}%
\begin{pgfscope}%
\pgfpathrectangle{\pgfqpoint{1.250000in}{0.550000in}}{\pgfqpoint{7.750000in}{3.850000in}}%
\pgfusepath{clip}%
\pgfsetbuttcap%
\pgfsetmiterjoin%
\definecolor{currentfill}{rgb}{0.501961,0.501961,0.501961}%
\pgfsetfillcolor{currentfill}%
\pgfsetfillopacity{0.700000}%
\pgfsetlinewidth{0.000000pt}%
\definecolor{currentstroke}{rgb}{0.000000,0.000000,0.000000}%
\pgfsetstrokecolor{currentstroke}%
\pgfsetstrokeopacity{0.700000}%
\pgfsetdash{}{0pt}%
\pgfpathmoveto{\pgfqpoint{4.492010in}{0.550000in}}%
\pgfpathlineto{\pgfqpoint{4.519531in}{0.550000in}}%
\pgfpathlineto{\pgfqpoint{4.519531in}{0.648563in}}%
\pgfpathlineto{\pgfqpoint{4.492010in}{0.648563in}}%
\pgfpathlineto{\pgfqpoint{4.492010in}{0.550000in}}%
\pgfpathclose%
\pgfusepath{fill}%
\end{pgfscope}%
\begin{pgfscope}%
\pgfpathrectangle{\pgfqpoint{1.250000in}{0.550000in}}{\pgfqpoint{7.750000in}{3.850000in}}%
\pgfusepath{clip}%
\pgfsetbuttcap%
\pgfsetmiterjoin%
\definecolor{currentfill}{rgb}{0.501961,0.501961,0.501961}%
\pgfsetfillcolor{currentfill}%
\pgfsetfillopacity{0.700000}%
\pgfsetlinewidth{0.000000pt}%
\definecolor{currentstroke}{rgb}{0.000000,0.000000,0.000000}%
\pgfsetstrokecolor{currentstroke}%
\pgfsetstrokeopacity{0.700000}%
\pgfsetdash{}{0pt}%
\pgfpathmoveto{\pgfqpoint{4.519531in}{0.550000in}}%
\pgfpathlineto{\pgfqpoint{4.547053in}{0.550000in}}%
\pgfpathlineto{\pgfqpoint{4.547053in}{0.659576in}}%
\pgfpathlineto{\pgfqpoint{4.519531in}{0.659576in}}%
\pgfpathlineto{\pgfqpoint{4.519531in}{0.550000in}}%
\pgfpathclose%
\pgfusepath{fill}%
\end{pgfscope}%
\begin{pgfscope}%
\pgfpathrectangle{\pgfqpoint{1.250000in}{0.550000in}}{\pgfqpoint{7.750000in}{3.850000in}}%
\pgfusepath{clip}%
\pgfsetbuttcap%
\pgfsetmiterjoin%
\definecolor{currentfill}{rgb}{0.501961,0.501961,0.501961}%
\pgfsetfillcolor{currentfill}%
\pgfsetfillopacity{0.700000}%
\pgfsetlinewidth{0.000000pt}%
\definecolor{currentstroke}{rgb}{0.000000,0.000000,0.000000}%
\pgfsetstrokecolor{currentstroke}%
\pgfsetstrokeopacity{0.700000}%
\pgfsetdash{}{0pt}%
\pgfpathmoveto{\pgfqpoint{4.547053in}{0.550000in}}%
\pgfpathlineto{\pgfqpoint{4.574574in}{0.550000in}}%
\pgfpathlineto{\pgfqpoint{4.574574in}{0.652968in}}%
\pgfpathlineto{\pgfqpoint{4.547053in}{0.652968in}}%
\pgfpathlineto{\pgfqpoint{4.547053in}{0.550000in}}%
\pgfpathclose%
\pgfusepath{fill}%
\end{pgfscope}%
\begin{pgfscope}%
\pgfpathrectangle{\pgfqpoint{1.250000in}{0.550000in}}{\pgfqpoint{7.750000in}{3.850000in}}%
\pgfusepath{clip}%
\pgfsetbuttcap%
\pgfsetmiterjoin%
\definecolor{currentfill}{rgb}{0.501961,0.501961,0.501961}%
\pgfsetfillcolor{currentfill}%
\pgfsetfillopacity{0.700000}%
\pgfsetlinewidth{0.000000pt}%
\definecolor{currentstroke}{rgb}{0.000000,0.000000,0.000000}%
\pgfsetstrokecolor{currentstroke}%
\pgfsetstrokeopacity{0.700000}%
\pgfsetdash{}{0pt}%
\pgfpathmoveto{\pgfqpoint{4.574574in}{0.550000in}}%
\pgfpathlineto{\pgfqpoint{4.602095in}{0.550000in}}%
\pgfpathlineto{\pgfqpoint{4.602095in}{0.652968in}}%
\pgfpathlineto{\pgfqpoint{4.574574in}{0.652968in}}%
\pgfpathlineto{\pgfqpoint{4.574574in}{0.550000in}}%
\pgfpathclose%
\pgfusepath{fill}%
\end{pgfscope}%
\begin{pgfscope}%
\pgfpathrectangle{\pgfqpoint{1.250000in}{0.550000in}}{\pgfqpoint{7.750000in}{3.850000in}}%
\pgfusepath{clip}%
\pgfsetbuttcap%
\pgfsetmiterjoin%
\definecolor{currentfill}{rgb}{0.501961,0.501961,0.501961}%
\pgfsetfillcolor{currentfill}%
\pgfsetfillopacity{0.700000}%
\pgfsetlinewidth{0.000000pt}%
\definecolor{currentstroke}{rgb}{0.000000,0.000000,0.000000}%
\pgfsetstrokecolor{currentstroke}%
\pgfsetstrokeopacity{0.700000}%
\pgfsetdash{}{0pt}%
\pgfpathmoveto{\pgfqpoint{4.602095in}{0.550000in}}%
\pgfpathlineto{\pgfqpoint{4.629616in}{0.550000in}}%
\pgfpathlineto{\pgfqpoint{4.629616in}{0.652693in}}%
\pgfpathlineto{\pgfqpoint{4.602095in}{0.652693in}}%
\pgfpathlineto{\pgfqpoint{4.602095in}{0.550000in}}%
\pgfpathclose%
\pgfusepath{fill}%
\end{pgfscope}%
\begin{pgfscope}%
\pgfpathrectangle{\pgfqpoint{1.250000in}{0.550000in}}{\pgfqpoint{7.750000in}{3.850000in}}%
\pgfusepath{clip}%
\pgfsetbuttcap%
\pgfsetmiterjoin%
\definecolor{currentfill}{rgb}{0.501961,0.501961,0.501961}%
\pgfsetfillcolor{currentfill}%
\pgfsetfillopacity{0.700000}%
\pgfsetlinewidth{0.000000pt}%
\definecolor{currentstroke}{rgb}{0.000000,0.000000,0.000000}%
\pgfsetstrokecolor{currentstroke}%
\pgfsetstrokeopacity{0.700000}%
\pgfsetdash{}{0pt}%
\pgfpathmoveto{\pgfqpoint{4.629616in}{0.550000in}}%
\pgfpathlineto{\pgfqpoint{4.657138in}{0.550000in}}%
\pgfpathlineto{\pgfqpoint{4.657138in}{0.651317in}}%
\pgfpathlineto{\pgfqpoint{4.629616in}{0.651317in}}%
\pgfpathlineto{\pgfqpoint{4.629616in}{0.550000in}}%
\pgfpathclose%
\pgfusepath{fill}%
\end{pgfscope}%
\begin{pgfscope}%
\pgfpathrectangle{\pgfqpoint{1.250000in}{0.550000in}}{\pgfqpoint{7.750000in}{3.850000in}}%
\pgfusepath{clip}%
\pgfsetbuttcap%
\pgfsetmiterjoin%
\definecolor{currentfill}{rgb}{0.501961,0.501961,0.501961}%
\pgfsetfillcolor{currentfill}%
\pgfsetfillopacity{0.700000}%
\pgfsetlinewidth{0.000000pt}%
\definecolor{currentstroke}{rgb}{0.000000,0.000000,0.000000}%
\pgfsetstrokecolor{currentstroke}%
\pgfsetstrokeopacity{0.700000}%
\pgfsetdash{}{0pt}%
\pgfpathmoveto{\pgfqpoint{4.657138in}{0.550000in}}%
\pgfpathlineto{\pgfqpoint{4.684659in}{0.550000in}}%
\pgfpathlineto{\pgfqpoint{4.684659in}{0.655997in}}%
\pgfpathlineto{\pgfqpoint{4.657138in}{0.655997in}}%
\pgfpathlineto{\pgfqpoint{4.657138in}{0.550000in}}%
\pgfpathclose%
\pgfusepath{fill}%
\end{pgfscope}%
\begin{pgfscope}%
\pgfpathrectangle{\pgfqpoint{1.250000in}{0.550000in}}{\pgfqpoint{7.750000in}{3.850000in}}%
\pgfusepath{clip}%
\pgfsetbuttcap%
\pgfsetmiterjoin%
\definecolor{currentfill}{rgb}{0.501961,0.501961,0.501961}%
\pgfsetfillcolor{currentfill}%
\pgfsetfillopacity{0.700000}%
\pgfsetlinewidth{0.000000pt}%
\definecolor{currentstroke}{rgb}{0.000000,0.000000,0.000000}%
\pgfsetstrokecolor{currentstroke}%
\pgfsetstrokeopacity{0.700000}%
\pgfsetdash{}{0pt}%
\pgfpathmoveto{\pgfqpoint{4.684659in}{0.550000in}}%
\pgfpathlineto{\pgfqpoint{4.712180in}{0.550000in}}%
\pgfpathlineto{\pgfqpoint{4.712180in}{0.648839in}}%
\pgfpathlineto{\pgfqpoint{4.684659in}{0.648839in}}%
\pgfpathlineto{\pgfqpoint{4.684659in}{0.550000in}}%
\pgfpathclose%
\pgfusepath{fill}%
\end{pgfscope}%
\begin{pgfscope}%
\pgfpathrectangle{\pgfqpoint{1.250000in}{0.550000in}}{\pgfqpoint{7.750000in}{3.850000in}}%
\pgfusepath{clip}%
\pgfsetbuttcap%
\pgfsetmiterjoin%
\definecolor{currentfill}{rgb}{0.501961,0.501961,0.501961}%
\pgfsetfillcolor{currentfill}%
\pgfsetfillopacity{0.700000}%
\pgfsetlinewidth{0.000000pt}%
\definecolor{currentstroke}{rgb}{0.000000,0.000000,0.000000}%
\pgfsetstrokecolor{currentstroke}%
\pgfsetstrokeopacity{0.700000}%
\pgfsetdash{}{0pt}%
\pgfpathmoveto{\pgfqpoint{4.712180in}{0.550000in}}%
\pgfpathlineto{\pgfqpoint{4.739702in}{0.550000in}}%
\pgfpathlineto{\pgfqpoint{4.739702in}{0.654620in}}%
\pgfpathlineto{\pgfqpoint{4.712180in}{0.654620in}}%
\pgfpathlineto{\pgfqpoint{4.712180in}{0.550000in}}%
\pgfpathclose%
\pgfusepath{fill}%
\end{pgfscope}%
\begin{pgfscope}%
\pgfpathrectangle{\pgfqpoint{1.250000in}{0.550000in}}{\pgfqpoint{7.750000in}{3.850000in}}%
\pgfusepath{clip}%
\pgfsetbuttcap%
\pgfsetmiterjoin%
\definecolor{currentfill}{rgb}{0.501961,0.501961,0.501961}%
\pgfsetfillcolor{currentfill}%
\pgfsetfillopacity{0.700000}%
\pgfsetlinewidth{0.000000pt}%
\definecolor{currentstroke}{rgb}{0.000000,0.000000,0.000000}%
\pgfsetstrokecolor{currentstroke}%
\pgfsetstrokeopacity{0.700000}%
\pgfsetdash{}{0pt}%
\pgfpathmoveto{\pgfqpoint{4.739702in}{0.550000in}}%
\pgfpathlineto{\pgfqpoint{4.767223in}{0.550000in}}%
\pgfpathlineto{\pgfqpoint{4.767223in}{0.647187in}}%
\pgfpathlineto{\pgfqpoint{4.739702in}{0.647187in}}%
\pgfpathlineto{\pgfqpoint{4.739702in}{0.550000in}}%
\pgfpathclose%
\pgfusepath{fill}%
\end{pgfscope}%
\begin{pgfscope}%
\pgfpathrectangle{\pgfqpoint{1.250000in}{0.550000in}}{\pgfqpoint{7.750000in}{3.850000in}}%
\pgfusepath{clip}%
\pgfsetbuttcap%
\pgfsetmiterjoin%
\definecolor{currentfill}{rgb}{0.501961,0.501961,0.501961}%
\pgfsetfillcolor{currentfill}%
\pgfsetfillopacity{0.700000}%
\pgfsetlinewidth{0.000000pt}%
\definecolor{currentstroke}{rgb}{0.000000,0.000000,0.000000}%
\pgfsetstrokecolor{currentstroke}%
\pgfsetstrokeopacity{0.700000}%
\pgfsetdash{}{0pt}%
\pgfpathmoveto{\pgfqpoint{4.767223in}{0.550000in}}%
\pgfpathlineto{\pgfqpoint{4.794744in}{0.550000in}}%
\pgfpathlineto{\pgfqpoint{4.794744in}{0.657373in}}%
\pgfpathlineto{\pgfqpoint{4.767223in}{0.657373in}}%
\pgfpathlineto{\pgfqpoint{4.767223in}{0.550000in}}%
\pgfpathclose%
\pgfusepath{fill}%
\end{pgfscope}%
\begin{pgfscope}%
\pgfpathrectangle{\pgfqpoint{1.250000in}{0.550000in}}{\pgfqpoint{7.750000in}{3.850000in}}%
\pgfusepath{clip}%
\pgfsetbuttcap%
\pgfsetmiterjoin%
\definecolor{currentfill}{rgb}{0.501961,0.501961,0.501961}%
\pgfsetfillcolor{currentfill}%
\pgfsetfillopacity{0.700000}%
\pgfsetlinewidth{0.000000pt}%
\definecolor{currentstroke}{rgb}{0.000000,0.000000,0.000000}%
\pgfsetstrokecolor{currentstroke}%
\pgfsetstrokeopacity{0.700000}%
\pgfsetdash{}{0pt}%
\pgfpathmoveto{\pgfqpoint{4.794744in}{0.550000in}}%
\pgfpathlineto{\pgfqpoint{4.822266in}{0.550000in}}%
\pgfpathlineto{\pgfqpoint{4.822266in}{0.655997in}}%
\pgfpathlineto{\pgfqpoint{4.794744in}{0.655997in}}%
\pgfpathlineto{\pgfqpoint{4.794744in}{0.550000in}}%
\pgfpathclose%
\pgfusepath{fill}%
\end{pgfscope}%
\begin{pgfscope}%
\pgfpathrectangle{\pgfqpoint{1.250000in}{0.550000in}}{\pgfqpoint{7.750000in}{3.850000in}}%
\pgfusepath{clip}%
\pgfsetbuttcap%
\pgfsetmiterjoin%
\definecolor{currentfill}{rgb}{0.501961,0.501961,0.501961}%
\pgfsetfillcolor{currentfill}%
\pgfsetfillopacity{0.700000}%
\pgfsetlinewidth{0.000000pt}%
\definecolor{currentstroke}{rgb}{0.000000,0.000000,0.000000}%
\pgfsetstrokecolor{currentstroke}%
\pgfsetstrokeopacity{0.700000}%
\pgfsetdash{}{0pt}%
\pgfpathmoveto{\pgfqpoint{4.822266in}{0.550000in}}%
\pgfpathlineto{\pgfqpoint{4.849787in}{0.550000in}}%
\pgfpathlineto{\pgfqpoint{4.849787in}{0.659301in}}%
\pgfpathlineto{\pgfqpoint{4.822266in}{0.659301in}}%
\pgfpathlineto{\pgfqpoint{4.822266in}{0.550000in}}%
\pgfpathclose%
\pgfusepath{fill}%
\end{pgfscope}%
\begin{pgfscope}%
\pgfpathrectangle{\pgfqpoint{1.250000in}{0.550000in}}{\pgfqpoint{7.750000in}{3.850000in}}%
\pgfusepath{clip}%
\pgfsetbuttcap%
\pgfsetmiterjoin%
\definecolor{currentfill}{rgb}{0.501961,0.501961,0.501961}%
\pgfsetfillcolor{currentfill}%
\pgfsetfillopacity{0.700000}%
\pgfsetlinewidth{0.000000pt}%
\definecolor{currentstroke}{rgb}{0.000000,0.000000,0.000000}%
\pgfsetstrokecolor{currentstroke}%
\pgfsetstrokeopacity{0.700000}%
\pgfsetdash{}{0pt}%
\pgfpathmoveto{\pgfqpoint{4.849787in}{0.550000in}}%
\pgfpathlineto{\pgfqpoint{4.877308in}{0.550000in}}%
\pgfpathlineto{\pgfqpoint{4.877308in}{0.658750in}}%
\pgfpathlineto{\pgfqpoint{4.849787in}{0.658750in}}%
\pgfpathlineto{\pgfqpoint{4.849787in}{0.550000in}}%
\pgfpathclose%
\pgfusepath{fill}%
\end{pgfscope}%
\begin{pgfscope}%
\pgfpathrectangle{\pgfqpoint{1.250000in}{0.550000in}}{\pgfqpoint{7.750000in}{3.850000in}}%
\pgfusepath{clip}%
\pgfsetbuttcap%
\pgfsetmiterjoin%
\definecolor{currentfill}{rgb}{0.501961,0.501961,0.501961}%
\pgfsetfillcolor{currentfill}%
\pgfsetfillopacity{0.700000}%
\pgfsetlinewidth{0.000000pt}%
\definecolor{currentstroke}{rgb}{0.000000,0.000000,0.000000}%
\pgfsetstrokecolor{currentstroke}%
\pgfsetstrokeopacity{0.700000}%
\pgfsetdash{}{0pt}%
\pgfpathmoveto{\pgfqpoint{4.877308in}{0.550000in}}%
\pgfpathlineto{\pgfqpoint{4.904830in}{0.550000in}}%
\pgfpathlineto{\pgfqpoint{4.904830in}{0.658750in}}%
\pgfpathlineto{\pgfqpoint{4.877308in}{0.658750in}}%
\pgfpathlineto{\pgfqpoint{4.877308in}{0.550000in}}%
\pgfpathclose%
\pgfusepath{fill}%
\end{pgfscope}%
\begin{pgfscope}%
\pgfpathrectangle{\pgfqpoint{1.250000in}{0.550000in}}{\pgfqpoint{7.750000in}{3.850000in}}%
\pgfusepath{clip}%
\pgfsetbuttcap%
\pgfsetmiterjoin%
\definecolor{currentfill}{rgb}{0.501961,0.501961,0.501961}%
\pgfsetfillcolor{currentfill}%
\pgfsetfillopacity{0.700000}%
\pgfsetlinewidth{0.000000pt}%
\definecolor{currentstroke}{rgb}{0.000000,0.000000,0.000000}%
\pgfsetstrokecolor{currentstroke}%
\pgfsetstrokeopacity{0.700000}%
\pgfsetdash{}{0pt}%
\pgfpathmoveto{\pgfqpoint{4.904830in}{0.550000in}}%
\pgfpathlineto{\pgfqpoint{4.932351in}{0.550000in}}%
\pgfpathlineto{\pgfqpoint{4.932351in}{0.651041in}}%
\pgfpathlineto{\pgfqpoint{4.904830in}{0.651041in}}%
\pgfpathlineto{\pgfqpoint{4.904830in}{0.550000in}}%
\pgfpathclose%
\pgfusepath{fill}%
\end{pgfscope}%
\begin{pgfscope}%
\pgfpathrectangle{\pgfqpoint{1.250000in}{0.550000in}}{\pgfqpoint{7.750000in}{3.850000in}}%
\pgfusepath{clip}%
\pgfsetbuttcap%
\pgfsetmiterjoin%
\definecolor{currentfill}{rgb}{0.501961,0.501961,0.501961}%
\pgfsetfillcolor{currentfill}%
\pgfsetfillopacity{0.700000}%
\pgfsetlinewidth{0.000000pt}%
\definecolor{currentstroke}{rgb}{0.000000,0.000000,0.000000}%
\pgfsetstrokecolor{currentstroke}%
\pgfsetstrokeopacity{0.700000}%
\pgfsetdash{}{0pt}%
\pgfpathmoveto{\pgfqpoint{4.932351in}{0.550000in}}%
\pgfpathlineto{\pgfqpoint{4.959872in}{0.550000in}}%
\pgfpathlineto{\pgfqpoint{4.959872in}{0.659576in}}%
\pgfpathlineto{\pgfqpoint{4.932351in}{0.659576in}}%
\pgfpathlineto{\pgfqpoint{4.932351in}{0.550000in}}%
\pgfpathclose%
\pgfusepath{fill}%
\end{pgfscope}%
\begin{pgfscope}%
\pgfpathrectangle{\pgfqpoint{1.250000in}{0.550000in}}{\pgfqpoint{7.750000in}{3.850000in}}%
\pgfusepath{clip}%
\pgfsetbuttcap%
\pgfsetmiterjoin%
\definecolor{currentfill}{rgb}{0.501961,0.501961,0.501961}%
\pgfsetfillcolor{currentfill}%
\pgfsetfillopacity{0.700000}%
\pgfsetlinewidth{0.000000pt}%
\definecolor{currentstroke}{rgb}{0.000000,0.000000,0.000000}%
\pgfsetstrokecolor{currentstroke}%
\pgfsetstrokeopacity{0.700000}%
\pgfsetdash{}{0pt}%
\pgfpathmoveto{\pgfqpoint{4.959872in}{0.550000in}}%
\pgfpathlineto{\pgfqpoint{4.987393in}{0.550000in}}%
\pgfpathlineto{\pgfqpoint{4.987393in}{0.652142in}}%
\pgfpathlineto{\pgfqpoint{4.959872in}{0.652142in}}%
\pgfpathlineto{\pgfqpoint{4.959872in}{0.550000in}}%
\pgfpathclose%
\pgfusepath{fill}%
\end{pgfscope}%
\begin{pgfscope}%
\pgfpathrectangle{\pgfqpoint{1.250000in}{0.550000in}}{\pgfqpoint{7.750000in}{3.850000in}}%
\pgfusepath{clip}%
\pgfsetbuttcap%
\pgfsetmiterjoin%
\definecolor{currentfill}{rgb}{0.501961,0.501961,0.501961}%
\pgfsetfillcolor{currentfill}%
\pgfsetfillopacity{0.700000}%
\pgfsetlinewidth{0.000000pt}%
\definecolor{currentstroke}{rgb}{0.000000,0.000000,0.000000}%
\pgfsetstrokecolor{currentstroke}%
\pgfsetstrokeopacity{0.700000}%
\pgfsetdash{}{0pt}%
\pgfpathmoveto{\pgfqpoint{4.987393in}{0.550000in}}%
\pgfpathlineto{\pgfqpoint{5.014915in}{0.550000in}}%
\pgfpathlineto{\pgfqpoint{5.014915in}{0.657098in}}%
\pgfpathlineto{\pgfqpoint{4.987393in}{0.657098in}}%
\pgfpathlineto{\pgfqpoint{4.987393in}{0.550000in}}%
\pgfpathclose%
\pgfusepath{fill}%
\end{pgfscope}%
\begin{pgfscope}%
\pgfpathrectangle{\pgfqpoint{1.250000in}{0.550000in}}{\pgfqpoint{7.750000in}{3.850000in}}%
\pgfusepath{clip}%
\pgfsetbuttcap%
\pgfsetmiterjoin%
\definecolor{currentfill}{rgb}{0.501961,0.501961,0.501961}%
\pgfsetfillcolor{currentfill}%
\pgfsetfillopacity{0.700000}%
\pgfsetlinewidth{0.000000pt}%
\definecolor{currentstroke}{rgb}{0.000000,0.000000,0.000000}%
\pgfsetstrokecolor{currentstroke}%
\pgfsetstrokeopacity{0.700000}%
\pgfsetdash{}{0pt}%
\pgfpathmoveto{\pgfqpoint{5.014915in}{0.550000in}}%
\pgfpathlineto{\pgfqpoint{5.042436in}{0.550000in}}%
\pgfpathlineto{\pgfqpoint{5.042436in}{0.646911in}}%
\pgfpathlineto{\pgfqpoint{5.014915in}{0.646911in}}%
\pgfpathlineto{\pgfqpoint{5.014915in}{0.550000in}}%
\pgfpathclose%
\pgfusepath{fill}%
\end{pgfscope}%
\begin{pgfscope}%
\pgfpathrectangle{\pgfqpoint{1.250000in}{0.550000in}}{\pgfqpoint{7.750000in}{3.850000in}}%
\pgfusepath{clip}%
\pgfsetbuttcap%
\pgfsetmiterjoin%
\definecolor{currentfill}{rgb}{0.501961,0.501961,0.501961}%
\pgfsetfillcolor{currentfill}%
\pgfsetfillopacity{0.700000}%
\pgfsetlinewidth{0.000000pt}%
\definecolor{currentstroke}{rgb}{0.000000,0.000000,0.000000}%
\pgfsetstrokecolor{currentstroke}%
\pgfsetstrokeopacity{0.700000}%
\pgfsetdash{}{0pt}%
\pgfpathmoveto{\pgfqpoint{5.042436in}{0.550000in}}%
\pgfpathlineto{\pgfqpoint{5.069957in}{0.550000in}}%
\pgfpathlineto{\pgfqpoint{5.069957in}{0.664256in}}%
\pgfpathlineto{\pgfqpoint{5.042436in}{0.664256in}}%
\pgfpathlineto{\pgfqpoint{5.042436in}{0.550000in}}%
\pgfpathclose%
\pgfusepath{fill}%
\end{pgfscope}%
\begin{pgfscope}%
\pgfpathrectangle{\pgfqpoint{1.250000in}{0.550000in}}{\pgfqpoint{7.750000in}{3.850000in}}%
\pgfusepath{clip}%
\pgfsetbuttcap%
\pgfsetmiterjoin%
\definecolor{currentfill}{rgb}{0.501961,0.501961,0.501961}%
\pgfsetfillcolor{currentfill}%
\pgfsetfillopacity{0.700000}%
\pgfsetlinewidth{0.000000pt}%
\definecolor{currentstroke}{rgb}{0.000000,0.000000,0.000000}%
\pgfsetstrokecolor{currentstroke}%
\pgfsetstrokeopacity{0.700000}%
\pgfsetdash{}{0pt}%
\pgfpathmoveto{\pgfqpoint{5.069957in}{0.550000in}}%
\pgfpathlineto{\pgfqpoint{5.097479in}{0.550000in}}%
\pgfpathlineto{\pgfqpoint{5.097479in}{0.657924in}}%
\pgfpathlineto{\pgfqpoint{5.069957in}{0.657924in}}%
\pgfpathlineto{\pgfqpoint{5.069957in}{0.550000in}}%
\pgfpathclose%
\pgfusepath{fill}%
\end{pgfscope}%
\begin{pgfscope}%
\pgfpathrectangle{\pgfqpoint{1.250000in}{0.550000in}}{\pgfqpoint{7.750000in}{3.850000in}}%
\pgfusepath{clip}%
\pgfsetbuttcap%
\pgfsetmiterjoin%
\definecolor{currentfill}{rgb}{0.501961,0.501961,0.501961}%
\pgfsetfillcolor{currentfill}%
\pgfsetfillopacity{0.700000}%
\pgfsetlinewidth{0.000000pt}%
\definecolor{currentstroke}{rgb}{0.000000,0.000000,0.000000}%
\pgfsetstrokecolor{currentstroke}%
\pgfsetstrokeopacity{0.700000}%
\pgfsetdash{}{0pt}%
\pgfpathmoveto{\pgfqpoint{5.097479in}{0.550000in}}%
\pgfpathlineto{\pgfqpoint{5.125000in}{0.550000in}}%
\pgfpathlineto{\pgfqpoint{5.125000in}{0.660127in}}%
\pgfpathlineto{\pgfqpoint{5.097479in}{0.660127in}}%
\pgfpathlineto{\pgfqpoint{5.097479in}{0.550000in}}%
\pgfpathclose%
\pgfusepath{fill}%
\end{pgfscope}%
\begin{pgfscope}%
\pgfpathrectangle{\pgfqpoint{1.250000in}{0.550000in}}{\pgfqpoint{7.750000in}{3.850000in}}%
\pgfusepath{clip}%
\pgfsetbuttcap%
\pgfsetmiterjoin%
\definecolor{currentfill}{rgb}{0.501961,0.501961,0.501961}%
\pgfsetfillcolor{currentfill}%
\pgfsetfillopacity{0.700000}%
\pgfsetlinewidth{0.000000pt}%
\definecolor{currentstroke}{rgb}{0.000000,0.000000,0.000000}%
\pgfsetstrokecolor{currentstroke}%
\pgfsetstrokeopacity{0.700000}%
\pgfsetdash{}{0pt}%
\pgfpathmoveto{\pgfqpoint{5.125000in}{0.550000in}}%
\pgfpathlineto{\pgfqpoint{5.152521in}{0.550000in}}%
\pgfpathlineto{\pgfqpoint{5.152521in}{0.662880in}}%
\pgfpathlineto{\pgfqpoint{5.125000in}{0.662880in}}%
\pgfpathlineto{\pgfqpoint{5.125000in}{0.550000in}}%
\pgfpathclose%
\pgfusepath{fill}%
\end{pgfscope}%
\begin{pgfscope}%
\pgfpathrectangle{\pgfqpoint{1.250000in}{0.550000in}}{\pgfqpoint{7.750000in}{3.850000in}}%
\pgfusepath{clip}%
\pgfsetbuttcap%
\pgfsetmiterjoin%
\definecolor{currentfill}{rgb}{0.501961,0.501961,0.501961}%
\pgfsetfillcolor{currentfill}%
\pgfsetfillopacity{0.700000}%
\pgfsetlinewidth{0.000000pt}%
\definecolor{currentstroke}{rgb}{0.000000,0.000000,0.000000}%
\pgfsetstrokecolor{currentstroke}%
\pgfsetstrokeopacity{0.700000}%
\pgfsetdash{}{0pt}%
\pgfpathmoveto{\pgfqpoint{5.152521in}{0.550000in}}%
\pgfpathlineto{\pgfqpoint{5.180043in}{0.550000in}}%
\pgfpathlineto{\pgfqpoint{5.180043in}{0.668937in}}%
\pgfpathlineto{\pgfqpoint{5.152521in}{0.668937in}}%
\pgfpathlineto{\pgfqpoint{5.152521in}{0.550000in}}%
\pgfpathclose%
\pgfusepath{fill}%
\end{pgfscope}%
\begin{pgfscope}%
\pgfpathrectangle{\pgfqpoint{1.250000in}{0.550000in}}{\pgfqpoint{7.750000in}{3.850000in}}%
\pgfusepath{clip}%
\pgfsetbuttcap%
\pgfsetmiterjoin%
\definecolor{currentfill}{rgb}{0.501961,0.501961,0.501961}%
\pgfsetfillcolor{currentfill}%
\pgfsetfillopacity{0.700000}%
\pgfsetlinewidth{0.000000pt}%
\definecolor{currentstroke}{rgb}{0.000000,0.000000,0.000000}%
\pgfsetstrokecolor{currentstroke}%
\pgfsetstrokeopacity{0.700000}%
\pgfsetdash{}{0pt}%
\pgfpathmoveto{\pgfqpoint{5.180043in}{0.550000in}}%
\pgfpathlineto{\pgfqpoint{5.207564in}{0.550000in}}%
\pgfpathlineto{\pgfqpoint{5.207564in}{0.678298in}}%
\pgfpathlineto{\pgfqpoint{5.180043in}{0.678298in}}%
\pgfpathlineto{\pgfqpoint{5.180043in}{0.550000in}}%
\pgfpathclose%
\pgfusepath{fill}%
\end{pgfscope}%
\begin{pgfscope}%
\pgfpathrectangle{\pgfqpoint{1.250000in}{0.550000in}}{\pgfqpoint{7.750000in}{3.850000in}}%
\pgfusepath{clip}%
\pgfsetbuttcap%
\pgfsetmiterjoin%
\definecolor{currentfill}{rgb}{0.501961,0.501961,0.501961}%
\pgfsetfillcolor{currentfill}%
\pgfsetfillopacity{0.700000}%
\pgfsetlinewidth{0.000000pt}%
\definecolor{currentstroke}{rgb}{0.000000,0.000000,0.000000}%
\pgfsetstrokecolor{currentstroke}%
\pgfsetstrokeopacity{0.700000}%
\pgfsetdash{}{0pt}%
\pgfpathmoveto{\pgfqpoint{5.207564in}{0.550000in}}%
\pgfpathlineto{\pgfqpoint{5.235085in}{0.550000in}}%
\pgfpathlineto{\pgfqpoint{5.235085in}{0.680500in}}%
\pgfpathlineto{\pgfqpoint{5.207564in}{0.680500in}}%
\pgfpathlineto{\pgfqpoint{5.207564in}{0.550000in}}%
\pgfpathclose%
\pgfusepath{fill}%
\end{pgfscope}%
\begin{pgfscope}%
\pgfpathrectangle{\pgfqpoint{1.250000in}{0.550000in}}{\pgfqpoint{7.750000in}{3.850000in}}%
\pgfusepath{clip}%
\pgfsetbuttcap%
\pgfsetmiterjoin%
\definecolor{currentfill}{rgb}{0.501961,0.501961,0.501961}%
\pgfsetfillcolor{currentfill}%
\pgfsetfillopacity{0.700000}%
\pgfsetlinewidth{0.000000pt}%
\definecolor{currentstroke}{rgb}{0.000000,0.000000,0.000000}%
\pgfsetstrokecolor{currentstroke}%
\pgfsetstrokeopacity{0.700000}%
\pgfsetdash{}{0pt}%
\pgfpathmoveto{\pgfqpoint{5.235085in}{0.550000in}}%
\pgfpathlineto{\pgfqpoint{5.262607in}{0.550000in}}%
\pgfpathlineto{\pgfqpoint{5.262607in}{0.677472in}}%
\pgfpathlineto{\pgfqpoint{5.235085in}{0.677472in}}%
\pgfpathlineto{\pgfqpoint{5.235085in}{0.550000in}}%
\pgfpathclose%
\pgfusepath{fill}%
\end{pgfscope}%
\begin{pgfscope}%
\pgfpathrectangle{\pgfqpoint{1.250000in}{0.550000in}}{\pgfqpoint{7.750000in}{3.850000in}}%
\pgfusepath{clip}%
\pgfsetbuttcap%
\pgfsetmiterjoin%
\definecolor{currentfill}{rgb}{0.501961,0.501961,0.501961}%
\pgfsetfillcolor{currentfill}%
\pgfsetfillopacity{0.700000}%
\pgfsetlinewidth{0.000000pt}%
\definecolor{currentstroke}{rgb}{0.000000,0.000000,0.000000}%
\pgfsetstrokecolor{currentstroke}%
\pgfsetstrokeopacity{0.700000}%
\pgfsetdash{}{0pt}%
\pgfpathmoveto{\pgfqpoint{5.262607in}{0.550000in}}%
\pgfpathlineto{\pgfqpoint{5.290128in}{0.550000in}}%
\pgfpathlineto{\pgfqpoint{5.290128in}{0.677196in}}%
\pgfpathlineto{\pgfqpoint{5.262607in}{0.677196in}}%
\pgfpathlineto{\pgfqpoint{5.262607in}{0.550000in}}%
\pgfpathclose%
\pgfusepath{fill}%
\end{pgfscope}%
\begin{pgfscope}%
\pgfpathrectangle{\pgfqpoint{1.250000in}{0.550000in}}{\pgfqpoint{7.750000in}{3.850000in}}%
\pgfusepath{clip}%
\pgfsetbuttcap%
\pgfsetmiterjoin%
\definecolor{currentfill}{rgb}{0.501961,0.501961,0.501961}%
\pgfsetfillcolor{currentfill}%
\pgfsetfillopacity{0.700000}%
\pgfsetlinewidth{0.000000pt}%
\definecolor{currentstroke}{rgb}{0.000000,0.000000,0.000000}%
\pgfsetstrokecolor{currentstroke}%
\pgfsetstrokeopacity{0.700000}%
\pgfsetdash{}{0pt}%
\pgfpathmoveto{\pgfqpoint{5.290128in}{0.550000in}}%
\pgfpathlineto{\pgfqpoint{5.317649in}{0.550000in}}%
\pgfpathlineto{\pgfqpoint{5.317649in}{0.683253in}}%
\pgfpathlineto{\pgfqpoint{5.290128in}{0.683253in}}%
\pgfpathlineto{\pgfqpoint{5.290128in}{0.550000in}}%
\pgfpathclose%
\pgfusepath{fill}%
\end{pgfscope}%
\begin{pgfscope}%
\pgfpathrectangle{\pgfqpoint{1.250000in}{0.550000in}}{\pgfqpoint{7.750000in}{3.850000in}}%
\pgfusepath{clip}%
\pgfsetbuttcap%
\pgfsetmiterjoin%
\definecolor{currentfill}{rgb}{0.501961,0.501961,0.501961}%
\pgfsetfillcolor{currentfill}%
\pgfsetfillopacity{0.700000}%
\pgfsetlinewidth{0.000000pt}%
\definecolor{currentstroke}{rgb}{0.000000,0.000000,0.000000}%
\pgfsetstrokecolor{currentstroke}%
\pgfsetstrokeopacity{0.700000}%
\pgfsetdash{}{0pt}%
\pgfpathmoveto{\pgfqpoint{5.317649in}{0.550000in}}%
\pgfpathlineto{\pgfqpoint{5.345170in}{0.550000in}}%
\pgfpathlineto{\pgfqpoint{5.345170in}{0.675269in}}%
\pgfpathlineto{\pgfqpoint{5.317649in}{0.675269in}}%
\pgfpathlineto{\pgfqpoint{5.317649in}{0.550000in}}%
\pgfpathclose%
\pgfusepath{fill}%
\end{pgfscope}%
\begin{pgfscope}%
\pgfpathrectangle{\pgfqpoint{1.250000in}{0.550000in}}{\pgfqpoint{7.750000in}{3.850000in}}%
\pgfusepath{clip}%
\pgfsetbuttcap%
\pgfsetmiterjoin%
\definecolor{currentfill}{rgb}{0.501961,0.501961,0.501961}%
\pgfsetfillcolor{currentfill}%
\pgfsetfillopacity{0.700000}%
\pgfsetlinewidth{0.000000pt}%
\definecolor{currentstroke}{rgb}{0.000000,0.000000,0.000000}%
\pgfsetstrokecolor{currentstroke}%
\pgfsetstrokeopacity{0.700000}%
\pgfsetdash{}{0pt}%
\pgfpathmoveto{\pgfqpoint{5.345170in}{0.550000in}}%
\pgfpathlineto{\pgfqpoint{5.372692in}{0.550000in}}%
\pgfpathlineto{\pgfqpoint{5.372692in}{0.679949in}}%
\pgfpathlineto{\pgfqpoint{5.345170in}{0.679949in}}%
\pgfpathlineto{\pgfqpoint{5.345170in}{0.550000in}}%
\pgfpathclose%
\pgfusepath{fill}%
\end{pgfscope}%
\begin{pgfscope}%
\pgfpathrectangle{\pgfqpoint{1.250000in}{0.550000in}}{\pgfqpoint{7.750000in}{3.850000in}}%
\pgfusepath{clip}%
\pgfsetbuttcap%
\pgfsetmiterjoin%
\definecolor{currentfill}{rgb}{0.501961,0.501961,0.501961}%
\pgfsetfillcolor{currentfill}%
\pgfsetfillopacity{0.700000}%
\pgfsetlinewidth{0.000000pt}%
\definecolor{currentstroke}{rgb}{0.000000,0.000000,0.000000}%
\pgfsetstrokecolor{currentstroke}%
\pgfsetstrokeopacity{0.700000}%
\pgfsetdash{}{0pt}%
\pgfpathmoveto{\pgfqpoint{5.372692in}{0.550000in}}%
\pgfpathlineto{\pgfqpoint{5.400213in}{0.550000in}}%
\pgfpathlineto{\pgfqpoint{5.400213in}{0.679949in}}%
\pgfpathlineto{\pgfqpoint{5.372692in}{0.679949in}}%
\pgfpathlineto{\pgfqpoint{5.372692in}{0.550000in}}%
\pgfpathclose%
\pgfusepath{fill}%
\end{pgfscope}%
\begin{pgfscope}%
\pgfpathrectangle{\pgfqpoint{1.250000in}{0.550000in}}{\pgfqpoint{7.750000in}{3.850000in}}%
\pgfusepath{clip}%
\pgfsetbuttcap%
\pgfsetmiterjoin%
\definecolor{currentfill}{rgb}{0.501961,0.501961,0.501961}%
\pgfsetfillcolor{currentfill}%
\pgfsetfillopacity{0.700000}%
\pgfsetlinewidth{0.000000pt}%
\definecolor{currentstroke}{rgb}{0.000000,0.000000,0.000000}%
\pgfsetstrokecolor{currentstroke}%
\pgfsetstrokeopacity{0.700000}%
\pgfsetdash{}{0pt}%
\pgfpathmoveto{\pgfqpoint{5.400213in}{0.550000in}}%
\pgfpathlineto{\pgfqpoint{5.427734in}{0.550000in}}%
\pgfpathlineto{\pgfqpoint{5.427734in}{0.684079in}}%
\pgfpathlineto{\pgfqpoint{5.400213in}{0.684079in}}%
\pgfpathlineto{\pgfqpoint{5.400213in}{0.550000in}}%
\pgfpathclose%
\pgfusepath{fill}%
\end{pgfscope}%
\begin{pgfscope}%
\pgfpathrectangle{\pgfqpoint{1.250000in}{0.550000in}}{\pgfqpoint{7.750000in}{3.850000in}}%
\pgfusepath{clip}%
\pgfsetbuttcap%
\pgfsetmiterjoin%
\definecolor{currentfill}{rgb}{0.501961,0.501961,0.501961}%
\pgfsetfillcolor{currentfill}%
\pgfsetfillopacity{0.700000}%
\pgfsetlinewidth{0.000000pt}%
\definecolor{currentstroke}{rgb}{0.000000,0.000000,0.000000}%
\pgfsetstrokecolor{currentstroke}%
\pgfsetstrokeopacity{0.700000}%
\pgfsetdash{}{0pt}%
\pgfpathmoveto{\pgfqpoint{5.427734in}{0.550000in}}%
\pgfpathlineto{\pgfqpoint{5.455256in}{0.550000in}}%
\pgfpathlineto{\pgfqpoint{5.455256in}{0.697570in}}%
\pgfpathlineto{\pgfqpoint{5.427734in}{0.697570in}}%
\pgfpathlineto{\pgfqpoint{5.427734in}{0.550000in}}%
\pgfpathclose%
\pgfusepath{fill}%
\end{pgfscope}%
\begin{pgfscope}%
\pgfpathrectangle{\pgfqpoint{1.250000in}{0.550000in}}{\pgfqpoint{7.750000in}{3.850000in}}%
\pgfusepath{clip}%
\pgfsetbuttcap%
\pgfsetmiterjoin%
\definecolor{currentfill}{rgb}{0.501961,0.501961,0.501961}%
\pgfsetfillcolor{currentfill}%
\pgfsetfillopacity{0.700000}%
\pgfsetlinewidth{0.000000pt}%
\definecolor{currentstroke}{rgb}{0.000000,0.000000,0.000000}%
\pgfsetstrokecolor{currentstroke}%
\pgfsetstrokeopacity{0.700000}%
\pgfsetdash{}{0pt}%
\pgfpathmoveto{\pgfqpoint{5.455256in}{0.550000in}}%
\pgfpathlineto{\pgfqpoint{5.482777in}{0.550000in}}%
\pgfpathlineto{\pgfqpoint{5.482777in}{0.686282in}}%
\pgfpathlineto{\pgfqpoint{5.455256in}{0.686282in}}%
\pgfpathlineto{\pgfqpoint{5.455256in}{0.550000in}}%
\pgfpathclose%
\pgfusepath{fill}%
\end{pgfscope}%
\begin{pgfscope}%
\pgfpathrectangle{\pgfqpoint{1.250000in}{0.550000in}}{\pgfqpoint{7.750000in}{3.850000in}}%
\pgfusepath{clip}%
\pgfsetbuttcap%
\pgfsetmiterjoin%
\definecolor{currentfill}{rgb}{0.501961,0.501961,0.501961}%
\pgfsetfillcolor{currentfill}%
\pgfsetfillopacity{0.700000}%
\pgfsetlinewidth{0.000000pt}%
\definecolor{currentstroke}{rgb}{0.000000,0.000000,0.000000}%
\pgfsetstrokecolor{currentstroke}%
\pgfsetstrokeopacity{0.700000}%
\pgfsetdash{}{0pt}%
\pgfpathmoveto{\pgfqpoint{5.482777in}{0.550000in}}%
\pgfpathlineto{\pgfqpoint{5.510298in}{0.550000in}}%
\pgfpathlineto{\pgfqpoint{5.510298in}{0.682427in}}%
\pgfpathlineto{\pgfqpoint{5.482777in}{0.682427in}}%
\pgfpathlineto{\pgfqpoint{5.482777in}{0.550000in}}%
\pgfpathclose%
\pgfusepath{fill}%
\end{pgfscope}%
\begin{pgfscope}%
\pgfpathrectangle{\pgfqpoint{1.250000in}{0.550000in}}{\pgfqpoint{7.750000in}{3.850000in}}%
\pgfusepath{clip}%
\pgfsetbuttcap%
\pgfsetmiterjoin%
\definecolor{currentfill}{rgb}{0.501961,0.501961,0.501961}%
\pgfsetfillcolor{currentfill}%
\pgfsetfillopacity{0.700000}%
\pgfsetlinewidth{0.000000pt}%
\definecolor{currentstroke}{rgb}{0.000000,0.000000,0.000000}%
\pgfsetstrokecolor{currentstroke}%
\pgfsetstrokeopacity{0.700000}%
\pgfsetdash{}{0pt}%
\pgfpathmoveto{\pgfqpoint{5.510298in}{0.550000in}}%
\pgfpathlineto{\pgfqpoint{5.537820in}{0.550000in}}%
\pgfpathlineto{\pgfqpoint{5.537820in}{0.683529in}}%
\pgfpathlineto{\pgfqpoint{5.510298in}{0.683529in}}%
\pgfpathlineto{\pgfqpoint{5.510298in}{0.550000in}}%
\pgfpathclose%
\pgfusepath{fill}%
\end{pgfscope}%
\begin{pgfscope}%
\pgfpathrectangle{\pgfqpoint{1.250000in}{0.550000in}}{\pgfqpoint{7.750000in}{3.850000in}}%
\pgfusepath{clip}%
\pgfsetbuttcap%
\pgfsetmiterjoin%
\definecolor{currentfill}{rgb}{0.501961,0.501961,0.501961}%
\pgfsetfillcolor{currentfill}%
\pgfsetfillopacity{0.700000}%
\pgfsetlinewidth{0.000000pt}%
\definecolor{currentstroke}{rgb}{0.000000,0.000000,0.000000}%
\pgfsetstrokecolor{currentstroke}%
\pgfsetstrokeopacity{0.700000}%
\pgfsetdash{}{0pt}%
\pgfpathmoveto{\pgfqpoint{5.537820in}{0.550000in}}%
\pgfpathlineto{\pgfqpoint{5.565341in}{0.550000in}}%
\pgfpathlineto{\pgfqpoint{5.565341in}{0.684079in}}%
\pgfpathlineto{\pgfqpoint{5.537820in}{0.684079in}}%
\pgfpathlineto{\pgfqpoint{5.537820in}{0.550000in}}%
\pgfpathclose%
\pgfusepath{fill}%
\end{pgfscope}%
\begin{pgfscope}%
\pgfpathrectangle{\pgfqpoint{1.250000in}{0.550000in}}{\pgfqpoint{7.750000in}{3.850000in}}%
\pgfusepath{clip}%
\pgfsetbuttcap%
\pgfsetmiterjoin%
\definecolor{currentfill}{rgb}{0.501961,0.501961,0.501961}%
\pgfsetfillcolor{currentfill}%
\pgfsetfillopacity{0.700000}%
\pgfsetlinewidth{0.000000pt}%
\definecolor{currentstroke}{rgb}{0.000000,0.000000,0.000000}%
\pgfsetstrokecolor{currentstroke}%
\pgfsetstrokeopacity{0.700000}%
\pgfsetdash{}{0pt}%
\pgfpathmoveto{\pgfqpoint{5.565341in}{0.550000in}}%
\pgfpathlineto{\pgfqpoint{5.592862in}{0.550000in}}%
\pgfpathlineto{\pgfqpoint{5.592862in}{0.678022in}}%
\pgfpathlineto{\pgfqpoint{5.565341in}{0.678022in}}%
\pgfpathlineto{\pgfqpoint{5.565341in}{0.550000in}}%
\pgfpathclose%
\pgfusepath{fill}%
\end{pgfscope}%
\begin{pgfscope}%
\pgfpathrectangle{\pgfqpoint{1.250000in}{0.550000in}}{\pgfqpoint{7.750000in}{3.850000in}}%
\pgfusepath{clip}%
\pgfsetbuttcap%
\pgfsetmiterjoin%
\definecolor{currentfill}{rgb}{0.501961,0.501961,0.501961}%
\pgfsetfillcolor{currentfill}%
\pgfsetfillopacity{0.700000}%
\pgfsetlinewidth{0.000000pt}%
\definecolor{currentstroke}{rgb}{0.000000,0.000000,0.000000}%
\pgfsetstrokecolor{currentstroke}%
\pgfsetstrokeopacity{0.700000}%
\pgfsetdash{}{0pt}%
\pgfpathmoveto{\pgfqpoint{5.592862in}{0.550000in}}%
\pgfpathlineto{\pgfqpoint{5.620384in}{0.550000in}}%
\pgfpathlineto{\pgfqpoint{5.620384in}{0.692063in}}%
\pgfpathlineto{\pgfqpoint{5.592862in}{0.692063in}}%
\pgfpathlineto{\pgfqpoint{5.592862in}{0.550000in}}%
\pgfpathclose%
\pgfusepath{fill}%
\end{pgfscope}%
\begin{pgfscope}%
\pgfpathrectangle{\pgfqpoint{1.250000in}{0.550000in}}{\pgfqpoint{7.750000in}{3.850000in}}%
\pgfusepath{clip}%
\pgfsetbuttcap%
\pgfsetmiterjoin%
\definecolor{currentfill}{rgb}{0.501961,0.501961,0.501961}%
\pgfsetfillcolor{currentfill}%
\pgfsetfillopacity{0.700000}%
\pgfsetlinewidth{0.000000pt}%
\definecolor{currentstroke}{rgb}{0.000000,0.000000,0.000000}%
\pgfsetstrokecolor{currentstroke}%
\pgfsetstrokeopacity{0.700000}%
\pgfsetdash{}{0pt}%
\pgfpathmoveto{\pgfqpoint{5.620384in}{0.550000in}}%
\pgfpathlineto{\pgfqpoint{5.647905in}{0.550000in}}%
\pgfpathlineto{\pgfqpoint{5.647905in}{0.681051in}}%
\pgfpathlineto{\pgfqpoint{5.620384in}{0.681051in}}%
\pgfpathlineto{\pgfqpoint{5.620384in}{0.550000in}}%
\pgfpathclose%
\pgfusepath{fill}%
\end{pgfscope}%
\begin{pgfscope}%
\pgfpathrectangle{\pgfqpoint{1.250000in}{0.550000in}}{\pgfqpoint{7.750000in}{3.850000in}}%
\pgfusepath{clip}%
\pgfsetbuttcap%
\pgfsetmiterjoin%
\definecolor{currentfill}{rgb}{0.501961,0.501961,0.501961}%
\pgfsetfillcolor{currentfill}%
\pgfsetfillopacity{0.700000}%
\pgfsetlinewidth{0.000000pt}%
\definecolor{currentstroke}{rgb}{0.000000,0.000000,0.000000}%
\pgfsetstrokecolor{currentstroke}%
\pgfsetstrokeopacity{0.700000}%
\pgfsetdash{}{0pt}%
\pgfpathmoveto{\pgfqpoint{5.647905in}{0.550000in}}%
\pgfpathlineto{\pgfqpoint{5.675426in}{0.550000in}}%
\pgfpathlineto{\pgfqpoint{5.675426in}{0.697570in}}%
\pgfpathlineto{\pgfqpoint{5.647905in}{0.697570in}}%
\pgfpathlineto{\pgfqpoint{5.647905in}{0.550000in}}%
\pgfpathclose%
\pgfusepath{fill}%
\end{pgfscope}%
\begin{pgfscope}%
\pgfpathrectangle{\pgfqpoint{1.250000in}{0.550000in}}{\pgfqpoint{7.750000in}{3.850000in}}%
\pgfusepath{clip}%
\pgfsetbuttcap%
\pgfsetmiterjoin%
\definecolor{currentfill}{rgb}{0.501961,0.501961,0.501961}%
\pgfsetfillcolor{currentfill}%
\pgfsetfillopacity{0.700000}%
\pgfsetlinewidth{0.000000pt}%
\definecolor{currentstroke}{rgb}{0.000000,0.000000,0.000000}%
\pgfsetstrokecolor{currentstroke}%
\pgfsetstrokeopacity{0.700000}%
\pgfsetdash{}{0pt}%
\pgfpathmoveto{\pgfqpoint{5.675426in}{0.550000in}}%
\pgfpathlineto{\pgfqpoint{5.702947in}{0.550000in}}%
\pgfpathlineto{\pgfqpoint{5.702947in}{0.702250in}}%
\pgfpathlineto{\pgfqpoint{5.675426in}{0.702250in}}%
\pgfpathlineto{\pgfqpoint{5.675426in}{0.550000in}}%
\pgfpathclose%
\pgfusepath{fill}%
\end{pgfscope}%
\begin{pgfscope}%
\pgfpathrectangle{\pgfqpoint{1.250000in}{0.550000in}}{\pgfqpoint{7.750000in}{3.850000in}}%
\pgfusepath{clip}%
\pgfsetbuttcap%
\pgfsetmiterjoin%
\definecolor{currentfill}{rgb}{0.501961,0.501961,0.501961}%
\pgfsetfillcolor{currentfill}%
\pgfsetfillopacity{0.700000}%
\pgfsetlinewidth{0.000000pt}%
\definecolor{currentstroke}{rgb}{0.000000,0.000000,0.000000}%
\pgfsetstrokecolor{currentstroke}%
\pgfsetstrokeopacity{0.700000}%
\pgfsetdash{}{0pt}%
\pgfpathmoveto{\pgfqpoint{5.702947in}{0.550000in}}%
\pgfpathlineto{\pgfqpoint{5.730469in}{0.550000in}}%
\pgfpathlineto{\pgfqpoint{5.730469in}{0.697845in}}%
\pgfpathlineto{\pgfqpoint{5.702947in}{0.697845in}}%
\pgfpathlineto{\pgfqpoint{5.702947in}{0.550000in}}%
\pgfpathclose%
\pgfusepath{fill}%
\end{pgfscope}%
\begin{pgfscope}%
\pgfpathrectangle{\pgfqpoint{1.250000in}{0.550000in}}{\pgfqpoint{7.750000in}{3.850000in}}%
\pgfusepath{clip}%
\pgfsetbuttcap%
\pgfsetmiterjoin%
\definecolor{currentfill}{rgb}{0.501961,0.501961,0.501961}%
\pgfsetfillcolor{currentfill}%
\pgfsetfillopacity{0.700000}%
\pgfsetlinewidth{0.000000pt}%
\definecolor{currentstroke}{rgb}{0.000000,0.000000,0.000000}%
\pgfsetstrokecolor{currentstroke}%
\pgfsetstrokeopacity{0.700000}%
\pgfsetdash{}{0pt}%
\pgfpathmoveto{\pgfqpoint{5.730469in}{0.550000in}}%
\pgfpathlineto{\pgfqpoint{5.757990in}{0.550000in}}%
\pgfpathlineto{\pgfqpoint{5.757990in}{0.703627in}}%
\pgfpathlineto{\pgfqpoint{5.730469in}{0.703627in}}%
\pgfpathlineto{\pgfqpoint{5.730469in}{0.550000in}}%
\pgfpathclose%
\pgfusepath{fill}%
\end{pgfscope}%
\begin{pgfscope}%
\pgfpathrectangle{\pgfqpoint{1.250000in}{0.550000in}}{\pgfqpoint{7.750000in}{3.850000in}}%
\pgfusepath{clip}%
\pgfsetbuttcap%
\pgfsetmiterjoin%
\definecolor{currentfill}{rgb}{0.501961,0.501961,0.501961}%
\pgfsetfillcolor{currentfill}%
\pgfsetfillopacity{0.700000}%
\pgfsetlinewidth{0.000000pt}%
\definecolor{currentstroke}{rgb}{0.000000,0.000000,0.000000}%
\pgfsetstrokecolor{currentstroke}%
\pgfsetstrokeopacity{0.700000}%
\pgfsetdash{}{0pt}%
\pgfpathmoveto{\pgfqpoint{5.757990in}{0.550000in}}%
\pgfpathlineto{\pgfqpoint{5.785511in}{0.550000in}}%
\pgfpathlineto{\pgfqpoint{5.785511in}{0.709133in}}%
\pgfpathlineto{\pgfqpoint{5.757990in}{0.709133in}}%
\pgfpathlineto{\pgfqpoint{5.757990in}{0.550000in}}%
\pgfpathclose%
\pgfusepath{fill}%
\end{pgfscope}%
\begin{pgfscope}%
\pgfpathrectangle{\pgfqpoint{1.250000in}{0.550000in}}{\pgfqpoint{7.750000in}{3.850000in}}%
\pgfusepath{clip}%
\pgfsetbuttcap%
\pgfsetmiterjoin%
\definecolor{currentfill}{rgb}{0.501961,0.501961,0.501961}%
\pgfsetfillcolor{currentfill}%
\pgfsetfillopacity{0.700000}%
\pgfsetlinewidth{0.000000pt}%
\definecolor{currentstroke}{rgb}{0.000000,0.000000,0.000000}%
\pgfsetstrokecolor{currentstroke}%
\pgfsetstrokeopacity{0.700000}%
\pgfsetdash{}{0pt}%
\pgfpathmoveto{\pgfqpoint{5.785511in}{0.550000in}}%
\pgfpathlineto{\pgfqpoint{5.813033in}{0.550000in}}%
\pgfpathlineto{\pgfqpoint{5.813033in}{0.703351in}}%
\pgfpathlineto{\pgfqpoint{5.785511in}{0.703351in}}%
\pgfpathlineto{\pgfqpoint{5.785511in}{0.550000in}}%
\pgfpathclose%
\pgfusepath{fill}%
\end{pgfscope}%
\begin{pgfscope}%
\pgfpathrectangle{\pgfqpoint{1.250000in}{0.550000in}}{\pgfqpoint{7.750000in}{3.850000in}}%
\pgfusepath{clip}%
\pgfsetbuttcap%
\pgfsetmiterjoin%
\definecolor{currentfill}{rgb}{0.501961,0.501961,0.501961}%
\pgfsetfillcolor{currentfill}%
\pgfsetfillopacity{0.700000}%
\pgfsetlinewidth{0.000000pt}%
\definecolor{currentstroke}{rgb}{0.000000,0.000000,0.000000}%
\pgfsetstrokecolor{currentstroke}%
\pgfsetstrokeopacity{0.700000}%
\pgfsetdash{}{0pt}%
\pgfpathmoveto{\pgfqpoint{5.813033in}{0.550000in}}%
\pgfpathlineto{\pgfqpoint{5.840554in}{0.550000in}}%
\pgfpathlineto{\pgfqpoint{5.840554in}{0.717117in}}%
\pgfpathlineto{\pgfqpoint{5.813033in}{0.717117in}}%
\pgfpathlineto{\pgfqpoint{5.813033in}{0.550000in}}%
\pgfpathclose%
\pgfusepath{fill}%
\end{pgfscope}%
\begin{pgfscope}%
\pgfpathrectangle{\pgfqpoint{1.250000in}{0.550000in}}{\pgfqpoint{7.750000in}{3.850000in}}%
\pgfusepath{clip}%
\pgfsetbuttcap%
\pgfsetmiterjoin%
\definecolor{currentfill}{rgb}{0.501961,0.501961,0.501961}%
\pgfsetfillcolor{currentfill}%
\pgfsetfillopacity{0.700000}%
\pgfsetlinewidth{0.000000pt}%
\definecolor{currentstroke}{rgb}{0.000000,0.000000,0.000000}%
\pgfsetstrokecolor{currentstroke}%
\pgfsetstrokeopacity{0.700000}%
\pgfsetdash{}{0pt}%
\pgfpathmoveto{\pgfqpoint{5.840554in}{0.550000in}}%
\pgfpathlineto{\pgfqpoint{5.868075in}{0.550000in}}%
\pgfpathlineto{\pgfqpoint{5.868075in}{0.711060in}}%
\pgfpathlineto{\pgfqpoint{5.840554in}{0.711060in}}%
\pgfpathlineto{\pgfqpoint{5.840554in}{0.550000in}}%
\pgfpathclose%
\pgfusepath{fill}%
\end{pgfscope}%
\begin{pgfscope}%
\pgfpathrectangle{\pgfqpoint{1.250000in}{0.550000in}}{\pgfqpoint{7.750000in}{3.850000in}}%
\pgfusepath{clip}%
\pgfsetbuttcap%
\pgfsetmiterjoin%
\definecolor{currentfill}{rgb}{0.501961,0.501961,0.501961}%
\pgfsetfillcolor{currentfill}%
\pgfsetfillopacity{0.700000}%
\pgfsetlinewidth{0.000000pt}%
\definecolor{currentstroke}{rgb}{0.000000,0.000000,0.000000}%
\pgfsetstrokecolor{currentstroke}%
\pgfsetstrokeopacity{0.700000}%
\pgfsetdash{}{0pt}%
\pgfpathmoveto{\pgfqpoint{5.868075in}{0.550000in}}%
\pgfpathlineto{\pgfqpoint{5.895597in}{0.550000in}}%
\pgfpathlineto{\pgfqpoint{5.895597in}{0.701975in}}%
\pgfpathlineto{\pgfqpoint{5.868075in}{0.701975in}}%
\pgfpathlineto{\pgfqpoint{5.868075in}{0.550000in}}%
\pgfpathclose%
\pgfusepath{fill}%
\end{pgfscope}%
\begin{pgfscope}%
\pgfpathrectangle{\pgfqpoint{1.250000in}{0.550000in}}{\pgfqpoint{7.750000in}{3.850000in}}%
\pgfusepath{clip}%
\pgfsetbuttcap%
\pgfsetmiterjoin%
\definecolor{currentfill}{rgb}{0.501961,0.501961,0.501961}%
\pgfsetfillcolor{currentfill}%
\pgfsetfillopacity{0.700000}%
\pgfsetlinewidth{0.000000pt}%
\definecolor{currentstroke}{rgb}{0.000000,0.000000,0.000000}%
\pgfsetstrokecolor{currentstroke}%
\pgfsetstrokeopacity{0.700000}%
\pgfsetdash{}{0pt}%
\pgfpathmoveto{\pgfqpoint{5.895597in}{0.550000in}}%
\pgfpathlineto{\pgfqpoint{5.923118in}{0.550000in}}%
\pgfpathlineto{\pgfqpoint{5.923118in}{0.704728in}}%
\pgfpathlineto{\pgfqpoint{5.895597in}{0.704728in}}%
\pgfpathlineto{\pgfqpoint{5.895597in}{0.550000in}}%
\pgfpathclose%
\pgfusepath{fill}%
\end{pgfscope}%
\begin{pgfscope}%
\pgfpathrectangle{\pgfqpoint{1.250000in}{0.550000in}}{\pgfqpoint{7.750000in}{3.850000in}}%
\pgfusepath{clip}%
\pgfsetbuttcap%
\pgfsetmiterjoin%
\definecolor{currentfill}{rgb}{0.501961,0.501961,0.501961}%
\pgfsetfillcolor{currentfill}%
\pgfsetfillopacity{0.700000}%
\pgfsetlinewidth{0.000000pt}%
\definecolor{currentstroke}{rgb}{0.000000,0.000000,0.000000}%
\pgfsetstrokecolor{currentstroke}%
\pgfsetstrokeopacity{0.700000}%
\pgfsetdash{}{0pt}%
\pgfpathmoveto{\pgfqpoint{5.923118in}{0.550000in}}%
\pgfpathlineto{\pgfqpoint{5.950639in}{0.550000in}}%
\pgfpathlineto{\pgfqpoint{5.950639in}{0.705829in}}%
\pgfpathlineto{\pgfqpoint{5.923118in}{0.705829in}}%
\pgfpathlineto{\pgfqpoint{5.923118in}{0.550000in}}%
\pgfpathclose%
\pgfusepath{fill}%
\end{pgfscope}%
\begin{pgfscope}%
\pgfpathrectangle{\pgfqpoint{1.250000in}{0.550000in}}{\pgfqpoint{7.750000in}{3.850000in}}%
\pgfusepath{clip}%
\pgfsetbuttcap%
\pgfsetmiterjoin%
\definecolor{currentfill}{rgb}{0.501961,0.501961,0.501961}%
\pgfsetfillcolor{currentfill}%
\pgfsetfillopacity{0.700000}%
\pgfsetlinewidth{0.000000pt}%
\definecolor{currentstroke}{rgb}{0.000000,0.000000,0.000000}%
\pgfsetstrokecolor{currentstroke}%
\pgfsetstrokeopacity{0.700000}%
\pgfsetdash{}{0pt}%
\pgfpathmoveto{\pgfqpoint{5.950639in}{0.550000in}}%
\pgfpathlineto{\pgfqpoint{5.978161in}{0.550000in}}%
\pgfpathlineto{\pgfqpoint{5.978161in}{0.720972in}}%
\pgfpathlineto{\pgfqpoint{5.950639in}{0.720972in}}%
\pgfpathlineto{\pgfqpoint{5.950639in}{0.550000in}}%
\pgfpathclose%
\pgfusepath{fill}%
\end{pgfscope}%
\begin{pgfscope}%
\pgfpathrectangle{\pgfqpoint{1.250000in}{0.550000in}}{\pgfqpoint{7.750000in}{3.850000in}}%
\pgfusepath{clip}%
\pgfsetbuttcap%
\pgfsetmiterjoin%
\definecolor{currentfill}{rgb}{0.501961,0.501961,0.501961}%
\pgfsetfillcolor{currentfill}%
\pgfsetfillopacity{0.700000}%
\pgfsetlinewidth{0.000000pt}%
\definecolor{currentstroke}{rgb}{0.000000,0.000000,0.000000}%
\pgfsetstrokecolor{currentstroke}%
\pgfsetstrokeopacity{0.700000}%
\pgfsetdash{}{0pt}%
\pgfpathmoveto{\pgfqpoint{5.978161in}{0.550000in}}%
\pgfpathlineto{\pgfqpoint{6.005682in}{0.550000in}}%
\pgfpathlineto{\pgfqpoint{6.005682in}{0.697294in}}%
\pgfpathlineto{\pgfqpoint{5.978161in}{0.697294in}}%
\pgfpathlineto{\pgfqpoint{5.978161in}{0.550000in}}%
\pgfpathclose%
\pgfusepath{fill}%
\end{pgfscope}%
\begin{pgfscope}%
\pgfpathrectangle{\pgfqpoint{1.250000in}{0.550000in}}{\pgfqpoint{7.750000in}{3.850000in}}%
\pgfusepath{clip}%
\pgfsetbuttcap%
\pgfsetmiterjoin%
\definecolor{currentfill}{rgb}{0.501961,0.501961,0.501961}%
\pgfsetfillcolor{currentfill}%
\pgfsetfillopacity{0.700000}%
\pgfsetlinewidth{0.000000pt}%
\definecolor{currentstroke}{rgb}{0.000000,0.000000,0.000000}%
\pgfsetstrokecolor{currentstroke}%
\pgfsetstrokeopacity{0.700000}%
\pgfsetdash{}{0pt}%
\pgfpathmoveto{\pgfqpoint{6.005682in}{0.550000in}}%
\pgfpathlineto{\pgfqpoint{6.033203in}{0.550000in}}%
\pgfpathlineto{\pgfqpoint{6.033203in}{0.688484in}}%
\pgfpathlineto{\pgfqpoint{6.005682in}{0.688484in}}%
\pgfpathlineto{\pgfqpoint{6.005682in}{0.550000in}}%
\pgfpathclose%
\pgfusepath{fill}%
\end{pgfscope}%
\begin{pgfscope}%
\pgfpathrectangle{\pgfqpoint{1.250000in}{0.550000in}}{\pgfqpoint{7.750000in}{3.850000in}}%
\pgfusepath{clip}%
\pgfsetbuttcap%
\pgfsetmiterjoin%
\definecolor{currentfill}{rgb}{0.501961,0.501961,0.501961}%
\pgfsetfillcolor{currentfill}%
\pgfsetfillopacity{0.700000}%
\pgfsetlinewidth{0.000000pt}%
\definecolor{currentstroke}{rgb}{0.000000,0.000000,0.000000}%
\pgfsetstrokecolor{currentstroke}%
\pgfsetstrokeopacity{0.700000}%
\pgfsetdash{}{0pt}%
\pgfpathmoveto{\pgfqpoint{6.033203in}{0.550000in}}%
\pgfpathlineto{\pgfqpoint{6.060724in}{0.550000in}}%
\pgfpathlineto{\pgfqpoint{6.060724in}{0.681877in}}%
\pgfpathlineto{\pgfqpoint{6.033203in}{0.681877in}}%
\pgfpathlineto{\pgfqpoint{6.033203in}{0.550000in}}%
\pgfpathclose%
\pgfusepath{fill}%
\end{pgfscope}%
\begin{pgfscope}%
\pgfpathrectangle{\pgfqpoint{1.250000in}{0.550000in}}{\pgfqpoint{7.750000in}{3.850000in}}%
\pgfusepath{clip}%
\pgfsetbuttcap%
\pgfsetmiterjoin%
\definecolor{currentfill}{rgb}{0.501961,0.501961,0.501961}%
\pgfsetfillcolor{currentfill}%
\pgfsetfillopacity{0.700000}%
\pgfsetlinewidth{0.000000pt}%
\definecolor{currentstroke}{rgb}{0.000000,0.000000,0.000000}%
\pgfsetstrokecolor{currentstroke}%
\pgfsetstrokeopacity{0.700000}%
\pgfsetdash{}{0pt}%
\pgfpathmoveto{\pgfqpoint{6.060724in}{0.550000in}}%
\pgfpathlineto{\pgfqpoint{6.088246in}{0.550000in}}%
\pgfpathlineto{\pgfqpoint{6.088246in}{0.675269in}}%
\pgfpathlineto{\pgfqpoint{6.060724in}{0.675269in}}%
\pgfpathlineto{\pgfqpoint{6.060724in}{0.550000in}}%
\pgfpathclose%
\pgfusepath{fill}%
\end{pgfscope}%
\begin{pgfscope}%
\pgfpathrectangle{\pgfqpoint{1.250000in}{0.550000in}}{\pgfqpoint{7.750000in}{3.850000in}}%
\pgfusepath{clip}%
\pgfsetbuttcap%
\pgfsetmiterjoin%
\definecolor{currentfill}{rgb}{0.501961,0.501961,0.501961}%
\pgfsetfillcolor{currentfill}%
\pgfsetfillopacity{0.700000}%
\pgfsetlinewidth{0.000000pt}%
\definecolor{currentstroke}{rgb}{0.000000,0.000000,0.000000}%
\pgfsetstrokecolor{currentstroke}%
\pgfsetstrokeopacity{0.700000}%
\pgfsetdash{}{0pt}%
\pgfpathmoveto{\pgfqpoint{6.088246in}{0.550000in}}%
\pgfpathlineto{\pgfqpoint{6.115767in}{0.550000in}}%
\pgfpathlineto{\pgfqpoint{6.115767in}{0.667285in}}%
\pgfpathlineto{\pgfqpoint{6.088246in}{0.667285in}}%
\pgfpathlineto{\pgfqpoint{6.088246in}{0.550000in}}%
\pgfpathclose%
\pgfusepath{fill}%
\end{pgfscope}%
\begin{pgfscope}%
\pgfpathrectangle{\pgfqpoint{1.250000in}{0.550000in}}{\pgfqpoint{7.750000in}{3.850000in}}%
\pgfusepath{clip}%
\pgfsetbuttcap%
\pgfsetmiterjoin%
\definecolor{currentfill}{rgb}{0.501961,0.501961,0.501961}%
\pgfsetfillcolor{currentfill}%
\pgfsetfillopacity{0.700000}%
\pgfsetlinewidth{0.000000pt}%
\definecolor{currentstroke}{rgb}{0.000000,0.000000,0.000000}%
\pgfsetstrokecolor{currentstroke}%
\pgfsetstrokeopacity{0.700000}%
\pgfsetdash{}{0pt}%
\pgfpathmoveto{\pgfqpoint{6.115767in}{0.550000in}}%
\pgfpathlineto{\pgfqpoint{6.143288in}{0.550000in}}%
\pgfpathlineto{\pgfqpoint{6.143288in}{0.662604in}}%
\pgfpathlineto{\pgfqpoint{6.115767in}{0.662604in}}%
\pgfpathlineto{\pgfqpoint{6.115767in}{0.550000in}}%
\pgfpathclose%
\pgfusepath{fill}%
\end{pgfscope}%
\begin{pgfscope}%
\pgfpathrectangle{\pgfqpoint{1.250000in}{0.550000in}}{\pgfqpoint{7.750000in}{3.850000in}}%
\pgfusepath{clip}%
\pgfsetbuttcap%
\pgfsetmiterjoin%
\definecolor{currentfill}{rgb}{0.501961,0.501961,0.501961}%
\pgfsetfillcolor{currentfill}%
\pgfsetfillopacity{0.700000}%
\pgfsetlinewidth{0.000000pt}%
\definecolor{currentstroke}{rgb}{0.000000,0.000000,0.000000}%
\pgfsetstrokecolor{currentstroke}%
\pgfsetstrokeopacity{0.700000}%
\pgfsetdash{}{0pt}%
\pgfpathmoveto{\pgfqpoint{6.143288in}{0.550000in}}%
\pgfpathlineto{\pgfqpoint{6.170810in}{0.550000in}}%
\pgfpathlineto{\pgfqpoint{6.170810in}{0.665082in}}%
\pgfpathlineto{\pgfqpoint{6.143288in}{0.665082in}}%
\pgfpathlineto{\pgfqpoint{6.143288in}{0.550000in}}%
\pgfpathclose%
\pgfusepath{fill}%
\end{pgfscope}%
\begin{pgfscope}%
\pgfpathrectangle{\pgfqpoint{1.250000in}{0.550000in}}{\pgfqpoint{7.750000in}{3.850000in}}%
\pgfusepath{clip}%
\pgfsetbuttcap%
\pgfsetmiterjoin%
\definecolor{currentfill}{rgb}{0.501961,0.501961,0.501961}%
\pgfsetfillcolor{currentfill}%
\pgfsetfillopacity{0.700000}%
\pgfsetlinewidth{0.000000pt}%
\definecolor{currentstroke}{rgb}{0.000000,0.000000,0.000000}%
\pgfsetstrokecolor{currentstroke}%
\pgfsetstrokeopacity{0.700000}%
\pgfsetdash{}{0pt}%
\pgfpathmoveto{\pgfqpoint{6.170810in}{0.550000in}}%
\pgfpathlineto{\pgfqpoint{6.198331in}{0.550000in}}%
\pgfpathlineto{\pgfqpoint{6.198331in}{0.672241in}}%
\pgfpathlineto{\pgfqpoint{6.170810in}{0.672241in}}%
\pgfpathlineto{\pgfqpoint{6.170810in}{0.550000in}}%
\pgfpathclose%
\pgfusepath{fill}%
\end{pgfscope}%
\begin{pgfscope}%
\pgfpathrectangle{\pgfqpoint{1.250000in}{0.550000in}}{\pgfqpoint{7.750000in}{3.850000in}}%
\pgfusepath{clip}%
\pgfsetbuttcap%
\pgfsetmiterjoin%
\definecolor{currentfill}{rgb}{0.501961,0.501961,0.501961}%
\pgfsetfillcolor{currentfill}%
\pgfsetfillopacity{0.700000}%
\pgfsetlinewidth{0.000000pt}%
\definecolor{currentstroke}{rgb}{0.000000,0.000000,0.000000}%
\pgfsetstrokecolor{currentstroke}%
\pgfsetstrokeopacity{0.700000}%
\pgfsetdash{}{0pt}%
\pgfpathmoveto{\pgfqpoint{6.198331in}{0.550000in}}%
\pgfpathlineto{\pgfqpoint{6.225852in}{0.550000in}}%
\pgfpathlineto{\pgfqpoint{6.225852in}{0.654345in}}%
\pgfpathlineto{\pgfqpoint{6.198331in}{0.654345in}}%
\pgfpathlineto{\pgfqpoint{6.198331in}{0.550000in}}%
\pgfpathclose%
\pgfusepath{fill}%
\end{pgfscope}%
\begin{pgfscope}%
\pgfpathrectangle{\pgfqpoint{1.250000in}{0.550000in}}{\pgfqpoint{7.750000in}{3.850000in}}%
\pgfusepath{clip}%
\pgfsetbuttcap%
\pgfsetmiterjoin%
\definecolor{currentfill}{rgb}{0.501961,0.501961,0.501961}%
\pgfsetfillcolor{currentfill}%
\pgfsetfillopacity{0.700000}%
\pgfsetlinewidth{0.000000pt}%
\definecolor{currentstroke}{rgb}{0.000000,0.000000,0.000000}%
\pgfsetstrokecolor{currentstroke}%
\pgfsetstrokeopacity{0.700000}%
\pgfsetdash{}{0pt}%
\pgfpathmoveto{\pgfqpoint{6.225852in}{0.550000in}}%
\pgfpathlineto{\pgfqpoint{6.253374in}{0.550000in}}%
\pgfpathlineto{\pgfqpoint{6.253374in}{0.648563in}}%
\pgfpathlineto{\pgfqpoint{6.225852in}{0.648563in}}%
\pgfpathlineto{\pgfqpoint{6.225852in}{0.550000in}}%
\pgfpathclose%
\pgfusepath{fill}%
\end{pgfscope}%
\begin{pgfscope}%
\pgfpathrectangle{\pgfqpoint{1.250000in}{0.550000in}}{\pgfqpoint{7.750000in}{3.850000in}}%
\pgfusepath{clip}%
\pgfsetbuttcap%
\pgfsetmiterjoin%
\definecolor{currentfill}{rgb}{0.501961,0.501961,0.501961}%
\pgfsetfillcolor{currentfill}%
\pgfsetfillopacity{0.700000}%
\pgfsetlinewidth{0.000000pt}%
\definecolor{currentstroke}{rgb}{0.000000,0.000000,0.000000}%
\pgfsetstrokecolor{currentstroke}%
\pgfsetstrokeopacity{0.700000}%
\pgfsetdash{}{0pt}%
\pgfpathmoveto{\pgfqpoint{6.253374in}{0.550000in}}%
\pgfpathlineto{\pgfqpoint{6.280895in}{0.550000in}}%
\pgfpathlineto{\pgfqpoint{6.280895in}{0.646085in}}%
\pgfpathlineto{\pgfqpoint{6.253374in}{0.646085in}}%
\pgfpathlineto{\pgfqpoint{6.253374in}{0.550000in}}%
\pgfpathclose%
\pgfusepath{fill}%
\end{pgfscope}%
\begin{pgfscope}%
\pgfpathrectangle{\pgfqpoint{1.250000in}{0.550000in}}{\pgfqpoint{7.750000in}{3.850000in}}%
\pgfusepath{clip}%
\pgfsetbuttcap%
\pgfsetmiterjoin%
\definecolor{currentfill}{rgb}{0.501961,0.501961,0.501961}%
\pgfsetfillcolor{currentfill}%
\pgfsetfillopacity{0.700000}%
\pgfsetlinewidth{0.000000pt}%
\definecolor{currentstroke}{rgb}{0.000000,0.000000,0.000000}%
\pgfsetstrokecolor{currentstroke}%
\pgfsetstrokeopacity{0.700000}%
\pgfsetdash{}{0pt}%
\pgfpathmoveto{\pgfqpoint{6.280895in}{0.550000in}}%
\pgfpathlineto{\pgfqpoint{6.308416in}{0.550000in}}%
\pgfpathlineto{\pgfqpoint{6.308416in}{0.643608in}}%
\pgfpathlineto{\pgfqpoint{6.280895in}{0.643608in}}%
\pgfpathlineto{\pgfqpoint{6.280895in}{0.550000in}}%
\pgfpathclose%
\pgfusepath{fill}%
\end{pgfscope}%
\begin{pgfscope}%
\pgfpathrectangle{\pgfqpoint{1.250000in}{0.550000in}}{\pgfqpoint{7.750000in}{3.850000in}}%
\pgfusepath{clip}%
\pgfsetbuttcap%
\pgfsetmiterjoin%
\definecolor{currentfill}{rgb}{0.501961,0.501961,0.501961}%
\pgfsetfillcolor{currentfill}%
\pgfsetfillopacity{0.700000}%
\pgfsetlinewidth{0.000000pt}%
\definecolor{currentstroke}{rgb}{0.000000,0.000000,0.000000}%
\pgfsetstrokecolor{currentstroke}%
\pgfsetstrokeopacity{0.700000}%
\pgfsetdash{}{0pt}%
\pgfpathmoveto{\pgfqpoint{6.308416in}{0.550000in}}%
\pgfpathlineto{\pgfqpoint{6.335937in}{0.550000in}}%
\pgfpathlineto{\pgfqpoint{6.335937in}{0.642506in}}%
\pgfpathlineto{\pgfqpoint{6.308416in}{0.642506in}}%
\pgfpathlineto{\pgfqpoint{6.308416in}{0.550000in}}%
\pgfpathclose%
\pgfusepath{fill}%
\end{pgfscope}%
\begin{pgfscope}%
\pgfpathrectangle{\pgfqpoint{1.250000in}{0.550000in}}{\pgfqpoint{7.750000in}{3.850000in}}%
\pgfusepath{clip}%
\pgfsetbuttcap%
\pgfsetmiterjoin%
\definecolor{currentfill}{rgb}{0.501961,0.501961,0.501961}%
\pgfsetfillcolor{currentfill}%
\pgfsetfillopacity{0.700000}%
\pgfsetlinewidth{0.000000pt}%
\definecolor{currentstroke}{rgb}{0.000000,0.000000,0.000000}%
\pgfsetstrokecolor{currentstroke}%
\pgfsetstrokeopacity{0.700000}%
\pgfsetdash{}{0pt}%
\pgfpathmoveto{\pgfqpoint{6.335938in}{0.550000in}}%
\pgfpathlineto{\pgfqpoint{6.363459in}{0.550000in}}%
\pgfpathlineto{\pgfqpoint{6.363459in}{0.635348in}}%
\pgfpathlineto{\pgfqpoint{6.335938in}{0.635348in}}%
\pgfpathlineto{\pgfqpoint{6.335938in}{0.550000in}}%
\pgfpathclose%
\pgfusepath{fill}%
\end{pgfscope}%
\begin{pgfscope}%
\pgfpathrectangle{\pgfqpoint{1.250000in}{0.550000in}}{\pgfqpoint{7.750000in}{3.850000in}}%
\pgfusepath{clip}%
\pgfsetbuttcap%
\pgfsetmiterjoin%
\definecolor{currentfill}{rgb}{0.501961,0.501961,0.501961}%
\pgfsetfillcolor{currentfill}%
\pgfsetfillopacity{0.700000}%
\pgfsetlinewidth{0.000000pt}%
\definecolor{currentstroke}{rgb}{0.000000,0.000000,0.000000}%
\pgfsetstrokecolor{currentstroke}%
\pgfsetstrokeopacity{0.700000}%
\pgfsetdash{}{0pt}%
\pgfpathmoveto{\pgfqpoint{6.363459in}{0.550000in}}%
\pgfpathlineto{\pgfqpoint{6.390980in}{0.550000in}}%
\pgfpathlineto{\pgfqpoint{6.390980in}{0.636449in}}%
\pgfpathlineto{\pgfqpoint{6.363459in}{0.636449in}}%
\pgfpathlineto{\pgfqpoint{6.363459in}{0.550000in}}%
\pgfpathclose%
\pgfusepath{fill}%
\end{pgfscope}%
\begin{pgfscope}%
\pgfpathrectangle{\pgfqpoint{1.250000in}{0.550000in}}{\pgfqpoint{7.750000in}{3.850000in}}%
\pgfusepath{clip}%
\pgfsetbuttcap%
\pgfsetmiterjoin%
\definecolor{currentfill}{rgb}{0.501961,0.501961,0.501961}%
\pgfsetfillcolor{currentfill}%
\pgfsetfillopacity{0.700000}%
\pgfsetlinewidth{0.000000pt}%
\definecolor{currentstroke}{rgb}{0.000000,0.000000,0.000000}%
\pgfsetstrokecolor{currentstroke}%
\pgfsetstrokeopacity{0.700000}%
\pgfsetdash{}{0pt}%
\pgfpathmoveto{\pgfqpoint{6.390980in}{0.550000in}}%
\pgfpathlineto{\pgfqpoint{6.418501in}{0.550000in}}%
\pgfpathlineto{\pgfqpoint{6.418501in}{0.619655in}}%
\pgfpathlineto{\pgfqpoint{6.390980in}{0.619655in}}%
\pgfpathlineto{\pgfqpoint{6.390980in}{0.550000in}}%
\pgfpathclose%
\pgfusepath{fill}%
\end{pgfscope}%
\begin{pgfscope}%
\pgfpathrectangle{\pgfqpoint{1.250000in}{0.550000in}}{\pgfqpoint{7.750000in}{3.850000in}}%
\pgfusepath{clip}%
\pgfsetbuttcap%
\pgfsetmiterjoin%
\definecolor{currentfill}{rgb}{0.501961,0.501961,0.501961}%
\pgfsetfillcolor{currentfill}%
\pgfsetfillopacity{0.700000}%
\pgfsetlinewidth{0.000000pt}%
\definecolor{currentstroke}{rgb}{0.000000,0.000000,0.000000}%
\pgfsetstrokecolor{currentstroke}%
\pgfsetstrokeopacity{0.700000}%
\pgfsetdash{}{0pt}%
\pgfpathmoveto{\pgfqpoint{6.418501in}{0.550000in}}%
\pgfpathlineto{\pgfqpoint{6.446023in}{0.550000in}}%
\pgfpathlineto{\pgfqpoint{6.446023in}{0.624886in}}%
\pgfpathlineto{\pgfqpoint{6.418501in}{0.624886in}}%
\pgfpathlineto{\pgfqpoint{6.418501in}{0.550000in}}%
\pgfpathclose%
\pgfusepath{fill}%
\end{pgfscope}%
\begin{pgfscope}%
\pgfpathrectangle{\pgfqpoint{1.250000in}{0.550000in}}{\pgfqpoint{7.750000in}{3.850000in}}%
\pgfusepath{clip}%
\pgfsetbuttcap%
\pgfsetmiterjoin%
\definecolor{currentfill}{rgb}{0.501961,0.501961,0.501961}%
\pgfsetfillcolor{currentfill}%
\pgfsetfillopacity{0.700000}%
\pgfsetlinewidth{0.000000pt}%
\definecolor{currentstroke}{rgb}{0.000000,0.000000,0.000000}%
\pgfsetstrokecolor{currentstroke}%
\pgfsetstrokeopacity{0.700000}%
\pgfsetdash{}{0pt}%
\pgfpathmoveto{\pgfqpoint{6.446023in}{0.550000in}}%
\pgfpathlineto{\pgfqpoint{6.473544in}{0.550000in}}%
\pgfpathlineto{\pgfqpoint{6.473544in}{0.616627in}}%
\pgfpathlineto{\pgfqpoint{6.446023in}{0.616627in}}%
\pgfpathlineto{\pgfqpoint{6.446023in}{0.550000in}}%
\pgfpathclose%
\pgfusepath{fill}%
\end{pgfscope}%
\begin{pgfscope}%
\pgfpathrectangle{\pgfqpoint{1.250000in}{0.550000in}}{\pgfqpoint{7.750000in}{3.850000in}}%
\pgfusepath{clip}%
\pgfsetbuttcap%
\pgfsetmiterjoin%
\definecolor{currentfill}{rgb}{0.501961,0.501961,0.501961}%
\pgfsetfillcolor{currentfill}%
\pgfsetfillopacity{0.700000}%
\pgfsetlinewidth{0.000000pt}%
\definecolor{currentstroke}{rgb}{0.000000,0.000000,0.000000}%
\pgfsetstrokecolor{currentstroke}%
\pgfsetstrokeopacity{0.700000}%
\pgfsetdash{}{0pt}%
\pgfpathmoveto{\pgfqpoint{6.473544in}{0.550000in}}%
\pgfpathlineto{\pgfqpoint{6.501065in}{0.550000in}}%
\pgfpathlineto{\pgfqpoint{6.501065in}{0.621582in}}%
\pgfpathlineto{\pgfqpoint{6.473544in}{0.621582in}}%
\pgfpathlineto{\pgfqpoint{6.473544in}{0.550000in}}%
\pgfpathclose%
\pgfusepath{fill}%
\end{pgfscope}%
\begin{pgfscope}%
\pgfpathrectangle{\pgfqpoint{1.250000in}{0.550000in}}{\pgfqpoint{7.750000in}{3.850000in}}%
\pgfusepath{clip}%
\pgfsetbuttcap%
\pgfsetmiterjoin%
\definecolor{currentfill}{rgb}{0.501961,0.501961,0.501961}%
\pgfsetfillcolor{currentfill}%
\pgfsetfillopacity{0.700000}%
\pgfsetlinewidth{0.000000pt}%
\definecolor{currentstroke}{rgb}{0.000000,0.000000,0.000000}%
\pgfsetstrokecolor{currentstroke}%
\pgfsetstrokeopacity{0.700000}%
\pgfsetdash{}{0pt}%
\pgfpathmoveto{\pgfqpoint{6.501065in}{0.550000in}}%
\pgfpathlineto{\pgfqpoint{6.528587in}{0.550000in}}%
\pgfpathlineto{\pgfqpoint{6.528587in}{0.608367in}}%
\pgfpathlineto{\pgfqpoint{6.501065in}{0.608367in}}%
\pgfpathlineto{\pgfqpoint{6.501065in}{0.550000in}}%
\pgfpathclose%
\pgfusepath{fill}%
\end{pgfscope}%
\begin{pgfscope}%
\pgfpathrectangle{\pgfqpoint{1.250000in}{0.550000in}}{\pgfqpoint{7.750000in}{3.850000in}}%
\pgfusepath{clip}%
\pgfsetbuttcap%
\pgfsetmiterjoin%
\definecolor{currentfill}{rgb}{0.501961,0.501961,0.501961}%
\pgfsetfillcolor{currentfill}%
\pgfsetfillopacity{0.700000}%
\pgfsetlinewidth{0.000000pt}%
\definecolor{currentstroke}{rgb}{0.000000,0.000000,0.000000}%
\pgfsetstrokecolor{currentstroke}%
\pgfsetstrokeopacity{0.700000}%
\pgfsetdash{}{0pt}%
\pgfpathmoveto{\pgfqpoint{6.528587in}{0.550000in}}%
\pgfpathlineto{\pgfqpoint{6.556108in}{0.550000in}}%
\pgfpathlineto{\pgfqpoint{6.556108in}{0.604788in}}%
\pgfpathlineto{\pgfqpoint{6.528587in}{0.604788in}}%
\pgfpathlineto{\pgfqpoint{6.528587in}{0.550000in}}%
\pgfpathclose%
\pgfusepath{fill}%
\end{pgfscope}%
\begin{pgfscope}%
\pgfpathrectangle{\pgfqpoint{1.250000in}{0.550000in}}{\pgfqpoint{7.750000in}{3.850000in}}%
\pgfusepath{clip}%
\pgfsetbuttcap%
\pgfsetmiterjoin%
\definecolor{currentfill}{rgb}{0.501961,0.501961,0.501961}%
\pgfsetfillcolor{currentfill}%
\pgfsetfillopacity{0.700000}%
\pgfsetlinewidth{0.000000pt}%
\definecolor{currentstroke}{rgb}{0.000000,0.000000,0.000000}%
\pgfsetstrokecolor{currentstroke}%
\pgfsetstrokeopacity{0.700000}%
\pgfsetdash{}{0pt}%
\pgfpathmoveto{\pgfqpoint{6.556108in}{0.550000in}}%
\pgfpathlineto{\pgfqpoint{6.583629in}{0.550000in}}%
\pgfpathlineto{\pgfqpoint{6.583629in}{0.611396in}}%
\pgfpathlineto{\pgfqpoint{6.556108in}{0.611396in}}%
\pgfpathlineto{\pgfqpoint{6.556108in}{0.550000in}}%
\pgfpathclose%
\pgfusepath{fill}%
\end{pgfscope}%
\begin{pgfscope}%
\pgfpathrectangle{\pgfqpoint{1.250000in}{0.550000in}}{\pgfqpoint{7.750000in}{3.850000in}}%
\pgfusepath{clip}%
\pgfsetbuttcap%
\pgfsetmiterjoin%
\definecolor{currentfill}{rgb}{0.501961,0.501961,0.501961}%
\pgfsetfillcolor{currentfill}%
\pgfsetfillopacity{0.700000}%
\pgfsetlinewidth{0.000000pt}%
\definecolor{currentstroke}{rgb}{0.000000,0.000000,0.000000}%
\pgfsetstrokecolor{currentstroke}%
\pgfsetstrokeopacity{0.700000}%
\pgfsetdash{}{0pt}%
\pgfpathmoveto{\pgfqpoint{6.583629in}{0.550000in}}%
\pgfpathlineto{\pgfqpoint{6.611151in}{0.550000in}}%
\pgfpathlineto{\pgfqpoint{6.611151in}{0.598180in}}%
\pgfpathlineto{\pgfqpoint{6.583629in}{0.598180in}}%
\pgfpathlineto{\pgfqpoint{6.583629in}{0.550000in}}%
\pgfpathclose%
\pgfusepath{fill}%
\end{pgfscope}%
\begin{pgfscope}%
\pgfpathrectangle{\pgfqpoint{1.250000in}{0.550000in}}{\pgfqpoint{7.750000in}{3.850000in}}%
\pgfusepath{clip}%
\pgfsetbuttcap%
\pgfsetmiterjoin%
\definecolor{currentfill}{rgb}{0.501961,0.501961,0.501961}%
\pgfsetfillcolor{currentfill}%
\pgfsetfillopacity{0.700000}%
\pgfsetlinewidth{0.000000pt}%
\definecolor{currentstroke}{rgb}{0.000000,0.000000,0.000000}%
\pgfsetstrokecolor{currentstroke}%
\pgfsetstrokeopacity{0.700000}%
\pgfsetdash{}{0pt}%
\pgfpathmoveto{\pgfqpoint{6.611151in}{0.550000in}}%
\pgfpathlineto{\pgfqpoint{6.638672in}{0.550000in}}%
\pgfpathlineto{\pgfqpoint{6.638672in}{0.611396in}}%
\pgfpathlineto{\pgfqpoint{6.611151in}{0.611396in}}%
\pgfpathlineto{\pgfqpoint{6.611151in}{0.550000in}}%
\pgfpathclose%
\pgfusepath{fill}%
\end{pgfscope}%
\begin{pgfscope}%
\pgfpathrectangle{\pgfqpoint{1.250000in}{0.550000in}}{\pgfqpoint{7.750000in}{3.850000in}}%
\pgfusepath{clip}%
\pgfsetbuttcap%
\pgfsetmiterjoin%
\definecolor{currentfill}{rgb}{0.501961,0.501961,0.501961}%
\pgfsetfillcolor{currentfill}%
\pgfsetfillopacity{0.700000}%
\pgfsetlinewidth{0.000000pt}%
\definecolor{currentstroke}{rgb}{0.000000,0.000000,0.000000}%
\pgfsetstrokecolor{currentstroke}%
\pgfsetstrokeopacity{0.700000}%
\pgfsetdash{}{0pt}%
\pgfpathmoveto{\pgfqpoint{6.638672in}{0.550000in}}%
\pgfpathlineto{\pgfqpoint{6.666193in}{0.550000in}}%
\pgfpathlineto{\pgfqpoint{6.666193in}{0.606715in}}%
\pgfpathlineto{\pgfqpoint{6.638672in}{0.606715in}}%
\pgfpathlineto{\pgfqpoint{6.638672in}{0.550000in}}%
\pgfpathclose%
\pgfusepath{fill}%
\end{pgfscope}%
\begin{pgfscope}%
\pgfpathrectangle{\pgfqpoint{1.250000in}{0.550000in}}{\pgfqpoint{7.750000in}{3.850000in}}%
\pgfusepath{clip}%
\pgfsetbuttcap%
\pgfsetmiterjoin%
\definecolor{currentfill}{rgb}{0.501961,0.501961,0.501961}%
\pgfsetfillcolor{currentfill}%
\pgfsetfillopacity{0.700000}%
\pgfsetlinewidth{0.000000pt}%
\definecolor{currentstroke}{rgb}{0.000000,0.000000,0.000000}%
\pgfsetstrokecolor{currentstroke}%
\pgfsetstrokeopacity{0.700000}%
\pgfsetdash{}{0pt}%
\pgfpathmoveto{\pgfqpoint{6.666193in}{0.550000in}}%
\pgfpathlineto{\pgfqpoint{6.693714in}{0.550000in}}%
\pgfpathlineto{\pgfqpoint{6.693714in}{0.609468in}}%
\pgfpathlineto{\pgfqpoint{6.666193in}{0.609468in}}%
\pgfpathlineto{\pgfqpoint{6.666193in}{0.550000in}}%
\pgfpathclose%
\pgfusepath{fill}%
\end{pgfscope}%
\begin{pgfscope}%
\pgfpathrectangle{\pgfqpoint{1.250000in}{0.550000in}}{\pgfqpoint{7.750000in}{3.850000in}}%
\pgfusepath{clip}%
\pgfsetbuttcap%
\pgfsetmiterjoin%
\definecolor{currentfill}{rgb}{0.501961,0.501961,0.501961}%
\pgfsetfillcolor{currentfill}%
\pgfsetfillopacity{0.700000}%
\pgfsetlinewidth{0.000000pt}%
\definecolor{currentstroke}{rgb}{0.000000,0.000000,0.000000}%
\pgfsetstrokecolor{currentstroke}%
\pgfsetstrokeopacity{0.700000}%
\pgfsetdash{}{0pt}%
\pgfpathmoveto{\pgfqpoint{6.693714in}{0.550000in}}%
\pgfpathlineto{\pgfqpoint{6.721236in}{0.550000in}}%
\pgfpathlineto{\pgfqpoint{6.721236in}{0.601760in}}%
\pgfpathlineto{\pgfqpoint{6.693714in}{0.601760in}}%
\pgfpathlineto{\pgfqpoint{6.693714in}{0.550000in}}%
\pgfpathclose%
\pgfusepath{fill}%
\end{pgfscope}%
\begin{pgfscope}%
\pgfpathrectangle{\pgfqpoint{1.250000in}{0.550000in}}{\pgfqpoint{7.750000in}{3.850000in}}%
\pgfusepath{clip}%
\pgfsetbuttcap%
\pgfsetmiterjoin%
\definecolor{currentfill}{rgb}{0.501961,0.501961,0.501961}%
\pgfsetfillcolor{currentfill}%
\pgfsetfillopacity{0.700000}%
\pgfsetlinewidth{0.000000pt}%
\definecolor{currentstroke}{rgb}{0.000000,0.000000,0.000000}%
\pgfsetstrokecolor{currentstroke}%
\pgfsetstrokeopacity{0.700000}%
\pgfsetdash{}{0pt}%
\pgfpathmoveto{\pgfqpoint{6.721236in}{0.550000in}}%
\pgfpathlineto{\pgfqpoint{6.748757in}{0.550000in}}%
\pgfpathlineto{\pgfqpoint{6.748757in}{0.609193in}}%
\pgfpathlineto{\pgfqpoint{6.721236in}{0.609193in}}%
\pgfpathlineto{\pgfqpoint{6.721236in}{0.550000in}}%
\pgfpathclose%
\pgfusepath{fill}%
\end{pgfscope}%
\begin{pgfscope}%
\pgfpathrectangle{\pgfqpoint{1.250000in}{0.550000in}}{\pgfqpoint{7.750000in}{3.850000in}}%
\pgfusepath{clip}%
\pgfsetbuttcap%
\pgfsetmiterjoin%
\definecolor{currentfill}{rgb}{0.501961,0.501961,0.501961}%
\pgfsetfillcolor{currentfill}%
\pgfsetfillopacity{0.700000}%
\pgfsetlinewidth{0.000000pt}%
\definecolor{currentstroke}{rgb}{0.000000,0.000000,0.000000}%
\pgfsetstrokecolor{currentstroke}%
\pgfsetstrokeopacity{0.700000}%
\pgfsetdash{}{0pt}%
\pgfpathmoveto{\pgfqpoint{6.748757in}{0.550000in}}%
\pgfpathlineto{\pgfqpoint{6.776278in}{0.550000in}}%
\pgfpathlineto{\pgfqpoint{6.776278in}{0.603962in}}%
\pgfpathlineto{\pgfqpoint{6.748757in}{0.603962in}}%
\pgfpathlineto{\pgfqpoint{6.748757in}{0.550000in}}%
\pgfpathclose%
\pgfusepath{fill}%
\end{pgfscope}%
\begin{pgfscope}%
\pgfpathrectangle{\pgfqpoint{1.250000in}{0.550000in}}{\pgfqpoint{7.750000in}{3.850000in}}%
\pgfusepath{clip}%
\pgfsetbuttcap%
\pgfsetmiterjoin%
\definecolor{currentfill}{rgb}{0.501961,0.501961,0.501961}%
\pgfsetfillcolor{currentfill}%
\pgfsetfillopacity{0.700000}%
\pgfsetlinewidth{0.000000pt}%
\definecolor{currentstroke}{rgb}{0.000000,0.000000,0.000000}%
\pgfsetstrokecolor{currentstroke}%
\pgfsetstrokeopacity{0.700000}%
\pgfsetdash{}{0pt}%
\pgfpathmoveto{\pgfqpoint{6.776278in}{0.550000in}}%
\pgfpathlineto{\pgfqpoint{6.803800in}{0.550000in}}%
\pgfpathlineto{\pgfqpoint{6.803800in}{0.599832in}}%
\pgfpathlineto{\pgfqpoint{6.776278in}{0.599832in}}%
\pgfpathlineto{\pgfqpoint{6.776278in}{0.550000in}}%
\pgfpathclose%
\pgfusepath{fill}%
\end{pgfscope}%
\begin{pgfscope}%
\pgfpathrectangle{\pgfqpoint{1.250000in}{0.550000in}}{\pgfqpoint{7.750000in}{3.850000in}}%
\pgfusepath{clip}%
\pgfsetbuttcap%
\pgfsetmiterjoin%
\definecolor{currentfill}{rgb}{0.501961,0.501961,0.501961}%
\pgfsetfillcolor{currentfill}%
\pgfsetfillopacity{0.700000}%
\pgfsetlinewidth{0.000000pt}%
\definecolor{currentstroke}{rgb}{0.000000,0.000000,0.000000}%
\pgfsetstrokecolor{currentstroke}%
\pgfsetstrokeopacity{0.700000}%
\pgfsetdash{}{0pt}%
\pgfpathmoveto{\pgfqpoint{6.803800in}{0.550000in}}%
\pgfpathlineto{\pgfqpoint{6.831321in}{0.550000in}}%
\pgfpathlineto{\pgfqpoint{6.831321in}{0.600108in}}%
\pgfpathlineto{\pgfqpoint{6.803800in}{0.600108in}}%
\pgfpathlineto{\pgfqpoint{6.803800in}{0.550000in}}%
\pgfpathclose%
\pgfusepath{fill}%
\end{pgfscope}%
\begin{pgfscope}%
\pgfpathrectangle{\pgfqpoint{1.250000in}{0.550000in}}{\pgfqpoint{7.750000in}{3.850000in}}%
\pgfusepath{clip}%
\pgfsetbuttcap%
\pgfsetmiterjoin%
\definecolor{currentfill}{rgb}{0.501961,0.501961,0.501961}%
\pgfsetfillcolor{currentfill}%
\pgfsetfillopacity{0.700000}%
\pgfsetlinewidth{0.000000pt}%
\definecolor{currentstroke}{rgb}{0.000000,0.000000,0.000000}%
\pgfsetstrokecolor{currentstroke}%
\pgfsetstrokeopacity{0.700000}%
\pgfsetdash{}{0pt}%
\pgfpathmoveto{\pgfqpoint{6.831321in}{0.550000in}}%
\pgfpathlineto{\pgfqpoint{6.858842in}{0.550000in}}%
\pgfpathlineto{\pgfqpoint{6.858842in}{0.599832in}}%
\pgfpathlineto{\pgfqpoint{6.831321in}{0.599832in}}%
\pgfpathlineto{\pgfqpoint{6.831321in}{0.550000in}}%
\pgfpathclose%
\pgfusepath{fill}%
\end{pgfscope}%
\begin{pgfscope}%
\pgfpathrectangle{\pgfqpoint{1.250000in}{0.550000in}}{\pgfqpoint{7.750000in}{3.850000in}}%
\pgfusepath{clip}%
\pgfsetbuttcap%
\pgfsetmiterjoin%
\definecolor{currentfill}{rgb}{0.501961,0.501961,0.501961}%
\pgfsetfillcolor{currentfill}%
\pgfsetfillopacity{0.700000}%
\pgfsetlinewidth{0.000000pt}%
\definecolor{currentstroke}{rgb}{0.000000,0.000000,0.000000}%
\pgfsetstrokecolor{currentstroke}%
\pgfsetstrokeopacity{0.700000}%
\pgfsetdash{}{0pt}%
\pgfpathmoveto{\pgfqpoint{6.858842in}{0.550000in}}%
\pgfpathlineto{\pgfqpoint{6.886364in}{0.550000in}}%
\pgfpathlineto{\pgfqpoint{6.886364in}{0.593225in}}%
\pgfpathlineto{\pgfqpoint{6.858842in}{0.593225in}}%
\pgfpathlineto{\pgfqpoint{6.858842in}{0.550000in}}%
\pgfpathclose%
\pgfusepath{fill}%
\end{pgfscope}%
\begin{pgfscope}%
\pgfpathrectangle{\pgfqpoint{1.250000in}{0.550000in}}{\pgfqpoint{7.750000in}{3.850000in}}%
\pgfusepath{clip}%
\pgfsetbuttcap%
\pgfsetmiterjoin%
\definecolor{currentfill}{rgb}{0.501961,0.501961,0.501961}%
\pgfsetfillcolor{currentfill}%
\pgfsetfillopacity{0.700000}%
\pgfsetlinewidth{0.000000pt}%
\definecolor{currentstroke}{rgb}{0.000000,0.000000,0.000000}%
\pgfsetstrokecolor{currentstroke}%
\pgfsetstrokeopacity{0.700000}%
\pgfsetdash{}{0pt}%
\pgfpathmoveto{\pgfqpoint{6.886364in}{0.550000in}}%
\pgfpathlineto{\pgfqpoint{6.913885in}{0.550000in}}%
\pgfpathlineto{\pgfqpoint{6.913885in}{0.600108in}}%
\pgfpathlineto{\pgfqpoint{6.886364in}{0.600108in}}%
\pgfpathlineto{\pgfqpoint{6.886364in}{0.550000in}}%
\pgfpathclose%
\pgfusepath{fill}%
\end{pgfscope}%
\begin{pgfscope}%
\pgfpathrectangle{\pgfqpoint{1.250000in}{0.550000in}}{\pgfqpoint{7.750000in}{3.850000in}}%
\pgfusepath{clip}%
\pgfsetbuttcap%
\pgfsetmiterjoin%
\definecolor{currentfill}{rgb}{0.501961,0.501961,0.501961}%
\pgfsetfillcolor{currentfill}%
\pgfsetfillopacity{0.700000}%
\pgfsetlinewidth{0.000000pt}%
\definecolor{currentstroke}{rgb}{0.000000,0.000000,0.000000}%
\pgfsetstrokecolor{currentstroke}%
\pgfsetstrokeopacity{0.700000}%
\pgfsetdash{}{0pt}%
\pgfpathmoveto{\pgfqpoint{6.913885in}{0.550000in}}%
\pgfpathlineto{\pgfqpoint{6.941406in}{0.550000in}}%
\pgfpathlineto{\pgfqpoint{6.941406in}{0.595978in}}%
\pgfpathlineto{\pgfqpoint{6.913885in}{0.595978in}}%
\pgfpathlineto{\pgfqpoint{6.913885in}{0.550000in}}%
\pgfpathclose%
\pgfusepath{fill}%
\end{pgfscope}%
\begin{pgfscope}%
\pgfpathrectangle{\pgfqpoint{1.250000in}{0.550000in}}{\pgfqpoint{7.750000in}{3.850000in}}%
\pgfusepath{clip}%
\pgfsetbuttcap%
\pgfsetmiterjoin%
\definecolor{currentfill}{rgb}{0.501961,0.501961,0.501961}%
\pgfsetfillcolor{currentfill}%
\pgfsetfillopacity{0.700000}%
\pgfsetlinewidth{0.000000pt}%
\definecolor{currentstroke}{rgb}{0.000000,0.000000,0.000000}%
\pgfsetstrokecolor{currentstroke}%
\pgfsetstrokeopacity{0.700000}%
\pgfsetdash{}{0pt}%
\pgfpathmoveto{\pgfqpoint{6.941406in}{0.550000in}}%
\pgfpathlineto{\pgfqpoint{6.968928in}{0.550000in}}%
\pgfpathlineto{\pgfqpoint{6.968928in}{0.591848in}}%
\pgfpathlineto{\pgfqpoint{6.941406in}{0.591848in}}%
\pgfpathlineto{\pgfqpoint{6.941406in}{0.550000in}}%
\pgfpathclose%
\pgfusepath{fill}%
\end{pgfscope}%
\begin{pgfscope}%
\pgfpathrectangle{\pgfqpoint{1.250000in}{0.550000in}}{\pgfqpoint{7.750000in}{3.850000in}}%
\pgfusepath{clip}%
\pgfsetbuttcap%
\pgfsetmiterjoin%
\definecolor{currentfill}{rgb}{0.501961,0.501961,0.501961}%
\pgfsetfillcolor{currentfill}%
\pgfsetfillopacity{0.700000}%
\pgfsetlinewidth{0.000000pt}%
\definecolor{currentstroke}{rgb}{0.000000,0.000000,0.000000}%
\pgfsetstrokecolor{currentstroke}%
\pgfsetstrokeopacity{0.700000}%
\pgfsetdash{}{0pt}%
\pgfpathmoveto{\pgfqpoint{6.968928in}{0.550000in}}%
\pgfpathlineto{\pgfqpoint{6.996449in}{0.550000in}}%
\pgfpathlineto{\pgfqpoint{6.996449in}{0.588269in}}%
\pgfpathlineto{\pgfqpoint{6.968928in}{0.588269in}}%
\pgfpathlineto{\pgfqpoint{6.968928in}{0.550000in}}%
\pgfpathclose%
\pgfusepath{fill}%
\end{pgfscope}%
\begin{pgfscope}%
\pgfpathrectangle{\pgfqpoint{1.250000in}{0.550000in}}{\pgfqpoint{7.750000in}{3.850000in}}%
\pgfusepath{clip}%
\pgfsetbuttcap%
\pgfsetmiterjoin%
\definecolor{currentfill}{rgb}{0.501961,0.501961,0.501961}%
\pgfsetfillcolor{currentfill}%
\pgfsetfillopacity{0.700000}%
\pgfsetlinewidth{0.000000pt}%
\definecolor{currentstroke}{rgb}{0.000000,0.000000,0.000000}%
\pgfsetstrokecolor{currentstroke}%
\pgfsetstrokeopacity{0.700000}%
\pgfsetdash{}{0pt}%
\pgfpathmoveto{\pgfqpoint{6.996449in}{0.550000in}}%
\pgfpathlineto{\pgfqpoint{7.023970in}{0.550000in}}%
\pgfpathlineto{\pgfqpoint{7.023970in}{0.600108in}}%
\pgfpathlineto{\pgfqpoint{6.996449in}{0.600108in}}%
\pgfpathlineto{\pgfqpoint{6.996449in}{0.550000in}}%
\pgfpathclose%
\pgfusepath{fill}%
\end{pgfscope}%
\begin{pgfscope}%
\pgfpathrectangle{\pgfqpoint{1.250000in}{0.550000in}}{\pgfqpoint{7.750000in}{3.850000in}}%
\pgfusepath{clip}%
\pgfsetbuttcap%
\pgfsetmiterjoin%
\definecolor{currentfill}{rgb}{0.501961,0.501961,0.501961}%
\pgfsetfillcolor{currentfill}%
\pgfsetfillopacity{0.700000}%
\pgfsetlinewidth{0.000000pt}%
\definecolor{currentstroke}{rgb}{0.000000,0.000000,0.000000}%
\pgfsetstrokecolor{currentstroke}%
\pgfsetstrokeopacity{0.700000}%
\pgfsetdash{}{0pt}%
\pgfpathmoveto{\pgfqpoint{7.023970in}{0.550000in}}%
\pgfpathlineto{\pgfqpoint{7.051491in}{0.550000in}}%
\pgfpathlineto{\pgfqpoint{7.051491in}{0.589646in}}%
\pgfpathlineto{\pgfqpoint{7.023970in}{0.589646in}}%
\pgfpathlineto{\pgfqpoint{7.023970in}{0.550000in}}%
\pgfpathclose%
\pgfusepath{fill}%
\end{pgfscope}%
\begin{pgfscope}%
\pgfpathrectangle{\pgfqpoint{1.250000in}{0.550000in}}{\pgfqpoint{7.750000in}{3.850000in}}%
\pgfusepath{clip}%
\pgfsetbuttcap%
\pgfsetmiterjoin%
\definecolor{currentfill}{rgb}{0.501961,0.501961,0.501961}%
\pgfsetfillcolor{currentfill}%
\pgfsetfillopacity{0.700000}%
\pgfsetlinewidth{0.000000pt}%
\definecolor{currentstroke}{rgb}{0.000000,0.000000,0.000000}%
\pgfsetstrokecolor{currentstroke}%
\pgfsetstrokeopacity{0.700000}%
\pgfsetdash{}{0pt}%
\pgfpathmoveto{\pgfqpoint{7.051491in}{0.550000in}}%
\pgfpathlineto{\pgfqpoint{7.079013in}{0.550000in}}%
\pgfpathlineto{\pgfqpoint{7.079013in}{0.597630in}}%
\pgfpathlineto{\pgfqpoint{7.051491in}{0.597630in}}%
\pgfpathlineto{\pgfqpoint{7.051491in}{0.550000in}}%
\pgfpathclose%
\pgfusepath{fill}%
\end{pgfscope}%
\begin{pgfscope}%
\pgfpathrectangle{\pgfqpoint{1.250000in}{0.550000in}}{\pgfqpoint{7.750000in}{3.850000in}}%
\pgfusepath{clip}%
\pgfsetbuttcap%
\pgfsetmiterjoin%
\definecolor{currentfill}{rgb}{0.501961,0.501961,0.501961}%
\pgfsetfillcolor{currentfill}%
\pgfsetfillopacity{0.700000}%
\pgfsetlinewidth{0.000000pt}%
\definecolor{currentstroke}{rgb}{0.000000,0.000000,0.000000}%
\pgfsetstrokecolor{currentstroke}%
\pgfsetstrokeopacity{0.700000}%
\pgfsetdash{}{0pt}%
\pgfpathmoveto{\pgfqpoint{7.079013in}{0.550000in}}%
\pgfpathlineto{\pgfqpoint{7.106534in}{0.550000in}}%
\pgfpathlineto{\pgfqpoint{7.106534in}{0.591573in}}%
\pgfpathlineto{\pgfqpoint{7.079013in}{0.591573in}}%
\pgfpathlineto{\pgfqpoint{7.079013in}{0.550000in}}%
\pgfpathclose%
\pgfusepath{fill}%
\end{pgfscope}%
\begin{pgfscope}%
\pgfpathrectangle{\pgfqpoint{1.250000in}{0.550000in}}{\pgfqpoint{7.750000in}{3.850000in}}%
\pgfusepath{clip}%
\pgfsetbuttcap%
\pgfsetmiterjoin%
\definecolor{currentfill}{rgb}{0.501961,0.501961,0.501961}%
\pgfsetfillcolor{currentfill}%
\pgfsetfillopacity{0.700000}%
\pgfsetlinewidth{0.000000pt}%
\definecolor{currentstroke}{rgb}{0.000000,0.000000,0.000000}%
\pgfsetstrokecolor{currentstroke}%
\pgfsetstrokeopacity{0.700000}%
\pgfsetdash{}{0pt}%
\pgfpathmoveto{\pgfqpoint{7.106534in}{0.550000in}}%
\pgfpathlineto{\pgfqpoint{7.134055in}{0.550000in}}%
\pgfpathlineto{\pgfqpoint{7.134055in}{0.592399in}}%
\pgfpathlineto{\pgfqpoint{7.106534in}{0.592399in}}%
\pgfpathlineto{\pgfqpoint{7.106534in}{0.550000in}}%
\pgfpathclose%
\pgfusepath{fill}%
\end{pgfscope}%
\begin{pgfscope}%
\pgfpathrectangle{\pgfqpoint{1.250000in}{0.550000in}}{\pgfqpoint{7.750000in}{3.850000in}}%
\pgfusepath{clip}%
\pgfsetbuttcap%
\pgfsetmiterjoin%
\definecolor{currentfill}{rgb}{0.501961,0.501961,0.501961}%
\pgfsetfillcolor{currentfill}%
\pgfsetfillopacity{0.700000}%
\pgfsetlinewidth{0.000000pt}%
\definecolor{currentstroke}{rgb}{0.000000,0.000000,0.000000}%
\pgfsetstrokecolor{currentstroke}%
\pgfsetstrokeopacity{0.700000}%
\pgfsetdash{}{0pt}%
\pgfpathmoveto{\pgfqpoint{7.134055in}{0.550000in}}%
\pgfpathlineto{\pgfqpoint{7.161577in}{0.550000in}}%
\pgfpathlineto{\pgfqpoint{7.161577in}{0.598731in}}%
\pgfpathlineto{\pgfqpoint{7.134055in}{0.598731in}}%
\pgfpathlineto{\pgfqpoint{7.134055in}{0.550000in}}%
\pgfpathclose%
\pgfusepath{fill}%
\end{pgfscope}%
\begin{pgfscope}%
\pgfpathrectangle{\pgfqpoint{1.250000in}{0.550000in}}{\pgfqpoint{7.750000in}{3.850000in}}%
\pgfusepath{clip}%
\pgfsetbuttcap%
\pgfsetmiterjoin%
\definecolor{currentfill}{rgb}{0.501961,0.501961,0.501961}%
\pgfsetfillcolor{currentfill}%
\pgfsetfillopacity{0.700000}%
\pgfsetlinewidth{0.000000pt}%
\definecolor{currentstroke}{rgb}{0.000000,0.000000,0.000000}%
\pgfsetstrokecolor{currentstroke}%
\pgfsetstrokeopacity{0.700000}%
\pgfsetdash{}{0pt}%
\pgfpathmoveto{\pgfqpoint{7.161577in}{0.550000in}}%
\pgfpathlineto{\pgfqpoint{7.189098in}{0.550000in}}%
\pgfpathlineto{\pgfqpoint{7.189098in}{0.584965in}}%
\pgfpathlineto{\pgfqpoint{7.161577in}{0.584965in}}%
\pgfpathlineto{\pgfqpoint{7.161577in}{0.550000in}}%
\pgfpathclose%
\pgfusepath{fill}%
\end{pgfscope}%
\begin{pgfscope}%
\pgfpathrectangle{\pgfqpoint{1.250000in}{0.550000in}}{\pgfqpoint{7.750000in}{3.850000in}}%
\pgfusepath{clip}%
\pgfsetbuttcap%
\pgfsetmiterjoin%
\definecolor{currentfill}{rgb}{0.501961,0.501961,0.501961}%
\pgfsetfillcolor{currentfill}%
\pgfsetfillopacity{0.700000}%
\pgfsetlinewidth{0.000000pt}%
\definecolor{currentstroke}{rgb}{0.000000,0.000000,0.000000}%
\pgfsetstrokecolor{currentstroke}%
\pgfsetstrokeopacity{0.700000}%
\pgfsetdash{}{0pt}%
\pgfpathmoveto{\pgfqpoint{7.189098in}{0.550000in}}%
\pgfpathlineto{\pgfqpoint{7.216619in}{0.550000in}}%
\pgfpathlineto{\pgfqpoint{7.216619in}{0.594326in}}%
\pgfpathlineto{\pgfqpoint{7.189098in}{0.594326in}}%
\pgfpathlineto{\pgfqpoint{7.189098in}{0.550000in}}%
\pgfpathclose%
\pgfusepath{fill}%
\end{pgfscope}%
\begin{pgfscope}%
\pgfpathrectangle{\pgfqpoint{1.250000in}{0.550000in}}{\pgfqpoint{7.750000in}{3.850000in}}%
\pgfusepath{clip}%
\pgfsetbuttcap%
\pgfsetmiterjoin%
\definecolor{currentfill}{rgb}{0.501961,0.501961,0.501961}%
\pgfsetfillcolor{currentfill}%
\pgfsetfillopacity{0.700000}%
\pgfsetlinewidth{0.000000pt}%
\definecolor{currentstroke}{rgb}{0.000000,0.000000,0.000000}%
\pgfsetstrokecolor{currentstroke}%
\pgfsetstrokeopacity{0.700000}%
\pgfsetdash{}{0pt}%
\pgfpathmoveto{\pgfqpoint{7.216619in}{0.550000in}}%
\pgfpathlineto{\pgfqpoint{7.244141in}{0.550000in}}%
\pgfpathlineto{\pgfqpoint{7.244141in}{0.590747in}}%
\pgfpathlineto{\pgfqpoint{7.216619in}{0.590747in}}%
\pgfpathlineto{\pgfqpoint{7.216619in}{0.550000in}}%
\pgfpathclose%
\pgfusepath{fill}%
\end{pgfscope}%
\begin{pgfscope}%
\pgfpathrectangle{\pgfqpoint{1.250000in}{0.550000in}}{\pgfqpoint{7.750000in}{3.850000in}}%
\pgfusepath{clip}%
\pgfsetbuttcap%
\pgfsetmiterjoin%
\definecolor{currentfill}{rgb}{0.501961,0.501961,0.501961}%
\pgfsetfillcolor{currentfill}%
\pgfsetfillopacity{0.700000}%
\pgfsetlinewidth{0.000000pt}%
\definecolor{currentstroke}{rgb}{0.000000,0.000000,0.000000}%
\pgfsetstrokecolor{currentstroke}%
\pgfsetstrokeopacity{0.700000}%
\pgfsetdash{}{0pt}%
\pgfpathmoveto{\pgfqpoint{7.244141in}{0.550000in}}%
\pgfpathlineto{\pgfqpoint{7.271662in}{0.550000in}}%
\pgfpathlineto{\pgfqpoint{7.271662in}{0.591573in}}%
\pgfpathlineto{\pgfqpoint{7.244141in}{0.591573in}}%
\pgfpathlineto{\pgfqpoint{7.244141in}{0.550000in}}%
\pgfpathclose%
\pgfusepath{fill}%
\end{pgfscope}%
\begin{pgfscope}%
\pgfpathrectangle{\pgfqpoint{1.250000in}{0.550000in}}{\pgfqpoint{7.750000in}{3.850000in}}%
\pgfusepath{clip}%
\pgfsetbuttcap%
\pgfsetmiterjoin%
\definecolor{currentfill}{rgb}{0.501961,0.501961,0.501961}%
\pgfsetfillcolor{currentfill}%
\pgfsetfillopacity{0.700000}%
\pgfsetlinewidth{0.000000pt}%
\definecolor{currentstroke}{rgb}{0.000000,0.000000,0.000000}%
\pgfsetstrokecolor{currentstroke}%
\pgfsetstrokeopacity{0.700000}%
\pgfsetdash{}{0pt}%
\pgfpathmoveto{\pgfqpoint{7.271662in}{0.550000in}}%
\pgfpathlineto{\pgfqpoint{7.299183in}{0.550000in}}%
\pgfpathlineto{\pgfqpoint{7.299183in}{0.593775in}}%
\pgfpathlineto{\pgfqpoint{7.271662in}{0.593775in}}%
\pgfpathlineto{\pgfqpoint{7.271662in}{0.550000in}}%
\pgfpathclose%
\pgfusepath{fill}%
\end{pgfscope}%
\begin{pgfscope}%
\pgfpathrectangle{\pgfqpoint{1.250000in}{0.550000in}}{\pgfqpoint{7.750000in}{3.850000in}}%
\pgfusepath{clip}%
\pgfsetbuttcap%
\pgfsetmiterjoin%
\definecolor{currentfill}{rgb}{0.501961,0.501961,0.501961}%
\pgfsetfillcolor{currentfill}%
\pgfsetfillopacity{0.700000}%
\pgfsetlinewidth{0.000000pt}%
\definecolor{currentstroke}{rgb}{0.000000,0.000000,0.000000}%
\pgfsetstrokecolor{currentstroke}%
\pgfsetstrokeopacity{0.700000}%
\pgfsetdash{}{0pt}%
\pgfpathmoveto{\pgfqpoint{7.299183in}{0.550000in}}%
\pgfpathlineto{\pgfqpoint{7.326705in}{0.550000in}}%
\pgfpathlineto{\pgfqpoint{7.326705in}{0.588820in}}%
\pgfpathlineto{\pgfqpoint{7.299183in}{0.588820in}}%
\pgfpathlineto{\pgfqpoint{7.299183in}{0.550000in}}%
\pgfpathclose%
\pgfusepath{fill}%
\end{pgfscope}%
\begin{pgfscope}%
\pgfpathrectangle{\pgfqpoint{1.250000in}{0.550000in}}{\pgfqpoint{7.750000in}{3.850000in}}%
\pgfusepath{clip}%
\pgfsetbuttcap%
\pgfsetmiterjoin%
\definecolor{currentfill}{rgb}{0.501961,0.501961,0.501961}%
\pgfsetfillcolor{currentfill}%
\pgfsetfillopacity{0.700000}%
\pgfsetlinewidth{0.000000pt}%
\definecolor{currentstroke}{rgb}{0.000000,0.000000,0.000000}%
\pgfsetstrokecolor{currentstroke}%
\pgfsetstrokeopacity{0.700000}%
\pgfsetdash{}{0pt}%
\pgfpathmoveto{\pgfqpoint{7.326705in}{0.550000in}}%
\pgfpathlineto{\pgfqpoint{7.354226in}{0.550000in}}%
\pgfpathlineto{\pgfqpoint{7.354226in}{0.587443in}}%
\pgfpathlineto{\pgfqpoint{7.326705in}{0.587443in}}%
\pgfpathlineto{\pgfqpoint{7.326705in}{0.550000in}}%
\pgfpathclose%
\pgfusepath{fill}%
\end{pgfscope}%
\begin{pgfscope}%
\pgfpathrectangle{\pgfqpoint{1.250000in}{0.550000in}}{\pgfqpoint{7.750000in}{3.850000in}}%
\pgfusepath{clip}%
\pgfsetbuttcap%
\pgfsetmiterjoin%
\definecolor{currentfill}{rgb}{0.501961,0.501961,0.501961}%
\pgfsetfillcolor{currentfill}%
\pgfsetfillopacity{0.700000}%
\pgfsetlinewidth{0.000000pt}%
\definecolor{currentstroke}{rgb}{0.000000,0.000000,0.000000}%
\pgfsetstrokecolor{currentstroke}%
\pgfsetstrokeopacity{0.700000}%
\pgfsetdash{}{0pt}%
\pgfpathmoveto{\pgfqpoint{7.354226in}{0.550000in}}%
\pgfpathlineto{\pgfqpoint{7.381747in}{0.550000in}}%
\pgfpathlineto{\pgfqpoint{7.381747in}{0.583589in}}%
\pgfpathlineto{\pgfqpoint{7.354226in}{0.583589in}}%
\pgfpathlineto{\pgfqpoint{7.354226in}{0.550000in}}%
\pgfpathclose%
\pgfusepath{fill}%
\end{pgfscope}%
\begin{pgfscope}%
\pgfpathrectangle{\pgfqpoint{1.250000in}{0.550000in}}{\pgfqpoint{7.750000in}{3.850000in}}%
\pgfusepath{clip}%
\pgfsetbuttcap%
\pgfsetmiterjoin%
\definecolor{currentfill}{rgb}{0.501961,0.501961,0.501961}%
\pgfsetfillcolor{currentfill}%
\pgfsetfillopacity{0.700000}%
\pgfsetlinewidth{0.000000pt}%
\definecolor{currentstroke}{rgb}{0.000000,0.000000,0.000000}%
\pgfsetstrokecolor{currentstroke}%
\pgfsetstrokeopacity{0.700000}%
\pgfsetdash{}{0pt}%
\pgfpathmoveto{\pgfqpoint{7.381747in}{0.550000in}}%
\pgfpathlineto{\pgfqpoint{7.409268in}{0.550000in}}%
\pgfpathlineto{\pgfqpoint{7.409268in}{0.588269in}}%
\pgfpathlineto{\pgfqpoint{7.381747in}{0.588269in}}%
\pgfpathlineto{\pgfqpoint{7.381747in}{0.550000in}}%
\pgfpathclose%
\pgfusepath{fill}%
\end{pgfscope}%
\begin{pgfscope}%
\pgfpathrectangle{\pgfqpoint{1.250000in}{0.550000in}}{\pgfqpoint{7.750000in}{3.850000in}}%
\pgfusepath{clip}%
\pgfsetbuttcap%
\pgfsetmiterjoin%
\definecolor{currentfill}{rgb}{0.501961,0.501961,0.501961}%
\pgfsetfillcolor{currentfill}%
\pgfsetfillopacity{0.700000}%
\pgfsetlinewidth{0.000000pt}%
\definecolor{currentstroke}{rgb}{0.000000,0.000000,0.000000}%
\pgfsetstrokecolor{currentstroke}%
\pgfsetstrokeopacity{0.700000}%
\pgfsetdash{}{0pt}%
\pgfpathmoveto{\pgfqpoint{7.409268in}{0.550000in}}%
\pgfpathlineto{\pgfqpoint{7.436790in}{0.550000in}}%
\pgfpathlineto{\pgfqpoint{7.436790in}{0.584965in}}%
\pgfpathlineto{\pgfqpoint{7.409268in}{0.584965in}}%
\pgfpathlineto{\pgfqpoint{7.409268in}{0.550000in}}%
\pgfpathclose%
\pgfusepath{fill}%
\end{pgfscope}%
\begin{pgfscope}%
\pgfpathrectangle{\pgfqpoint{1.250000in}{0.550000in}}{\pgfqpoint{7.750000in}{3.850000in}}%
\pgfusepath{clip}%
\pgfsetbuttcap%
\pgfsetmiterjoin%
\definecolor{currentfill}{rgb}{0.501961,0.501961,0.501961}%
\pgfsetfillcolor{currentfill}%
\pgfsetfillopacity{0.700000}%
\pgfsetlinewidth{0.000000pt}%
\definecolor{currentstroke}{rgb}{0.000000,0.000000,0.000000}%
\pgfsetstrokecolor{currentstroke}%
\pgfsetstrokeopacity{0.700000}%
\pgfsetdash{}{0pt}%
\pgfpathmoveto{\pgfqpoint{7.436790in}{0.550000in}}%
\pgfpathlineto{\pgfqpoint{7.464311in}{0.550000in}}%
\pgfpathlineto{\pgfqpoint{7.464311in}{0.587443in}}%
\pgfpathlineto{\pgfqpoint{7.436790in}{0.587443in}}%
\pgfpathlineto{\pgfqpoint{7.436790in}{0.550000in}}%
\pgfpathclose%
\pgfusepath{fill}%
\end{pgfscope}%
\begin{pgfscope}%
\pgfpathrectangle{\pgfqpoint{1.250000in}{0.550000in}}{\pgfqpoint{7.750000in}{3.850000in}}%
\pgfusepath{clip}%
\pgfsetbuttcap%
\pgfsetmiterjoin%
\definecolor{currentfill}{rgb}{0.501961,0.501961,0.501961}%
\pgfsetfillcolor{currentfill}%
\pgfsetfillopacity{0.700000}%
\pgfsetlinewidth{0.000000pt}%
\definecolor{currentstroke}{rgb}{0.000000,0.000000,0.000000}%
\pgfsetstrokecolor{currentstroke}%
\pgfsetstrokeopacity{0.700000}%
\pgfsetdash{}{0pt}%
\pgfpathmoveto{\pgfqpoint{7.464311in}{0.550000in}}%
\pgfpathlineto{\pgfqpoint{7.491832in}{0.550000in}}%
\pgfpathlineto{\pgfqpoint{7.491832in}{0.586892in}}%
\pgfpathlineto{\pgfqpoint{7.464311in}{0.586892in}}%
\pgfpathlineto{\pgfqpoint{7.464311in}{0.550000in}}%
\pgfpathclose%
\pgfusepath{fill}%
\end{pgfscope}%
\begin{pgfscope}%
\pgfpathrectangle{\pgfqpoint{1.250000in}{0.550000in}}{\pgfqpoint{7.750000in}{3.850000in}}%
\pgfusepath{clip}%
\pgfsetbuttcap%
\pgfsetmiterjoin%
\definecolor{currentfill}{rgb}{0.501961,0.501961,0.501961}%
\pgfsetfillcolor{currentfill}%
\pgfsetfillopacity{0.700000}%
\pgfsetlinewidth{0.000000pt}%
\definecolor{currentstroke}{rgb}{0.000000,0.000000,0.000000}%
\pgfsetstrokecolor{currentstroke}%
\pgfsetstrokeopacity{0.700000}%
\pgfsetdash{}{0pt}%
\pgfpathmoveto{\pgfqpoint{7.491832in}{0.550000in}}%
\pgfpathlineto{\pgfqpoint{7.519354in}{0.550000in}}%
\pgfpathlineto{\pgfqpoint{7.519354in}{0.584415in}}%
\pgfpathlineto{\pgfqpoint{7.491832in}{0.584415in}}%
\pgfpathlineto{\pgfqpoint{7.491832in}{0.550000in}}%
\pgfpathclose%
\pgfusepath{fill}%
\end{pgfscope}%
\begin{pgfscope}%
\pgfpathrectangle{\pgfqpoint{1.250000in}{0.550000in}}{\pgfqpoint{7.750000in}{3.850000in}}%
\pgfusepath{clip}%
\pgfsetbuttcap%
\pgfsetmiterjoin%
\definecolor{currentfill}{rgb}{0.501961,0.501961,0.501961}%
\pgfsetfillcolor{currentfill}%
\pgfsetfillopacity{0.700000}%
\pgfsetlinewidth{0.000000pt}%
\definecolor{currentstroke}{rgb}{0.000000,0.000000,0.000000}%
\pgfsetstrokecolor{currentstroke}%
\pgfsetstrokeopacity{0.700000}%
\pgfsetdash{}{0pt}%
\pgfpathmoveto{\pgfqpoint{7.519354in}{0.550000in}}%
\pgfpathlineto{\pgfqpoint{7.546875in}{0.550000in}}%
\pgfpathlineto{\pgfqpoint{7.546875in}{0.578082in}}%
\pgfpathlineto{\pgfqpoint{7.519354in}{0.578082in}}%
\pgfpathlineto{\pgfqpoint{7.519354in}{0.550000in}}%
\pgfpathclose%
\pgfusepath{fill}%
\end{pgfscope}%
\begin{pgfscope}%
\pgfpathrectangle{\pgfqpoint{1.250000in}{0.550000in}}{\pgfqpoint{7.750000in}{3.850000in}}%
\pgfusepath{clip}%
\pgfsetbuttcap%
\pgfsetmiterjoin%
\definecolor{currentfill}{rgb}{0.501961,0.501961,0.501961}%
\pgfsetfillcolor{currentfill}%
\pgfsetfillopacity{0.700000}%
\pgfsetlinewidth{0.000000pt}%
\definecolor{currentstroke}{rgb}{0.000000,0.000000,0.000000}%
\pgfsetstrokecolor{currentstroke}%
\pgfsetstrokeopacity{0.700000}%
\pgfsetdash{}{0pt}%
\pgfpathmoveto{\pgfqpoint{7.546875in}{0.550000in}}%
\pgfpathlineto{\pgfqpoint{7.574396in}{0.550000in}}%
\pgfpathlineto{\pgfqpoint{7.574396in}{0.581111in}}%
\pgfpathlineto{\pgfqpoint{7.546875in}{0.581111in}}%
\pgfpathlineto{\pgfqpoint{7.546875in}{0.550000in}}%
\pgfpathclose%
\pgfusepath{fill}%
\end{pgfscope}%
\begin{pgfscope}%
\pgfpathrectangle{\pgfqpoint{1.250000in}{0.550000in}}{\pgfqpoint{7.750000in}{3.850000in}}%
\pgfusepath{clip}%
\pgfsetbuttcap%
\pgfsetmiterjoin%
\definecolor{currentfill}{rgb}{0.501961,0.501961,0.501961}%
\pgfsetfillcolor{currentfill}%
\pgfsetfillopacity{0.700000}%
\pgfsetlinewidth{0.000000pt}%
\definecolor{currentstroke}{rgb}{0.000000,0.000000,0.000000}%
\pgfsetstrokecolor{currentstroke}%
\pgfsetstrokeopacity{0.700000}%
\pgfsetdash{}{0pt}%
\pgfpathmoveto{\pgfqpoint{7.574396in}{0.550000in}}%
\pgfpathlineto{\pgfqpoint{7.601918in}{0.550000in}}%
\pgfpathlineto{\pgfqpoint{7.601918in}{0.583313in}}%
\pgfpathlineto{\pgfqpoint{7.574396in}{0.583313in}}%
\pgfpathlineto{\pgfqpoint{7.574396in}{0.550000in}}%
\pgfpathclose%
\pgfusepath{fill}%
\end{pgfscope}%
\begin{pgfscope}%
\pgfpathrectangle{\pgfqpoint{1.250000in}{0.550000in}}{\pgfqpoint{7.750000in}{3.850000in}}%
\pgfusepath{clip}%
\pgfsetbuttcap%
\pgfsetmiterjoin%
\definecolor{currentfill}{rgb}{0.501961,0.501961,0.501961}%
\pgfsetfillcolor{currentfill}%
\pgfsetfillopacity{0.700000}%
\pgfsetlinewidth{0.000000pt}%
\definecolor{currentstroke}{rgb}{0.000000,0.000000,0.000000}%
\pgfsetstrokecolor{currentstroke}%
\pgfsetstrokeopacity{0.700000}%
\pgfsetdash{}{0pt}%
\pgfpathmoveto{\pgfqpoint{7.601918in}{0.550000in}}%
\pgfpathlineto{\pgfqpoint{7.629439in}{0.550000in}}%
\pgfpathlineto{\pgfqpoint{7.629439in}{0.583038in}}%
\pgfpathlineto{\pgfqpoint{7.601918in}{0.583038in}}%
\pgfpathlineto{\pgfqpoint{7.601918in}{0.550000in}}%
\pgfpathclose%
\pgfusepath{fill}%
\end{pgfscope}%
\begin{pgfscope}%
\pgfpathrectangle{\pgfqpoint{1.250000in}{0.550000in}}{\pgfqpoint{7.750000in}{3.850000in}}%
\pgfusepath{clip}%
\pgfsetbuttcap%
\pgfsetmiterjoin%
\definecolor{currentfill}{rgb}{0.501961,0.501961,0.501961}%
\pgfsetfillcolor{currentfill}%
\pgfsetfillopacity{0.700000}%
\pgfsetlinewidth{0.000000pt}%
\definecolor{currentstroke}{rgb}{0.000000,0.000000,0.000000}%
\pgfsetstrokecolor{currentstroke}%
\pgfsetstrokeopacity{0.700000}%
\pgfsetdash{}{0pt}%
\pgfpathmoveto{\pgfqpoint{7.629439in}{0.550000in}}%
\pgfpathlineto{\pgfqpoint{7.656960in}{0.550000in}}%
\pgfpathlineto{\pgfqpoint{7.656960in}{0.580560in}}%
\pgfpathlineto{\pgfqpoint{7.629439in}{0.580560in}}%
\pgfpathlineto{\pgfqpoint{7.629439in}{0.550000in}}%
\pgfpathclose%
\pgfusepath{fill}%
\end{pgfscope}%
\begin{pgfscope}%
\pgfpathrectangle{\pgfqpoint{1.250000in}{0.550000in}}{\pgfqpoint{7.750000in}{3.850000in}}%
\pgfusepath{clip}%
\pgfsetbuttcap%
\pgfsetmiterjoin%
\definecolor{currentfill}{rgb}{0.501961,0.501961,0.501961}%
\pgfsetfillcolor{currentfill}%
\pgfsetfillopacity{0.700000}%
\pgfsetlinewidth{0.000000pt}%
\definecolor{currentstroke}{rgb}{0.000000,0.000000,0.000000}%
\pgfsetstrokecolor{currentstroke}%
\pgfsetstrokeopacity{0.700000}%
\pgfsetdash{}{0pt}%
\pgfpathmoveto{\pgfqpoint{7.656960in}{0.550000in}}%
\pgfpathlineto{\pgfqpoint{7.684482in}{0.550000in}}%
\pgfpathlineto{\pgfqpoint{7.684482in}{0.582763in}}%
\pgfpathlineto{\pgfqpoint{7.656960in}{0.582763in}}%
\pgfpathlineto{\pgfqpoint{7.656960in}{0.550000in}}%
\pgfpathclose%
\pgfusepath{fill}%
\end{pgfscope}%
\begin{pgfscope}%
\pgfpathrectangle{\pgfqpoint{1.250000in}{0.550000in}}{\pgfqpoint{7.750000in}{3.850000in}}%
\pgfusepath{clip}%
\pgfsetbuttcap%
\pgfsetmiterjoin%
\definecolor{currentfill}{rgb}{0.501961,0.501961,0.501961}%
\pgfsetfillcolor{currentfill}%
\pgfsetfillopacity{0.700000}%
\pgfsetlinewidth{0.000000pt}%
\definecolor{currentstroke}{rgb}{0.000000,0.000000,0.000000}%
\pgfsetstrokecolor{currentstroke}%
\pgfsetstrokeopacity{0.700000}%
\pgfsetdash{}{0pt}%
\pgfpathmoveto{\pgfqpoint{7.684482in}{0.550000in}}%
\pgfpathlineto{\pgfqpoint{7.712003in}{0.550000in}}%
\pgfpathlineto{\pgfqpoint{7.712003in}{0.581661in}}%
\pgfpathlineto{\pgfqpoint{7.684482in}{0.581661in}}%
\pgfpathlineto{\pgfqpoint{7.684482in}{0.550000in}}%
\pgfpathclose%
\pgfusepath{fill}%
\end{pgfscope}%
\begin{pgfscope}%
\pgfpathrectangle{\pgfqpoint{1.250000in}{0.550000in}}{\pgfqpoint{7.750000in}{3.850000in}}%
\pgfusepath{clip}%
\pgfsetbuttcap%
\pgfsetmiterjoin%
\definecolor{currentfill}{rgb}{0.501961,0.501961,0.501961}%
\pgfsetfillcolor{currentfill}%
\pgfsetfillopacity{0.700000}%
\pgfsetlinewidth{0.000000pt}%
\definecolor{currentstroke}{rgb}{0.000000,0.000000,0.000000}%
\pgfsetstrokecolor{currentstroke}%
\pgfsetstrokeopacity{0.700000}%
\pgfsetdash{}{0pt}%
\pgfpathmoveto{\pgfqpoint{7.712003in}{0.550000in}}%
\pgfpathlineto{\pgfqpoint{7.739524in}{0.550000in}}%
\pgfpathlineto{\pgfqpoint{7.739524in}{0.586066in}}%
\pgfpathlineto{\pgfqpoint{7.712003in}{0.586066in}}%
\pgfpathlineto{\pgfqpoint{7.712003in}{0.550000in}}%
\pgfpathclose%
\pgfusepath{fill}%
\end{pgfscope}%
\begin{pgfscope}%
\pgfpathrectangle{\pgfqpoint{1.250000in}{0.550000in}}{\pgfqpoint{7.750000in}{3.850000in}}%
\pgfusepath{clip}%
\pgfsetbuttcap%
\pgfsetmiterjoin%
\definecolor{currentfill}{rgb}{0.501961,0.501961,0.501961}%
\pgfsetfillcolor{currentfill}%
\pgfsetfillopacity{0.700000}%
\pgfsetlinewidth{0.000000pt}%
\definecolor{currentstroke}{rgb}{0.000000,0.000000,0.000000}%
\pgfsetstrokecolor{currentstroke}%
\pgfsetstrokeopacity{0.700000}%
\pgfsetdash{}{0pt}%
\pgfpathmoveto{\pgfqpoint{7.739524in}{0.550000in}}%
\pgfpathlineto{\pgfqpoint{7.767045in}{0.550000in}}%
\pgfpathlineto{\pgfqpoint{7.767045in}{0.587443in}}%
\pgfpathlineto{\pgfqpoint{7.739524in}{0.587443in}}%
\pgfpathlineto{\pgfqpoint{7.739524in}{0.550000in}}%
\pgfpathclose%
\pgfusepath{fill}%
\end{pgfscope}%
\begin{pgfscope}%
\pgfpathrectangle{\pgfqpoint{1.250000in}{0.550000in}}{\pgfqpoint{7.750000in}{3.850000in}}%
\pgfusepath{clip}%
\pgfsetbuttcap%
\pgfsetmiterjoin%
\definecolor{currentfill}{rgb}{0.501961,0.501961,0.501961}%
\pgfsetfillcolor{currentfill}%
\pgfsetfillopacity{0.700000}%
\pgfsetlinewidth{0.000000pt}%
\definecolor{currentstroke}{rgb}{0.000000,0.000000,0.000000}%
\pgfsetstrokecolor{currentstroke}%
\pgfsetstrokeopacity{0.700000}%
\pgfsetdash{}{0pt}%
\pgfpathmoveto{\pgfqpoint{7.767045in}{0.550000in}}%
\pgfpathlineto{\pgfqpoint{7.794567in}{0.550000in}}%
\pgfpathlineto{\pgfqpoint{7.794567in}{0.582487in}}%
\pgfpathlineto{\pgfqpoint{7.767045in}{0.582487in}}%
\pgfpathlineto{\pgfqpoint{7.767045in}{0.550000in}}%
\pgfpathclose%
\pgfusepath{fill}%
\end{pgfscope}%
\begin{pgfscope}%
\pgfpathrectangle{\pgfqpoint{1.250000in}{0.550000in}}{\pgfqpoint{7.750000in}{3.850000in}}%
\pgfusepath{clip}%
\pgfsetbuttcap%
\pgfsetmiterjoin%
\definecolor{currentfill}{rgb}{0.501961,0.501961,0.501961}%
\pgfsetfillcolor{currentfill}%
\pgfsetfillopacity{0.700000}%
\pgfsetlinewidth{0.000000pt}%
\definecolor{currentstroke}{rgb}{0.000000,0.000000,0.000000}%
\pgfsetstrokecolor{currentstroke}%
\pgfsetstrokeopacity{0.700000}%
\pgfsetdash{}{0pt}%
\pgfpathmoveto{\pgfqpoint{7.794567in}{0.550000in}}%
\pgfpathlineto{\pgfqpoint{7.822088in}{0.550000in}}%
\pgfpathlineto{\pgfqpoint{7.822088in}{0.589095in}}%
\pgfpathlineto{\pgfqpoint{7.794567in}{0.589095in}}%
\pgfpathlineto{\pgfqpoint{7.794567in}{0.550000in}}%
\pgfpathclose%
\pgfusepath{fill}%
\end{pgfscope}%
\begin{pgfscope}%
\pgfpathrectangle{\pgfqpoint{1.250000in}{0.550000in}}{\pgfqpoint{7.750000in}{3.850000in}}%
\pgfusepath{clip}%
\pgfsetbuttcap%
\pgfsetmiterjoin%
\definecolor{currentfill}{rgb}{0.501961,0.501961,0.501961}%
\pgfsetfillcolor{currentfill}%
\pgfsetfillopacity{0.700000}%
\pgfsetlinewidth{0.000000pt}%
\definecolor{currentstroke}{rgb}{0.000000,0.000000,0.000000}%
\pgfsetstrokecolor{currentstroke}%
\pgfsetstrokeopacity{0.700000}%
\pgfsetdash{}{0pt}%
\pgfpathmoveto{\pgfqpoint{7.822088in}{0.550000in}}%
\pgfpathlineto{\pgfqpoint{7.849609in}{0.550000in}}%
\pgfpathlineto{\pgfqpoint{7.849609in}{0.580835in}}%
\pgfpathlineto{\pgfqpoint{7.822088in}{0.580835in}}%
\pgfpathlineto{\pgfqpoint{7.822088in}{0.550000in}}%
\pgfpathclose%
\pgfusepath{fill}%
\end{pgfscope}%
\begin{pgfscope}%
\pgfpathrectangle{\pgfqpoint{1.250000in}{0.550000in}}{\pgfqpoint{7.750000in}{3.850000in}}%
\pgfusepath{clip}%
\pgfsetbuttcap%
\pgfsetmiterjoin%
\definecolor{currentfill}{rgb}{0.501961,0.501961,0.501961}%
\pgfsetfillcolor{currentfill}%
\pgfsetfillopacity{0.700000}%
\pgfsetlinewidth{0.000000pt}%
\definecolor{currentstroke}{rgb}{0.000000,0.000000,0.000000}%
\pgfsetstrokecolor{currentstroke}%
\pgfsetstrokeopacity{0.700000}%
\pgfsetdash{}{0pt}%
\pgfpathmoveto{\pgfqpoint{7.849609in}{0.550000in}}%
\pgfpathlineto{\pgfqpoint{7.877131in}{0.550000in}}%
\pgfpathlineto{\pgfqpoint{7.877131in}{0.578082in}}%
\pgfpathlineto{\pgfqpoint{7.849609in}{0.578082in}}%
\pgfpathlineto{\pgfqpoint{7.849609in}{0.550000in}}%
\pgfpathclose%
\pgfusepath{fill}%
\end{pgfscope}%
\begin{pgfscope}%
\pgfpathrectangle{\pgfqpoint{1.250000in}{0.550000in}}{\pgfqpoint{7.750000in}{3.850000in}}%
\pgfusepath{clip}%
\pgfsetbuttcap%
\pgfsetmiterjoin%
\definecolor{currentfill}{rgb}{0.501961,0.501961,0.501961}%
\pgfsetfillcolor{currentfill}%
\pgfsetfillopacity{0.700000}%
\pgfsetlinewidth{0.000000pt}%
\definecolor{currentstroke}{rgb}{0.000000,0.000000,0.000000}%
\pgfsetstrokecolor{currentstroke}%
\pgfsetstrokeopacity{0.700000}%
\pgfsetdash{}{0pt}%
\pgfpathmoveto{\pgfqpoint{7.877131in}{0.550000in}}%
\pgfpathlineto{\pgfqpoint{7.904652in}{0.550000in}}%
\pgfpathlineto{\pgfqpoint{7.904652in}{0.588544in}}%
\pgfpathlineto{\pgfqpoint{7.877131in}{0.588544in}}%
\pgfpathlineto{\pgfqpoint{7.877131in}{0.550000in}}%
\pgfpathclose%
\pgfusepath{fill}%
\end{pgfscope}%
\begin{pgfscope}%
\pgfpathrectangle{\pgfqpoint{1.250000in}{0.550000in}}{\pgfqpoint{7.750000in}{3.850000in}}%
\pgfusepath{clip}%
\pgfsetbuttcap%
\pgfsetmiterjoin%
\definecolor{currentfill}{rgb}{0.501961,0.501961,0.501961}%
\pgfsetfillcolor{currentfill}%
\pgfsetfillopacity{0.700000}%
\pgfsetlinewidth{0.000000pt}%
\definecolor{currentstroke}{rgb}{0.000000,0.000000,0.000000}%
\pgfsetstrokecolor{currentstroke}%
\pgfsetstrokeopacity{0.700000}%
\pgfsetdash{}{0pt}%
\pgfpathmoveto{\pgfqpoint{7.904652in}{0.550000in}}%
\pgfpathlineto{\pgfqpoint{7.932173in}{0.550000in}}%
\pgfpathlineto{\pgfqpoint{7.932173in}{0.582212in}}%
\pgfpathlineto{\pgfqpoint{7.904652in}{0.582212in}}%
\pgfpathlineto{\pgfqpoint{7.904652in}{0.550000in}}%
\pgfpathclose%
\pgfusepath{fill}%
\end{pgfscope}%
\begin{pgfscope}%
\pgfpathrectangle{\pgfqpoint{1.250000in}{0.550000in}}{\pgfqpoint{7.750000in}{3.850000in}}%
\pgfusepath{clip}%
\pgfsetbuttcap%
\pgfsetmiterjoin%
\definecolor{currentfill}{rgb}{0.501961,0.501961,0.501961}%
\pgfsetfillcolor{currentfill}%
\pgfsetfillopacity{0.700000}%
\pgfsetlinewidth{0.000000pt}%
\definecolor{currentstroke}{rgb}{0.000000,0.000000,0.000000}%
\pgfsetstrokecolor{currentstroke}%
\pgfsetstrokeopacity{0.700000}%
\pgfsetdash{}{0pt}%
\pgfpathmoveto{\pgfqpoint{7.932173in}{0.550000in}}%
\pgfpathlineto{\pgfqpoint{7.959695in}{0.550000in}}%
\pgfpathlineto{\pgfqpoint{7.959695in}{0.583589in}}%
\pgfpathlineto{\pgfqpoint{7.932173in}{0.583589in}}%
\pgfpathlineto{\pgfqpoint{7.932173in}{0.550000in}}%
\pgfpathclose%
\pgfusepath{fill}%
\end{pgfscope}%
\begin{pgfscope}%
\pgfpathrectangle{\pgfqpoint{1.250000in}{0.550000in}}{\pgfqpoint{7.750000in}{3.850000in}}%
\pgfusepath{clip}%
\pgfsetbuttcap%
\pgfsetmiterjoin%
\definecolor{currentfill}{rgb}{0.501961,0.501961,0.501961}%
\pgfsetfillcolor{currentfill}%
\pgfsetfillopacity{0.700000}%
\pgfsetlinewidth{0.000000pt}%
\definecolor{currentstroke}{rgb}{0.000000,0.000000,0.000000}%
\pgfsetstrokecolor{currentstroke}%
\pgfsetstrokeopacity{0.700000}%
\pgfsetdash{}{0pt}%
\pgfpathmoveto{\pgfqpoint{7.959695in}{0.550000in}}%
\pgfpathlineto{\pgfqpoint{7.987216in}{0.550000in}}%
\pgfpathlineto{\pgfqpoint{7.987216in}{0.582487in}}%
\pgfpathlineto{\pgfqpoint{7.959695in}{0.582487in}}%
\pgfpathlineto{\pgfqpoint{7.959695in}{0.550000in}}%
\pgfpathclose%
\pgfusepath{fill}%
\end{pgfscope}%
\begin{pgfscope}%
\pgfpathrectangle{\pgfqpoint{1.250000in}{0.550000in}}{\pgfqpoint{7.750000in}{3.850000in}}%
\pgfusepath{clip}%
\pgfsetbuttcap%
\pgfsetmiterjoin%
\definecolor{currentfill}{rgb}{0.501961,0.501961,0.501961}%
\pgfsetfillcolor{currentfill}%
\pgfsetfillopacity{0.700000}%
\pgfsetlinewidth{0.000000pt}%
\definecolor{currentstroke}{rgb}{0.000000,0.000000,0.000000}%
\pgfsetstrokecolor{currentstroke}%
\pgfsetstrokeopacity{0.700000}%
\pgfsetdash{}{0pt}%
\pgfpathmoveto{\pgfqpoint{7.987216in}{0.550000in}}%
\pgfpathlineto{\pgfqpoint{8.014737in}{0.550000in}}%
\pgfpathlineto{\pgfqpoint{8.014737in}{0.589646in}}%
\pgfpathlineto{\pgfqpoint{7.987216in}{0.589646in}}%
\pgfpathlineto{\pgfqpoint{7.987216in}{0.550000in}}%
\pgfpathclose%
\pgfusepath{fill}%
\end{pgfscope}%
\begin{pgfscope}%
\pgfpathrectangle{\pgfqpoint{1.250000in}{0.550000in}}{\pgfqpoint{7.750000in}{3.850000in}}%
\pgfusepath{clip}%
\pgfsetbuttcap%
\pgfsetmiterjoin%
\definecolor{currentfill}{rgb}{0.501961,0.501961,0.501961}%
\pgfsetfillcolor{currentfill}%
\pgfsetfillopacity{0.700000}%
\pgfsetlinewidth{0.000000pt}%
\definecolor{currentstroke}{rgb}{0.000000,0.000000,0.000000}%
\pgfsetstrokecolor{currentstroke}%
\pgfsetstrokeopacity{0.700000}%
\pgfsetdash{}{0pt}%
\pgfpathmoveto{\pgfqpoint{8.014737in}{0.550000in}}%
\pgfpathlineto{\pgfqpoint{8.042259in}{0.550000in}}%
\pgfpathlineto{\pgfqpoint{8.042259in}{0.584139in}}%
\pgfpathlineto{\pgfqpoint{8.014737in}{0.584139in}}%
\pgfpathlineto{\pgfqpoint{8.014737in}{0.550000in}}%
\pgfpathclose%
\pgfusepath{fill}%
\end{pgfscope}%
\begin{pgfscope}%
\pgfpathrectangle{\pgfqpoint{1.250000in}{0.550000in}}{\pgfqpoint{7.750000in}{3.850000in}}%
\pgfusepath{clip}%
\pgfsetbuttcap%
\pgfsetmiterjoin%
\definecolor{currentfill}{rgb}{0.501961,0.501961,0.501961}%
\pgfsetfillcolor{currentfill}%
\pgfsetfillopacity{0.700000}%
\pgfsetlinewidth{0.000000pt}%
\definecolor{currentstroke}{rgb}{0.000000,0.000000,0.000000}%
\pgfsetstrokecolor{currentstroke}%
\pgfsetstrokeopacity{0.700000}%
\pgfsetdash{}{0pt}%
\pgfpathmoveto{\pgfqpoint{8.042259in}{0.550000in}}%
\pgfpathlineto{\pgfqpoint{8.069780in}{0.550000in}}%
\pgfpathlineto{\pgfqpoint{8.069780in}{0.581937in}}%
\pgfpathlineto{\pgfqpoint{8.042259in}{0.581937in}}%
\pgfpathlineto{\pgfqpoint{8.042259in}{0.550000in}}%
\pgfpathclose%
\pgfusepath{fill}%
\end{pgfscope}%
\begin{pgfscope}%
\pgfpathrectangle{\pgfqpoint{1.250000in}{0.550000in}}{\pgfqpoint{7.750000in}{3.850000in}}%
\pgfusepath{clip}%
\pgfsetbuttcap%
\pgfsetmiterjoin%
\definecolor{currentfill}{rgb}{0.501961,0.501961,0.501961}%
\pgfsetfillcolor{currentfill}%
\pgfsetfillopacity{0.700000}%
\pgfsetlinewidth{0.000000pt}%
\definecolor{currentstroke}{rgb}{0.000000,0.000000,0.000000}%
\pgfsetstrokecolor{currentstroke}%
\pgfsetstrokeopacity{0.700000}%
\pgfsetdash{}{0pt}%
\pgfpathmoveto{\pgfqpoint{8.069780in}{0.550000in}}%
\pgfpathlineto{\pgfqpoint{8.097301in}{0.550000in}}%
\pgfpathlineto{\pgfqpoint{8.097301in}{0.585241in}}%
\pgfpathlineto{\pgfqpoint{8.069780in}{0.585241in}}%
\pgfpathlineto{\pgfqpoint{8.069780in}{0.550000in}}%
\pgfpathclose%
\pgfusepath{fill}%
\end{pgfscope}%
\begin{pgfscope}%
\pgfpathrectangle{\pgfqpoint{1.250000in}{0.550000in}}{\pgfqpoint{7.750000in}{3.850000in}}%
\pgfusepath{clip}%
\pgfsetbuttcap%
\pgfsetmiterjoin%
\definecolor{currentfill}{rgb}{0.501961,0.501961,0.501961}%
\pgfsetfillcolor{currentfill}%
\pgfsetfillopacity{0.700000}%
\pgfsetlinewidth{0.000000pt}%
\definecolor{currentstroke}{rgb}{0.000000,0.000000,0.000000}%
\pgfsetstrokecolor{currentstroke}%
\pgfsetstrokeopacity{0.700000}%
\pgfsetdash{}{0pt}%
\pgfpathmoveto{\pgfqpoint{8.097301in}{0.550000in}}%
\pgfpathlineto{\pgfqpoint{8.124822in}{0.550000in}}%
\pgfpathlineto{\pgfqpoint{8.124822in}{0.580835in}}%
\pgfpathlineto{\pgfqpoint{8.097301in}{0.580835in}}%
\pgfpathlineto{\pgfqpoint{8.097301in}{0.550000in}}%
\pgfpathclose%
\pgfusepath{fill}%
\end{pgfscope}%
\begin{pgfscope}%
\pgfpathrectangle{\pgfqpoint{1.250000in}{0.550000in}}{\pgfqpoint{7.750000in}{3.850000in}}%
\pgfusepath{clip}%
\pgfsetbuttcap%
\pgfsetmiterjoin%
\definecolor{currentfill}{rgb}{0.501961,0.501961,0.501961}%
\pgfsetfillcolor{currentfill}%
\pgfsetfillopacity{0.700000}%
\pgfsetlinewidth{0.000000pt}%
\definecolor{currentstroke}{rgb}{0.000000,0.000000,0.000000}%
\pgfsetstrokecolor{currentstroke}%
\pgfsetstrokeopacity{0.700000}%
\pgfsetdash{}{0pt}%
\pgfpathmoveto{\pgfqpoint{8.124822in}{0.550000in}}%
\pgfpathlineto{\pgfqpoint{8.152344in}{0.550000in}}%
\pgfpathlineto{\pgfqpoint{8.152344in}{0.577532in}}%
\pgfpathlineto{\pgfqpoint{8.124822in}{0.577532in}}%
\pgfpathlineto{\pgfqpoint{8.124822in}{0.550000in}}%
\pgfpathclose%
\pgfusepath{fill}%
\end{pgfscope}%
\begin{pgfscope}%
\pgfpathrectangle{\pgfqpoint{1.250000in}{0.550000in}}{\pgfqpoint{7.750000in}{3.850000in}}%
\pgfusepath{clip}%
\pgfsetbuttcap%
\pgfsetmiterjoin%
\definecolor{currentfill}{rgb}{0.501961,0.501961,0.501961}%
\pgfsetfillcolor{currentfill}%
\pgfsetfillopacity{0.700000}%
\pgfsetlinewidth{0.000000pt}%
\definecolor{currentstroke}{rgb}{0.000000,0.000000,0.000000}%
\pgfsetstrokecolor{currentstroke}%
\pgfsetstrokeopacity{0.700000}%
\pgfsetdash{}{0pt}%
\pgfpathmoveto{\pgfqpoint{8.152344in}{0.550000in}}%
\pgfpathlineto{\pgfqpoint{8.179865in}{0.550000in}}%
\pgfpathlineto{\pgfqpoint{8.179865in}{0.583589in}}%
\pgfpathlineto{\pgfqpoint{8.152344in}{0.583589in}}%
\pgfpathlineto{\pgfqpoint{8.152344in}{0.550000in}}%
\pgfpathclose%
\pgfusepath{fill}%
\end{pgfscope}%
\begin{pgfscope}%
\pgfpathrectangle{\pgfqpoint{1.250000in}{0.550000in}}{\pgfqpoint{7.750000in}{3.850000in}}%
\pgfusepath{clip}%
\pgfsetbuttcap%
\pgfsetmiterjoin%
\definecolor{currentfill}{rgb}{0.501961,0.501961,0.501961}%
\pgfsetfillcolor{currentfill}%
\pgfsetfillopacity{0.700000}%
\pgfsetlinewidth{0.000000pt}%
\definecolor{currentstroke}{rgb}{0.000000,0.000000,0.000000}%
\pgfsetstrokecolor{currentstroke}%
\pgfsetstrokeopacity{0.700000}%
\pgfsetdash{}{0pt}%
\pgfpathmoveto{\pgfqpoint{8.179865in}{0.550000in}}%
\pgfpathlineto{\pgfqpoint{8.207386in}{0.550000in}}%
\pgfpathlineto{\pgfqpoint{8.207386in}{0.579734in}}%
\pgfpathlineto{\pgfqpoint{8.179865in}{0.579734in}}%
\pgfpathlineto{\pgfqpoint{8.179865in}{0.550000in}}%
\pgfpathclose%
\pgfusepath{fill}%
\end{pgfscope}%
\begin{pgfscope}%
\pgfpathrectangle{\pgfqpoint{1.250000in}{0.550000in}}{\pgfqpoint{7.750000in}{3.850000in}}%
\pgfusepath{clip}%
\pgfsetbuttcap%
\pgfsetmiterjoin%
\definecolor{currentfill}{rgb}{0.501961,0.501961,0.501961}%
\pgfsetfillcolor{currentfill}%
\pgfsetfillopacity{0.700000}%
\pgfsetlinewidth{0.000000pt}%
\definecolor{currentstroke}{rgb}{0.000000,0.000000,0.000000}%
\pgfsetstrokecolor{currentstroke}%
\pgfsetstrokeopacity{0.700000}%
\pgfsetdash{}{0pt}%
\pgfpathmoveto{\pgfqpoint{8.207386in}{0.550000in}}%
\pgfpathlineto{\pgfqpoint{8.234908in}{0.550000in}}%
\pgfpathlineto{\pgfqpoint{8.234908in}{0.583313in}}%
\pgfpathlineto{\pgfqpoint{8.207386in}{0.583313in}}%
\pgfpathlineto{\pgfqpoint{8.207386in}{0.550000in}}%
\pgfpathclose%
\pgfusepath{fill}%
\end{pgfscope}%
\begin{pgfscope}%
\pgfpathrectangle{\pgfqpoint{1.250000in}{0.550000in}}{\pgfqpoint{7.750000in}{3.850000in}}%
\pgfusepath{clip}%
\pgfsetbuttcap%
\pgfsetmiterjoin%
\definecolor{currentfill}{rgb}{0.501961,0.501961,0.501961}%
\pgfsetfillcolor{currentfill}%
\pgfsetfillopacity{0.700000}%
\pgfsetlinewidth{0.000000pt}%
\definecolor{currentstroke}{rgb}{0.000000,0.000000,0.000000}%
\pgfsetstrokecolor{currentstroke}%
\pgfsetstrokeopacity{0.700000}%
\pgfsetdash{}{0pt}%
\pgfpathmoveto{\pgfqpoint{8.234908in}{0.550000in}}%
\pgfpathlineto{\pgfqpoint{8.262429in}{0.550000in}}%
\pgfpathlineto{\pgfqpoint{8.262429in}{0.580285in}}%
\pgfpathlineto{\pgfqpoint{8.234908in}{0.580285in}}%
\pgfpathlineto{\pgfqpoint{8.234908in}{0.550000in}}%
\pgfpathclose%
\pgfusepath{fill}%
\end{pgfscope}%
\begin{pgfscope}%
\pgfpathrectangle{\pgfqpoint{1.250000in}{0.550000in}}{\pgfqpoint{7.750000in}{3.850000in}}%
\pgfusepath{clip}%
\pgfsetbuttcap%
\pgfsetmiterjoin%
\definecolor{currentfill}{rgb}{0.501961,0.501961,0.501961}%
\pgfsetfillcolor{currentfill}%
\pgfsetfillopacity{0.700000}%
\pgfsetlinewidth{0.000000pt}%
\definecolor{currentstroke}{rgb}{0.000000,0.000000,0.000000}%
\pgfsetstrokecolor{currentstroke}%
\pgfsetstrokeopacity{0.700000}%
\pgfsetdash{}{0pt}%
\pgfpathmoveto{\pgfqpoint{8.262429in}{0.550000in}}%
\pgfpathlineto{\pgfqpoint{8.289950in}{0.550000in}}%
\pgfpathlineto{\pgfqpoint{8.289950in}{0.578082in}}%
\pgfpathlineto{\pgfqpoint{8.262429in}{0.578082in}}%
\pgfpathlineto{\pgfqpoint{8.262429in}{0.550000in}}%
\pgfpathclose%
\pgfusepath{fill}%
\end{pgfscope}%
\begin{pgfscope}%
\pgfpathrectangle{\pgfqpoint{1.250000in}{0.550000in}}{\pgfqpoint{7.750000in}{3.850000in}}%
\pgfusepath{clip}%
\pgfsetbuttcap%
\pgfsetmiterjoin%
\definecolor{currentfill}{rgb}{0.501961,0.501961,0.501961}%
\pgfsetfillcolor{currentfill}%
\pgfsetfillopacity{0.700000}%
\pgfsetlinewidth{0.000000pt}%
\definecolor{currentstroke}{rgb}{0.000000,0.000000,0.000000}%
\pgfsetstrokecolor{currentstroke}%
\pgfsetstrokeopacity{0.700000}%
\pgfsetdash{}{0pt}%
\pgfpathmoveto{\pgfqpoint{8.289950in}{0.550000in}}%
\pgfpathlineto{\pgfqpoint{8.317472in}{0.550000in}}%
\pgfpathlineto{\pgfqpoint{8.317472in}{0.584965in}}%
\pgfpathlineto{\pgfqpoint{8.289950in}{0.584965in}}%
\pgfpathlineto{\pgfqpoint{8.289950in}{0.550000in}}%
\pgfpathclose%
\pgfusepath{fill}%
\end{pgfscope}%
\begin{pgfscope}%
\pgfpathrectangle{\pgfqpoint{1.250000in}{0.550000in}}{\pgfqpoint{7.750000in}{3.850000in}}%
\pgfusepath{clip}%
\pgfsetbuttcap%
\pgfsetmiterjoin%
\definecolor{currentfill}{rgb}{0.501961,0.501961,0.501961}%
\pgfsetfillcolor{currentfill}%
\pgfsetfillopacity{0.700000}%
\pgfsetlinewidth{0.000000pt}%
\definecolor{currentstroke}{rgb}{0.000000,0.000000,0.000000}%
\pgfsetstrokecolor{currentstroke}%
\pgfsetstrokeopacity{0.700000}%
\pgfsetdash{}{0pt}%
\pgfpathmoveto{\pgfqpoint{8.317472in}{0.550000in}}%
\pgfpathlineto{\pgfqpoint{8.344993in}{0.550000in}}%
\pgfpathlineto{\pgfqpoint{8.344993in}{0.577532in}}%
\pgfpathlineto{\pgfqpoint{8.317472in}{0.577532in}}%
\pgfpathlineto{\pgfqpoint{8.317472in}{0.550000in}}%
\pgfpathclose%
\pgfusepath{fill}%
\end{pgfscope}%
\begin{pgfscope}%
\pgfpathrectangle{\pgfqpoint{1.250000in}{0.550000in}}{\pgfqpoint{7.750000in}{3.850000in}}%
\pgfusepath{clip}%
\pgfsetbuttcap%
\pgfsetmiterjoin%
\definecolor{currentfill}{rgb}{0.501961,0.501961,0.501961}%
\pgfsetfillcolor{currentfill}%
\pgfsetfillopacity{0.700000}%
\pgfsetlinewidth{0.000000pt}%
\definecolor{currentstroke}{rgb}{0.000000,0.000000,0.000000}%
\pgfsetstrokecolor{currentstroke}%
\pgfsetstrokeopacity{0.700000}%
\pgfsetdash{}{0pt}%
\pgfpathmoveto{\pgfqpoint{8.344993in}{0.550000in}}%
\pgfpathlineto{\pgfqpoint{8.372514in}{0.550000in}}%
\pgfpathlineto{\pgfqpoint{8.372514in}{0.578082in}}%
\pgfpathlineto{\pgfqpoint{8.344993in}{0.578082in}}%
\pgfpathlineto{\pgfqpoint{8.344993in}{0.550000in}}%
\pgfpathclose%
\pgfusepath{fill}%
\end{pgfscope}%
\begin{pgfscope}%
\pgfpathrectangle{\pgfqpoint{1.250000in}{0.550000in}}{\pgfqpoint{7.750000in}{3.850000in}}%
\pgfusepath{clip}%
\pgfsetbuttcap%
\pgfsetmiterjoin%
\definecolor{currentfill}{rgb}{0.501961,0.501961,0.501961}%
\pgfsetfillcolor{currentfill}%
\pgfsetfillopacity{0.700000}%
\pgfsetlinewidth{0.000000pt}%
\definecolor{currentstroke}{rgb}{0.000000,0.000000,0.000000}%
\pgfsetstrokecolor{currentstroke}%
\pgfsetstrokeopacity{0.700000}%
\pgfsetdash{}{0pt}%
\pgfpathmoveto{\pgfqpoint{8.372514in}{0.550000in}}%
\pgfpathlineto{\pgfqpoint{8.400036in}{0.550000in}}%
\pgfpathlineto{\pgfqpoint{8.400036in}{0.581937in}}%
\pgfpathlineto{\pgfqpoint{8.372514in}{0.581937in}}%
\pgfpathlineto{\pgfqpoint{8.372514in}{0.550000in}}%
\pgfpathclose%
\pgfusepath{fill}%
\end{pgfscope}%
\begin{pgfscope}%
\pgfpathrectangle{\pgfqpoint{1.250000in}{0.550000in}}{\pgfqpoint{7.750000in}{3.850000in}}%
\pgfusepath{clip}%
\pgfsetbuttcap%
\pgfsetmiterjoin%
\definecolor{currentfill}{rgb}{0.501961,0.501961,0.501961}%
\pgfsetfillcolor{currentfill}%
\pgfsetfillopacity{0.700000}%
\pgfsetlinewidth{0.000000pt}%
\definecolor{currentstroke}{rgb}{0.000000,0.000000,0.000000}%
\pgfsetstrokecolor{currentstroke}%
\pgfsetstrokeopacity{0.700000}%
\pgfsetdash{}{0pt}%
\pgfpathmoveto{\pgfqpoint{8.400036in}{0.550000in}}%
\pgfpathlineto{\pgfqpoint{8.427557in}{0.550000in}}%
\pgfpathlineto{\pgfqpoint{8.427557in}{0.585791in}}%
\pgfpathlineto{\pgfqpoint{8.400036in}{0.585791in}}%
\pgfpathlineto{\pgfqpoint{8.400036in}{0.550000in}}%
\pgfpathclose%
\pgfusepath{fill}%
\end{pgfscope}%
\begin{pgfscope}%
\pgfpathrectangle{\pgfqpoint{1.250000in}{0.550000in}}{\pgfqpoint{7.750000in}{3.850000in}}%
\pgfusepath{clip}%
\pgfsetbuttcap%
\pgfsetmiterjoin%
\definecolor{currentfill}{rgb}{0.501961,0.501961,0.501961}%
\pgfsetfillcolor{currentfill}%
\pgfsetfillopacity{0.700000}%
\pgfsetlinewidth{0.000000pt}%
\definecolor{currentstroke}{rgb}{0.000000,0.000000,0.000000}%
\pgfsetstrokecolor{currentstroke}%
\pgfsetstrokeopacity{0.700000}%
\pgfsetdash{}{0pt}%
\pgfpathmoveto{\pgfqpoint{8.427557in}{0.550000in}}%
\pgfpathlineto{\pgfqpoint{8.455078in}{0.550000in}}%
\pgfpathlineto{\pgfqpoint{8.455078in}{0.577807in}}%
\pgfpathlineto{\pgfqpoint{8.427557in}{0.577807in}}%
\pgfpathlineto{\pgfqpoint{8.427557in}{0.550000in}}%
\pgfpathclose%
\pgfusepath{fill}%
\end{pgfscope}%
\begin{pgfscope}%
\pgfpathrectangle{\pgfqpoint{1.250000in}{0.550000in}}{\pgfqpoint{7.750000in}{3.850000in}}%
\pgfusepath{clip}%
\pgfsetbuttcap%
\pgfsetmiterjoin%
\definecolor{currentfill}{rgb}{0.501961,0.501961,0.501961}%
\pgfsetfillcolor{currentfill}%
\pgfsetfillopacity{0.700000}%
\pgfsetlinewidth{0.000000pt}%
\definecolor{currentstroke}{rgb}{0.000000,0.000000,0.000000}%
\pgfsetstrokecolor{currentstroke}%
\pgfsetstrokeopacity{0.700000}%
\pgfsetdash{}{0pt}%
\pgfpathmoveto{\pgfqpoint{8.455078in}{0.550000in}}%
\pgfpathlineto{\pgfqpoint{8.482599in}{0.550000in}}%
\pgfpathlineto{\pgfqpoint{8.482599in}{0.585791in}}%
\pgfpathlineto{\pgfqpoint{8.455078in}{0.585791in}}%
\pgfpathlineto{\pgfqpoint{8.455078in}{0.550000in}}%
\pgfpathclose%
\pgfusepath{fill}%
\end{pgfscope}%
\begin{pgfscope}%
\pgfpathrectangle{\pgfqpoint{1.250000in}{0.550000in}}{\pgfqpoint{7.750000in}{3.850000in}}%
\pgfusepath{clip}%
\pgfsetbuttcap%
\pgfsetmiterjoin%
\definecolor{currentfill}{rgb}{0.501961,0.501961,0.501961}%
\pgfsetfillcolor{currentfill}%
\pgfsetfillopacity{0.700000}%
\pgfsetlinewidth{0.000000pt}%
\definecolor{currentstroke}{rgb}{0.000000,0.000000,0.000000}%
\pgfsetstrokecolor{currentstroke}%
\pgfsetstrokeopacity{0.700000}%
\pgfsetdash{}{0pt}%
\pgfpathmoveto{\pgfqpoint{8.482599in}{0.550000in}}%
\pgfpathlineto{\pgfqpoint{8.510121in}{0.550000in}}%
\pgfpathlineto{\pgfqpoint{8.510121in}{0.582487in}}%
\pgfpathlineto{\pgfqpoint{8.482599in}{0.582487in}}%
\pgfpathlineto{\pgfqpoint{8.482599in}{0.550000in}}%
\pgfpathclose%
\pgfusepath{fill}%
\end{pgfscope}%
\begin{pgfscope}%
\pgfpathrectangle{\pgfqpoint{1.250000in}{0.550000in}}{\pgfqpoint{7.750000in}{3.850000in}}%
\pgfusepath{clip}%
\pgfsetbuttcap%
\pgfsetmiterjoin%
\definecolor{currentfill}{rgb}{0.501961,0.501961,0.501961}%
\pgfsetfillcolor{currentfill}%
\pgfsetfillopacity{0.700000}%
\pgfsetlinewidth{0.000000pt}%
\definecolor{currentstroke}{rgb}{0.000000,0.000000,0.000000}%
\pgfsetstrokecolor{currentstroke}%
\pgfsetstrokeopacity{0.700000}%
\pgfsetdash{}{0pt}%
\pgfpathmoveto{\pgfqpoint{8.510121in}{0.550000in}}%
\pgfpathlineto{\pgfqpoint{8.537642in}{0.550000in}}%
\pgfpathlineto{\pgfqpoint{8.537642in}{0.587994in}}%
\pgfpathlineto{\pgfqpoint{8.510121in}{0.587994in}}%
\pgfpathlineto{\pgfqpoint{8.510121in}{0.550000in}}%
\pgfpathclose%
\pgfusepath{fill}%
\end{pgfscope}%
\begin{pgfscope}%
\pgfpathrectangle{\pgfqpoint{1.250000in}{0.550000in}}{\pgfqpoint{7.750000in}{3.850000in}}%
\pgfusepath{clip}%
\pgfsetbuttcap%
\pgfsetmiterjoin%
\definecolor{currentfill}{rgb}{0.501961,0.501961,0.501961}%
\pgfsetfillcolor{currentfill}%
\pgfsetfillopacity{0.700000}%
\pgfsetlinewidth{0.000000pt}%
\definecolor{currentstroke}{rgb}{0.000000,0.000000,0.000000}%
\pgfsetstrokecolor{currentstroke}%
\pgfsetstrokeopacity{0.700000}%
\pgfsetdash{}{0pt}%
\pgfpathmoveto{\pgfqpoint{8.537642in}{0.550000in}}%
\pgfpathlineto{\pgfqpoint{8.565163in}{0.550000in}}%
\pgfpathlineto{\pgfqpoint{8.565163in}{0.584139in}}%
\pgfpathlineto{\pgfqpoint{8.537642in}{0.584139in}}%
\pgfpathlineto{\pgfqpoint{8.537642in}{0.550000in}}%
\pgfpathclose%
\pgfusepath{fill}%
\end{pgfscope}%
\begin{pgfscope}%
\pgfpathrectangle{\pgfqpoint{1.250000in}{0.550000in}}{\pgfqpoint{7.750000in}{3.850000in}}%
\pgfusepath{clip}%
\pgfsetbuttcap%
\pgfsetmiterjoin%
\definecolor{currentfill}{rgb}{0.501961,0.501961,0.501961}%
\pgfsetfillcolor{currentfill}%
\pgfsetfillopacity{0.700000}%
\pgfsetlinewidth{0.000000pt}%
\definecolor{currentstroke}{rgb}{0.000000,0.000000,0.000000}%
\pgfsetstrokecolor{currentstroke}%
\pgfsetstrokeopacity{0.700000}%
\pgfsetdash{}{0pt}%
\pgfpathmoveto{\pgfqpoint{8.565163in}{0.550000in}}%
\pgfpathlineto{\pgfqpoint{8.592685in}{0.550000in}}%
\pgfpathlineto{\pgfqpoint{8.592685in}{0.575604in}}%
\pgfpathlineto{\pgfqpoint{8.565163in}{0.575604in}}%
\pgfpathlineto{\pgfqpoint{8.565163in}{0.550000in}}%
\pgfpathclose%
\pgfusepath{fill}%
\end{pgfscope}%
\begin{pgfscope}%
\pgfpathrectangle{\pgfqpoint{1.250000in}{0.550000in}}{\pgfqpoint{7.750000in}{3.850000in}}%
\pgfusepath{clip}%
\pgfsetbuttcap%
\pgfsetmiterjoin%
\definecolor{currentfill}{rgb}{0.501961,0.501961,0.501961}%
\pgfsetfillcolor{currentfill}%
\pgfsetfillopacity{0.700000}%
\pgfsetlinewidth{0.000000pt}%
\definecolor{currentstroke}{rgb}{0.000000,0.000000,0.000000}%
\pgfsetstrokecolor{currentstroke}%
\pgfsetstrokeopacity{0.700000}%
\pgfsetdash{}{0pt}%
\pgfpathmoveto{\pgfqpoint{8.592685in}{0.550000in}}%
\pgfpathlineto{\pgfqpoint{8.620206in}{0.550000in}}%
\pgfpathlineto{\pgfqpoint{8.620206in}{0.563215in}}%
\pgfpathlineto{\pgfqpoint{8.592685in}{0.563215in}}%
\pgfpathlineto{\pgfqpoint{8.592685in}{0.550000in}}%
\pgfpathclose%
\pgfusepath{fill}%
\end{pgfscope}%
\begin{pgfscope}%
\pgfpathrectangle{\pgfqpoint{1.250000in}{0.550000in}}{\pgfqpoint{7.750000in}{3.850000in}}%
\pgfusepath{clip}%
\pgfsetbuttcap%
\pgfsetmiterjoin%
\definecolor{currentfill}{rgb}{0.501961,0.501961,0.501961}%
\pgfsetfillcolor{currentfill}%
\pgfsetfillopacity{0.700000}%
\pgfsetlinewidth{0.000000pt}%
\definecolor{currentstroke}{rgb}{0.000000,0.000000,0.000000}%
\pgfsetstrokecolor{currentstroke}%
\pgfsetstrokeopacity{0.700000}%
\pgfsetdash{}{0pt}%
\pgfpathmoveto{\pgfqpoint{8.620206in}{0.550000in}}%
\pgfpathlineto{\pgfqpoint{8.647727in}{0.550000in}}%
\pgfpathlineto{\pgfqpoint{8.647727in}{0.559361in}}%
\pgfpathlineto{\pgfqpoint{8.620206in}{0.559361in}}%
\pgfpathlineto{\pgfqpoint{8.620206in}{0.550000in}}%
\pgfpathclose%
\pgfusepath{fill}%
\end{pgfscope}%
\begin{pgfscope}%
\pgfsetbuttcap%
\pgfsetroundjoin%
\definecolor{currentfill}{rgb}{0.000000,0.000000,0.000000}%
\pgfsetfillcolor{currentfill}%
\pgfsetlinewidth{0.803000pt}%
\definecolor{currentstroke}{rgb}{0.000000,0.000000,0.000000}%
\pgfsetstrokecolor{currentstroke}%
\pgfsetdash{}{0pt}%
\pgfsys@defobject{currentmarker}{\pgfqpoint{0.000000in}{-0.048611in}}{\pgfqpoint{0.000000in}{0.000000in}}{%
\pgfpathmoveto{\pgfqpoint{0.000000in}{0.000000in}}%
\pgfpathlineto{\pgfqpoint{0.000000in}{-0.048611in}}%
\pgfusepath{stroke,fill}%
}%
\begin{pgfscope}%
\pgfsys@transformshift{1.602273in}{0.550000in}%
\pgfsys@useobject{currentmarker}{}%
\end{pgfscope}%
\end{pgfscope}%
\begin{pgfscope}%
\definecolor{textcolor}{rgb}{0.000000,0.000000,0.000000}%
\pgfsetstrokecolor{textcolor}%
\pgfsetfillcolor{textcolor}%
\pgftext[x=1.602273in,y=0.452778in,,top]{\color{textcolor}{\sffamily\fontsize{10.000000}{12.000000}\selectfont\catcode`\^=\active\def^{\ifmmode\sp\else\^{}\fi}\catcode`\%=\active\def%{\%}0}}%
\end{pgfscope}%
\begin{pgfscope}%
\pgfsetbuttcap%
\pgfsetroundjoin%
\definecolor{currentfill}{rgb}{0.000000,0.000000,0.000000}%
\pgfsetfillcolor{currentfill}%
\pgfsetlinewidth{0.803000pt}%
\definecolor{currentstroke}{rgb}{0.000000,0.000000,0.000000}%
\pgfsetstrokecolor{currentstroke}%
\pgfsetdash{}{0pt}%
\pgfsys@defobject{currentmarker}{\pgfqpoint{0.000000in}{-0.048611in}}{\pgfqpoint{0.000000in}{0.000000in}}{%
\pgfpathmoveto{\pgfqpoint{0.000000in}{0.000000in}}%
\pgfpathlineto{\pgfqpoint{0.000000in}{-0.048611in}}%
\pgfusepath{stroke,fill}%
}%
\begin{pgfscope}%
\pgfsys@transformshift{2.978338in}{0.550000in}%
\pgfsys@useobject{currentmarker}{}%
\end{pgfscope}%
\end{pgfscope}%
\begin{pgfscope}%
\definecolor{textcolor}{rgb}{0.000000,0.000000,0.000000}%
\pgfsetstrokecolor{textcolor}%
\pgfsetfillcolor{textcolor}%
\pgftext[x=2.978338in,y=0.452778in,,top]{\color{textcolor}{\sffamily\fontsize{10.000000}{12.000000}\selectfont\catcode`\^=\active\def^{\ifmmode\sp\else\^{}\fi}\catcode`\%=\active\def%{\%}50}}%
\end{pgfscope}%
\begin{pgfscope}%
\pgfsetbuttcap%
\pgfsetroundjoin%
\definecolor{currentfill}{rgb}{0.000000,0.000000,0.000000}%
\pgfsetfillcolor{currentfill}%
\pgfsetlinewidth{0.803000pt}%
\definecolor{currentstroke}{rgb}{0.000000,0.000000,0.000000}%
\pgfsetstrokecolor{currentstroke}%
\pgfsetdash{}{0pt}%
\pgfsys@defobject{currentmarker}{\pgfqpoint{0.000000in}{-0.048611in}}{\pgfqpoint{0.000000in}{0.000000in}}{%
\pgfpathmoveto{\pgfqpoint{0.000000in}{0.000000in}}%
\pgfpathlineto{\pgfqpoint{0.000000in}{-0.048611in}}%
\pgfusepath{stroke,fill}%
}%
\begin{pgfscope}%
\pgfsys@transformshift{4.354403in}{0.550000in}%
\pgfsys@useobject{currentmarker}{}%
\end{pgfscope}%
\end{pgfscope}%
\begin{pgfscope}%
\definecolor{textcolor}{rgb}{0.000000,0.000000,0.000000}%
\pgfsetstrokecolor{textcolor}%
\pgfsetfillcolor{textcolor}%
\pgftext[x=4.354403in,y=0.452778in,,top]{\color{textcolor}{\sffamily\fontsize{10.000000}{12.000000}\selectfont\catcode`\^=\active\def^{\ifmmode\sp\else\^{}\fi}\catcode`\%=\active\def%{\%}100}}%
\end{pgfscope}%
\begin{pgfscope}%
\pgfsetbuttcap%
\pgfsetroundjoin%
\definecolor{currentfill}{rgb}{0.000000,0.000000,0.000000}%
\pgfsetfillcolor{currentfill}%
\pgfsetlinewidth{0.803000pt}%
\definecolor{currentstroke}{rgb}{0.000000,0.000000,0.000000}%
\pgfsetstrokecolor{currentstroke}%
\pgfsetdash{}{0pt}%
\pgfsys@defobject{currentmarker}{\pgfqpoint{0.000000in}{-0.048611in}}{\pgfqpoint{0.000000in}{0.000000in}}{%
\pgfpathmoveto{\pgfqpoint{0.000000in}{0.000000in}}%
\pgfpathlineto{\pgfqpoint{0.000000in}{-0.048611in}}%
\pgfusepath{stroke,fill}%
}%
\begin{pgfscope}%
\pgfsys@transformshift{5.730469in}{0.550000in}%
\pgfsys@useobject{currentmarker}{}%
\end{pgfscope}%
\end{pgfscope}%
\begin{pgfscope}%
\definecolor{textcolor}{rgb}{0.000000,0.000000,0.000000}%
\pgfsetstrokecolor{textcolor}%
\pgfsetfillcolor{textcolor}%
\pgftext[x=5.730469in,y=0.452778in,,top]{\color{textcolor}{\sffamily\fontsize{10.000000}{12.000000}\selectfont\catcode`\^=\active\def^{\ifmmode\sp\else\^{}\fi}\catcode`\%=\active\def%{\%}150}}%
\end{pgfscope}%
\begin{pgfscope}%
\pgfsetbuttcap%
\pgfsetroundjoin%
\definecolor{currentfill}{rgb}{0.000000,0.000000,0.000000}%
\pgfsetfillcolor{currentfill}%
\pgfsetlinewidth{0.803000pt}%
\definecolor{currentstroke}{rgb}{0.000000,0.000000,0.000000}%
\pgfsetstrokecolor{currentstroke}%
\pgfsetdash{}{0pt}%
\pgfsys@defobject{currentmarker}{\pgfqpoint{0.000000in}{-0.048611in}}{\pgfqpoint{0.000000in}{0.000000in}}{%
\pgfpathmoveto{\pgfqpoint{0.000000in}{0.000000in}}%
\pgfpathlineto{\pgfqpoint{0.000000in}{-0.048611in}}%
\pgfusepath{stroke,fill}%
}%
\begin{pgfscope}%
\pgfsys@transformshift{7.106534in}{0.550000in}%
\pgfsys@useobject{currentmarker}{}%
\end{pgfscope}%
\end{pgfscope}%
\begin{pgfscope}%
\definecolor{textcolor}{rgb}{0.000000,0.000000,0.000000}%
\pgfsetstrokecolor{textcolor}%
\pgfsetfillcolor{textcolor}%
\pgftext[x=7.106534in,y=0.452778in,,top]{\color{textcolor}{\sffamily\fontsize{10.000000}{12.000000}\selectfont\catcode`\^=\active\def^{\ifmmode\sp\else\^{}\fi}\catcode`\%=\active\def%{\%}200}}%
\end{pgfscope}%
\begin{pgfscope}%
\pgfsetbuttcap%
\pgfsetroundjoin%
\definecolor{currentfill}{rgb}{0.000000,0.000000,0.000000}%
\pgfsetfillcolor{currentfill}%
\pgfsetlinewidth{0.803000pt}%
\definecolor{currentstroke}{rgb}{0.000000,0.000000,0.000000}%
\pgfsetstrokecolor{currentstroke}%
\pgfsetdash{}{0pt}%
\pgfsys@defobject{currentmarker}{\pgfqpoint{0.000000in}{-0.048611in}}{\pgfqpoint{0.000000in}{0.000000in}}{%
\pgfpathmoveto{\pgfqpoint{0.000000in}{0.000000in}}%
\pgfpathlineto{\pgfqpoint{0.000000in}{-0.048611in}}%
\pgfusepath{stroke,fill}%
}%
\begin{pgfscope}%
\pgfsys@transformshift{8.482599in}{0.550000in}%
\pgfsys@useobject{currentmarker}{}%
\end{pgfscope}%
\end{pgfscope}%
\begin{pgfscope}%
\definecolor{textcolor}{rgb}{0.000000,0.000000,0.000000}%
\pgfsetstrokecolor{textcolor}%
\pgfsetfillcolor{textcolor}%
\pgftext[x=8.482599in,y=0.452778in,,top]{\color{textcolor}{\sffamily\fontsize{10.000000}{12.000000}\selectfont\catcode`\^=\active\def^{\ifmmode\sp\else\^{}\fi}\catcode`\%=\active\def%{\%}250}}%
\end{pgfscope}%
\begin{pgfscope}%
\definecolor{textcolor}{rgb}{0.000000,0.000000,0.000000}%
\pgfsetstrokecolor{textcolor}%
\pgfsetfillcolor{textcolor}%
\pgftext[x=5.125000in,y=0.262809in,,top]{\color{textcolor}{\sffamily\fontsize{10.000000}{12.000000}\selectfont\catcode`\^=\active\def^{\ifmmode\sp\else\^{}\fi}\catcode`\%=\active\def%{\%}Intensidad}}%
\end{pgfscope}%
\begin{pgfscope}%
\pgfsetbuttcap%
\pgfsetroundjoin%
\definecolor{currentfill}{rgb}{0.000000,0.000000,0.000000}%
\pgfsetfillcolor{currentfill}%
\pgfsetlinewidth{0.803000pt}%
\definecolor{currentstroke}{rgb}{0.000000,0.000000,0.000000}%
\pgfsetstrokecolor{currentstroke}%
\pgfsetdash{}{0pt}%
\pgfsys@defobject{currentmarker}{\pgfqpoint{-0.048611in}{0.000000in}}{\pgfqpoint{-0.000000in}{0.000000in}}{%
\pgfpathmoveto{\pgfqpoint{-0.000000in}{0.000000in}}%
\pgfpathlineto{\pgfqpoint{-0.048611in}{0.000000in}}%
\pgfusepath{stroke,fill}%
}%
\begin{pgfscope}%
\pgfsys@transformshift{1.250000in}{0.550000in}%
\pgfsys@useobject{currentmarker}{}%
\end{pgfscope}%
\end{pgfscope}%
\begin{pgfscope}%
\definecolor{textcolor}{rgb}{0.000000,0.000000,0.000000}%
\pgfsetstrokecolor{textcolor}%
\pgfsetfillcolor{textcolor}%
\pgftext[x=1.064412in, y=0.497238in, left, base]{\color{textcolor}{\sffamily\fontsize{10.000000}{12.000000}\selectfont\catcode`\^=\active\def^{\ifmmode\sp\else\^{}\fi}\catcode`\%=\active\def%{\%}0}}%
\end{pgfscope}%
\begin{pgfscope}%
\pgfsetbuttcap%
\pgfsetroundjoin%
\definecolor{currentfill}{rgb}{0.000000,0.000000,0.000000}%
\pgfsetfillcolor{currentfill}%
\pgfsetlinewidth{0.803000pt}%
\definecolor{currentstroke}{rgb}{0.000000,0.000000,0.000000}%
\pgfsetstrokecolor{currentstroke}%
\pgfsetdash{}{0pt}%
\pgfsys@defobject{currentmarker}{\pgfqpoint{-0.048611in}{0.000000in}}{\pgfqpoint{-0.000000in}{0.000000in}}{%
\pgfpathmoveto{\pgfqpoint{-0.000000in}{0.000000in}}%
\pgfpathlineto{\pgfqpoint{-0.048611in}{0.000000in}}%
\pgfusepath{stroke,fill}%
}%
\begin{pgfscope}%
\pgfsys@transformshift{1.250000in}{1.100633in}%
\pgfsys@useobject{currentmarker}{}%
\end{pgfscope}%
\end{pgfscope}%
\begin{pgfscope}%
\definecolor{textcolor}{rgb}{0.000000,0.000000,0.000000}%
\pgfsetstrokecolor{textcolor}%
\pgfsetfillcolor{textcolor}%
\pgftext[x=0.799316in, y=1.047872in, left, base]{\color{textcolor}{\sffamily\fontsize{10.000000}{12.000000}\selectfont\catcode`\^=\active\def^{\ifmmode\sp\else\^{}\fi}\catcode`\%=\active\def%{\%}2000}}%
\end{pgfscope}%
\begin{pgfscope}%
\pgfsetbuttcap%
\pgfsetroundjoin%
\definecolor{currentfill}{rgb}{0.000000,0.000000,0.000000}%
\pgfsetfillcolor{currentfill}%
\pgfsetlinewidth{0.803000pt}%
\definecolor{currentstroke}{rgb}{0.000000,0.000000,0.000000}%
\pgfsetstrokecolor{currentstroke}%
\pgfsetdash{}{0pt}%
\pgfsys@defobject{currentmarker}{\pgfqpoint{-0.048611in}{0.000000in}}{\pgfqpoint{-0.000000in}{0.000000in}}{%
\pgfpathmoveto{\pgfqpoint{-0.000000in}{0.000000in}}%
\pgfpathlineto{\pgfqpoint{-0.048611in}{0.000000in}}%
\pgfusepath{stroke,fill}%
}%
\begin{pgfscope}%
\pgfsys@transformshift{1.250000in}{1.651266in}%
\pgfsys@useobject{currentmarker}{}%
\end{pgfscope}%
\end{pgfscope}%
\begin{pgfscope}%
\definecolor{textcolor}{rgb}{0.000000,0.000000,0.000000}%
\pgfsetstrokecolor{textcolor}%
\pgfsetfillcolor{textcolor}%
\pgftext[x=0.799316in, y=1.598505in, left, base]{\color{textcolor}{\sffamily\fontsize{10.000000}{12.000000}\selectfont\catcode`\^=\active\def^{\ifmmode\sp\else\^{}\fi}\catcode`\%=\active\def%{\%}4000}}%
\end{pgfscope}%
\begin{pgfscope}%
\pgfsetbuttcap%
\pgfsetroundjoin%
\definecolor{currentfill}{rgb}{0.000000,0.000000,0.000000}%
\pgfsetfillcolor{currentfill}%
\pgfsetlinewidth{0.803000pt}%
\definecolor{currentstroke}{rgb}{0.000000,0.000000,0.000000}%
\pgfsetstrokecolor{currentstroke}%
\pgfsetdash{}{0pt}%
\pgfsys@defobject{currentmarker}{\pgfqpoint{-0.048611in}{0.000000in}}{\pgfqpoint{-0.000000in}{0.000000in}}{%
\pgfpathmoveto{\pgfqpoint{-0.000000in}{0.000000in}}%
\pgfpathlineto{\pgfqpoint{-0.048611in}{0.000000in}}%
\pgfusepath{stroke,fill}%
}%
\begin{pgfscope}%
\pgfsys@transformshift{1.250000in}{2.201900in}%
\pgfsys@useobject{currentmarker}{}%
\end{pgfscope}%
\end{pgfscope}%
\begin{pgfscope}%
\definecolor{textcolor}{rgb}{0.000000,0.000000,0.000000}%
\pgfsetstrokecolor{textcolor}%
\pgfsetfillcolor{textcolor}%
\pgftext[x=0.799316in, y=2.149138in, left, base]{\color{textcolor}{\sffamily\fontsize{10.000000}{12.000000}\selectfont\catcode`\^=\active\def^{\ifmmode\sp\else\^{}\fi}\catcode`\%=\active\def%{\%}6000}}%
\end{pgfscope}%
\begin{pgfscope}%
\pgfsetbuttcap%
\pgfsetroundjoin%
\definecolor{currentfill}{rgb}{0.000000,0.000000,0.000000}%
\pgfsetfillcolor{currentfill}%
\pgfsetlinewidth{0.803000pt}%
\definecolor{currentstroke}{rgb}{0.000000,0.000000,0.000000}%
\pgfsetstrokecolor{currentstroke}%
\pgfsetdash{}{0pt}%
\pgfsys@defobject{currentmarker}{\pgfqpoint{-0.048611in}{0.000000in}}{\pgfqpoint{-0.000000in}{0.000000in}}{%
\pgfpathmoveto{\pgfqpoint{-0.000000in}{0.000000in}}%
\pgfpathlineto{\pgfqpoint{-0.048611in}{0.000000in}}%
\pgfusepath{stroke,fill}%
}%
\begin{pgfscope}%
\pgfsys@transformshift{1.250000in}{2.752533in}%
\pgfsys@useobject{currentmarker}{}%
\end{pgfscope}%
\end{pgfscope}%
\begin{pgfscope}%
\definecolor{textcolor}{rgb}{0.000000,0.000000,0.000000}%
\pgfsetstrokecolor{textcolor}%
\pgfsetfillcolor{textcolor}%
\pgftext[x=0.799316in, y=2.699771in, left, base]{\color{textcolor}{\sffamily\fontsize{10.000000}{12.000000}\selectfont\catcode`\^=\active\def^{\ifmmode\sp\else\^{}\fi}\catcode`\%=\active\def%{\%}8000}}%
\end{pgfscope}%
\begin{pgfscope}%
\pgfsetbuttcap%
\pgfsetroundjoin%
\definecolor{currentfill}{rgb}{0.000000,0.000000,0.000000}%
\pgfsetfillcolor{currentfill}%
\pgfsetlinewidth{0.803000pt}%
\definecolor{currentstroke}{rgb}{0.000000,0.000000,0.000000}%
\pgfsetstrokecolor{currentstroke}%
\pgfsetdash{}{0pt}%
\pgfsys@defobject{currentmarker}{\pgfqpoint{-0.048611in}{0.000000in}}{\pgfqpoint{-0.000000in}{0.000000in}}{%
\pgfpathmoveto{\pgfqpoint{-0.000000in}{0.000000in}}%
\pgfpathlineto{\pgfqpoint{-0.048611in}{0.000000in}}%
\pgfusepath{stroke,fill}%
}%
\begin{pgfscope}%
\pgfsys@transformshift{1.250000in}{3.303166in}%
\pgfsys@useobject{currentmarker}{}%
\end{pgfscope}%
\end{pgfscope}%
\begin{pgfscope}%
\definecolor{textcolor}{rgb}{0.000000,0.000000,0.000000}%
\pgfsetstrokecolor{textcolor}%
\pgfsetfillcolor{textcolor}%
\pgftext[x=0.710951in, y=3.250405in, left, base]{\color{textcolor}{\sffamily\fontsize{10.000000}{12.000000}\selectfont\catcode`\^=\active\def^{\ifmmode\sp\else\^{}\fi}\catcode`\%=\active\def%{\%}10000}}%
\end{pgfscope}%
\begin{pgfscope}%
\pgfsetbuttcap%
\pgfsetroundjoin%
\definecolor{currentfill}{rgb}{0.000000,0.000000,0.000000}%
\pgfsetfillcolor{currentfill}%
\pgfsetlinewidth{0.803000pt}%
\definecolor{currentstroke}{rgb}{0.000000,0.000000,0.000000}%
\pgfsetstrokecolor{currentstroke}%
\pgfsetdash{}{0pt}%
\pgfsys@defobject{currentmarker}{\pgfqpoint{-0.048611in}{0.000000in}}{\pgfqpoint{-0.000000in}{0.000000in}}{%
\pgfpathmoveto{\pgfqpoint{-0.000000in}{0.000000in}}%
\pgfpathlineto{\pgfqpoint{-0.048611in}{0.000000in}}%
\pgfusepath{stroke,fill}%
}%
\begin{pgfscope}%
\pgfsys@transformshift{1.250000in}{3.853799in}%
\pgfsys@useobject{currentmarker}{}%
\end{pgfscope}%
\end{pgfscope}%
\begin{pgfscope}%
\definecolor{textcolor}{rgb}{0.000000,0.000000,0.000000}%
\pgfsetstrokecolor{textcolor}%
\pgfsetfillcolor{textcolor}%
\pgftext[x=0.710951in, y=3.801038in, left, base]{\color{textcolor}{\sffamily\fontsize{10.000000}{12.000000}\selectfont\catcode`\^=\active\def^{\ifmmode\sp\else\^{}\fi}\catcode`\%=\active\def%{\%}12000}}%
\end{pgfscope}%
\begin{pgfscope}%
\definecolor{textcolor}{rgb}{0.000000,0.000000,0.000000}%
\pgfsetstrokecolor{textcolor}%
\pgfsetfillcolor{textcolor}%
\pgftext[x=0.655396in,y=2.475000in,,bottom,rotate=90.000000]{\color{textcolor}{\sffamily\fontsize{10.000000}{12.000000}\selectfont\catcode`\^=\active\def^{\ifmmode\sp\else\^{}\fi}\catcode`\%=\active\def%{\%}Frecuencia}}%
\end{pgfscope}%
\begin{pgfscope}%
\pgfsetrectcap%
\pgfsetmiterjoin%
\pgfsetlinewidth{0.803000pt}%
\definecolor{currentstroke}{rgb}{0.000000,0.000000,0.000000}%
\pgfsetstrokecolor{currentstroke}%
\pgfsetdash{}{0pt}%
\pgfpathmoveto{\pgfqpoint{1.250000in}{0.550000in}}%
\pgfpathlineto{\pgfqpoint{1.250000in}{4.400000in}}%
\pgfusepath{stroke}%
\end{pgfscope}%
\begin{pgfscope}%
\pgfsetrectcap%
\pgfsetmiterjoin%
\pgfsetlinewidth{0.803000pt}%
\definecolor{currentstroke}{rgb}{0.000000,0.000000,0.000000}%
\pgfsetstrokecolor{currentstroke}%
\pgfsetdash{}{0pt}%
\pgfpathmoveto{\pgfqpoint{9.000000in}{0.550000in}}%
\pgfpathlineto{\pgfqpoint{9.000000in}{4.400000in}}%
\pgfusepath{stroke}%
\end{pgfscope}%
\begin{pgfscope}%
\pgfsetrectcap%
\pgfsetmiterjoin%
\pgfsetlinewidth{0.803000pt}%
\definecolor{currentstroke}{rgb}{0.000000,0.000000,0.000000}%
\pgfsetstrokecolor{currentstroke}%
\pgfsetdash{}{0pt}%
\pgfpathmoveto{\pgfqpoint{1.250000in}{0.550000in}}%
\pgfpathlineto{\pgfqpoint{9.000000in}{0.550000in}}%
\pgfusepath{stroke}%
\end{pgfscope}%
\begin{pgfscope}%
\pgfsetrectcap%
\pgfsetmiterjoin%
\pgfsetlinewidth{0.803000pt}%
\definecolor{currentstroke}{rgb}{0.000000,0.000000,0.000000}%
\pgfsetstrokecolor{currentstroke}%
\pgfsetdash{}{0pt}%
\pgfpathmoveto{\pgfqpoint{1.250000in}{4.400000in}}%
\pgfpathlineto{\pgfqpoint{9.000000in}{4.400000in}}%
\pgfusepath{stroke}%
\end{pgfscope}%
\begin{pgfscope}%
\definecolor{textcolor}{rgb}{0.000000,0.000000,0.000000}%
\pgfsetstrokecolor{textcolor}%
\pgfsetfillcolor{textcolor}%
\pgftext[x=5.125000in,y=4.483333in,,base]{\color{textcolor}{\sffamily\fontsize{12.000000}{14.400000}\selectfont\catcode`\^=\active\def^{\ifmmode\sp\else\^{}\fi}\catcode`\%=\active\def%{\%}Histograma de Original}}%
\end{pgfscope}%
\end{pgfpicture}%
\makeatother%
\endgroup%
}
		\caption{Original}
		\label{fig:original_2}
	\end{subfigure}
	\hfill
	\begin{subfigure}[b]{0.45\textwidth}
		\resizebox{\linewidth}{!}{%% Creator: Matplotlib, PGF backend
%%
%% To include the figure in your LaTeX document, write
%%   \input{<filename>.pgf}
%%
%% Make sure the required packages are loaded in your preamble
%%   \usepackage{pgf}
%%
%% Also ensure that all the required font packages are loaded; for instance,
%% the lmodern package is sometimes necessary when using math font.
%%   \usepackage{lmodern}
%%
%% Figures using additional raster images can only be included by \input if
%% they are in the same directory as the main LaTeX file. For loading figures
%% from other directories you can use the `import` package
%%   \usepackage{import}
%%
%% and then include the figures with
%%   \import{<path to file>}{<filename>.pgf}
%%
%% Matplotlib used the following preamble
%%   \def\mathdefault#1{#1}
%%   \everymath=\expandafter{\the\everymath\displaystyle}
%%   \IfFileExists{scrextend.sty}{
%%     \usepackage[fontsize=10.000000pt]{scrextend}
%%   }{
%%     \renewcommand{\normalsize}{\fontsize{10.000000}{12.000000}\selectfont}
%%     \normalsize
%%   }
%%   
%%   \ifdefined\pdftexversion\else  % non-pdftex case.
%%     \usepackage{fontspec}
%%     \setmainfont{DejaVuSerif.ttf}[Path=\detokenize{/home/elias/.virtualenvs/src-pway/lib/python3.13/site-packages/matplotlib/mpl-data/fonts/ttf/}]
%%     \setsansfont{DejaVuSans.ttf}[Path=\detokenize{/home/elias/.virtualenvs/src-pway/lib/python3.13/site-packages/matplotlib/mpl-data/fonts/ttf/}]
%%     \setmonofont{DejaVuSansMono.ttf}[Path=\detokenize{/home/elias/.virtualenvs/src-pway/lib/python3.13/site-packages/matplotlib/mpl-data/fonts/ttf/}]
%%   \fi
%%   \makeatletter\@ifpackageloaded{underscore}{}{\usepackage[strings]{underscore}}\makeatother
%%
\begingroup%
\makeatletter%
\begin{pgfpicture}%
\pgfpathrectangle{\pgfpointorigin}{\pgfqpoint{10.000000in}{5.000000in}}%
\pgfusepath{use as bounding box, clip}%
\begin{pgfscope}%
\pgfsetbuttcap%
\pgfsetmiterjoin%
\definecolor{currentfill}{rgb}{1.000000,1.000000,1.000000}%
\pgfsetfillcolor{currentfill}%
\pgfsetlinewidth{0.000000pt}%
\definecolor{currentstroke}{rgb}{1.000000,1.000000,1.000000}%
\pgfsetstrokecolor{currentstroke}%
\pgfsetdash{}{0pt}%
\pgfpathmoveto{\pgfqpoint{0.000000in}{0.000000in}}%
\pgfpathlineto{\pgfqpoint{10.000000in}{0.000000in}}%
\pgfpathlineto{\pgfqpoint{10.000000in}{5.000000in}}%
\pgfpathlineto{\pgfqpoint{0.000000in}{5.000000in}}%
\pgfpathlineto{\pgfqpoint{0.000000in}{0.000000in}}%
\pgfpathclose%
\pgfusepath{fill}%
\end{pgfscope}%
\begin{pgfscope}%
\pgfsetbuttcap%
\pgfsetmiterjoin%
\definecolor{currentfill}{rgb}{1.000000,1.000000,1.000000}%
\pgfsetfillcolor{currentfill}%
\pgfsetlinewidth{0.000000pt}%
\definecolor{currentstroke}{rgb}{0.000000,0.000000,0.000000}%
\pgfsetstrokecolor{currentstroke}%
\pgfsetstrokeopacity{0.000000}%
\pgfsetdash{}{0pt}%
\pgfpathmoveto{\pgfqpoint{1.250000in}{0.550000in}}%
\pgfpathlineto{\pgfqpoint{9.000000in}{0.550000in}}%
\pgfpathlineto{\pgfqpoint{9.000000in}{4.400000in}}%
\pgfpathlineto{\pgfqpoint{1.250000in}{4.400000in}}%
\pgfpathlineto{\pgfqpoint{1.250000in}{0.550000in}}%
\pgfpathclose%
\pgfusepath{fill}%
\end{pgfscope}%
\begin{pgfscope}%
\pgfpathrectangle{\pgfqpoint{1.250000in}{0.550000in}}{\pgfqpoint{7.750000in}{3.850000in}}%
\pgfusepath{clip}%
\pgfsetbuttcap%
\pgfsetmiterjoin%
\definecolor{currentfill}{rgb}{0.501961,0.501961,0.501961}%
\pgfsetfillcolor{currentfill}%
\pgfsetfillopacity{0.700000}%
\pgfsetlinewidth{0.000000pt}%
\definecolor{currentstroke}{rgb}{0.000000,0.000000,0.000000}%
\pgfsetstrokecolor{currentstroke}%
\pgfsetstrokeopacity{0.700000}%
\pgfsetdash{}{0pt}%
\pgfpathmoveto{\pgfqpoint{1.602273in}{0.550000in}}%
\pgfpathlineto{\pgfqpoint{1.629794in}{0.550000in}}%
\pgfpathlineto{\pgfqpoint{1.629794in}{1.041835in}}%
\pgfpathlineto{\pgfqpoint{1.602273in}{1.041835in}}%
\pgfpathlineto{\pgfqpoint{1.602273in}{0.550000in}}%
\pgfpathclose%
\pgfusepath{fill}%
\end{pgfscope}%
\begin{pgfscope}%
\pgfpathrectangle{\pgfqpoint{1.250000in}{0.550000in}}{\pgfqpoint{7.750000in}{3.850000in}}%
\pgfusepath{clip}%
\pgfsetbuttcap%
\pgfsetmiterjoin%
\definecolor{currentfill}{rgb}{0.501961,0.501961,0.501961}%
\pgfsetfillcolor{currentfill}%
\pgfsetfillopacity{0.700000}%
\pgfsetlinewidth{0.000000pt}%
\definecolor{currentstroke}{rgb}{0.000000,0.000000,0.000000}%
\pgfsetstrokecolor{currentstroke}%
\pgfsetstrokeopacity{0.700000}%
\pgfsetdash{}{0pt}%
\pgfpathmoveto{\pgfqpoint{1.629794in}{0.550000in}}%
\pgfpathlineto{\pgfqpoint{1.657315in}{0.550000in}}%
\pgfpathlineto{\pgfqpoint{1.657315in}{2.069699in}}%
\pgfpathlineto{\pgfqpoint{1.629794in}{2.069699in}}%
\pgfpathlineto{\pgfqpoint{1.629794in}{0.550000in}}%
\pgfpathclose%
\pgfusepath{fill}%
\end{pgfscope}%
\begin{pgfscope}%
\pgfpathrectangle{\pgfqpoint{1.250000in}{0.550000in}}{\pgfqpoint{7.750000in}{3.850000in}}%
\pgfusepath{clip}%
\pgfsetbuttcap%
\pgfsetmiterjoin%
\definecolor{currentfill}{rgb}{0.501961,0.501961,0.501961}%
\pgfsetfillcolor{currentfill}%
\pgfsetfillopacity{0.700000}%
\pgfsetlinewidth{0.000000pt}%
\definecolor{currentstroke}{rgb}{0.000000,0.000000,0.000000}%
\pgfsetstrokecolor{currentstroke}%
\pgfsetstrokeopacity{0.700000}%
\pgfsetdash{}{0pt}%
\pgfpathmoveto{\pgfqpoint{1.657315in}{0.550000in}}%
\pgfpathlineto{\pgfqpoint{1.684837in}{0.550000in}}%
\pgfpathlineto{\pgfqpoint{1.684837in}{1.567885in}}%
\pgfpathlineto{\pgfqpoint{1.657315in}{1.567885in}}%
\pgfpathlineto{\pgfqpoint{1.657315in}{0.550000in}}%
\pgfpathclose%
\pgfusepath{fill}%
\end{pgfscope}%
\begin{pgfscope}%
\pgfpathrectangle{\pgfqpoint{1.250000in}{0.550000in}}{\pgfqpoint{7.750000in}{3.850000in}}%
\pgfusepath{clip}%
\pgfsetbuttcap%
\pgfsetmiterjoin%
\definecolor{currentfill}{rgb}{0.501961,0.501961,0.501961}%
\pgfsetfillcolor{currentfill}%
\pgfsetfillopacity{0.700000}%
\pgfsetlinewidth{0.000000pt}%
\definecolor{currentstroke}{rgb}{0.000000,0.000000,0.000000}%
\pgfsetstrokecolor{currentstroke}%
\pgfsetstrokeopacity{0.700000}%
\pgfsetdash{}{0pt}%
\pgfpathmoveto{\pgfqpoint{1.684837in}{0.550000in}}%
\pgfpathlineto{\pgfqpoint{1.712358in}{0.550000in}}%
\pgfpathlineto{\pgfqpoint{1.712358in}{1.999844in}}%
\pgfpathlineto{\pgfqpoint{1.684837in}{1.999844in}}%
\pgfpathlineto{\pgfqpoint{1.684837in}{0.550000in}}%
\pgfpathclose%
\pgfusepath{fill}%
\end{pgfscope}%
\begin{pgfscope}%
\pgfpathrectangle{\pgfqpoint{1.250000in}{0.550000in}}{\pgfqpoint{7.750000in}{3.850000in}}%
\pgfusepath{clip}%
\pgfsetbuttcap%
\pgfsetmiterjoin%
\definecolor{currentfill}{rgb}{0.501961,0.501961,0.501961}%
\pgfsetfillcolor{currentfill}%
\pgfsetfillopacity{0.700000}%
\pgfsetlinewidth{0.000000pt}%
\definecolor{currentstroke}{rgb}{0.000000,0.000000,0.000000}%
\pgfsetstrokecolor{currentstroke}%
\pgfsetstrokeopacity{0.700000}%
\pgfsetdash{}{0pt}%
\pgfpathmoveto{\pgfqpoint{1.712358in}{0.550000in}}%
\pgfpathlineto{\pgfqpoint{1.739879in}{0.550000in}}%
\pgfpathlineto{\pgfqpoint{1.739879in}{2.326309in}}%
\pgfpathlineto{\pgfqpoint{1.712358in}{2.326309in}}%
\pgfpathlineto{\pgfqpoint{1.712358in}{0.550000in}}%
\pgfpathclose%
\pgfusepath{fill}%
\end{pgfscope}%
\begin{pgfscope}%
\pgfpathrectangle{\pgfqpoint{1.250000in}{0.550000in}}{\pgfqpoint{7.750000in}{3.850000in}}%
\pgfusepath{clip}%
\pgfsetbuttcap%
\pgfsetmiterjoin%
\definecolor{currentfill}{rgb}{0.501961,0.501961,0.501961}%
\pgfsetfillcolor{currentfill}%
\pgfsetfillopacity{0.700000}%
\pgfsetlinewidth{0.000000pt}%
\definecolor{currentstroke}{rgb}{0.000000,0.000000,0.000000}%
\pgfsetstrokecolor{currentstroke}%
\pgfsetstrokeopacity{0.700000}%
\pgfsetdash{}{0pt}%
\pgfpathmoveto{\pgfqpoint{1.739879in}{0.550000in}}%
\pgfpathlineto{\pgfqpoint{1.767401in}{0.550000in}}%
\pgfpathlineto{\pgfqpoint{1.767401in}{2.290669in}}%
\pgfpathlineto{\pgfqpoint{1.739879in}{2.290669in}}%
\pgfpathlineto{\pgfqpoint{1.739879in}{0.550000in}}%
\pgfpathclose%
\pgfusepath{fill}%
\end{pgfscope}%
\begin{pgfscope}%
\pgfpathrectangle{\pgfqpoint{1.250000in}{0.550000in}}{\pgfqpoint{7.750000in}{3.850000in}}%
\pgfusepath{clip}%
\pgfsetbuttcap%
\pgfsetmiterjoin%
\definecolor{currentfill}{rgb}{0.501961,0.501961,0.501961}%
\pgfsetfillcolor{currentfill}%
\pgfsetfillopacity{0.700000}%
\pgfsetlinewidth{0.000000pt}%
\definecolor{currentstroke}{rgb}{0.000000,0.000000,0.000000}%
\pgfsetstrokecolor{currentstroke}%
\pgfsetstrokeopacity{0.700000}%
\pgfsetdash{}{0pt}%
\pgfpathmoveto{\pgfqpoint{1.767401in}{0.550000in}}%
\pgfpathlineto{\pgfqpoint{1.794922in}{0.550000in}}%
\pgfpathlineto{\pgfqpoint{1.794922in}{2.237921in}}%
\pgfpathlineto{\pgfqpoint{1.767401in}{2.237921in}}%
\pgfpathlineto{\pgfqpoint{1.767401in}{0.550000in}}%
\pgfpathclose%
\pgfusepath{fill}%
\end{pgfscope}%
\begin{pgfscope}%
\pgfpathrectangle{\pgfqpoint{1.250000in}{0.550000in}}{\pgfqpoint{7.750000in}{3.850000in}}%
\pgfusepath{clip}%
\pgfsetbuttcap%
\pgfsetmiterjoin%
\definecolor{currentfill}{rgb}{0.501961,0.501961,0.501961}%
\pgfsetfillcolor{currentfill}%
\pgfsetfillopacity{0.700000}%
\pgfsetlinewidth{0.000000pt}%
\definecolor{currentstroke}{rgb}{0.000000,0.000000,0.000000}%
\pgfsetstrokecolor{currentstroke}%
\pgfsetstrokeopacity{0.700000}%
\pgfsetdash{}{0pt}%
\pgfpathmoveto{\pgfqpoint{1.794922in}{0.550000in}}%
\pgfpathlineto{\pgfqpoint{1.822443in}{0.550000in}}%
\pgfpathlineto{\pgfqpoint{1.822443in}{2.269285in}}%
\pgfpathlineto{\pgfqpoint{1.794922in}{2.269285in}}%
\pgfpathlineto{\pgfqpoint{1.794922in}{0.550000in}}%
\pgfpathclose%
\pgfusepath{fill}%
\end{pgfscope}%
\begin{pgfscope}%
\pgfpathrectangle{\pgfqpoint{1.250000in}{0.550000in}}{\pgfqpoint{7.750000in}{3.850000in}}%
\pgfusepath{clip}%
\pgfsetbuttcap%
\pgfsetmiterjoin%
\definecolor{currentfill}{rgb}{0.501961,0.501961,0.501961}%
\pgfsetfillcolor{currentfill}%
\pgfsetfillopacity{0.700000}%
\pgfsetlinewidth{0.000000pt}%
\definecolor{currentstroke}{rgb}{0.000000,0.000000,0.000000}%
\pgfsetstrokecolor{currentstroke}%
\pgfsetstrokeopacity{0.700000}%
\pgfsetdash{}{0pt}%
\pgfpathmoveto{\pgfqpoint{1.822443in}{0.550000in}}%
\pgfpathlineto{\pgfqpoint{1.849964in}{0.550000in}}%
\pgfpathlineto{\pgfqpoint{1.849964in}{2.058294in}}%
\pgfpathlineto{\pgfqpoint{1.822443in}{2.058294in}}%
\pgfpathlineto{\pgfqpoint{1.822443in}{0.550000in}}%
\pgfpathclose%
\pgfusepath{fill}%
\end{pgfscope}%
\begin{pgfscope}%
\pgfpathrectangle{\pgfqpoint{1.250000in}{0.550000in}}{\pgfqpoint{7.750000in}{3.850000in}}%
\pgfusepath{clip}%
\pgfsetbuttcap%
\pgfsetmiterjoin%
\definecolor{currentfill}{rgb}{0.501961,0.501961,0.501961}%
\pgfsetfillcolor{currentfill}%
\pgfsetfillopacity{0.700000}%
\pgfsetlinewidth{0.000000pt}%
\definecolor{currentstroke}{rgb}{0.000000,0.000000,0.000000}%
\pgfsetstrokecolor{currentstroke}%
\pgfsetstrokeopacity{0.700000}%
\pgfsetdash{}{0pt}%
\pgfpathmoveto{\pgfqpoint{1.849964in}{0.550000in}}%
\pgfpathlineto{\pgfqpoint{1.877486in}{0.550000in}}%
\pgfpathlineto{\pgfqpoint{1.877486in}{2.049741in}}%
\pgfpathlineto{\pgfqpoint{1.849964in}{2.049741in}}%
\pgfpathlineto{\pgfqpoint{1.849964in}{0.550000in}}%
\pgfpathclose%
\pgfusepath{fill}%
\end{pgfscope}%
\begin{pgfscope}%
\pgfpathrectangle{\pgfqpoint{1.250000in}{0.550000in}}{\pgfqpoint{7.750000in}{3.850000in}}%
\pgfusepath{clip}%
\pgfsetbuttcap%
\pgfsetmiterjoin%
\definecolor{currentfill}{rgb}{0.501961,0.501961,0.501961}%
\pgfsetfillcolor{currentfill}%
\pgfsetfillopacity{0.700000}%
\pgfsetlinewidth{0.000000pt}%
\definecolor{currentstroke}{rgb}{0.000000,0.000000,0.000000}%
\pgfsetstrokecolor{currentstroke}%
\pgfsetstrokeopacity{0.700000}%
\pgfsetdash{}{0pt}%
\pgfpathmoveto{\pgfqpoint{1.877486in}{0.550000in}}%
\pgfpathlineto{\pgfqpoint{1.905007in}{0.550000in}}%
\pgfpathlineto{\pgfqpoint{1.905007in}{2.001270in}}%
\pgfpathlineto{\pgfqpoint{1.877486in}{2.001270in}}%
\pgfpathlineto{\pgfqpoint{1.877486in}{0.550000in}}%
\pgfpathclose%
\pgfusepath{fill}%
\end{pgfscope}%
\begin{pgfscope}%
\pgfpathrectangle{\pgfqpoint{1.250000in}{0.550000in}}{\pgfqpoint{7.750000in}{3.850000in}}%
\pgfusepath{clip}%
\pgfsetbuttcap%
\pgfsetmiterjoin%
\definecolor{currentfill}{rgb}{0.501961,0.501961,0.501961}%
\pgfsetfillcolor{currentfill}%
\pgfsetfillopacity{0.700000}%
\pgfsetlinewidth{0.000000pt}%
\definecolor{currentstroke}{rgb}{0.000000,0.000000,0.000000}%
\pgfsetstrokecolor{currentstroke}%
\pgfsetstrokeopacity{0.700000}%
\pgfsetdash{}{0pt}%
\pgfpathmoveto{\pgfqpoint{1.905007in}{0.550000in}}%
\pgfpathlineto{\pgfqpoint{1.932528in}{0.550000in}}%
\pgfpathlineto{\pgfqpoint{1.932528in}{2.109616in}}%
\pgfpathlineto{\pgfqpoint{1.905007in}{2.109616in}}%
\pgfpathlineto{\pgfqpoint{1.905007in}{0.550000in}}%
\pgfpathclose%
\pgfusepath{fill}%
\end{pgfscope}%
\begin{pgfscope}%
\pgfpathrectangle{\pgfqpoint{1.250000in}{0.550000in}}{\pgfqpoint{7.750000in}{3.850000in}}%
\pgfusepath{clip}%
\pgfsetbuttcap%
\pgfsetmiterjoin%
\definecolor{currentfill}{rgb}{0.501961,0.501961,0.501961}%
\pgfsetfillcolor{currentfill}%
\pgfsetfillopacity{0.700000}%
\pgfsetlinewidth{0.000000pt}%
\definecolor{currentstroke}{rgb}{0.000000,0.000000,0.000000}%
\pgfsetstrokecolor{currentstroke}%
\pgfsetstrokeopacity{0.700000}%
\pgfsetdash{}{0pt}%
\pgfpathmoveto{\pgfqpoint{1.932528in}{0.550000in}}%
\pgfpathlineto{\pgfqpoint{1.960050in}{0.550000in}}%
\pgfpathlineto{\pgfqpoint{1.960050in}{2.207983in}}%
\pgfpathlineto{\pgfqpoint{1.932528in}{2.207983in}}%
\pgfpathlineto{\pgfqpoint{1.932528in}{0.550000in}}%
\pgfpathclose%
\pgfusepath{fill}%
\end{pgfscope}%
\begin{pgfscope}%
\pgfpathrectangle{\pgfqpoint{1.250000in}{0.550000in}}{\pgfqpoint{7.750000in}{3.850000in}}%
\pgfusepath{clip}%
\pgfsetbuttcap%
\pgfsetmiterjoin%
\definecolor{currentfill}{rgb}{0.501961,0.501961,0.501961}%
\pgfsetfillcolor{currentfill}%
\pgfsetfillopacity{0.700000}%
\pgfsetlinewidth{0.000000pt}%
\definecolor{currentstroke}{rgb}{0.000000,0.000000,0.000000}%
\pgfsetstrokecolor{currentstroke}%
\pgfsetstrokeopacity{0.700000}%
\pgfsetdash{}{0pt}%
\pgfpathmoveto{\pgfqpoint{1.960050in}{0.550000in}}%
\pgfpathlineto{\pgfqpoint{1.987571in}{0.550000in}}%
\pgfpathlineto{\pgfqpoint{1.987571in}{2.302074in}}%
\pgfpathlineto{\pgfqpoint{1.960050in}{2.302074in}}%
\pgfpathlineto{\pgfqpoint{1.960050in}{0.550000in}}%
\pgfpathclose%
\pgfusepath{fill}%
\end{pgfscope}%
\begin{pgfscope}%
\pgfpathrectangle{\pgfqpoint{1.250000in}{0.550000in}}{\pgfqpoint{7.750000in}{3.850000in}}%
\pgfusepath{clip}%
\pgfsetbuttcap%
\pgfsetmiterjoin%
\definecolor{currentfill}{rgb}{0.501961,0.501961,0.501961}%
\pgfsetfillcolor{currentfill}%
\pgfsetfillopacity{0.700000}%
\pgfsetlinewidth{0.000000pt}%
\definecolor{currentstroke}{rgb}{0.000000,0.000000,0.000000}%
\pgfsetstrokecolor{currentstroke}%
\pgfsetstrokeopacity{0.700000}%
\pgfsetdash{}{0pt}%
\pgfpathmoveto{\pgfqpoint{1.987571in}{0.550000in}}%
\pgfpathlineto{\pgfqpoint{2.015092in}{0.550000in}}%
\pgfpathlineto{\pgfqpoint{2.015092in}{2.145257in}}%
\pgfpathlineto{\pgfqpoint{1.987571in}{2.145257in}}%
\pgfpathlineto{\pgfqpoint{1.987571in}{0.550000in}}%
\pgfpathclose%
\pgfusepath{fill}%
\end{pgfscope}%
\begin{pgfscope}%
\pgfpathrectangle{\pgfqpoint{1.250000in}{0.550000in}}{\pgfqpoint{7.750000in}{3.850000in}}%
\pgfusepath{clip}%
\pgfsetbuttcap%
\pgfsetmiterjoin%
\definecolor{currentfill}{rgb}{0.501961,0.501961,0.501961}%
\pgfsetfillcolor{currentfill}%
\pgfsetfillopacity{0.700000}%
\pgfsetlinewidth{0.000000pt}%
\definecolor{currentstroke}{rgb}{0.000000,0.000000,0.000000}%
\pgfsetstrokecolor{currentstroke}%
\pgfsetstrokeopacity{0.700000}%
\pgfsetdash{}{0pt}%
\pgfpathmoveto{\pgfqpoint{2.015092in}{0.550000in}}%
\pgfpathlineto{\pgfqpoint{2.042614in}{0.550000in}}%
\pgfpathlineto{\pgfqpoint{2.042614in}{2.188025in}}%
\pgfpathlineto{\pgfqpoint{2.015092in}{2.188025in}}%
\pgfpathlineto{\pgfqpoint{2.015092in}{0.550000in}}%
\pgfpathclose%
\pgfusepath{fill}%
\end{pgfscope}%
\begin{pgfscope}%
\pgfpathrectangle{\pgfqpoint{1.250000in}{0.550000in}}{\pgfqpoint{7.750000in}{3.850000in}}%
\pgfusepath{clip}%
\pgfsetbuttcap%
\pgfsetmiterjoin%
\definecolor{currentfill}{rgb}{0.501961,0.501961,0.501961}%
\pgfsetfillcolor{currentfill}%
\pgfsetfillopacity{0.700000}%
\pgfsetlinewidth{0.000000pt}%
\definecolor{currentstroke}{rgb}{0.000000,0.000000,0.000000}%
\pgfsetstrokecolor{currentstroke}%
\pgfsetstrokeopacity{0.700000}%
\pgfsetdash{}{0pt}%
\pgfpathmoveto{\pgfqpoint{2.042614in}{0.550000in}}%
\pgfpathlineto{\pgfqpoint{2.070135in}{0.550000in}}%
\pgfpathlineto{\pgfqpoint{2.070135in}{2.222240in}}%
\pgfpathlineto{\pgfqpoint{2.042614in}{2.222240in}}%
\pgfpathlineto{\pgfqpoint{2.042614in}{0.550000in}}%
\pgfpathclose%
\pgfusepath{fill}%
\end{pgfscope}%
\begin{pgfscope}%
\pgfpathrectangle{\pgfqpoint{1.250000in}{0.550000in}}{\pgfqpoint{7.750000in}{3.850000in}}%
\pgfusepath{clip}%
\pgfsetbuttcap%
\pgfsetmiterjoin%
\definecolor{currentfill}{rgb}{0.501961,0.501961,0.501961}%
\pgfsetfillcolor{currentfill}%
\pgfsetfillopacity{0.700000}%
\pgfsetlinewidth{0.000000pt}%
\definecolor{currentstroke}{rgb}{0.000000,0.000000,0.000000}%
\pgfsetstrokecolor{currentstroke}%
\pgfsetstrokeopacity{0.700000}%
\pgfsetdash{}{0pt}%
\pgfpathmoveto{\pgfqpoint{2.070135in}{0.550000in}}%
\pgfpathlineto{\pgfqpoint{2.097656in}{0.550000in}}%
\pgfpathlineto{\pgfqpoint{2.097656in}{2.168066in}}%
\pgfpathlineto{\pgfqpoint{2.070135in}{2.168066in}}%
\pgfpathlineto{\pgfqpoint{2.070135in}{0.550000in}}%
\pgfpathclose%
\pgfusepath{fill}%
\end{pgfscope}%
\begin{pgfscope}%
\pgfpathrectangle{\pgfqpoint{1.250000in}{0.550000in}}{\pgfqpoint{7.750000in}{3.850000in}}%
\pgfusepath{clip}%
\pgfsetbuttcap%
\pgfsetmiterjoin%
\definecolor{currentfill}{rgb}{0.501961,0.501961,0.501961}%
\pgfsetfillcolor{currentfill}%
\pgfsetfillopacity{0.700000}%
\pgfsetlinewidth{0.000000pt}%
\definecolor{currentstroke}{rgb}{0.000000,0.000000,0.000000}%
\pgfsetstrokecolor{currentstroke}%
\pgfsetstrokeopacity{0.700000}%
\pgfsetdash{}{0pt}%
\pgfpathmoveto{\pgfqpoint{2.097656in}{0.550000in}}%
\pgfpathlineto{\pgfqpoint{2.125178in}{0.550000in}}%
\pgfpathlineto{\pgfqpoint{2.125178in}{2.132426in}}%
\pgfpathlineto{\pgfqpoint{2.097656in}{2.132426in}}%
\pgfpathlineto{\pgfqpoint{2.097656in}{0.550000in}}%
\pgfpathclose%
\pgfusepath{fill}%
\end{pgfscope}%
\begin{pgfscope}%
\pgfpathrectangle{\pgfqpoint{1.250000in}{0.550000in}}{\pgfqpoint{7.750000in}{3.850000in}}%
\pgfusepath{clip}%
\pgfsetbuttcap%
\pgfsetmiterjoin%
\definecolor{currentfill}{rgb}{0.501961,0.501961,0.501961}%
\pgfsetfillcolor{currentfill}%
\pgfsetfillopacity{0.700000}%
\pgfsetlinewidth{0.000000pt}%
\definecolor{currentstroke}{rgb}{0.000000,0.000000,0.000000}%
\pgfsetstrokecolor{currentstroke}%
\pgfsetstrokeopacity{0.700000}%
\pgfsetdash{}{0pt}%
\pgfpathmoveto{\pgfqpoint{2.125178in}{0.550000in}}%
\pgfpathlineto{\pgfqpoint{2.152699in}{0.550000in}}%
\pgfpathlineto{\pgfqpoint{2.152699in}{2.276413in}}%
\pgfpathlineto{\pgfqpoint{2.125178in}{2.276413in}}%
\pgfpathlineto{\pgfqpoint{2.125178in}{0.550000in}}%
\pgfpathclose%
\pgfusepath{fill}%
\end{pgfscope}%
\begin{pgfscope}%
\pgfpathrectangle{\pgfqpoint{1.250000in}{0.550000in}}{\pgfqpoint{7.750000in}{3.850000in}}%
\pgfusepath{clip}%
\pgfsetbuttcap%
\pgfsetmiterjoin%
\definecolor{currentfill}{rgb}{0.501961,0.501961,0.501961}%
\pgfsetfillcolor{currentfill}%
\pgfsetfillopacity{0.700000}%
\pgfsetlinewidth{0.000000pt}%
\definecolor{currentstroke}{rgb}{0.000000,0.000000,0.000000}%
\pgfsetstrokecolor{currentstroke}%
\pgfsetstrokeopacity{0.700000}%
\pgfsetdash{}{0pt}%
\pgfpathmoveto{\pgfqpoint{2.152699in}{0.550000in}}%
\pgfpathlineto{\pgfqpoint{2.180220in}{0.550000in}}%
\pgfpathlineto{\pgfqpoint{2.180220in}{2.206558in}}%
\pgfpathlineto{\pgfqpoint{2.152699in}{2.206558in}}%
\pgfpathlineto{\pgfqpoint{2.152699in}{0.550000in}}%
\pgfpathclose%
\pgfusepath{fill}%
\end{pgfscope}%
\begin{pgfscope}%
\pgfpathrectangle{\pgfqpoint{1.250000in}{0.550000in}}{\pgfqpoint{7.750000in}{3.850000in}}%
\pgfusepath{clip}%
\pgfsetbuttcap%
\pgfsetmiterjoin%
\definecolor{currentfill}{rgb}{0.501961,0.501961,0.501961}%
\pgfsetfillcolor{currentfill}%
\pgfsetfillopacity{0.700000}%
\pgfsetlinewidth{0.000000pt}%
\definecolor{currentstroke}{rgb}{0.000000,0.000000,0.000000}%
\pgfsetstrokecolor{currentstroke}%
\pgfsetstrokeopacity{0.700000}%
\pgfsetdash{}{0pt}%
\pgfpathmoveto{\pgfqpoint{2.180220in}{0.550000in}}%
\pgfpathlineto{\pgfqpoint{2.207741in}{0.550000in}}%
\pgfpathlineto{\pgfqpoint{2.207741in}{2.303499in}}%
\pgfpathlineto{\pgfqpoint{2.180220in}{2.303499in}}%
\pgfpathlineto{\pgfqpoint{2.180220in}{0.550000in}}%
\pgfpathclose%
\pgfusepath{fill}%
\end{pgfscope}%
\begin{pgfscope}%
\pgfpathrectangle{\pgfqpoint{1.250000in}{0.550000in}}{\pgfqpoint{7.750000in}{3.850000in}}%
\pgfusepath{clip}%
\pgfsetbuttcap%
\pgfsetmiterjoin%
\definecolor{currentfill}{rgb}{0.501961,0.501961,0.501961}%
\pgfsetfillcolor{currentfill}%
\pgfsetfillopacity{0.700000}%
\pgfsetlinewidth{0.000000pt}%
\definecolor{currentstroke}{rgb}{0.000000,0.000000,0.000000}%
\pgfsetstrokecolor{currentstroke}%
\pgfsetstrokeopacity{0.700000}%
\pgfsetdash{}{0pt}%
\pgfpathmoveto{\pgfqpoint{2.207741in}{0.550000in}}%
\pgfpathlineto{\pgfqpoint{2.235263in}{0.550000in}}%
\pgfpathlineto{\pgfqpoint{2.235263in}{2.363375in}}%
\pgfpathlineto{\pgfqpoint{2.207741in}{2.363375in}}%
\pgfpathlineto{\pgfqpoint{2.207741in}{0.550000in}}%
\pgfpathclose%
\pgfusepath{fill}%
\end{pgfscope}%
\begin{pgfscope}%
\pgfpathrectangle{\pgfqpoint{1.250000in}{0.550000in}}{\pgfqpoint{7.750000in}{3.850000in}}%
\pgfusepath{clip}%
\pgfsetbuttcap%
\pgfsetmiterjoin%
\definecolor{currentfill}{rgb}{0.501961,0.501961,0.501961}%
\pgfsetfillcolor{currentfill}%
\pgfsetfillopacity{0.700000}%
\pgfsetlinewidth{0.000000pt}%
\definecolor{currentstroke}{rgb}{0.000000,0.000000,0.000000}%
\pgfsetstrokecolor{currentstroke}%
\pgfsetstrokeopacity{0.700000}%
\pgfsetdash{}{0pt}%
\pgfpathmoveto{\pgfqpoint{2.235263in}{0.550000in}}%
\pgfpathlineto{\pgfqpoint{2.262784in}{0.550000in}}%
\pgfpathlineto{\pgfqpoint{2.262784in}{2.363375in}}%
\pgfpathlineto{\pgfqpoint{2.235263in}{2.363375in}}%
\pgfpathlineto{\pgfqpoint{2.235263in}{0.550000in}}%
\pgfpathclose%
\pgfusepath{fill}%
\end{pgfscope}%
\begin{pgfscope}%
\pgfpathrectangle{\pgfqpoint{1.250000in}{0.550000in}}{\pgfqpoint{7.750000in}{3.850000in}}%
\pgfusepath{clip}%
\pgfsetbuttcap%
\pgfsetmiterjoin%
\definecolor{currentfill}{rgb}{0.501961,0.501961,0.501961}%
\pgfsetfillcolor{currentfill}%
\pgfsetfillopacity{0.700000}%
\pgfsetlinewidth{0.000000pt}%
\definecolor{currentstroke}{rgb}{0.000000,0.000000,0.000000}%
\pgfsetstrokecolor{currentstroke}%
\pgfsetstrokeopacity{0.700000}%
\pgfsetdash{}{0pt}%
\pgfpathmoveto{\pgfqpoint{2.262784in}{0.550000in}}%
\pgfpathlineto{\pgfqpoint{2.290305in}{0.550000in}}%
\pgfpathlineto{\pgfqpoint{2.290305in}{2.451763in}}%
\pgfpathlineto{\pgfqpoint{2.262784in}{2.451763in}}%
\pgfpathlineto{\pgfqpoint{2.262784in}{0.550000in}}%
\pgfpathclose%
\pgfusepath{fill}%
\end{pgfscope}%
\begin{pgfscope}%
\pgfpathrectangle{\pgfqpoint{1.250000in}{0.550000in}}{\pgfqpoint{7.750000in}{3.850000in}}%
\pgfusepath{clip}%
\pgfsetbuttcap%
\pgfsetmiterjoin%
\definecolor{currentfill}{rgb}{0.501961,0.501961,0.501961}%
\pgfsetfillcolor{currentfill}%
\pgfsetfillopacity{0.700000}%
\pgfsetlinewidth{0.000000pt}%
\definecolor{currentstroke}{rgb}{0.000000,0.000000,0.000000}%
\pgfsetstrokecolor{currentstroke}%
\pgfsetstrokeopacity{0.700000}%
\pgfsetdash{}{0pt}%
\pgfpathmoveto{\pgfqpoint{2.290305in}{0.550000in}}%
\pgfpathlineto{\pgfqpoint{2.317827in}{0.550000in}}%
\pgfpathlineto{\pgfqpoint{2.317827in}{2.527320in}}%
\pgfpathlineto{\pgfqpoint{2.290305in}{2.527320in}}%
\pgfpathlineto{\pgfqpoint{2.290305in}{0.550000in}}%
\pgfpathclose%
\pgfusepath{fill}%
\end{pgfscope}%
\begin{pgfscope}%
\pgfpathrectangle{\pgfqpoint{1.250000in}{0.550000in}}{\pgfqpoint{7.750000in}{3.850000in}}%
\pgfusepath{clip}%
\pgfsetbuttcap%
\pgfsetmiterjoin%
\definecolor{currentfill}{rgb}{0.501961,0.501961,0.501961}%
\pgfsetfillcolor{currentfill}%
\pgfsetfillopacity{0.700000}%
\pgfsetlinewidth{0.000000pt}%
\definecolor{currentstroke}{rgb}{0.000000,0.000000,0.000000}%
\pgfsetstrokecolor{currentstroke}%
\pgfsetstrokeopacity{0.700000}%
\pgfsetdash{}{0pt}%
\pgfpathmoveto{\pgfqpoint{2.317827in}{0.550000in}}%
\pgfpathlineto{\pgfqpoint{2.345348in}{0.550000in}}%
\pgfpathlineto{\pgfqpoint{2.345348in}{2.610005in}}%
\pgfpathlineto{\pgfqpoint{2.317827in}{2.610005in}}%
\pgfpathlineto{\pgfqpoint{2.317827in}{0.550000in}}%
\pgfpathclose%
\pgfusepath{fill}%
\end{pgfscope}%
\begin{pgfscope}%
\pgfpathrectangle{\pgfqpoint{1.250000in}{0.550000in}}{\pgfqpoint{7.750000in}{3.850000in}}%
\pgfusepath{clip}%
\pgfsetbuttcap%
\pgfsetmiterjoin%
\definecolor{currentfill}{rgb}{0.501961,0.501961,0.501961}%
\pgfsetfillcolor{currentfill}%
\pgfsetfillopacity{0.700000}%
\pgfsetlinewidth{0.000000pt}%
\definecolor{currentstroke}{rgb}{0.000000,0.000000,0.000000}%
\pgfsetstrokecolor{currentstroke}%
\pgfsetstrokeopacity{0.700000}%
\pgfsetdash{}{0pt}%
\pgfpathmoveto{\pgfqpoint{2.345348in}{0.550000in}}%
\pgfpathlineto{\pgfqpoint{2.372869in}{0.550000in}}%
\pgfpathlineto{\pgfqpoint{2.372869in}{2.493105in}}%
\pgfpathlineto{\pgfqpoint{2.345348in}{2.493105in}}%
\pgfpathlineto{\pgfqpoint{2.345348in}{0.550000in}}%
\pgfpathclose%
\pgfusepath{fill}%
\end{pgfscope}%
\begin{pgfscope}%
\pgfpathrectangle{\pgfqpoint{1.250000in}{0.550000in}}{\pgfqpoint{7.750000in}{3.850000in}}%
\pgfusepath{clip}%
\pgfsetbuttcap%
\pgfsetmiterjoin%
\definecolor{currentfill}{rgb}{0.501961,0.501961,0.501961}%
\pgfsetfillcolor{currentfill}%
\pgfsetfillopacity{0.700000}%
\pgfsetlinewidth{0.000000pt}%
\definecolor{currentstroke}{rgb}{0.000000,0.000000,0.000000}%
\pgfsetstrokecolor{currentstroke}%
\pgfsetstrokeopacity{0.700000}%
\pgfsetdash{}{0pt}%
\pgfpathmoveto{\pgfqpoint{2.372869in}{0.550000in}}%
\pgfpathlineto{\pgfqpoint{2.400391in}{0.550000in}}%
\pgfpathlineto{\pgfqpoint{2.400391in}{0.550000in}}%
\pgfpathlineto{\pgfqpoint{2.372869in}{0.550000in}}%
\pgfpathlineto{\pgfqpoint{2.372869in}{0.550000in}}%
\pgfpathclose%
\pgfusepath{fill}%
\end{pgfscope}%
\begin{pgfscope}%
\pgfpathrectangle{\pgfqpoint{1.250000in}{0.550000in}}{\pgfqpoint{7.750000in}{3.850000in}}%
\pgfusepath{clip}%
\pgfsetbuttcap%
\pgfsetmiterjoin%
\definecolor{currentfill}{rgb}{0.501961,0.501961,0.501961}%
\pgfsetfillcolor{currentfill}%
\pgfsetfillopacity{0.700000}%
\pgfsetlinewidth{0.000000pt}%
\definecolor{currentstroke}{rgb}{0.000000,0.000000,0.000000}%
\pgfsetstrokecolor{currentstroke}%
\pgfsetstrokeopacity{0.700000}%
\pgfsetdash{}{0pt}%
\pgfpathmoveto{\pgfqpoint{2.400391in}{0.550000in}}%
\pgfpathlineto{\pgfqpoint{2.427912in}{0.550000in}}%
\pgfpathlineto{\pgfqpoint{2.427912in}{2.655625in}}%
\pgfpathlineto{\pgfqpoint{2.400391in}{2.655625in}}%
\pgfpathlineto{\pgfqpoint{2.400391in}{0.550000in}}%
\pgfpathclose%
\pgfusepath{fill}%
\end{pgfscope}%
\begin{pgfscope}%
\pgfpathrectangle{\pgfqpoint{1.250000in}{0.550000in}}{\pgfqpoint{7.750000in}{3.850000in}}%
\pgfusepath{clip}%
\pgfsetbuttcap%
\pgfsetmiterjoin%
\definecolor{currentfill}{rgb}{0.501961,0.501961,0.501961}%
\pgfsetfillcolor{currentfill}%
\pgfsetfillopacity{0.700000}%
\pgfsetlinewidth{0.000000pt}%
\definecolor{currentstroke}{rgb}{0.000000,0.000000,0.000000}%
\pgfsetstrokecolor{currentstroke}%
\pgfsetstrokeopacity{0.700000}%
\pgfsetdash{}{0pt}%
\pgfpathmoveto{\pgfqpoint{2.427912in}{0.550000in}}%
\pgfpathlineto{\pgfqpoint{2.455433in}{0.550000in}}%
\pgfpathlineto{\pgfqpoint{2.455433in}{2.493105in}}%
\pgfpathlineto{\pgfqpoint{2.427912in}{2.493105in}}%
\pgfpathlineto{\pgfqpoint{2.427912in}{0.550000in}}%
\pgfpathclose%
\pgfusepath{fill}%
\end{pgfscope}%
\begin{pgfscope}%
\pgfpathrectangle{\pgfqpoint{1.250000in}{0.550000in}}{\pgfqpoint{7.750000in}{3.850000in}}%
\pgfusepath{clip}%
\pgfsetbuttcap%
\pgfsetmiterjoin%
\definecolor{currentfill}{rgb}{0.501961,0.501961,0.501961}%
\pgfsetfillcolor{currentfill}%
\pgfsetfillopacity{0.700000}%
\pgfsetlinewidth{0.000000pt}%
\definecolor{currentstroke}{rgb}{0.000000,0.000000,0.000000}%
\pgfsetstrokecolor{currentstroke}%
\pgfsetstrokeopacity{0.700000}%
\pgfsetdash{}{0pt}%
\pgfpathmoveto{\pgfqpoint{2.455433in}{0.550000in}}%
\pgfpathlineto{\pgfqpoint{2.482955in}{0.550000in}}%
\pgfpathlineto{\pgfqpoint{2.482955in}{2.574365in}}%
\pgfpathlineto{\pgfqpoint{2.455433in}{2.574365in}}%
\pgfpathlineto{\pgfqpoint{2.455433in}{0.550000in}}%
\pgfpathclose%
\pgfusepath{fill}%
\end{pgfscope}%
\begin{pgfscope}%
\pgfpathrectangle{\pgfqpoint{1.250000in}{0.550000in}}{\pgfqpoint{7.750000in}{3.850000in}}%
\pgfusepath{clip}%
\pgfsetbuttcap%
\pgfsetmiterjoin%
\definecolor{currentfill}{rgb}{0.501961,0.501961,0.501961}%
\pgfsetfillcolor{currentfill}%
\pgfsetfillopacity{0.700000}%
\pgfsetlinewidth{0.000000pt}%
\definecolor{currentstroke}{rgb}{0.000000,0.000000,0.000000}%
\pgfsetstrokecolor{currentstroke}%
\pgfsetstrokeopacity{0.700000}%
\pgfsetdash{}{0pt}%
\pgfpathmoveto{\pgfqpoint{2.482955in}{0.550000in}}%
\pgfpathlineto{\pgfqpoint{2.510476in}{0.550000in}}%
\pgfpathlineto{\pgfqpoint{2.510476in}{2.531597in}}%
\pgfpathlineto{\pgfqpoint{2.482955in}{2.531597in}}%
\pgfpathlineto{\pgfqpoint{2.482955in}{0.550000in}}%
\pgfpathclose%
\pgfusepath{fill}%
\end{pgfscope}%
\begin{pgfscope}%
\pgfpathrectangle{\pgfqpoint{1.250000in}{0.550000in}}{\pgfqpoint{7.750000in}{3.850000in}}%
\pgfusepath{clip}%
\pgfsetbuttcap%
\pgfsetmiterjoin%
\definecolor{currentfill}{rgb}{0.501961,0.501961,0.501961}%
\pgfsetfillcolor{currentfill}%
\pgfsetfillopacity{0.700000}%
\pgfsetlinewidth{0.000000pt}%
\definecolor{currentstroke}{rgb}{0.000000,0.000000,0.000000}%
\pgfsetstrokecolor{currentstroke}%
\pgfsetstrokeopacity{0.700000}%
\pgfsetdash{}{0pt}%
\pgfpathmoveto{\pgfqpoint{2.510476in}{0.550000in}}%
\pgfpathlineto{\pgfqpoint{2.537997in}{0.550000in}}%
\pgfpathlineto{\pgfqpoint{2.537997in}{2.591472in}}%
\pgfpathlineto{\pgfqpoint{2.510476in}{2.591472in}}%
\pgfpathlineto{\pgfqpoint{2.510476in}{0.550000in}}%
\pgfpathclose%
\pgfusepath{fill}%
\end{pgfscope}%
\begin{pgfscope}%
\pgfpathrectangle{\pgfqpoint{1.250000in}{0.550000in}}{\pgfqpoint{7.750000in}{3.850000in}}%
\pgfusepath{clip}%
\pgfsetbuttcap%
\pgfsetmiterjoin%
\definecolor{currentfill}{rgb}{0.501961,0.501961,0.501961}%
\pgfsetfillcolor{currentfill}%
\pgfsetfillopacity{0.700000}%
\pgfsetlinewidth{0.000000pt}%
\definecolor{currentstroke}{rgb}{0.000000,0.000000,0.000000}%
\pgfsetstrokecolor{currentstroke}%
\pgfsetstrokeopacity{0.700000}%
\pgfsetdash{}{0pt}%
\pgfpathmoveto{\pgfqpoint{2.537997in}{0.550000in}}%
\pgfpathlineto{\pgfqpoint{2.565518in}{0.550000in}}%
\pgfpathlineto{\pgfqpoint{2.565518in}{2.456039in}}%
\pgfpathlineto{\pgfqpoint{2.537997in}{2.456039in}}%
\pgfpathlineto{\pgfqpoint{2.537997in}{0.550000in}}%
\pgfpathclose%
\pgfusepath{fill}%
\end{pgfscope}%
\begin{pgfscope}%
\pgfpathrectangle{\pgfqpoint{1.250000in}{0.550000in}}{\pgfqpoint{7.750000in}{3.850000in}}%
\pgfusepath{clip}%
\pgfsetbuttcap%
\pgfsetmiterjoin%
\definecolor{currentfill}{rgb}{0.501961,0.501961,0.501961}%
\pgfsetfillcolor{currentfill}%
\pgfsetfillopacity{0.700000}%
\pgfsetlinewidth{0.000000pt}%
\definecolor{currentstroke}{rgb}{0.000000,0.000000,0.000000}%
\pgfsetstrokecolor{currentstroke}%
\pgfsetstrokeopacity{0.700000}%
\pgfsetdash{}{0pt}%
\pgfpathmoveto{\pgfqpoint{2.565518in}{0.550000in}}%
\pgfpathlineto{\pgfqpoint{2.593040in}{0.550000in}}%
\pgfpathlineto{\pgfqpoint{2.593040in}{2.538725in}}%
\pgfpathlineto{\pgfqpoint{2.565518in}{2.538725in}}%
\pgfpathlineto{\pgfqpoint{2.565518in}{0.550000in}}%
\pgfpathclose%
\pgfusepath{fill}%
\end{pgfscope}%
\begin{pgfscope}%
\pgfpathrectangle{\pgfqpoint{1.250000in}{0.550000in}}{\pgfqpoint{7.750000in}{3.850000in}}%
\pgfusepath{clip}%
\pgfsetbuttcap%
\pgfsetmiterjoin%
\definecolor{currentfill}{rgb}{0.501961,0.501961,0.501961}%
\pgfsetfillcolor{currentfill}%
\pgfsetfillopacity{0.700000}%
\pgfsetlinewidth{0.000000pt}%
\definecolor{currentstroke}{rgb}{0.000000,0.000000,0.000000}%
\pgfsetstrokecolor{currentstroke}%
\pgfsetstrokeopacity{0.700000}%
\pgfsetdash{}{0pt}%
\pgfpathmoveto{\pgfqpoint{2.593040in}{0.550000in}}%
\pgfpathlineto{\pgfqpoint{2.620561in}{0.550000in}}%
\pgfpathlineto{\pgfqpoint{2.620561in}{2.456039in}}%
\pgfpathlineto{\pgfqpoint{2.593040in}{2.456039in}}%
\pgfpathlineto{\pgfqpoint{2.593040in}{0.550000in}}%
\pgfpathclose%
\pgfusepath{fill}%
\end{pgfscope}%
\begin{pgfscope}%
\pgfpathrectangle{\pgfqpoint{1.250000in}{0.550000in}}{\pgfqpoint{7.750000in}{3.850000in}}%
\pgfusepath{clip}%
\pgfsetbuttcap%
\pgfsetmiterjoin%
\definecolor{currentfill}{rgb}{0.501961,0.501961,0.501961}%
\pgfsetfillcolor{currentfill}%
\pgfsetfillopacity{0.700000}%
\pgfsetlinewidth{0.000000pt}%
\definecolor{currentstroke}{rgb}{0.000000,0.000000,0.000000}%
\pgfsetstrokecolor{currentstroke}%
\pgfsetstrokeopacity{0.700000}%
\pgfsetdash{}{0pt}%
\pgfpathmoveto{\pgfqpoint{2.620561in}{0.550000in}}%
\pgfpathlineto{\pgfqpoint{2.648082in}{0.550000in}}%
\pgfpathlineto{\pgfqpoint{2.648082in}{2.438932in}}%
\pgfpathlineto{\pgfqpoint{2.620561in}{2.438932in}}%
\pgfpathlineto{\pgfqpoint{2.620561in}{0.550000in}}%
\pgfpathclose%
\pgfusepath{fill}%
\end{pgfscope}%
\begin{pgfscope}%
\pgfpathrectangle{\pgfqpoint{1.250000in}{0.550000in}}{\pgfqpoint{7.750000in}{3.850000in}}%
\pgfusepath{clip}%
\pgfsetbuttcap%
\pgfsetmiterjoin%
\definecolor{currentfill}{rgb}{0.501961,0.501961,0.501961}%
\pgfsetfillcolor{currentfill}%
\pgfsetfillopacity{0.700000}%
\pgfsetlinewidth{0.000000pt}%
\definecolor{currentstroke}{rgb}{0.000000,0.000000,0.000000}%
\pgfsetstrokecolor{currentstroke}%
\pgfsetstrokeopacity{0.700000}%
\pgfsetdash{}{0pt}%
\pgfpathmoveto{\pgfqpoint{2.648082in}{0.550000in}}%
\pgfpathlineto{\pgfqpoint{2.675604in}{0.550000in}}%
\pgfpathlineto{\pgfqpoint{2.675604in}{2.427527in}}%
\pgfpathlineto{\pgfqpoint{2.648082in}{2.427527in}}%
\pgfpathlineto{\pgfqpoint{2.648082in}{0.550000in}}%
\pgfpathclose%
\pgfusepath{fill}%
\end{pgfscope}%
\begin{pgfscope}%
\pgfpathrectangle{\pgfqpoint{1.250000in}{0.550000in}}{\pgfqpoint{7.750000in}{3.850000in}}%
\pgfusepath{clip}%
\pgfsetbuttcap%
\pgfsetmiterjoin%
\definecolor{currentfill}{rgb}{0.501961,0.501961,0.501961}%
\pgfsetfillcolor{currentfill}%
\pgfsetfillopacity{0.700000}%
\pgfsetlinewidth{0.000000pt}%
\definecolor{currentstroke}{rgb}{0.000000,0.000000,0.000000}%
\pgfsetstrokecolor{currentstroke}%
\pgfsetstrokeopacity{0.700000}%
\pgfsetdash{}{0pt}%
\pgfpathmoveto{\pgfqpoint{2.675604in}{0.550000in}}%
\pgfpathlineto{\pgfqpoint{2.703125in}{0.550000in}}%
\pgfpathlineto{\pgfqpoint{2.703125in}{2.340565in}}%
\pgfpathlineto{\pgfqpoint{2.675604in}{2.340565in}}%
\pgfpathlineto{\pgfqpoint{2.675604in}{0.550000in}}%
\pgfpathclose%
\pgfusepath{fill}%
\end{pgfscope}%
\begin{pgfscope}%
\pgfpathrectangle{\pgfqpoint{1.250000in}{0.550000in}}{\pgfqpoint{7.750000in}{3.850000in}}%
\pgfusepath{clip}%
\pgfsetbuttcap%
\pgfsetmiterjoin%
\definecolor{currentfill}{rgb}{0.501961,0.501961,0.501961}%
\pgfsetfillcolor{currentfill}%
\pgfsetfillopacity{0.700000}%
\pgfsetlinewidth{0.000000pt}%
\definecolor{currentstroke}{rgb}{0.000000,0.000000,0.000000}%
\pgfsetstrokecolor{currentstroke}%
\pgfsetstrokeopacity{0.700000}%
\pgfsetdash{}{0pt}%
\pgfpathmoveto{\pgfqpoint{2.703125in}{0.550000in}}%
\pgfpathlineto{\pgfqpoint{2.730646in}{0.550000in}}%
\pgfpathlineto{\pgfqpoint{2.730646in}{2.427527in}}%
\pgfpathlineto{\pgfqpoint{2.703125in}{2.427527in}}%
\pgfpathlineto{\pgfqpoint{2.703125in}{0.550000in}}%
\pgfpathclose%
\pgfusepath{fill}%
\end{pgfscope}%
\begin{pgfscope}%
\pgfpathrectangle{\pgfqpoint{1.250000in}{0.550000in}}{\pgfqpoint{7.750000in}{3.850000in}}%
\pgfusepath{clip}%
\pgfsetbuttcap%
\pgfsetmiterjoin%
\definecolor{currentfill}{rgb}{0.501961,0.501961,0.501961}%
\pgfsetfillcolor{currentfill}%
\pgfsetfillopacity{0.700000}%
\pgfsetlinewidth{0.000000pt}%
\definecolor{currentstroke}{rgb}{0.000000,0.000000,0.000000}%
\pgfsetstrokecolor{currentstroke}%
\pgfsetstrokeopacity{0.700000}%
\pgfsetdash{}{0pt}%
\pgfpathmoveto{\pgfqpoint{2.730646in}{0.550000in}}%
\pgfpathlineto{\pgfqpoint{2.758168in}{0.550000in}}%
\pgfpathlineto{\pgfqpoint{2.758168in}{2.427527in}}%
\pgfpathlineto{\pgfqpoint{2.730646in}{2.427527in}}%
\pgfpathlineto{\pgfqpoint{2.730646in}{0.550000in}}%
\pgfpathclose%
\pgfusepath{fill}%
\end{pgfscope}%
\begin{pgfscope}%
\pgfpathrectangle{\pgfqpoint{1.250000in}{0.550000in}}{\pgfqpoint{7.750000in}{3.850000in}}%
\pgfusepath{clip}%
\pgfsetbuttcap%
\pgfsetmiterjoin%
\definecolor{currentfill}{rgb}{0.501961,0.501961,0.501961}%
\pgfsetfillcolor{currentfill}%
\pgfsetfillopacity{0.700000}%
\pgfsetlinewidth{0.000000pt}%
\definecolor{currentstroke}{rgb}{0.000000,0.000000,0.000000}%
\pgfsetstrokecolor{currentstroke}%
\pgfsetstrokeopacity{0.700000}%
\pgfsetdash{}{0pt}%
\pgfpathmoveto{\pgfqpoint{2.758168in}{0.550000in}}%
\pgfpathlineto{\pgfqpoint{2.785689in}{0.550000in}}%
\pgfpathlineto{\pgfqpoint{2.785689in}{2.363375in}}%
\pgfpathlineto{\pgfqpoint{2.758168in}{2.363375in}}%
\pgfpathlineto{\pgfqpoint{2.758168in}{0.550000in}}%
\pgfpathclose%
\pgfusepath{fill}%
\end{pgfscope}%
\begin{pgfscope}%
\pgfpathrectangle{\pgfqpoint{1.250000in}{0.550000in}}{\pgfqpoint{7.750000in}{3.850000in}}%
\pgfusepath{clip}%
\pgfsetbuttcap%
\pgfsetmiterjoin%
\definecolor{currentfill}{rgb}{0.501961,0.501961,0.501961}%
\pgfsetfillcolor{currentfill}%
\pgfsetfillopacity{0.700000}%
\pgfsetlinewidth{0.000000pt}%
\definecolor{currentstroke}{rgb}{0.000000,0.000000,0.000000}%
\pgfsetstrokecolor{currentstroke}%
\pgfsetstrokeopacity{0.700000}%
\pgfsetdash{}{0pt}%
\pgfpathmoveto{\pgfqpoint{2.785689in}{0.550000in}}%
\pgfpathlineto{\pgfqpoint{2.813210in}{0.550000in}}%
\pgfpathlineto{\pgfqpoint{2.813210in}{2.387610in}}%
\pgfpathlineto{\pgfqpoint{2.785689in}{2.387610in}}%
\pgfpathlineto{\pgfqpoint{2.785689in}{0.550000in}}%
\pgfpathclose%
\pgfusepath{fill}%
\end{pgfscope}%
\begin{pgfscope}%
\pgfpathrectangle{\pgfqpoint{1.250000in}{0.550000in}}{\pgfqpoint{7.750000in}{3.850000in}}%
\pgfusepath{clip}%
\pgfsetbuttcap%
\pgfsetmiterjoin%
\definecolor{currentfill}{rgb}{0.501961,0.501961,0.501961}%
\pgfsetfillcolor{currentfill}%
\pgfsetfillopacity{0.700000}%
\pgfsetlinewidth{0.000000pt}%
\definecolor{currentstroke}{rgb}{0.000000,0.000000,0.000000}%
\pgfsetstrokecolor{currentstroke}%
\pgfsetstrokeopacity{0.700000}%
\pgfsetdash{}{0pt}%
\pgfpathmoveto{\pgfqpoint{2.813210in}{0.550000in}}%
\pgfpathlineto{\pgfqpoint{2.840732in}{0.550000in}}%
\pgfpathlineto{\pgfqpoint{2.840732in}{4.151089in}}%
\pgfpathlineto{\pgfqpoint{2.813210in}{4.151089in}}%
\pgfpathlineto{\pgfqpoint{2.813210in}{0.550000in}}%
\pgfpathclose%
\pgfusepath{fill}%
\end{pgfscope}%
\begin{pgfscope}%
\pgfpathrectangle{\pgfqpoint{1.250000in}{0.550000in}}{\pgfqpoint{7.750000in}{3.850000in}}%
\pgfusepath{clip}%
\pgfsetbuttcap%
\pgfsetmiterjoin%
\definecolor{currentfill}{rgb}{0.501961,0.501961,0.501961}%
\pgfsetfillcolor{currentfill}%
\pgfsetfillopacity{0.700000}%
\pgfsetlinewidth{0.000000pt}%
\definecolor{currentstroke}{rgb}{0.000000,0.000000,0.000000}%
\pgfsetstrokecolor{currentstroke}%
\pgfsetstrokeopacity{0.700000}%
\pgfsetdash{}{0pt}%
\pgfpathmoveto{\pgfqpoint{2.840732in}{0.550000in}}%
\pgfpathlineto{\pgfqpoint{2.868253in}{0.550000in}}%
\pgfpathlineto{\pgfqpoint{2.868253in}{2.406143in}}%
\pgfpathlineto{\pgfqpoint{2.840732in}{2.406143in}}%
\pgfpathlineto{\pgfqpoint{2.840732in}{0.550000in}}%
\pgfpathclose%
\pgfusepath{fill}%
\end{pgfscope}%
\begin{pgfscope}%
\pgfpathrectangle{\pgfqpoint{1.250000in}{0.550000in}}{\pgfqpoint{7.750000in}{3.850000in}}%
\pgfusepath{clip}%
\pgfsetbuttcap%
\pgfsetmiterjoin%
\definecolor{currentfill}{rgb}{0.501961,0.501961,0.501961}%
\pgfsetfillcolor{currentfill}%
\pgfsetfillopacity{0.700000}%
\pgfsetlinewidth{0.000000pt}%
\definecolor{currentstroke}{rgb}{0.000000,0.000000,0.000000}%
\pgfsetstrokecolor{currentstroke}%
\pgfsetstrokeopacity{0.700000}%
\pgfsetdash{}{0pt}%
\pgfpathmoveto{\pgfqpoint{2.868253in}{0.550000in}}%
\pgfpathlineto{\pgfqpoint{2.895774in}{0.550000in}}%
\pgfpathlineto{\pgfqpoint{2.895774in}{2.424676in}}%
\pgfpathlineto{\pgfqpoint{2.868253in}{2.424676in}}%
\pgfpathlineto{\pgfqpoint{2.868253in}{0.550000in}}%
\pgfpathclose%
\pgfusepath{fill}%
\end{pgfscope}%
\begin{pgfscope}%
\pgfpathrectangle{\pgfqpoint{1.250000in}{0.550000in}}{\pgfqpoint{7.750000in}{3.850000in}}%
\pgfusepath{clip}%
\pgfsetbuttcap%
\pgfsetmiterjoin%
\definecolor{currentfill}{rgb}{0.501961,0.501961,0.501961}%
\pgfsetfillcolor{currentfill}%
\pgfsetfillopacity{0.700000}%
\pgfsetlinewidth{0.000000pt}%
\definecolor{currentstroke}{rgb}{0.000000,0.000000,0.000000}%
\pgfsetstrokecolor{currentstroke}%
\pgfsetstrokeopacity{0.700000}%
\pgfsetdash{}{0pt}%
\pgfpathmoveto{\pgfqpoint{2.895774in}{0.550000in}}%
\pgfpathlineto{\pgfqpoint{2.923295in}{0.550000in}}%
\pgfpathlineto{\pgfqpoint{2.923295in}{2.393313in}}%
\pgfpathlineto{\pgfqpoint{2.895774in}{2.393313in}}%
\pgfpathlineto{\pgfqpoint{2.895774in}{0.550000in}}%
\pgfpathclose%
\pgfusepath{fill}%
\end{pgfscope}%
\begin{pgfscope}%
\pgfpathrectangle{\pgfqpoint{1.250000in}{0.550000in}}{\pgfqpoint{7.750000in}{3.850000in}}%
\pgfusepath{clip}%
\pgfsetbuttcap%
\pgfsetmiterjoin%
\definecolor{currentfill}{rgb}{0.501961,0.501961,0.501961}%
\pgfsetfillcolor{currentfill}%
\pgfsetfillopacity{0.700000}%
\pgfsetlinewidth{0.000000pt}%
\definecolor{currentstroke}{rgb}{0.000000,0.000000,0.000000}%
\pgfsetstrokecolor{currentstroke}%
\pgfsetstrokeopacity{0.700000}%
\pgfsetdash{}{0pt}%
\pgfpathmoveto{\pgfqpoint{2.923295in}{0.550000in}}%
\pgfpathlineto{\pgfqpoint{2.950817in}{0.550000in}}%
\pgfpathlineto{\pgfqpoint{2.950817in}{2.420399in}}%
\pgfpathlineto{\pgfqpoint{2.923295in}{2.420399in}}%
\pgfpathlineto{\pgfqpoint{2.923295in}{0.550000in}}%
\pgfpathclose%
\pgfusepath{fill}%
\end{pgfscope}%
\begin{pgfscope}%
\pgfpathrectangle{\pgfqpoint{1.250000in}{0.550000in}}{\pgfqpoint{7.750000in}{3.850000in}}%
\pgfusepath{clip}%
\pgfsetbuttcap%
\pgfsetmiterjoin%
\definecolor{currentfill}{rgb}{0.501961,0.501961,0.501961}%
\pgfsetfillcolor{currentfill}%
\pgfsetfillopacity{0.700000}%
\pgfsetlinewidth{0.000000pt}%
\definecolor{currentstroke}{rgb}{0.000000,0.000000,0.000000}%
\pgfsetstrokecolor{currentstroke}%
\pgfsetstrokeopacity{0.700000}%
\pgfsetdash{}{0pt}%
\pgfpathmoveto{\pgfqpoint{2.950817in}{0.550000in}}%
\pgfpathlineto{\pgfqpoint{2.978338in}{0.550000in}}%
\pgfpathlineto{\pgfqpoint{2.978338in}{2.467444in}}%
\pgfpathlineto{\pgfqpoint{2.950817in}{2.467444in}}%
\pgfpathlineto{\pgfqpoint{2.950817in}{0.550000in}}%
\pgfpathclose%
\pgfusepath{fill}%
\end{pgfscope}%
\begin{pgfscope}%
\pgfpathrectangle{\pgfqpoint{1.250000in}{0.550000in}}{\pgfqpoint{7.750000in}{3.850000in}}%
\pgfusepath{clip}%
\pgfsetbuttcap%
\pgfsetmiterjoin%
\definecolor{currentfill}{rgb}{0.501961,0.501961,0.501961}%
\pgfsetfillcolor{currentfill}%
\pgfsetfillopacity{0.700000}%
\pgfsetlinewidth{0.000000pt}%
\definecolor{currentstroke}{rgb}{0.000000,0.000000,0.000000}%
\pgfsetstrokecolor{currentstroke}%
\pgfsetstrokeopacity{0.700000}%
\pgfsetdash{}{0pt}%
\pgfpathmoveto{\pgfqpoint{2.978338in}{0.550000in}}%
\pgfpathlineto{\pgfqpoint{3.005859in}{0.550000in}}%
\pgfpathlineto{\pgfqpoint{3.005859in}{2.426102in}}%
\pgfpathlineto{\pgfqpoint{2.978338in}{2.426102in}}%
\pgfpathlineto{\pgfqpoint{2.978338in}{0.550000in}}%
\pgfpathclose%
\pgfusepath{fill}%
\end{pgfscope}%
\begin{pgfscope}%
\pgfpathrectangle{\pgfqpoint{1.250000in}{0.550000in}}{\pgfqpoint{7.750000in}{3.850000in}}%
\pgfusepath{clip}%
\pgfsetbuttcap%
\pgfsetmiterjoin%
\definecolor{currentfill}{rgb}{0.501961,0.501961,0.501961}%
\pgfsetfillcolor{currentfill}%
\pgfsetfillopacity{0.700000}%
\pgfsetlinewidth{0.000000pt}%
\definecolor{currentstroke}{rgb}{0.000000,0.000000,0.000000}%
\pgfsetstrokecolor{currentstroke}%
\pgfsetstrokeopacity{0.700000}%
\pgfsetdash{}{0pt}%
\pgfpathmoveto{\pgfqpoint{3.005859in}{0.550000in}}%
\pgfpathlineto{\pgfqpoint{3.033381in}{0.550000in}}%
\pgfpathlineto{\pgfqpoint{3.033381in}{2.615708in}}%
\pgfpathlineto{\pgfqpoint{3.005859in}{2.615708in}}%
\pgfpathlineto{\pgfqpoint{3.005859in}{0.550000in}}%
\pgfpathclose%
\pgfusepath{fill}%
\end{pgfscope}%
\begin{pgfscope}%
\pgfpathrectangle{\pgfqpoint{1.250000in}{0.550000in}}{\pgfqpoint{7.750000in}{3.850000in}}%
\pgfusepath{clip}%
\pgfsetbuttcap%
\pgfsetmiterjoin%
\definecolor{currentfill}{rgb}{0.501961,0.501961,0.501961}%
\pgfsetfillcolor{currentfill}%
\pgfsetfillopacity{0.700000}%
\pgfsetlinewidth{0.000000pt}%
\definecolor{currentstroke}{rgb}{0.000000,0.000000,0.000000}%
\pgfsetstrokecolor{currentstroke}%
\pgfsetstrokeopacity{0.700000}%
\pgfsetdash{}{0pt}%
\pgfpathmoveto{\pgfqpoint{3.033381in}{0.550000in}}%
\pgfpathlineto{\pgfqpoint{3.060902in}{0.550000in}}%
\pgfpathlineto{\pgfqpoint{3.060902in}{2.421825in}}%
\pgfpathlineto{\pgfqpoint{3.033381in}{2.421825in}}%
\pgfpathlineto{\pgfqpoint{3.033381in}{0.550000in}}%
\pgfpathclose%
\pgfusepath{fill}%
\end{pgfscope}%
\begin{pgfscope}%
\pgfpathrectangle{\pgfqpoint{1.250000in}{0.550000in}}{\pgfqpoint{7.750000in}{3.850000in}}%
\pgfusepath{clip}%
\pgfsetbuttcap%
\pgfsetmiterjoin%
\definecolor{currentfill}{rgb}{0.501961,0.501961,0.501961}%
\pgfsetfillcolor{currentfill}%
\pgfsetfillopacity{0.700000}%
\pgfsetlinewidth{0.000000pt}%
\definecolor{currentstroke}{rgb}{0.000000,0.000000,0.000000}%
\pgfsetstrokecolor{currentstroke}%
\pgfsetstrokeopacity{0.700000}%
\pgfsetdash{}{0pt}%
\pgfpathmoveto{\pgfqpoint{3.060902in}{0.550000in}}%
\pgfpathlineto{\pgfqpoint{3.088423in}{0.550000in}}%
\pgfpathlineto{\pgfqpoint{3.088423in}{2.584344in}}%
\pgfpathlineto{\pgfqpoint{3.060902in}{2.584344in}}%
\pgfpathlineto{\pgfqpoint{3.060902in}{0.550000in}}%
\pgfpathclose%
\pgfusepath{fill}%
\end{pgfscope}%
\begin{pgfscope}%
\pgfpathrectangle{\pgfqpoint{1.250000in}{0.550000in}}{\pgfqpoint{7.750000in}{3.850000in}}%
\pgfusepath{clip}%
\pgfsetbuttcap%
\pgfsetmiterjoin%
\definecolor{currentfill}{rgb}{0.501961,0.501961,0.501961}%
\pgfsetfillcolor{currentfill}%
\pgfsetfillopacity{0.700000}%
\pgfsetlinewidth{0.000000pt}%
\definecolor{currentstroke}{rgb}{0.000000,0.000000,0.000000}%
\pgfsetstrokecolor{currentstroke}%
\pgfsetstrokeopacity{0.700000}%
\pgfsetdash{}{0pt}%
\pgfpathmoveto{\pgfqpoint{3.088423in}{0.550000in}}%
\pgfpathlineto{\pgfqpoint{3.115945in}{0.550000in}}%
\pgfpathlineto{\pgfqpoint{3.115945in}{2.537299in}}%
\pgfpathlineto{\pgfqpoint{3.088423in}{2.537299in}}%
\pgfpathlineto{\pgfqpoint{3.088423in}{0.550000in}}%
\pgfpathclose%
\pgfusepath{fill}%
\end{pgfscope}%
\begin{pgfscope}%
\pgfpathrectangle{\pgfqpoint{1.250000in}{0.550000in}}{\pgfqpoint{7.750000in}{3.850000in}}%
\pgfusepath{clip}%
\pgfsetbuttcap%
\pgfsetmiterjoin%
\definecolor{currentfill}{rgb}{0.501961,0.501961,0.501961}%
\pgfsetfillcolor{currentfill}%
\pgfsetfillopacity{0.700000}%
\pgfsetlinewidth{0.000000pt}%
\definecolor{currentstroke}{rgb}{0.000000,0.000000,0.000000}%
\pgfsetstrokecolor{currentstroke}%
\pgfsetstrokeopacity{0.700000}%
\pgfsetdash{}{0pt}%
\pgfpathmoveto{\pgfqpoint{3.115945in}{0.550000in}}%
\pgfpathlineto{\pgfqpoint{3.143466in}{0.550000in}}%
\pgfpathlineto{\pgfqpoint{3.143466in}{2.658476in}}%
\pgfpathlineto{\pgfqpoint{3.115945in}{2.658476in}}%
\pgfpathlineto{\pgfqpoint{3.115945in}{0.550000in}}%
\pgfpathclose%
\pgfusepath{fill}%
\end{pgfscope}%
\begin{pgfscope}%
\pgfpathrectangle{\pgfqpoint{1.250000in}{0.550000in}}{\pgfqpoint{7.750000in}{3.850000in}}%
\pgfusepath{clip}%
\pgfsetbuttcap%
\pgfsetmiterjoin%
\definecolor{currentfill}{rgb}{0.501961,0.501961,0.501961}%
\pgfsetfillcolor{currentfill}%
\pgfsetfillopacity{0.700000}%
\pgfsetlinewidth{0.000000pt}%
\definecolor{currentstroke}{rgb}{0.000000,0.000000,0.000000}%
\pgfsetstrokecolor{currentstroke}%
\pgfsetstrokeopacity{0.700000}%
\pgfsetdash{}{0pt}%
\pgfpathmoveto{\pgfqpoint{3.143466in}{0.550000in}}%
\pgfpathlineto{\pgfqpoint{3.170987in}{0.550000in}}%
\pgfpathlineto{\pgfqpoint{3.170987in}{2.588621in}}%
\pgfpathlineto{\pgfqpoint{3.143466in}{2.588621in}}%
\pgfpathlineto{\pgfqpoint{3.143466in}{0.550000in}}%
\pgfpathclose%
\pgfusepath{fill}%
\end{pgfscope}%
\begin{pgfscope}%
\pgfpathrectangle{\pgfqpoint{1.250000in}{0.550000in}}{\pgfqpoint{7.750000in}{3.850000in}}%
\pgfusepath{clip}%
\pgfsetbuttcap%
\pgfsetmiterjoin%
\definecolor{currentfill}{rgb}{0.501961,0.501961,0.501961}%
\pgfsetfillcolor{currentfill}%
\pgfsetfillopacity{0.700000}%
\pgfsetlinewidth{0.000000pt}%
\definecolor{currentstroke}{rgb}{0.000000,0.000000,0.000000}%
\pgfsetstrokecolor{currentstroke}%
\pgfsetstrokeopacity{0.700000}%
\pgfsetdash{}{0pt}%
\pgfpathmoveto{\pgfqpoint{3.170987in}{0.550000in}}%
\pgfpathlineto{\pgfqpoint{3.198509in}{0.550000in}}%
\pgfpathlineto{\pgfqpoint{3.198509in}{0.550000in}}%
\pgfpathlineto{\pgfqpoint{3.170987in}{0.550000in}}%
\pgfpathlineto{\pgfqpoint{3.170987in}{0.550000in}}%
\pgfpathclose%
\pgfusepath{fill}%
\end{pgfscope}%
\begin{pgfscope}%
\pgfpathrectangle{\pgfqpoint{1.250000in}{0.550000in}}{\pgfqpoint{7.750000in}{3.850000in}}%
\pgfusepath{clip}%
\pgfsetbuttcap%
\pgfsetmiterjoin%
\definecolor{currentfill}{rgb}{0.501961,0.501961,0.501961}%
\pgfsetfillcolor{currentfill}%
\pgfsetfillopacity{0.700000}%
\pgfsetlinewidth{0.000000pt}%
\definecolor{currentstroke}{rgb}{0.000000,0.000000,0.000000}%
\pgfsetstrokecolor{currentstroke}%
\pgfsetstrokeopacity{0.700000}%
\pgfsetdash{}{0pt}%
\pgfpathmoveto{\pgfqpoint{3.198509in}{0.550000in}}%
\pgfpathlineto{\pgfqpoint{3.226030in}{0.550000in}}%
\pgfpathlineto{\pgfqpoint{3.226030in}{2.631389in}}%
\pgfpathlineto{\pgfqpoint{3.198509in}{2.631389in}}%
\pgfpathlineto{\pgfqpoint{3.198509in}{0.550000in}}%
\pgfpathclose%
\pgfusepath{fill}%
\end{pgfscope}%
\begin{pgfscope}%
\pgfpathrectangle{\pgfqpoint{1.250000in}{0.550000in}}{\pgfqpoint{7.750000in}{3.850000in}}%
\pgfusepath{clip}%
\pgfsetbuttcap%
\pgfsetmiterjoin%
\definecolor{currentfill}{rgb}{0.501961,0.501961,0.501961}%
\pgfsetfillcolor{currentfill}%
\pgfsetfillopacity{0.700000}%
\pgfsetlinewidth{0.000000pt}%
\definecolor{currentstroke}{rgb}{0.000000,0.000000,0.000000}%
\pgfsetstrokecolor{currentstroke}%
\pgfsetstrokeopacity{0.700000}%
\pgfsetdash{}{0pt}%
\pgfpathmoveto{\pgfqpoint{3.226030in}{0.550000in}}%
\pgfpathlineto{\pgfqpoint{3.253551in}{0.550000in}}%
\pgfpathlineto{\pgfqpoint{3.253551in}{2.712649in}}%
\pgfpathlineto{\pgfqpoint{3.226030in}{2.712649in}}%
\pgfpathlineto{\pgfqpoint{3.226030in}{0.550000in}}%
\pgfpathclose%
\pgfusepath{fill}%
\end{pgfscope}%
\begin{pgfscope}%
\pgfpathrectangle{\pgfqpoint{1.250000in}{0.550000in}}{\pgfqpoint{7.750000in}{3.850000in}}%
\pgfusepath{clip}%
\pgfsetbuttcap%
\pgfsetmiterjoin%
\definecolor{currentfill}{rgb}{0.501961,0.501961,0.501961}%
\pgfsetfillcolor{currentfill}%
\pgfsetfillopacity{0.700000}%
\pgfsetlinewidth{0.000000pt}%
\definecolor{currentstroke}{rgb}{0.000000,0.000000,0.000000}%
\pgfsetstrokecolor{currentstroke}%
\pgfsetstrokeopacity{0.700000}%
\pgfsetdash{}{0pt}%
\pgfpathmoveto{\pgfqpoint{3.253551in}{0.550000in}}%
\pgfpathlineto{\pgfqpoint{3.281072in}{0.550000in}}%
\pgfpathlineto{\pgfqpoint{3.281072in}{2.662753in}}%
\pgfpathlineto{\pgfqpoint{3.253551in}{2.662753in}}%
\pgfpathlineto{\pgfqpoint{3.253551in}{0.550000in}}%
\pgfpathclose%
\pgfusepath{fill}%
\end{pgfscope}%
\begin{pgfscope}%
\pgfpathrectangle{\pgfqpoint{1.250000in}{0.550000in}}{\pgfqpoint{7.750000in}{3.850000in}}%
\pgfusepath{clip}%
\pgfsetbuttcap%
\pgfsetmiterjoin%
\definecolor{currentfill}{rgb}{0.501961,0.501961,0.501961}%
\pgfsetfillcolor{currentfill}%
\pgfsetfillopacity{0.700000}%
\pgfsetlinewidth{0.000000pt}%
\definecolor{currentstroke}{rgb}{0.000000,0.000000,0.000000}%
\pgfsetstrokecolor{currentstroke}%
\pgfsetstrokeopacity{0.700000}%
\pgfsetdash{}{0pt}%
\pgfpathmoveto{\pgfqpoint{3.281072in}{0.550000in}}%
\pgfpathlineto{\pgfqpoint{3.308594in}{0.550000in}}%
\pgfpathlineto{\pgfqpoint{3.308594in}{2.627112in}}%
\pgfpathlineto{\pgfqpoint{3.281072in}{2.627112in}}%
\pgfpathlineto{\pgfqpoint{3.281072in}{0.550000in}}%
\pgfpathclose%
\pgfusepath{fill}%
\end{pgfscope}%
\begin{pgfscope}%
\pgfpathrectangle{\pgfqpoint{1.250000in}{0.550000in}}{\pgfqpoint{7.750000in}{3.850000in}}%
\pgfusepath{clip}%
\pgfsetbuttcap%
\pgfsetmiterjoin%
\definecolor{currentfill}{rgb}{0.501961,0.501961,0.501961}%
\pgfsetfillcolor{currentfill}%
\pgfsetfillopacity{0.700000}%
\pgfsetlinewidth{0.000000pt}%
\definecolor{currentstroke}{rgb}{0.000000,0.000000,0.000000}%
\pgfsetstrokecolor{currentstroke}%
\pgfsetstrokeopacity{0.700000}%
\pgfsetdash{}{0pt}%
\pgfpathmoveto{\pgfqpoint{3.308594in}{0.550000in}}%
\pgfpathlineto{\pgfqpoint{3.336115in}{0.550000in}}%
\pgfpathlineto{\pgfqpoint{3.336115in}{2.829549in}}%
\pgfpathlineto{\pgfqpoint{3.308594in}{2.829549in}}%
\pgfpathlineto{\pgfqpoint{3.308594in}{0.550000in}}%
\pgfpathclose%
\pgfusepath{fill}%
\end{pgfscope}%
\begin{pgfscope}%
\pgfpathrectangle{\pgfqpoint{1.250000in}{0.550000in}}{\pgfqpoint{7.750000in}{3.850000in}}%
\pgfusepath{clip}%
\pgfsetbuttcap%
\pgfsetmiterjoin%
\definecolor{currentfill}{rgb}{0.501961,0.501961,0.501961}%
\pgfsetfillcolor{currentfill}%
\pgfsetfillopacity{0.700000}%
\pgfsetlinewidth{0.000000pt}%
\definecolor{currentstroke}{rgb}{0.000000,0.000000,0.000000}%
\pgfsetstrokecolor{currentstroke}%
\pgfsetstrokeopacity{0.700000}%
\pgfsetdash{}{0pt}%
\pgfpathmoveto{\pgfqpoint{3.336115in}{0.550000in}}%
\pgfpathlineto{\pgfqpoint{3.363636in}{0.550000in}}%
\pgfpathlineto{\pgfqpoint{3.363636in}{2.724054in}}%
\pgfpathlineto{\pgfqpoint{3.336115in}{2.724054in}}%
\pgfpathlineto{\pgfqpoint{3.336115in}{0.550000in}}%
\pgfpathclose%
\pgfusepath{fill}%
\end{pgfscope}%
\begin{pgfscope}%
\pgfpathrectangle{\pgfqpoint{1.250000in}{0.550000in}}{\pgfqpoint{7.750000in}{3.850000in}}%
\pgfusepath{clip}%
\pgfsetbuttcap%
\pgfsetmiterjoin%
\definecolor{currentfill}{rgb}{0.501961,0.501961,0.501961}%
\pgfsetfillcolor{currentfill}%
\pgfsetfillopacity{0.700000}%
\pgfsetlinewidth{0.000000pt}%
\definecolor{currentstroke}{rgb}{0.000000,0.000000,0.000000}%
\pgfsetstrokecolor{currentstroke}%
\pgfsetstrokeopacity{0.700000}%
\pgfsetdash{}{0pt}%
\pgfpathmoveto{\pgfqpoint{3.363636in}{0.550000in}}%
\pgfpathlineto{\pgfqpoint{3.391158in}{0.550000in}}%
\pgfpathlineto{\pgfqpoint{3.391158in}{2.806739in}}%
\pgfpathlineto{\pgfqpoint{3.363636in}{2.806739in}}%
\pgfpathlineto{\pgfqpoint{3.363636in}{0.550000in}}%
\pgfpathclose%
\pgfusepath{fill}%
\end{pgfscope}%
\begin{pgfscope}%
\pgfpathrectangle{\pgfqpoint{1.250000in}{0.550000in}}{\pgfqpoint{7.750000in}{3.850000in}}%
\pgfusepath{clip}%
\pgfsetbuttcap%
\pgfsetmiterjoin%
\definecolor{currentfill}{rgb}{0.501961,0.501961,0.501961}%
\pgfsetfillcolor{currentfill}%
\pgfsetfillopacity{0.700000}%
\pgfsetlinewidth{0.000000pt}%
\definecolor{currentstroke}{rgb}{0.000000,0.000000,0.000000}%
\pgfsetstrokecolor{currentstroke}%
\pgfsetstrokeopacity{0.700000}%
\pgfsetdash{}{0pt}%
\pgfpathmoveto{\pgfqpoint{3.391158in}{0.550000in}}%
\pgfpathlineto{\pgfqpoint{3.418679in}{0.550000in}}%
\pgfpathlineto{\pgfqpoint{3.418679in}{2.734033in}}%
\pgfpathlineto{\pgfqpoint{3.391158in}{2.734033in}}%
\pgfpathlineto{\pgfqpoint{3.391158in}{0.550000in}}%
\pgfpathclose%
\pgfusepath{fill}%
\end{pgfscope}%
\begin{pgfscope}%
\pgfpathrectangle{\pgfqpoint{1.250000in}{0.550000in}}{\pgfqpoint{7.750000in}{3.850000in}}%
\pgfusepath{clip}%
\pgfsetbuttcap%
\pgfsetmiterjoin%
\definecolor{currentfill}{rgb}{0.501961,0.501961,0.501961}%
\pgfsetfillcolor{currentfill}%
\pgfsetfillopacity{0.700000}%
\pgfsetlinewidth{0.000000pt}%
\definecolor{currentstroke}{rgb}{0.000000,0.000000,0.000000}%
\pgfsetstrokecolor{currentstroke}%
\pgfsetstrokeopacity{0.700000}%
\pgfsetdash{}{0pt}%
\pgfpathmoveto{\pgfqpoint{3.418679in}{0.550000in}}%
\pgfpathlineto{\pgfqpoint{3.446200in}{0.550000in}}%
\pgfpathlineto{\pgfqpoint{3.446200in}{2.811016in}}%
\pgfpathlineto{\pgfqpoint{3.418679in}{2.811016in}}%
\pgfpathlineto{\pgfqpoint{3.418679in}{0.550000in}}%
\pgfpathclose%
\pgfusepath{fill}%
\end{pgfscope}%
\begin{pgfscope}%
\pgfpathrectangle{\pgfqpoint{1.250000in}{0.550000in}}{\pgfqpoint{7.750000in}{3.850000in}}%
\pgfusepath{clip}%
\pgfsetbuttcap%
\pgfsetmiterjoin%
\definecolor{currentfill}{rgb}{0.501961,0.501961,0.501961}%
\pgfsetfillcolor{currentfill}%
\pgfsetfillopacity{0.700000}%
\pgfsetlinewidth{0.000000pt}%
\definecolor{currentstroke}{rgb}{0.000000,0.000000,0.000000}%
\pgfsetstrokecolor{currentstroke}%
\pgfsetstrokeopacity{0.700000}%
\pgfsetdash{}{0pt}%
\pgfpathmoveto{\pgfqpoint{3.446200in}{0.550000in}}%
\pgfpathlineto{\pgfqpoint{3.473722in}{0.550000in}}%
\pgfpathlineto{\pgfqpoint{3.473722in}{2.752566in}}%
\pgfpathlineto{\pgfqpoint{3.446200in}{2.752566in}}%
\pgfpathlineto{\pgfqpoint{3.446200in}{0.550000in}}%
\pgfpathclose%
\pgfusepath{fill}%
\end{pgfscope}%
\begin{pgfscope}%
\pgfpathrectangle{\pgfqpoint{1.250000in}{0.550000in}}{\pgfqpoint{7.750000in}{3.850000in}}%
\pgfusepath{clip}%
\pgfsetbuttcap%
\pgfsetmiterjoin%
\definecolor{currentfill}{rgb}{0.501961,0.501961,0.501961}%
\pgfsetfillcolor{currentfill}%
\pgfsetfillopacity{0.700000}%
\pgfsetlinewidth{0.000000pt}%
\definecolor{currentstroke}{rgb}{0.000000,0.000000,0.000000}%
\pgfsetstrokecolor{currentstroke}%
\pgfsetstrokeopacity{0.700000}%
\pgfsetdash{}{0pt}%
\pgfpathmoveto{\pgfqpoint{3.473722in}{0.550000in}}%
\pgfpathlineto{\pgfqpoint{3.501243in}{0.550000in}}%
\pgfpathlineto{\pgfqpoint{3.501243in}{2.714075in}}%
\pgfpathlineto{\pgfqpoint{3.473722in}{2.714075in}}%
\pgfpathlineto{\pgfqpoint{3.473722in}{0.550000in}}%
\pgfpathclose%
\pgfusepath{fill}%
\end{pgfscope}%
\begin{pgfscope}%
\pgfpathrectangle{\pgfqpoint{1.250000in}{0.550000in}}{\pgfqpoint{7.750000in}{3.850000in}}%
\pgfusepath{clip}%
\pgfsetbuttcap%
\pgfsetmiterjoin%
\definecolor{currentfill}{rgb}{0.501961,0.501961,0.501961}%
\pgfsetfillcolor{currentfill}%
\pgfsetfillopacity{0.700000}%
\pgfsetlinewidth{0.000000pt}%
\definecolor{currentstroke}{rgb}{0.000000,0.000000,0.000000}%
\pgfsetstrokecolor{currentstroke}%
\pgfsetstrokeopacity{0.700000}%
\pgfsetdash{}{0pt}%
\pgfpathmoveto{\pgfqpoint{3.501243in}{0.550000in}}%
\pgfpathlineto{\pgfqpoint{3.528764in}{0.550000in}}%
\pgfpathlineto{\pgfqpoint{3.528764in}{2.724054in}}%
\pgfpathlineto{\pgfqpoint{3.501243in}{2.724054in}}%
\pgfpathlineto{\pgfqpoint{3.501243in}{0.550000in}}%
\pgfpathclose%
\pgfusepath{fill}%
\end{pgfscope}%
\begin{pgfscope}%
\pgfpathrectangle{\pgfqpoint{1.250000in}{0.550000in}}{\pgfqpoint{7.750000in}{3.850000in}}%
\pgfusepath{clip}%
\pgfsetbuttcap%
\pgfsetmiterjoin%
\definecolor{currentfill}{rgb}{0.501961,0.501961,0.501961}%
\pgfsetfillcolor{currentfill}%
\pgfsetfillopacity{0.700000}%
\pgfsetlinewidth{0.000000pt}%
\definecolor{currentstroke}{rgb}{0.000000,0.000000,0.000000}%
\pgfsetstrokecolor{currentstroke}%
\pgfsetstrokeopacity{0.700000}%
\pgfsetdash{}{0pt}%
\pgfpathmoveto{\pgfqpoint{3.528764in}{0.550000in}}%
\pgfpathlineto{\pgfqpoint{3.556286in}{0.550000in}}%
\pgfpathlineto{\pgfqpoint{3.556286in}{2.719777in}}%
\pgfpathlineto{\pgfqpoint{3.528764in}{2.719777in}}%
\pgfpathlineto{\pgfqpoint{3.528764in}{0.550000in}}%
\pgfpathclose%
\pgfusepath{fill}%
\end{pgfscope}%
\begin{pgfscope}%
\pgfpathrectangle{\pgfqpoint{1.250000in}{0.550000in}}{\pgfqpoint{7.750000in}{3.850000in}}%
\pgfusepath{clip}%
\pgfsetbuttcap%
\pgfsetmiterjoin%
\definecolor{currentfill}{rgb}{0.501961,0.501961,0.501961}%
\pgfsetfillcolor{currentfill}%
\pgfsetfillopacity{0.700000}%
\pgfsetlinewidth{0.000000pt}%
\definecolor{currentstroke}{rgb}{0.000000,0.000000,0.000000}%
\pgfsetstrokecolor{currentstroke}%
\pgfsetstrokeopacity{0.700000}%
\pgfsetdash{}{0pt}%
\pgfpathmoveto{\pgfqpoint{3.556286in}{0.550000in}}%
\pgfpathlineto{\pgfqpoint{3.583807in}{0.550000in}}%
\pgfpathlineto{\pgfqpoint{3.583807in}{2.849508in}}%
\pgfpathlineto{\pgfqpoint{3.556286in}{2.849508in}}%
\pgfpathlineto{\pgfqpoint{3.556286in}{0.550000in}}%
\pgfpathclose%
\pgfusepath{fill}%
\end{pgfscope}%
\begin{pgfscope}%
\pgfpathrectangle{\pgfqpoint{1.250000in}{0.550000in}}{\pgfqpoint{7.750000in}{3.850000in}}%
\pgfusepath{clip}%
\pgfsetbuttcap%
\pgfsetmiterjoin%
\definecolor{currentfill}{rgb}{0.501961,0.501961,0.501961}%
\pgfsetfillcolor{currentfill}%
\pgfsetfillopacity{0.700000}%
\pgfsetlinewidth{0.000000pt}%
\definecolor{currentstroke}{rgb}{0.000000,0.000000,0.000000}%
\pgfsetstrokecolor{currentstroke}%
\pgfsetstrokeopacity{0.700000}%
\pgfsetdash{}{0pt}%
\pgfpathmoveto{\pgfqpoint{3.583807in}{0.550000in}}%
\pgfpathlineto{\pgfqpoint{3.611328in}{0.550000in}}%
\pgfpathlineto{\pgfqpoint{3.611328in}{3.017729in}}%
\pgfpathlineto{\pgfqpoint{3.583807in}{3.017729in}}%
\pgfpathlineto{\pgfqpoint{3.583807in}{0.550000in}}%
\pgfpathclose%
\pgfusepath{fill}%
\end{pgfscope}%
\begin{pgfscope}%
\pgfpathrectangle{\pgfqpoint{1.250000in}{0.550000in}}{\pgfqpoint{7.750000in}{3.850000in}}%
\pgfusepath{clip}%
\pgfsetbuttcap%
\pgfsetmiterjoin%
\definecolor{currentfill}{rgb}{0.501961,0.501961,0.501961}%
\pgfsetfillcolor{currentfill}%
\pgfsetfillopacity{0.700000}%
\pgfsetlinewidth{0.000000pt}%
\definecolor{currentstroke}{rgb}{0.000000,0.000000,0.000000}%
\pgfsetstrokecolor{currentstroke}%
\pgfsetstrokeopacity{0.700000}%
\pgfsetdash{}{0pt}%
\pgfpathmoveto{\pgfqpoint{3.611328in}{0.550000in}}%
\pgfpathlineto{\pgfqpoint{3.638849in}{0.550000in}}%
\pgfpathlineto{\pgfqpoint{3.638849in}{2.993494in}}%
\pgfpathlineto{\pgfqpoint{3.611328in}{2.993494in}}%
\pgfpathlineto{\pgfqpoint{3.611328in}{0.550000in}}%
\pgfpathclose%
\pgfusepath{fill}%
\end{pgfscope}%
\begin{pgfscope}%
\pgfpathrectangle{\pgfqpoint{1.250000in}{0.550000in}}{\pgfqpoint{7.750000in}{3.850000in}}%
\pgfusepath{clip}%
\pgfsetbuttcap%
\pgfsetmiterjoin%
\definecolor{currentfill}{rgb}{0.501961,0.501961,0.501961}%
\pgfsetfillcolor{currentfill}%
\pgfsetfillopacity{0.700000}%
\pgfsetlinewidth{0.000000pt}%
\definecolor{currentstroke}{rgb}{0.000000,0.000000,0.000000}%
\pgfsetstrokecolor{currentstroke}%
\pgfsetstrokeopacity{0.700000}%
\pgfsetdash{}{0pt}%
\pgfpathmoveto{\pgfqpoint{3.638849in}{0.550000in}}%
\pgfpathlineto{\pgfqpoint{3.666371in}{0.550000in}}%
\pgfpathlineto{\pgfqpoint{3.666371in}{2.900829in}}%
\pgfpathlineto{\pgfqpoint{3.638849in}{2.900829in}}%
\pgfpathlineto{\pgfqpoint{3.638849in}{0.550000in}}%
\pgfpathclose%
\pgfusepath{fill}%
\end{pgfscope}%
\begin{pgfscope}%
\pgfpathrectangle{\pgfqpoint{1.250000in}{0.550000in}}{\pgfqpoint{7.750000in}{3.850000in}}%
\pgfusepath{clip}%
\pgfsetbuttcap%
\pgfsetmiterjoin%
\definecolor{currentfill}{rgb}{0.501961,0.501961,0.501961}%
\pgfsetfillcolor{currentfill}%
\pgfsetfillopacity{0.700000}%
\pgfsetlinewidth{0.000000pt}%
\definecolor{currentstroke}{rgb}{0.000000,0.000000,0.000000}%
\pgfsetstrokecolor{currentstroke}%
\pgfsetstrokeopacity{0.700000}%
\pgfsetdash{}{0pt}%
\pgfpathmoveto{\pgfqpoint{3.666371in}{0.550000in}}%
\pgfpathlineto{\pgfqpoint{3.693892in}{0.550000in}}%
\pgfpathlineto{\pgfqpoint{3.693892in}{2.853784in}}%
\pgfpathlineto{\pgfqpoint{3.666371in}{2.853784in}}%
\pgfpathlineto{\pgfqpoint{3.666371in}{0.550000in}}%
\pgfpathclose%
\pgfusepath{fill}%
\end{pgfscope}%
\begin{pgfscope}%
\pgfpathrectangle{\pgfqpoint{1.250000in}{0.550000in}}{\pgfqpoint{7.750000in}{3.850000in}}%
\pgfusepath{clip}%
\pgfsetbuttcap%
\pgfsetmiterjoin%
\definecolor{currentfill}{rgb}{0.501961,0.501961,0.501961}%
\pgfsetfillcolor{currentfill}%
\pgfsetfillopacity{0.700000}%
\pgfsetlinewidth{0.000000pt}%
\definecolor{currentstroke}{rgb}{0.000000,0.000000,0.000000}%
\pgfsetstrokecolor{currentstroke}%
\pgfsetstrokeopacity{0.700000}%
\pgfsetdash{}{0pt}%
\pgfpathmoveto{\pgfqpoint{3.693892in}{0.550000in}}%
\pgfpathlineto{\pgfqpoint{3.721413in}{0.550000in}}%
\pgfpathlineto{\pgfqpoint{3.721413in}{2.791058in}}%
\pgfpathlineto{\pgfqpoint{3.693892in}{2.791058in}}%
\pgfpathlineto{\pgfqpoint{3.693892in}{0.550000in}}%
\pgfpathclose%
\pgfusepath{fill}%
\end{pgfscope}%
\begin{pgfscope}%
\pgfpathrectangle{\pgfqpoint{1.250000in}{0.550000in}}{\pgfqpoint{7.750000in}{3.850000in}}%
\pgfusepath{clip}%
\pgfsetbuttcap%
\pgfsetmiterjoin%
\definecolor{currentfill}{rgb}{0.501961,0.501961,0.501961}%
\pgfsetfillcolor{currentfill}%
\pgfsetfillopacity{0.700000}%
\pgfsetlinewidth{0.000000pt}%
\definecolor{currentstroke}{rgb}{0.000000,0.000000,0.000000}%
\pgfsetstrokecolor{currentstroke}%
\pgfsetstrokeopacity{0.700000}%
\pgfsetdash{}{0pt}%
\pgfpathmoveto{\pgfqpoint{3.721413in}{0.550000in}}%
\pgfpathlineto{\pgfqpoint{3.748935in}{0.550000in}}%
\pgfpathlineto{\pgfqpoint{3.748935in}{2.816719in}}%
\pgfpathlineto{\pgfqpoint{3.721413in}{2.816719in}}%
\pgfpathlineto{\pgfqpoint{3.721413in}{0.550000in}}%
\pgfpathclose%
\pgfusepath{fill}%
\end{pgfscope}%
\begin{pgfscope}%
\pgfpathrectangle{\pgfqpoint{1.250000in}{0.550000in}}{\pgfqpoint{7.750000in}{3.850000in}}%
\pgfusepath{clip}%
\pgfsetbuttcap%
\pgfsetmiterjoin%
\definecolor{currentfill}{rgb}{0.501961,0.501961,0.501961}%
\pgfsetfillcolor{currentfill}%
\pgfsetfillopacity{0.700000}%
\pgfsetlinewidth{0.000000pt}%
\definecolor{currentstroke}{rgb}{0.000000,0.000000,0.000000}%
\pgfsetstrokecolor{currentstroke}%
\pgfsetstrokeopacity{0.700000}%
\pgfsetdash{}{0pt}%
\pgfpathmoveto{\pgfqpoint{3.748935in}{0.550000in}}%
\pgfpathlineto{\pgfqpoint{3.776456in}{0.550000in}}%
\pgfpathlineto{\pgfqpoint{3.776456in}{0.550000in}}%
\pgfpathlineto{\pgfqpoint{3.748935in}{0.550000in}}%
\pgfpathlineto{\pgfqpoint{3.748935in}{0.550000in}}%
\pgfpathclose%
\pgfusepath{fill}%
\end{pgfscope}%
\begin{pgfscope}%
\pgfpathrectangle{\pgfqpoint{1.250000in}{0.550000in}}{\pgfqpoint{7.750000in}{3.850000in}}%
\pgfusepath{clip}%
\pgfsetbuttcap%
\pgfsetmiterjoin%
\definecolor{currentfill}{rgb}{0.501961,0.501961,0.501961}%
\pgfsetfillcolor{currentfill}%
\pgfsetfillopacity{0.700000}%
\pgfsetlinewidth{0.000000pt}%
\definecolor{currentstroke}{rgb}{0.000000,0.000000,0.000000}%
\pgfsetstrokecolor{currentstroke}%
\pgfsetstrokeopacity{0.700000}%
\pgfsetdash{}{0pt}%
\pgfpathmoveto{\pgfqpoint{3.776456in}{0.550000in}}%
\pgfpathlineto{\pgfqpoint{3.803977in}{0.550000in}}%
\pgfpathlineto{\pgfqpoint{3.803977in}{2.843805in}}%
\pgfpathlineto{\pgfqpoint{3.776456in}{2.843805in}}%
\pgfpathlineto{\pgfqpoint{3.776456in}{0.550000in}}%
\pgfpathclose%
\pgfusepath{fill}%
\end{pgfscope}%
\begin{pgfscope}%
\pgfpathrectangle{\pgfqpoint{1.250000in}{0.550000in}}{\pgfqpoint{7.750000in}{3.850000in}}%
\pgfusepath{clip}%
\pgfsetbuttcap%
\pgfsetmiterjoin%
\definecolor{currentfill}{rgb}{0.501961,0.501961,0.501961}%
\pgfsetfillcolor{currentfill}%
\pgfsetfillopacity{0.700000}%
\pgfsetlinewidth{0.000000pt}%
\definecolor{currentstroke}{rgb}{0.000000,0.000000,0.000000}%
\pgfsetstrokecolor{currentstroke}%
\pgfsetstrokeopacity{0.700000}%
\pgfsetdash{}{0pt}%
\pgfpathmoveto{\pgfqpoint{3.803977in}{0.550000in}}%
\pgfpathlineto{\pgfqpoint{3.831499in}{0.550000in}}%
\pgfpathlineto{\pgfqpoint{3.831499in}{2.785355in}}%
\pgfpathlineto{\pgfqpoint{3.803977in}{2.785355in}}%
\pgfpathlineto{\pgfqpoint{3.803977in}{0.550000in}}%
\pgfpathclose%
\pgfusepath{fill}%
\end{pgfscope}%
\begin{pgfscope}%
\pgfpathrectangle{\pgfqpoint{1.250000in}{0.550000in}}{\pgfqpoint{7.750000in}{3.850000in}}%
\pgfusepath{clip}%
\pgfsetbuttcap%
\pgfsetmiterjoin%
\definecolor{currentfill}{rgb}{0.501961,0.501961,0.501961}%
\pgfsetfillcolor{currentfill}%
\pgfsetfillopacity{0.700000}%
\pgfsetlinewidth{0.000000pt}%
\definecolor{currentstroke}{rgb}{0.000000,0.000000,0.000000}%
\pgfsetstrokecolor{currentstroke}%
\pgfsetstrokeopacity{0.700000}%
\pgfsetdash{}{0pt}%
\pgfpathmoveto{\pgfqpoint{3.831499in}{0.550000in}}%
\pgfpathlineto{\pgfqpoint{3.859020in}{0.550000in}}%
\pgfpathlineto{\pgfqpoint{3.859020in}{2.869466in}}%
\pgfpathlineto{\pgfqpoint{3.831499in}{2.869466in}}%
\pgfpathlineto{\pgfqpoint{3.831499in}{0.550000in}}%
\pgfpathclose%
\pgfusepath{fill}%
\end{pgfscope}%
\begin{pgfscope}%
\pgfpathrectangle{\pgfqpoint{1.250000in}{0.550000in}}{\pgfqpoint{7.750000in}{3.850000in}}%
\pgfusepath{clip}%
\pgfsetbuttcap%
\pgfsetmiterjoin%
\definecolor{currentfill}{rgb}{0.501961,0.501961,0.501961}%
\pgfsetfillcolor{currentfill}%
\pgfsetfillopacity{0.700000}%
\pgfsetlinewidth{0.000000pt}%
\definecolor{currentstroke}{rgb}{0.000000,0.000000,0.000000}%
\pgfsetstrokecolor{currentstroke}%
\pgfsetstrokeopacity{0.700000}%
\pgfsetdash{}{0pt}%
\pgfpathmoveto{\pgfqpoint{3.859020in}{0.550000in}}%
\pgfpathlineto{\pgfqpoint{3.886541in}{0.550000in}}%
\pgfpathlineto{\pgfqpoint{3.886541in}{2.759694in}}%
\pgfpathlineto{\pgfqpoint{3.859020in}{2.759694in}}%
\pgfpathlineto{\pgfqpoint{3.859020in}{0.550000in}}%
\pgfpathclose%
\pgfusepath{fill}%
\end{pgfscope}%
\begin{pgfscope}%
\pgfpathrectangle{\pgfqpoint{1.250000in}{0.550000in}}{\pgfqpoint{7.750000in}{3.850000in}}%
\pgfusepath{clip}%
\pgfsetbuttcap%
\pgfsetmiterjoin%
\definecolor{currentfill}{rgb}{0.501961,0.501961,0.501961}%
\pgfsetfillcolor{currentfill}%
\pgfsetfillopacity{0.700000}%
\pgfsetlinewidth{0.000000pt}%
\definecolor{currentstroke}{rgb}{0.000000,0.000000,0.000000}%
\pgfsetstrokecolor{currentstroke}%
\pgfsetstrokeopacity{0.700000}%
\pgfsetdash{}{0pt}%
\pgfpathmoveto{\pgfqpoint{3.886541in}{0.550000in}}%
\pgfpathlineto{\pgfqpoint{3.914062in}{0.550000in}}%
\pgfpathlineto{\pgfqpoint{3.914062in}{2.836677in}}%
\pgfpathlineto{\pgfqpoint{3.886541in}{2.836677in}}%
\pgfpathlineto{\pgfqpoint{3.886541in}{0.550000in}}%
\pgfpathclose%
\pgfusepath{fill}%
\end{pgfscope}%
\begin{pgfscope}%
\pgfpathrectangle{\pgfqpoint{1.250000in}{0.550000in}}{\pgfqpoint{7.750000in}{3.850000in}}%
\pgfusepath{clip}%
\pgfsetbuttcap%
\pgfsetmiterjoin%
\definecolor{currentfill}{rgb}{0.501961,0.501961,0.501961}%
\pgfsetfillcolor{currentfill}%
\pgfsetfillopacity{0.700000}%
\pgfsetlinewidth{0.000000pt}%
\definecolor{currentstroke}{rgb}{0.000000,0.000000,0.000000}%
\pgfsetstrokecolor{currentstroke}%
\pgfsetstrokeopacity{0.700000}%
\pgfsetdash{}{0pt}%
\pgfpathmoveto{\pgfqpoint{3.914062in}{0.550000in}}%
\pgfpathlineto{\pgfqpoint{3.941584in}{0.550000in}}%
\pgfpathlineto{\pgfqpoint{3.941584in}{2.749715in}}%
\pgfpathlineto{\pgfqpoint{3.914062in}{2.749715in}}%
\pgfpathlineto{\pgfqpoint{3.914062in}{0.550000in}}%
\pgfpathclose%
\pgfusepath{fill}%
\end{pgfscope}%
\begin{pgfscope}%
\pgfpathrectangle{\pgfqpoint{1.250000in}{0.550000in}}{\pgfqpoint{7.750000in}{3.850000in}}%
\pgfusepath{clip}%
\pgfsetbuttcap%
\pgfsetmiterjoin%
\definecolor{currentfill}{rgb}{0.501961,0.501961,0.501961}%
\pgfsetfillcolor{currentfill}%
\pgfsetfillopacity{0.700000}%
\pgfsetlinewidth{0.000000pt}%
\definecolor{currentstroke}{rgb}{0.000000,0.000000,0.000000}%
\pgfsetstrokecolor{currentstroke}%
\pgfsetstrokeopacity{0.700000}%
\pgfsetdash{}{0pt}%
\pgfpathmoveto{\pgfqpoint{3.941584in}{0.550000in}}%
\pgfpathlineto{\pgfqpoint{3.969105in}{0.550000in}}%
\pgfpathlineto{\pgfqpoint{3.969105in}{2.762545in}}%
\pgfpathlineto{\pgfqpoint{3.941584in}{2.762545in}}%
\pgfpathlineto{\pgfqpoint{3.941584in}{0.550000in}}%
\pgfpathclose%
\pgfusepath{fill}%
\end{pgfscope}%
\begin{pgfscope}%
\pgfpathrectangle{\pgfqpoint{1.250000in}{0.550000in}}{\pgfqpoint{7.750000in}{3.850000in}}%
\pgfusepath{clip}%
\pgfsetbuttcap%
\pgfsetmiterjoin%
\definecolor{currentfill}{rgb}{0.501961,0.501961,0.501961}%
\pgfsetfillcolor{currentfill}%
\pgfsetfillopacity{0.700000}%
\pgfsetlinewidth{0.000000pt}%
\definecolor{currentstroke}{rgb}{0.000000,0.000000,0.000000}%
\pgfsetstrokecolor{currentstroke}%
\pgfsetstrokeopacity{0.700000}%
\pgfsetdash{}{0pt}%
\pgfpathmoveto{\pgfqpoint{3.969105in}{0.550000in}}%
\pgfpathlineto{\pgfqpoint{3.996626in}{0.550000in}}%
\pgfpathlineto{\pgfqpoint{3.996626in}{2.719777in}}%
\pgfpathlineto{\pgfqpoint{3.969105in}{2.719777in}}%
\pgfpathlineto{\pgfqpoint{3.969105in}{0.550000in}}%
\pgfpathclose%
\pgfusepath{fill}%
\end{pgfscope}%
\begin{pgfscope}%
\pgfpathrectangle{\pgfqpoint{1.250000in}{0.550000in}}{\pgfqpoint{7.750000in}{3.850000in}}%
\pgfusepath{clip}%
\pgfsetbuttcap%
\pgfsetmiterjoin%
\definecolor{currentfill}{rgb}{0.501961,0.501961,0.501961}%
\pgfsetfillcolor{currentfill}%
\pgfsetfillopacity{0.700000}%
\pgfsetlinewidth{0.000000pt}%
\definecolor{currentstroke}{rgb}{0.000000,0.000000,0.000000}%
\pgfsetstrokecolor{currentstroke}%
\pgfsetstrokeopacity{0.700000}%
\pgfsetdash{}{0pt}%
\pgfpathmoveto{\pgfqpoint{3.996626in}{0.550000in}}%
\pgfpathlineto{\pgfqpoint{4.024148in}{0.550000in}}%
\pgfpathlineto{\pgfqpoint{4.024148in}{2.637092in}}%
\pgfpathlineto{\pgfqpoint{3.996626in}{2.637092in}}%
\pgfpathlineto{\pgfqpoint{3.996626in}{0.550000in}}%
\pgfpathclose%
\pgfusepath{fill}%
\end{pgfscope}%
\begin{pgfscope}%
\pgfpathrectangle{\pgfqpoint{1.250000in}{0.550000in}}{\pgfqpoint{7.750000in}{3.850000in}}%
\pgfusepath{clip}%
\pgfsetbuttcap%
\pgfsetmiterjoin%
\definecolor{currentfill}{rgb}{0.501961,0.501961,0.501961}%
\pgfsetfillcolor{currentfill}%
\pgfsetfillopacity{0.700000}%
\pgfsetlinewidth{0.000000pt}%
\definecolor{currentstroke}{rgb}{0.000000,0.000000,0.000000}%
\pgfsetstrokecolor{currentstroke}%
\pgfsetstrokeopacity{0.700000}%
\pgfsetdash{}{0pt}%
\pgfpathmoveto{\pgfqpoint{4.024148in}{0.550000in}}%
\pgfpathlineto{\pgfqpoint{4.051669in}{0.550000in}}%
\pgfpathlineto{\pgfqpoint{4.051669in}{2.608580in}}%
\pgfpathlineto{\pgfqpoint{4.024148in}{2.608580in}}%
\pgfpathlineto{\pgfqpoint{4.024148in}{0.550000in}}%
\pgfpathclose%
\pgfusepath{fill}%
\end{pgfscope}%
\begin{pgfscope}%
\pgfpathrectangle{\pgfqpoint{1.250000in}{0.550000in}}{\pgfqpoint{7.750000in}{3.850000in}}%
\pgfusepath{clip}%
\pgfsetbuttcap%
\pgfsetmiterjoin%
\definecolor{currentfill}{rgb}{0.501961,0.501961,0.501961}%
\pgfsetfillcolor{currentfill}%
\pgfsetfillopacity{0.700000}%
\pgfsetlinewidth{0.000000pt}%
\definecolor{currentstroke}{rgb}{0.000000,0.000000,0.000000}%
\pgfsetstrokecolor{currentstroke}%
\pgfsetstrokeopacity{0.700000}%
\pgfsetdash{}{0pt}%
\pgfpathmoveto{\pgfqpoint{4.051669in}{0.550000in}}%
\pgfpathlineto{\pgfqpoint{4.079190in}{0.550000in}}%
\pgfpathlineto{\pgfqpoint{4.079190in}{2.685562in}}%
\pgfpathlineto{\pgfqpoint{4.051669in}{2.685562in}}%
\pgfpathlineto{\pgfqpoint{4.051669in}{0.550000in}}%
\pgfpathclose%
\pgfusepath{fill}%
\end{pgfscope}%
\begin{pgfscope}%
\pgfpathrectangle{\pgfqpoint{1.250000in}{0.550000in}}{\pgfqpoint{7.750000in}{3.850000in}}%
\pgfusepath{clip}%
\pgfsetbuttcap%
\pgfsetmiterjoin%
\definecolor{currentfill}{rgb}{0.501961,0.501961,0.501961}%
\pgfsetfillcolor{currentfill}%
\pgfsetfillopacity{0.700000}%
\pgfsetlinewidth{0.000000pt}%
\definecolor{currentstroke}{rgb}{0.000000,0.000000,0.000000}%
\pgfsetstrokecolor{currentstroke}%
\pgfsetstrokeopacity{0.700000}%
\pgfsetdash{}{0pt}%
\pgfpathmoveto{\pgfqpoint{4.079190in}{0.550000in}}%
\pgfpathlineto{\pgfqpoint{4.106712in}{0.550000in}}%
\pgfpathlineto{\pgfqpoint{4.106712in}{2.578642in}}%
\pgfpathlineto{\pgfqpoint{4.079190in}{2.578642in}}%
\pgfpathlineto{\pgfqpoint{4.079190in}{0.550000in}}%
\pgfpathclose%
\pgfusepath{fill}%
\end{pgfscope}%
\begin{pgfscope}%
\pgfpathrectangle{\pgfqpoint{1.250000in}{0.550000in}}{\pgfqpoint{7.750000in}{3.850000in}}%
\pgfusepath{clip}%
\pgfsetbuttcap%
\pgfsetmiterjoin%
\definecolor{currentfill}{rgb}{0.501961,0.501961,0.501961}%
\pgfsetfillcolor{currentfill}%
\pgfsetfillopacity{0.700000}%
\pgfsetlinewidth{0.000000pt}%
\definecolor{currentstroke}{rgb}{0.000000,0.000000,0.000000}%
\pgfsetstrokecolor{currentstroke}%
\pgfsetstrokeopacity{0.700000}%
\pgfsetdash{}{0pt}%
\pgfpathmoveto{\pgfqpoint{4.106712in}{0.550000in}}%
\pgfpathlineto{\pgfqpoint{4.134233in}{0.550000in}}%
\pgfpathlineto{\pgfqpoint{4.134233in}{2.448911in}}%
\pgfpathlineto{\pgfqpoint{4.106712in}{2.448911in}}%
\pgfpathlineto{\pgfqpoint{4.106712in}{0.550000in}}%
\pgfpathclose%
\pgfusepath{fill}%
\end{pgfscope}%
\begin{pgfscope}%
\pgfpathrectangle{\pgfqpoint{1.250000in}{0.550000in}}{\pgfqpoint{7.750000in}{3.850000in}}%
\pgfusepath{clip}%
\pgfsetbuttcap%
\pgfsetmiterjoin%
\definecolor{currentfill}{rgb}{0.501961,0.501961,0.501961}%
\pgfsetfillcolor{currentfill}%
\pgfsetfillopacity{0.700000}%
\pgfsetlinewidth{0.000000pt}%
\definecolor{currentstroke}{rgb}{0.000000,0.000000,0.000000}%
\pgfsetstrokecolor{currentstroke}%
\pgfsetstrokeopacity{0.700000}%
\pgfsetdash{}{0pt}%
\pgfpathmoveto{\pgfqpoint{4.134233in}{0.550000in}}%
\pgfpathlineto{\pgfqpoint{4.161754in}{0.550000in}}%
\pgfpathlineto{\pgfqpoint{4.161754in}{4.216667in}}%
\pgfpathlineto{\pgfqpoint{4.134233in}{4.216667in}}%
\pgfpathlineto{\pgfqpoint{4.134233in}{0.550000in}}%
\pgfpathclose%
\pgfusepath{fill}%
\end{pgfscope}%
\begin{pgfscope}%
\pgfpathrectangle{\pgfqpoint{1.250000in}{0.550000in}}{\pgfqpoint{7.750000in}{3.850000in}}%
\pgfusepath{clip}%
\pgfsetbuttcap%
\pgfsetmiterjoin%
\definecolor{currentfill}{rgb}{0.501961,0.501961,0.501961}%
\pgfsetfillcolor{currentfill}%
\pgfsetfillopacity{0.700000}%
\pgfsetlinewidth{0.000000pt}%
\definecolor{currentstroke}{rgb}{0.000000,0.000000,0.000000}%
\pgfsetstrokecolor{currentstroke}%
\pgfsetstrokeopacity{0.700000}%
\pgfsetdash{}{0pt}%
\pgfpathmoveto{\pgfqpoint{4.161754in}{0.550000in}}%
\pgfpathlineto{\pgfqpoint{4.189276in}{0.550000in}}%
\pgfpathlineto{\pgfqpoint{4.189276in}{2.441783in}}%
\pgfpathlineto{\pgfqpoint{4.161754in}{2.441783in}}%
\pgfpathlineto{\pgfqpoint{4.161754in}{0.550000in}}%
\pgfpathclose%
\pgfusepath{fill}%
\end{pgfscope}%
\begin{pgfscope}%
\pgfpathrectangle{\pgfqpoint{1.250000in}{0.550000in}}{\pgfqpoint{7.750000in}{3.850000in}}%
\pgfusepath{clip}%
\pgfsetbuttcap%
\pgfsetmiterjoin%
\definecolor{currentfill}{rgb}{0.501961,0.501961,0.501961}%
\pgfsetfillcolor{currentfill}%
\pgfsetfillopacity{0.700000}%
\pgfsetlinewidth{0.000000pt}%
\definecolor{currentstroke}{rgb}{0.000000,0.000000,0.000000}%
\pgfsetstrokecolor{currentstroke}%
\pgfsetstrokeopacity{0.700000}%
\pgfsetdash{}{0pt}%
\pgfpathmoveto{\pgfqpoint{4.189276in}{0.550000in}}%
\pgfpathlineto{\pgfqpoint{4.216797in}{0.550000in}}%
\pgfpathlineto{\pgfqpoint{4.216797in}{2.304925in}}%
\pgfpathlineto{\pgfqpoint{4.189276in}{2.304925in}}%
\pgfpathlineto{\pgfqpoint{4.189276in}{0.550000in}}%
\pgfpathclose%
\pgfusepath{fill}%
\end{pgfscope}%
\begin{pgfscope}%
\pgfpathrectangle{\pgfqpoint{1.250000in}{0.550000in}}{\pgfqpoint{7.750000in}{3.850000in}}%
\pgfusepath{clip}%
\pgfsetbuttcap%
\pgfsetmiterjoin%
\definecolor{currentfill}{rgb}{0.501961,0.501961,0.501961}%
\pgfsetfillcolor{currentfill}%
\pgfsetfillopacity{0.700000}%
\pgfsetlinewidth{0.000000pt}%
\definecolor{currentstroke}{rgb}{0.000000,0.000000,0.000000}%
\pgfsetstrokecolor{currentstroke}%
\pgfsetstrokeopacity{0.700000}%
\pgfsetdash{}{0pt}%
\pgfpathmoveto{\pgfqpoint{4.216797in}{0.550000in}}%
\pgfpathlineto{\pgfqpoint{4.244318in}{0.550000in}}%
\pgfpathlineto{\pgfqpoint{4.244318in}{2.282115in}}%
\pgfpathlineto{\pgfqpoint{4.216797in}{2.282115in}}%
\pgfpathlineto{\pgfqpoint{4.216797in}{0.550000in}}%
\pgfpathclose%
\pgfusepath{fill}%
\end{pgfscope}%
\begin{pgfscope}%
\pgfpathrectangle{\pgfqpoint{1.250000in}{0.550000in}}{\pgfqpoint{7.750000in}{3.850000in}}%
\pgfusepath{clip}%
\pgfsetbuttcap%
\pgfsetmiterjoin%
\definecolor{currentfill}{rgb}{0.501961,0.501961,0.501961}%
\pgfsetfillcolor{currentfill}%
\pgfsetfillopacity{0.700000}%
\pgfsetlinewidth{0.000000pt}%
\definecolor{currentstroke}{rgb}{0.000000,0.000000,0.000000}%
\pgfsetstrokecolor{currentstroke}%
\pgfsetstrokeopacity{0.700000}%
\pgfsetdash{}{0pt}%
\pgfpathmoveto{\pgfqpoint{4.244318in}{0.550000in}}%
\pgfpathlineto{\pgfqpoint{4.271839in}{0.550000in}}%
\pgfpathlineto{\pgfqpoint{4.271839in}{0.550000in}}%
\pgfpathlineto{\pgfqpoint{4.244318in}{0.550000in}}%
\pgfpathlineto{\pgfqpoint{4.244318in}{0.550000in}}%
\pgfpathclose%
\pgfusepath{fill}%
\end{pgfscope}%
\begin{pgfscope}%
\pgfpathrectangle{\pgfqpoint{1.250000in}{0.550000in}}{\pgfqpoint{7.750000in}{3.850000in}}%
\pgfusepath{clip}%
\pgfsetbuttcap%
\pgfsetmiterjoin%
\definecolor{currentfill}{rgb}{0.501961,0.501961,0.501961}%
\pgfsetfillcolor{currentfill}%
\pgfsetfillopacity{0.700000}%
\pgfsetlinewidth{0.000000pt}%
\definecolor{currentstroke}{rgb}{0.000000,0.000000,0.000000}%
\pgfsetstrokecolor{currentstroke}%
\pgfsetstrokeopacity{0.700000}%
\pgfsetdash{}{0pt}%
\pgfpathmoveto{\pgfqpoint{4.271839in}{0.550000in}}%
\pgfpathlineto{\pgfqpoint{4.299361in}{0.550000in}}%
\pgfpathlineto{\pgfqpoint{4.299361in}{2.289243in}}%
\pgfpathlineto{\pgfqpoint{4.271839in}{2.289243in}}%
\pgfpathlineto{\pgfqpoint{4.271839in}{0.550000in}}%
\pgfpathclose%
\pgfusepath{fill}%
\end{pgfscope}%
\begin{pgfscope}%
\pgfpathrectangle{\pgfqpoint{1.250000in}{0.550000in}}{\pgfqpoint{7.750000in}{3.850000in}}%
\pgfusepath{clip}%
\pgfsetbuttcap%
\pgfsetmiterjoin%
\definecolor{currentfill}{rgb}{0.501961,0.501961,0.501961}%
\pgfsetfillcolor{currentfill}%
\pgfsetfillopacity{0.700000}%
\pgfsetlinewidth{0.000000pt}%
\definecolor{currentstroke}{rgb}{0.000000,0.000000,0.000000}%
\pgfsetstrokecolor{currentstroke}%
\pgfsetstrokeopacity{0.700000}%
\pgfsetdash{}{0pt}%
\pgfpathmoveto{\pgfqpoint{4.299361in}{0.550000in}}%
\pgfpathlineto{\pgfqpoint{4.326882in}{0.550000in}}%
\pgfpathlineto{\pgfqpoint{4.326882in}{0.550000in}}%
\pgfpathlineto{\pgfqpoint{4.299361in}{0.550000in}}%
\pgfpathlineto{\pgfqpoint{4.299361in}{0.550000in}}%
\pgfpathclose%
\pgfusepath{fill}%
\end{pgfscope}%
\begin{pgfscope}%
\pgfpathrectangle{\pgfqpoint{1.250000in}{0.550000in}}{\pgfqpoint{7.750000in}{3.850000in}}%
\pgfusepath{clip}%
\pgfsetbuttcap%
\pgfsetmiterjoin%
\definecolor{currentfill}{rgb}{0.501961,0.501961,0.501961}%
\pgfsetfillcolor{currentfill}%
\pgfsetfillopacity{0.700000}%
\pgfsetlinewidth{0.000000pt}%
\definecolor{currentstroke}{rgb}{0.000000,0.000000,0.000000}%
\pgfsetstrokecolor{currentstroke}%
\pgfsetstrokeopacity{0.700000}%
\pgfsetdash{}{0pt}%
\pgfpathmoveto{\pgfqpoint{4.326882in}{0.550000in}}%
\pgfpathlineto{\pgfqpoint{4.354403in}{0.550000in}}%
\pgfpathlineto{\pgfqpoint{4.354403in}{2.129575in}}%
\pgfpathlineto{\pgfqpoint{4.326882in}{2.129575in}}%
\pgfpathlineto{\pgfqpoint{4.326882in}{0.550000in}}%
\pgfpathclose%
\pgfusepath{fill}%
\end{pgfscope}%
\begin{pgfscope}%
\pgfpathrectangle{\pgfqpoint{1.250000in}{0.550000in}}{\pgfqpoint{7.750000in}{3.850000in}}%
\pgfusepath{clip}%
\pgfsetbuttcap%
\pgfsetmiterjoin%
\definecolor{currentfill}{rgb}{0.501961,0.501961,0.501961}%
\pgfsetfillcolor{currentfill}%
\pgfsetfillopacity{0.700000}%
\pgfsetlinewidth{0.000000pt}%
\definecolor{currentstroke}{rgb}{0.000000,0.000000,0.000000}%
\pgfsetstrokecolor{currentstroke}%
\pgfsetstrokeopacity{0.700000}%
\pgfsetdash{}{0pt}%
\pgfpathmoveto{\pgfqpoint{4.354403in}{0.550000in}}%
\pgfpathlineto{\pgfqpoint{4.381925in}{0.550000in}}%
\pgfpathlineto{\pgfqpoint{4.381925in}{0.550000in}}%
\pgfpathlineto{\pgfqpoint{4.354403in}{0.550000in}}%
\pgfpathlineto{\pgfqpoint{4.354403in}{0.550000in}}%
\pgfpathclose%
\pgfusepath{fill}%
\end{pgfscope}%
\begin{pgfscope}%
\pgfpathrectangle{\pgfqpoint{1.250000in}{0.550000in}}{\pgfqpoint{7.750000in}{3.850000in}}%
\pgfusepath{clip}%
\pgfsetbuttcap%
\pgfsetmiterjoin%
\definecolor{currentfill}{rgb}{0.501961,0.501961,0.501961}%
\pgfsetfillcolor{currentfill}%
\pgfsetfillopacity{0.700000}%
\pgfsetlinewidth{0.000000pt}%
\definecolor{currentstroke}{rgb}{0.000000,0.000000,0.000000}%
\pgfsetstrokecolor{currentstroke}%
\pgfsetstrokeopacity{0.700000}%
\pgfsetdash{}{0pt}%
\pgfpathmoveto{\pgfqpoint{4.381925in}{0.550000in}}%
\pgfpathlineto{\pgfqpoint{4.409446in}{0.550000in}}%
\pgfpathlineto{\pgfqpoint{4.409446in}{2.280689in}}%
\pgfpathlineto{\pgfqpoint{4.381925in}{2.280689in}}%
\pgfpathlineto{\pgfqpoint{4.381925in}{0.550000in}}%
\pgfpathclose%
\pgfusepath{fill}%
\end{pgfscope}%
\begin{pgfscope}%
\pgfpathrectangle{\pgfqpoint{1.250000in}{0.550000in}}{\pgfqpoint{7.750000in}{3.850000in}}%
\pgfusepath{clip}%
\pgfsetbuttcap%
\pgfsetmiterjoin%
\definecolor{currentfill}{rgb}{0.501961,0.501961,0.501961}%
\pgfsetfillcolor{currentfill}%
\pgfsetfillopacity{0.700000}%
\pgfsetlinewidth{0.000000pt}%
\definecolor{currentstroke}{rgb}{0.000000,0.000000,0.000000}%
\pgfsetstrokecolor{currentstroke}%
\pgfsetstrokeopacity{0.700000}%
\pgfsetdash{}{0pt}%
\pgfpathmoveto{\pgfqpoint{4.409446in}{0.550000in}}%
\pgfpathlineto{\pgfqpoint{4.436967in}{0.550000in}}%
\pgfpathlineto{\pgfqpoint{4.436967in}{0.550000in}}%
\pgfpathlineto{\pgfqpoint{4.409446in}{0.550000in}}%
\pgfpathlineto{\pgfqpoint{4.409446in}{0.550000in}}%
\pgfpathclose%
\pgfusepath{fill}%
\end{pgfscope}%
\begin{pgfscope}%
\pgfpathrectangle{\pgfqpoint{1.250000in}{0.550000in}}{\pgfqpoint{7.750000in}{3.850000in}}%
\pgfusepath{clip}%
\pgfsetbuttcap%
\pgfsetmiterjoin%
\definecolor{currentfill}{rgb}{0.501961,0.501961,0.501961}%
\pgfsetfillcolor{currentfill}%
\pgfsetfillopacity{0.700000}%
\pgfsetlinewidth{0.000000pt}%
\definecolor{currentstroke}{rgb}{0.000000,0.000000,0.000000}%
\pgfsetstrokecolor{currentstroke}%
\pgfsetstrokeopacity{0.700000}%
\pgfsetdash{}{0pt}%
\pgfpathmoveto{\pgfqpoint{4.436967in}{0.550000in}}%
\pgfpathlineto{\pgfqpoint{4.464489in}{0.550000in}}%
\pgfpathlineto{\pgfqpoint{4.464489in}{2.073976in}}%
\pgfpathlineto{\pgfqpoint{4.436967in}{2.073976in}}%
\pgfpathlineto{\pgfqpoint{4.436967in}{0.550000in}}%
\pgfpathclose%
\pgfusepath{fill}%
\end{pgfscope}%
\begin{pgfscope}%
\pgfpathrectangle{\pgfqpoint{1.250000in}{0.550000in}}{\pgfqpoint{7.750000in}{3.850000in}}%
\pgfusepath{clip}%
\pgfsetbuttcap%
\pgfsetmiterjoin%
\definecolor{currentfill}{rgb}{0.501961,0.501961,0.501961}%
\pgfsetfillcolor{currentfill}%
\pgfsetfillopacity{0.700000}%
\pgfsetlinewidth{0.000000pt}%
\definecolor{currentstroke}{rgb}{0.000000,0.000000,0.000000}%
\pgfsetstrokecolor{currentstroke}%
\pgfsetstrokeopacity{0.700000}%
\pgfsetdash{}{0pt}%
\pgfpathmoveto{\pgfqpoint{4.464489in}{0.550000in}}%
\pgfpathlineto{\pgfqpoint{4.492010in}{0.550000in}}%
\pgfpathlineto{\pgfqpoint{4.492010in}{0.550000in}}%
\pgfpathlineto{\pgfqpoint{4.464489in}{0.550000in}}%
\pgfpathlineto{\pgfqpoint{4.464489in}{0.550000in}}%
\pgfpathclose%
\pgfusepath{fill}%
\end{pgfscope}%
\begin{pgfscope}%
\pgfpathrectangle{\pgfqpoint{1.250000in}{0.550000in}}{\pgfqpoint{7.750000in}{3.850000in}}%
\pgfusepath{clip}%
\pgfsetbuttcap%
\pgfsetmiterjoin%
\definecolor{currentfill}{rgb}{0.501961,0.501961,0.501961}%
\pgfsetfillcolor{currentfill}%
\pgfsetfillopacity{0.700000}%
\pgfsetlinewidth{0.000000pt}%
\definecolor{currentstroke}{rgb}{0.000000,0.000000,0.000000}%
\pgfsetstrokecolor{currentstroke}%
\pgfsetstrokeopacity{0.700000}%
\pgfsetdash{}{0pt}%
\pgfpathmoveto{\pgfqpoint{4.492010in}{0.550000in}}%
\pgfpathlineto{\pgfqpoint{4.519531in}{0.550000in}}%
\pgfpathlineto{\pgfqpoint{4.519531in}{2.142405in}}%
\pgfpathlineto{\pgfqpoint{4.492010in}{2.142405in}}%
\pgfpathlineto{\pgfqpoint{4.492010in}{0.550000in}}%
\pgfpathclose%
\pgfusepath{fill}%
\end{pgfscope}%
\begin{pgfscope}%
\pgfpathrectangle{\pgfqpoint{1.250000in}{0.550000in}}{\pgfqpoint{7.750000in}{3.850000in}}%
\pgfusepath{clip}%
\pgfsetbuttcap%
\pgfsetmiterjoin%
\definecolor{currentfill}{rgb}{0.501961,0.501961,0.501961}%
\pgfsetfillcolor{currentfill}%
\pgfsetfillopacity{0.700000}%
\pgfsetlinewidth{0.000000pt}%
\definecolor{currentstroke}{rgb}{0.000000,0.000000,0.000000}%
\pgfsetstrokecolor{currentstroke}%
\pgfsetstrokeopacity{0.700000}%
\pgfsetdash{}{0pt}%
\pgfpathmoveto{\pgfqpoint{4.519531in}{0.550000in}}%
\pgfpathlineto{\pgfqpoint{4.547053in}{0.550000in}}%
\pgfpathlineto{\pgfqpoint{4.547053in}{0.550000in}}%
\pgfpathlineto{\pgfqpoint{4.519531in}{0.550000in}}%
\pgfpathlineto{\pgfqpoint{4.519531in}{0.550000in}}%
\pgfpathclose%
\pgfusepath{fill}%
\end{pgfscope}%
\begin{pgfscope}%
\pgfpathrectangle{\pgfqpoint{1.250000in}{0.550000in}}{\pgfqpoint{7.750000in}{3.850000in}}%
\pgfusepath{clip}%
\pgfsetbuttcap%
\pgfsetmiterjoin%
\definecolor{currentfill}{rgb}{0.501961,0.501961,0.501961}%
\pgfsetfillcolor{currentfill}%
\pgfsetfillopacity{0.700000}%
\pgfsetlinewidth{0.000000pt}%
\definecolor{currentstroke}{rgb}{0.000000,0.000000,0.000000}%
\pgfsetstrokecolor{currentstroke}%
\pgfsetstrokeopacity{0.700000}%
\pgfsetdash{}{0pt}%
\pgfpathmoveto{\pgfqpoint{4.547053in}{0.550000in}}%
\pgfpathlineto{\pgfqpoint{4.574574in}{0.550000in}}%
\pgfpathlineto{\pgfqpoint{4.574574in}{2.086807in}}%
\pgfpathlineto{\pgfqpoint{4.547053in}{2.086807in}}%
\pgfpathlineto{\pgfqpoint{4.547053in}{0.550000in}}%
\pgfpathclose%
\pgfusepath{fill}%
\end{pgfscope}%
\begin{pgfscope}%
\pgfpathrectangle{\pgfqpoint{1.250000in}{0.550000in}}{\pgfqpoint{7.750000in}{3.850000in}}%
\pgfusepath{clip}%
\pgfsetbuttcap%
\pgfsetmiterjoin%
\definecolor{currentfill}{rgb}{0.501961,0.501961,0.501961}%
\pgfsetfillcolor{currentfill}%
\pgfsetfillopacity{0.700000}%
\pgfsetlinewidth{0.000000pt}%
\definecolor{currentstroke}{rgb}{0.000000,0.000000,0.000000}%
\pgfsetstrokecolor{currentstroke}%
\pgfsetstrokeopacity{0.700000}%
\pgfsetdash{}{0pt}%
\pgfpathmoveto{\pgfqpoint{4.574574in}{0.550000in}}%
\pgfpathlineto{\pgfqpoint{4.602095in}{0.550000in}}%
\pgfpathlineto{\pgfqpoint{4.602095in}{0.550000in}}%
\pgfpathlineto{\pgfqpoint{4.574574in}{0.550000in}}%
\pgfpathlineto{\pgfqpoint{4.574574in}{0.550000in}}%
\pgfpathclose%
\pgfusepath{fill}%
\end{pgfscope}%
\begin{pgfscope}%
\pgfpathrectangle{\pgfqpoint{1.250000in}{0.550000in}}{\pgfqpoint{7.750000in}{3.850000in}}%
\pgfusepath{clip}%
\pgfsetbuttcap%
\pgfsetmiterjoin%
\definecolor{currentfill}{rgb}{0.501961,0.501961,0.501961}%
\pgfsetfillcolor{currentfill}%
\pgfsetfillopacity{0.700000}%
\pgfsetlinewidth{0.000000pt}%
\definecolor{currentstroke}{rgb}{0.000000,0.000000,0.000000}%
\pgfsetstrokecolor{currentstroke}%
\pgfsetstrokeopacity{0.700000}%
\pgfsetdash{}{0pt}%
\pgfpathmoveto{\pgfqpoint{4.602095in}{0.550000in}}%
\pgfpathlineto{\pgfqpoint{4.629616in}{0.550000in}}%
\pgfpathlineto{\pgfqpoint{4.629616in}{2.039762in}}%
\pgfpathlineto{\pgfqpoint{4.602095in}{2.039762in}}%
\pgfpathlineto{\pgfqpoint{4.602095in}{0.550000in}}%
\pgfpathclose%
\pgfusepath{fill}%
\end{pgfscope}%
\begin{pgfscope}%
\pgfpathrectangle{\pgfqpoint{1.250000in}{0.550000in}}{\pgfqpoint{7.750000in}{3.850000in}}%
\pgfusepath{clip}%
\pgfsetbuttcap%
\pgfsetmiterjoin%
\definecolor{currentfill}{rgb}{0.501961,0.501961,0.501961}%
\pgfsetfillcolor{currentfill}%
\pgfsetfillopacity{0.700000}%
\pgfsetlinewidth{0.000000pt}%
\definecolor{currentstroke}{rgb}{0.000000,0.000000,0.000000}%
\pgfsetstrokecolor{currentstroke}%
\pgfsetstrokeopacity{0.700000}%
\pgfsetdash{}{0pt}%
\pgfpathmoveto{\pgfqpoint{4.629616in}{0.550000in}}%
\pgfpathlineto{\pgfqpoint{4.657138in}{0.550000in}}%
\pgfpathlineto{\pgfqpoint{4.657138in}{0.550000in}}%
\pgfpathlineto{\pgfqpoint{4.629616in}{0.550000in}}%
\pgfpathlineto{\pgfqpoint{4.629616in}{0.550000in}}%
\pgfpathclose%
\pgfusepath{fill}%
\end{pgfscope}%
\begin{pgfscope}%
\pgfpathrectangle{\pgfqpoint{1.250000in}{0.550000in}}{\pgfqpoint{7.750000in}{3.850000in}}%
\pgfusepath{clip}%
\pgfsetbuttcap%
\pgfsetmiterjoin%
\definecolor{currentfill}{rgb}{0.501961,0.501961,0.501961}%
\pgfsetfillcolor{currentfill}%
\pgfsetfillopacity{0.700000}%
\pgfsetlinewidth{0.000000pt}%
\definecolor{currentstroke}{rgb}{0.000000,0.000000,0.000000}%
\pgfsetstrokecolor{currentstroke}%
\pgfsetstrokeopacity{0.700000}%
\pgfsetdash{}{0pt}%
\pgfpathmoveto{\pgfqpoint{4.657138in}{0.550000in}}%
\pgfpathlineto{\pgfqpoint{4.684659in}{0.550000in}}%
\pgfpathlineto{\pgfqpoint{4.684659in}{2.078253in}}%
\pgfpathlineto{\pgfqpoint{4.657138in}{2.078253in}}%
\pgfpathlineto{\pgfqpoint{4.657138in}{0.550000in}}%
\pgfpathclose%
\pgfusepath{fill}%
\end{pgfscope}%
\begin{pgfscope}%
\pgfpathrectangle{\pgfqpoint{1.250000in}{0.550000in}}{\pgfqpoint{7.750000in}{3.850000in}}%
\pgfusepath{clip}%
\pgfsetbuttcap%
\pgfsetmiterjoin%
\definecolor{currentfill}{rgb}{0.501961,0.501961,0.501961}%
\pgfsetfillcolor{currentfill}%
\pgfsetfillopacity{0.700000}%
\pgfsetlinewidth{0.000000pt}%
\definecolor{currentstroke}{rgb}{0.000000,0.000000,0.000000}%
\pgfsetstrokecolor{currentstroke}%
\pgfsetstrokeopacity{0.700000}%
\pgfsetdash{}{0pt}%
\pgfpathmoveto{\pgfqpoint{4.684659in}{0.550000in}}%
\pgfpathlineto{\pgfqpoint{4.712180in}{0.550000in}}%
\pgfpathlineto{\pgfqpoint{4.712180in}{0.550000in}}%
\pgfpathlineto{\pgfqpoint{4.684659in}{0.550000in}}%
\pgfpathlineto{\pgfqpoint{4.684659in}{0.550000in}}%
\pgfpathclose%
\pgfusepath{fill}%
\end{pgfscope}%
\begin{pgfscope}%
\pgfpathrectangle{\pgfqpoint{1.250000in}{0.550000in}}{\pgfqpoint{7.750000in}{3.850000in}}%
\pgfusepath{clip}%
\pgfsetbuttcap%
\pgfsetmiterjoin%
\definecolor{currentfill}{rgb}{0.501961,0.501961,0.501961}%
\pgfsetfillcolor{currentfill}%
\pgfsetfillopacity{0.700000}%
\pgfsetlinewidth{0.000000pt}%
\definecolor{currentstroke}{rgb}{0.000000,0.000000,0.000000}%
\pgfsetstrokecolor{currentstroke}%
\pgfsetstrokeopacity{0.700000}%
\pgfsetdash{}{0pt}%
\pgfpathmoveto{\pgfqpoint{4.712180in}{0.550000in}}%
\pgfpathlineto{\pgfqpoint{4.739702in}{0.550000in}}%
\pgfpathlineto{\pgfqpoint{4.739702in}{2.098212in}}%
\pgfpathlineto{\pgfqpoint{4.712180in}{2.098212in}}%
\pgfpathlineto{\pgfqpoint{4.712180in}{0.550000in}}%
\pgfpathclose%
\pgfusepath{fill}%
\end{pgfscope}%
\begin{pgfscope}%
\pgfpathrectangle{\pgfqpoint{1.250000in}{0.550000in}}{\pgfqpoint{7.750000in}{3.850000in}}%
\pgfusepath{clip}%
\pgfsetbuttcap%
\pgfsetmiterjoin%
\definecolor{currentfill}{rgb}{0.501961,0.501961,0.501961}%
\pgfsetfillcolor{currentfill}%
\pgfsetfillopacity{0.700000}%
\pgfsetlinewidth{0.000000pt}%
\definecolor{currentstroke}{rgb}{0.000000,0.000000,0.000000}%
\pgfsetstrokecolor{currentstroke}%
\pgfsetstrokeopacity{0.700000}%
\pgfsetdash{}{0pt}%
\pgfpathmoveto{\pgfqpoint{4.739702in}{0.550000in}}%
\pgfpathlineto{\pgfqpoint{4.767223in}{0.550000in}}%
\pgfpathlineto{\pgfqpoint{4.767223in}{0.550000in}}%
\pgfpathlineto{\pgfqpoint{4.739702in}{0.550000in}}%
\pgfpathlineto{\pgfqpoint{4.739702in}{0.550000in}}%
\pgfpathclose%
\pgfusepath{fill}%
\end{pgfscope}%
\begin{pgfscope}%
\pgfpathrectangle{\pgfqpoint{1.250000in}{0.550000in}}{\pgfqpoint{7.750000in}{3.850000in}}%
\pgfusepath{clip}%
\pgfsetbuttcap%
\pgfsetmiterjoin%
\definecolor{currentfill}{rgb}{0.501961,0.501961,0.501961}%
\pgfsetfillcolor{currentfill}%
\pgfsetfillopacity{0.700000}%
\pgfsetlinewidth{0.000000pt}%
\definecolor{currentstroke}{rgb}{0.000000,0.000000,0.000000}%
\pgfsetstrokecolor{currentstroke}%
\pgfsetstrokeopacity{0.700000}%
\pgfsetdash{}{0pt}%
\pgfpathmoveto{\pgfqpoint{4.767223in}{0.550000in}}%
\pgfpathlineto{\pgfqpoint{4.794744in}{0.550000in}}%
\pgfpathlineto{\pgfqpoint{4.794744in}{1.937118in}}%
\pgfpathlineto{\pgfqpoint{4.767223in}{1.937118in}}%
\pgfpathlineto{\pgfqpoint{4.767223in}{0.550000in}}%
\pgfpathclose%
\pgfusepath{fill}%
\end{pgfscope}%
\begin{pgfscope}%
\pgfpathrectangle{\pgfqpoint{1.250000in}{0.550000in}}{\pgfqpoint{7.750000in}{3.850000in}}%
\pgfusepath{clip}%
\pgfsetbuttcap%
\pgfsetmiterjoin%
\definecolor{currentfill}{rgb}{0.501961,0.501961,0.501961}%
\pgfsetfillcolor{currentfill}%
\pgfsetfillopacity{0.700000}%
\pgfsetlinewidth{0.000000pt}%
\definecolor{currentstroke}{rgb}{0.000000,0.000000,0.000000}%
\pgfsetstrokecolor{currentstroke}%
\pgfsetstrokeopacity{0.700000}%
\pgfsetdash{}{0pt}%
\pgfpathmoveto{\pgfqpoint{4.794744in}{0.550000in}}%
\pgfpathlineto{\pgfqpoint{4.822266in}{0.550000in}}%
\pgfpathlineto{\pgfqpoint{4.822266in}{0.550000in}}%
\pgfpathlineto{\pgfqpoint{4.794744in}{0.550000in}}%
\pgfpathlineto{\pgfqpoint{4.794744in}{0.550000in}}%
\pgfpathclose%
\pgfusepath{fill}%
\end{pgfscope}%
\begin{pgfscope}%
\pgfpathrectangle{\pgfqpoint{1.250000in}{0.550000in}}{\pgfqpoint{7.750000in}{3.850000in}}%
\pgfusepath{clip}%
\pgfsetbuttcap%
\pgfsetmiterjoin%
\definecolor{currentfill}{rgb}{0.501961,0.501961,0.501961}%
\pgfsetfillcolor{currentfill}%
\pgfsetfillopacity{0.700000}%
\pgfsetlinewidth{0.000000pt}%
\definecolor{currentstroke}{rgb}{0.000000,0.000000,0.000000}%
\pgfsetstrokecolor{currentstroke}%
\pgfsetstrokeopacity{0.700000}%
\pgfsetdash{}{0pt}%
\pgfpathmoveto{\pgfqpoint{4.822266in}{0.550000in}}%
\pgfpathlineto{\pgfqpoint{4.849787in}{0.550000in}}%
\pgfpathlineto{\pgfqpoint{4.849787in}{1.928564in}}%
\pgfpathlineto{\pgfqpoint{4.822266in}{1.928564in}}%
\pgfpathlineto{\pgfqpoint{4.822266in}{0.550000in}}%
\pgfpathclose%
\pgfusepath{fill}%
\end{pgfscope}%
\begin{pgfscope}%
\pgfpathrectangle{\pgfqpoint{1.250000in}{0.550000in}}{\pgfqpoint{7.750000in}{3.850000in}}%
\pgfusepath{clip}%
\pgfsetbuttcap%
\pgfsetmiterjoin%
\definecolor{currentfill}{rgb}{0.501961,0.501961,0.501961}%
\pgfsetfillcolor{currentfill}%
\pgfsetfillopacity{0.700000}%
\pgfsetlinewidth{0.000000pt}%
\definecolor{currentstroke}{rgb}{0.000000,0.000000,0.000000}%
\pgfsetstrokecolor{currentstroke}%
\pgfsetstrokeopacity{0.700000}%
\pgfsetdash{}{0pt}%
\pgfpathmoveto{\pgfqpoint{4.849787in}{0.550000in}}%
\pgfpathlineto{\pgfqpoint{4.877308in}{0.550000in}}%
\pgfpathlineto{\pgfqpoint{4.877308in}{0.550000in}}%
\pgfpathlineto{\pgfqpoint{4.849787in}{0.550000in}}%
\pgfpathlineto{\pgfqpoint{4.849787in}{0.550000in}}%
\pgfpathclose%
\pgfusepath{fill}%
\end{pgfscope}%
\begin{pgfscope}%
\pgfpathrectangle{\pgfqpoint{1.250000in}{0.550000in}}{\pgfqpoint{7.750000in}{3.850000in}}%
\pgfusepath{clip}%
\pgfsetbuttcap%
\pgfsetmiterjoin%
\definecolor{currentfill}{rgb}{0.501961,0.501961,0.501961}%
\pgfsetfillcolor{currentfill}%
\pgfsetfillopacity{0.700000}%
\pgfsetlinewidth{0.000000pt}%
\definecolor{currentstroke}{rgb}{0.000000,0.000000,0.000000}%
\pgfsetstrokecolor{currentstroke}%
\pgfsetstrokeopacity{0.700000}%
\pgfsetdash{}{0pt}%
\pgfpathmoveto{\pgfqpoint{4.877308in}{0.550000in}}%
\pgfpathlineto{\pgfqpoint{4.904830in}{0.550000in}}%
\pgfpathlineto{\pgfqpoint{4.904830in}{1.942820in}}%
\pgfpathlineto{\pgfqpoint{4.877308in}{1.942820in}}%
\pgfpathlineto{\pgfqpoint{4.877308in}{0.550000in}}%
\pgfpathclose%
\pgfusepath{fill}%
\end{pgfscope}%
\begin{pgfscope}%
\pgfpathrectangle{\pgfqpoint{1.250000in}{0.550000in}}{\pgfqpoint{7.750000in}{3.850000in}}%
\pgfusepath{clip}%
\pgfsetbuttcap%
\pgfsetmiterjoin%
\definecolor{currentfill}{rgb}{0.501961,0.501961,0.501961}%
\pgfsetfillcolor{currentfill}%
\pgfsetfillopacity{0.700000}%
\pgfsetlinewidth{0.000000pt}%
\definecolor{currentstroke}{rgb}{0.000000,0.000000,0.000000}%
\pgfsetstrokecolor{currentstroke}%
\pgfsetstrokeopacity{0.700000}%
\pgfsetdash{}{0pt}%
\pgfpathmoveto{\pgfqpoint{4.904830in}{0.550000in}}%
\pgfpathlineto{\pgfqpoint{4.932351in}{0.550000in}}%
\pgfpathlineto{\pgfqpoint{4.932351in}{0.550000in}}%
\pgfpathlineto{\pgfqpoint{4.904830in}{0.550000in}}%
\pgfpathlineto{\pgfqpoint{4.904830in}{0.550000in}}%
\pgfpathclose%
\pgfusepath{fill}%
\end{pgfscope}%
\begin{pgfscope}%
\pgfpathrectangle{\pgfqpoint{1.250000in}{0.550000in}}{\pgfqpoint{7.750000in}{3.850000in}}%
\pgfusepath{clip}%
\pgfsetbuttcap%
\pgfsetmiterjoin%
\definecolor{currentfill}{rgb}{0.501961,0.501961,0.501961}%
\pgfsetfillcolor{currentfill}%
\pgfsetfillopacity{0.700000}%
\pgfsetlinewidth{0.000000pt}%
\definecolor{currentstroke}{rgb}{0.000000,0.000000,0.000000}%
\pgfsetstrokecolor{currentstroke}%
\pgfsetstrokeopacity{0.700000}%
\pgfsetdash{}{0pt}%
\pgfpathmoveto{\pgfqpoint{4.932351in}{0.550000in}}%
\pgfpathlineto{\pgfqpoint{4.959872in}{0.550000in}}%
\pgfpathlineto{\pgfqpoint{4.959872in}{1.854432in}}%
\pgfpathlineto{\pgfqpoint{4.932351in}{1.854432in}}%
\pgfpathlineto{\pgfqpoint{4.932351in}{0.550000in}}%
\pgfpathclose%
\pgfusepath{fill}%
\end{pgfscope}%
\begin{pgfscope}%
\pgfpathrectangle{\pgfqpoint{1.250000in}{0.550000in}}{\pgfqpoint{7.750000in}{3.850000in}}%
\pgfusepath{clip}%
\pgfsetbuttcap%
\pgfsetmiterjoin%
\definecolor{currentfill}{rgb}{0.501961,0.501961,0.501961}%
\pgfsetfillcolor{currentfill}%
\pgfsetfillopacity{0.700000}%
\pgfsetlinewidth{0.000000pt}%
\definecolor{currentstroke}{rgb}{0.000000,0.000000,0.000000}%
\pgfsetstrokecolor{currentstroke}%
\pgfsetstrokeopacity{0.700000}%
\pgfsetdash{}{0pt}%
\pgfpathmoveto{\pgfqpoint{4.959872in}{0.550000in}}%
\pgfpathlineto{\pgfqpoint{4.987393in}{0.550000in}}%
\pgfpathlineto{\pgfqpoint{4.987393in}{0.550000in}}%
\pgfpathlineto{\pgfqpoint{4.959872in}{0.550000in}}%
\pgfpathlineto{\pgfqpoint{4.959872in}{0.550000in}}%
\pgfpathclose%
\pgfusepath{fill}%
\end{pgfscope}%
\begin{pgfscope}%
\pgfpathrectangle{\pgfqpoint{1.250000in}{0.550000in}}{\pgfqpoint{7.750000in}{3.850000in}}%
\pgfusepath{clip}%
\pgfsetbuttcap%
\pgfsetmiterjoin%
\definecolor{currentfill}{rgb}{0.501961,0.501961,0.501961}%
\pgfsetfillcolor{currentfill}%
\pgfsetfillopacity{0.700000}%
\pgfsetlinewidth{0.000000pt}%
\definecolor{currentstroke}{rgb}{0.000000,0.000000,0.000000}%
\pgfsetstrokecolor{currentstroke}%
\pgfsetstrokeopacity{0.700000}%
\pgfsetdash{}{0pt}%
\pgfpathmoveto{\pgfqpoint{4.987393in}{0.550000in}}%
\pgfpathlineto{\pgfqpoint{5.014915in}{0.550000in}}%
\pgfpathlineto{\pgfqpoint{5.014915in}{1.827346in}}%
\pgfpathlineto{\pgfqpoint{4.987393in}{1.827346in}}%
\pgfpathlineto{\pgfqpoint{4.987393in}{0.550000in}}%
\pgfpathclose%
\pgfusepath{fill}%
\end{pgfscope}%
\begin{pgfscope}%
\pgfpathrectangle{\pgfqpoint{1.250000in}{0.550000in}}{\pgfqpoint{7.750000in}{3.850000in}}%
\pgfusepath{clip}%
\pgfsetbuttcap%
\pgfsetmiterjoin%
\definecolor{currentfill}{rgb}{0.501961,0.501961,0.501961}%
\pgfsetfillcolor{currentfill}%
\pgfsetfillopacity{0.700000}%
\pgfsetlinewidth{0.000000pt}%
\definecolor{currentstroke}{rgb}{0.000000,0.000000,0.000000}%
\pgfsetstrokecolor{currentstroke}%
\pgfsetstrokeopacity{0.700000}%
\pgfsetdash{}{0pt}%
\pgfpathmoveto{\pgfqpoint{5.014915in}{0.550000in}}%
\pgfpathlineto{\pgfqpoint{5.042436in}{0.550000in}}%
\pgfpathlineto{\pgfqpoint{5.042436in}{0.550000in}}%
\pgfpathlineto{\pgfqpoint{5.014915in}{0.550000in}}%
\pgfpathlineto{\pgfqpoint{5.014915in}{0.550000in}}%
\pgfpathclose%
\pgfusepath{fill}%
\end{pgfscope}%
\begin{pgfscope}%
\pgfpathrectangle{\pgfqpoint{1.250000in}{0.550000in}}{\pgfqpoint{7.750000in}{3.850000in}}%
\pgfusepath{clip}%
\pgfsetbuttcap%
\pgfsetmiterjoin%
\definecolor{currentfill}{rgb}{0.501961,0.501961,0.501961}%
\pgfsetfillcolor{currentfill}%
\pgfsetfillopacity{0.700000}%
\pgfsetlinewidth{0.000000pt}%
\definecolor{currentstroke}{rgb}{0.000000,0.000000,0.000000}%
\pgfsetstrokecolor{currentstroke}%
\pgfsetstrokeopacity{0.700000}%
\pgfsetdash{}{0pt}%
\pgfpathmoveto{\pgfqpoint{5.042436in}{0.550000in}}%
\pgfpathlineto{\pgfqpoint{5.069957in}{0.550000in}}%
\pgfpathlineto{\pgfqpoint{5.069957in}{1.766045in}}%
\pgfpathlineto{\pgfqpoint{5.042436in}{1.766045in}}%
\pgfpathlineto{\pgfqpoint{5.042436in}{0.550000in}}%
\pgfpathclose%
\pgfusepath{fill}%
\end{pgfscope}%
\begin{pgfscope}%
\pgfpathrectangle{\pgfqpoint{1.250000in}{0.550000in}}{\pgfqpoint{7.750000in}{3.850000in}}%
\pgfusepath{clip}%
\pgfsetbuttcap%
\pgfsetmiterjoin%
\definecolor{currentfill}{rgb}{0.501961,0.501961,0.501961}%
\pgfsetfillcolor{currentfill}%
\pgfsetfillopacity{0.700000}%
\pgfsetlinewidth{0.000000pt}%
\definecolor{currentstroke}{rgb}{0.000000,0.000000,0.000000}%
\pgfsetstrokecolor{currentstroke}%
\pgfsetstrokeopacity{0.700000}%
\pgfsetdash{}{0pt}%
\pgfpathmoveto{\pgfqpoint{5.069957in}{0.550000in}}%
\pgfpathlineto{\pgfqpoint{5.097479in}{0.550000in}}%
\pgfpathlineto{\pgfqpoint{5.097479in}{0.550000in}}%
\pgfpathlineto{\pgfqpoint{5.069957in}{0.550000in}}%
\pgfpathlineto{\pgfqpoint{5.069957in}{0.550000in}}%
\pgfpathclose%
\pgfusepath{fill}%
\end{pgfscope}%
\begin{pgfscope}%
\pgfpathrectangle{\pgfqpoint{1.250000in}{0.550000in}}{\pgfqpoint{7.750000in}{3.850000in}}%
\pgfusepath{clip}%
\pgfsetbuttcap%
\pgfsetmiterjoin%
\definecolor{currentfill}{rgb}{0.501961,0.501961,0.501961}%
\pgfsetfillcolor{currentfill}%
\pgfsetfillopacity{0.700000}%
\pgfsetlinewidth{0.000000pt}%
\definecolor{currentstroke}{rgb}{0.000000,0.000000,0.000000}%
\pgfsetstrokecolor{currentstroke}%
\pgfsetstrokeopacity{0.700000}%
\pgfsetdash{}{0pt}%
\pgfpathmoveto{\pgfqpoint{5.097479in}{0.550000in}}%
\pgfpathlineto{\pgfqpoint{5.125000in}{0.550000in}}%
\pgfpathlineto{\pgfqpoint{5.125000in}{1.784578in}}%
\pgfpathlineto{\pgfqpoint{5.097479in}{1.784578in}}%
\pgfpathlineto{\pgfqpoint{5.097479in}{0.550000in}}%
\pgfpathclose%
\pgfusepath{fill}%
\end{pgfscope}%
\begin{pgfscope}%
\pgfpathrectangle{\pgfqpoint{1.250000in}{0.550000in}}{\pgfqpoint{7.750000in}{3.850000in}}%
\pgfusepath{clip}%
\pgfsetbuttcap%
\pgfsetmiterjoin%
\definecolor{currentfill}{rgb}{0.501961,0.501961,0.501961}%
\pgfsetfillcolor{currentfill}%
\pgfsetfillopacity{0.700000}%
\pgfsetlinewidth{0.000000pt}%
\definecolor{currentstroke}{rgb}{0.000000,0.000000,0.000000}%
\pgfsetstrokecolor{currentstroke}%
\pgfsetstrokeopacity{0.700000}%
\pgfsetdash{}{0pt}%
\pgfpathmoveto{\pgfqpoint{5.125000in}{0.550000in}}%
\pgfpathlineto{\pgfqpoint{5.152521in}{0.550000in}}%
\pgfpathlineto{\pgfqpoint{5.152521in}{0.550000in}}%
\pgfpathlineto{\pgfqpoint{5.125000in}{0.550000in}}%
\pgfpathlineto{\pgfqpoint{5.125000in}{0.550000in}}%
\pgfpathclose%
\pgfusepath{fill}%
\end{pgfscope}%
\begin{pgfscope}%
\pgfpathrectangle{\pgfqpoint{1.250000in}{0.550000in}}{\pgfqpoint{7.750000in}{3.850000in}}%
\pgfusepath{clip}%
\pgfsetbuttcap%
\pgfsetmiterjoin%
\definecolor{currentfill}{rgb}{0.501961,0.501961,0.501961}%
\pgfsetfillcolor{currentfill}%
\pgfsetfillopacity{0.700000}%
\pgfsetlinewidth{0.000000pt}%
\definecolor{currentstroke}{rgb}{0.000000,0.000000,0.000000}%
\pgfsetstrokecolor{currentstroke}%
\pgfsetstrokeopacity{0.700000}%
\pgfsetdash{}{0pt}%
\pgfpathmoveto{\pgfqpoint{5.152521in}{0.550000in}}%
\pgfpathlineto{\pgfqpoint{5.180043in}{0.550000in}}%
\pgfpathlineto{\pgfqpoint{5.180043in}{1.734681in}}%
\pgfpathlineto{\pgfqpoint{5.152521in}{1.734681in}}%
\pgfpathlineto{\pgfqpoint{5.152521in}{0.550000in}}%
\pgfpathclose%
\pgfusepath{fill}%
\end{pgfscope}%
\begin{pgfscope}%
\pgfpathrectangle{\pgfqpoint{1.250000in}{0.550000in}}{\pgfqpoint{7.750000in}{3.850000in}}%
\pgfusepath{clip}%
\pgfsetbuttcap%
\pgfsetmiterjoin%
\definecolor{currentfill}{rgb}{0.501961,0.501961,0.501961}%
\pgfsetfillcolor{currentfill}%
\pgfsetfillopacity{0.700000}%
\pgfsetlinewidth{0.000000pt}%
\definecolor{currentstroke}{rgb}{0.000000,0.000000,0.000000}%
\pgfsetstrokecolor{currentstroke}%
\pgfsetstrokeopacity{0.700000}%
\pgfsetdash{}{0pt}%
\pgfpathmoveto{\pgfqpoint{5.180043in}{0.550000in}}%
\pgfpathlineto{\pgfqpoint{5.207564in}{0.550000in}}%
\pgfpathlineto{\pgfqpoint{5.207564in}{0.550000in}}%
\pgfpathlineto{\pgfqpoint{5.180043in}{0.550000in}}%
\pgfpathlineto{\pgfqpoint{5.180043in}{0.550000in}}%
\pgfpathclose%
\pgfusepath{fill}%
\end{pgfscope}%
\begin{pgfscope}%
\pgfpathrectangle{\pgfqpoint{1.250000in}{0.550000in}}{\pgfqpoint{7.750000in}{3.850000in}}%
\pgfusepath{clip}%
\pgfsetbuttcap%
\pgfsetmiterjoin%
\definecolor{currentfill}{rgb}{0.501961,0.501961,0.501961}%
\pgfsetfillcolor{currentfill}%
\pgfsetfillopacity{0.700000}%
\pgfsetlinewidth{0.000000pt}%
\definecolor{currentstroke}{rgb}{0.000000,0.000000,0.000000}%
\pgfsetstrokecolor{currentstroke}%
\pgfsetstrokeopacity{0.700000}%
\pgfsetdash{}{0pt}%
\pgfpathmoveto{\pgfqpoint{5.207564in}{0.550000in}}%
\pgfpathlineto{\pgfqpoint{5.235085in}{0.550000in}}%
\pgfpathlineto{\pgfqpoint{5.235085in}{1.690487in}}%
\pgfpathlineto{\pgfqpoint{5.207564in}{1.690487in}}%
\pgfpathlineto{\pgfqpoint{5.207564in}{0.550000in}}%
\pgfpathclose%
\pgfusepath{fill}%
\end{pgfscope}%
\begin{pgfscope}%
\pgfpathrectangle{\pgfqpoint{1.250000in}{0.550000in}}{\pgfqpoint{7.750000in}{3.850000in}}%
\pgfusepath{clip}%
\pgfsetbuttcap%
\pgfsetmiterjoin%
\definecolor{currentfill}{rgb}{0.501961,0.501961,0.501961}%
\pgfsetfillcolor{currentfill}%
\pgfsetfillopacity{0.700000}%
\pgfsetlinewidth{0.000000pt}%
\definecolor{currentstroke}{rgb}{0.000000,0.000000,0.000000}%
\pgfsetstrokecolor{currentstroke}%
\pgfsetstrokeopacity{0.700000}%
\pgfsetdash{}{0pt}%
\pgfpathmoveto{\pgfqpoint{5.235085in}{0.550000in}}%
\pgfpathlineto{\pgfqpoint{5.262607in}{0.550000in}}%
\pgfpathlineto{\pgfqpoint{5.262607in}{0.550000in}}%
\pgfpathlineto{\pgfqpoint{5.235085in}{0.550000in}}%
\pgfpathlineto{\pgfqpoint{5.235085in}{0.550000in}}%
\pgfpathclose%
\pgfusepath{fill}%
\end{pgfscope}%
\begin{pgfscope}%
\pgfpathrectangle{\pgfqpoint{1.250000in}{0.550000in}}{\pgfqpoint{7.750000in}{3.850000in}}%
\pgfusepath{clip}%
\pgfsetbuttcap%
\pgfsetmiterjoin%
\definecolor{currentfill}{rgb}{0.501961,0.501961,0.501961}%
\pgfsetfillcolor{currentfill}%
\pgfsetfillopacity{0.700000}%
\pgfsetlinewidth{0.000000pt}%
\definecolor{currentstroke}{rgb}{0.000000,0.000000,0.000000}%
\pgfsetstrokecolor{currentstroke}%
\pgfsetstrokeopacity{0.700000}%
\pgfsetdash{}{0pt}%
\pgfpathmoveto{\pgfqpoint{5.262607in}{0.550000in}}%
\pgfpathlineto{\pgfqpoint{5.290128in}{0.550000in}}%
\pgfpathlineto{\pgfqpoint{5.290128in}{1.644868in}}%
\pgfpathlineto{\pgfqpoint{5.262607in}{1.644868in}}%
\pgfpathlineto{\pgfqpoint{5.262607in}{0.550000in}}%
\pgfpathclose%
\pgfusepath{fill}%
\end{pgfscope}%
\begin{pgfscope}%
\pgfpathrectangle{\pgfqpoint{1.250000in}{0.550000in}}{\pgfqpoint{7.750000in}{3.850000in}}%
\pgfusepath{clip}%
\pgfsetbuttcap%
\pgfsetmiterjoin%
\definecolor{currentfill}{rgb}{0.501961,0.501961,0.501961}%
\pgfsetfillcolor{currentfill}%
\pgfsetfillopacity{0.700000}%
\pgfsetlinewidth{0.000000pt}%
\definecolor{currentstroke}{rgb}{0.000000,0.000000,0.000000}%
\pgfsetstrokecolor{currentstroke}%
\pgfsetstrokeopacity{0.700000}%
\pgfsetdash{}{0pt}%
\pgfpathmoveto{\pgfqpoint{5.290128in}{0.550000in}}%
\pgfpathlineto{\pgfqpoint{5.317649in}{0.550000in}}%
\pgfpathlineto{\pgfqpoint{5.317649in}{0.550000in}}%
\pgfpathlineto{\pgfqpoint{5.290128in}{0.550000in}}%
\pgfpathlineto{\pgfqpoint{5.290128in}{0.550000in}}%
\pgfpathclose%
\pgfusepath{fill}%
\end{pgfscope}%
\begin{pgfscope}%
\pgfpathrectangle{\pgfqpoint{1.250000in}{0.550000in}}{\pgfqpoint{7.750000in}{3.850000in}}%
\pgfusepath{clip}%
\pgfsetbuttcap%
\pgfsetmiterjoin%
\definecolor{currentfill}{rgb}{0.501961,0.501961,0.501961}%
\pgfsetfillcolor{currentfill}%
\pgfsetfillopacity{0.700000}%
\pgfsetlinewidth{0.000000pt}%
\definecolor{currentstroke}{rgb}{0.000000,0.000000,0.000000}%
\pgfsetstrokecolor{currentstroke}%
\pgfsetstrokeopacity{0.700000}%
\pgfsetdash{}{0pt}%
\pgfpathmoveto{\pgfqpoint{5.317649in}{0.550000in}}%
\pgfpathlineto{\pgfqpoint{5.345170in}{0.550000in}}%
\pgfpathlineto{\pgfqpoint{5.345170in}{1.602100in}}%
\pgfpathlineto{\pgfqpoint{5.317649in}{1.602100in}}%
\pgfpathlineto{\pgfqpoint{5.317649in}{0.550000in}}%
\pgfpathclose%
\pgfusepath{fill}%
\end{pgfscope}%
\begin{pgfscope}%
\pgfpathrectangle{\pgfqpoint{1.250000in}{0.550000in}}{\pgfqpoint{7.750000in}{3.850000in}}%
\pgfusepath{clip}%
\pgfsetbuttcap%
\pgfsetmiterjoin%
\definecolor{currentfill}{rgb}{0.501961,0.501961,0.501961}%
\pgfsetfillcolor{currentfill}%
\pgfsetfillopacity{0.700000}%
\pgfsetlinewidth{0.000000pt}%
\definecolor{currentstroke}{rgb}{0.000000,0.000000,0.000000}%
\pgfsetstrokecolor{currentstroke}%
\pgfsetstrokeopacity{0.700000}%
\pgfsetdash{}{0pt}%
\pgfpathmoveto{\pgfqpoint{5.345170in}{0.550000in}}%
\pgfpathlineto{\pgfqpoint{5.372692in}{0.550000in}}%
\pgfpathlineto{\pgfqpoint{5.372692in}{0.550000in}}%
\pgfpathlineto{\pgfqpoint{5.345170in}{0.550000in}}%
\pgfpathlineto{\pgfqpoint{5.345170in}{0.550000in}}%
\pgfpathclose%
\pgfusepath{fill}%
\end{pgfscope}%
\begin{pgfscope}%
\pgfpathrectangle{\pgfqpoint{1.250000in}{0.550000in}}{\pgfqpoint{7.750000in}{3.850000in}}%
\pgfusepath{clip}%
\pgfsetbuttcap%
\pgfsetmiterjoin%
\definecolor{currentfill}{rgb}{0.501961,0.501961,0.501961}%
\pgfsetfillcolor{currentfill}%
\pgfsetfillopacity{0.700000}%
\pgfsetlinewidth{0.000000pt}%
\definecolor{currentstroke}{rgb}{0.000000,0.000000,0.000000}%
\pgfsetstrokecolor{currentstroke}%
\pgfsetstrokeopacity{0.700000}%
\pgfsetdash{}{0pt}%
\pgfpathmoveto{\pgfqpoint{5.372692in}{0.550000in}}%
\pgfpathlineto{\pgfqpoint{5.400213in}{0.550000in}}%
\pgfpathlineto{\pgfqpoint{5.400213in}{1.592120in}}%
\pgfpathlineto{\pgfqpoint{5.372692in}{1.592120in}}%
\pgfpathlineto{\pgfqpoint{5.372692in}{0.550000in}}%
\pgfpathclose%
\pgfusepath{fill}%
\end{pgfscope}%
\begin{pgfscope}%
\pgfpathrectangle{\pgfqpoint{1.250000in}{0.550000in}}{\pgfqpoint{7.750000in}{3.850000in}}%
\pgfusepath{clip}%
\pgfsetbuttcap%
\pgfsetmiterjoin%
\definecolor{currentfill}{rgb}{0.501961,0.501961,0.501961}%
\pgfsetfillcolor{currentfill}%
\pgfsetfillopacity{0.700000}%
\pgfsetlinewidth{0.000000pt}%
\definecolor{currentstroke}{rgb}{0.000000,0.000000,0.000000}%
\pgfsetstrokecolor{currentstroke}%
\pgfsetstrokeopacity{0.700000}%
\pgfsetdash{}{0pt}%
\pgfpathmoveto{\pgfqpoint{5.400213in}{0.550000in}}%
\pgfpathlineto{\pgfqpoint{5.427734in}{0.550000in}}%
\pgfpathlineto{\pgfqpoint{5.427734in}{0.550000in}}%
\pgfpathlineto{\pgfqpoint{5.400213in}{0.550000in}}%
\pgfpathlineto{\pgfqpoint{5.400213in}{0.550000in}}%
\pgfpathclose%
\pgfusepath{fill}%
\end{pgfscope}%
\begin{pgfscope}%
\pgfpathrectangle{\pgfqpoint{1.250000in}{0.550000in}}{\pgfqpoint{7.750000in}{3.850000in}}%
\pgfusepath{clip}%
\pgfsetbuttcap%
\pgfsetmiterjoin%
\definecolor{currentfill}{rgb}{0.501961,0.501961,0.501961}%
\pgfsetfillcolor{currentfill}%
\pgfsetfillopacity{0.700000}%
\pgfsetlinewidth{0.000000pt}%
\definecolor{currentstroke}{rgb}{0.000000,0.000000,0.000000}%
\pgfsetstrokecolor{currentstroke}%
\pgfsetstrokeopacity{0.700000}%
\pgfsetdash{}{0pt}%
\pgfpathmoveto{\pgfqpoint{5.427734in}{0.550000in}}%
\pgfpathlineto{\pgfqpoint{5.455256in}{0.550000in}}%
\pgfpathlineto{\pgfqpoint{5.455256in}{1.510861in}}%
\pgfpathlineto{\pgfqpoint{5.427734in}{1.510861in}}%
\pgfpathlineto{\pgfqpoint{5.427734in}{0.550000in}}%
\pgfpathclose%
\pgfusepath{fill}%
\end{pgfscope}%
\begin{pgfscope}%
\pgfpathrectangle{\pgfqpoint{1.250000in}{0.550000in}}{\pgfqpoint{7.750000in}{3.850000in}}%
\pgfusepath{clip}%
\pgfsetbuttcap%
\pgfsetmiterjoin%
\definecolor{currentfill}{rgb}{0.501961,0.501961,0.501961}%
\pgfsetfillcolor{currentfill}%
\pgfsetfillopacity{0.700000}%
\pgfsetlinewidth{0.000000pt}%
\definecolor{currentstroke}{rgb}{0.000000,0.000000,0.000000}%
\pgfsetstrokecolor{currentstroke}%
\pgfsetstrokeopacity{0.700000}%
\pgfsetdash{}{0pt}%
\pgfpathmoveto{\pgfqpoint{5.455256in}{0.550000in}}%
\pgfpathlineto{\pgfqpoint{5.482777in}{0.550000in}}%
\pgfpathlineto{\pgfqpoint{5.482777in}{0.550000in}}%
\pgfpathlineto{\pgfqpoint{5.455256in}{0.550000in}}%
\pgfpathlineto{\pgfqpoint{5.455256in}{0.550000in}}%
\pgfpathclose%
\pgfusepath{fill}%
\end{pgfscope}%
\begin{pgfscope}%
\pgfpathrectangle{\pgfqpoint{1.250000in}{0.550000in}}{\pgfqpoint{7.750000in}{3.850000in}}%
\pgfusepath{clip}%
\pgfsetbuttcap%
\pgfsetmiterjoin%
\definecolor{currentfill}{rgb}{0.501961,0.501961,0.501961}%
\pgfsetfillcolor{currentfill}%
\pgfsetfillopacity{0.700000}%
\pgfsetlinewidth{0.000000pt}%
\definecolor{currentstroke}{rgb}{0.000000,0.000000,0.000000}%
\pgfsetstrokecolor{currentstroke}%
\pgfsetstrokeopacity{0.700000}%
\pgfsetdash{}{0pt}%
\pgfpathmoveto{\pgfqpoint{5.482777in}{0.550000in}}%
\pgfpathlineto{\pgfqpoint{5.510298in}{0.550000in}}%
\pgfpathlineto{\pgfqpoint{5.510298in}{1.508009in}}%
\pgfpathlineto{\pgfqpoint{5.482777in}{1.508009in}}%
\pgfpathlineto{\pgfqpoint{5.482777in}{0.550000in}}%
\pgfpathclose%
\pgfusepath{fill}%
\end{pgfscope}%
\begin{pgfscope}%
\pgfpathrectangle{\pgfqpoint{1.250000in}{0.550000in}}{\pgfqpoint{7.750000in}{3.850000in}}%
\pgfusepath{clip}%
\pgfsetbuttcap%
\pgfsetmiterjoin%
\definecolor{currentfill}{rgb}{0.501961,0.501961,0.501961}%
\pgfsetfillcolor{currentfill}%
\pgfsetfillopacity{0.700000}%
\pgfsetlinewidth{0.000000pt}%
\definecolor{currentstroke}{rgb}{0.000000,0.000000,0.000000}%
\pgfsetstrokecolor{currentstroke}%
\pgfsetstrokeopacity{0.700000}%
\pgfsetdash{}{0pt}%
\pgfpathmoveto{\pgfqpoint{5.510298in}{0.550000in}}%
\pgfpathlineto{\pgfqpoint{5.537820in}{0.550000in}}%
\pgfpathlineto{\pgfqpoint{5.537820in}{0.550000in}}%
\pgfpathlineto{\pgfqpoint{5.510298in}{0.550000in}}%
\pgfpathlineto{\pgfqpoint{5.510298in}{0.550000in}}%
\pgfpathclose%
\pgfusepath{fill}%
\end{pgfscope}%
\begin{pgfscope}%
\pgfpathrectangle{\pgfqpoint{1.250000in}{0.550000in}}{\pgfqpoint{7.750000in}{3.850000in}}%
\pgfusepath{clip}%
\pgfsetbuttcap%
\pgfsetmiterjoin%
\definecolor{currentfill}{rgb}{0.501961,0.501961,0.501961}%
\pgfsetfillcolor{currentfill}%
\pgfsetfillopacity{0.700000}%
\pgfsetlinewidth{0.000000pt}%
\definecolor{currentstroke}{rgb}{0.000000,0.000000,0.000000}%
\pgfsetstrokecolor{currentstroke}%
\pgfsetstrokeopacity{0.700000}%
\pgfsetdash{}{0pt}%
\pgfpathmoveto{\pgfqpoint{5.537820in}{0.550000in}}%
\pgfpathlineto{\pgfqpoint{5.565341in}{0.550000in}}%
\pgfpathlineto{\pgfqpoint{5.565341in}{1.382556in}}%
\pgfpathlineto{\pgfqpoint{5.537820in}{1.382556in}}%
\pgfpathlineto{\pgfqpoint{5.537820in}{0.550000in}}%
\pgfpathclose%
\pgfusepath{fill}%
\end{pgfscope}%
\begin{pgfscope}%
\pgfpathrectangle{\pgfqpoint{1.250000in}{0.550000in}}{\pgfqpoint{7.750000in}{3.850000in}}%
\pgfusepath{clip}%
\pgfsetbuttcap%
\pgfsetmiterjoin%
\definecolor{currentfill}{rgb}{0.501961,0.501961,0.501961}%
\pgfsetfillcolor{currentfill}%
\pgfsetfillopacity{0.700000}%
\pgfsetlinewidth{0.000000pt}%
\definecolor{currentstroke}{rgb}{0.000000,0.000000,0.000000}%
\pgfsetstrokecolor{currentstroke}%
\pgfsetstrokeopacity{0.700000}%
\pgfsetdash{}{0pt}%
\pgfpathmoveto{\pgfqpoint{5.565341in}{0.550000in}}%
\pgfpathlineto{\pgfqpoint{5.592862in}{0.550000in}}%
\pgfpathlineto{\pgfqpoint{5.592862in}{0.550000in}}%
\pgfpathlineto{\pgfqpoint{5.565341in}{0.550000in}}%
\pgfpathlineto{\pgfqpoint{5.565341in}{0.550000in}}%
\pgfpathclose%
\pgfusepath{fill}%
\end{pgfscope}%
\begin{pgfscope}%
\pgfpathrectangle{\pgfqpoint{1.250000in}{0.550000in}}{\pgfqpoint{7.750000in}{3.850000in}}%
\pgfusepath{clip}%
\pgfsetbuttcap%
\pgfsetmiterjoin%
\definecolor{currentfill}{rgb}{0.501961,0.501961,0.501961}%
\pgfsetfillcolor{currentfill}%
\pgfsetfillopacity{0.700000}%
\pgfsetlinewidth{0.000000pt}%
\definecolor{currentstroke}{rgb}{0.000000,0.000000,0.000000}%
\pgfsetstrokecolor{currentstroke}%
\pgfsetstrokeopacity{0.700000}%
\pgfsetdash{}{0pt}%
\pgfpathmoveto{\pgfqpoint{5.592862in}{0.550000in}}%
\pgfpathlineto{\pgfqpoint{5.620384in}{0.550000in}}%
\pgfpathlineto{\pgfqpoint{5.620384in}{1.449559in}}%
\pgfpathlineto{\pgfqpoint{5.592862in}{1.449559in}}%
\pgfpathlineto{\pgfqpoint{5.592862in}{0.550000in}}%
\pgfpathclose%
\pgfusepath{fill}%
\end{pgfscope}%
\begin{pgfscope}%
\pgfpathrectangle{\pgfqpoint{1.250000in}{0.550000in}}{\pgfqpoint{7.750000in}{3.850000in}}%
\pgfusepath{clip}%
\pgfsetbuttcap%
\pgfsetmiterjoin%
\definecolor{currentfill}{rgb}{0.501961,0.501961,0.501961}%
\pgfsetfillcolor{currentfill}%
\pgfsetfillopacity{0.700000}%
\pgfsetlinewidth{0.000000pt}%
\definecolor{currentstroke}{rgb}{0.000000,0.000000,0.000000}%
\pgfsetstrokecolor{currentstroke}%
\pgfsetstrokeopacity{0.700000}%
\pgfsetdash{}{0pt}%
\pgfpathmoveto{\pgfqpoint{5.620384in}{0.550000in}}%
\pgfpathlineto{\pgfqpoint{5.647905in}{0.550000in}}%
\pgfpathlineto{\pgfqpoint{5.647905in}{0.550000in}}%
\pgfpathlineto{\pgfqpoint{5.620384in}{0.550000in}}%
\pgfpathlineto{\pgfqpoint{5.620384in}{0.550000in}}%
\pgfpathclose%
\pgfusepath{fill}%
\end{pgfscope}%
\begin{pgfscope}%
\pgfpathrectangle{\pgfqpoint{1.250000in}{0.550000in}}{\pgfqpoint{7.750000in}{3.850000in}}%
\pgfusepath{clip}%
\pgfsetbuttcap%
\pgfsetmiterjoin%
\definecolor{currentfill}{rgb}{0.501961,0.501961,0.501961}%
\pgfsetfillcolor{currentfill}%
\pgfsetfillopacity{0.700000}%
\pgfsetlinewidth{0.000000pt}%
\definecolor{currentstroke}{rgb}{0.000000,0.000000,0.000000}%
\pgfsetstrokecolor{currentstroke}%
\pgfsetstrokeopacity{0.700000}%
\pgfsetdash{}{0pt}%
\pgfpathmoveto{\pgfqpoint{5.647905in}{0.550000in}}%
\pgfpathlineto{\pgfqpoint{5.675426in}{0.550000in}}%
\pgfpathlineto{\pgfqpoint{5.675426in}{1.425324in}}%
\pgfpathlineto{\pgfqpoint{5.647905in}{1.425324in}}%
\pgfpathlineto{\pgfqpoint{5.647905in}{0.550000in}}%
\pgfpathclose%
\pgfusepath{fill}%
\end{pgfscope}%
\begin{pgfscope}%
\pgfpathrectangle{\pgfqpoint{1.250000in}{0.550000in}}{\pgfqpoint{7.750000in}{3.850000in}}%
\pgfusepath{clip}%
\pgfsetbuttcap%
\pgfsetmiterjoin%
\definecolor{currentfill}{rgb}{0.501961,0.501961,0.501961}%
\pgfsetfillcolor{currentfill}%
\pgfsetfillopacity{0.700000}%
\pgfsetlinewidth{0.000000pt}%
\definecolor{currentstroke}{rgb}{0.000000,0.000000,0.000000}%
\pgfsetstrokecolor{currentstroke}%
\pgfsetstrokeopacity{0.700000}%
\pgfsetdash{}{0pt}%
\pgfpathmoveto{\pgfqpoint{5.675426in}{0.550000in}}%
\pgfpathlineto{\pgfqpoint{5.702947in}{0.550000in}}%
\pgfpathlineto{\pgfqpoint{5.702947in}{0.550000in}}%
\pgfpathlineto{\pgfqpoint{5.675426in}{0.550000in}}%
\pgfpathlineto{\pgfqpoint{5.675426in}{0.550000in}}%
\pgfpathclose%
\pgfusepath{fill}%
\end{pgfscope}%
\begin{pgfscope}%
\pgfpathrectangle{\pgfqpoint{1.250000in}{0.550000in}}{\pgfqpoint{7.750000in}{3.850000in}}%
\pgfusepath{clip}%
\pgfsetbuttcap%
\pgfsetmiterjoin%
\definecolor{currentfill}{rgb}{0.501961,0.501961,0.501961}%
\pgfsetfillcolor{currentfill}%
\pgfsetfillopacity{0.700000}%
\pgfsetlinewidth{0.000000pt}%
\definecolor{currentstroke}{rgb}{0.000000,0.000000,0.000000}%
\pgfsetstrokecolor{currentstroke}%
\pgfsetstrokeopacity{0.700000}%
\pgfsetdash{}{0pt}%
\pgfpathmoveto{\pgfqpoint{5.702947in}{0.550000in}}%
\pgfpathlineto{\pgfqpoint{5.730469in}{0.550000in}}%
\pgfpathlineto{\pgfqpoint{5.730469in}{1.445283in}}%
\pgfpathlineto{\pgfqpoint{5.702947in}{1.445283in}}%
\pgfpathlineto{\pgfqpoint{5.702947in}{0.550000in}}%
\pgfpathclose%
\pgfusepath{fill}%
\end{pgfscope}%
\begin{pgfscope}%
\pgfpathrectangle{\pgfqpoint{1.250000in}{0.550000in}}{\pgfqpoint{7.750000in}{3.850000in}}%
\pgfusepath{clip}%
\pgfsetbuttcap%
\pgfsetmiterjoin%
\definecolor{currentfill}{rgb}{0.501961,0.501961,0.501961}%
\pgfsetfillcolor{currentfill}%
\pgfsetfillopacity{0.700000}%
\pgfsetlinewidth{0.000000pt}%
\definecolor{currentstroke}{rgb}{0.000000,0.000000,0.000000}%
\pgfsetstrokecolor{currentstroke}%
\pgfsetstrokeopacity{0.700000}%
\pgfsetdash{}{0pt}%
\pgfpathmoveto{\pgfqpoint{5.730469in}{0.550000in}}%
\pgfpathlineto{\pgfqpoint{5.757990in}{0.550000in}}%
\pgfpathlineto{\pgfqpoint{5.757990in}{0.550000in}}%
\pgfpathlineto{\pgfqpoint{5.730469in}{0.550000in}}%
\pgfpathlineto{\pgfqpoint{5.730469in}{0.550000in}}%
\pgfpathclose%
\pgfusepath{fill}%
\end{pgfscope}%
\begin{pgfscope}%
\pgfpathrectangle{\pgfqpoint{1.250000in}{0.550000in}}{\pgfqpoint{7.750000in}{3.850000in}}%
\pgfusepath{clip}%
\pgfsetbuttcap%
\pgfsetmiterjoin%
\definecolor{currentfill}{rgb}{0.501961,0.501961,0.501961}%
\pgfsetfillcolor{currentfill}%
\pgfsetfillopacity{0.700000}%
\pgfsetlinewidth{0.000000pt}%
\definecolor{currentstroke}{rgb}{0.000000,0.000000,0.000000}%
\pgfsetstrokecolor{currentstroke}%
\pgfsetstrokeopacity{0.700000}%
\pgfsetdash{}{0pt}%
\pgfpathmoveto{\pgfqpoint{5.757990in}{0.550000in}}%
\pgfpathlineto{\pgfqpoint{5.785511in}{0.550000in}}%
\pgfpathlineto{\pgfqpoint{5.785511in}{1.374002in}}%
\pgfpathlineto{\pgfqpoint{5.757990in}{1.374002in}}%
\pgfpathlineto{\pgfqpoint{5.757990in}{0.550000in}}%
\pgfpathclose%
\pgfusepath{fill}%
\end{pgfscope}%
\begin{pgfscope}%
\pgfpathrectangle{\pgfqpoint{1.250000in}{0.550000in}}{\pgfqpoint{7.750000in}{3.850000in}}%
\pgfusepath{clip}%
\pgfsetbuttcap%
\pgfsetmiterjoin%
\definecolor{currentfill}{rgb}{0.501961,0.501961,0.501961}%
\pgfsetfillcolor{currentfill}%
\pgfsetfillopacity{0.700000}%
\pgfsetlinewidth{0.000000pt}%
\definecolor{currentstroke}{rgb}{0.000000,0.000000,0.000000}%
\pgfsetstrokecolor{currentstroke}%
\pgfsetstrokeopacity{0.700000}%
\pgfsetdash{}{0pt}%
\pgfpathmoveto{\pgfqpoint{5.785511in}{0.550000in}}%
\pgfpathlineto{\pgfqpoint{5.813033in}{0.550000in}}%
\pgfpathlineto{\pgfqpoint{5.813033in}{0.550000in}}%
\pgfpathlineto{\pgfqpoint{5.785511in}{0.550000in}}%
\pgfpathlineto{\pgfqpoint{5.785511in}{0.550000in}}%
\pgfpathclose%
\pgfusepath{fill}%
\end{pgfscope}%
\begin{pgfscope}%
\pgfpathrectangle{\pgfqpoint{1.250000in}{0.550000in}}{\pgfqpoint{7.750000in}{3.850000in}}%
\pgfusepath{clip}%
\pgfsetbuttcap%
\pgfsetmiterjoin%
\definecolor{currentfill}{rgb}{0.501961,0.501961,0.501961}%
\pgfsetfillcolor{currentfill}%
\pgfsetfillopacity{0.700000}%
\pgfsetlinewidth{0.000000pt}%
\definecolor{currentstroke}{rgb}{0.000000,0.000000,0.000000}%
\pgfsetstrokecolor{currentstroke}%
\pgfsetstrokeopacity{0.700000}%
\pgfsetdash{}{0pt}%
\pgfpathmoveto{\pgfqpoint{5.813033in}{0.550000in}}%
\pgfpathlineto{\pgfqpoint{5.840554in}{0.550000in}}%
\pgfpathlineto{\pgfqpoint{5.840554in}{1.379705in}}%
\pgfpathlineto{\pgfqpoint{5.813033in}{1.379705in}}%
\pgfpathlineto{\pgfqpoint{5.813033in}{0.550000in}}%
\pgfpathclose%
\pgfusepath{fill}%
\end{pgfscope}%
\begin{pgfscope}%
\pgfpathrectangle{\pgfqpoint{1.250000in}{0.550000in}}{\pgfqpoint{7.750000in}{3.850000in}}%
\pgfusepath{clip}%
\pgfsetbuttcap%
\pgfsetmiterjoin%
\definecolor{currentfill}{rgb}{0.501961,0.501961,0.501961}%
\pgfsetfillcolor{currentfill}%
\pgfsetfillopacity{0.700000}%
\pgfsetlinewidth{0.000000pt}%
\definecolor{currentstroke}{rgb}{0.000000,0.000000,0.000000}%
\pgfsetstrokecolor{currentstroke}%
\pgfsetstrokeopacity{0.700000}%
\pgfsetdash{}{0pt}%
\pgfpathmoveto{\pgfqpoint{5.840554in}{0.550000in}}%
\pgfpathlineto{\pgfqpoint{5.868075in}{0.550000in}}%
\pgfpathlineto{\pgfqpoint{5.868075in}{0.550000in}}%
\pgfpathlineto{\pgfqpoint{5.840554in}{0.550000in}}%
\pgfpathlineto{\pgfqpoint{5.840554in}{0.550000in}}%
\pgfpathclose%
\pgfusepath{fill}%
\end{pgfscope}%
\begin{pgfscope}%
\pgfpathrectangle{\pgfqpoint{1.250000in}{0.550000in}}{\pgfqpoint{7.750000in}{3.850000in}}%
\pgfusepath{clip}%
\pgfsetbuttcap%
\pgfsetmiterjoin%
\definecolor{currentfill}{rgb}{0.501961,0.501961,0.501961}%
\pgfsetfillcolor{currentfill}%
\pgfsetfillopacity{0.700000}%
\pgfsetlinewidth{0.000000pt}%
\definecolor{currentstroke}{rgb}{0.000000,0.000000,0.000000}%
\pgfsetstrokecolor{currentstroke}%
\pgfsetstrokeopacity{0.700000}%
\pgfsetdash{}{0pt}%
\pgfpathmoveto{\pgfqpoint{5.868075in}{0.550000in}}%
\pgfpathlineto{\pgfqpoint{5.895597in}{0.550000in}}%
\pgfpathlineto{\pgfqpoint{5.895597in}{1.278486in}}%
\pgfpathlineto{\pgfqpoint{5.868075in}{1.278486in}}%
\pgfpathlineto{\pgfqpoint{5.868075in}{0.550000in}}%
\pgfpathclose%
\pgfusepath{fill}%
\end{pgfscope}%
\begin{pgfscope}%
\pgfpathrectangle{\pgfqpoint{1.250000in}{0.550000in}}{\pgfqpoint{7.750000in}{3.850000in}}%
\pgfusepath{clip}%
\pgfsetbuttcap%
\pgfsetmiterjoin%
\definecolor{currentfill}{rgb}{0.501961,0.501961,0.501961}%
\pgfsetfillcolor{currentfill}%
\pgfsetfillopacity{0.700000}%
\pgfsetlinewidth{0.000000pt}%
\definecolor{currentstroke}{rgb}{0.000000,0.000000,0.000000}%
\pgfsetstrokecolor{currentstroke}%
\pgfsetstrokeopacity{0.700000}%
\pgfsetdash{}{0pt}%
\pgfpathmoveto{\pgfqpoint{5.895597in}{0.550000in}}%
\pgfpathlineto{\pgfqpoint{5.923118in}{0.550000in}}%
\pgfpathlineto{\pgfqpoint{5.923118in}{0.550000in}}%
\pgfpathlineto{\pgfqpoint{5.895597in}{0.550000in}}%
\pgfpathlineto{\pgfqpoint{5.895597in}{0.550000in}}%
\pgfpathclose%
\pgfusepath{fill}%
\end{pgfscope}%
\begin{pgfscope}%
\pgfpathrectangle{\pgfqpoint{1.250000in}{0.550000in}}{\pgfqpoint{7.750000in}{3.850000in}}%
\pgfusepath{clip}%
\pgfsetbuttcap%
\pgfsetmiterjoin%
\definecolor{currentfill}{rgb}{0.501961,0.501961,0.501961}%
\pgfsetfillcolor{currentfill}%
\pgfsetfillopacity{0.700000}%
\pgfsetlinewidth{0.000000pt}%
\definecolor{currentstroke}{rgb}{0.000000,0.000000,0.000000}%
\pgfsetstrokecolor{currentstroke}%
\pgfsetstrokeopacity{0.700000}%
\pgfsetdash{}{0pt}%
\pgfpathmoveto{\pgfqpoint{5.923118in}{0.550000in}}%
\pgfpathlineto{\pgfqpoint{5.950639in}{0.550000in}}%
\pgfpathlineto{\pgfqpoint{5.950639in}{1.272784in}}%
\pgfpathlineto{\pgfqpoint{5.923118in}{1.272784in}}%
\pgfpathlineto{\pgfqpoint{5.923118in}{0.550000in}}%
\pgfpathclose%
\pgfusepath{fill}%
\end{pgfscope}%
\begin{pgfscope}%
\pgfpathrectangle{\pgfqpoint{1.250000in}{0.550000in}}{\pgfqpoint{7.750000in}{3.850000in}}%
\pgfusepath{clip}%
\pgfsetbuttcap%
\pgfsetmiterjoin%
\definecolor{currentfill}{rgb}{0.501961,0.501961,0.501961}%
\pgfsetfillcolor{currentfill}%
\pgfsetfillopacity{0.700000}%
\pgfsetlinewidth{0.000000pt}%
\definecolor{currentstroke}{rgb}{0.000000,0.000000,0.000000}%
\pgfsetstrokecolor{currentstroke}%
\pgfsetstrokeopacity{0.700000}%
\pgfsetdash{}{0pt}%
\pgfpathmoveto{\pgfqpoint{5.950639in}{0.550000in}}%
\pgfpathlineto{\pgfqpoint{5.978161in}{0.550000in}}%
\pgfpathlineto{\pgfqpoint{5.978161in}{1.314126in}}%
\pgfpathlineto{\pgfqpoint{5.950639in}{1.314126in}}%
\pgfpathlineto{\pgfqpoint{5.950639in}{0.550000in}}%
\pgfpathclose%
\pgfusepath{fill}%
\end{pgfscope}%
\begin{pgfscope}%
\pgfpathrectangle{\pgfqpoint{1.250000in}{0.550000in}}{\pgfqpoint{7.750000in}{3.850000in}}%
\pgfusepath{clip}%
\pgfsetbuttcap%
\pgfsetmiterjoin%
\definecolor{currentfill}{rgb}{0.501961,0.501961,0.501961}%
\pgfsetfillcolor{currentfill}%
\pgfsetfillopacity{0.700000}%
\pgfsetlinewidth{0.000000pt}%
\definecolor{currentstroke}{rgb}{0.000000,0.000000,0.000000}%
\pgfsetstrokecolor{currentstroke}%
\pgfsetstrokeopacity{0.700000}%
\pgfsetdash{}{0pt}%
\pgfpathmoveto{\pgfqpoint{5.978161in}{0.550000in}}%
\pgfpathlineto{\pgfqpoint{6.005682in}{0.550000in}}%
\pgfpathlineto{\pgfqpoint{6.005682in}{0.550000in}}%
\pgfpathlineto{\pgfqpoint{5.978161in}{0.550000in}}%
\pgfpathlineto{\pgfqpoint{5.978161in}{0.550000in}}%
\pgfpathclose%
\pgfusepath{fill}%
\end{pgfscope}%
\begin{pgfscope}%
\pgfpathrectangle{\pgfqpoint{1.250000in}{0.550000in}}{\pgfqpoint{7.750000in}{3.850000in}}%
\pgfusepath{clip}%
\pgfsetbuttcap%
\pgfsetmiterjoin%
\definecolor{currentfill}{rgb}{0.501961,0.501961,0.501961}%
\pgfsetfillcolor{currentfill}%
\pgfsetfillopacity{0.700000}%
\pgfsetlinewidth{0.000000pt}%
\definecolor{currentstroke}{rgb}{0.000000,0.000000,0.000000}%
\pgfsetstrokecolor{currentstroke}%
\pgfsetstrokeopacity{0.700000}%
\pgfsetdash{}{0pt}%
\pgfpathmoveto{\pgfqpoint{6.005682in}{0.550000in}}%
\pgfpathlineto{\pgfqpoint{6.033203in}{0.550000in}}%
\pgfpathlineto{\pgfqpoint{6.033203in}{1.261379in}}%
\pgfpathlineto{\pgfqpoint{6.005682in}{1.261379in}}%
\pgfpathlineto{\pgfqpoint{6.005682in}{0.550000in}}%
\pgfpathclose%
\pgfusepath{fill}%
\end{pgfscope}%
\begin{pgfscope}%
\pgfpathrectangle{\pgfqpoint{1.250000in}{0.550000in}}{\pgfqpoint{7.750000in}{3.850000in}}%
\pgfusepath{clip}%
\pgfsetbuttcap%
\pgfsetmiterjoin%
\definecolor{currentfill}{rgb}{0.501961,0.501961,0.501961}%
\pgfsetfillcolor{currentfill}%
\pgfsetfillopacity{0.700000}%
\pgfsetlinewidth{0.000000pt}%
\definecolor{currentstroke}{rgb}{0.000000,0.000000,0.000000}%
\pgfsetstrokecolor{currentstroke}%
\pgfsetstrokeopacity{0.700000}%
\pgfsetdash{}{0pt}%
\pgfpathmoveto{\pgfqpoint{6.033203in}{0.550000in}}%
\pgfpathlineto{\pgfqpoint{6.060724in}{0.550000in}}%
\pgfpathlineto{\pgfqpoint{6.060724in}{0.550000in}}%
\pgfpathlineto{\pgfqpoint{6.033203in}{0.550000in}}%
\pgfpathlineto{\pgfqpoint{6.033203in}{0.550000in}}%
\pgfpathclose%
\pgfusepath{fill}%
\end{pgfscope}%
\begin{pgfscope}%
\pgfpathrectangle{\pgfqpoint{1.250000in}{0.550000in}}{\pgfqpoint{7.750000in}{3.850000in}}%
\pgfusepath{clip}%
\pgfsetbuttcap%
\pgfsetmiterjoin%
\definecolor{currentfill}{rgb}{0.501961,0.501961,0.501961}%
\pgfsetfillcolor{currentfill}%
\pgfsetfillopacity{0.700000}%
\pgfsetlinewidth{0.000000pt}%
\definecolor{currentstroke}{rgb}{0.000000,0.000000,0.000000}%
\pgfsetstrokecolor{currentstroke}%
\pgfsetstrokeopacity{0.700000}%
\pgfsetdash{}{0pt}%
\pgfpathmoveto{\pgfqpoint{6.060724in}{0.550000in}}%
\pgfpathlineto{\pgfqpoint{6.088246in}{0.550000in}}%
\pgfpathlineto{\pgfqpoint{6.088246in}{1.207206in}}%
\pgfpathlineto{\pgfqpoint{6.060724in}{1.207206in}}%
\pgfpathlineto{\pgfqpoint{6.060724in}{0.550000in}}%
\pgfpathclose%
\pgfusepath{fill}%
\end{pgfscope}%
\begin{pgfscope}%
\pgfpathrectangle{\pgfqpoint{1.250000in}{0.550000in}}{\pgfqpoint{7.750000in}{3.850000in}}%
\pgfusepath{clip}%
\pgfsetbuttcap%
\pgfsetmiterjoin%
\definecolor{currentfill}{rgb}{0.501961,0.501961,0.501961}%
\pgfsetfillcolor{currentfill}%
\pgfsetfillopacity{0.700000}%
\pgfsetlinewidth{0.000000pt}%
\definecolor{currentstroke}{rgb}{0.000000,0.000000,0.000000}%
\pgfsetstrokecolor{currentstroke}%
\pgfsetstrokeopacity{0.700000}%
\pgfsetdash{}{0pt}%
\pgfpathmoveto{\pgfqpoint{6.088246in}{0.550000in}}%
\pgfpathlineto{\pgfqpoint{6.115767in}{0.550000in}}%
\pgfpathlineto{\pgfqpoint{6.115767in}{0.550000in}}%
\pgfpathlineto{\pgfqpoint{6.088246in}{0.550000in}}%
\pgfpathlineto{\pgfqpoint{6.088246in}{0.550000in}}%
\pgfpathclose%
\pgfusepath{fill}%
\end{pgfscope}%
\begin{pgfscope}%
\pgfpathrectangle{\pgfqpoint{1.250000in}{0.550000in}}{\pgfqpoint{7.750000in}{3.850000in}}%
\pgfusepath{clip}%
\pgfsetbuttcap%
\pgfsetmiterjoin%
\definecolor{currentfill}{rgb}{0.501961,0.501961,0.501961}%
\pgfsetfillcolor{currentfill}%
\pgfsetfillopacity{0.700000}%
\pgfsetlinewidth{0.000000pt}%
\definecolor{currentstroke}{rgb}{0.000000,0.000000,0.000000}%
\pgfsetstrokecolor{currentstroke}%
\pgfsetstrokeopacity{0.700000}%
\pgfsetdash{}{0pt}%
\pgfpathmoveto{\pgfqpoint{6.115767in}{0.550000in}}%
\pgfpathlineto{\pgfqpoint{6.143288in}{0.550000in}}%
\pgfpathlineto{\pgfqpoint{6.143288in}{1.194375in}}%
\pgfpathlineto{\pgfqpoint{6.115767in}{1.194375in}}%
\pgfpathlineto{\pgfqpoint{6.115767in}{0.550000in}}%
\pgfpathclose%
\pgfusepath{fill}%
\end{pgfscope}%
\begin{pgfscope}%
\pgfpathrectangle{\pgfqpoint{1.250000in}{0.550000in}}{\pgfqpoint{7.750000in}{3.850000in}}%
\pgfusepath{clip}%
\pgfsetbuttcap%
\pgfsetmiterjoin%
\definecolor{currentfill}{rgb}{0.501961,0.501961,0.501961}%
\pgfsetfillcolor{currentfill}%
\pgfsetfillopacity{0.700000}%
\pgfsetlinewidth{0.000000pt}%
\definecolor{currentstroke}{rgb}{0.000000,0.000000,0.000000}%
\pgfsetstrokecolor{currentstroke}%
\pgfsetstrokeopacity{0.700000}%
\pgfsetdash{}{0pt}%
\pgfpathmoveto{\pgfqpoint{6.143288in}{0.550000in}}%
\pgfpathlineto{\pgfqpoint{6.170810in}{0.550000in}}%
\pgfpathlineto{\pgfqpoint{6.170810in}{0.550000in}}%
\pgfpathlineto{\pgfqpoint{6.143288in}{0.550000in}}%
\pgfpathlineto{\pgfqpoint{6.143288in}{0.550000in}}%
\pgfpathclose%
\pgfusepath{fill}%
\end{pgfscope}%
\begin{pgfscope}%
\pgfpathrectangle{\pgfqpoint{1.250000in}{0.550000in}}{\pgfqpoint{7.750000in}{3.850000in}}%
\pgfusepath{clip}%
\pgfsetbuttcap%
\pgfsetmiterjoin%
\definecolor{currentfill}{rgb}{0.501961,0.501961,0.501961}%
\pgfsetfillcolor{currentfill}%
\pgfsetfillopacity{0.700000}%
\pgfsetlinewidth{0.000000pt}%
\definecolor{currentstroke}{rgb}{0.000000,0.000000,0.000000}%
\pgfsetstrokecolor{currentstroke}%
\pgfsetstrokeopacity{0.700000}%
\pgfsetdash{}{0pt}%
\pgfpathmoveto{\pgfqpoint{6.170810in}{0.550000in}}%
\pgfpathlineto{\pgfqpoint{6.198331in}{0.550000in}}%
\pgfpathlineto{\pgfqpoint{6.198331in}{1.222888in}}%
\pgfpathlineto{\pgfqpoint{6.170810in}{1.222888in}}%
\pgfpathlineto{\pgfqpoint{6.170810in}{0.550000in}}%
\pgfpathclose%
\pgfusepath{fill}%
\end{pgfscope}%
\begin{pgfscope}%
\pgfpathrectangle{\pgfqpoint{1.250000in}{0.550000in}}{\pgfqpoint{7.750000in}{3.850000in}}%
\pgfusepath{clip}%
\pgfsetbuttcap%
\pgfsetmiterjoin%
\definecolor{currentfill}{rgb}{0.501961,0.501961,0.501961}%
\pgfsetfillcolor{currentfill}%
\pgfsetfillopacity{0.700000}%
\pgfsetlinewidth{0.000000pt}%
\definecolor{currentstroke}{rgb}{0.000000,0.000000,0.000000}%
\pgfsetstrokecolor{currentstroke}%
\pgfsetstrokeopacity{0.700000}%
\pgfsetdash{}{0pt}%
\pgfpathmoveto{\pgfqpoint{6.198331in}{0.550000in}}%
\pgfpathlineto{\pgfqpoint{6.225852in}{0.550000in}}%
\pgfpathlineto{\pgfqpoint{6.225852in}{0.550000in}}%
\pgfpathlineto{\pgfqpoint{6.198331in}{0.550000in}}%
\pgfpathlineto{\pgfqpoint{6.198331in}{0.550000in}}%
\pgfpathclose%
\pgfusepath{fill}%
\end{pgfscope}%
\begin{pgfscope}%
\pgfpathrectangle{\pgfqpoint{1.250000in}{0.550000in}}{\pgfqpoint{7.750000in}{3.850000in}}%
\pgfusepath{clip}%
\pgfsetbuttcap%
\pgfsetmiterjoin%
\definecolor{currentfill}{rgb}{0.501961,0.501961,0.501961}%
\pgfsetfillcolor{currentfill}%
\pgfsetfillopacity{0.700000}%
\pgfsetlinewidth{0.000000pt}%
\definecolor{currentstroke}{rgb}{0.000000,0.000000,0.000000}%
\pgfsetstrokecolor{currentstroke}%
\pgfsetstrokeopacity{0.700000}%
\pgfsetdash{}{0pt}%
\pgfpathmoveto{\pgfqpoint{6.225852in}{0.550000in}}%
\pgfpathlineto{\pgfqpoint{6.253374in}{0.550000in}}%
\pgfpathlineto{\pgfqpoint{6.253374in}{1.191524in}}%
\pgfpathlineto{\pgfqpoint{6.225852in}{1.191524in}}%
\pgfpathlineto{\pgfqpoint{6.225852in}{0.550000in}}%
\pgfpathclose%
\pgfusepath{fill}%
\end{pgfscope}%
\begin{pgfscope}%
\pgfpathrectangle{\pgfqpoint{1.250000in}{0.550000in}}{\pgfqpoint{7.750000in}{3.850000in}}%
\pgfusepath{clip}%
\pgfsetbuttcap%
\pgfsetmiterjoin%
\definecolor{currentfill}{rgb}{0.501961,0.501961,0.501961}%
\pgfsetfillcolor{currentfill}%
\pgfsetfillopacity{0.700000}%
\pgfsetlinewidth{0.000000pt}%
\definecolor{currentstroke}{rgb}{0.000000,0.000000,0.000000}%
\pgfsetstrokecolor{currentstroke}%
\pgfsetstrokeopacity{0.700000}%
\pgfsetdash{}{0pt}%
\pgfpathmoveto{\pgfqpoint{6.253374in}{0.550000in}}%
\pgfpathlineto{\pgfqpoint{6.280895in}{0.550000in}}%
\pgfpathlineto{\pgfqpoint{6.280895in}{0.550000in}}%
\pgfpathlineto{\pgfqpoint{6.253374in}{0.550000in}}%
\pgfpathlineto{\pgfqpoint{6.253374in}{0.550000in}}%
\pgfpathclose%
\pgfusepath{fill}%
\end{pgfscope}%
\begin{pgfscope}%
\pgfpathrectangle{\pgfqpoint{1.250000in}{0.550000in}}{\pgfqpoint{7.750000in}{3.850000in}}%
\pgfusepath{clip}%
\pgfsetbuttcap%
\pgfsetmiterjoin%
\definecolor{currentfill}{rgb}{0.501961,0.501961,0.501961}%
\pgfsetfillcolor{currentfill}%
\pgfsetfillopacity{0.700000}%
\pgfsetlinewidth{0.000000pt}%
\definecolor{currentstroke}{rgb}{0.000000,0.000000,0.000000}%
\pgfsetstrokecolor{currentstroke}%
\pgfsetstrokeopacity{0.700000}%
\pgfsetdash{}{0pt}%
\pgfpathmoveto{\pgfqpoint{6.280895in}{0.550000in}}%
\pgfpathlineto{\pgfqpoint{6.308416in}{0.550000in}}%
\pgfpathlineto{\pgfqpoint{6.308416in}{1.161586in}}%
\pgfpathlineto{\pgfqpoint{6.280895in}{1.161586in}}%
\pgfpathlineto{\pgfqpoint{6.280895in}{0.550000in}}%
\pgfpathclose%
\pgfusepath{fill}%
\end{pgfscope}%
\begin{pgfscope}%
\pgfpathrectangle{\pgfqpoint{1.250000in}{0.550000in}}{\pgfqpoint{7.750000in}{3.850000in}}%
\pgfusepath{clip}%
\pgfsetbuttcap%
\pgfsetmiterjoin%
\definecolor{currentfill}{rgb}{0.501961,0.501961,0.501961}%
\pgfsetfillcolor{currentfill}%
\pgfsetfillopacity{0.700000}%
\pgfsetlinewidth{0.000000pt}%
\definecolor{currentstroke}{rgb}{0.000000,0.000000,0.000000}%
\pgfsetstrokecolor{currentstroke}%
\pgfsetstrokeopacity{0.700000}%
\pgfsetdash{}{0pt}%
\pgfpathmoveto{\pgfqpoint{6.308416in}{0.550000in}}%
\pgfpathlineto{\pgfqpoint{6.335937in}{0.550000in}}%
\pgfpathlineto{\pgfqpoint{6.335937in}{0.550000in}}%
\pgfpathlineto{\pgfqpoint{6.308416in}{0.550000in}}%
\pgfpathlineto{\pgfqpoint{6.308416in}{0.550000in}}%
\pgfpathclose%
\pgfusepath{fill}%
\end{pgfscope}%
\begin{pgfscope}%
\pgfpathrectangle{\pgfqpoint{1.250000in}{0.550000in}}{\pgfqpoint{7.750000in}{3.850000in}}%
\pgfusepath{clip}%
\pgfsetbuttcap%
\pgfsetmiterjoin%
\definecolor{currentfill}{rgb}{0.501961,0.501961,0.501961}%
\pgfsetfillcolor{currentfill}%
\pgfsetfillopacity{0.700000}%
\pgfsetlinewidth{0.000000pt}%
\definecolor{currentstroke}{rgb}{0.000000,0.000000,0.000000}%
\pgfsetstrokecolor{currentstroke}%
\pgfsetstrokeopacity{0.700000}%
\pgfsetdash{}{0pt}%
\pgfpathmoveto{\pgfqpoint{6.335938in}{0.550000in}}%
\pgfpathlineto{\pgfqpoint{6.363459in}{0.550000in}}%
\pgfpathlineto{\pgfqpoint{6.363459in}{1.110264in}}%
\pgfpathlineto{\pgfqpoint{6.335938in}{1.110264in}}%
\pgfpathlineto{\pgfqpoint{6.335938in}{0.550000in}}%
\pgfpathclose%
\pgfusepath{fill}%
\end{pgfscope}%
\begin{pgfscope}%
\pgfpathrectangle{\pgfqpoint{1.250000in}{0.550000in}}{\pgfqpoint{7.750000in}{3.850000in}}%
\pgfusepath{clip}%
\pgfsetbuttcap%
\pgfsetmiterjoin%
\definecolor{currentfill}{rgb}{0.501961,0.501961,0.501961}%
\pgfsetfillcolor{currentfill}%
\pgfsetfillopacity{0.700000}%
\pgfsetlinewidth{0.000000pt}%
\definecolor{currentstroke}{rgb}{0.000000,0.000000,0.000000}%
\pgfsetstrokecolor{currentstroke}%
\pgfsetstrokeopacity{0.700000}%
\pgfsetdash{}{0pt}%
\pgfpathmoveto{\pgfqpoint{6.363459in}{0.550000in}}%
\pgfpathlineto{\pgfqpoint{6.390980in}{0.550000in}}%
\pgfpathlineto{\pgfqpoint{6.390980in}{0.550000in}}%
\pgfpathlineto{\pgfqpoint{6.363459in}{0.550000in}}%
\pgfpathlineto{\pgfqpoint{6.363459in}{0.550000in}}%
\pgfpathclose%
\pgfusepath{fill}%
\end{pgfscope}%
\begin{pgfscope}%
\pgfpathrectangle{\pgfqpoint{1.250000in}{0.550000in}}{\pgfqpoint{7.750000in}{3.850000in}}%
\pgfusepath{clip}%
\pgfsetbuttcap%
\pgfsetmiterjoin%
\definecolor{currentfill}{rgb}{0.501961,0.501961,0.501961}%
\pgfsetfillcolor{currentfill}%
\pgfsetfillopacity{0.700000}%
\pgfsetlinewidth{0.000000pt}%
\definecolor{currentstroke}{rgb}{0.000000,0.000000,0.000000}%
\pgfsetstrokecolor{currentstroke}%
\pgfsetstrokeopacity{0.700000}%
\pgfsetdash{}{0pt}%
\pgfpathmoveto{\pgfqpoint{6.390980in}{0.550000in}}%
\pgfpathlineto{\pgfqpoint{6.418501in}{0.550000in}}%
\pgfpathlineto{\pgfqpoint{6.418501in}{1.125946in}}%
\pgfpathlineto{\pgfqpoint{6.390980in}{1.125946in}}%
\pgfpathlineto{\pgfqpoint{6.390980in}{0.550000in}}%
\pgfpathclose%
\pgfusepath{fill}%
\end{pgfscope}%
\begin{pgfscope}%
\pgfpathrectangle{\pgfqpoint{1.250000in}{0.550000in}}{\pgfqpoint{7.750000in}{3.850000in}}%
\pgfusepath{clip}%
\pgfsetbuttcap%
\pgfsetmiterjoin%
\definecolor{currentfill}{rgb}{0.501961,0.501961,0.501961}%
\pgfsetfillcolor{currentfill}%
\pgfsetfillopacity{0.700000}%
\pgfsetlinewidth{0.000000pt}%
\definecolor{currentstroke}{rgb}{0.000000,0.000000,0.000000}%
\pgfsetstrokecolor{currentstroke}%
\pgfsetstrokeopacity{0.700000}%
\pgfsetdash{}{0pt}%
\pgfpathmoveto{\pgfqpoint{6.418501in}{0.550000in}}%
\pgfpathlineto{\pgfqpoint{6.446023in}{0.550000in}}%
\pgfpathlineto{\pgfqpoint{6.446023in}{0.550000in}}%
\pgfpathlineto{\pgfqpoint{6.418501in}{0.550000in}}%
\pgfpathlineto{\pgfqpoint{6.418501in}{0.550000in}}%
\pgfpathclose%
\pgfusepath{fill}%
\end{pgfscope}%
\begin{pgfscope}%
\pgfpathrectangle{\pgfqpoint{1.250000in}{0.550000in}}{\pgfqpoint{7.750000in}{3.850000in}}%
\pgfusepath{clip}%
\pgfsetbuttcap%
\pgfsetmiterjoin%
\definecolor{currentfill}{rgb}{0.501961,0.501961,0.501961}%
\pgfsetfillcolor{currentfill}%
\pgfsetfillopacity{0.700000}%
\pgfsetlinewidth{0.000000pt}%
\definecolor{currentstroke}{rgb}{0.000000,0.000000,0.000000}%
\pgfsetstrokecolor{currentstroke}%
\pgfsetstrokeopacity{0.700000}%
\pgfsetdash{}{0pt}%
\pgfpathmoveto{\pgfqpoint{6.446023in}{0.550000in}}%
\pgfpathlineto{\pgfqpoint{6.473544in}{0.550000in}}%
\pgfpathlineto{\pgfqpoint{6.473544in}{1.086029in}}%
\pgfpathlineto{\pgfqpoint{6.446023in}{1.086029in}}%
\pgfpathlineto{\pgfqpoint{6.446023in}{0.550000in}}%
\pgfpathclose%
\pgfusepath{fill}%
\end{pgfscope}%
\begin{pgfscope}%
\pgfpathrectangle{\pgfqpoint{1.250000in}{0.550000in}}{\pgfqpoint{7.750000in}{3.850000in}}%
\pgfusepath{clip}%
\pgfsetbuttcap%
\pgfsetmiterjoin%
\definecolor{currentfill}{rgb}{0.501961,0.501961,0.501961}%
\pgfsetfillcolor{currentfill}%
\pgfsetfillopacity{0.700000}%
\pgfsetlinewidth{0.000000pt}%
\definecolor{currentstroke}{rgb}{0.000000,0.000000,0.000000}%
\pgfsetstrokecolor{currentstroke}%
\pgfsetstrokeopacity{0.700000}%
\pgfsetdash{}{0pt}%
\pgfpathmoveto{\pgfqpoint{6.473544in}{0.550000in}}%
\pgfpathlineto{\pgfqpoint{6.501065in}{0.550000in}}%
\pgfpathlineto{\pgfqpoint{6.501065in}{0.550000in}}%
\pgfpathlineto{\pgfqpoint{6.473544in}{0.550000in}}%
\pgfpathlineto{\pgfqpoint{6.473544in}{0.550000in}}%
\pgfpathclose%
\pgfusepath{fill}%
\end{pgfscope}%
\begin{pgfscope}%
\pgfpathrectangle{\pgfqpoint{1.250000in}{0.550000in}}{\pgfqpoint{7.750000in}{3.850000in}}%
\pgfusepath{clip}%
\pgfsetbuttcap%
\pgfsetmiterjoin%
\definecolor{currentfill}{rgb}{0.501961,0.501961,0.501961}%
\pgfsetfillcolor{currentfill}%
\pgfsetfillopacity{0.700000}%
\pgfsetlinewidth{0.000000pt}%
\definecolor{currentstroke}{rgb}{0.000000,0.000000,0.000000}%
\pgfsetstrokecolor{currentstroke}%
\pgfsetstrokeopacity{0.700000}%
\pgfsetdash{}{0pt}%
\pgfpathmoveto{\pgfqpoint{6.501065in}{0.550000in}}%
\pgfpathlineto{\pgfqpoint{6.528587in}{0.550000in}}%
\pgfpathlineto{\pgfqpoint{6.528587in}{1.103136in}}%
\pgfpathlineto{\pgfqpoint{6.501065in}{1.103136in}}%
\pgfpathlineto{\pgfqpoint{6.501065in}{0.550000in}}%
\pgfpathclose%
\pgfusepath{fill}%
\end{pgfscope}%
\begin{pgfscope}%
\pgfpathrectangle{\pgfqpoint{1.250000in}{0.550000in}}{\pgfqpoint{7.750000in}{3.850000in}}%
\pgfusepath{clip}%
\pgfsetbuttcap%
\pgfsetmiterjoin%
\definecolor{currentfill}{rgb}{0.501961,0.501961,0.501961}%
\pgfsetfillcolor{currentfill}%
\pgfsetfillopacity{0.700000}%
\pgfsetlinewidth{0.000000pt}%
\definecolor{currentstroke}{rgb}{0.000000,0.000000,0.000000}%
\pgfsetstrokecolor{currentstroke}%
\pgfsetstrokeopacity{0.700000}%
\pgfsetdash{}{0pt}%
\pgfpathmoveto{\pgfqpoint{6.528587in}{0.550000in}}%
\pgfpathlineto{\pgfqpoint{6.556108in}{0.550000in}}%
\pgfpathlineto{\pgfqpoint{6.556108in}{0.550000in}}%
\pgfpathlineto{\pgfqpoint{6.528587in}{0.550000in}}%
\pgfpathlineto{\pgfqpoint{6.528587in}{0.550000in}}%
\pgfpathclose%
\pgfusepath{fill}%
\end{pgfscope}%
\begin{pgfscope}%
\pgfpathrectangle{\pgfqpoint{1.250000in}{0.550000in}}{\pgfqpoint{7.750000in}{3.850000in}}%
\pgfusepath{clip}%
\pgfsetbuttcap%
\pgfsetmiterjoin%
\definecolor{currentfill}{rgb}{0.501961,0.501961,0.501961}%
\pgfsetfillcolor{currentfill}%
\pgfsetfillopacity{0.700000}%
\pgfsetlinewidth{0.000000pt}%
\definecolor{currentstroke}{rgb}{0.000000,0.000000,0.000000}%
\pgfsetstrokecolor{currentstroke}%
\pgfsetstrokeopacity{0.700000}%
\pgfsetdash{}{0pt}%
\pgfpathmoveto{\pgfqpoint{6.556108in}{0.550000in}}%
\pgfpathlineto{\pgfqpoint{6.583629in}{0.550000in}}%
\pgfpathlineto{\pgfqpoint{6.583629in}{1.044686in}}%
\pgfpathlineto{\pgfqpoint{6.556108in}{1.044686in}}%
\pgfpathlineto{\pgfqpoint{6.556108in}{0.550000in}}%
\pgfpathclose%
\pgfusepath{fill}%
\end{pgfscope}%
\begin{pgfscope}%
\pgfpathrectangle{\pgfqpoint{1.250000in}{0.550000in}}{\pgfqpoint{7.750000in}{3.850000in}}%
\pgfusepath{clip}%
\pgfsetbuttcap%
\pgfsetmiterjoin%
\definecolor{currentfill}{rgb}{0.501961,0.501961,0.501961}%
\pgfsetfillcolor{currentfill}%
\pgfsetfillopacity{0.700000}%
\pgfsetlinewidth{0.000000pt}%
\definecolor{currentstroke}{rgb}{0.000000,0.000000,0.000000}%
\pgfsetstrokecolor{currentstroke}%
\pgfsetstrokeopacity{0.700000}%
\pgfsetdash{}{0pt}%
\pgfpathmoveto{\pgfqpoint{6.583629in}{0.550000in}}%
\pgfpathlineto{\pgfqpoint{6.611151in}{0.550000in}}%
\pgfpathlineto{\pgfqpoint{6.611151in}{0.550000in}}%
\pgfpathlineto{\pgfqpoint{6.583629in}{0.550000in}}%
\pgfpathlineto{\pgfqpoint{6.583629in}{0.550000in}}%
\pgfpathclose%
\pgfusepath{fill}%
\end{pgfscope}%
\begin{pgfscope}%
\pgfpathrectangle{\pgfqpoint{1.250000in}{0.550000in}}{\pgfqpoint{7.750000in}{3.850000in}}%
\pgfusepath{clip}%
\pgfsetbuttcap%
\pgfsetmiterjoin%
\definecolor{currentfill}{rgb}{0.501961,0.501961,0.501961}%
\pgfsetfillcolor{currentfill}%
\pgfsetfillopacity{0.700000}%
\pgfsetlinewidth{0.000000pt}%
\definecolor{currentstroke}{rgb}{0.000000,0.000000,0.000000}%
\pgfsetstrokecolor{currentstroke}%
\pgfsetstrokeopacity{0.700000}%
\pgfsetdash{}{0pt}%
\pgfpathmoveto{\pgfqpoint{6.611151in}{0.550000in}}%
\pgfpathlineto{\pgfqpoint{6.638672in}{0.550000in}}%
\pgfpathlineto{\pgfqpoint{6.638672in}{1.033281in}}%
\pgfpathlineto{\pgfqpoint{6.611151in}{1.033281in}}%
\pgfpathlineto{\pgfqpoint{6.611151in}{0.550000in}}%
\pgfpathclose%
\pgfusepath{fill}%
\end{pgfscope}%
\begin{pgfscope}%
\pgfpathrectangle{\pgfqpoint{1.250000in}{0.550000in}}{\pgfqpoint{7.750000in}{3.850000in}}%
\pgfusepath{clip}%
\pgfsetbuttcap%
\pgfsetmiterjoin%
\definecolor{currentfill}{rgb}{0.501961,0.501961,0.501961}%
\pgfsetfillcolor{currentfill}%
\pgfsetfillopacity{0.700000}%
\pgfsetlinewidth{0.000000pt}%
\definecolor{currentstroke}{rgb}{0.000000,0.000000,0.000000}%
\pgfsetstrokecolor{currentstroke}%
\pgfsetstrokeopacity{0.700000}%
\pgfsetdash{}{0pt}%
\pgfpathmoveto{\pgfqpoint{6.638672in}{0.550000in}}%
\pgfpathlineto{\pgfqpoint{6.666193in}{0.550000in}}%
\pgfpathlineto{\pgfqpoint{6.666193in}{0.550000in}}%
\pgfpathlineto{\pgfqpoint{6.638672in}{0.550000in}}%
\pgfpathlineto{\pgfqpoint{6.638672in}{0.550000in}}%
\pgfpathclose%
\pgfusepath{fill}%
\end{pgfscope}%
\begin{pgfscope}%
\pgfpathrectangle{\pgfqpoint{1.250000in}{0.550000in}}{\pgfqpoint{7.750000in}{3.850000in}}%
\pgfusepath{clip}%
\pgfsetbuttcap%
\pgfsetmiterjoin%
\definecolor{currentfill}{rgb}{0.501961,0.501961,0.501961}%
\pgfsetfillcolor{currentfill}%
\pgfsetfillopacity{0.700000}%
\pgfsetlinewidth{0.000000pt}%
\definecolor{currentstroke}{rgb}{0.000000,0.000000,0.000000}%
\pgfsetstrokecolor{currentstroke}%
\pgfsetstrokeopacity{0.700000}%
\pgfsetdash{}{0pt}%
\pgfpathmoveto{\pgfqpoint{6.666193in}{0.550000in}}%
\pgfpathlineto{\pgfqpoint{6.693714in}{0.550000in}}%
\pgfpathlineto{\pgfqpoint{6.693714in}{1.063219in}}%
\pgfpathlineto{\pgfqpoint{6.666193in}{1.063219in}}%
\pgfpathlineto{\pgfqpoint{6.666193in}{0.550000in}}%
\pgfpathclose%
\pgfusepath{fill}%
\end{pgfscope}%
\begin{pgfscope}%
\pgfpathrectangle{\pgfqpoint{1.250000in}{0.550000in}}{\pgfqpoint{7.750000in}{3.850000in}}%
\pgfusepath{clip}%
\pgfsetbuttcap%
\pgfsetmiterjoin%
\definecolor{currentfill}{rgb}{0.501961,0.501961,0.501961}%
\pgfsetfillcolor{currentfill}%
\pgfsetfillopacity{0.700000}%
\pgfsetlinewidth{0.000000pt}%
\definecolor{currentstroke}{rgb}{0.000000,0.000000,0.000000}%
\pgfsetstrokecolor{currentstroke}%
\pgfsetstrokeopacity{0.700000}%
\pgfsetdash{}{0pt}%
\pgfpathmoveto{\pgfqpoint{6.693714in}{0.550000in}}%
\pgfpathlineto{\pgfqpoint{6.721236in}{0.550000in}}%
\pgfpathlineto{\pgfqpoint{6.721236in}{0.550000in}}%
\pgfpathlineto{\pgfqpoint{6.693714in}{0.550000in}}%
\pgfpathlineto{\pgfqpoint{6.693714in}{0.550000in}}%
\pgfpathclose%
\pgfusepath{fill}%
\end{pgfscope}%
\begin{pgfscope}%
\pgfpathrectangle{\pgfqpoint{1.250000in}{0.550000in}}{\pgfqpoint{7.750000in}{3.850000in}}%
\pgfusepath{clip}%
\pgfsetbuttcap%
\pgfsetmiterjoin%
\definecolor{currentfill}{rgb}{0.501961,0.501961,0.501961}%
\pgfsetfillcolor{currentfill}%
\pgfsetfillopacity{0.700000}%
\pgfsetlinewidth{0.000000pt}%
\definecolor{currentstroke}{rgb}{0.000000,0.000000,0.000000}%
\pgfsetstrokecolor{currentstroke}%
\pgfsetstrokeopacity{0.700000}%
\pgfsetdash{}{0pt}%
\pgfpathmoveto{\pgfqpoint{6.721236in}{0.550000in}}%
\pgfpathlineto{\pgfqpoint{6.748757in}{0.550000in}}%
\pgfpathlineto{\pgfqpoint{6.748757in}{1.017600in}}%
\pgfpathlineto{\pgfqpoint{6.721236in}{1.017600in}}%
\pgfpathlineto{\pgfqpoint{6.721236in}{0.550000in}}%
\pgfpathclose%
\pgfusepath{fill}%
\end{pgfscope}%
\begin{pgfscope}%
\pgfpathrectangle{\pgfqpoint{1.250000in}{0.550000in}}{\pgfqpoint{7.750000in}{3.850000in}}%
\pgfusepath{clip}%
\pgfsetbuttcap%
\pgfsetmiterjoin%
\definecolor{currentfill}{rgb}{0.501961,0.501961,0.501961}%
\pgfsetfillcolor{currentfill}%
\pgfsetfillopacity{0.700000}%
\pgfsetlinewidth{0.000000pt}%
\definecolor{currentstroke}{rgb}{0.000000,0.000000,0.000000}%
\pgfsetstrokecolor{currentstroke}%
\pgfsetstrokeopacity{0.700000}%
\pgfsetdash{}{0pt}%
\pgfpathmoveto{\pgfqpoint{6.748757in}{0.550000in}}%
\pgfpathlineto{\pgfqpoint{6.776278in}{0.550000in}}%
\pgfpathlineto{\pgfqpoint{6.776278in}{0.550000in}}%
\pgfpathlineto{\pgfqpoint{6.748757in}{0.550000in}}%
\pgfpathlineto{\pgfqpoint{6.748757in}{0.550000in}}%
\pgfpathclose%
\pgfusepath{fill}%
\end{pgfscope}%
\begin{pgfscope}%
\pgfpathrectangle{\pgfqpoint{1.250000in}{0.550000in}}{\pgfqpoint{7.750000in}{3.850000in}}%
\pgfusepath{clip}%
\pgfsetbuttcap%
\pgfsetmiterjoin%
\definecolor{currentfill}{rgb}{0.501961,0.501961,0.501961}%
\pgfsetfillcolor{currentfill}%
\pgfsetfillopacity{0.700000}%
\pgfsetlinewidth{0.000000pt}%
\definecolor{currentstroke}{rgb}{0.000000,0.000000,0.000000}%
\pgfsetstrokecolor{currentstroke}%
\pgfsetstrokeopacity{0.700000}%
\pgfsetdash{}{0pt}%
\pgfpathmoveto{\pgfqpoint{6.776278in}{0.550000in}}%
\pgfpathlineto{\pgfqpoint{6.803800in}{0.550000in}}%
\pgfpathlineto{\pgfqpoint{6.803800in}{0.977683in}}%
\pgfpathlineto{\pgfqpoint{6.776278in}{0.977683in}}%
\pgfpathlineto{\pgfqpoint{6.776278in}{0.550000in}}%
\pgfpathclose%
\pgfusepath{fill}%
\end{pgfscope}%
\begin{pgfscope}%
\pgfpathrectangle{\pgfqpoint{1.250000in}{0.550000in}}{\pgfqpoint{7.750000in}{3.850000in}}%
\pgfusepath{clip}%
\pgfsetbuttcap%
\pgfsetmiterjoin%
\definecolor{currentfill}{rgb}{0.501961,0.501961,0.501961}%
\pgfsetfillcolor{currentfill}%
\pgfsetfillopacity{0.700000}%
\pgfsetlinewidth{0.000000pt}%
\definecolor{currentstroke}{rgb}{0.000000,0.000000,0.000000}%
\pgfsetstrokecolor{currentstroke}%
\pgfsetstrokeopacity{0.700000}%
\pgfsetdash{}{0pt}%
\pgfpathmoveto{\pgfqpoint{6.803800in}{0.550000in}}%
\pgfpathlineto{\pgfqpoint{6.831321in}{0.550000in}}%
\pgfpathlineto{\pgfqpoint{6.831321in}{0.550000in}}%
\pgfpathlineto{\pgfqpoint{6.803800in}{0.550000in}}%
\pgfpathlineto{\pgfqpoint{6.803800in}{0.550000in}}%
\pgfpathclose%
\pgfusepath{fill}%
\end{pgfscope}%
\begin{pgfscope}%
\pgfpathrectangle{\pgfqpoint{1.250000in}{0.550000in}}{\pgfqpoint{7.750000in}{3.850000in}}%
\pgfusepath{clip}%
\pgfsetbuttcap%
\pgfsetmiterjoin%
\definecolor{currentfill}{rgb}{0.501961,0.501961,0.501961}%
\pgfsetfillcolor{currentfill}%
\pgfsetfillopacity{0.700000}%
\pgfsetlinewidth{0.000000pt}%
\definecolor{currentstroke}{rgb}{0.000000,0.000000,0.000000}%
\pgfsetstrokecolor{currentstroke}%
\pgfsetstrokeopacity{0.700000}%
\pgfsetdash{}{0pt}%
\pgfpathmoveto{\pgfqpoint{6.831321in}{0.550000in}}%
\pgfpathlineto{\pgfqpoint{6.858842in}{0.550000in}}%
\pgfpathlineto{\pgfqpoint{6.858842in}{1.001918in}}%
\pgfpathlineto{\pgfqpoint{6.831321in}{1.001918in}}%
\pgfpathlineto{\pgfqpoint{6.831321in}{0.550000in}}%
\pgfpathclose%
\pgfusepath{fill}%
\end{pgfscope}%
\begin{pgfscope}%
\pgfpathrectangle{\pgfqpoint{1.250000in}{0.550000in}}{\pgfqpoint{7.750000in}{3.850000in}}%
\pgfusepath{clip}%
\pgfsetbuttcap%
\pgfsetmiterjoin%
\definecolor{currentfill}{rgb}{0.501961,0.501961,0.501961}%
\pgfsetfillcolor{currentfill}%
\pgfsetfillopacity{0.700000}%
\pgfsetlinewidth{0.000000pt}%
\definecolor{currentstroke}{rgb}{0.000000,0.000000,0.000000}%
\pgfsetstrokecolor{currentstroke}%
\pgfsetstrokeopacity{0.700000}%
\pgfsetdash{}{0pt}%
\pgfpathmoveto{\pgfqpoint{6.858842in}{0.550000in}}%
\pgfpathlineto{\pgfqpoint{6.886364in}{0.550000in}}%
\pgfpathlineto{\pgfqpoint{6.886364in}{0.550000in}}%
\pgfpathlineto{\pgfqpoint{6.858842in}{0.550000in}}%
\pgfpathlineto{\pgfqpoint{6.858842in}{0.550000in}}%
\pgfpathclose%
\pgfusepath{fill}%
\end{pgfscope}%
\begin{pgfscope}%
\pgfpathrectangle{\pgfqpoint{1.250000in}{0.550000in}}{\pgfqpoint{7.750000in}{3.850000in}}%
\pgfusepath{clip}%
\pgfsetbuttcap%
\pgfsetmiterjoin%
\definecolor{currentfill}{rgb}{0.501961,0.501961,0.501961}%
\pgfsetfillcolor{currentfill}%
\pgfsetfillopacity{0.700000}%
\pgfsetlinewidth{0.000000pt}%
\definecolor{currentstroke}{rgb}{0.000000,0.000000,0.000000}%
\pgfsetstrokecolor{currentstroke}%
\pgfsetstrokeopacity{0.700000}%
\pgfsetdash{}{0pt}%
\pgfpathmoveto{\pgfqpoint{6.886364in}{0.550000in}}%
\pgfpathlineto{\pgfqpoint{6.913885in}{0.550000in}}%
\pgfpathlineto{\pgfqpoint{6.913885in}{1.009046in}}%
\pgfpathlineto{\pgfqpoint{6.886364in}{1.009046in}}%
\pgfpathlineto{\pgfqpoint{6.886364in}{0.550000in}}%
\pgfpathclose%
\pgfusepath{fill}%
\end{pgfscope}%
\begin{pgfscope}%
\pgfpathrectangle{\pgfqpoint{1.250000in}{0.550000in}}{\pgfqpoint{7.750000in}{3.850000in}}%
\pgfusepath{clip}%
\pgfsetbuttcap%
\pgfsetmiterjoin%
\definecolor{currentfill}{rgb}{0.501961,0.501961,0.501961}%
\pgfsetfillcolor{currentfill}%
\pgfsetfillopacity{0.700000}%
\pgfsetlinewidth{0.000000pt}%
\definecolor{currentstroke}{rgb}{0.000000,0.000000,0.000000}%
\pgfsetstrokecolor{currentstroke}%
\pgfsetstrokeopacity{0.700000}%
\pgfsetdash{}{0pt}%
\pgfpathmoveto{\pgfqpoint{6.913885in}{0.550000in}}%
\pgfpathlineto{\pgfqpoint{6.941406in}{0.550000in}}%
\pgfpathlineto{\pgfqpoint{6.941406in}{0.550000in}}%
\pgfpathlineto{\pgfqpoint{6.913885in}{0.550000in}}%
\pgfpathlineto{\pgfqpoint{6.913885in}{0.550000in}}%
\pgfpathclose%
\pgfusepath{fill}%
\end{pgfscope}%
\begin{pgfscope}%
\pgfpathrectangle{\pgfqpoint{1.250000in}{0.550000in}}{\pgfqpoint{7.750000in}{3.850000in}}%
\pgfusepath{clip}%
\pgfsetbuttcap%
\pgfsetmiterjoin%
\definecolor{currentfill}{rgb}{0.501961,0.501961,0.501961}%
\pgfsetfillcolor{currentfill}%
\pgfsetfillopacity{0.700000}%
\pgfsetlinewidth{0.000000pt}%
\definecolor{currentstroke}{rgb}{0.000000,0.000000,0.000000}%
\pgfsetstrokecolor{currentstroke}%
\pgfsetstrokeopacity{0.700000}%
\pgfsetdash{}{0pt}%
\pgfpathmoveto{\pgfqpoint{6.941406in}{0.550000in}}%
\pgfpathlineto{\pgfqpoint{6.968928in}{0.550000in}}%
\pgfpathlineto{\pgfqpoint{6.968928in}{0.990513in}}%
\pgfpathlineto{\pgfqpoint{6.941406in}{0.990513in}}%
\pgfpathlineto{\pgfqpoint{6.941406in}{0.550000in}}%
\pgfpathclose%
\pgfusepath{fill}%
\end{pgfscope}%
\begin{pgfscope}%
\pgfpathrectangle{\pgfqpoint{1.250000in}{0.550000in}}{\pgfqpoint{7.750000in}{3.850000in}}%
\pgfusepath{clip}%
\pgfsetbuttcap%
\pgfsetmiterjoin%
\definecolor{currentfill}{rgb}{0.501961,0.501961,0.501961}%
\pgfsetfillcolor{currentfill}%
\pgfsetfillopacity{0.700000}%
\pgfsetlinewidth{0.000000pt}%
\definecolor{currentstroke}{rgb}{0.000000,0.000000,0.000000}%
\pgfsetstrokecolor{currentstroke}%
\pgfsetstrokeopacity{0.700000}%
\pgfsetdash{}{0pt}%
\pgfpathmoveto{\pgfqpoint{6.968928in}{0.550000in}}%
\pgfpathlineto{\pgfqpoint{6.996449in}{0.550000in}}%
\pgfpathlineto{\pgfqpoint{6.996449in}{0.550000in}}%
\pgfpathlineto{\pgfqpoint{6.968928in}{0.550000in}}%
\pgfpathlineto{\pgfqpoint{6.968928in}{0.550000in}}%
\pgfpathclose%
\pgfusepath{fill}%
\end{pgfscope}%
\begin{pgfscope}%
\pgfpathrectangle{\pgfqpoint{1.250000in}{0.550000in}}{\pgfqpoint{7.750000in}{3.850000in}}%
\pgfusepath{clip}%
\pgfsetbuttcap%
\pgfsetmiterjoin%
\definecolor{currentfill}{rgb}{0.501961,0.501961,0.501961}%
\pgfsetfillcolor{currentfill}%
\pgfsetfillopacity{0.700000}%
\pgfsetlinewidth{0.000000pt}%
\definecolor{currentstroke}{rgb}{0.000000,0.000000,0.000000}%
\pgfsetstrokecolor{currentstroke}%
\pgfsetstrokeopacity{0.700000}%
\pgfsetdash{}{0pt}%
\pgfpathmoveto{\pgfqpoint{6.996449in}{0.550000in}}%
\pgfpathlineto{\pgfqpoint{7.023970in}{0.550000in}}%
\pgfpathlineto{\pgfqpoint{7.023970in}{0.943468in}}%
\pgfpathlineto{\pgfqpoint{6.996449in}{0.943468in}}%
\pgfpathlineto{\pgfqpoint{6.996449in}{0.550000in}}%
\pgfpathclose%
\pgfusepath{fill}%
\end{pgfscope}%
\begin{pgfscope}%
\pgfpathrectangle{\pgfqpoint{1.250000in}{0.550000in}}{\pgfqpoint{7.750000in}{3.850000in}}%
\pgfusepath{clip}%
\pgfsetbuttcap%
\pgfsetmiterjoin%
\definecolor{currentfill}{rgb}{0.501961,0.501961,0.501961}%
\pgfsetfillcolor{currentfill}%
\pgfsetfillopacity{0.700000}%
\pgfsetlinewidth{0.000000pt}%
\definecolor{currentstroke}{rgb}{0.000000,0.000000,0.000000}%
\pgfsetstrokecolor{currentstroke}%
\pgfsetstrokeopacity{0.700000}%
\pgfsetdash{}{0pt}%
\pgfpathmoveto{\pgfqpoint{7.023970in}{0.550000in}}%
\pgfpathlineto{\pgfqpoint{7.051491in}{0.550000in}}%
\pgfpathlineto{\pgfqpoint{7.051491in}{0.550000in}}%
\pgfpathlineto{\pgfqpoint{7.023970in}{0.550000in}}%
\pgfpathlineto{\pgfqpoint{7.023970in}{0.550000in}}%
\pgfpathclose%
\pgfusepath{fill}%
\end{pgfscope}%
\begin{pgfscope}%
\pgfpathrectangle{\pgfqpoint{1.250000in}{0.550000in}}{\pgfqpoint{7.750000in}{3.850000in}}%
\pgfusepath{clip}%
\pgfsetbuttcap%
\pgfsetmiterjoin%
\definecolor{currentfill}{rgb}{0.501961,0.501961,0.501961}%
\pgfsetfillcolor{currentfill}%
\pgfsetfillopacity{0.700000}%
\pgfsetlinewidth{0.000000pt}%
\definecolor{currentstroke}{rgb}{0.000000,0.000000,0.000000}%
\pgfsetstrokecolor{currentstroke}%
\pgfsetstrokeopacity{0.700000}%
\pgfsetdash{}{0pt}%
\pgfpathmoveto{\pgfqpoint{7.051491in}{0.550000in}}%
\pgfpathlineto{\pgfqpoint{7.079013in}{0.550000in}}%
\pgfpathlineto{\pgfqpoint{7.079013in}{0.974832in}}%
\pgfpathlineto{\pgfqpoint{7.051491in}{0.974832in}}%
\pgfpathlineto{\pgfqpoint{7.051491in}{0.550000in}}%
\pgfpathclose%
\pgfusepath{fill}%
\end{pgfscope}%
\begin{pgfscope}%
\pgfpathrectangle{\pgfqpoint{1.250000in}{0.550000in}}{\pgfqpoint{7.750000in}{3.850000in}}%
\pgfusepath{clip}%
\pgfsetbuttcap%
\pgfsetmiterjoin%
\definecolor{currentfill}{rgb}{0.501961,0.501961,0.501961}%
\pgfsetfillcolor{currentfill}%
\pgfsetfillopacity{0.700000}%
\pgfsetlinewidth{0.000000pt}%
\definecolor{currentstroke}{rgb}{0.000000,0.000000,0.000000}%
\pgfsetstrokecolor{currentstroke}%
\pgfsetstrokeopacity{0.700000}%
\pgfsetdash{}{0pt}%
\pgfpathmoveto{\pgfqpoint{7.079013in}{0.550000in}}%
\pgfpathlineto{\pgfqpoint{7.106534in}{0.550000in}}%
\pgfpathlineto{\pgfqpoint{7.106534in}{0.550000in}}%
\pgfpathlineto{\pgfqpoint{7.079013in}{0.550000in}}%
\pgfpathlineto{\pgfqpoint{7.079013in}{0.550000in}}%
\pgfpathclose%
\pgfusepath{fill}%
\end{pgfscope}%
\begin{pgfscope}%
\pgfpathrectangle{\pgfqpoint{1.250000in}{0.550000in}}{\pgfqpoint{7.750000in}{3.850000in}}%
\pgfusepath{clip}%
\pgfsetbuttcap%
\pgfsetmiterjoin%
\definecolor{currentfill}{rgb}{0.501961,0.501961,0.501961}%
\pgfsetfillcolor{currentfill}%
\pgfsetfillopacity{0.700000}%
\pgfsetlinewidth{0.000000pt}%
\definecolor{currentstroke}{rgb}{0.000000,0.000000,0.000000}%
\pgfsetstrokecolor{currentstroke}%
\pgfsetstrokeopacity{0.700000}%
\pgfsetdash{}{0pt}%
\pgfpathmoveto{\pgfqpoint{7.106534in}{0.550000in}}%
\pgfpathlineto{\pgfqpoint{7.134055in}{0.550000in}}%
\pgfpathlineto{\pgfqpoint{7.134055in}{0.953447in}}%
\pgfpathlineto{\pgfqpoint{7.106534in}{0.953447in}}%
\pgfpathlineto{\pgfqpoint{7.106534in}{0.550000in}}%
\pgfpathclose%
\pgfusepath{fill}%
\end{pgfscope}%
\begin{pgfscope}%
\pgfpathrectangle{\pgfqpoint{1.250000in}{0.550000in}}{\pgfqpoint{7.750000in}{3.850000in}}%
\pgfusepath{clip}%
\pgfsetbuttcap%
\pgfsetmiterjoin%
\definecolor{currentfill}{rgb}{0.501961,0.501961,0.501961}%
\pgfsetfillcolor{currentfill}%
\pgfsetfillopacity{0.700000}%
\pgfsetlinewidth{0.000000pt}%
\definecolor{currentstroke}{rgb}{0.000000,0.000000,0.000000}%
\pgfsetstrokecolor{currentstroke}%
\pgfsetstrokeopacity{0.700000}%
\pgfsetdash{}{0pt}%
\pgfpathmoveto{\pgfqpoint{7.134055in}{0.550000in}}%
\pgfpathlineto{\pgfqpoint{7.161577in}{0.550000in}}%
\pgfpathlineto{\pgfqpoint{7.161577in}{0.550000in}}%
\pgfpathlineto{\pgfqpoint{7.134055in}{0.550000in}}%
\pgfpathlineto{\pgfqpoint{7.134055in}{0.550000in}}%
\pgfpathclose%
\pgfusepath{fill}%
\end{pgfscope}%
\begin{pgfscope}%
\pgfpathrectangle{\pgfqpoint{1.250000in}{0.550000in}}{\pgfqpoint{7.750000in}{3.850000in}}%
\pgfusepath{clip}%
\pgfsetbuttcap%
\pgfsetmiterjoin%
\definecolor{currentfill}{rgb}{0.501961,0.501961,0.501961}%
\pgfsetfillcolor{currentfill}%
\pgfsetfillopacity{0.700000}%
\pgfsetlinewidth{0.000000pt}%
\definecolor{currentstroke}{rgb}{0.000000,0.000000,0.000000}%
\pgfsetstrokecolor{currentstroke}%
\pgfsetstrokeopacity{0.700000}%
\pgfsetdash{}{0pt}%
\pgfpathmoveto{\pgfqpoint{7.161577in}{0.550000in}}%
\pgfpathlineto{\pgfqpoint{7.189098in}{0.550000in}}%
\pgfpathlineto{\pgfqpoint{7.189098in}{0.949171in}}%
\pgfpathlineto{\pgfqpoint{7.161577in}{0.949171in}}%
\pgfpathlineto{\pgfqpoint{7.161577in}{0.550000in}}%
\pgfpathclose%
\pgfusepath{fill}%
\end{pgfscope}%
\begin{pgfscope}%
\pgfpathrectangle{\pgfqpoint{1.250000in}{0.550000in}}{\pgfqpoint{7.750000in}{3.850000in}}%
\pgfusepath{clip}%
\pgfsetbuttcap%
\pgfsetmiterjoin%
\definecolor{currentfill}{rgb}{0.501961,0.501961,0.501961}%
\pgfsetfillcolor{currentfill}%
\pgfsetfillopacity{0.700000}%
\pgfsetlinewidth{0.000000pt}%
\definecolor{currentstroke}{rgb}{0.000000,0.000000,0.000000}%
\pgfsetstrokecolor{currentstroke}%
\pgfsetstrokeopacity{0.700000}%
\pgfsetdash{}{0pt}%
\pgfpathmoveto{\pgfqpoint{7.189098in}{0.550000in}}%
\pgfpathlineto{\pgfqpoint{7.216619in}{0.550000in}}%
\pgfpathlineto{\pgfqpoint{7.216619in}{0.550000in}}%
\pgfpathlineto{\pgfqpoint{7.189098in}{0.550000in}}%
\pgfpathlineto{\pgfqpoint{7.189098in}{0.550000in}}%
\pgfpathclose%
\pgfusepath{fill}%
\end{pgfscope}%
\begin{pgfscope}%
\pgfpathrectangle{\pgfqpoint{1.250000in}{0.550000in}}{\pgfqpoint{7.750000in}{3.850000in}}%
\pgfusepath{clip}%
\pgfsetbuttcap%
\pgfsetmiterjoin%
\definecolor{currentfill}{rgb}{0.501961,0.501961,0.501961}%
\pgfsetfillcolor{currentfill}%
\pgfsetfillopacity{0.700000}%
\pgfsetlinewidth{0.000000pt}%
\definecolor{currentstroke}{rgb}{0.000000,0.000000,0.000000}%
\pgfsetstrokecolor{currentstroke}%
\pgfsetstrokeopacity{0.700000}%
\pgfsetdash{}{0pt}%
\pgfpathmoveto{\pgfqpoint{7.216619in}{0.550000in}}%
\pgfpathlineto{\pgfqpoint{7.244141in}{0.550000in}}%
\pgfpathlineto{\pgfqpoint{7.244141in}{0.934914in}}%
\pgfpathlineto{\pgfqpoint{7.216619in}{0.934914in}}%
\pgfpathlineto{\pgfqpoint{7.216619in}{0.550000in}}%
\pgfpathclose%
\pgfusepath{fill}%
\end{pgfscope}%
\begin{pgfscope}%
\pgfpathrectangle{\pgfqpoint{1.250000in}{0.550000in}}{\pgfqpoint{7.750000in}{3.850000in}}%
\pgfusepath{clip}%
\pgfsetbuttcap%
\pgfsetmiterjoin%
\definecolor{currentfill}{rgb}{0.501961,0.501961,0.501961}%
\pgfsetfillcolor{currentfill}%
\pgfsetfillopacity{0.700000}%
\pgfsetlinewidth{0.000000pt}%
\definecolor{currentstroke}{rgb}{0.000000,0.000000,0.000000}%
\pgfsetstrokecolor{currentstroke}%
\pgfsetstrokeopacity{0.700000}%
\pgfsetdash{}{0pt}%
\pgfpathmoveto{\pgfqpoint{7.244141in}{0.550000in}}%
\pgfpathlineto{\pgfqpoint{7.271662in}{0.550000in}}%
\pgfpathlineto{\pgfqpoint{7.271662in}{0.550000in}}%
\pgfpathlineto{\pgfqpoint{7.244141in}{0.550000in}}%
\pgfpathlineto{\pgfqpoint{7.244141in}{0.550000in}}%
\pgfpathclose%
\pgfusepath{fill}%
\end{pgfscope}%
\begin{pgfscope}%
\pgfpathrectangle{\pgfqpoint{1.250000in}{0.550000in}}{\pgfqpoint{7.750000in}{3.850000in}}%
\pgfusepath{clip}%
\pgfsetbuttcap%
\pgfsetmiterjoin%
\definecolor{currentfill}{rgb}{0.501961,0.501961,0.501961}%
\pgfsetfillcolor{currentfill}%
\pgfsetfillopacity{0.700000}%
\pgfsetlinewidth{0.000000pt}%
\definecolor{currentstroke}{rgb}{0.000000,0.000000,0.000000}%
\pgfsetstrokecolor{currentstroke}%
\pgfsetstrokeopacity{0.700000}%
\pgfsetdash{}{0pt}%
\pgfpathmoveto{\pgfqpoint{7.271662in}{0.550000in}}%
\pgfpathlineto{\pgfqpoint{7.299183in}{0.550000in}}%
\pgfpathlineto{\pgfqpoint{7.299183in}{0.855080in}}%
\pgfpathlineto{\pgfqpoint{7.271662in}{0.855080in}}%
\pgfpathlineto{\pgfqpoint{7.271662in}{0.550000in}}%
\pgfpathclose%
\pgfusepath{fill}%
\end{pgfscope}%
\begin{pgfscope}%
\pgfpathrectangle{\pgfqpoint{1.250000in}{0.550000in}}{\pgfqpoint{7.750000in}{3.850000in}}%
\pgfusepath{clip}%
\pgfsetbuttcap%
\pgfsetmiterjoin%
\definecolor{currentfill}{rgb}{0.501961,0.501961,0.501961}%
\pgfsetfillcolor{currentfill}%
\pgfsetfillopacity{0.700000}%
\pgfsetlinewidth{0.000000pt}%
\definecolor{currentstroke}{rgb}{0.000000,0.000000,0.000000}%
\pgfsetstrokecolor{currentstroke}%
\pgfsetstrokeopacity{0.700000}%
\pgfsetdash{}{0pt}%
\pgfpathmoveto{\pgfqpoint{7.299183in}{0.550000in}}%
\pgfpathlineto{\pgfqpoint{7.326705in}{0.550000in}}%
\pgfpathlineto{\pgfqpoint{7.326705in}{0.550000in}}%
\pgfpathlineto{\pgfqpoint{7.299183in}{0.550000in}}%
\pgfpathlineto{\pgfqpoint{7.299183in}{0.550000in}}%
\pgfpathclose%
\pgfusepath{fill}%
\end{pgfscope}%
\begin{pgfscope}%
\pgfpathrectangle{\pgfqpoint{1.250000in}{0.550000in}}{\pgfqpoint{7.750000in}{3.850000in}}%
\pgfusepath{clip}%
\pgfsetbuttcap%
\pgfsetmiterjoin%
\definecolor{currentfill}{rgb}{0.501961,0.501961,0.501961}%
\pgfsetfillcolor{currentfill}%
\pgfsetfillopacity{0.700000}%
\pgfsetlinewidth{0.000000pt}%
\definecolor{currentstroke}{rgb}{0.000000,0.000000,0.000000}%
\pgfsetstrokecolor{currentstroke}%
\pgfsetstrokeopacity{0.700000}%
\pgfsetdash{}{0pt}%
\pgfpathmoveto{\pgfqpoint{7.326705in}{0.550000in}}%
\pgfpathlineto{\pgfqpoint{7.354226in}{0.550000in}}%
\pgfpathlineto{\pgfqpoint{7.354226in}{0.904977in}}%
\pgfpathlineto{\pgfqpoint{7.326705in}{0.904977in}}%
\pgfpathlineto{\pgfqpoint{7.326705in}{0.550000in}}%
\pgfpathclose%
\pgfusepath{fill}%
\end{pgfscope}%
\begin{pgfscope}%
\pgfpathrectangle{\pgfqpoint{1.250000in}{0.550000in}}{\pgfqpoint{7.750000in}{3.850000in}}%
\pgfusepath{clip}%
\pgfsetbuttcap%
\pgfsetmiterjoin%
\definecolor{currentfill}{rgb}{0.501961,0.501961,0.501961}%
\pgfsetfillcolor{currentfill}%
\pgfsetfillopacity{0.700000}%
\pgfsetlinewidth{0.000000pt}%
\definecolor{currentstroke}{rgb}{0.000000,0.000000,0.000000}%
\pgfsetstrokecolor{currentstroke}%
\pgfsetstrokeopacity{0.700000}%
\pgfsetdash{}{0pt}%
\pgfpathmoveto{\pgfqpoint{7.354226in}{0.550000in}}%
\pgfpathlineto{\pgfqpoint{7.381747in}{0.550000in}}%
\pgfpathlineto{\pgfqpoint{7.381747in}{0.880741in}}%
\pgfpathlineto{\pgfqpoint{7.354226in}{0.880741in}}%
\pgfpathlineto{\pgfqpoint{7.354226in}{0.550000in}}%
\pgfpathclose%
\pgfusepath{fill}%
\end{pgfscope}%
\begin{pgfscope}%
\pgfpathrectangle{\pgfqpoint{1.250000in}{0.550000in}}{\pgfqpoint{7.750000in}{3.850000in}}%
\pgfusepath{clip}%
\pgfsetbuttcap%
\pgfsetmiterjoin%
\definecolor{currentfill}{rgb}{0.501961,0.501961,0.501961}%
\pgfsetfillcolor{currentfill}%
\pgfsetfillopacity{0.700000}%
\pgfsetlinewidth{0.000000pt}%
\definecolor{currentstroke}{rgb}{0.000000,0.000000,0.000000}%
\pgfsetstrokecolor{currentstroke}%
\pgfsetstrokeopacity{0.700000}%
\pgfsetdash{}{0pt}%
\pgfpathmoveto{\pgfqpoint{7.381747in}{0.550000in}}%
\pgfpathlineto{\pgfqpoint{7.409268in}{0.550000in}}%
\pgfpathlineto{\pgfqpoint{7.409268in}{0.550000in}}%
\pgfpathlineto{\pgfqpoint{7.381747in}{0.550000in}}%
\pgfpathlineto{\pgfqpoint{7.381747in}{0.550000in}}%
\pgfpathclose%
\pgfusepath{fill}%
\end{pgfscope}%
\begin{pgfscope}%
\pgfpathrectangle{\pgfqpoint{1.250000in}{0.550000in}}{\pgfqpoint{7.750000in}{3.850000in}}%
\pgfusepath{clip}%
\pgfsetbuttcap%
\pgfsetmiterjoin%
\definecolor{currentfill}{rgb}{0.501961,0.501961,0.501961}%
\pgfsetfillcolor{currentfill}%
\pgfsetfillopacity{0.700000}%
\pgfsetlinewidth{0.000000pt}%
\definecolor{currentstroke}{rgb}{0.000000,0.000000,0.000000}%
\pgfsetstrokecolor{currentstroke}%
\pgfsetstrokeopacity{0.700000}%
\pgfsetdash{}{0pt}%
\pgfpathmoveto{\pgfqpoint{7.409268in}{0.550000in}}%
\pgfpathlineto{\pgfqpoint{7.436790in}{0.550000in}}%
\pgfpathlineto{\pgfqpoint{7.436790in}{0.875039in}}%
\pgfpathlineto{\pgfqpoint{7.409268in}{0.875039in}}%
\pgfpathlineto{\pgfqpoint{7.409268in}{0.550000in}}%
\pgfpathclose%
\pgfusepath{fill}%
\end{pgfscope}%
\begin{pgfscope}%
\pgfpathrectangle{\pgfqpoint{1.250000in}{0.550000in}}{\pgfqpoint{7.750000in}{3.850000in}}%
\pgfusepath{clip}%
\pgfsetbuttcap%
\pgfsetmiterjoin%
\definecolor{currentfill}{rgb}{0.501961,0.501961,0.501961}%
\pgfsetfillcolor{currentfill}%
\pgfsetfillopacity{0.700000}%
\pgfsetlinewidth{0.000000pt}%
\definecolor{currentstroke}{rgb}{0.000000,0.000000,0.000000}%
\pgfsetstrokecolor{currentstroke}%
\pgfsetstrokeopacity{0.700000}%
\pgfsetdash{}{0pt}%
\pgfpathmoveto{\pgfqpoint{7.436790in}{0.550000in}}%
\pgfpathlineto{\pgfqpoint{7.464311in}{0.550000in}}%
\pgfpathlineto{\pgfqpoint{7.464311in}{0.550000in}}%
\pgfpathlineto{\pgfqpoint{7.436790in}{0.550000in}}%
\pgfpathlineto{\pgfqpoint{7.436790in}{0.550000in}}%
\pgfpathclose%
\pgfusepath{fill}%
\end{pgfscope}%
\begin{pgfscope}%
\pgfpathrectangle{\pgfqpoint{1.250000in}{0.550000in}}{\pgfqpoint{7.750000in}{3.850000in}}%
\pgfusepath{clip}%
\pgfsetbuttcap%
\pgfsetmiterjoin%
\definecolor{currentfill}{rgb}{0.501961,0.501961,0.501961}%
\pgfsetfillcolor{currentfill}%
\pgfsetfillopacity{0.700000}%
\pgfsetlinewidth{0.000000pt}%
\definecolor{currentstroke}{rgb}{0.000000,0.000000,0.000000}%
\pgfsetstrokecolor{currentstroke}%
\pgfsetstrokeopacity{0.700000}%
\pgfsetdash{}{0pt}%
\pgfpathmoveto{\pgfqpoint{7.464311in}{0.550000in}}%
\pgfpathlineto{\pgfqpoint{7.491832in}{0.550000in}}%
\pgfpathlineto{\pgfqpoint{7.491832in}{0.929212in}}%
\pgfpathlineto{\pgfqpoint{7.464311in}{0.929212in}}%
\pgfpathlineto{\pgfqpoint{7.464311in}{0.550000in}}%
\pgfpathclose%
\pgfusepath{fill}%
\end{pgfscope}%
\begin{pgfscope}%
\pgfpathrectangle{\pgfqpoint{1.250000in}{0.550000in}}{\pgfqpoint{7.750000in}{3.850000in}}%
\pgfusepath{clip}%
\pgfsetbuttcap%
\pgfsetmiterjoin%
\definecolor{currentfill}{rgb}{0.501961,0.501961,0.501961}%
\pgfsetfillcolor{currentfill}%
\pgfsetfillopacity{0.700000}%
\pgfsetlinewidth{0.000000pt}%
\definecolor{currentstroke}{rgb}{0.000000,0.000000,0.000000}%
\pgfsetstrokecolor{currentstroke}%
\pgfsetstrokeopacity{0.700000}%
\pgfsetdash{}{0pt}%
\pgfpathmoveto{\pgfqpoint{7.491832in}{0.550000in}}%
\pgfpathlineto{\pgfqpoint{7.519354in}{0.550000in}}%
\pgfpathlineto{\pgfqpoint{7.519354in}{0.550000in}}%
\pgfpathlineto{\pgfqpoint{7.491832in}{0.550000in}}%
\pgfpathlineto{\pgfqpoint{7.491832in}{0.550000in}}%
\pgfpathclose%
\pgfusepath{fill}%
\end{pgfscope}%
\begin{pgfscope}%
\pgfpathrectangle{\pgfqpoint{1.250000in}{0.550000in}}{\pgfqpoint{7.750000in}{3.850000in}}%
\pgfusepath{clip}%
\pgfsetbuttcap%
\pgfsetmiterjoin%
\definecolor{currentfill}{rgb}{0.501961,0.501961,0.501961}%
\pgfsetfillcolor{currentfill}%
\pgfsetfillopacity{0.700000}%
\pgfsetlinewidth{0.000000pt}%
\definecolor{currentstroke}{rgb}{0.000000,0.000000,0.000000}%
\pgfsetstrokecolor{currentstroke}%
\pgfsetstrokeopacity{0.700000}%
\pgfsetdash{}{0pt}%
\pgfpathmoveto{\pgfqpoint{7.519354in}{0.550000in}}%
\pgfpathlineto{\pgfqpoint{7.546875in}{0.550000in}}%
\pgfpathlineto{\pgfqpoint{7.546875in}{0.856506in}}%
\pgfpathlineto{\pgfqpoint{7.519354in}{0.856506in}}%
\pgfpathlineto{\pgfqpoint{7.519354in}{0.550000in}}%
\pgfpathclose%
\pgfusepath{fill}%
\end{pgfscope}%
\begin{pgfscope}%
\pgfpathrectangle{\pgfqpoint{1.250000in}{0.550000in}}{\pgfqpoint{7.750000in}{3.850000in}}%
\pgfusepath{clip}%
\pgfsetbuttcap%
\pgfsetmiterjoin%
\definecolor{currentfill}{rgb}{0.501961,0.501961,0.501961}%
\pgfsetfillcolor{currentfill}%
\pgfsetfillopacity{0.700000}%
\pgfsetlinewidth{0.000000pt}%
\definecolor{currentstroke}{rgb}{0.000000,0.000000,0.000000}%
\pgfsetstrokecolor{currentstroke}%
\pgfsetstrokeopacity{0.700000}%
\pgfsetdash{}{0pt}%
\pgfpathmoveto{\pgfqpoint{7.546875in}{0.550000in}}%
\pgfpathlineto{\pgfqpoint{7.574396in}{0.550000in}}%
\pgfpathlineto{\pgfqpoint{7.574396in}{0.805184in}}%
\pgfpathlineto{\pgfqpoint{7.546875in}{0.805184in}}%
\pgfpathlineto{\pgfqpoint{7.546875in}{0.550000in}}%
\pgfpathclose%
\pgfusepath{fill}%
\end{pgfscope}%
\begin{pgfscope}%
\pgfpathrectangle{\pgfqpoint{1.250000in}{0.550000in}}{\pgfqpoint{7.750000in}{3.850000in}}%
\pgfusepath{clip}%
\pgfsetbuttcap%
\pgfsetmiterjoin%
\definecolor{currentfill}{rgb}{0.501961,0.501961,0.501961}%
\pgfsetfillcolor{currentfill}%
\pgfsetfillopacity{0.700000}%
\pgfsetlinewidth{0.000000pt}%
\definecolor{currentstroke}{rgb}{0.000000,0.000000,0.000000}%
\pgfsetstrokecolor{currentstroke}%
\pgfsetstrokeopacity{0.700000}%
\pgfsetdash{}{0pt}%
\pgfpathmoveto{\pgfqpoint{7.574396in}{0.550000in}}%
\pgfpathlineto{\pgfqpoint{7.601918in}{0.550000in}}%
\pgfpathlineto{\pgfqpoint{7.601918in}{0.799482in}}%
\pgfpathlineto{\pgfqpoint{7.574396in}{0.799482in}}%
\pgfpathlineto{\pgfqpoint{7.574396in}{0.550000in}}%
\pgfpathclose%
\pgfusepath{fill}%
\end{pgfscope}%
\begin{pgfscope}%
\pgfpathrectangle{\pgfqpoint{1.250000in}{0.550000in}}{\pgfqpoint{7.750000in}{3.850000in}}%
\pgfusepath{clip}%
\pgfsetbuttcap%
\pgfsetmiterjoin%
\definecolor{currentfill}{rgb}{0.501961,0.501961,0.501961}%
\pgfsetfillcolor{currentfill}%
\pgfsetfillopacity{0.700000}%
\pgfsetlinewidth{0.000000pt}%
\definecolor{currentstroke}{rgb}{0.000000,0.000000,0.000000}%
\pgfsetstrokecolor{currentstroke}%
\pgfsetstrokeopacity{0.700000}%
\pgfsetdash{}{0pt}%
\pgfpathmoveto{\pgfqpoint{7.601918in}{0.550000in}}%
\pgfpathlineto{\pgfqpoint{7.629439in}{0.550000in}}%
\pgfpathlineto{\pgfqpoint{7.629439in}{0.550000in}}%
\pgfpathlineto{\pgfqpoint{7.601918in}{0.550000in}}%
\pgfpathlineto{\pgfqpoint{7.601918in}{0.550000in}}%
\pgfpathclose%
\pgfusepath{fill}%
\end{pgfscope}%
\begin{pgfscope}%
\pgfpathrectangle{\pgfqpoint{1.250000in}{0.550000in}}{\pgfqpoint{7.750000in}{3.850000in}}%
\pgfusepath{clip}%
\pgfsetbuttcap%
\pgfsetmiterjoin%
\definecolor{currentfill}{rgb}{0.501961,0.501961,0.501961}%
\pgfsetfillcolor{currentfill}%
\pgfsetfillopacity{0.700000}%
\pgfsetlinewidth{0.000000pt}%
\definecolor{currentstroke}{rgb}{0.000000,0.000000,0.000000}%
\pgfsetstrokecolor{currentstroke}%
\pgfsetstrokeopacity{0.700000}%
\pgfsetdash{}{0pt}%
\pgfpathmoveto{\pgfqpoint{7.629439in}{0.550000in}}%
\pgfpathlineto{\pgfqpoint{7.656960in}{0.550000in}}%
\pgfpathlineto{\pgfqpoint{7.656960in}{0.809461in}}%
\pgfpathlineto{\pgfqpoint{7.629439in}{0.809461in}}%
\pgfpathlineto{\pgfqpoint{7.629439in}{0.550000in}}%
\pgfpathclose%
\pgfusepath{fill}%
\end{pgfscope}%
\begin{pgfscope}%
\pgfpathrectangle{\pgfqpoint{1.250000in}{0.550000in}}{\pgfqpoint{7.750000in}{3.850000in}}%
\pgfusepath{clip}%
\pgfsetbuttcap%
\pgfsetmiterjoin%
\definecolor{currentfill}{rgb}{0.501961,0.501961,0.501961}%
\pgfsetfillcolor{currentfill}%
\pgfsetfillopacity{0.700000}%
\pgfsetlinewidth{0.000000pt}%
\definecolor{currentstroke}{rgb}{0.000000,0.000000,0.000000}%
\pgfsetstrokecolor{currentstroke}%
\pgfsetstrokeopacity{0.700000}%
\pgfsetdash{}{0pt}%
\pgfpathmoveto{\pgfqpoint{7.656960in}{0.550000in}}%
\pgfpathlineto{\pgfqpoint{7.684482in}{0.550000in}}%
\pgfpathlineto{\pgfqpoint{7.684482in}{0.805184in}}%
\pgfpathlineto{\pgfqpoint{7.656960in}{0.805184in}}%
\pgfpathlineto{\pgfqpoint{7.656960in}{0.550000in}}%
\pgfpathclose%
\pgfusepath{fill}%
\end{pgfscope}%
\begin{pgfscope}%
\pgfpathrectangle{\pgfqpoint{1.250000in}{0.550000in}}{\pgfqpoint{7.750000in}{3.850000in}}%
\pgfusepath{clip}%
\pgfsetbuttcap%
\pgfsetmiterjoin%
\definecolor{currentfill}{rgb}{0.501961,0.501961,0.501961}%
\pgfsetfillcolor{currentfill}%
\pgfsetfillopacity{0.700000}%
\pgfsetlinewidth{0.000000pt}%
\definecolor{currentstroke}{rgb}{0.000000,0.000000,0.000000}%
\pgfsetstrokecolor{currentstroke}%
\pgfsetstrokeopacity{0.700000}%
\pgfsetdash{}{0pt}%
\pgfpathmoveto{\pgfqpoint{7.684482in}{0.550000in}}%
\pgfpathlineto{\pgfqpoint{7.712003in}{0.550000in}}%
\pgfpathlineto{\pgfqpoint{7.712003in}{0.813738in}}%
\pgfpathlineto{\pgfqpoint{7.684482in}{0.813738in}}%
\pgfpathlineto{\pgfqpoint{7.684482in}{0.550000in}}%
\pgfpathclose%
\pgfusepath{fill}%
\end{pgfscope}%
\begin{pgfscope}%
\pgfpathrectangle{\pgfqpoint{1.250000in}{0.550000in}}{\pgfqpoint{7.750000in}{3.850000in}}%
\pgfusepath{clip}%
\pgfsetbuttcap%
\pgfsetmiterjoin%
\definecolor{currentfill}{rgb}{0.501961,0.501961,0.501961}%
\pgfsetfillcolor{currentfill}%
\pgfsetfillopacity{0.700000}%
\pgfsetlinewidth{0.000000pt}%
\definecolor{currentstroke}{rgb}{0.000000,0.000000,0.000000}%
\pgfsetstrokecolor{currentstroke}%
\pgfsetstrokeopacity{0.700000}%
\pgfsetdash{}{0pt}%
\pgfpathmoveto{\pgfqpoint{7.712003in}{0.550000in}}%
\pgfpathlineto{\pgfqpoint{7.739524in}{0.550000in}}%
\pgfpathlineto{\pgfqpoint{7.739524in}{0.550000in}}%
\pgfpathlineto{\pgfqpoint{7.712003in}{0.550000in}}%
\pgfpathlineto{\pgfqpoint{7.712003in}{0.550000in}}%
\pgfpathclose%
\pgfusepath{fill}%
\end{pgfscope}%
\begin{pgfscope}%
\pgfpathrectangle{\pgfqpoint{1.250000in}{0.550000in}}{\pgfqpoint{7.750000in}{3.850000in}}%
\pgfusepath{clip}%
\pgfsetbuttcap%
\pgfsetmiterjoin%
\definecolor{currentfill}{rgb}{0.501961,0.501961,0.501961}%
\pgfsetfillcolor{currentfill}%
\pgfsetfillopacity{0.700000}%
\pgfsetlinewidth{0.000000pt}%
\definecolor{currentstroke}{rgb}{0.000000,0.000000,0.000000}%
\pgfsetstrokecolor{currentstroke}%
\pgfsetstrokeopacity{0.700000}%
\pgfsetdash{}{0pt}%
\pgfpathmoveto{\pgfqpoint{7.739524in}{0.550000in}}%
\pgfpathlineto{\pgfqpoint{7.767045in}{0.550000in}}%
\pgfpathlineto{\pgfqpoint{7.767045in}{0.782374in}}%
\pgfpathlineto{\pgfqpoint{7.739524in}{0.782374in}}%
\pgfpathlineto{\pgfqpoint{7.739524in}{0.550000in}}%
\pgfpathclose%
\pgfusepath{fill}%
\end{pgfscope}%
\begin{pgfscope}%
\pgfpathrectangle{\pgfqpoint{1.250000in}{0.550000in}}{\pgfqpoint{7.750000in}{3.850000in}}%
\pgfusepath{clip}%
\pgfsetbuttcap%
\pgfsetmiterjoin%
\definecolor{currentfill}{rgb}{0.501961,0.501961,0.501961}%
\pgfsetfillcolor{currentfill}%
\pgfsetfillopacity{0.700000}%
\pgfsetlinewidth{0.000000pt}%
\definecolor{currentstroke}{rgb}{0.000000,0.000000,0.000000}%
\pgfsetstrokecolor{currentstroke}%
\pgfsetstrokeopacity{0.700000}%
\pgfsetdash{}{0pt}%
\pgfpathmoveto{\pgfqpoint{7.767045in}{0.550000in}}%
\pgfpathlineto{\pgfqpoint{7.794567in}{0.550000in}}%
\pgfpathlineto{\pgfqpoint{7.794567in}{0.768118in}}%
\pgfpathlineto{\pgfqpoint{7.767045in}{0.768118in}}%
\pgfpathlineto{\pgfqpoint{7.767045in}{0.550000in}}%
\pgfpathclose%
\pgfusepath{fill}%
\end{pgfscope}%
\begin{pgfscope}%
\pgfpathrectangle{\pgfqpoint{1.250000in}{0.550000in}}{\pgfqpoint{7.750000in}{3.850000in}}%
\pgfusepath{clip}%
\pgfsetbuttcap%
\pgfsetmiterjoin%
\definecolor{currentfill}{rgb}{0.501961,0.501961,0.501961}%
\pgfsetfillcolor{currentfill}%
\pgfsetfillopacity{0.700000}%
\pgfsetlinewidth{0.000000pt}%
\definecolor{currentstroke}{rgb}{0.000000,0.000000,0.000000}%
\pgfsetstrokecolor{currentstroke}%
\pgfsetstrokeopacity{0.700000}%
\pgfsetdash{}{0pt}%
\pgfpathmoveto{\pgfqpoint{7.794567in}{0.550000in}}%
\pgfpathlineto{\pgfqpoint{7.822088in}{0.550000in}}%
\pgfpathlineto{\pgfqpoint{7.822088in}{0.775246in}}%
\pgfpathlineto{\pgfqpoint{7.794567in}{0.775246in}}%
\pgfpathlineto{\pgfqpoint{7.794567in}{0.550000in}}%
\pgfpathclose%
\pgfusepath{fill}%
\end{pgfscope}%
\begin{pgfscope}%
\pgfpathrectangle{\pgfqpoint{1.250000in}{0.550000in}}{\pgfqpoint{7.750000in}{3.850000in}}%
\pgfusepath{clip}%
\pgfsetbuttcap%
\pgfsetmiterjoin%
\definecolor{currentfill}{rgb}{0.501961,0.501961,0.501961}%
\pgfsetfillcolor{currentfill}%
\pgfsetfillopacity{0.700000}%
\pgfsetlinewidth{0.000000pt}%
\definecolor{currentstroke}{rgb}{0.000000,0.000000,0.000000}%
\pgfsetstrokecolor{currentstroke}%
\pgfsetstrokeopacity{0.700000}%
\pgfsetdash{}{0pt}%
\pgfpathmoveto{\pgfqpoint{7.822088in}{0.550000in}}%
\pgfpathlineto{\pgfqpoint{7.849609in}{0.550000in}}%
\pgfpathlineto{\pgfqpoint{7.849609in}{0.773821in}}%
\pgfpathlineto{\pgfqpoint{7.822088in}{0.773821in}}%
\pgfpathlineto{\pgfqpoint{7.822088in}{0.550000in}}%
\pgfpathclose%
\pgfusepath{fill}%
\end{pgfscope}%
\begin{pgfscope}%
\pgfpathrectangle{\pgfqpoint{1.250000in}{0.550000in}}{\pgfqpoint{7.750000in}{3.850000in}}%
\pgfusepath{clip}%
\pgfsetbuttcap%
\pgfsetmiterjoin%
\definecolor{currentfill}{rgb}{0.501961,0.501961,0.501961}%
\pgfsetfillcolor{currentfill}%
\pgfsetfillopacity{0.700000}%
\pgfsetlinewidth{0.000000pt}%
\definecolor{currentstroke}{rgb}{0.000000,0.000000,0.000000}%
\pgfsetstrokecolor{currentstroke}%
\pgfsetstrokeopacity{0.700000}%
\pgfsetdash{}{0pt}%
\pgfpathmoveto{\pgfqpoint{7.849609in}{0.550000in}}%
\pgfpathlineto{\pgfqpoint{7.877131in}{0.550000in}}%
\pgfpathlineto{\pgfqpoint{7.877131in}{0.745308in}}%
\pgfpathlineto{\pgfqpoint{7.849609in}{0.745308in}}%
\pgfpathlineto{\pgfqpoint{7.849609in}{0.550000in}}%
\pgfpathclose%
\pgfusepath{fill}%
\end{pgfscope}%
\begin{pgfscope}%
\pgfpathrectangle{\pgfqpoint{1.250000in}{0.550000in}}{\pgfqpoint{7.750000in}{3.850000in}}%
\pgfusepath{clip}%
\pgfsetbuttcap%
\pgfsetmiterjoin%
\definecolor{currentfill}{rgb}{0.501961,0.501961,0.501961}%
\pgfsetfillcolor{currentfill}%
\pgfsetfillopacity{0.700000}%
\pgfsetlinewidth{0.000000pt}%
\definecolor{currentstroke}{rgb}{0.000000,0.000000,0.000000}%
\pgfsetstrokecolor{currentstroke}%
\pgfsetstrokeopacity{0.700000}%
\pgfsetdash{}{0pt}%
\pgfpathmoveto{\pgfqpoint{7.877131in}{0.550000in}}%
\pgfpathlineto{\pgfqpoint{7.904652in}{0.550000in}}%
\pgfpathlineto{\pgfqpoint{7.904652in}{0.749585in}}%
\pgfpathlineto{\pgfqpoint{7.877131in}{0.749585in}}%
\pgfpathlineto{\pgfqpoint{7.877131in}{0.550000in}}%
\pgfpathclose%
\pgfusepath{fill}%
\end{pgfscope}%
\begin{pgfscope}%
\pgfpathrectangle{\pgfqpoint{1.250000in}{0.550000in}}{\pgfqpoint{7.750000in}{3.850000in}}%
\pgfusepath{clip}%
\pgfsetbuttcap%
\pgfsetmiterjoin%
\definecolor{currentfill}{rgb}{0.501961,0.501961,0.501961}%
\pgfsetfillcolor{currentfill}%
\pgfsetfillopacity{0.700000}%
\pgfsetlinewidth{0.000000pt}%
\definecolor{currentstroke}{rgb}{0.000000,0.000000,0.000000}%
\pgfsetstrokecolor{currentstroke}%
\pgfsetstrokeopacity{0.700000}%
\pgfsetdash{}{0pt}%
\pgfpathmoveto{\pgfqpoint{7.904652in}{0.550000in}}%
\pgfpathlineto{\pgfqpoint{7.932173in}{0.550000in}}%
\pgfpathlineto{\pgfqpoint{7.932173in}{0.709668in}}%
\pgfpathlineto{\pgfqpoint{7.904652in}{0.709668in}}%
\pgfpathlineto{\pgfqpoint{7.904652in}{0.550000in}}%
\pgfpathclose%
\pgfusepath{fill}%
\end{pgfscope}%
\begin{pgfscope}%
\pgfpathrectangle{\pgfqpoint{1.250000in}{0.550000in}}{\pgfqpoint{7.750000in}{3.850000in}}%
\pgfusepath{clip}%
\pgfsetbuttcap%
\pgfsetmiterjoin%
\definecolor{currentfill}{rgb}{0.501961,0.501961,0.501961}%
\pgfsetfillcolor{currentfill}%
\pgfsetfillopacity{0.700000}%
\pgfsetlinewidth{0.000000pt}%
\definecolor{currentstroke}{rgb}{0.000000,0.000000,0.000000}%
\pgfsetstrokecolor{currentstroke}%
\pgfsetstrokeopacity{0.700000}%
\pgfsetdash{}{0pt}%
\pgfpathmoveto{\pgfqpoint{7.932173in}{0.550000in}}%
\pgfpathlineto{\pgfqpoint{7.959695in}{0.550000in}}%
\pgfpathlineto{\pgfqpoint{7.959695in}{0.729627in}}%
\pgfpathlineto{\pgfqpoint{7.932173in}{0.729627in}}%
\pgfpathlineto{\pgfqpoint{7.932173in}{0.550000in}}%
\pgfpathclose%
\pgfusepath{fill}%
\end{pgfscope}%
\begin{pgfscope}%
\pgfpathrectangle{\pgfqpoint{1.250000in}{0.550000in}}{\pgfqpoint{7.750000in}{3.850000in}}%
\pgfusepath{clip}%
\pgfsetbuttcap%
\pgfsetmiterjoin%
\definecolor{currentfill}{rgb}{0.501961,0.501961,0.501961}%
\pgfsetfillcolor{currentfill}%
\pgfsetfillopacity{0.700000}%
\pgfsetlinewidth{0.000000pt}%
\definecolor{currentstroke}{rgb}{0.000000,0.000000,0.000000}%
\pgfsetstrokecolor{currentstroke}%
\pgfsetstrokeopacity{0.700000}%
\pgfsetdash{}{0pt}%
\pgfpathmoveto{\pgfqpoint{7.959695in}{0.550000in}}%
\pgfpathlineto{\pgfqpoint{7.987216in}{0.550000in}}%
\pgfpathlineto{\pgfqpoint{7.987216in}{0.726776in}}%
\pgfpathlineto{\pgfqpoint{7.959695in}{0.726776in}}%
\pgfpathlineto{\pgfqpoint{7.959695in}{0.550000in}}%
\pgfpathclose%
\pgfusepath{fill}%
\end{pgfscope}%
\begin{pgfscope}%
\pgfpathrectangle{\pgfqpoint{1.250000in}{0.550000in}}{\pgfqpoint{7.750000in}{3.850000in}}%
\pgfusepath{clip}%
\pgfsetbuttcap%
\pgfsetmiterjoin%
\definecolor{currentfill}{rgb}{0.501961,0.501961,0.501961}%
\pgfsetfillcolor{currentfill}%
\pgfsetfillopacity{0.700000}%
\pgfsetlinewidth{0.000000pt}%
\definecolor{currentstroke}{rgb}{0.000000,0.000000,0.000000}%
\pgfsetstrokecolor{currentstroke}%
\pgfsetstrokeopacity{0.700000}%
\pgfsetdash{}{0pt}%
\pgfpathmoveto{\pgfqpoint{7.987216in}{0.550000in}}%
\pgfpathlineto{\pgfqpoint{8.014737in}{0.550000in}}%
\pgfpathlineto{\pgfqpoint{8.014737in}{0.709668in}}%
\pgfpathlineto{\pgfqpoint{7.987216in}{0.709668in}}%
\pgfpathlineto{\pgfqpoint{7.987216in}{0.550000in}}%
\pgfpathclose%
\pgfusepath{fill}%
\end{pgfscope}%
\begin{pgfscope}%
\pgfpathrectangle{\pgfqpoint{1.250000in}{0.550000in}}{\pgfqpoint{7.750000in}{3.850000in}}%
\pgfusepath{clip}%
\pgfsetbuttcap%
\pgfsetmiterjoin%
\definecolor{currentfill}{rgb}{0.501961,0.501961,0.501961}%
\pgfsetfillcolor{currentfill}%
\pgfsetfillopacity{0.700000}%
\pgfsetlinewidth{0.000000pt}%
\definecolor{currentstroke}{rgb}{0.000000,0.000000,0.000000}%
\pgfsetstrokecolor{currentstroke}%
\pgfsetstrokeopacity{0.700000}%
\pgfsetdash{}{0pt}%
\pgfpathmoveto{\pgfqpoint{8.014737in}{0.550000in}}%
\pgfpathlineto{\pgfqpoint{8.042259in}{0.550000in}}%
\pgfpathlineto{\pgfqpoint{8.042259in}{0.693987in}}%
\pgfpathlineto{\pgfqpoint{8.014737in}{0.693987in}}%
\pgfpathlineto{\pgfqpoint{8.014737in}{0.550000in}}%
\pgfpathclose%
\pgfusepath{fill}%
\end{pgfscope}%
\begin{pgfscope}%
\pgfpathrectangle{\pgfqpoint{1.250000in}{0.550000in}}{\pgfqpoint{7.750000in}{3.850000in}}%
\pgfusepath{clip}%
\pgfsetbuttcap%
\pgfsetmiterjoin%
\definecolor{currentfill}{rgb}{0.501961,0.501961,0.501961}%
\pgfsetfillcolor{currentfill}%
\pgfsetfillopacity{0.700000}%
\pgfsetlinewidth{0.000000pt}%
\definecolor{currentstroke}{rgb}{0.000000,0.000000,0.000000}%
\pgfsetstrokecolor{currentstroke}%
\pgfsetstrokeopacity{0.700000}%
\pgfsetdash{}{0pt}%
\pgfpathmoveto{\pgfqpoint{8.042259in}{0.550000in}}%
\pgfpathlineto{\pgfqpoint{8.069780in}{0.550000in}}%
\pgfpathlineto{\pgfqpoint{8.069780in}{0.856506in}}%
\pgfpathlineto{\pgfqpoint{8.042259in}{0.856506in}}%
\pgfpathlineto{\pgfqpoint{8.042259in}{0.550000in}}%
\pgfpathclose%
\pgfusepath{fill}%
\end{pgfscope}%
\begin{pgfscope}%
\pgfpathrectangle{\pgfqpoint{1.250000in}{0.550000in}}{\pgfqpoint{7.750000in}{3.850000in}}%
\pgfusepath{clip}%
\pgfsetbuttcap%
\pgfsetmiterjoin%
\definecolor{currentfill}{rgb}{0.501961,0.501961,0.501961}%
\pgfsetfillcolor{currentfill}%
\pgfsetfillopacity{0.700000}%
\pgfsetlinewidth{0.000000pt}%
\definecolor{currentstroke}{rgb}{0.000000,0.000000,0.000000}%
\pgfsetstrokecolor{currentstroke}%
\pgfsetstrokeopacity{0.700000}%
\pgfsetdash{}{0pt}%
\pgfpathmoveto{\pgfqpoint{8.069780in}{0.550000in}}%
\pgfpathlineto{\pgfqpoint{8.097301in}{0.550000in}}%
\pgfpathlineto{\pgfqpoint{8.097301in}{0.684007in}}%
\pgfpathlineto{\pgfqpoint{8.069780in}{0.684007in}}%
\pgfpathlineto{\pgfqpoint{8.069780in}{0.550000in}}%
\pgfpathclose%
\pgfusepath{fill}%
\end{pgfscope}%
\begin{pgfscope}%
\pgfpathrectangle{\pgfqpoint{1.250000in}{0.550000in}}{\pgfqpoint{7.750000in}{3.850000in}}%
\pgfusepath{clip}%
\pgfsetbuttcap%
\pgfsetmiterjoin%
\definecolor{currentfill}{rgb}{0.501961,0.501961,0.501961}%
\pgfsetfillcolor{currentfill}%
\pgfsetfillopacity{0.700000}%
\pgfsetlinewidth{0.000000pt}%
\definecolor{currentstroke}{rgb}{0.000000,0.000000,0.000000}%
\pgfsetstrokecolor{currentstroke}%
\pgfsetstrokeopacity{0.700000}%
\pgfsetdash{}{0pt}%
\pgfpathmoveto{\pgfqpoint{8.097301in}{0.550000in}}%
\pgfpathlineto{\pgfqpoint{8.124822in}{0.550000in}}%
\pgfpathlineto{\pgfqpoint{8.124822in}{0.800907in}}%
\pgfpathlineto{\pgfqpoint{8.097301in}{0.800907in}}%
\pgfpathlineto{\pgfqpoint{8.097301in}{0.550000in}}%
\pgfpathclose%
\pgfusepath{fill}%
\end{pgfscope}%
\begin{pgfscope}%
\pgfpathrectangle{\pgfqpoint{1.250000in}{0.550000in}}{\pgfqpoint{7.750000in}{3.850000in}}%
\pgfusepath{clip}%
\pgfsetbuttcap%
\pgfsetmiterjoin%
\definecolor{currentfill}{rgb}{0.501961,0.501961,0.501961}%
\pgfsetfillcolor{currentfill}%
\pgfsetfillopacity{0.700000}%
\pgfsetlinewidth{0.000000pt}%
\definecolor{currentstroke}{rgb}{0.000000,0.000000,0.000000}%
\pgfsetstrokecolor{currentstroke}%
\pgfsetstrokeopacity{0.700000}%
\pgfsetdash{}{0pt}%
\pgfpathmoveto{\pgfqpoint{8.124822in}{0.550000in}}%
\pgfpathlineto{\pgfqpoint{8.152344in}{0.550000in}}%
\pgfpathlineto{\pgfqpoint{8.152344in}{0.691135in}}%
\pgfpathlineto{\pgfqpoint{8.124822in}{0.691135in}}%
\pgfpathlineto{\pgfqpoint{8.124822in}{0.550000in}}%
\pgfpathclose%
\pgfusepath{fill}%
\end{pgfscope}%
\begin{pgfscope}%
\pgfpathrectangle{\pgfqpoint{1.250000in}{0.550000in}}{\pgfqpoint{7.750000in}{3.850000in}}%
\pgfusepath{clip}%
\pgfsetbuttcap%
\pgfsetmiterjoin%
\definecolor{currentfill}{rgb}{0.501961,0.501961,0.501961}%
\pgfsetfillcolor{currentfill}%
\pgfsetfillopacity{0.700000}%
\pgfsetlinewidth{0.000000pt}%
\definecolor{currentstroke}{rgb}{0.000000,0.000000,0.000000}%
\pgfsetstrokecolor{currentstroke}%
\pgfsetstrokeopacity{0.700000}%
\pgfsetdash{}{0pt}%
\pgfpathmoveto{\pgfqpoint{8.152344in}{0.550000in}}%
\pgfpathlineto{\pgfqpoint{8.179865in}{0.550000in}}%
\pgfpathlineto{\pgfqpoint{8.179865in}{0.776672in}}%
\pgfpathlineto{\pgfqpoint{8.152344in}{0.776672in}}%
\pgfpathlineto{\pgfqpoint{8.152344in}{0.550000in}}%
\pgfpathclose%
\pgfusepath{fill}%
\end{pgfscope}%
\begin{pgfscope}%
\pgfpathrectangle{\pgfqpoint{1.250000in}{0.550000in}}{\pgfqpoint{7.750000in}{3.850000in}}%
\pgfusepath{clip}%
\pgfsetbuttcap%
\pgfsetmiterjoin%
\definecolor{currentfill}{rgb}{0.501961,0.501961,0.501961}%
\pgfsetfillcolor{currentfill}%
\pgfsetfillopacity{0.700000}%
\pgfsetlinewidth{0.000000pt}%
\definecolor{currentstroke}{rgb}{0.000000,0.000000,0.000000}%
\pgfsetstrokecolor{currentstroke}%
\pgfsetstrokeopacity{0.700000}%
\pgfsetdash{}{0pt}%
\pgfpathmoveto{\pgfqpoint{8.179865in}{0.550000in}}%
\pgfpathlineto{\pgfqpoint{8.207386in}{0.550000in}}%
\pgfpathlineto{\pgfqpoint{8.207386in}{0.782374in}}%
\pgfpathlineto{\pgfqpoint{8.179865in}{0.782374in}}%
\pgfpathlineto{\pgfqpoint{8.179865in}{0.550000in}}%
\pgfpathclose%
\pgfusepath{fill}%
\end{pgfscope}%
\begin{pgfscope}%
\pgfpathrectangle{\pgfqpoint{1.250000in}{0.550000in}}{\pgfqpoint{7.750000in}{3.850000in}}%
\pgfusepath{clip}%
\pgfsetbuttcap%
\pgfsetmiterjoin%
\definecolor{currentfill}{rgb}{0.501961,0.501961,0.501961}%
\pgfsetfillcolor{currentfill}%
\pgfsetfillopacity{0.700000}%
\pgfsetlinewidth{0.000000pt}%
\definecolor{currentstroke}{rgb}{0.000000,0.000000,0.000000}%
\pgfsetstrokecolor{currentstroke}%
\pgfsetstrokeopacity{0.700000}%
\pgfsetdash{}{0pt}%
\pgfpathmoveto{\pgfqpoint{8.207386in}{0.550000in}}%
\pgfpathlineto{\pgfqpoint{8.234908in}{0.550000in}}%
\pgfpathlineto{\pgfqpoint{8.234908in}{0.718222in}}%
\pgfpathlineto{\pgfqpoint{8.207386in}{0.718222in}}%
\pgfpathlineto{\pgfqpoint{8.207386in}{0.550000in}}%
\pgfpathclose%
\pgfusepath{fill}%
\end{pgfscope}%
\begin{pgfscope}%
\pgfpathrectangle{\pgfqpoint{1.250000in}{0.550000in}}{\pgfqpoint{7.750000in}{3.850000in}}%
\pgfusepath{clip}%
\pgfsetbuttcap%
\pgfsetmiterjoin%
\definecolor{currentfill}{rgb}{0.501961,0.501961,0.501961}%
\pgfsetfillcolor{currentfill}%
\pgfsetfillopacity{0.700000}%
\pgfsetlinewidth{0.000000pt}%
\definecolor{currentstroke}{rgb}{0.000000,0.000000,0.000000}%
\pgfsetstrokecolor{currentstroke}%
\pgfsetstrokeopacity{0.700000}%
\pgfsetdash{}{0pt}%
\pgfpathmoveto{\pgfqpoint{8.234908in}{0.550000in}}%
\pgfpathlineto{\pgfqpoint{8.262429in}{0.550000in}}%
\pgfpathlineto{\pgfqpoint{8.262429in}{0.679730in}}%
\pgfpathlineto{\pgfqpoint{8.234908in}{0.679730in}}%
\pgfpathlineto{\pgfqpoint{8.234908in}{0.550000in}}%
\pgfpathclose%
\pgfusepath{fill}%
\end{pgfscope}%
\begin{pgfscope}%
\pgfpathrectangle{\pgfqpoint{1.250000in}{0.550000in}}{\pgfqpoint{7.750000in}{3.850000in}}%
\pgfusepath{clip}%
\pgfsetbuttcap%
\pgfsetmiterjoin%
\definecolor{currentfill}{rgb}{0.501961,0.501961,0.501961}%
\pgfsetfillcolor{currentfill}%
\pgfsetfillopacity{0.700000}%
\pgfsetlinewidth{0.000000pt}%
\definecolor{currentstroke}{rgb}{0.000000,0.000000,0.000000}%
\pgfsetstrokecolor{currentstroke}%
\pgfsetstrokeopacity{0.700000}%
\pgfsetdash{}{0pt}%
\pgfpathmoveto{\pgfqpoint{8.262429in}{0.550000in}}%
\pgfpathlineto{\pgfqpoint{8.289950in}{0.550000in}}%
\pgfpathlineto{\pgfqpoint{8.289950in}{0.749585in}}%
\pgfpathlineto{\pgfqpoint{8.262429in}{0.749585in}}%
\pgfpathlineto{\pgfqpoint{8.262429in}{0.550000in}}%
\pgfpathclose%
\pgfusepath{fill}%
\end{pgfscope}%
\begin{pgfscope}%
\pgfpathrectangle{\pgfqpoint{1.250000in}{0.550000in}}{\pgfqpoint{7.750000in}{3.850000in}}%
\pgfusepath{clip}%
\pgfsetbuttcap%
\pgfsetmiterjoin%
\definecolor{currentfill}{rgb}{0.501961,0.501961,0.501961}%
\pgfsetfillcolor{currentfill}%
\pgfsetfillopacity{0.700000}%
\pgfsetlinewidth{0.000000pt}%
\definecolor{currentstroke}{rgb}{0.000000,0.000000,0.000000}%
\pgfsetstrokecolor{currentstroke}%
\pgfsetstrokeopacity{0.700000}%
\pgfsetdash{}{0pt}%
\pgfpathmoveto{\pgfqpoint{8.289950in}{0.550000in}}%
\pgfpathlineto{\pgfqpoint{8.317472in}{0.550000in}}%
\pgfpathlineto{\pgfqpoint{8.317472in}{0.792354in}}%
\pgfpathlineto{\pgfqpoint{8.289950in}{0.792354in}}%
\pgfpathlineto{\pgfqpoint{8.289950in}{0.550000in}}%
\pgfpathclose%
\pgfusepath{fill}%
\end{pgfscope}%
\begin{pgfscope}%
\pgfpathrectangle{\pgfqpoint{1.250000in}{0.550000in}}{\pgfqpoint{7.750000in}{3.850000in}}%
\pgfusepath{clip}%
\pgfsetbuttcap%
\pgfsetmiterjoin%
\definecolor{currentfill}{rgb}{0.501961,0.501961,0.501961}%
\pgfsetfillcolor{currentfill}%
\pgfsetfillopacity{0.700000}%
\pgfsetlinewidth{0.000000pt}%
\definecolor{currentstroke}{rgb}{0.000000,0.000000,0.000000}%
\pgfsetstrokecolor{currentstroke}%
\pgfsetstrokeopacity{0.700000}%
\pgfsetdash{}{0pt}%
\pgfpathmoveto{\pgfqpoint{8.317472in}{0.550000in}}%
\pgfpathlineto{\pgfqpoint{8.344993in}{0.550000in}}%
\pgfpathlineto{\pgfqpoint{8.344993in}{0.709668in}}%
\pgfpathlineto{\pgfqpoint{8.317472in}{0.709668in}}%
\pgfpathlineto{\pgfqpoint{8.317472in}{0.550000in}}%
\pgfpathclose%
\pgfusepath{fill}%
\end{pgfscope}%
\begin{pgfscope}%
\pgfpathrectangle{\pgfqpoint{1.250000in}{0.550000in}}{\pgfqpoint{7.750000in}{3.850000in}}%
\pgfusepath{clip}%
\pgfsetbuttcap%
\pgfsetmiterjoin%
\definecolor{currentfill}{rgb}{0.501961,0.501961,0.501961}%
\pgfsetfillcolor{currentfill}%
\pgfsetfillopacity{0.700000}%
\pgfsetlinewidth{0.000000pt}%
\definecolor{currentstroke}{rgb}{0.000000,0.000000,0.000000}%
\pgfsetstrokecolor{currentstroke}%
\pgfsetstrokeopacity{0.700000}%
\pgfsetdash{}{0pt}%
\pgfpathmoveto{\pgfqpoint{8.344993in}{0.550000in}}%
\pgfpathlineto{\pgfqpoint{8.372514in}{0.550000in}}%
\pgfpathlineto{\pgfqpoint{8.372514in}{0.785226in}}%
\pgfpathlineto{\pgfqpoint{8.344993in}{0.785226in}}%
\pgfpathlineto{\pgfqpoint{8.344993in}{0.550000in}}%
\pgfpathclose%
\pgfusepath{fill}%
\end{pgfscope}%
\begin{pgfscope}%
\pgfpathrectangle{\pgfqpoint{1.250000in}{0.550000in}}{\pgfqpoint{7.750000in}{3.850000in}}%
\pgfusepath{clip}%
\pgfsetbuttcap%
\pgfsetmiterjoin%
\definecolor{currentfill}{rgb}{0.501961,0.501961,0.501961}%
\pgfsetfillcolor{currentfill}%
\pgfsetfillopacity{0.700000}%
\pgfsetlinewidth{0.000000pt}%
\definecolor{currentstroke}{rgb}{0.000000,0.000000,0.000000}%
\pgfsetstrokecolor{currentstroke}%
\pgfsetstrokeopacity{0.700000}%
\pgfsetdash{}{0pt}%
\pgfpathmoveto{\pgfqpoint{8.372514in}{0.550000in}}%
\pgfpathlineto{\pgfqpoint{8.400036in}{0.550000in}}%
\pgfpathlineto{\pgfqpoint{8.400036in}{0.705391in}}%
\pgfpathlineto{\pgfqpoint{8.372514in}{0.705391in}}%
\pgfpathlineto{\pgfqpoint{8.372514in}{0.550000in}}%
\pgfpathclose%
\pgfusepath{fill}%
\end{pgfscope}%
\begin{pgfscope}%
\pgfpathrectangle{\pgfqpoint{1.250000in}{0.550000in}}{\pgfqpoint{7.750000in}{3.850000in}}%
\pgfusepath{clip}%
\pgfsetbuttcap%
\pgfsetmiterjoin%
\definecolor{currentfill}{rgb}{0.501961,0.501961,0.501961}%
\pgfsetfillcolor{currentfill}%
\pgfsetfillopacity{0.700000}%
\pgfsetlinewidth{0.000000pt}%
\definecolor{currentstroke}{rgb}{0.000000,0.000000,0.000000}%
\pgfsetstrokecolor{currentstroke}%
\pgfsetstrokeopacity{0.700000}%
\pgfsetdash{}{0pt}%
\pgfpathmoveto{\pgfqpoint{8.400036in}{0.550000in}}%
\pgfpathlineto{\pgfqpoint{8.427557in}{0.550000in}}%
\pgfpathlineto{\pgfqpoint{8.427557in}{0.772395in}}%
\pgfpathlineto{\pgfqpoint{8.400036in}{0.772395in}}%
\pgfpathlineto{\pgfqpoint{8.400036in}{0.550000in}}%
\pgfpathclose%
\pgfusepath{fill}%
\end{pgfscope}%
\begin{pgfscope}%
\pgfpathrectangle{\pgfqpoint{1.250000in}{0.550000in}}{\pgfqpoint{7.750000in}{3.850000in}}%
\pgfusepath{clip}%
\pgfsetbuttcap%
\pgfsetmiterjoin%
\definecolor{currentfill}{rgb}{0.501961,0.501961,0.501961}%
\pgfsetfillcolor{currentfill}%
\pgfsetfillopacity{0.700000}%
\pgfsetlinewidth{0.000000pt}%
\definecolor{currentstroke}{rgb}{0.000000,0.000000,0.000000}%
\pgfsetstrokecolor{currentstroke}%
\pgfsetstrokeopacity{0.700000}%
\pgfsetdash{}{0pt}%
\pgfpathmoveto{\pgfqpoint{8.427557in}{0.550000in}}%
\pgfpathlineto{\pgfqpoint{8.455078in}{0.550000in}}%
\pgfpathlineto{\pgfqpoint{8.455078in}{0.742457in}}%
\pgfpathlineto{\pgfqpoint{8.427557in}{0.742457in}}%
\pgfpathlineto{\pgfqpoint{8.427557in}{0.550000in}}%
\pgfpathclose%
\pgfusepath{fill}%
\end{pgfscope}%
\begin{pgfscope}%
\pgfpathrectangle{\pgfqpoint{1.250000in}{0.550000in}}{\pgfqpoint{7.750000in}{3.850000in}}%
\pgfusepath{clip}%
\pgfsetbuttcap%
\pgfsetmiterjoin%
\definecolor{currentfill}{rgb}{0.501961,0.501961,0.501961}%
\pgfsetfillcolor{currentfill}%
\pgfsetfillopacity{0.700000}%
\pgfsetlinewidth{0.000000pt}%
\definecolor{currentstroke}{rgb}{0.000000,0.000000,0.000000}%
\pgfsetstrokecolor{currentstroke}%
\pgfsetstrokeopacity{0.700000}%
\pgfsetdash{}{0pt}%
\pgfpathmoveto{\pgfqpoint{8.455078in}{0.550000in}}%
\pgfpathlineto{\pgfqpoint{8.482599in}{0.550000in}}%
\pgfpathlineto{\pgfqpoint{8.482599in}{0.738180in}}%
\pgfpathlineto{\pgfqpoint{8.455078in}{0.738180in}}%
\pgfpathlineto{\pgfqpoint{8.455078in}{0.550000in}}%
\pgfpathclose%
\pgfusepath{fill}%
\end{pgfscope}%
\begin{pgfscope}%
\pgfpathrectangle{\pgfqpoint{1.250000in}{0.550000in}}{\pgfqpoint{7.750000in}{3.850000in}}%
\pgfusepath{clip}%
\pgfsetbuttcap%
\pgfsetmiterjoin%
\definecolor{currentfill}{rgb}{0.501961,0.501961,0.501961}%
\pgfsetfillcolor{currentfill}%
\pgfsetfillopacity{0.700000}%
\pgfsetlinewidth{0.000000pt}%
\definecolor{currentstroke}{rgb}{0.000000,0.000000,0.000000}%
\pgfsetstrokecolor{currentstroke}%
\pgfsetstrokeopacity{0.700000}%
\pgfsetdash{}{0pt}%
\pgfpathmoveto{\pgfqpoint{8.482599in}{0.550000in}}%
\pgfpathlineto{\pgfqpoint{8.510121in}{0.550000in}}%
\pgfpathlineto{\pgfqpoint{8.510121in}{0.760990in}}%
\pgfpathlineto{\pgfqpoint{8.482599in}{0.760990in}}%
\pgfpathlineto{\pgfqpoint{8.482599in}{0.550000in}}%
\pgfpathclose%
\pgfusepath{fill}%
\end{pgfscope}%
\begin{pgfscope}%
\pgfpathrectangle{\pgfqpoint{1.250000in}{0.550000in}}{\pgfqpoint{7.750000in}{3.850000in}}%
\pgfusepath{clip}%
\pgfsetbuttcap%
\pgfsetmiterjoin%
\definecolor{currentfill}{rgb}{0.501961,0.501961,0.501961}%
\pgfsetfillcolor{currentfill}%
\pgfsetfillopacity{0.700000}%
\pgfsetlinewidth{0.000000pt}%
\definecolor{currentstroke}{rgb}{0.000000,0.000000,0.000000}%
\pgfsetstrokecolor{currentstroke}%
\pgfsetstrokeopacity{0.700000}%
\pgfsetdash{}{0pt}%
\pgfpathmoveto{\pgfqpoint{8.510121in}{0.550000in}}%
\pgfpathlineto{\pgfqpoint{8.537642in}{0.550000in}}%
\pgfpathlineto{\pgfqpoint{8.537642in}{0.733904in}}%
\pgfpathlineto{\pgfqpoint{8.510121in}{0.733904in}}%
\pgfpathlineto{\pgfqpoint{8.510121in}{0.550000in}}%
\pgfpathclose%
\pgfusepath{fill}%
\end{pgfscope}%
\begin{pgfscope}%
\pgfpathrectangle{\pgfqpoint{1.250000in}{0.550000in}}{\pgfqpoint{7.750000in}{3.850000in}}%
\pgfusepath{clip}%
\pgfsetbuttcap%
\pgfsetmiterjoin%
\definecolor{currentfill}{rgb}{0.501961,0.501961,0.501961}%
\pgfsetfillcolor{currentfill}%
\pgfsetfillopacity{0.700000}%
\pgfsetlinewidth{0.000000pt}%
\definecolor{currentstroke}{rgb}{0.000000,0.000000,0.000000}%
\pgfsetstrokecolor{currentstroke}%
\pgfsetstrokeopacity{0.700000}%
\pgfsetdash{}{0pt}%
\pgfpathmoveto{\pgfqpoint{8.537642in}{0.550000in}}%
\pgfpathlineto{\pgfqpoint{8.565163in}{0.550000in}}%
\pgfpathlineto{\pgfqpoint{8.565163in}{0.741032in}}%
\pgfpathlineto{\pgfqpoint{8.537642in}{0.741032in}}%
\pgfpathlineto{\pgfqpoint{8.537642in}{0.550000in}}%
\pgfpathclose%
\pgfusepath{fill}%
\end{pgfscope}%
\begin{pgfscope}%
\pgfpathrectangle{\pgfqpoint{1.250000in}{0.550000in}}{\pgfqpoint{7.750000in}{3.850000in}}%
\pgfusepath{clip}%
\pgfsetbuttcap%
\pgfsetmiterjoin%
\definecolor{currentfill}{rgb}{0.501961,0.501961,0.501961}%
\pgfsetfillcolor{currentfill}%
\pgfsetfillopacity{0.700000}%
\pgfsetlinewidth{0.000000pt}%
\definecolor{currentstroke}{rgb}{0.000000,0.000000,0.000000}%
\pgfsetstrokecolor{currentstroke}%
\pgfsetstrokeopacity{0.700000}%
\pgfsetdash{}{0pt}%
\pgfpathmoveto{\pgfqpoint{8.565163in}{0.550000in}}%
\pgfpathlineto{\pgfqpoint{8.592685in}{0.550000in}}%
\pgfpathlineto{\pgfqpoint{8.592685in}{0.728201in}}%
\pgfpathlineto{\pgfqpoint{8.565163in}{0.728201in}}%
\pgfpathlineto{\pgfqpoint{8.565163in}{0.550000in}}%
\pgfpathclose%
\pgfusepath{fill}%
\end{pgfscope}%
\begin{pgfscope}%
\pgfpathrectangle{\pgfqpoint{1.250000in}{0.550000in}}{\pgfqpoint{7.750000in}{3.850000in}}%
\pgfusepath{clip}%
\pgfsetbuttcap%
\pgfsetmiterjoin%
\definecolor{currentfill}{rgb}{0.501961,0.501961,0.501961}%
\pgfsetfillcolor{currentfill}%
\pgfsetfillopacity{0.700000}%
\pgfsetlinewidth{0.000000pt}%
\definecolor{currentstroke}{rgb}{0.000000,0.000000,0.000000}%
\pgfsetstrokecolor{currentstroke}%
\pgfsetstrokeopacity{0.700000}%
\pgfsetdash{}{0pt}%
\pgfpathmoveto{\pgfqpoint{8.592685in}{0.550000in}}%
\pgfpathlineto{\pgfqpoint{8.620206in}{0.550000in}}%
\pgfpathlineto{\pgfqpoint{8.620206in}{0.735329in}}%
\pgfpathlineto{\pgfqpoint{8.592685in}{0.735329in}}%
\pgfpathlineto{\pgfqpoint{8.592685in}{0.550000in}}%
\pgfpathclose%
\pgfusepath{fill}%
\end{pgfscope}%
\begin{pgfscope}%
\pgfpathrectangle{\pgfqpoint{1.250000in}{0.550000in}}{\pgfqpoint{7.750000in}{3.850000in}}%
\pgfusepath{clip}%
\pgfsetbuttcap%
\pgfsetmiterjoin%
\definecolor{currentfill}{rgb}{0.501961,0.501961,0.501961}%
\pgfsetfillcolor{currentfill}%
\pgfsetfillopacity{0.700000}%
\pgfsetlinewidth{0.000000pt}%
\definecolor{currentstroke}{rgb}{0.000000,0.000000,0.000000}%
\pgfsetstrokecolor{currentstroke}%
\pgfsetstrokeopacity{0.700000}%
\pgfsetdash{}{0pt}%
\pgfpathmoveto{\pgfqpoint{8.620206in}{0.550000in}}%
\pgfpathlineto{\pgfqpoint{8.647727in}{0.550000in}}%
\pgfpathlineto{\pgfqpoint{8.647727in}{0.666900in}}%
\pgfpathlineto{\pgfqpoint{8.620206in}{0.666900in}}%
\pgfpathlineto{\pgfqpoint{8.620206in}{0.550000in}}%
\pgfpathclose%
\pgfusepath{fill}%
\end{pgfscope}%
\begin{pgfscope}%
\pgfsetbuttcap%
\pgfsetroundjoin%
\definecolor{currentfill}{rgb}{0.000000,0.000000,0.000000}%
\pgfsetfillcolor{currentfill}%
\pgfsetlinewidth{0.803000pt}%
\definecolor{currentstroke}{rgb}{0.000000,0.000000,0.000000}%
\pgfsetstrokecolor{currentstroke}%
\pgfsetdash{}{0pt}%
\pgfsys@defobject{currentmarker}{\pgfqpoint{0.000000in}{-0.048611in}}{\pgfqpoint{0.000000in}{0.000000in}}{%
\pgfpathmoveto{\pgfqpoint{0.000000in}{0.000000in}}%
\pgfpathlineto{\pgfqpoint{0.000000in}{-0.048611in}}%
\pgfusepath{stroke,fill}%
}%
\begin{pgfscope}%
\pgfsys@transformshift{1.602273in}{0.550000in}%
\pgfsys@useobject{currentmarker}{}%
\end{pgfscope}%
\end{pgfscope}%
\begin{pgfscope}%
\definecolor{textcolor}{rgb}{0.000000,0.000000,0.000000}%
\pgfsetstrokecolor{textcolor}%
\pgfsetfillcolor{textcolor}%
\pgftext[x=1.602273in,y=0.452778in,,top]{\color{textcolor}{\sffamily\fontsize{10.000000}{12.000000}\selectfont\catcode`\^=\active\def^{\ifmmode\sp\else\^{}\fi}\catcode`\%=\active\def%{\%}0}}%
\end{pgfscope}%
\begin{pgfscope}%
\pgfsetbuttcap%
\pgfsetroundjoin%
\definecolor{currentfill}{rgb}{0.000000,0.000000,0.000000}%
\pgfsetfillcolor{currentfill}%
\pgfsetlinewidth{0.803000pt}%
\definecolor{currentstroke}{rgb}{0.000000,0.000000,0.000000}%
\pgfsetstrokecolor{currentstroke}%
\pgfsetdash{}{0pt}%
\pgfsys@defobject{currentmarker}{\pgfqpoint{0.000000in}{-0.048611in}}{\pgfqpoint{0.000000in}{0.000000in}}{%
\pgfpathmoveto{\pgfqpoint{0.000000in}{0.000000in}}%
\pgfpathlineto{\pgfqpoint{0.000000in}{-0.048611in}}%
\pgfusepath{stroke,fill}%
}%
\begin{pgfscope}%
\pgfsys@transformshift{2.978338in}{0.550000in}%
\pgfsys@useobject{currentmarker}{}%
\end{pgfscope}%
\end{pgfscope}%
\begin{pgfscope}%
\definecolor{textcolor}{rgb}{0.000000,0.000000,0.000000}%
\pgfsetstrokecolor{textcolor}%
\pgfsetfillcolor{textcolor}%
\pgftext[x=2.978338in,y=0.452778in,,top]{\color{textcolor}{\sffamily\fontsize{10.000000}{12.000000}\selectfont\catcode`\^=\active\def^{\ifmmode\sp\else\^{}\fi}\catcode`\%=\active\def%{\%}50}}%
\end{pgfscope}%
\begin{pgfscope}%
\pgfsetbuttcap%
\pgfsetroundjoin%
\definecolor{currentfill}{rgb}{0.000000,0.000000,0.000000}%
\pgfsetfillcolor{currentfill}%
\pgfsetlinewidth{0.803000pt}%
\definecolor{currentstroke}{rgb}{0.000000,0.000000,0.000000}%
\pgfsetstrokecolor{currentstroke}%
\pgfsetdash{}{0pt}%
\pgfsys@defobject{currentmarker}{\pgfqpoint{0.000000in}{-0.048611in}}{\pgfqpoint{0.000000in}{0.000000in}}{%
\pgfpathmoveto{\pgfqpoint{0.000000in}{0.000000in}}%
\pgfpathlineto{\pgfqpoint{0.000000in}{-0.048611in}}%
\pgfusepath{stroke,fill}%
}%
\begin{pgfscope}%
\pgfsys@transformshift{4.354403in}{0.550000in}%
\pgfsys@useobject{currentmarker}{}%
\end{pgfscope}%
\end{pgfscope}%
\begin{pgfscope}%
\definecolor{textcolor}{rgb}{0.000000,0.000000,0.000000}%
\pgfsetstrokecolor{textcolor}%
\pgfsetfillcolor{textcolor}%
\pgftext[x=4.354403in,y=0.452778in,,top]{\color{textcolor}{\sffamily\fontsize{10.000000}{12.000000}\selectfont\catcode`\^=\active\def^{\ifmmode\sp\else\^{}\fi}\catcode`\%=\active\def%{\%}100}}%
\end{pgfscope}%
\begin{pgfscope}%
\pgfsetbuttcap%
\pgfsetroundjoin%
\definecolor{currentfill}{rgb}{0.000000,0.000000,0.000000}%
\pgfsetfillcolor{currentfill}%
\pgfsetlinewidth{0.803000pt}%
\definecolor{currentstroke}{rgb}{0.000000,0.000000,0.000000}%
\pgfsetstrokecolor{currentstroke}%
\pgfsetdash{}{0pt}%
\pgfsys@defobject{currentmarker}{\pgfqpoint{0.000000in}{-0.048611in}}{\pgfqpoint{0.000000in}{0.000000in}}{%
\pgfpathmoveto{\pgfqpoint{0.000000in}{0.000000in}}%
\pgfpathlineto{\pgfqpoint{0.000000in}{-0.048611in}}%
\pgfusepath{stroke,fill}%
}%
\begin{pgfscope}%
\pgfsys@transformshift{5.730469in}{0.550000in}%
\pgfsys@useobject{currentmarker}{}%
\end{pgfscope}%
\end{pgfscope}%
\begin{pgfscope}%
\definecolor{textcolor}{rgb}{0.000000,0.000000,0.000000}%
\pgfsetstrokecolor{textcolor}%
\pgfsetfillcolor{textcolor}%
\pgftext[x=5.730469in,y=0.452778in,,top]{\color{textcolor}{\sffamily\fontsize{10.000000}{12.000000}\selectfont\catcode`\^=\active\def^{\ifmmode\sp\else\^{}\fi}\catcode`\%=\active\def%{\%}150}}%
\end{pgfscope}%
\begin{pgfscope}%
\pgfsetbuttcap%
\pgfsetroundjoin%
\definecolor{currentfill}{rgb}{0.000000,0.000000,0.000000}%
\pgfsetfillcolor{currentfill}%
\pgfsetlinewidth{0.803000pt}%
\definecolor{currentstroke}{rgb}{0.000000,0.000000,0.000000}%
\pgfsetstrokecolor{currentstroke}%
\pgfsetdash{}{0pt}%
\pgfsys@defobject{currentmarker}{\pgfqpoint{0.000000in}{-0.048611in}}{\pgfqpoint{0.000000in}{0.000000in}}{%
\pgfpathmoveto{\pgfqpoint{0.000000in}{0.000000in}}%
\pgfpathlineto{\pgfqpoint{0.000000in}{-0.048611in}}%
\pgfusepath{stroke,fill}%
}%
\begin{pgfscope}%
\pgfsys@transformshift{7.106534in}{0.550000in}%
\pgfsys@useobject{currentmarker}{}%
\end{pgfscope}%
\end{pgfscope}%
\begin{pgfscope}%
\definecolor{textcolor}{rgb}{0.000000,0.000000,0.000000}%
\pgfsetstrokecolor{textcolor}%
\pgfsetfillcolor{textcolor}%
\pgftext[x=7.106534in,y=0.452778in,,top]{\color{textcolor}{\sffamily\fontsize{10.000000}{12.000000}\selectfont\catcode`\^=\active\def^{\ifmmode\sp\else\^{}\fi}\catcode`\%=\active\def%{\%}200}}%
\end{pgfscope}%
\begin{pgfscope}%
\pgfsetbuttcap%
\pgfsetroundjoin%
\definecolor{currentfill}{rgb}{0.000000,0.000000,0.000000}%
\pgfsetfillcolor{currentfill}%
\pgfsetlinewidth{0.803000pt}%
\definecolor{currentstroke}{rgb}{0.000000,0.000000,0.000000}%
\pgfsetstrokecolor{currentstroke}%
\pgfsetdash{}{0pt}%
\pgfsys@defobject{currentmarker}{\pgfqpoint{0.000000in}{-0.048611in}}{\pgfqpoint{0.000000in}{0.000000in}}{%
\pgfpathmoveto{\pgfqpoint{0.000000in}{0.000000in}}%
\pgfpathlineto{\pgfqpoint{0.000000in}{-0.048611in}}%
\pgfusepath{stroke,fill}%
}%
\begin{pgfscope}%
\pgfsys@transformshift{8.482599in}{0.550000in}%
\pgfsys@useobject{currentmarker}{}%
\end{pgfscope}%
\end{pgfscope}%
\begin{pgfscope}%
\definecolor{textcolor}{rgb}{0.000000,0.000000,0.000000}%
\pgfsetstrokecolor{textcolor}%
\pgfsetfillcolor{textcolor}%
\pgftext[x=8.482599in,y=0.452778in,,top]{\color{textcolor}{\sffamily\fontsize{10.000000}{12.000000}\selectfont\catcode`\^=\active\def^{\ifmmode\sp\else\^{}\fi}\catcode`\%=\active\def%{\%}250}}%
\end{pgfscope}%
\begin{pgfscope}%
\definecolor{textcolor}{rgb}{0.000000,0.000000,0.000000}%
\pgfsetstrokecolor{textcolor}%
\pgfsetfillcolor{textcolor}%
\pgftext[x=5.125000in,y=0.262809in,,top]{\color{textcolor}{\sffamily\fontsize{10.000000}{12.000000}\selectfont\catcode`\^=\active\def^{\ifmmode\sp\else\^{}\fi}\catcode`\%=\active\def%{\%}Intensidad}}%
\end{pgfscope}%
\begin{pgfscope}%
\pgfsetbuttcap%
\pgfsetroundjoin%
\definecolor{currentfill}{rgb}{0.000000,0.000000,0.000000}%
\pgfsetfillcolor{currentfill}%
\pgfsetlinewidth{0.803000pt}%
\definecolor{currentstroke}{rgb}{0.000000,0.000000,0.000000}%
\pgfsetstrokecolor{currentstroke}%
\pgfsetdash{}{0pt}%
\pgfsys@defobject{currentmarker}{\pgfqpoint{-0.048611in}{0.000000in}}{\pgfqpoint{-0.000000in}{0.000000in}}{%
\pgfpathmoveto{\pgfqpoint{-0.000000in}{0.000000in}}%
\pgfpathlineto{\pgfqpoint{-0.048611in}{0.000000in}}%
\pgfusepath{stroke,fill}%
}%
\begin{pgfscope}%
\pgfsys@transformshift{1.250000in}{0.550000in}%
\pgfsys@useobject{currentmarker}{}%
\end{pgfscope}%
\end{pgfscope}%
\begin{pgfscope}%
\definecolor{textcolor}{rgb}{0.000000,0.000000,0.000000}%
\pgfsetstrokecolor{textcolor}%
\pgfsetfillcolor{textcolor}%
\pgftext[x=1.064412in, y=0.497238in, left, base]{\color{textcolor}{\sffamily\fontsize{10.000000}{12.000000}\selectfont\catcode`\^=\active\def^{\ifmmode\sp\else\^{}\fi}\catcode`\%=\active\def%{\%}0}}%
\end{pgfscope}%
\begin{pgfscope}%
\pgfsetbuttcap%
\pgfsetroundjoin%
\definecolor{currentfill}{rgb}{0.000000,0.000000,0.000000}%
\pgfsetfillcolor{currentfill}%
\pgfsetlinewidth{0.803000pt}%
\definecolor{currentstroke}{rgb}{0.000000,0.000000,0.000000}%
\pgfsetstrokecolor{currentstroke}%
\pgfsetdash{}{0pt}%
\pgfsys@defobject{currentmarker}{\pgfqpoint{-0.048611in}{0.000000in}}{\pgfqpoint{-0.000000in}{0.000000in}}{%
\pgfpathmoveto{\pgfqpoint{-0.000000in}{0.000000in}}%
\pgfpathlineto{\pgfqpoint{-0.048611in}{0.000000in}}%
\pgfusepath{stroke,fill}%
}%
\begin{pgfscope}%
\pgfsys@transformshift{1.250000in}{1.262805in}%
\pgfsys@useobject{currentmarker}{}%
\end{pgfscope}%
\end{pgfscope}%
\begin{pgfscope}%
\definecolor{textcolor}{rgb}{0.000000,0.000000,0.000000}%
\pgfsetstrokecolor{textcolor}%
\pgfsetfillcolor{textcolor}%
\pgftext[x=0.887682in, y=1.210043in, left, base]{\color{textcolor}{\sffamily\fontsize{10.000000}{12.000000}\selectfont\catcode`\^=\active\def^{\ifmmode\sp\else\^{}\fi}\catcode`\%=\active\def%{\%}500}}%
\end{pgfscope}%
\begin{pgfscope}%
\pgfsetbuttcap%
\pgfsetroundjoin%
\definecolor{currentfill}{rgb}{0.000000,0.000000,0.000000}%
\pgfsetfillcolor{currentfill}%
\pgfsetlinewidth{0.803000pt}%
\definecolor{currentstroke}{rgb}{0.000000,0.000000,0.000000}%
\pgfsetstrokecolor{currentstroke}%
\pgfsetdash{}{0pt}%
\pgfsys@defobject{currentmarker}{\pgfqpoint{-0.048611in}{0.000000in}}{\pgfqpoint{-0.000000in}{0.000000in}}{%
\pgfpathmoveto{\pgfqpoint{-0.000000in}{0.000000in}}%
\pgfpathlineto{\pgfqpoint{-0.048611in}{0.000000in}}%
\pgfusepath{stroke,fill}%
}%
\begin{pgfscope}%
\pgfsys@transformshift{1.250000in}{1.975609in}%
\pgfsys@useobject{currentmarker}{}%
\end{pgfscope}%
\end{pgfscope}%
\begin{pgfscope}%
\definecolor{textcolor}{rgb}{0.000000,0.000000,0.000000}%
\pgfsetstrokecolor{textcolor}%
\pgfsetfillcolor{textcolor}%
\pgftext[x=0.799316in, y=1.922848in, left, base]{\color{textcolor}{\sffamily\fontsize{10.000000}{12.000000}\selectfont\catcode`\^=\active\def^{\ifmmode\sp\else\^{}\fi}\catcode`\%=\active\def%{\%}1000}}%
\end{pgfscope}%
\begin{pgfscope}%
\pgfsetbuttcap%
\pgfsetroundjoin%
\definecolor{currentfill}{rgb}{0.000000,0.000000,0.000000}%
\pgfsetfillcolor{currentfill}%
\pgfsetlinewidth{0.803000pt}%
\definecolor{currentstroke}{rgb}{0.000000,0.000000,0.000000}%
\pgfsetstrokecolor{currentstroke}%
\pgfsetdash{}{0pt}%
\pgfsys@defobject{currentmarker}{\pgfqpoint{-0.048611in}{0.000000in}}{\pgfqpoint{-0.000000in}{0.000000in}}{%
\pgfpathmoveto{\pgfqpoint{-0.000000in}{0.000000in}}%
\pgfpathlineto{\pgfqpoint{-0.048611in}{0.000000in}}%
\pgfusepath{stroke,fill}%
}%
\begin{pgfscope}%
\pgfsys@transformshift{1.250000in}{2.688414in}%
\pgfsys@useobject{currentmarker}{}%
\end{pgfscope}%
\end{pgfscope}%
\begin{pgfscope}%
\definecolor{textcolor}{rgb}{0.000000,0.000000,0.000000}%
\pgfsetstrokecolor{textcolor}%
\pgfsetfillcolor{textcolor}%
\pgftext[x=0.799316in, y=2.635652in, left, base]{\color{textcolor}{\sffamily\fontsize{10.000000}{12.000000}\selectfont\catcode`\^=\active\def^{\ifmmode\sp\else\^{}\fi}\catcode`\%=\active\def%{\%}1500}}%
\end{pgfscope}%
\begin{pgfscope}%
\pgfsetbuttcap%
\pgfsetroundjoin%
\definecolor{currentfill}{rgb}{0.000000,0.000000,0.000000}%
\pgfsetfillcolor{currentfill}%
\pgfsetlinewidth{0.803000pt}%
\definecolor{currentstroke}{rgb}{0.000000,0.000000,0.000000}%
\pgfsetstrokecolor{currentstroke}%
\pgfsetdash{}{0pt}%
\pgfsys@defobject{currentmarker}{\pgfqpoint{-0.048611in}{0.000000in}}{\pgfqpoint{-0.000000in}{0.000000in}}{%
\pgfpathmoveto{\pgfqpoint{-0.000000in}{0.000000in}}%
\pgfpathlineto{\pgfqpoint{-0.048611in}{0.000000in}}%
\pgfusepath{stroke,fill}%
}%
\begin{pgfscope}%
\pgfsys@transformshift{1.250000in}{3.401218in}%
\pgfsys@useobject{currentmarker}{}%
\end{pgfscope}%
\end{pgfscope}%
\begin{pgfscope}%
\definecolor{textcolor}{rgb}{0.000000,0.000000,0.000000}%
\pgfsetstrokecolor{textcolor}%
\pgfsetfillcolor{textcolor}%
\pgftext[x=0.799316in, y=3.348457in, left, base]{\color{textcolor}{\sffamily\fontsize{10.000000}{12.000000}\selectfont\catcode`\^=\active\def^{\ifmmode\sp\else\^{}\fi}\catcode`\%=\active\def%{\%}2000}}%
\end{pgfscope}%
\begin{pgfscope}%
\pgfsetbuttcap%
\pgfsetroundjoin%
\definecolor{currentfill}{rgb}{0.000000,0.000000,0.000000}%
\pgfsetfillcolor{currentfill}%
\pgfsetlinewidth{0.803000pt}%
\definecolor{currentstroke}{rgb}{0.000000,0.000000,0.000000}%
\pgfsetstrokecolor{currentstroke}%
\pgfsetdash{}{0pt}%
\pgfsys@defobject{currentmarker}{\pgfqpoint{-0.048611in}{0.000000in}}{\pgfqpoint{-0.000000in}{0.000000in}}{%
\pgfpathmoveto{\pgfqpoint{-0.000000in}{0.000000in}}%
\pgfpathlineto{\pgfqpoint{-0.048611in}{0.000000in}}%
\pgfusepath{stroke,fill}%
}%
\begin{pgfscope}%
\pgfsys@transformshift{1.250000in}{4.114023in}%
\pgfsys@useobject{currentmarker}{}%
\end{pgfscope}%
\end{pgfscope}%
\begin{pgfscope}%
\definecolor{textcolor}{rgb}{0.000000,0.000000,0.000000}%
\pgfsetstrokecolor{textcolor}%
\pgfsetfillcolor{textcolor}%
\pgftext[x=0.799316in, y=4.061261in, left, base]{\color{textcolor}{\sffamily\fontsize{10.000000}{12.000000}\selectfont\catcode`\^=\active\def^{\ifmmode\sp\else\^{}\fi}\catcode`\%=\active\def%{\%}2500}}%
\end{pgfscope}%
\begin{pgfscope}%
\definecolor{textcolor}{rgb}{0.000000,0.000000,0.000000}%
\pgfsetstrokecolor{textcolor}%
\pgfsetfillcolor{textcolor}%
\pgftext[x=0.743761in,y=2.475000in,,bottom,rotate=90.000000]{\color{textcolor}{\sffamily\fontsize{10.000000}{12.000000}\selectfont\catcode`\^=\active\def^{\ifmmode\sp\else\^{}\fi}\catcode`\%=\active\def%{\%}Frecuencia}}%
\end{pgfscope}%
\begin{pgfscope}%
\pgfsetrectcap%
\pgfsetmiterjoin%
\pgfsetlinewidth{0.803000pt}%
\definecolor{currentstroke}{rgb}{0.000000,0.000000,0.000000}%
\pgfsetstrokecolor{currentstroke}%
\pgfsetdash{}{0pt}%
\pgfpathmoveto{\pgfqpoint{1.250000in}{0.550000in}}%
\pgfpathlineto{\pgfqpoint{1.250000in}{4.400000in}}%
\pgfusepath{stroke}%
\end{pgfscope}%
\begin{pgfscope}%
\pgfsetrectcap%
\pgfsetmiterjoin%
\pgfsetlinewidth{0.803000pt}%
\definecolor{currentstroke}{rgb}{0.000000,0.000000,0.000000}%
\pgfsetstrokecolor{currentstroke}%
\pgfsetdash{}{0pt}%
\pgfpathmoveto{\pgfqpoint{9.000000in}{0.550000in}}%
\pgfpathlineto{\pgfqpoint{9.000000in}{4.400000in}}%
\pgfusepath{stroke}%
\end{pgfscope}%
\begin{pgfscope}%
\pgfsetrectcap%
\pgfsetmiterjoin%
\pgfsetlinewidth{0.803000pt}%
\definecolor{currentstroke}{rgb}{0.000000,0.000000,0.000000}%
\pgfsetstrokecolor{currentstroke}%
\pgfsetdash{}{0pt}%
\pgfpathmoveto{\pgfqpoint{1.250000in}{0.550000in}}%
\pgfpathlineto{\pgfqpoint{9.000000in}{0.550000in}}%
\pgfusepath{stroke}%
\end{pgfscope}%
\begin{pgfscope}%
\pgfsetrectcap%
\pgfsetmiterjoin%
\pgfsetlinewidth{0.803000pt}%
\definecolor{currentstroke}{rgb}{0.000000,0.000000,0.000000}%
\pgfsetstrokecolor{currentstroke}%
\pgfsetdash{}{0pt}%
\pgfpathmoveto{\pgfqpoint{1.250000in}{4.400000in}}%
\pgfpathlineto{\pgfqpoint{9.000000in}{4.400000in}}%
\pgfusepath{stroke}%
\end{pgfscope}%
\begin{pgfscope}%
\definecolor{textcolor}{rgb}{0.000000,0.000000,0.000000}%
\pgfsetstrokecolor{textcolor}%
\pgfsetfillcolor{textcolor}%
\pgftext[x=5.125000in,y=4.483333in,,base]{\color{textcolor}{\sffamily\fontsize{12.000000}{14.400000}\selectfont\catcode`\^=\active\def^{\ifmmode\sp\else\^{}\fi}\catcode`\%=\active\def%{\%}Histograma de DQHEPL}}%
\end{pgfscope}%
\end{pgfpicture}%
\makeatother%
\endgroup%
}
		\caption{DQHEPL}
		\label{fig:dqhepl_2}
	\end{subfigure}

	\vspace{5mm}

	\begin{subfigure}[b]{0.45\textwidth}
		\resizebox{\linewidth}{!}{%% Creator: Matplotlib, PGF backend
%%
%% To include the figure in your LaTeX document, write
%%   \input{<filename>.pgf}
%%
%% Make sure the required packages are loaded in your preamble
%%   \usepackage{pgf}
%%
%% Also ensure that all the required font packages are loaded; for instance,
%% the lmodern package is sometimes necessary when using math font.
%%   \usepackage{lmodern}
%%
%% Figures using additional raster images can only be included by \input if
%% they are in the same directory as the main LaTeX file. For loading figures
%% from other directories you can use the `import` package
%%   \usepackage{import}
%%
%% and then include the figures with
%%   \import{<path to file>}{<filename>.pgf}
%%
%% Matplotlib used the following preamble
%%   \def\mathdefault#1{#1}
%%   \everymath=\expandafter{\the\everymath\displaystyle}
%%   \IfFileExists{scrextend.sty}{
%%     \usepackage[fontsize=10.000000pt]{scrextend}
%%   }{
%%     \renewcommand{\normalsize}{\fontsize{10.000000}{12.000000}\selectfont}
%%     \normalsize
%%   }
%%   
%%   \ifdefined\pdftexversion\else  % non-pdftex case.
%%     \usepackage{fontspec}
%%     \setmainfont{DejaVuSerif.ttf}[Path=\detokenize{/home/elias/.virtualenvs/src-pway/lib/python3.13/site-packages/matplotlib/mpl-data/fonts/ttf/}]
%%     \setsansfont{DejaVuSans.ttf}[Path=\detokenize{/home/elias/.virtualenvs/src-pway/lib/python3.13/site-packages/matplotlib/mpl-data/fonts/ttf/}]
%%     \setmonofont{DejaVuSansMono.ttf}[Path=\detokenize{/home/elias/.virtualenvs/src-pway/lib/python3.13/site-packages/matplotlib/mpl-data/fonts/ttf/}]
%%   \fi
%%   \makeatletter\@ifpackageloaded{underscore}{}{\usepackage[strings]{underscore}}\makeatother
%%
\begingroup%
\makeatletter%
\begin{pgfpicture}%
\pgfpathrectangle{\pgfpointorigin}{\pgfqpoint{10.000000in}{5.000000in}}%
\pgfusepath{use as bounding box, clip}%
\begin{pgfscope}%
\pgfsetbuttcap%
\pgfsetmiterjoin%
\definecolor{currentfill}{rgb}{1.000000,1.000000,1.000000}%
\pgfsetfillcolor{currentfill}%
\pgfsetlinewidth{0.000000pt}%
\definecolor{currentstroke}{rgb}{1.000000,1.000000,1.000000}%
\pgfsetstrokecolor{currentstroke}%
\pgfsetdash{}{0pt}%
\pgfpathmoveto{\pgfqpoint{0.000000in}{0.000000in}}%
\pgfpathlineto{\pgfqpoint{10.000000in}{0.000000in}}%
\pgfpathlineto{\pgfqpoint{10.000000in}{5.000000in}}%
\pgfpathlineto{\pgfqpoint{0.000000in}{5.000000in}}%
\pgfpathlineto{\pgfqpoint{0.000000in}{0.000000in}}%
\pgfpathclose%
\pgfusepath{fill}%
\end{pgfscope}%
\begin{pgfscope}%
\pgfsetbuttcap%
\pgfsetmiterjoin%
\definecolor{currentfill}{rgb}{1.000000,1.000000,1.000000}%
\pgfsetfillcolor{currentfill}%
\pgfsetlinewidth{0.000000pt}%
\definecolor{currentstroke}{rgb}{0.000000,0.000000,0.000000}%
\pgfsetstrokecolor{currentstroke}%
\pgfsetstrokeopacity{0.000000}%
\pgfsetdash{}{0pt}%
\pgfpathmoveto{\pgfqpoint{1.250000in}{0.550000in}}%
\pgfpathlineto{\pgfqpoint{9.000000in}{0.550000in}}%
\pgfpathlineto{\pgfqpoint{9.000000in}{4.400000in}}%
\pgfpathlineto{\pgfqpoint{1.250000in}{4.400000in}}%
\pgfpathlineto{\pgfqpoint{1.250000in}{0.550000in}}%
\pgfpathclose%
\pgfusepath{fill}%
\end{pgfscope}%
\begin{pgfscope}%
\pgfpathrectangle{\pgfqpoint{1.250000in}{0.550000in}}{\pgfqpoint{7.750000in}{3.850000in}}%
\pgfusepath{clip}%
\pgfsetbuttcap%
\pgfsetmiterjoin%
\definecolor{currentfill}{rgb}{0.501961,0.501961,0.501961}%
\pgfsetfillcolor{currentfill}%
\pgfsetfillopacity{0.700000}%
\pgfsetlinewidth{0.000000pt}%
\definecolor{currentstroke}{rgb}{0.000000,0.000000,0.000000}%
\pgfsetstrokecolor{currentstroke}%
\pgfsetstrokeopacity{0.700000}%
\pgfsetdash{}{0pt}%
\pgfpathmoveto{\pgfqpoint{1.602273in}{0.550000in}}%
\pgfpathlineto{\pgfqpoint{1.629794in}{0.550000in}}%
\pgfpathlineto{\pgfqpoint{1.629794in}{4.216667in}}%
\pgfpathlineto{\pgfqpoint{1.602273in}{4.216667in}}%
\pgfpathlineto{\pgfqpoint{1.602273in}{0.550000in}}%
\pgfpathclose%
\pgfusepath{fill}%
\end{pgfscope}%
\begin{pgfscope}%
\pgfpathrectangle{\pgfqpoint{1.250000in}{0.550000in}}{\pgfqpoint{7.750000in}{3.850000in}}%
\pgfusepath{clip}%
\pgfsetbuttcap%
\pgfsetmiterjoin%
\definecolor{currentfill}{rgb}{0.501961,0.501961,0.501961}%
\pgfsetfillcolor{currentfill}%
\pgfsetfillopacity{0.700000}%
\pgfsetlinewidth{0.000000pt}%
\definecolor{currentstroke}{rgb}{0.000000,0.000000,0.000000}%
\pgfsetstrokecolor{currentstroke}%
\pgfsetstrokeopacity{0.700000}%
\pgfsetdash{}{0pt}%
\pgfpathmoveto{\pgfqpoint{1.629794in}{0.550000in}}%
\pgfpathlineto{\pgfqpoint{1.657315in}{0.550000in}}%
\pgfpathlineto{\pgfqpoint{1.657315in}{0.550000in}}%
\pgfpathlineto{\pgfqpoint{1.629794in}{0.550000in}}%
\pgfpathlineto{\pgfqpoint{1.629794in}{0.550000in}}%
\pgfpathclose%
\pgfusepath{fill}%
\end{pgfscope}%
\begin{pgfscope}%
\pgfpathrectangle{\pgfqpoint{1.250000in}{0.550000in}}{\pgfqpoint{7.750000in}{3.850000in}}%
\pgfusepath{clip}%
\pgfsetbuttcap%
\pgfsetmiterjoin%
\definecolor{currentfill}{rgb}{0.501961,0.501961,0.501961}%
\pgfsetfillcolor{currentfill}%
\pgfsetfillopacity{0.700000}%
\pgfsetlinewidth{0.000000pt}%
\definecolor{currentstroke}{rgb}{0.000000,0.000000,0.000000}%
\pgfsetstrokecolor{currentstroke}%
\pgfsetstrokeopacity{0.700000}%
\pgfsetdash{}{0pt}%
\pgfpathmoveto{\pgfqpoint{1.657315in}{0.550000in}}%
\pgfpathlineto{\pgfqpoint{1.684837in}{0.550000in}}%
\pgfpathlineto{\pgfqpoint{1.684837in}{0.550000in}}%
\pgfpathlineto{\pgfqpoint{1.657315in}{0.550000in}}%
\pgfpathlineto{\pgfqpoint{1.657315in}{0.550000in}}%
\pgfpathclose%
\pgfusepath{fill}%
\end{pgfscope}%
\begin{pgfscope}%
\pgfpathrectangle{\pgfqpoint{1.250000in}{0.550000in}}{\pgfqpoint{7.750000in}{3.850000in}}%
\pgfusepath{clip}%
\pgfsetbuttcap%
\pgfsetmiterjoin%
\definecolor{currentfill}{rgb}{0.501961,0.501961,0.501961}%
\pgfsetfillcolor{currentfill}%
\pgfsetfillopacity{0.700000}%
\pgfsetlinewidth{0.000000pt}%
\definecolor{currentstroke}{rgb}{0.000000,0.000000,0.000000}%
\pgfsetstrokecolor{currentstroke}%
\pgfsetstrokeopacity{0.700000}%
\pgfsetdash{}{0pt}%
\pgfpathmoveto{\pgfqpoint{1.684837in}{0.550000in}}%
\pgfpathlineto{\pgfqpoint{1.712358in}{0.550000in}}%
\pgfpathlineto{\pgfqpoint{1.712358in}{0.550000in}}%
\pgfpathlineto{\pgfqpoint{1.684837in}{0.550000in}}%
\pgfpathlineto{\pgfqpoint{1.684837in}{0.550000in}}%
\pgfpathclose%
\pgfusepath{fill}%
\end{pgfscope}%
\begin{pgfscope}%
\pgfpathrectangle{\pgfqpoint{1.250000in}{0.550000in}}{\pgfqpoint{7.750000in}{3.850000in}}%
\pgfusepath{clip}%
\pgfsetbuttcap%
\pgfsetmiterjoin%
\definecolor{currentfill}{rgb}{0.501961,0.501961,0.501961}%
\pgfsetfillcolor{currentfill}%
\pgfsetfillopacity{0.700000}%
\pgfsetlinewidth{0.000000pt}%
\definecolor{currentstroke}{rgb}{0.000000,0.000000,0.000000}%
\pgfsetstrokecolor{currentstroke}%
\pgfsetstrokeopacity{0.700000}%
\pgfsetdash{}{0pt}%
\pgfpathmoveto{\pgfqpoint{1.712358in}{0.550000in}}%
\pgfpathlineto{\pgfqpoint{1.739879in}{0.550000in}}%
\pgfpathlineto{\pgfqpoint{1.739879in}{0.550000in}}%
\pgfpathlineto{\pgfqpoint{1.712358in}{0.550000in}}%
\pgfpathlineto{\pgfqpoint{1.712358in}{0.550000in}}%
\pgfpathclose%
\pgfusepath{fill}%
\end{pgfscope}%
\begin{pgfscope}%
\pgfpathrectangle{\pgfqpoint{1.250000in}{0.550000in}}{\pgfqpoint{7.750000in}{3.850000in}}%
\pgfusepath{clip}%
\pgfsetbuttcap%
\pgfsetmiterjoin%
\definecolor{currentfill}{rgb}{0.501961,0.501961,0.501961}%
\pgfsetfillcolor{currentfill}%
\pgfsetfillopacity{0.700000}%
\pgfsetlinewidth{0.000000pt}%
\definecolor{currentstroke}{rgb}{0.000000,0.000000,0.000000}%
\pgfsetstrokecolor{currentstroke}%
\pgfsetstrokeopacity{0.700000}%
\pgfsetdash{}{0pt}%
\pgfpathmoveto{\pgfqpoint{1.739879in}{0.550000in}}%
\pgfpathlineto{\pgfqpoint{1.767401in}{0.550000in}}%
\pgfpathlineto{\pgfqpoint{1.767401in}{0.550000in}}%
\pgfpathlineto{\pgfqpoint{1.739879in}{0.550000in}}%
\pgfpathlineto{\pgfqpoint{1.739879in}{0.550000in}}%
\pgfpathclose%
\pgfusepath{fill}%
\end{pgfscope}%
\begin{pgfscope}%
\pgfpathrectangle{\pgfqpoint{1.250000in}{0.550000in}}{\pgfqpoint{7.750000in}{3.850000in}}%
\pgfusepath{clip}%
\pgfsetbuttcap%
\pgfsetmiterjoin%
\definecolor{currentfill}{rgb}{0.501961,0.501961,0.501961}%
\pgfsetfillcolor{currentfill}%
\pgfsetfillopacity{0.700000}%
\pgfsetlinewidth{0.000000pt}%
\definecolor{currentstroke}{rgb}{0.000000,0.000000,0.000000}%
\pgfsetstrokecolor{currentstroke}%
\pgfsetstrokeopacity{0.700000}%
\pgfsetdash{}{0pt}%
\pgfpathmoveto{\pgfqpoint{1.767401in}{0.550000in}}%
\pgfpathlineto{\pgfqpoint{1.794922in}{0.550000in}}%
\pgfpathlineto{\pgfqpoint{1.794922in}{0.550000in}}%
\pgfpathlineto{\pgfqpoint{1.767401in}{0.550000in}}%
\pgfpathlineto{\pgfqpoint{1.767401in}{0.550000in}}%
\pgfpathclose%
\pgfusepath{fill}%
\end{pgfscope}%
\begin{pgfscope}%
\pgfpathrectangle{\pgfqpoint{1.250000in}{0.550000in}}{\pgfqpoint{7.750000in}{3.850000in}}%
\pgfusepath{clip}%
\pgfsetbuttcap%
\pgfsetmiterjoin%
\definecolor{currentfill}{rgb}{0.501961,0.501961,0.501961}%
\pgfsetfillcolor{currentfill}%
\pgfsetfillopacity{0.700000}%
\pgfsetlinewidth{0.000000pt}%
\definecolor{currentstroke}{rgb}{0.000000,0.000000,0.000000}%
\pgfsetstrokecolor{currentstroke}%
\pgfsetstrokeopacity{0.700000}%
\pgfsetdash{}{0pt}%
\pgfpathmoveto{\pgfqpoint{1.794922in}{0.550000in}}%
\pgfpathlineto{\pgfqpoint{1.822443in}{0.550000in}}%
\pgfpathlineto{\pgfqpoint{1.822443in}{0.550000in}}%
\pgfpathlineto{\pgfqpoint{1.794922in}{0.550000in}}%
\pgfpathlineto{\pgfqpoint{1.794922in}{0.550000in}}%
\pgfpathclose%
\pgfusepath{fill}%
\end{pgfscope}%
\begin{pgfscope}%
\pgfpathrectangle{\pgfqpoint{1.250000in}{0.550000in}}{\pgfqpoint{7.750000in}{3.850000in}}%
\pgfusepath{clip}%
\pgfsetbuttcap%
\pgfsetmiterjoin%
\definecolor{currentfill}{rgb}{0.501961,0.501961,0.501961}%
\pgfsetfillcolor{currentfill}%
\pgfsetfillopacity{0.700000}%
\pgfsetlinewidth{0.000000pt}%
\definecolor{currentstroke}{rgb}{0.000000,0.000000,0.000000}%
\pgfsetstrokecolor{currentstroke}%
\pgfsetstrokeopacity{0.700000}%
\pgfsetdash{}{0pt}%
\pgfpathmoveto{\pgfqpoint{1.822443in}{0.550000in}}%
\pgfpathlineto{\pgfqpoint{1.849964in}{0.550000in}}%
\pgfpathlineto{\pgfqpoint{1.849964in}{0.550000in}}%
\pgfpathlineto{\pgfqpoint{1.822443in}{0.550000in}}%
\pgfpathlineto{\pgfqpoint{1.822443in}{0.550000in}}%
\pgfpathclose%
\pgfusepath{fill}%
\end{pgfscope}%
\begin{pgfscope}%
\pgfpathrectangle{\pgfqpoint{1.250000in}{0.550000in}}{\pgfqpoint{7.750000in}{3.850000in}}%
\pgfusepath{clip}%
\pgfsetbuttcap%
\pgfsetmiterjoin%
\definecolor{currentfill}{rgb}{0.501961,0.501961,0.501961}%
\pgfsetfillcolor{currentfill}%
\pgfsetfillopacity{0.700000}%
\pgfsetlinewidth{0.000000pt}%
\definecolor{currentstroke}{rgb}{0.000000,0.000000,0.000000}%
\pgfsetstrokecolor{currentstroke}%
\pgfsetstrokeopacity{0.700000}%
\pgfsetdash{}{0pt}%
\pgfpathmoveto{\pgfqpoint{1.849964in}{0.550000in}}%
\pgfpathlineto{\pgfqpoint{1.877486in}{0.550000in}}%
\pgfpathlineto{\pgfqpoint{1.877486in}{0.550000in}}%
\pgfpathlineto{\pgfqpoint{1.849964in}{0.550000in}}%
\pgfpathlineto{\pgfqpoint{1.849964in}{0.550000in}}%
\pgfpathclose%
\pgfusepath{fill}%
\end{pgfscope}%
\begin{pgfscope}%
\pgfpathrectangle{\pgfqpoint{1.250000in}{0.550000in}}{\pgfqpoint{7.750000in}{3.850000in}}%
\pgfusepath{clip}%
\pgfsetbuttcap%
\pgfsetmiterjoin%
\definecolor{currentfill}{rgb}{0.501961,0.501961,0.501961}%
\pgfsetfillcolor{currentfill}%
\pgfsetfillopacity{0.700000}%
\pgfsetlinewidth{0.000000pt}%
\definecolor{currentstroke}{rgb}{0.000000,0.000000,0.000000}%
\pgfsetstrokecolor{currentstroke}%
\pgfsetstrokeopacity{0.700000}%
\pgfsetdash{}{0pt}%
\pgfpathmoveto{\pgfqpoint{1.877486in}{0.550000in}}%
\pgfpathlineto{\pgfqpoint{1.905007in}{0.550000in}}%
\pgfpathlineto{\pgfqpoint{1.905007in}{0.550000in}}%
\pgfpathlineto{\pgfqpoint{1.877486in}{0.550000in}}%
\pgfpathlineto{\pgfqpoint{1.877486in}{0.550000in}}%
\pgfpathclose%
\pgfusepath{fill}%
\end{pgfscope}%
\begin{pgfscope}%
\pgfpathrectangle{\pgfqpoint{1.250000in}{0.550000in}}{\pgfqpoint{7.750000in}{3.850000in}}%
\pgfusepath{clip}%
\pgfsetbuttcap%
\pgfsetmiterjoin%
\definecolor{currentfill}{rgb}{0.501961,0.501961,0.501961}%
\pgfsetfillcolor{currentfill}%
\pgfsetfillopacity{0.700000}%
\pgfsetlinewidth{0.000000pt}%
\definecolor{currentstroke}{rgb}{0.000000,0.000000,0.000000}%
\pgfsetstrokecolor{currentstroke}%
\pgfsetstrokeopacity{0.700000}%
\pgfsetdash{}{0pt}%
\pgfpathmoveto{\pgfqpoint{1.905007in}{0.550000in}}%
\pgfpathlineto{\pgfqpoint{1.932528in}{0.550000in}}%
\pgfpathlineto{\pgfqpoint{1.932528in}{0.550000in}}%
\pgfpathlineto{\pgfqpoint{1.905007in}{0.550000in}}%
\pgfpathlineto{\pgfqpoint{1.905007in}{0.550000in}}%
\pgfpathclose%
\pgfusepath{fill}%
\end{pgfscope}%
\begin{pgfscope}%
\pgfpathrectangle{\pgfqpoint{1.250000in}{0.550000in}}{\pgfqpoint{7.750000in}{3.850000in}}%
\pgfusepath{clip}%
\pgfsetbuttcap%
\pgfsetmiterjoin%
\definecolor{currentfill}{rgb}{0.501961,0.501961,0.501961}%
\pgfsetfillcolor{currentfill}%
\pgfsetfillopacity{0.700000}%
\pgfsetlinewidth{0.000000pt}%
\definecolor{currentstroke}{rgb}{0.000000,0.000000,0.000000}%
\pgfsetstrokecolor{currentstroke}%
\pgfsetstrokeopacity{0.700000}%
\pgfsetdash{}{0pt}%
\pgfpathmoveto{\pgfqpoint{1.932528in}{0.550000in}}%
\pgfpathlineto{\pgfqpoint{1.960050in}{0.550000in}}%
\pgfpathlineto{\pgfqpoint{1.960050in}{0.550000in}}%
\pgfpathlineto{\pgfqpoint{1.932528in}{0.550000in}}%
\pgfpathlineto{\pgfqpoint{1.932528in}{0.550000in}}%
\pgfpathclose%
\pgfusepath{fill}%
\end{pgfscope}%
\begin{pgfscope}%
\pgfpathrectangle{\pgfqpoint{1.250000in}{0.550000in}}{\pgfqpoint{7.750000in}{3.850000in}}%
\pgfusepath{clip}%
\pgfsetbuttcap%
\pgfsetmiterjoin%
\definecolor{currentfill}{rgb}{0.501961,0.501961,0.501961}%
\pgfsetfillcolor{currentfill}%
\pgfsetfillopacity{0.700000}%
\pgfsetlinewidth{0.000000pt}%
\definecolor{currentstroke}{rgb}{0.000000,0.000000,0.000000}%
\pgfsetstrokecolor{currentstroke}%
\pgfsetstrokeopacity{0.700000}%
\pgfsetdash{}{0pt}%
\pgfpathmoveto{\pgfqpoint{1.960050in}{0.550000in}}%
\pgfpathlineto{\pgfqpoint{1.987571in}{0.550000in}}%
\pgfpathlineto{\pgfqpoint{1.987571in}{0.550000in}}%
\pgfpathlineto{\pgfqpoint{1.960050in}{0.550000in}}%
\pgfpathlineto{\pgfqpoint{1.960050in}{0.550000in}}%
\pgfpathclose%
\pgfusepath{fill}%
\end{pgfscope}%
\begin{pgfscope}%
\pgfpathrectangle{\pgfqpoint{1.250000in}{0.550000in}}{\pgfqpoint{7.750000in}{3.850000in}}%
\pgfusepath{clip}%
\pgfsetbuttcap%
\pgfsetmiterjoin%
\definecolor{currentfill}{rgb}{0.501961,0.501961,0.501961}%
\pgfsetfillcolor{currentfill}%
\pgfsetfillopacity{0.700000}%
\pgfsetlinewidth{0.000000pt}%
\definecolor{currentstroke}{rgb}{0.000000,0.000000,0.000000}%
\pgfsetstrokecolor{currentstroke}%
\pgfsetstrokeopacity{0.700000}%
\pgfsetdash{}{0pt}%
\pgfpathmoveto{\pgfqpoint{1.987571in}{0.550000in}}%
\pgfpathlineto{\pgfqpoint{2.015092in}{0.550000in}}%
\pgfpathlineto{\pgfqpoint{2.015092in}{0.550000in}}%
\pgfpathlineto{\pgfqpoint{1.987571in}{0.550000in}}%
\pgfpathlineto{\pgfqpoint{1.987571in}{0.550000in}}%
\pgfpathclose%
\pgfusepath{fill}%
\end{pgfscope}%
\begin{pgfscope}%
\pgfpathrectangle{\pgfqpoint{1.250000in}{0.550000in}}{\pgfqpoint{7.750000in}{3.850000in}}%
\pgfusepath{clip}%
\pgfsetbuttcap%
\pgfsetmiterjoin%
\definecolor{currentfill}{rgb}{0.501961,0.501961,0.501961}%
\pgfsetfillcolor{currentfill}%
\pgfsetfillopacity{0.700000}%
\pgfsetlinewidth{0.000000pt}%
\definecolor{currentstroke}{rgb}{0.000000,0.000000,0.000000}%
\pgfsetstrokecolor{currentstroke}%
\pgfsetstrokeopacity{0.700000}%
\pgfsetdash{}{0pt}%
\pgfpathmoveto{\pgfqpoint{2.015092in}{0.550000in}}%
\pgfpathlineto{\pgfqpoint{2.042614in}{0.550000in}}%
\pgfpathlineto{\pgfqpoint{2.042614in}{0.550000in}}%
\pgfpathlineto{\pgfqpoint{2.015092in}{0.550000in}}%
\pgfpathlineto{\pgfqpoint{2.015092in}{0.550000in}}%
\pgfpathclose%
\pgfusepath{fill}%
\end{pgfscope}%
\begin{pgfscope}%
\pgfpathrectangle{\pgfqpoint{1.250000in}{0.550000in}}{\pgfqpoint{7.750000in}{3.850000in}}%
\pgfusepath{clip}%
\pgfsetbuttcap%
\pgfsetmiterjoin%
\definecolor{currentfill}{rgb}{0.501961,0.501961,0.501961}%
\pgfsetfillcolor{currentfill}%
\pgfsetfillopacity{0.700000}%
\pgfsetlinewidth{0.000000pt}%
\definecolor{currentstroke}{rgb}{0.000000,0.000000,0.000000}%
\pgfsetstrokecolor{currentstroke}%
\pgfsetstrokeopacity{0.700000}%
\pgfsetdash{}{0pt}%
\pgfpathmoveto{\pgfqpoint{2.042614in}{0.550000in}}%
\pgfpathlineto{\pgfqpoint{2.070135in}{0.550000in}}%
\pgfpathlineto{\pgfqpoint{2.070135in}{0.550000in}}%
\pgfpathlineto{\pgfqpoint{2.042614in}{0.550000in}}%
\pgfpathlineto{\pgfqpoint{2.042614in}{0.550000in}}%
\pgfpathclose%
\pgfusepath{fill}%
\end{pgfscope}%
\begin{pgfscope}%
\pgfpathrectangle{\pgfqpoint{1.250000in}{0.550000in}}{\pgfqpoint{7.750000in}{3.850000in}}%
\pgfusepath{clip}%
\pgfsetbuttcap%
\pgfsetmiterjoin%
\definecolor{currentfill}{rgb}{0.501961,0.501961,0.501961}%
\pgfsetfillcolor{currentfill}%
\pgfsetfillopacity{0.700000}%
\pgfsetlinewidth{0.000000pt}%
\definecolor{currentstroke}{rgb}{0.000000,0.000000,0.000000}%
\pgfsetstrokecolor{currentstroke}%
\pgfsetstrokeopacity{0.700000}%
\pgfsetdash{}{0pt}%
\pgfpathmoveto{\pgfqpoint{2.070135in}{0.550000in}}%
\pgfpathlineto{\pgfqpoint{2.097656in}{0.550000in}}%
\pgfpathlineto{\pgfqpoint{2.097656in}{0.550000in}}%
\pgfpathlineto{\pgfqpoint{2.070135in}{0.550000in}}%
\pgfpathlineto{\pgfqpoint{2.070135in}{0.550000in}}%
\pgfpathclose%
\pgfusepath{fill}%
\end{pgfscope}%
\begin{pgfscope}%
\pgfpathrectangle{\pgfqpoint{1.250000in}{0.550000in}}{\pgfqpoint{7.750000in}{3.850000in}}%
\pgfusepath{clip}%
\pgfsetbuttcap%
\pgfsetmiterjoin%
\definecolor{currentfill}{rgb}{0.501961,0.501961,0.501961}%
\pgfsetfillcolor{currentfill}%
\pgfsetfillopacity{0.700000}%
\pgfsetlinewidth{0.000000pt}%
\definecolor{currentstroke}{rgb}{0.000000,0.000000,0.000000}%
\pgfsetstrokecolor{currentstroke}%
\pgfsetstrokeopacity{0.700000}%
\pgfsetdash{}{0pt}%
\pgfpathmoveto{\pgfqpoint{2.097656in}{0.550000in}}%
\pgfpathlineto{\pgfqpoint{2.125178in}{0.550000in}}%
\pgfpathlineto{\pgfqpoint{2.125178in}{0.550000in}}%
\pgfpathlineto{\pgfqpoint{2.097656in}{0.550000in}}%
\pgfpathlineto{\pgfqpoint{2.097656in}{0.550000in}}%
\pgfpathclose%
\pgfusepath{fill}%
\end{pgfscope}%
\begin{pgfscope}%
\pgfpathrectangle{\pgfqpoint{1.250000in}{0.550000in}}{\pgfqpoint{7.750000in}{3.850000in}}%
\pgfusepath{clip}%
\pgfsetbuttcap%
\pgfsetmiterjoin%
\definecolor{currentfill}{rgb}{0.501961,0.501961,0.501961}%
\pgfsetfillcolor{currentfill}%
\pgfsetfillopacity{0.700000}%
\pgfsetlinewidth{0.000000pt}%
\definecolor{currentstroke}{rgb}{0.000000,0.000000,0.000000}%
\pgfsetstrokecolor{currentstroke}%
\pgfsetstrokeopacity{0.700000}%
\pgfsetdash{}{0pt}%
\pgfpathmoveto{\pgfqpoint{2.125178in}{0.550000in}}%
\pgfpathlineto{\pgfqpoint{2.152699in}{0.550000in}}%
\pgfpathlineto{\pgfqpoint{2.152699in}{0.550000in}}%
\pgfpathlineto{\pgfqpoint{2.125178in}{0.550000in}}%
\pgfpathlineto{\pgfqpoint{2.125178in}{0.550000in}}%
\pgfpathclose%
\pgfusepath{fill}%
\end{pgfscope}%
\begin{pgfscope}%
\pgfpathrectangle{\pgfqpoint{1.250000in}{0.550000in}}{\pgfqpoint{7.750000in}{3.850000in}}%
\pgfusepath{clip}%
\pgfsetbuttcap%
\pgfsetmiterjoin%
\definecolor{currentfill}{rgb}{0.501961,0.501961,0.501961}%
\pgfsetfillcolor{currentfill}%
\pgfsetfillopacity{0.700000}%
\pgfsetlinewidth{0.000000pt}%
\definecolor{currentstroke}{rgb}{0.000000,0.000000,0.000000}%
\pgfsetstrokecolor{currentstroke}%
\pgfsetstrokeopacity{0.700000}%
\pgfsetdash{}{0pt}%
\pgfpathmoveto{\pgfqpoint{2.152699in}{0.550000in}}%
\pgfpathlineto{\pgfqpoint{2.180220in}{0.550000in}}%
\pgfpathlineto{\pgfqpoint{2.180220in}{3.683059in}}%
\pgfpathlineto{\pgfqpoint{2.152699in}{3.683059in}}%
\pgfpathlineto{\pgfqpoint{2.152699in}{0.550000in}}%
\pgfpathclose%
\pgfusepath{fill}%
\end{pgfscope}%
\begin{pgfscope}%
\pgfpathrectangle{\pgfqpoint{1.250000in}{0.550000in}}{\pgfqpoint{7.750000in}{3.850000in}}%
\pgfusepath{clip}%
\pgfsetbuttcap%
\pgfsetmiterjoin%
\definecolor{currentfill}{rgb}{0.501961,0.501961,0.501961}%
\pgfsetfillcolor{currentfill}%
\pgfsetfillopacity{0.700000}%
\pgfsetlinewidth{0.000000pt}%
\definecolor{currentstroke}{rgb}{0.000000,0.000000,0.000000}%
\pgfsetstrokecolor{currentstroke}%
\pgfsetstrokeopacity{0.700000}%
\pgfsetdash{}{0pt}%
\pgfpathmoveto{\pgfqpoint{2.180220in}{0.550000in}}%
\pgfpathlineto{\pgfqpoint{2.207741in}{0.550000in}}%
\pgfpathlineto{\pgfqpoint{2.207741in}{0.550000in}}%
\pgfpathlineto{\pgfqpoint{2.180220in}{0.550000in}}%
\pgfpathlineto{\pgfqpoint{2.180220in}{0.550000in}}%
\pgfpathclose%
\pgfusepath{fill}%
\end{pgfscope}%
\begin{pgfscope}%
\pgfpathrectangle{\pgfqpoint{1.250000in}{0.550000in}}{\pgfqpoint{7.750000in}{3.850000in}}%
\pgfusepath{clip}%
\pgfsetbuttcap%
\pgfsetmiterjoin%
\definecolor{currentfill}{rgb}{0.501961,0.501961,0.501961}%
\pgfsetfillcolor{currentfill}%
\pgfsetfillopacity{0.700000}%
\pgfsetlinewidth{0.000000pt}%
\definecolor{currentstroke}{rgb}{0.000000,0.000000,0.000000}%
\pgfsetstrokecolor{currentstroke}%
\pgfsetstrokeopacity{0.700000}%
\pgfsetdash{}{0pt}%
\pgfpathmoveto{\pgfqpoint{2.207741in}{0.550000in}}%
\pgfpathlineto{\pgfqpoint{2.235263in}{0.550000in}}%
\pgfpathlineto{\pgfqpoint{2.235263in}{0.550000in}}%
\pgfpathlineto{\pgfqpoint{2.207741in}{0.550000in}}%
\pgfpathlineto{\pgfqpoint{2.207741in}{0.550000in}}%
\pgfpathclose%
\pgfusepath{fill}%
\end{pgfscope}%
\begin{pgfscope}%
\pgfpathrectangle{\pgfqpoint{1.250000in}{0.550000in}}{\pgfqpoint{7.750000in}{3.850000in}}%
\pgfusepath{clip}%
\pgfsetbuttcap%
\pgfsetmiterjoin%
\definecolor{currentfill}{rgb}{0.501961,0.501961,0.501961}%
\pgfsetfillcolor{currentfill}%
\pgfsetfillopacity{0.700000}%
\pgfsetlinewidth{0.000000pt}%
\definecolor{currentstroke}{rgb}{0.000000,0.000000,0.000000}%
\pgfsetstrokecolor{currentstroke}%
\pgfsetstrokeopacity{0.700000}%
\pgfsetdash{}{0pt}%
\pgfpathmoveto{\pgfqpoint{2.235263in}{0.550000in}}%
\pgfpathlineto{\pgfqpoint{2.262784in}{0.550000in}}%
\pgfpathlineto{\pgfqpoint{2.262784in}{0.550000in}}%
\pgfpathlineto{\pgfqpoint{2.235263in}{0.550000in}}%
\pgfpathlineto{\pgfqpoint{2.235263in}{0.550000in}}%
\pgfpathclose%
\pgfusepath{fill}%
\end{pgfscope}%
\begin{pgfscope}%
\pgfpathrectangle{\pgfqpoint{1.250000in}{0.550000in}}{\pgfqpoint{7.750000in}{3.850000in}}%
\pgfusepath{clip}%
\pgfsetbuttcap%
\pgfsetmiterjoin%
\definecolor{currentfill}{rgb}{0.501961,0.501961,0.501961}%
\pgfsetfillcolor{currentfill}%
\pgfsetfillopacity{0.700000}%
\pgfsetlinewidth{0.000000pt}%
\definecolor{currentstroke}{rgb}{0.000000,0.000000,0.000000}%
\pgfsetstrokecolor{currentstroke}%
\pgfsetstrokeopacity{0.700000}%
\pgfsetdash{}{0pt}%
\pgfpathmoveto{\pgfqpoint{2.262784in}{0.550000in}}%
\pgfpathlineto{\pgfqpoint{2.290305in}{0.550000in}}%
\pgfpathlineto{\pgfqpoint{2.290305in}{0.550000in}}%
\pgfpathlineto{\pgfqpoint{2.262784in}{0.550000in}}%
\pgfpathlineto{\pgfqpoint{2.262784in}{0.550000in}}%
\pgfpathclose%
\pgfusepath{fill}%
\end{pgfscope}%
\begin{pgfscope}%
\pgfpathrectangle{\pgfqpoint{1.250000in}{0.550000in}}{\pgfqpoint{7.750000in}{3.850000in}}%
\pgfusepath{clip}%
\pgfsetbuttcap%
\pgfsetmiterjoin%
\definecolor{currentfill}{rgb}{0.501961,0.501961,0.501961}%
\pgfsetfillcolor{currentfill}%
\pgfsetfillopacity{0.700000}%
\pgfsetlinewidth{0.000000pt}%
\definecolor{currentstroke}{rgb}{0.000000,0.000000,0.000000}%
\pgfsetstrokecolor{currentstroke}%
\pgfsetstrokeopacity{0.700000}%
\pgfsetdash{}{0pt}%
\pgfpathmoveto{\pgfqpoint{2.290305in}{0.550000in}}%
\pgfpathlineto{\pgfqpoint{2.317827in}{0.550000in}}%
\pgfpathlineto{\pgfqpoint{2.317827in}{0.550000in}}%
\pgfpathlineto{\pgfqpoint{2.290305in}{0.550000in}}%
\pgfpathlineto{\pgfqpoint{2.290305in}{0.550000in}}%
\pgfpathclose%
\pgfusepath{fill}%
\end{pgfscope}%
\begin{pgfscope}%
\pgfpathrectangle{\pgfqpoint{1.250000in}{0.550000in}}{\pgfqpoint{7.750000in}{3.850000in}}%
\pgfusepath{clip}%
\pgfsetbuttcap%
\pgfsetmiterjoin%
\definecolor{currentfill}{rgb}{0.501961,0.501961,0.501961}%
\pgfsetfillcolor{currentfill}%
\pgfsetfillopacity{0.700000}%
\pgfsetlinewidth{0.000000pt}%
\definecolor{currentstroke}{rgb}{0.000000,0.000000,0.000000}%
\pgfsetstrokecolor{currentstroke}%
\pgfsetstrokeopacity{0.700000}%
\pgfsetdash{}{0pt}%
\pgfpathmoveto{\pgfqpoint{2.317827in}{0.550000in}}%
\pgfpathlineto{\pgfqpoint{2.345348in}{0.550000in}}%
\pgfpathlineto{\pgfqpoint{2.345348in}{0.550000in}}%
\pgfpathlineto{\pgfqpoint{2.317827in}{0.550000in}}%
\pgfpathlineto{\pgfqpoint{2.317827in}{0.550000in}}%
\pgfpathclose%
\pgfusepath{fill}%
\end{pgfscope}%
\begin{pgfscope}%
\pgfpathrectangle{\pgfqpoint{1.250000in}{0.550000in}}{\pgfqpoint{7.750000in}{3.850000in}}%
\pgfusepath{clip}%
\pgfsetbuttcap%
\pgfsetmiterjoin%
\definecolor{currentfill}{rgb}{0.501961,0.501961,0.501961}%
\pgfsetfillcolor{currentfill}%
\pgfsetfillopacity{0.700000}%
\pgfsetlinewidth{0.000000pt}%
\definecolor{currentstroke}{rgb}{0.000000,0.000000,0.000000}%
\pgfsetstrokecolor{currentstroke}%
\pgfsetstrokeopacity{0.700000}%
\pgfsetdash{}{0pt}%
\pgfpathmoveto{\pgfqpoint{2.345348in}{0.550000in}}%
\pgfpathlineto{\pgfqpoint{2.372869in}{0.550000in}}%
\pgfpathlineto{\pgfqpoint{2.372869in}{0.550000in}}%
\pgfpathlineto{\pgfqpoint{2.345348in}{0.550000in}}%
\pgfpathlineto{\pgfqpoint{2.345348in}{0.550000in}}%
\pgfpathclose%
\pgfusepath{fill}%
\end{pgfscope}%
\begin{pgfscope}%
\pgfpathrectangle{\pgfqpoint{1.250000in}{0.550000in}}{\pgfqpoint{7.750000in}{3.850000in}}%
\pgfusepath{clip}%
\pgfsetbuttcap%
\pgfsetmiterjoin%
\definecolor{currentfill}{rgb}{0.501961,0.501961,0.501961}%
\pgfsetfillcolor{currentfill}%
\pgfsetfillopacity{0.700000}%
\pgfsetlinewidth{0.000000pt}%
\definecolor{currentstroke}{rgb}{0.000000,0.000000,0.000000}%
\pgfsetstrokecolor{currentstroke}%
\pgfsetstrokeopacity{0.700000}%
\pgfsetdash{}{0pt}%
\pgfpathmoveto{\pgfqpoint{2.372869in}{0.550000in}}%
\pgfpathlineto{\pgfqpoint{2.400391in}{0.550000in}}%
\pgfpathlineto{\pgfqpoint{2.400391in}{0.550000in}}%
\pgfpathlineto{\pgfqpoint{2.372869in}{0.550000in}}%
\pgfpathlineto{\pgfqpoint{2.372869in}{0.550000in}}%
\pgfpathclose%
\pgfusepath{fill}%
\end{pgfscope}%
\begin{pgfscope}%
\pgfpathrectangle{\pgfqpoint{1.250000in}{0.550000in}}{\pgfqpoint{7.750000in}{3.850000in}}%
\pgfusepath{clip}%
\pgfsetbuttcap%
\pgfsetmiterjoin%
\definecolor{currentfill}{rgb}{0.501961,0.501961,0.501961}%
\pgfsetfillcolor{currentfill}%
\pgfsetfillopacity{0.700000}%
\pgfsetlinewidth{0.000000pt}%
\definecolor{currentstroke}{rgb}{0.000000,0.000000,0.000000}%
\pgfsetstrokecolor{currentstroke}%
\pgfsetstrokeopacity{0.700000}%
\pgfsetdash{}{0pt}%
\pgfpathmoveto{\pgfqpoint{2.400391in}{0.550000in}}%
\pgfpathlineto{\pgfqpoint{2.427912in}{0.550000in}}%
\pgfpathlineto{\pgfqpoint{2.427912in}{0.550000in}}%
\pgfpathlineto{\pgfqpoint{2.400391in}{0.550000in}}%
\pgfpathlineto{\pgfqpoint{2.400391in}{0.550000in}}%
\pgfpathclose%
\pgfusepath{fill}%
\end{pgfscope}%
\begin{pgfscope}%
\pgfpathrectangle{\pgfqpoint{1.250000in}{0.550000in}}{\pgfqpoint{7.750000in}{3.850000in}}%
\pgfusepath{clip}%
\pgfsetbuttcap%
\pgfsetmiterjoin%
\definecolor{currentfill}{rgb}{0.501961,0.501961,0.501961}%
\pgfsetfillcolor{currentfill}%
\pgfsetfillopacity{0.700000}%
\pgfsetlinewidth{0.000000pt}%
\definecolor{currentstroke}{rgb}{0.000000,0.000000,0.000000}%
\pgfsetstrokecolor{currentstroke}%
\pgfsetstrokeopacity{0.700000}%
\pgfsetdash{}{0pt}%
\pgfpathmoveto{\pgfqpoint{2.427912in}{0.550000in}}%
\pgfpathlineto{\pgfqpoint{2.455433in}{0.550000in}}%
\pgfpathlineto{\pgfqpoint{2.455433in}{0.550000in}}%
\pgfpathlineto{\pgfqpoint{2.427912in}{0.550000in}}%
\pgfpathlineto{\pgfqpoint{2.427912in}{0.550000in}}%
\pgfpathclose%
\pgfusepath{fill}%
\end{pgfscope}%
\begin{pgfscope}%
\pgfpathrectangle{\pgfqpoint{1.250000in}{0.550000in}}{\pgfqpoint{7.750000in}{3.850000in}}%
\pgfusepath{clip}%
\pgfsetbuttcap%
\pgfsetmiterjoin%
\definecolor{currentfill}{rgb}{0.501961,0.501961,0.501961}%
\pgfsetfillcolor{currentfill}%
\pgfsetfillopacity{0.700000}%
\pgfsetlinewidth{0.000000pt}%
\definecolor{currentstroke}{rgb}{0.000000,0.000000,0.000000}%
\pgfsetstrokecolor{currentstroke}%
\pgfsetstrokeopacity{0.700000}%
\pgfsetdash{}{0pt}%
\pgfpathmoveto{\pgfqpoint{2.455433in}{0.550000in}}%
\pgfpathlineto{\pgfqpoint{2.482955in}{0.550000in}}%
\pgfpathlineto{\pgfqpoint{2.482955in}{2.356000in}}%
\pgfpathlineto{\pgfqpoint{2.455433in}{2.356000in}}%
\pgfpathlineto{\pgfqpoint{2.455433in}{0.550000in}}%
\pgfpathclose%
\pgfusepath{fill}%
\end{pgfscope}%
\begin{pgfscope}%
\pgfpathrectangle{\pgfqpoint{1.250000in}{0.550000in}}{\pgfqpoint{7.750000in}{3.850000in}}%
\pgfusepath{clip}%
\pgfsetbuttcap%
\pgfsetmiterjoin%
\definecolor{currentfill}{rgb}{0.501961,0.501961,0.501961}%
\pgfsetfillcolor{currentfill}%
\pgfsetfillopacity{0.700000}%
\pgfsetlinewidth{0.000000pt}%
\definecolor{currentstroke}{rgb}{0.000000,0.000000,0.000000}%
\pgfsetstrokecolor{currentstroke}%
\pgfsetstrokeopacity{0.700000}%
\pgfsetdash{}{0pt}%
\pgfpathmoveto{\pgfqpoint{2.482955in}{0.550000in}}%
\pgfpathlineto{\pgfqpoint{2.510476in}{0.550000in}}%
\pgfpathlineto{\pgfqpoint{2.510476in}{0.550000in}}%
\pgfpathlineto{\pgfqpoint{2.482955in}{0.550000in}}%
\pgfpathlineto{\pgfqpoint{2.482955in}{0.550000in}}%
\pgfpathclose%
\pgfusepath{fill}%
\end{pgfscope}%
\begin{pgfscope}%
\pgfpathrectangle{\pgfqpoint{1.250000in}{0.550000in}}{\pgfqpoint{7.750000in}{3.850000in}}%
\pgfusepath{clip}%
\pgfsetbuttcap%
\pgfsetmiterjoin%
\definecolor{currentfill}{rgb}{0.501961,0.501961,0.501961}%
\pgfsetfillcolor{currentfill}%
\pgfsetfillopacity{0.700000}%
\pgfsetlinewidth{0.000000pt}%
\definecolor{currentstroke}{rgb}{0.000000,0.000000,0.000000}%
\pgfsetstrokecolor{currentstroke}%
\pgfsetstrokeopacity{0.700000}%
\pgfsetdash{}{0pt}%
\pgfpathmoveto{\pgfqpoint{2.510476in}{0.550000in}}%
\pgfpathlineto{\pgfqpoint{2.537997in}{0.550000in}}%
\pgfpathlineto{\pgfqpoint{2.537997in}{0.550000in}}%
\pgfpathlineto{\pgfqpoint{2.510476in}{0.550000in}}%
\pgfpathlineto{\pgfqpoint{2.510476in}{0.550000in}}%
\pgfpathclose%
\pgfusepath{fill}%
\end{pgfscope}%
\begin{pgfscope}%
\pgfpathrectangle{\pgfqpoint{1.250000in}{0.550000in}}{\pgfqpoint{7.750000in}{3.850000in}}%
\pgfusepath{clip}%
\pgfsetbuttcap%
\pgfsetmiterjoin%
\definecolor{currentfill}{rgb}{0.501961,0.501961,0.501961}%
\pgfsetfillcolor{currentfill}%
\pgfsetfillopacity{0.700000}%
\pgfsetlinewidth{0.000000pt}%
\definecolor{currentstroke}{rgb}{0.000000,0.000000,0.000000}%
\pgfsetstrokecolor{currentstroke}%
\pgfsetstrokeopacity{0.700000}%
\pgfsetdash{}{0pt}%
\pgfpathmoveto{\pgfqpoint{2.537997in}{0.550000in}}%
\pgfpathlineto{\pgfqpoint{2.565518in}{0.550000in}}%
\pgfpathlineto{\pgfqpoint{2.565518in}{0.550000in}}%
\pgfpathlineto{\pgfqpoint{2.537997in}{0.550000in}}%
\pgfpathlineto{\pgfqpoint{2.537997in}{0.550000in}}%
\pgfpathclose%
\pgfusepath{fill}%
\end{pgfscope}%
\begin{pgfscope}%
\pgfpathrectangle{\pgfqpoint{1.250000in}{0.550000in}}{\pgfqpoint{7.750000in}{3.850000in}}%
\pgfusepath{clip}%
\pgfsetbuttcap%
\pgfsetmiterjoin%
\definecolor{currentfill}{rgb}{0.501961,0.501961,0.501961}%
\pgfsetfillcolor{currentfill}%
\pgfsetfillopacity{0.700000}%
\pgfsetlinewidth{0.000000pt}%
\definecolor{currentstroke}{rgb}{0.000000,0.000000,0.000000}%
\pgfsetstrokecolor{currentstroke}%
\pgfsetstrokeopacity{0.700000}%
\pgfsetdash{}{0pt}%
\pgfpathmoveto{\pgfqpoint{2.565518in}{0.550000in}}%
\pgfpathlineto{\pgfqpoint{2.593040in}{0.550000in}}%
\pgfpathlineto{\pgfqpoint{2.593040in}{0.550000in}}%
\pgfpathlineto{\pgfqpoint{2.565518in}{0.550000in}}%
\pgfpathlineto{\pgfqpoint{2.565518in}{0.550000in}}%
\pgfpathclose%
\pgfusepath{fill}%
\end{pgfscope}%
\begin{pgfscope}%
\pgfpathrectangle{\pgfqpoint{1.250000in}{0.550000in}}{\pgfqpoint{7.750000in}{3.850000in}}%
\pgfusepath{clip}%
\pgfsetbuttcap%
\pgfsetmiterjoin%
\definecolor{currentfill}{rgb}{0.501961,0.501961,0.501961}%
\pgfsetfillcolor{currentfill}%
\pgfsetfillopacity{0.700000}%
\pgfsetlinewidth{0.000000pt}%
\definecolor{currentstroke}{rgb}{0.000000,0.000000,0.000000}%
\pgfsetstrokecolor{currentstroke}%
\pgfsetstrokeopacity{0.700000}%
\pgfsetdash{}{0pt}%
\pgfpathmoveto{\pgfqpoint{2.593040in}{0.550000in}}%
\pgfpathlineto{\pgfqpoint{2.620561in}{0.550000in}}%
\pgfpathlineto{\pgfqpoint{2.620561in}{0.550000in}}%
\pgfpathlineto{\pgfqpoint{2.593040in}{0.550000in}}%
\pgfpathlineto{\pgfqpoint{2.593040in}{0.550000in}}%
\pgfpathclose%
\pgfusepath{fill}%
\end{pgfscope}%
\begin{pgfscope}%
\pgfpathrectangle{\pgfqpoint{1.250000in}{0.550000in}}{\pgfqpoint{7.750000in}{3.850000in}}%
\pgfusepath{clip}%
\pgfsetbuttcap%
\pgfsetmiterjoin%
\definecolor{currentfill}{rgb}{0.501961,0.501961,0.501961}%
\pgfsetfillcolor{currentfill}%
\pgfsetfillopacity{0.700000}%
\pgfsetlinewidth{0.000000pt}%
\definecolor{currentstroke}{rgb}{0.000000,0.000000,0.000000}%
\pgfsetstrokecolor{currentstroke}%
\pgfsetstrokeopacity{0.700000}%
\pgfsetdash{}{0pt}%
\pgfpathmoveto{\pgfqpoint{2.620561in}{0.550000in}}%
\pgfpathlineto{\pgfqpoint{2.648082in}{0.550000in}}%
\pgfpathlineto{\pgfqpoint{2.648082in}{0.550000in}}%
\pgfpathlineto{\pgfqpoint{2.620561in}{0.550000in}}%
\pgfpathlineto{\pgfqpoint{2.620561in}{0.550000in}}%
\pgfpathclose%
\pgfusepath{fill}%
\end{pgfscope}%
\begin{pgfscope}%
\pgfpathrectangle{\pgfqpoint{1.250000in}{0.550000in}}{\pgfqpoint{7.750000in}{3.850000in}}%
\pgfusepath{clip}%
\pgfsetbuttcap%
\pgfsetmiterjoin%
\definecolor{currentfill}{rgb}{0.501961,0.501961,0.501961}%
\pgfsetfillcolor{currentfill}%
\pgfsetfillopacity{0.700000}%
\pgfsetlinewidth{0.000000pt}%
\definecolor{currentstroke}{rgb}{0.000000,0.000000,0.000000}%
\pgfsetstrokecolor{currentstroke}%
\pgfsetstrokeopacity{0.700000}%
\pgfsetdash{}{0pt}%
\pgfpathmoveto{\pgfqpoint{2.648082in}{0.550000in}}%
\pgfpathlineto{\pgfqpoint{2.675604in}{0.550000in}}%
\pgfpathlineto{\pgfqpoint{2.675604in}{0.550000in}}%
\pgfpathlineto{\pgfqpoint{2.648082in}{0.550000in}}%
\pgfpathlineto{\pgfqpoint{2.648082in}{0.550000in}}%
\pgfpathclose%
\pgfusepath{fill}%
\end{pgfscope}%
\begin{pgfscope}%
\pgfpathrectangle{\pgfqpoint{1.250000in}{0.550000in}}{\pgfqpoint{7.750000in}{3.850000in}}%
\pgfusepath{clip}%
\pgfsetbuttcap%
\pgfsetmiterjoin%
\definecolor{currentfill}{rgb}{0.501961,0.501961,0.501961}%
\pgfsetfillcolor{currentfill}%
\pgfsetfillopacity{0.700000}%
\pgfsetlinewidth{0.000000pt}%
\definecolor{currentstroke}{rgb}{0.000000,0.000000,0.000000}%
\pgfsetstrokecolor{currentstroke}%
\pgfsetstrokeopacity{0.700000}%
\pgfsetdash{}{0pt}%
\pgfpathmoveto{\pgfqpoint{2.675604in}{0.550000in}}%
\pgfpathlineto{\pgfqpoint{2.703125in}{0.550000in}}%
\pgfpathlineto{\pgfqpoint{2.703125in}{0.550000in}}%
\pgfpathlineto{\pgfqpoint{2.675604in}{0.550000in}}%
\pgfpathlineto{\pgfqpoint{2.675604in}{0.550000in}}%
\pgfpathclose%
\pgfusepath{fill}%
\end{pgfscope}%
\begin{pgfscope}%
\pgfpathrectangle{\pgfqpoint{1.250000in}{0.550000in}}{\pgfqpoint{7.750000in}{3.850000in}}%
\pgfusepath{clip}%
\pgfsetbuttcap%
\pgfsetmiterjoin%
\definecolor{currentfill}{rgb}{0.501961,0.501961,0.501961}%
\pgfsetfillcolor{currentfill}%
\pgfsetfillopacity{0.700000}%
\pgfsetlinewidth{0.000000pt}%
\definecolor{currentstroke}{rgb}{0.000000,0.000000,0.000000}%
\pgfsetstrokecolor{currentstroke}%
\pgfsetstrokeopacity{0.700000}%
\pgfsetdash{}{0pt}%
\pgfpathmoveto{\pgfqpoint{2.703125in}{0.550000in}}%
\pgfpathlineto{\pgfqpoint{2.730646in}{0.550000in}}%
\pgfpathlineto{\pgfqpoint{2.730646in}{2.100507in}}%
\pgfpathlineto{\pgfqpoint{2.703125in}{2.100507in}}%
\pgfpathlineto{\pgfqpoint{2.703125in}{0.550000in}}%
\pgfpathclose%
\pgfusepath{fill}%
\end{pgfscope}%
\begin{pgfscope}%
\pgfpathrectangle{\pgfqpoint{1.250000in}{0.550000in}}{\pgfqpoint{7.750000in}{3.850000in}}%
\pgfusepath{clip}%
\pgfsetbuttcap%
\pgfsetmiterjoin%
\definecolor{currentfill}{rgb}{0.501961,0.501961,0.501961}%
\pgfsetfillcolor{currentfill}%
\pgfsetfillopacity{0.700000}%
\pgfsetlinewidth{0.000000pt}%
\definecolor{currentstroke}{rgb}{0.000000,0.000000,0.000000}%
\pgfsetstrokecolor{currentstroke}%
\pgfsetstrokeopacity{0.700000}%
\pgfsetdash{}{0pt}%
\pgfpathmoveto{\pgfqpoint{2.730646in}{0.550000in}}%
\pgfpathlineto{\pgfqpoint{2.758168in}{0.550000in}}%
\pgfpathlineto{\pgfqpoint{2.758168in}{0.550000in}}%
\pgfpathlineto{\pgfqpoint{2.730646in}{0.550000in}}%
\pgfpathlineto{\pgfqpoint{2.730646in}{0.550000in}}%
\pgfpathclose%
\pgfusepath{fill}%
\end{pgfscope}%
\begin{pgfscope}%
\pgfpathrectangle{\pgfqpoint{1.250000in}{0.550000in}}{\pgfqpoint{7.750000in}{3.850000in}}%
\pgfusepath{clip}%
\pgfsetbuttcap%
\pgfsetmiterjoin%
\definecolor{currentfill}{rgb}{0.501961,0.501961,0.501961}%
\pgfsetfillcolor{currentfill}%
\pgfsetfillopacity{0.700000}%
\pgfsetlinewidth{0.000000pt}%
\definecolor{currentstroke}{rgb}{0.000000,0.000000,0.000000}%
\pgfsetstrokecolor{currentstroke}%
\pgfsetstrokeopacity{0.700000}%
\pgfsetdash{}{0pt}%
\pgfpathmoveto{\pgfqpoint{2.758168in}{0.550000in}}%
\pgfpathlineto{\pgfqpoint{2.785689in}{0.550000in}}%
\pgfpathlineto{\pgfqpoint{2.785689in}{0.550000in}}%
\pgfpathlineto{\pgfqpoint{2.758168in}{0.550000in}}%
\pgfpathlineto{\pgfqpoint{2.758168in}{0.550000in}}%
\pgfpathclose%
\pgfusepath{fill}%
\end{pgfscope}%
\begin{pgfscope}%
\pgfpathrectangle{\pgfqpoint{1.250000in}{0.550000in}}{\pgfqpoint{7.750000in}{3.850000in}}%
\pgfusepath{clip}%
\pgfsetbuttcap%
\pgfsetmiterjoin%
\definecolor{currentfill}{rgb}{0.501961,0.501961,0.501961}%
\pgfsetfillcolor{currentfill}%
\pgfsetfillopacity{0.700000}%
\pgfsetlinewidth{0.000000pt}%
\definecolor{currentstroke}{rgb}{0.000000,0.000000,0.000000}%
\pgfsetstrokecolor{currentstroke}%
\pgfsetstrokeopacity{0.700000}%
\pgfsetdash{}{0pt}%
\pgfpathmoveto{\pgfqpoint{2.785689in}{0.550000in}}%
\pgfpathlineto{\pgfqpoint{2.813210in}{0.550000in}}%
\pgfpathlineto{\pgfqpoint{2.813210in}{0.550000in}}%
\pgfpathlineto{\pgfqpoint{2.785689in}{0.550000in}}%
\pgfpathlineto{\pgfqpoint{2.785689in}{0.550000in}}%
\pgfpathclose%
\pgfusepath{fill}%
\end{pgfscope}%
\begin{pgfscope}%
\pgfpathrectangle{\pgfqpoint{1.250000in}{0.550000in}}{\pgfqpoint{7.750000in}{3.850000in}}%
\pgfusepath{clip}%
\pgfsetbuttcap%
\pgfsetmiterjoin%
\definecolor{currentfill}{rgb}{0.501961,0.501961,0.501961}%
\pgfsetfillcolor{currentfill}%
\pgfsetfillopacity{0.700000}%
\pgfsetlinewidth{0.000000pt}%
\definecolor{currentstroke}{rgb}{0.000000,0.000000,0.000000}%
\pgfsetstrokecolor{currentstroke}%
\pgfsetstrokeopacity{0.700000}%
\pgfsetdash{}{0pt}%
\pgfpathmoveto{\pgfqpoint{2.813210in}{0.550000in}}%
\pgfpathlineto{\pgfqpoint{2.840732in}{0.550000in}}%
\pgfpathlineto{\pgfqpoint{2.840732in}{0.550000in}}%
\pgfpathlineto{\pgfqpoint{2.813210in}{0.550000in}}%
\pgfpathlineto{\pgfqpoint{2.813210in}{0.550000in}}%
\pgfpathclose%
\pgfusepath{fill}%
\end{pgfscope}%
\begin{pgfscope}%
\pgfpathrectangle{\pgfqpoint{1.250000in}{0.550000in}}{\pgfqpoint{7.750000in}{3.850000in}}%
\pgfusepath{clip}%
\pgfsetbuttcap%
\pgfsetmiterjoin%
\definecolor{currentfill}{rgb}{0.501961,0.501961,0.501961}%
\pgfsetfillcolor{currentfill}%
\pgfsetfillopacity{0.700000}%
\pgfsetlinewidth{0.000000pt}%
\definecolor{currentstroke}{rgb}{0.000000,0.000000,0.000000}%
\pgfsetstrokecolor{currentstroke}%
\pgfsetstrokeopacity{0.700000}%
\pgfsetdash{}{0pt}%
\pgfpathmoveto{\pgfqpoint{2.840732in}{0.550000in}}%
\pgfpathlineto{\pgfqpoint{2.868253in}{0.550000in}}%
\pgfpathlineto{\pgfqpoint{2.868253in}{0.550000in}}%
\pgfpathlineto{\pgfqpoint{2.840732in}{0.550000in}}%
\pgfpathlineto{\pgfqpoint{2.840732in}{0.550000in}}%
\pgfpathclose%
\pgfusepath{fill}%
\end{pgfscope}%
\begin{pgfscope}%
\pgfpathrectangle{\pgfqpoint{1.250000in}{0.550000in}}{\pgfqpoint{7.750000in}{3.850000in}}%
\pgfusepath{clip}%
\pgfsetbuttcap%
\pgfsetmiterjoin%
\definecolor{currentfill}{rgb}{0.501961,0.501961,0.501961}%
\pgfsetfillcolor{currentfill}%
\pgfsetfillopacity{0.700000}%
\pgfsetlinewidth{0.000000pt}%
\definecolor{currentstroke}{rgb}{0.000000,0.000000,0.000000}%
\pgfsetstrokecolor{currentstroke}%
\pgfsetstrokeopacity{0.700000}%
\pgfsetdash{}{0pt}%
\pgfpathmoveto{\pgfqpoint{2.868253in}{0.550000in}}%
\pgfpathlineto{\pgfqpoint{2.895774in}{0.550000in}}%
\pgfpathlineto{\pgfqpoint{2.895774in}{0.550000in}}%
\pgfpathlineto{\pgfqpoint{2.868253in}{0.550000in}}%
\pgfpathlineto{\pgfqpoint{2.868253in}{0.550000in}}%
\pgfpathclose%
\pgfusepath{fill}%
\end{pgfscope}%
\begin{pgfscope}%
\pgfpathrectangle{\pgfqpoint{1.250000in}{0.550000in}}{\pgfqpoint{7.750000in}{3.850000in}}%
\pgfusepath{clip}%
\pgfsetbuttcap%
\pgfsetmiterjoin%
\definecolor{currentfill}{rgb}{0.501961,0.501961,0.501961}%
\pgfsetfillcolor{currentfill}%
\pgfsetfillopacity{0.700000}%
\pgfsetlinewidth{0.000000pt}%
\definecolor{currentstroke}{rgb}{0.000000,0.000000,0.000000}%
\pgfsetstrokecolor{currentstroke}%
\pgfsetstrokeopacity{0.700000}%
\pgfsetdash{}{0pt}%
\pgfpathmoveto{\pgfqpoint{2.895774in}{0.550000in}}%
\pgfpathlineto{\pgfqpoint{2.923295in}{0.550000in}}%
\pgfpathlineto{\pgfqpoint{2.923295in}{0.550000in}}%
\pgfpathlineto{\pgfqpoint{2.895774in}{0.550000in}}%
\pgfpathlineto{\pgfqpoint{2.895774in}{0.550000in}}%
\pgfpathclose%
\pgfusepath{fill}%
\end{pgfscope}%
\begin{pgfscope}%
\pgfpathrectangle{\pgfqpoint{1.250000in}{0.550000in}}{\pgfqpoint{7.750000in}{3.850000in}}%
\pgfusepath{clip}%
\pgfsetbuttcap%
\pgfsetmiterjoin%
\definecolor{currentfill}{rgb}{0.501961,0.501961,0.501961}%
\pgfsetfillcolor{currentfill}%
\pgfsetfillopacity{0.700000}%
\pgfsetlinewidth{0.000000pt}%
\definecolor{currentstroke}{rgb}{0.000000,0.000000,0.000000}%
\pgfsetstrokecolor{currentstroke}%
\pgfsetstrokeopacity{0.700000}%
\pgfsetdash{}{0pt}%
\pgfpathmoveto{\pgfqpoint{2.923295in}{0.550000in}}%
\pgfpathlineto{\pgfqpoint{2.950817in}{0.550000in}}%
\pgfpathlineto{\pgfqpoint{2.950817in}{0.550000in}}%
\pgfpathlineto{\pgfqpoint{2.923295in}{0.550000in}}%
\pgfpathlineto{\pgfqpoint{2.923295in}{0.550000in}}%
\pgfpathclose%
\pgfusepath{fill}%
\end{pgfscope}%
\begin{pgfscope}%
\pgfpathrectangle{\pgfqpoint{1.250000in}{0.550000in}}{\pgfqpoint{7.750000in}{3.850000in}}%
\pgfusepath{clip}%
\pgfsetbuttcap%
\pgfsetmiterjoin%
\definecolor{currentfill}{rgb}{0.501961,0.501961,0.501961}%
\pgfsetfillcolor{currentfill}%
\pgfsetfillopacity{0.700000}%
\pgfsetlinewidth{0.000000pt}%
\definecolor{currentstroke}{rgb}{0.000000,0.000000,0.000000}%
\pgfsetstrokecolor{currentstroke}%
\pgfsetstrokeopacity{0.700000}%
\pgfsetdash{}{0pt}%
\pgfpathmoveto{\pgfqpoint{2.950817in}{0.550000in}}%
\pgfpathlineto{\pgfqpoint{2.978338in}{0.550000in}}%
\pgfpathlineto{\pgfqpoint{2.978338in}{1.845448in}}%
\pgfpathlineto{\pgfqpoint{2.950817in}{1.845448in}}%
\pgfpathlineto{\pgfqpoint{2.950817in}{0.550000in}}%
\pgfpathclose%
\pgfusepath{fill}%
\end{pgfscope}%
\begin{pgfscope}%
\pgfpathrectangle{\pgfqpoint{1.250000in}{0.550000in}}{\pgfqpoint{7.750000in}{3.850000in}}%
\pgfusepath{clip}%
\pgfsetbuttcap%
\pgfsetmiterjoin%
\definecolor{currentfill}{rgb}{0.501961,0.501961,0.501961}%
\pgfsetfillcolor{currentfill}%
\pgfsetfillopacity{0.700000}%
\pgfsetlinewidth{0.000000pt}%
\definecolor{currentstroke}{rgb}{0.000000,0.000000,0.000000}%
\pgfsetstrokecolor{currentstroke}%
\pgfsetstrokeopacity{0.700000}%
\pgfsetdash{}{0pt}%
\pgfpathmoveto{\pgfqpoint{2.978338in}{0.550000in}}%
\pgfpathlineto{\pgfqpoint{3.005859in}{0.550000in}}%
\pgfpathlineto{\pgfqpoint{3.005859in}{0.550000in}}%
\pgfpathlineto{\pgfqpoint{2.978338in}{0.550000in}}%
\pgfpathlineto{\pgfqpoint{2.978338in}{0.550000in}}%
\pgfpathclose%
\pgfusepath{fill}%
\end{pgfscope}%
\begin{pgfscope}%
\pgfpathrectangle{\pgfqpoint{1.250000in}{0.550000in}}{\pgfqpoint{7.750000in}{3.850000in}}%
\pgfusepath{clip}%
\pgfsetbuttcap%
\pgfsetmiterjoin%
\definecolor{currentfill}{rgb}{0.501961,0.501961,0.501961}%
\pgfsetfillcolor{currentfill}%
\pgfsetfillopacity{0.700000}%
\pgfsetlinewidth{0.000000pt}%
\definecolor{currentstroke}{rgb}{0.000000,0.000000,0.000000}%
\pgfsetstrokecolor{currentstroke}%
\pgfsetstrokeopacity{0.700000}%
\pgfsetdash{}{0pt}%
\pgfpathmoveto{\pgfqpoint{3.005859in}{0.550000in}}%
\pgfpathlineto{\pgfqpoint{3.033381in}{0.550000in}}%
\pgfpathlineto{\pgfqpoint{3.033381in}{0.550000in}}%
\pgfpathlineto{\pgfqpoint{3.005859in}{0.550000in}}%
\pgfpathlineto{\pgfqpoint{3.005859in}{0.550000in}}%
\pgfpathclose%
\pgfusepath{fill}%
\end{pgfscope}%
\begin{pgfscope}%
\pgfpathrectangle{\pgfqpoint{1.250000in}{0.550000in}}{\pgfqpoint{7.750000in}{3.850000in}}%
\pgfusepath{clip}%
\pgfsetbuttcap%
\pgfsetmiterjoin%
\definecolor{currentfill}{rgb}{0.501961,0.501961,0.501961}%
\pgfsetfillcolor{currentfill}%
\pgfsetfillopacity{0.700000}%
\pgfsetlinewidth{0.000000pt}%
\definecolor{currentstroke}{rgb}{0.000000,0.000000,0.000000}%
\pgfsetstrokecolor{currentstroke}%
\pgfsetstrokeopacity{0.700000}%
\pgfsetdash{}{0pt}%
\pgfpathmoveto{\pgfqpoint{3.033381in}{0.550000in}}%
\pgfpathlineto{\pgfqpoint{3.060902in}{0.550000in}}%
\pgfpathlineto{\pgfqpoint{3.060902in}{0.550000in}}%
\pgfpathlineto{\pgfqpoint{3.033381in}{0.550000in}}%
\pgfpathlineto{\pgfqpoint{3.033381in}{0.550000in}}%
\pgfpathclose%
\pgfusepath{fill}%
\end{pgfscope}%
\begin{pgfscope}%
\pgfpathrectangle{\pgfqpoint{1.250000in}{0.550000in}}{\pgfqpoint{7.750000in}{3.850000in}}%
\pgfusepath{clip}%
\pgfsetbuttcap%
\pgfsetmiterjoin%
\definecolor{currentfill}{rgb}{0.501961,0.501961,0.501961}%
\pgfsetfillcolor{currentfill}%
\pgfsetfillopacity{0.700000}%
\pgfsetlinewidth{0.000000pt}%
\definecolor{currentstroke}{rgb}{0.000000,0.000000,0.000000}%
\pgfsetstrokecolor{currentstroke}%
\pgfsetstrokeopacity{0.700000}%
\pgfsetdash{}{0pt}%
\pgfpathmoveto{\pgfqpoint{3.060902in}{0.550000in}}%
\pgfpathlineto{\pgfqpoint{3.088423in}{0.550000in}}%
\pgfpathlineto{\pgfqpoint{3.088423in}{0.550000in}}%
\pgfpathlineto{\pgfqpoint{3.060902in}{0.550000in}}%
\pgfpathlineto{\pgfqpoint{3.060902in}{0.550000in}}%
\pgfpathclose%
\pgfusepath{fill}%
\end{pgfscope}%
\begin{pgfscope}%
\pgfpathrectangle{\pgfqpoint{1.250000in}{0.550000in}}{\pgfqpoint{7.750000in}{3.850000in}}%
\pgfusepath{clip}%
\pgfsetbuttcap%
\pgfsetmiterjoin%
\definecolor{currentfill}{rgb}{0.501961,0.501961,0.501961}%
\pgfsetfillcolor{currentfill}%
\pgfsetfillopacity{0.700000}%
\pgfsetlinewidth{0.000000pt}%
\definecolor{currentstroke}{rgb}{0.000000,0.000000,0.000000}%
\pgfsetstrokecolor{currentstroke}%
\pgfsetstrokeopacity{0.700000}%
\pgfsetdash{}{0pt}%
\pgfpathmoveto{\pgfqpoint{3.088423in}{0.550000in}}%
\pgfpathlineto{\pgfqpoint{3.115945in}{0.550000in}}%
\pgfpathlineto{\pgfqpoint{3.115945in}{0.550000in}}%
\pgfpathlineto{\pgfqpoint{3.088423in}{0.550000in}}%
\pgfpathlineto{\pgfqpoint{3.088423in}{0.550000in}}%
\pgfpathclose%
\pgfusepath{fill}%
\end{pgfscope}%
\begin{pgfscope}%
\pgfpathrectangle{\pgfqpoint{1.250000in}{0.550000in}}{\pgfqpoint{7.750000in}{3.850000in}}%
\pgfusepath{clip}%
\pgfsetbuttcap%
\pgfsetmiterjoin%
\definecolor{currentfill}{rgb}{0.501961,0.501961,0.501961}%
\pgfsetfillcolor{currentfill}%
\pgfsetfillopacity{0.700000}%
\pgfsetlinewidth{0.000000pt}%
\definecolor{currentstroke}{rgb}{0.000000,0.000000,0.000000}%
\pgfsetstrokecolor{currentstroke}%
\pgfsetstrokeopacity{0.700000}%
\pgfsetdash{}{0pt}%
\pgfpathmoveto{\pgfqpoint{3.115945in}{0.550000in}}%
\pgfpathlineto{\pgfqpoint{3.143466in}{0.550000in}}%
\pgfpathlineto{\pgfqpoint{3.143466in}{1.619100in}}%
\pgfpathlineto{\pgfqpoint{3.115945in}{1.619100in}}%
\pgfpathlineto{\pgfqpoint{3.115945in}{0.550000in}}%
\pgfpathclose%
\pgfusepath{fill}%
\end{pgfscope}%
\begin{pgfscope}%
\pgfpathrectangle{\pgfqpoint{1.250000in}{0.550000in}}{\pgfqpoint{7.750000in}{3.850000in}}%
\pgfusepath{clip}%
\pgfsetbuttcap%
\pgfsetmiterjoin%
\definecolor{currentfill}{rgb}{0.501961,0.501961,0.501961}%
\pgfsetfillcolor{currentfill}%
\pgfsetfillopacity{0.700000}%
\pgfsetlinewidth{0.000000pt}%
\definecolor{currentstroke}{rgb}{0.000000,0.000000,0.000000}%
\pgfsetstrokecolor{currentstroke}%
\pgfsetstrokeopacity{0.700000}%
\pgfsetdash{}{0pt}%
\pgfpathmoveto{\pgfqpoint{3.143466in}{0.550000in}}%
\pgfpathlineto{\pgfqpoint{3.170987in}{0.550000in}}%
\pgfpathlineto{\pgfqpoint{3.170987in}{0.550000in}}%
\pgfpathlineto{\pgfqpoint{3.143466in}{0.550000in}}%
\pgfpathlineto{\pgfqpoint{3.143466in}{0.550000in}}%
\pgfpathclose%
\pgfusepath{fill}%
\end{pgfscope}%
\begin{pgfscope}%
\pgfpathrectangle{\pgfqpoint{1.250000in}{0.550000in}}{\pgfqpoint{7.750000in}{3.850000in}}%
\pgfusepath{clip}%
\pgfsetbuttcap%
\pgfsetmiterjoin%
\definecolor{currentfill}{rgb}{0.501961,0.501961,0.501961}%
\pgfsetfillcolor{currentfill}%
\pgfsetfillopacity{0.700000}%
\pgfsetlinewidth{0.000000pt}%
\definecolor{currentstroke}{rgb}{0.000000,0.000000,0.000000}%
\pgfsetstrokecolor{currentstroke}%
\pgfsetstrokeopacity{0.700000}%
\pgfsetdash{}{0pt}%
\pgfpathmoveto{\pgfqpoint{3.170987in}{0.550000in}}%
\pgfpathlineto{\pgfqpoint{3.198509in}{0.550000in}}%
\pgfpathlineto{\pgfqpoint{3.198509in}{0.550000in}}%
\pgfpathlineto{\pgfqpoint{3.170987in}{0.550000in}}%
\pgfpathlineto{\pgfqpoint{3.170987in}{0.550000in}}%
\pgfpathclose%
\pgfusepath{fill}%
\end{pgfscope}%
\begin{pgfscope}%
\pgfpathrectangle{\pgfqpoint{1.250000in}{0.550000in}}{\pgfqpoint{7.750000in}{3.850000in}}%
\pgfusepath{clip}%
\pgfsetbuttcap%
\pgfsetmiterjoin%
\definecolor{currentfill}{rgb}{0.501961,0.501961,0.501961}%
\pgfsetfillcolor{currentfill}%
\pgfsetfillopacity{0.700000}%
\pgfsetlinewidth{0.000000pt}%
\definecolor{currentstroke}{rgb}{0.000000,0.000000,0.000000}%
\pgfsetstrokecolor{currentstroke}%
\pgfsetstrokeopacity{0.700000}%
\pgfsetdash{}{0pt}%
\pgfpathmoveto{\pgfqpoint{3.198509in}{0.550000in}}%
\pgfpathlineto{\pgfqpoint{3.226030in}{0.550000in}}%
\pgfpathlineto{\pgfqpoint{3.226030in}{0.550000in}}%
\pgfpathlineto{\pgfqpoint{3.198509in}{0.550000in}}%
\pgfpathlineto{\pgfqpoint{3.198509in}{0.550000in}}%
\pgfpathclose%
\pgfusepath{fill}%
\end{pgfscope}%
\begin{pgfscope}%
\pgfpathrectangle{\pgfqpoint{1.250000in}{0.550000in}}{\pgfqpoint{7.750000in}{3.850000in}}%
\pgfusepath{clip}%
\pgfsetbuttcap%
\pgfsetmiterjoin%
\definecolor{currentfill}{rgb}{0.501961,0.501961,0.501961}%
\pgfsetfillcolor{currentfill}%
\pgfsetfillopacity{0.700000}%
\pgfsetlinewidth{0.000000pt}%
\definecolor{currentstroke}{rgb}{0.000000,0.000000,0.000000}%
\pgfsetstrokecolor{currentstroke}%
\pgfsetstrokeopacity{0.700000}%
\pgfsetdash{}{0pt}%
\pgfpathmoveto{\pgfqpoint{3.226030in}{0.550000in}}%
\pgfpathlineto{\pgfqpoint{3.253551in}{0.550000in}}%
\pgfpathlineto{\pgfqpoint{3.253551in}{0.550000in}}%
\pgfpathlineto{\pgfqpoint{3.226030in}{0.550000in}}%
\pgfpathlineto{\pgfqpoint{3.226030in}{0.550000in}}%
\pgfpathclose%
\pgfusepath{fill}%
\end{pgfscope}%
\begin{pgfscope}%
\pgfpathrectangle{\pgfqpoint{1.250000in}{0.550000in}}{\pgfqpoint{7.750000in}{3.850000in}}%
\pgfusepath{clip}%
\pgfsetbuttcap%
\pgfsetmiterjoin%
\definecolor{currentfill}{rgb}{0.501961,0.501961,0.501961}%
\pgfsetfillcolor{currentfill}%
\pgfsetfillopacity{0.700000}%
\pgfsetlinewidth{0.000000pt}%
\definecolor{currentstroke}{rgb}{0.000000,0.000000,0.000000}%
\pgfsetstrokecolor{currentstroke}%
\pgfsetstrokeopacity{0.700000}%
\pgfsetdash{}{0pt}%
\pgfpathmoveto{\pgfqpoint{3.253551in}{0.550000in}}%
\pgfpathlineto{\pgfqpoint{3.281072in}{0.550000in}}%
\pgfpathlineto{\pgfqpoint{3.281072in}{0.550000in}}%
\pgfpathlineto{\pgfqpoint{3.253551in}{0.550000in}}%
\pgfpathlineto{\pgfqpoint{3.253551in}{0.550000in}}%
\pgfpathclose%
\pgfusepath{fill}%
\end{pgfscope}%
\begin{pgfscope}%
\pgfpathrectangle{\pgfqpoint{1.250000in}{0.550000in}}{\pgfqpoint{7.750000in}{3.850000in}}%
\pgfusepath{clip}%
\pgfsetbuttcap%
\pgfsetmiterjoin%
\definecolor{currentfill}{rgb}{0.501961,0.501961,0.501961}%
\pgfsetfillcolor{currentfill}%
\pgfsetfillopacity{0.700000}%
\pgfsetlinewidth{0.000000pt}%
\definecolor{currentstroke}{rgb}{0.000000,0.000000,0.000000}%
\pgfsetstrokecolor{currentstroke}%
\pgfsetstrokeopacity{0.700000}%
\pgfsetdash{}{0pt}%
\pgfpathmoveto{\pgfqpoint{3.281072in}{0.550000in}}%
\pgfpathlineto{\pgfqpoint{3.308594in}{0.550000in}}%
\pgfpathlineto{\pgfqpoint{3.308594in}{1.464963in}}%
\pgfpathlineto{\pgfqpoint{3.281072in}{1.464963in}}%
\pgfpathlineto{\pgfqpoint{3.281072in}{0.550000in}}%
\pgfpathclose%
\pgfusepath{fill}%
\end{pgfscope}%
\begin{pgfscope}%
\pgfpathrectangle{\pgfqpoint{1.250000in}{0.550000in}}{\pgfqpoint{7.750000in}{3.850000in}}%
\pgfusepath{clip}%
\pgfsetbuttcap%
\pgfsetmiterjoin%
\definecolor{currentfill}{rgb}{0.501961,0.501961,0.501961}%
\pgfsetfillcolor{currentfill}%
\pgfsetfillopacity{0.700000}%
\pgfsetlinewidth{0.000000pt}%
\definecolor{currentstroke}{rgb}{0.000000,0.000000,0.000000}%
\pgfsetstrokecolor{currentstroke}%
\pgfsetstrokeopacity{0.700000}%
\pgfsetdash{}{0pt}%
\pgfpathmoveto{\pgfqpoint{3.308594in}{0.550000in}}%
\pgfpathlineto{\pgfqpoint{3.336115in}{0.550000in}}%
\pgfpathlineto{\pgfqpoint{3.336115in}{0.550000in}}%
\pgfpathlineto{\pgfqpoint{3.308594in}{0.550000in}}%
\pgfpathlineto{\pgfqpoint{3.308594in}{0.550000in}}%
\pgfpathclose%
\pgfusepath{fill}%
\end{pgfscope}%
\begin{pgfscope}%
\pgfpathrectangle{\pgfqpoint{1.250000in}{0.550000in}}{\pgfqpoint{7.750000in}{3.850000in}}%
\pgfusepath{clip}%
\pgfsetbuttcap%
\pgfsetmiterjoin%
\definecolor{currentfill}{rgb}{0.501961,0.501961,0.501961}%
\pgfsetfillcolor{currentfill}%
\pgfsetfillopacity{0.700000}%
\pgfsetlinewidth{0.000000pt}%
\definecolor{currentstroke}{rgb}{0.000000,0.000000,0.000000}%
\pgfsetstrokecolor{currentstroke}%
\pgfsetstrokeopacity{0.700000}%
\pgfsetdash{}{0pt}%
\pgfpathmoveto{\pgfqpoint{3.336115in}{0.550000in}}%
\pgfpathlineto{\pgfqpoint{3.363636in}{0.550000in}}%
\pgfpathlineto{\pgfqpoint{3.363636in}{0.550000in}}%
\pgfpathlineto{\pgfqpoint{3.336115in}{0.550000in}}%
\pgfpathlineto{\pgfqpoint{3.336115in}{0.550000in}}%
\pgfpathclose%
\pgfusepath{fill}%
\end{pgfscope}%
\begin{pgfscope}%
\pgfpathrectangle{\pgfqpoint{1.250000in}{0.550000in}}{\pgfqpoint{7.750000in}{3.850000in}}%
\pgfusepath{clip}%
\pgfsetbuttcap%
\pgfsetmiterjoin%
\definecolor{currentfill}{rgb}{0.501961,0.501961,0.501961}%
\pgfsetfillcolor{currentfill}%
\pgfsetfillopacity{0.700000}%
\pgfsetlinewidth{0.000000pt}%
\definecolor{currentstroke}{rgb}{0.000000,0.000000,0.000000}%
\pgfsetstrokecolor{currentstroke}%
\pgfsetstrokeopacity{0.700000}%
\pgfsetdash{}{0pt}%
\pgfpathmoveto{\pgfqpoint{3.363636in}{0.550000in}}%
\pgfpathlineto{\pgfqpoint{3.391158in}{0.550000in}}%
\pgfpathlineto{\pgfqpoint{3.391158in}{0.550000in}}%
\pgfpathlineto{\pgfqpoint{3.363636in}{0.550000in}}%
\pgfpathlineto{\pgfqpoint{3.363636in}{0.550000in}}%
\pgfpathclose%
\pgfusepath{fill}%
\end{pgfscope}%
\begin{pgfscope}%
\pgfpathrectangle{\pgfqpoint{1.250000in}{0.550000in}}{\pgfqpoint{7.750000in}{3.850000in}}%
\pgfusepath{clip}%
\pgfsetbuttcap%
\pgfsetmiterjoin%
\definecolor{currentfill}{rgb}{0.501961,0.501961,0.501961}%
\pgfsetfillcolor{currentfill}%
\pgfsetfillopacity{0.700000}%
\pgfsetlinewidth{0.000000pt}%
\definecolor{currentstroke}{rgb}{0.000000,0.000000,0.000000}%
\pgfsetstrokecolor{currentstroke}%
\pgfsetstrokeopacity{0.700000}%
\pgfsetdash{}{0pt}%
\pgfpathmoveto{\pgfqpoint{3.391158in}{0.550000in}}%
\pgfpathlineto{\pgfqpoint{3.418679in}{0.550000in}}%
\pgfpathlineto{\pgfqpoint{3.418679in}{0.550000in}}%
\pgfpathlineto{\pgfqpoint{3.391158in}{0.550000in}}%
\pgfpathlineto{\pgfqpoint{3.391158in}{0.550000in}}%
\pgfpathclose%
\pgfusepath{fill}%
\end{pgfscope}%
\begin{pgfscope}%
\pgfpathrectangle{\pgfqpoint{1.250000in}{0.550000in}}{\pgfqpoint{7.750000in}{3.850000in}}%
\pgfusepath{clip}%
\pgfsetbuttcap%
\pgfsetmiterjoin%
\definecolor{currentfill}{rgb}{0.501961,0.501961,0.501961}%
\pgfsetfillcolor{currentfill}%
\pgfsetfillopacity{0.700000}%
\pgfsetlinewidth{0.000000pt}%
\definecolor{currentstroke}{rgb}{0.000000,0.000000,0.000000}%
\pgfsetstrokecolor{currentstroke}%
\pgfsetstrokeopacity{0.700000}%
\pgfsetdash{}{0pt}%
\pgfpathmoveto{\pgfqpoint{3.418679in}{0.550000in}}%
\pgfpathlineto{\pgfqpoint{3.446200in}{0.550000in}}%
\pgfpathlineto{\pgfqpoint{3.446200in}{1.319671in}}%
\pgfpathlineto{\pgfqpoint{3.418679in}{1.319671in}}%
\pgfpathlineto{\pgfqpoint{3.418679in}{0.550000in}}%
\pgfpathclose%
\pgfusepath{fill}%
\end{pgfscope}%
\begin{pgfscope}%
\pgfpathrectangle{\pgfqpoint{1.250000in}{0.550000in}}{\pgfqpoint{7.750000in}{3.850000in}}%
\pgfusepath{clip}%
\pgfsetbuttcap%
\pgfsetmiterjoin%
\definecolor{currentfill}{rgb}{0.501961,0.501961,0.501961}%
\pgfsetfillcolor{currentfill}%
\pgfsetfillopacity{0.700000}%
\pgfsetlinewidth{0.000000pt}%
\definecolor{currentstroke}{rgb}{0.000000,0.000000,0.000000}%
\pgfsetstrokecolor{currentstroke}%
\pgfsetstrokeopacity{0.700000}%
\pgfsetdash{}{0pt}%
\pgfpathmoveto{\pgfqpoint{3.446200in}{0.550000in}}%
\pgfpathlineto{\pgfqpoint{3.473722in}{0.550000in}}%
\pgfpathlineto{\pgfqpoint{3.473722in}{0.550000in}}%
\pgfpathlineto{\pgfqpoint{3.446200in}{0.550000in}}%
\pgfpathlineto{\pgfqpoint{3.446200in}{0.550000in}}%
\pgfpathclose%
\pgfusepath{fill}%
\end{pgfscope}%
\begin{pgfscope}%
\pgfpathrectangle{\pgfqpoint{1.250000in}{0.550000in}}{\pgfqpoint{7.750000in}{3.850000in}}%
\pgfusepath{clip}%
\pgfsetbuttcap%
\pgfsetmiterjoin%
\definecolor{currentfill}{rgb}{0.501961,0.501961,0.501961}%
\pgfsetfillcolor{currentfill}%
\pgfsetfillopacity{0.700000}%
\pgfsetlinewidth{0.000000pt}%
\definecolor{currentstroke}{rgb}{0.000000,0.000000,0.000000}%
\pgfsetstrokecolor{currentstroke}%
\pgfsetstrokeopacity{0.700000}%
\pgfsetdash{}{0pt}%
\pgfpathmoveto{\pgfqpoint{3.473722in}{0.550000in}}%
\pgfpathlineto{\pgfqpoint{3.501243in}{0.550000in}}%
\pgfpathlineto{\pgfqpoint{3.501243in}{0.550000in}}%
\pgfpathlineto{\pgfqpoint{3.473722in}{0.550000in}}%
\pgfpathlineto{\pgfqpoint{3.473722in}{0.550000in}}%
\pgfpathclose%
\pgfusepath{fill}%
\end{pgfscope}%
\begin{pgfscope}%
\pgfpathrectangle{\pgfqpoint{1.250000in}{0.550000in}}{\pgfqpoint{7.750000in}{3.850000in}}%
\pgfusepath{clip}%
\pgfsetbuttcap%
\pgfsetmiterjoin%
\definecolor{currentfill}{rgb}{0.501961,0.501961,0.501961}%
\pgfsetfillcolor{currentfill}%
\pgfsetfillopacity{0.700000}%
\pgfsetlinewidth{0.000000pt}%
\definecolor{currentstroke}{rgb}{0.000000,0.000000,0.000000}%
\pgfsetstrokecolor{currentstroke}%
\pgfsetstrokeopacity{0.700000}%
\pgfsetdash{}{0pt}%
\pgfpathmoveto{\pgfqpoint{3.501243in}{0.550000in}}%
\pgfpathlineto{\pgfqpoint{3.528764in}{0.550000in}}%
\pgfpathlineto{\pgfqpoint{3.528764in}{0.550000in}}%
\pgfpathlineto{\pgfqpoint{3.501243in}{0.550000in}}%
\pgfpathlineto{\pgfqpoint{3.501243in}{0.550000in}}%
\pgfpathclose%
\pgfusepath{fill}%
\end{pgfscope}%
\begin{pgfscope}%
\pgfpathrectangle{\pgfqpoint{1.250000in}{0.550000in}}{\pgfqpoint{7.750000in}{3.850000in}}%
\pgfusepath{clip}%
\pgfsetbuttcap%
\pgfsetmiterjoin%
\definecolor{currentfill}{rgb}{0.501961,0.501961,0.501961}%
\pgfsetfillcolor{currentfill}%
\pgfsetfillopacity{0.700000}%
\pgfsetlinewidth{0.000000pt}%
\definecolor{currentstroke}{rgb}{0.000000,0.000000,0.000000}%
\pgfsetstrokecolor{currentstroke}%
\pgfsetstrokeopacity{0.700000}%
\pgfsetdash{}{0pt}%
\pgfpathmoveto{\pgfqpoint{3.528764in}{0.550000in}}%
\pgfpathlineto{\pgfqpoint{3.556286in}{0.550000in}}%
\pgfpathlineto{\pgfqpoint{3.556286in}{1.243110in}}%
\pgfpathlineto{\pgfqpoint{3.528764in}{1.243110in}}%
\pgfpathlineto{\pgfqpoint{3.528764in}{0.550000in}}%
\pgfpathclose%
\pgfusepath{fill}%
\end{pgfscope}%
\begin{pgfscope}%
\pgfpathrectangle{\pgfqpoint{1.250000in}{0.550000in}}{\pgfqpoint{7.750000in}{3.850000in}}%
\pgfusepath{clip}%
\pgfsetbuttcap%
\pgfsetmiterjoin%
\definecolor{currentfill}{rgb}{0.501961,0.501961,0.501961}%
\pgfsetfillcolor{currentfill}%
\pgfsetfillopacity{0.700000}%
\pgfsetlinewidth{0.000000pt}%
\definecolor{currentstroke}{rgb}{0.000000,0.000000,0.000000}%
\pgfsetstrokecolor{currentstroke}%
\pgfsetstrokeopacity{0.700000}%
\pgfsetdash{}{0pt}%
\pgfpathmoveto{\pgfqpoint{3.556286in}{0.550000in}}%
\pgfpathlineto{\pgfqpoint{3.583807in}{0.550000in}}%
\pgfpathlineto{\pgfqpoint{3.583807in}{0.550000in}}%
\pgfpathlineto{\pgfqpoint{3.556286in}{0.550000in}}%
\pgfpathlineto{\pgfqpoint{3.556286in}{0.550000in}}%
\pgfpathclose%
\pgfusepath{fill}%
\end{pgfscope}%
\begin{pgfscope}%
\pgfpathrectangle{\pgfqpoint{1.250000in}{0.550000in}}{\pgfqpoint{7.750000in}{3.850000in}}%
\pgfusepath{clip}%
\pgfsetbuttcap%
\pgfsetmiterjoin%
\definecolor{currentfill}{rgb}{0.501961,0.501961,0.501961}%
\pgfsetfillcolor{currentfill}%
\pgfsetfillopacity{0.700000}%
\pgfsetlinewidth{0.000000pt}%
\definecolor{currentstroke}{rgb}{0.000000,0.000000,0.000000}%
\pgfsetstrokecolor{currentstroke}%
\pgfsetstrokeopacity{0.700000}%
\pgfsetdash{}{0pt}%
\pgfpathmoveto{\pgfqpoint{3.583807in}{0.550000in}}%
\pgfpathlineto{\pgfqpoint{3.611328in}{0.550000in}}%
\pgfpathlineto{\pgfqpoint{3.611328in}{0.550000in}}%
\pgfpathlineto{\pgfqpoint{3.583807in}{0.550000in}}%
\pgfpathlineto{\pgfqpoint{3.583807in}{0.550000in}}%
\pgfpathclose%
\pgfusepath{fill}%
\end{pgfscope}%
\begin{pgfscope}%
\pgfpathrectangle{\pgfqpoint{1.250000in}{0.550000in}}{\pgfqpoint{7.750000in}{3.850000in}}%
\pgfusepath{clip}%
\pgfsetbuttcap%
\pgfsetmiterjoin%
\definecolor{currentfill}{rgb}{0.501961,0.501961,0.501961}%
\pgfsetfillcolor{currentfill}%
\pgfsetfillopacity{0.700000}%
\pgfsetlinewidth{0.000000pt}%
\definecolor{currentstroke}{rgb}{0.000000,0.000000,0.000000}%
\pgfsetstrokecolor{currentstroke}%
\pgfsetstrokeopacity{0.700000}%
\pgfsetdash{}{0pt}%
\pgfpathmoveto{\pgfqpoint{3.611328in}{0.550000in}}%
\pgfpathlineto{\pgfqpoint{3.638849in}{0.550000in}}%
\pgfpathlineto{\pgfqpoint{3.638849in}{0.550000in}}%
\pgfpathlineto{\pgfqpoint{3.611328in}{0.550000in}}%
\pgfpathlineto{\pgfqpoint{3.611328in}{0.550000in}}%
\pgfpathclose%
\pgfusepath{fill}%
\end{pgfscope}%
\begin{pgfscope}%
\pgfpathrectangle{\pgfqpoint{1.250000in}{0.550000in}}{\pgfqpoint{7.750000in}{3.850000in}}%
\pgfusepath{clip}%
\pgfsetbuttcap%
\pgfsetmiterjoin%
\definecolor{currentfill}{rgb}{0.501961,0.501961,0.501961}%
\pgfsetfillcolor{currentfill}%
\pgfsetfillopacity{0.700000}%
\pgfsetlinewidth{0.000000pt}%
\definecolor{currentstroke}{rgb}{0.000000,0.000000,0.000000}%
\pgfsetstrokecolor{currentstroke}%
\pgfsetstrokeopacity{0.700000}%
\pgfsetdash{}{0pt}%
\pgfpathmoveto{\pgfqpoint{3.638849in}{0.550000in}}%
\pgfpathlineto{\pgfqpoint{3.666371in}{0.550000in}}%
\pgfpathlineto{\pgfqpoint{3.666371in}{1.179454in}}%
\pgfpathlineto{\pgfqpoint{3.638849in}{1.179454in}}%
\pgfpathlineto{\pgfqpoint{3.638849in}{0.550000in}}%
\pgfpathclose%
\pgfusepath{fill}%
\end{pgfscope}%
\begin{pgfscope}%
\pgfpathrectangle{\pgfqpoint{1.250000in}{0.550000in}}{\pgfqpoint{7.750000in}{3.850000in}}%
\pgfusepath{clip}%
\pgfsetbuttcap%
\pgfsetmiterjoin%
\definecolor{currentfill}{rgb}{0.501961,0.501961,0.501961}%
\pgfsetfillcolor{currentfill}%
\pgfsetfillopacity{0.700000}%
\pgfsetlinewidth{0.000000pt}%
\definecolor{currentstroke}{rgb}{0.000000,0.000000,0.000000}%
\pgfsetstrokecolor{currentstroke}%
\pgfsetstrokeopacity{0.700000}%
\pgfsetdash{}{0pt}%
\pgfpathmoveto{\pgfqpoint{3.666371in}{0.550000in}}%
\pgfpathlineto{\pgfqpoint{3.693892in}{0.550000in}}%
\pgfpathlineto{\pgfqpoint{3.693892in}{0.550000in}}%
\pgfpathlineto{\pgfqpoint{3.666371in}{0.550000in}}%
\pgfpathlineto{\pgfqpoint{3.666371in}{0.550000in}}%
\pgfpathclose%
\pgfusepath{fill}%
\end{pgfscope}%
\begin{pgfscope}%
\pgfpathrectangle{\pgfqpoint{1.250000in}{0.550000in}}{\pgfqpoint{7.750000in}{3.850000in}}%
\pgfusepath{clip}%
\pgfsetbuttcap%
\pgfsetmiterjoin%
\definecolor{currentfill}{rgb}{0.501961,0.501961,0.501961}%
\pgfsetfillcolor{currentfill}%
\pgfsetfillopacity{0.700000}%
\pgfsetlinewidth{0.000000pt}%
\definecolor{currentstroke}{rgb}{0.000000,0.000000,0.000000}%
\pgfsetstrokecolor{currentstroke}%
\pgfsetstrokeopacity{0.700000}%
\pgfsetdash{}{0pt}%
\pgfpathmoveto{\pgfqpoint{3.693892in}{0.550000in}}%
\pgfpathlineto{\pgfqpoint{3.721413in}{0.550000in}}%
\pgfpathlineto{\pgfqpoint{3.721413in}{0.550000in}}%
\pgfpathlineto{\pgfqpoint{3.693892in}{0.550000in}}%
\pgfpathlineto{\pgfqpoint{3.693892in}{0.550000in}}%
\pgfpathclose%
\pgfusepath{fill}%
\end{pgfscope}%
\begin{pgfscope}%
\pgfpathrectangle{\pgfqpoint{1.250000in}{0.550000in}}{\pgfqpoint{7.750000in}{3.850000in}}%
\pgfusepath{clip}%
\pgfsetbuttcap%
\pgfsetmiterjoin%
\definecolor{currentfill}{rgb}{0.501961,0.501961,0.501961}%
\pgfsetfillcolor{currentfill}%
\pgfsetfillopacity{0.700000}%
\pgfsetlinewidth{0.000000pt}%
\definecolor{currentstroke}{rgb}{0.000000,0.000000,0.000000}%
\pgfsetstrokecolor{currentstroke}%
\pgfsetstrokeopacity{0.700000}%
\pgfsetdash{}{0pt}%
\pgfpathmoveto{\pgfqpoint{3.721413in}{0.550000in}}%
\pgfpathlineto{\pgfqpoint{3.748935in}{0.550000in}}%
\pgfpathlineto{\pgfqpoint{3.748935in}{0.550000in}}%
\pgfpathlineto{\pgfqpoint{3.721413in}{0.550000in}}%
\pgfpathlineto{\pgfqpoint{3.721413in}{0.550000in}}%
\pgfpathclose%
\pgfusepath{fill}%
\end{pgfscope}%
\begin{pgfscope}%
\pgfpathrectangle{\pgfqpoint{1.250000in}{0.550000in}}{\pgfqpoint{7.750000in}{3.850000in}}%
\pgfusepath{clip}%
\pgfsetbuttcap%
\pgfsetmiterjoin%
\definecolor{currentfill}{rgb}{0.501961,0.501961,0.501961}%
\pgfsetfillcolor{currentfill}%
\pgfsetfillopacity{0.700000}%
\pgfsetlinewidth{0.000000pt}%
\definecolor{currentstroke}{rgb}{0.000000,0.000000,0.000000}%
\pgfsetstrokecolor{currentstroke}%
\pgfsetstrokeopacity{0.700000}%
\pgfsetdash{}{0pt}%
\pgfpathmoveto{\pgfqpoint{3.748935in}{0.550000in}}%
\pgfpathlineto{\pgfqpoint{3.776456in}{0.550000in}}%
\pgfpathlineto{\pgfqpoint{3.776456in}{1.167274in}}%
\pgfpathlineto{\pgfqpoint{3.748935in}{1.167274in}}%
\pgfpathlineto{\pgfqpoint{3.748935in}{0.550000in}}%
\pgfpathclose%
\pgfusepath{fill}%
\end{pgfscope}%
\begin{pgfscope}%
\pgfpathrectangle{\pgfqpoint{1.250000in}{0.550000in}}{\pgfqpoint{7.750000in}{3.850000in}}%
\pgfusepath{clip}%
\pgfsetbuttcap%
\pgfsetmiterjoin%
\definecolor{currentfill}{rgb}{0.501961,0.501961,0.501961}%
\pgfsetfillcolor{currentfill}%
\pgfsetfillopacity{0.700000}%
\pgfsetlinewidth{0.000000pt}%
\definecolor{currentstroke}{rgb}{0.000000,0.000000,0.000000}%
\pgfsetstrokecolor{currentstroke}%
\pgfsetstrokeopacity{0.700000}%
\pgfsetdash{}{0pt}%
\pgfpathmoveto{\pgfqpoint{3.776456in}{0.550000in}}%
\pgfpathlineto{\pgfqpoint{3.803977in}{0.550000in}}%
\pgfpathlineto{\pgfqpoint{3.803977in}{0.550000in}}%
\pgfpathlineto{\pgfqpoint{3.776456in}{0.550000in}}%
\pgfpathlineto{\pgfqpoint{3.776456in}{0.550000in}}%
\pgfpathclose%
\pgfusepath{fill}%
\end{pgfscope}%
\begin{pgfscope}%
\pgfpathrectangle{\pgfqpoint{1.250000in}{0.550000in}}{\pgfqpoint{7.750000in}{3.850000in}}%
\pgfusepath{clip}%
\pgfsetbuttcap%
\pgfsetmiterjoin%
\definecolor{currentfill}{rgb}{0.501961,0.501961,0.501961}%
\pgfsetfillcolor{currentfill}%
\pgfsetfillopacity{0.700000}%
\pgfsetlinewidth{0.000000pt}%
\definecolor{currentstroke}{rgb}{0.000000,0.000000,0.000000}%
\pgfsetstrokecolor{currentstroke}%
\pgfsetstrokeopacity{0.700000}%
\pgfsetdash{}{0pt}%
\pgfpathmoveto{\pgfqpoint{3.803977in}{0.550000in}}%
\pgfpathlineto{\pgfqpoint{3.831499in}{0.550000in}}%
\pgfpathlineto{\pgfqpoint{3.831499in}{0.550000in}}%
\pgfpathlineto{\pgfqpoint{3.803977in}{0.550000in}}%
\pgfpathlineto{\pgfqpoint{3.803977in}{0.550000in}}%
\pgfpathclose%
\pgfusepath{fill}%
\end{pgfscope}%
\begin{pgfscope}%
\pgfpathrectangle{\pgfqpoint{1.250000in}{0.550000in}}{\pgfqpoint{7.750000in}{3.850000in}}%
\pgfusepath{clip}%
\pgfsetbuttcap%
\pgfsetmiterjoin%
\definecolor{currentfill}{rgb}{0.501961,0.501961,0.501961}%
\pgfsetfillcolor{currentfill}%
\pgfsetfillopacity{0.700000}%
\pgfsetlinewidth{0.000000pt}%
\definecolor{currentstroke}{rgb}{0.000000,0.000000,0.000000}%
\pgfsetstrokecolor{currentstroke}%
\pgfsetstrokeopacity{0.700000}%
\pgfsetdash{}{0pt}%
\pgfpathmoveto{\pgfqpoint{3.831499in}{0.550000in}}%
\pgfpathlineto{\pgfqpoint{3.859020in}{0.550000in}}%
\pgfpathlineto{\pgfqpoint{3.859020in}{1.126383in}}%
\pgfpathlineto{\pgfqpoint{3.831499in}{1.126383in}}%
\pgfpathlineto{\pgfqpoint{3.831499in}{0.550000in}}%
\pgfpathclose%
\pgfusepath{fill}%
\end{pgfscope}%
\begin{pgfscope}%
\pgfpathrectangle{\pgfqpoint{1.250000in}{0.550000in}}{\pgfqpoint{7.750000in}{3.850000in}}%
\pgfusepath{clip}%
\pgfsetbuttcap%
\pgfsetmiterjoin%
\definecolor{currentfill}{rgb}{0.501961,0.501961,0.501961}%
\pgfsetfillcolor{currentfill}%
\pgfsetfillopacity{0.700000}%
\pgfsetlinewidth{0.000000pt}%
\definecolor{currentstroke}{rgb}{0.000000,0.000000,0.000000}%
\pgfsetstrokecolor{currentstroke}%
\pgfsetstrokeopacity{0.700000}%
\pgfsetdash{}{0pt}%
\pgfpathmoveto{\pgfqpoint{3.859020in}{0.550000in}}%
\pgfpathlineto{\pgfqpoint{3.886541in}{0.550000in}}%
\pgfpathlineto{\pgfqpoint{3.886541in}{0.550000in}}%
\pgfpathlineto{\pgfqpoint{3.859020in}{0.550000in}}%
\pgfpathlineto{\pgfqpoint{3.859020in}{0.550000in}}%
\pgfpathclose%
\pgfusepath{fill}%
\end{pgfscope}%
\begin{pgfscope}%
\pgfpathrectangle{\pgfqpoint{1.250000in}{0.550000in}}{\pgfqpoint{7.750000in}{3.850000in}}%
\pgfusepath{clip}%
\pgfsetbuttcap%
\pgfsetmiterjoin%
\definecolor{currentfill}{rgb}{0.501961,0.501961,0.501961}%
\pgfsetfillcolor{currentfill}%
\pgfsetfillopacity{0.700000}%
\pgfsetlinewidth{0.000000pt}%
\definecolor{currentstroke}{rgb}{0.000000,0.000000,0.000000}%
\pgfsetstrokecolor{currentstroke}%
\pgfsetstrokeopacity{0.700000}%
\pgfsetdash{}{0pt}%
\pgfpathmoveto{\pgfqpoint{3.886541in}{0.550000in}}%
\pgfpathlineto{\pgfqpoint{3.914062in}{0.550000in}}%
\pgfpathlineto{\pgfqpoint{3.914062in}{0.550000in}}%
\pgfpathlineto{\pgfqpoint{3.886541in}{0.550000in}}%
\pgfpathlineto{\pgfqpoint{3.886541in}{0.550000in}}%
\pgfpathclose%
\pgfusepath{fill}%
\end{pgfscope}%
\begin{pgfscope}%
\pgfpathrectangle{\pgfqpoint{1.250000in}{0.550000in}}{\pgfqpoint{7.750000in}{3.850000in}}%
\pgfusepath{clip}%
\pgfsetbuttcap%
\pgfsetmiterjoin%
\definecolor{currentfill}{rgb}{0.501961,0.501961,0.501961}%
\pgfsetfillcolor{currentfill}%
\pgfsetfillopacity{0.700000}%
\pgfsetlinewidth{0.000000pt}%
\definecolor{currentstroke}{rgb}{0.000000,0.000000,0.000000}%
\pgfsetstrokecolor{currentstroke}%
\pgfsetstrokeopacity{0.700000}%
\pgfsetdash{}{0pt}%
\pgfpathmoveto{\pgfqpoint{3.914062in}{0.550000in}}%
\pgfpathlineto{\pgfqpoint{3.941584in}{0.550000in}}%
\pgfpathlineto{\pgfqpoint{3.941584in}{0.550000in}}%
\pgfpathlineto{\pgfqpoint{3.914062in}{0.550000in}}%
\pgfpathlineto{\pgfqpoint{3.914062in}{0.550000in}}%
\pgfpathclose%
\pgfusepath{fill}%
\end{pgfscope}%
\begin{pgfscope}%
\pgfpathrectangle{\pgfqpoint{1.250000in}{0.550000in}}{\pgfqpoint{7.750000in}{3.850000in}}%
\pgfusepath{clip}%
\pgfsetbuttcap%
\pgfsetmiterjoin%
\definecolor{currentfill}{rgb}{0.501961,0.501961,0.501961}%
\pgfsetfillcolor{currentfill}%
\pgfsetfillopacity{0.700000}%
\pgfsetlinewidth{0.000000pt}%
\definecolor{currentstroke}{rgb}{0.000000,0.000000,0.000000}%
\pgfsetstrokecolor{currentstroke}%
\pgfsetstrokeopacity{0.700000}%
\pgfsetdash{}{0pt}%
\pgfpathmoveto{\pgfqpoint{3.941584in}{0.550000in}}%
\pgfpathlineto{\pgfqpoint{3.969105in}{0.550000in}}%
\pgfpathlineto{\pgfqpoint{3.969105in}{1.112463in}}%
\pgfpathlineto{\pgfqpoint{3.941584in}{1.112463in}}%
\pgfpathlineto{\pgfqpoint{3.941584in}{0.550000in}}%
\pgfpathclose%
\pgfusepath{fill}%
\end{pgfscope}%
\begin{pgfscope}%
\pgfpathrectangle{\pgfqpoint{1.250000in}{0.550000in}}{\pgfqpoint{7.750000in}{3.850000in}}%
\pgfusepath{clip}%
\pgfsetbuttcap%
\pgfsetmiterjoin%
\definecolor{currentfill}{rgb}{0.501961,0.501961,0.501961}%
\pgfsetfillcolor{currentfill}%
\pgfsetfillopacity{0.700000}%
\pgfsetlinewidth{0.000000pt}%
\definecolor{currentstroke}{rgb}{0.000000,0.000000,0.000000}%
\pgfsetstrokecolor{currentstroke}%
\pgfsetstrokeopacity{0.700000}%
\pgfsetdash{}{0pt}%
\pgfpathmoveto{\pgfqpoint{3.969105in}{0.550000in}}%
\pgfpathlineto{\pgfqpoint{3.996626in}{0.550000in}}%
\pgfpathlineto{\pgfqpoint{3.996626in}{0.550000in}}%
\pgfpathlineto{\pgfqpoint{3.969105in}{0.550000in}}%
\pgfpathlineto{\pgfqpoint{3.969105in}{0.550000in}}%
\pgfpathclose%
\pgfusepath{fill}%
\end{pgfscope}%
\begin{pgfscope}%
\pgfpathrectangle{\pgfqpoint{1.250000in}{0.550000in}}{\pgfqpoint{7.750000in}{3.850000in}}%
\pgfusepath{clip}%
\pgfsetbuttcap%
\pgfsetmiterjoin%
\definecolor{currentfill}{rgb}{0.501961,0.501961,0.501961}%
\pgfsetfillcolor{currentfill}%
\pgfsetfillopacity{0.700000}%
\pgfsetlinewidth{0.000000pt}%
\definecolor{currentstroke}{rgb}{0.000000,0.000000,0.000000}%
\pgfsetstrokecolor{currentstroke}%
\pgfsetstrokeopacity{0.700000}%
\pgfsetdash{}{0pt}%
\pgfpathmoveto{\pgfqpoint{3.996626in}{0.550000in}}%
\pgfpathlineto{\pgfqpoint{4.024148in}{0.550000in}}%
\pgfpathlineto{\pgfqpoint{4.024148in}{0.550000in}}%
\pgfpathlineto{\pgfqpoint{3.996626in}{0.550000in}}%
\pgfpathlineto{\pgfqpoint{3.996626in}{0.550000in}}%
\pgfpathclose%
\pgfusepath{fill}%
\end{pgfscope}%
\begin{pgfscope}%
\pgfpathrectangle{\pgfqpoint{1.250000in}{0.550000in}}{\pgfqpoint{7.750000in}{3.850000in}}%
\pgfusepath{clip}%
\pgfsetbuttcap%
\pgfsetmiterjoin%
\definecolor{currentfill}{rgb}{0.501961,0.501961,0.501961}%
\pgfsetfillcolor{currentfill}%
\pgfsetfillopacity{0.700000}%
\pgfsetlinewidth{0.000000pt}%
\definecolor{currentstroke}{rgb}{0.000000,0.000000,0.000000}%
\pgfsetstrokecolor{currentstroke}%
\pgfsetstrokeopacity{0.700000}%
\pgfsetdash{}{0pt}%
\pgfpathmoveto{\pgfqpoint{4.024148in}{0.550000in}}%
\pgfpathlineto{\pgfqpoint{4.051669in}{0.550000in}}%
\pgfpathlineto{\pgfqpoint{4.051669in}{1.099848in}}%
\pgfpathlineto{\pgfqpoint{4.024148in}{1.099848in}}%
\pgfpathlineto{\pgfqpoint{4.024148in}{0.550000in}}%
\pgfpathclose%
\pgfusepath{fill}%
\end{pgfscope}%
\begin{pgfscope}%
\pgfpathrectangle{\pgfqpoint{1.250000in}{0.550000in}}{\pgfqpoint{7.750000in}{3.850000in}}%
\pgfusepath{clip}%
\pgfsetbuttcap%
\pgfsetmiterjoin%
\definecolor{currentfill}{rgb}{0.501961,0.501961,0.501961}%
\pgfsetfillcolor{currentfill}%
\pgfsetfillopacity{0.700000}%
\pgfsetlinewidth{0.000000pt}%
\definecolor{currentstroke}{rgb}{0.000000,0.000000,0.000000}%
\pgfsetstrokecolor{currentstroke}%
\pgfsetstrokeopacity{0.700000}%
\pgfsetdash{}{0pt}%
\pgfpathmoveto{\pgfqpoint{4.051669in}{0.550000in}}%
\pgfpathlineto{\pgfqpoint{4.079190in}{0.550000in}}%
\pgfpathlineto{\pgfqpoint{4.079190in}{0.550000in}}%
\pgfpathlineto{\pgfqpoint{4.051669in}{0.550000in}}%
\pgfpathlineto{\pgfqpoint{4.051669in}{0.550000in}}%
\pgfpathclose%
\pgfusepath{fill}%
\end{pgfscope}%
\begin{pgfscope}%
\pgfpathrectangle{\pgfqpoint{1.250000in}{0.550000in}}{\pgfqpoint{7.750000in}{3.850000in}}%
\pgfusepath{clip}%
\pgfsetbuttcap%
\pgfsetmiterjoin%
\definecolor{currentfill}{rgb}{0.501961,0.501961,0.501961}%
\pgfsetfillcolor{currentfill}%
\pgfsetfillopacity{0.700000}%
\pgfsetlinewidth{0.000000pt}%
\definecolor{currentstroke}{rgb}{0.000000,0.000000,0.000000}%
\pgfsetstrokecolor{currentstroke}%
\pgfsetstrokeopacity{0.700000}%
\pgfsetdash{}{0pt}%
\pgfpathmoveto{\pgfqpoint{4.079190in}{0.550000in}}%
\pgfpathlineto{\pgfqpoint{4.106712in}{0.550000in}}%
\pgfpathlineto{\pgfqpoint{4.106712in}{0.550000in}}%
\pgfpathlineto{\pgfqpoint{4.079190in}{0.550000in}}%
\pgfpathlineto{\pgfqpoint{4.079190in}{0.550000in}}%
\pgfpathclose%
\pgfusepath{fill}%
\end{pgfscope}%
\begin{pgfscope}%
\pgfpathrectangle{\pgfqpoint{1.250000in}{0.550000in}}{\pgfqpoint{7.750000in}{3.850000in}}%
\pgfusepath{clip}%
\pgfsetbuttcap%
\pgfsetmiterjoin%
\definecolor{currentfill}{rgb}{0.501961,0.501961,0.501961}%
\pgfsetfillcolor{currentfill}%
\pgfsetfillopacity{0.700000}%
\pgfsetlinewidth{0.000000pt}%
\definecolor{currentstroke}{rgb}{0.000000,0.000000,0.000000}%
\pgfsetstrokecolor{currentstroke}%
\pgfsetstrokeopacity{0.700000}%
\pgfsetdash{}{0pt}%
\pgfpathmoveto{\pgfqpoint{4.106712in}{0.550000in}}%
\pgfpathlineto{\pgfqpoint{4.134233in}{0.550000in}}%
\pgfpathlineto{\pgfqpoint{4.134233in}{0.550000in}}%
\pgfpathlineto{\pgfqpoint{4.106712in}{0.550000in}}%
\pgfpathlineto{\pgfqpoint{4.106712in}{0.550000in}}%
\pgfpathclose%
\pgfusepath{fill}%
\end{pgfscope}%
\begin{pgfscope}%
\pgfpathrectangle{\pgfqpoint{1.250000in}{0.550000in}}{\pgfqpoint{7.750000in}{3.850000in}}%
\pgfusepath{clip}%
\pgfsetbuttcap%
\pgfsetmiterjoin%
\definecolor{currentfill}{rgb}{0.501961,0.501961,0.501961}%
\pgfsetfillcolor{currentfill}%
\pgfsetfillopacity{0.700000}%
\pgfsetlinewidth{0.000000pt}%
\definecolor{currentstroke}{rgb}{0.000000,0.000000,0.000000}%
\pgfsetstrokecolor{currentstroke}%
\pgfsetstrokeopacity{0.700000}%
\pgfsetdash{}{0pt}%
\pgfpathmoveto{\pgfqpoint{4.134233in}{0.550000in}}%
\pgfpathlineto{\pgfqpoint{4.161754in}{0.550000in}}%
\pgfpathlineto{\pgfqpoint{4.161754in}{1.064902in}}%
\pgfpathlineto{\pgfqpoint{4.134233in}{1.064902in}}%
\pgfpathlineto{\pgfqpoint{4.134233in}{0.550000in}}%
\pgfpathclose%
\pgfusepath{fill}%
\end{pgfscope}%
\begin{pgfscope}%
\pgfpathrectangle{\pgfqpoint{1.250000in}{0.550000in}}{\pgfqpoint{7.750000in}{3.850000in}}%
\pgfusepath{clip}%
\pgfsetbuttcap%
\pgfsetmiterjoin%
\definecolor{currentfill}{rgb}{0.501961,0.501961,0.501961}%
\pgfsetfillcolor{currentfill}%
\pgfsetfillopacity{0.700000}%
\pgfsetlinewidth{0.000000pt}%
\definecolor{currentstroke}{rgb}{0.000000,0.000000,0.000000}%
\pgfsetstrokecolor{currentstroke}%
\pgfsetstrokeopacity{0.700000}%
\pgfsetdash{}{0pt}%
\pgfpathmoveto{\pgfqpoint{4.161754in}{0.550000in}}%
\pgfpathlineto{\pgfqpoint{4.189276in}{0.550000in}}%
\pgfpathlineto{\pgfqpoint{4.189276in}{0.550000in}}%
\pgfpathlineto{\pgfqpoint{4.161754in}{0.550000in}}%
\pgfpathlineto{\pgfqpoint{4.161754in}{0.550000in}}%
\pgfpathclose%
\pgfusepath{fill}%
\end{pgfscope}%
\begin{pgfscope}%
\pgfpathrectangle{\pgfqpoint{1.250000in}{0.550000in}}{\pgfqpoint{7.750000in}{3.850000in}}%
\pgfusepath{clip}%
\pgfsetbuttcap%
\pgfsetmiterjoin%
\definecolor{currentfill}{rgb}{0.501961,0.501961,0.501961}%
\pgfsetfillcolor{currentfill}%
\pgfsetfillopacity{0.700000}%
\pgfsetlinewidth{0.000000pt}%
\definecolor{currentstroke}{rgb}{0.000000,0.000000,0.000000}%
\pgfsetstrokecolor{currentstroke}%
\pgfsetstrokeopacity{0.700000}%
\pgfsetdash{}{0pt}%
\pgfpathmoveto{\pgfqpoint{4.189276in}{0.550000in}}%
\pgfpathlineto{\pgfqpoint{4.216797in}{0.550000in}}%
\pgfpathlineto{\pgfqpoint{4.216797in}{0.550000in}}%
\pgfpathlineto{\pgfqpoint{4.189276in}{0.550000in}}%
\pgfpathlineto{\pgfqpoint{4.189276in}{0.550000in}}%
\pgfpathclose%
\pgfusepath{fill}%
\end{pgfscope}%
\begin{pgfscope}%
\pgfpathrectangle{\pgfqpoint{1.250000in}{0.550000in}}{\pgfqpoint{7.750000in}{3.850000in}}%
\pgfusepath{clip}%
\pgfsetbuttcap%
\pgfsetmiterjoin%
\definecolor{currentfill}{rgb}{0.501961,0.501961,0.501961}%
\pgfsetfillcolor{currentfill}%
\pgfsetfillopacity{0.700000}%
\pgfsetlinewidth{0.000000pt}%
\definecolor{currentstroke}{rgb}{0.000000,0.000000,0.000000}%
\pgfsetstrokecolor{currentstroke}%
\pgfsetstrokeopacity{0.700000}%
\pgfsetdash{}{0pt}%
\pgfpathmoveto{\pgfqpoint{4.216797in}{0.550000in}}%
\pgfpathlineto{\pgfqpoint{4.244318in}{0.550000in}}%
\pgfpathlineto{\pgfqpoint{4.244318in}{1.074327in}}%
\pgfpathlineto{\pgfqpoint{4.216797in}{1.074327in}}%
\pgfpathlineto{\pgfqpoint{4.216797in}{0.550000in}}%
\pgfpathclose%
\pgfusepath{fill}%
\end{pgfscope}%
\begin{pgfscope}%
\pgfpathrectangle{\pgfqpoint{1.250000in}{0.550000in}}{\pgfqpoint{7.750000in}{3.850000in}}%
\pgfusepath{clip}%
\pgfsetbuttcap%
\pgfsetmiterjoin%
\definecolor{currentfill}{rgb}{0.501961,0.501961,0.501961}%
\pgfsetfillcolor{currentfill}%
\pgfsetfillopacity{0.700000}%
\pgfsetlinewidth{0.000000pt}%
\definecolor{currentstroke}{rgb}{0.000000,0.000000,0.000000}%
\pgfsetstrokecolor{currentstroke}%
\pgfsetstrokeopacity{0.700000}%
\pgfsetdash{}{0pt}%
\pgfpathmoveto{\pgfqpoint{4.244318in}{0.550000in}}%
\pgfpathlineto{\pgfqpoint{4.271839in}{0.550000in}}%
\pgfpathlineto{\pgfqpoint{4.271839in}{0.550000in}}%
\pgfpathlineto{\pgfqpoint{4.244318in}{0.550000in}}%
\pgfpathlineto{\pgfqpoint{4.244318in}{0.550000in}}%
\pgfpathclose%
\pgfusepath{fill}%
\end{pgfscope}%
\begin{pgfscope}%
\pgfpathrectangle{\pgfqpoint{1.250000in}{0.550000in}}{\pgfqpoint{7.750000in}{3.850000in}}%
\pgfusepath{clip}%
\pgfsetbuttcap%
\pgfsetmiterjoin%
\definecolor{currentfill}{rgb}{0.501961,0.501961,0.501961}%
\pgfsetfillcolor{currentfill}%
\pgfsetfillopacity{0.700000}%
\pgfsetlinewidth{0.000000pt}%
\definecolor{currentstroke}{rgb}{0.000000,0.000000,0.000000}%
\pgfsetstrokecolor{currentstroke}%
\pgfsetstrokeopacity{0.700000}%
\pgfsetdash{}{0pt}%
\pgfpathmoveto{\pgfqpoint{4.271839in}{0.550000in}}%
\pgfpathlineto{\pgfqpoint{4.299361in}{0.550000in}}%
\pgfpathlineto{\pgfqpoint{4.299361in}{0.550000in}}%
\pgfpathlineto{\pgfqpoint{4.271839in}{0.550000in}}%
\pgfpathlineto{\pgfqpoint{4.271839in}{0.550000in}}%
\pgfpathclose%
\pgfusepath{fill}%
\end{pgfscope}%
\begin{pgfscope}%
\pgfpathrectangle{\pgfqpoint{1.250000in}{0.550000in}}{\pgfqpoint{7.750000in}{3.850000in}}%
\pgfusepath{clip}%
\pgfsetbuttcap%
\pgfsetmiterjoin%
\definecolor{currentfill}{rgb}{0.501961,0.501961,0.501961}%
\pgfsetfillcolor{currentfill}%
\pgfsetfillopacity{0.700000}%
\pgfsetlinewidth{0.000000pt}%
\definecolor{currentstroke}{rgb}{0.000000,0.000000,0.000000}%
\pgfsetstrokecolor{currentstroke}%
\pgfsetstrokeopacity{0.700000}%
\pgfsetdash{}{0pt}%
\pgfpathmoveto{\pgfqpoint{4.299361in}{0.550000in}}%
\pgfpathlineto{\pgfqpoint{4.326882in}{0.550000in}}%
\pgfpathlineto{\pgfqpoint{4.326882in}{1.044312in}}%
\pgfpathlineto{\pgfqpoint{4.299361in}{1.044312in}}%
\pgfpathlineto{\pgfqpoint{4.299361in}{0.550000in}}%
\pgfpathclose%
\pgfusepath{fill}%
\end{pgfscope}%
\begin{pgfscope}%
\pgfpathrectangle{\pgfqpoint{1.250000in}{0.550000in}}{\pgfqpoint{7.750000in}{3.850000in}}%
\pgfusepath{clip}%
\pgfsetbuttcap%
\pgfsetmiterjoin%
\definecolor{currentfill}{rgb}{0.501961,0.501961,0.501961}%
\pgfsetfillcolor{currentfill}%
\pgfsetfillopacity{0.700000}%
\pgfsetlinewidth{0.000000pt}%
\definecolor{currentstroke}{rgb}{0.000000,0.000000,0.000000}%
\pgfsetstrokecolor{currentstroke}%
\pgfsetstrokeopacity{0.700000}%
\pgfsetdash{}{0pt}%
\pgfpathmoveto{\pgfqpoint{4.326882in}{0.550000in}}%
\pgfpathlineto{\pgfqpoint{4.354403in}{0.550000in}}%
\pgfpathlineto{\pgfqpoint{4.354403in}{0.550000in}}%
\pgfpathlineto{\pgfqpoint{4.326882in}{0.550000in}}%
\pgfpathlineto{\pgfqpoint{4.326882in}{0.550000in}}%
\pgfpathclose%
\pgfusepath{fill}%
\end{pgfscope}%
\begin{pgfscope}%
\pgfpathrectangle{\pgfqpoint{1.250000in}{0.550000in}}{\pgfqpoint{7.750000in}{3.850000in}}%
\pgfusepath{clip}%
\pgfsetbuttcap%
\pgfsetmiterjoin%
\definecolor{currentfill}{rgb}{0.501961,0.501961,0.501961}%
\pgfsetfillcolor{currentfill}%
\pgfsetfillopacity{0.700000}%
\pgfsetlinewidth{0.000000pt}%
\definecolor{currentstroke}{rgb}{0.000000,0.000000,0.000000}%
\pgfsetstrokecolor{currentstroke}%
\pgfsetstrokeopacity{0.700000}%
\pgfsetdash{}{0pt}%
\pgfpathmoveto{\pgfqpoint{4.354403in}{0.550000in}}%
\pgfpathlineto{\pgfqpoint{4.381925in}{0.550000in}}%
\pgfpathlineto{\pgfqpoint{4.381925in}{0.550000in}}%
\pgfpathlineto{\pgfqpoint{4.354403in}{0.550000in}}%
\pgfpathlineto{\pgfqpoint{4.354403in}{0.550000in}}%
\pgfpathclose%
\pgfusepath{fill}%
\end{pgfscope}%
\begin{pgfscope}%
\pgfpathrectangle{\pgfqpoint{1.250000in}{0.550000in}}{\pgfqpoint{7.750000in}{3.850000in}}%
\pgfusepath{clip}%
\pgfsetbuttcap%
\pgfsetmiterjoin%
\definecolor{currentfill}{rgb}{0.501961,0.501961,0.501961}%
\pgfsetfillcolor{currentfill}%
\pgfsetfillopacity{0.700000}%
\pgfsetlinewidth{0.000000pt}%
\definecolor{currentstroke}{rgb}{0.000000,0.000000,0.000000}%
\pgfsetstrokecolor{currentstroke}%
\pgfsetstrokeopacity{0.700000}%
\pgfsetdash{}{0pt}%
\pgfpathmoveto{\pgfqpoint{4.381925in}{0.550000in}}%
\pgfpathlineto{\pgfqpoint{4.409446in}{0.550000in}}%
\pgfpathlineto{\pgfqpoint{4.409446in}{1.041267in}}%
\pgfpathlineto{\pgfqpoint{4.381925in}{1.041267in}}%
\pgfpathlineto{\pgfqpoint{4.381925in}{0.550000in}}%
\pgfpathclose%
\pgfusepath{fill}%
\end{pgfscope}%
\begin{pgfscope}%
\pgfpathrectangle{\pgfqpoint{1.250000in}{0.550000in}}{\pgfqpoint{7.750000in}{3.850000in}}%
\pgfusepath{clip}%
\pgfsetbuttcap%
\pgfsetmiterjoin%
\definecolor{currentfill}{rgb}{0.501961,0.501961,0.501961}%
\pgfsetfillcolor{currentfill}%
\pgfsetfillopacity{0.700000}%
\pgfsetlinewidth{0.000000pt}%
\definecolor{currentstroke}{rgb}{0.000000,0.000000,0.000000}%
\pgfsetstrokecolor{currentstroke}%
\pgfsetstrokeopacity{0.700000}%
\pgfsetdash{}{0pt}%
\pgfpathmoveto{\pgfqpoint{4.409446in}{0.550000in}}%
\pgfpathlineto{\pgfqpoint{4.436967in}{0.550000in}}%
\pgfpathlineto{\pgfqpoint{4.436967in}{0.550000in}}%
\pgfpathlineto{\pgfqpoint{4.409446in}{0.550000in}}%
\pgfpathlineto{\pgfqpoint{4.409446in}{0.550000in}}%
\pgfpathclose%
\pgfusepath{fill}%
\end{pgfscope}%
\begin{pgfscope}%
\pgfpathrectangle{\pgfqpoint{1.250000in}{0.550000in}}{\pgfqpoint{7.750000in}{3.850000in}}%
\pgfusepath{clip}%
\pgfsetbuttcap%
\pgfsetmiterjoin%
\definecolor{currentfill}{rgb}{0.501961,0.501961,0.501961}%
\pgfsetfillcolor{currentfill}%
\pgfsetfillopacity{0.700000}%
\pgfsetlinewidth{0.000000pt}%
\definecolor{currentstroke}{rgb}{0.000000,0.000000,0.000000}%
\pgfsetstrokecolor{currentstroke}%
\pgfsetstrokeopacity{0.700000}%
\pgfsetdash{}{0pt}%
\pgfpathmoveto{\pgfqpoint{4.436967in}{0.550000in}}%
\pgfpathlineto{\pgfqpoint{4.464489in}{0.550000in}}%
\pgfpathlineto{\pgfqpoint{4.464489in}{0.550000in}}%
\pgfpathlineto{\pgfqpoint{4.436967in}{0.550000in}}%
\pgfpathlineto{\pgfqpoint{4.436967in}{0.550000in}}%
\pgfpathclose%
\pgfusepath{fill}%
\end{pgfscope}%
\begin{pgfscope}%
\pgfpathrectangle{\pgfqpoint{1.250000in}{0.550000in}}{\pgfqpoint{7.750000in}{3.850000in}}%
\pgfusepath{clip}%
\pgfsetbuttcap%
\pgfsetmiterjoin%
\definecolor{currentfill}{rgb}{0.501961,0.501961,0.501961}%
\pgfsetfillcolor{currentfill}%
\pgfsetfillopacity{0.700000}%
\pgfsetlinewidth{0.000000pt}%
\definecolor{currentstroke}{rgb}{0.000000,0.000000,0.000000}%
\pgfsetstrokecolor{currentstroke}%
\pgfsetstrokeopacity{0.700000}%
\pgfsetdash{}{0pt}%
\pgfpathmoveto{\pgfqpoint{4.464489in}{0.550000in}}%
\pgfpathlineto{\pgfqpoint{4.492010in}{0.550000in}}%
\pgfpathlineto{\pgfqpoint{4.492010in}{1.042282in}}%
\pgfpathlineto{\pgfqpoint{4.464489in}{1.042282in}}%
\pgfpathlineto{\pgfqpoint{4.464489in}{0.550000in}}%
\pgfpathclose%
\pgfusepath{fill}%
\end{pgfscope}%
\begin{pgfscope}%
\pgfpathrectangle{\pgfqpoint{1.250000in}{0.550000in}}{\pgfqpoint{7.750000in}{3.850000in}}%
\pgfusepath{clip}%
\pgfsetbuttcap%
\pgfsetmiterjoin%
\definecolor{currentfill}{rgb}{0.501961,0.501961,0.501961}%
\pgfsetfillcolor{currentfill}%
\pgfsetfillopacity{0.700000}%
\pgfsetlinewidth{0.000000pt}%
\definecolor{currentstroke}{rgb}{0.000000,0.000000,0.000000}%
\pgfsetstrokecolor{currentstroke}%
\pgfsetstrokeopacity{0.700000}%
\pgfsetdash{}{0pt}%
\pgfpathmoveto{\pgfqpoint{4.492010in}{0.550000in}}%
\pgfpathlineto{\pgfqpoint{4.519531in}{0.550000in}}%
\pgfpathlineto{\pgfqpoint{4.519531in}{0.550000in}}%
\pgfpathlineto{\pgfqpoint{4.492010in}{0.550000in}}%
\pgfpathlineto{\pgfqpoint{4.492010in}{0.550000in}}%
\pgfpathclose%
\pgfusepath{fill}%
\end{pgfscope}%
\begin{pgfscope}%
\pgfpathrectangle{\pgfqpoint{1.250000in}{0.550000in}}{\pgfqpoint{7.750000in}{3.850000in}}%
\pgfusepath{clip}%
\pgfsetbuttcap%
\pgfsetmiterjoin%
\definecolor{currentfill}{rgb}{0.501961,0.501961,0.501961}%
\pgfsetfillcolor{currentfill}%
\pgfsetfillopacity{0.700000}%
\pgfsetlinewidth{0.000000pt}%
\definecolor{currentstroke}{rgb}{0.000000,0.000000,0.000000}%
\pgfsetstrokecolor{currentstroke}%
\pgfsetstrokeopacity{0.700000}%
\pgfsetdash{}{0pt}%
\pgfpathmoveto{\pgfqpoint{4.519531in}{0.550000in}}%
\pgfpathlineto{\pgfqpoint{4.547053in}{0.550000in}}%
\pgfpathlineto{\pgfqpoint{4.547053in}{0.550000in}}%
\pgfpathlineto{\pgfqpoint{4.519531in}{0.550000in}}%
\pgfpathlineto{\pgfqpoint{4.519531in}{0.550000in}}%
\pgfpathclose%
\pgfusepath{fill}%
\end{pgfscope}%
\begin{pgfscope}%
\pgfpathrectangle{\pgfqpoint{1.250000in}{0.550000in}}{\pgfqpoint{7.750000in}{3.850000in}}%
\pgfusepath{clip}%
\pgfsetbuttcap%
\pgfsetmiterjoin%
\definecolor{currentfill}{rgb}{0.501961,0.501961,0.501961}%
\pgfsetfillcolor{currentfill}%
\pgfsetfillopacity{0.700000}%
\pgfsetlinewidth{0.000000pt}%
\definecolor{currentstroke}{rgb}{0.000000,0.000000,0.000000}%
\pgfsetstrokecolor{currentstroke}%
\pgfsetstrokeopacity{0.700000}%
\pgfsetdash{}{0pt}%
\pgfpathmoveto{\pgfqpoint{4.547053in}{0.550000in}}%
\pgfpathlineto{\pgfqpoint{4.574574in}{0.550000in}}%
\pgfpathlineto{\pgfqpoint{4.574574in}{1.032422in}}%
\pgfpathlineto{\pgfqpoint{4.547053in}{1.032422in}}%
\pgfpathlineto{\pgfqpoint{4.547053in}{0.550000in}}%
\pgfpathclose%
\pgfusepath{fill}%
\end{pgfscope}%
\begin{pgfscope}%
\pgfpathrectangle{\pgfqpoint{1.250000in}{0.550000in}}{\pgfqpoint{7.750000in}{3.850000in}}%
\pgfusepath{clip}%
\pgfsetbuttcap%
\pgfsetmiterjoin%
\definecolor{currentfill}{rgb}{0.501961,0.501961,0.501961}%
\pgfsetfillcolor{currentfill}%
\pgfsetfillopacity{0.700000}%
\pgfsetlinewidth{0.000000pt}%
\definecolor{currentstroke}{rgb}{0.000000,0.000000,0.000000}%
\pgfsetstrokecolor{currentstroke}%
\pgfsetstrokeopacity{0.700000}%
\pgfsetdash{}{0pt}%
\pgfpathmoveto{\pgfqpoint{4.574574in}{0.550000in}}%
\pgfpathlineto{\pgfqpoint{4.602095in}{0.550000in}}%
\pgfpathlineto{\pgfqpoint{4.602095in}{0.550000in}}%
\pgfpathlineto{\pgfqpoint{4.574574in}{0.550000in}}%
\pgfpathlineto{\pgfqpoint{4.574574in}{0.550000in}}%
\pgfpathclose%
\pgfusepath{fill}%
\end{pgfscope}%
\begin{pgfscope}%
\pgfpathrectangle{\pgfqpoint{1.250000in}{0.550000in}}{\pgfqpoint{7.750000in}{3.850000in}}%
\pgfusepath{clip}%
\pgfsetbuttcap%
\pgfsetmiterjoin%
\definecolor{currentfill}{rgb}{0.501961,0.501961,0.501961}%
\pgfsetfillcolor{currentfill}%
\pgfsetfillopacity{0.700000}%
\pgfsetlinewidth{0.000000pt}%
\definecolor{currentstroke}{rgb}{0.000000,0.000000,0.000000}%
\pgfsetstrokecolor{currentstroke}%
\pgfsetstrokeopacity{0.700000}%
\pgfsetdash{}{0pt}%
\pgfpathmoveto{\pgfqpoint{4.602095in}{0.550000in}}%
\pgfpathlineto{\pgfqpoint{4.629616in}{0.550000in}}%
\pgfpathlineto{\pgfqpoint{4.629616in}{0.550000in}}%
\pgfpathlineto{\pgfqpoint{4.602095in}{0.550000in}}%
\pgfpathlineto{\pgfqpoint{4.602095in}{0.550000in}}%
\pgfpathclose%
\pgfusepath{fill}%
\end{pgfscope}%
\begin{pgfscope}%
\pgfpathrectangle{\pgfqpoint{1.250000in}{0.550000in}}{\pgfqpoint{7.750000in}{3.850000in}}%
\pgfusepath{clip}%
\pgfsetbuttcap%
\pgfsetmiterjoin%
\definecolor{currentfill}{rgb}{0.501961,0.501961,0.501961}%
\pgfsetfillcolor{currentfill}%
\pgfsetfillopacity{0.700000}%
\pgfsetlinewidth{0.000000pt}%
\definecolor{currentstroke}{rgb}{0.000000,0.000000,0.000000}%
\pgfsetstrokecolor{currentstroke}%
\pgfsetstrokeopacity{0.700000}%
\pgfsetdash{}{0pt}%
\pgfpathmoveto{\pgfqpoint{4.629616in}{0.550000in}}%
\pgfpathlineto{\pgfqpoint{4.657138in}{0.550000in}}%
\pgfpathlineto{\pgfqpoint{4.657138in}{1.008351in}}%
\pgfpathlineto{\pgfqpoint{4.629616in}{1.008351in}}%
\pgfpathlineto{\pgfqpoint{4.629616in}{0.550000in}}%
\pgfpathclose%
\pgfusepath{fill}%
\end{pgfscope}%
\begin{pgfscope}%
\pgfpathrectangle{\pgfqpoint{1.250000in}{0.550000in}}{\pgfqpoint{7.750000in}{3.850000in}}%
\pgfusepath{clip}%
\pgfsetbuttcap%
\pgfsetmiterjoin%
\definecolor{currentfill}{rgb}{0.501961,0.501961,0.501961}%
\pgfsetfillcolor{currentfill}%
\pgfsetfillopacity{0.700000}%
\pgfsetlinewidth{0.000000pt}%
\definecolor{currentstroke}{rgb}{0.000000,0.000000,0.000000}%
\pgfsetstrokecolor{currentstroke}%
\pgfsetstrokeopacity{0.700000}%
\pgfsetdash{}{0pt}%
\pgfpathmoveto{\pgfqpoint{4.657138in}{0.550000in}}%
\pgfpathlineto{\pgfqpoint{4.684659in}{0.550000in}}%
\pgfpathlineto{\pgfqpoint{4.684659in}{0.550000in}}%
\pgfpathlineto{\pgfqpoint{4.657138in}{0.550000in}}%
\pgfpathlineto{\pgfqpoint{4.657138in}{0.550000in}}%
\pgfpathclose%
\pgfusepath{fill}%
\end{pgfscope}%
\begin{pgfscope}%
\pgfpathrectangle{\pgfqpoint{1.250000in}{0.550000in}}{\pgfqpoint{7.750000in}{3.850000in}}%
\pgfusepath{clip}%
\pgfsetbuttcap%
\pgfsetmiterjoin%
\definecolor{currentfill}{rgb}{0.501961,0.501961,0.501961}%
\pgfsetfillcolor{currentfill}%
\pgfsetfillopacity{0.700000}%
\pgfsetlinewidth{0.000000pt}%
\definecolor{currentstroke}{rgb}{0.000000,0.000000,0.000000}%
\pgfsetstrokecolor{currentstroke}%
\pgfsetstrokeopacity{0.700000}%
\pgfsetdash{}{0pt}%
\pgfpathmoveto{\pgfqpoint{4.684659in}{0.550000in}}%
\pgfpathlineto{\pgfqpoint{4.712180in}{0.550000in}}%
\pgfpathlineto{\pgfqpoint{4.712180in}{0.550000in}}%
\pgfpathlineto{\pgfqpoint{4.684659in}{0.550000in}}%
\pgfpathlineto{\pgfqpoint{4.684659in}{0.550000in}}%
\pgfpathclose%
\pgfusepath{fill}%
\end{pgfscope}%
\begin{pgfscope}%
\pgfpathrectangle{\pgfqpoint{1.250000in}{0.550000in}}{\pgfqpoint{7.750000in}{3.850000in}}%
\pgfusepath{clip}%
\pgfsetbuttcap%
\pgfsetmiterjoin%
\definecolor{currentfill}{rgb}{0.501961,0.501961,0.501961}%
\pgfsetfillcolor{currentfill}%
\pgfsetfillopacity{0.700000}%
\pgfsetlinewidth{0.000000pt}%
\definecolor{currentstroke}{rgb}{0.000000,0.000000,0.000000}%
\pgfsetstrokecolor{currentstroke}%
\pgfsetstrokeopacity{0.700000}%
\pgfsetdash{}{0pt}%
\pgfpathmoveto{\pgfqpoint{4.712180in}{0.550000in}}%
\pgfpathlineto{\pgfqpoint{4.739702in}{0.550000in}}%
\pgfpathlineto{\pgfqpoint{4.739702in}{1.021402in}}%
\pgfpathlineto{\pgfqpoint{4.712180in}{1.021402in}}%
\pgfpathlineto{\pgfqpoint{4.712180in}{0.550000in}}%
\pgfpathclose%
\pgfusepath{fill}%
\end{pgfscope}%
\begin{pgfscope}%
\pgfpathrectangle{\pgfqpoint{1.250000in}{0.550000in}}{\pgfqpoint{7.750000in}{3.850000in}}%
\pgfusepath{clip}%
\pgfsetbuttcap%
\pgfsetmiterjoin%
\definecolor{currentfill}{rgb}{0.501961,0.501961,0.501961}%
\pgfsetfillcolor{currentfill}%
\pgfsetfillopacity{0.700000}%
\pgfsetlinewidth{0.000000pt}%
\definecolor{currentstroke}{rgb}{0.000000,0.000000,0.000000}%
\pgfsetstrokecolor{currentstroke}%
\pgfsetstrokeopacity{0.700000}%
\pgfsetdash{}{0pt}%
\pgfpathmoveto{\pgfqpoint{4.739702in}{0.550000in}}%
\pgfpathlineto{\pgfqpoint{4.767223in}{0.550000in}}%
\pgfpathlineto{\pgfqpoint{4.767223in}{0.550000in}}%
\pgfpathlineto{\pgfqpoint{4.739702in}{0.550000in}}%
\pgfpathlineto{\pgfqpoint{4.739702in}{0.550000in}}%
\pgfpathclose%
\pgfusepath{fill}%
\end{pgfscope}%
\begin{pgfscope}%
\pgfpathrectangle{\pgfqpoint{1.250000in}{0.550000in}}{\pgfqpoint{7.750000in}{3.850000in}}%
\pgfusepath{clip}%
\pgfsetbuttcap%
\pgfsetmiterjoin%
\definecolor{currentfill}{rgb}{0.501961,0.501961,0.501961}%
\pgfsetfillcolor{currentfill}%
\pgfsetfillopacity{0.700000}%
\pgfsetlinewidth{0.000000pt}%
\definecolor{currentstroke}{rgb}{0.000000,0.000000,0.000000}%
\pgfsetstrokecolor{currentstroke}%
\pgfsetstrokeopacity{0.700000}%
\pgfsetdash{}{0pt}%
\pgfpathmoveto{\pgfqpoint{4.767223in}{0.550000in}}%
\pgfpathlineto{\pgfqpoint{4.794744in}{0.550000in}}%
\pgfpathlineto{\pgfqpoint{4.794744in}{0.550000in}}%
\pgfpathlineto{\pgfqpoint{4.767223in}{0.550000in}}%
\pgfpathlineto{\pgfqpoint{4.767223in}{0.550000in}}%
\pgfpathclose%
\pgfusepath{fill}%
\end{pgfscope}%
\begin{pgfscope}%
\pgfpathrectangle{\pgfqpoint{1.250000in}{0.550000in}}{\pgfqpoint{7.750000in}{3.850000in}}%
\pgfusepath{clip}%
\pgfsetbuttcap%
\pgfsetmiterjoin%
\definecolor{currentfill}{rgb}{0.501961,0.501961,0.501961}%
\pgfsetfillcolor{currentfill}%
\pgfsetfillopacity{0.700000}%
\pgfsetlinewidth{0.000000pt}%
\definecolor{currentstroke}{rgb}{0.000000,0.000000,0.000000}%
\pgfsetstrokecolor{currentstroke}%
\pgfsetstrokeopacity{0.700000}%
\pgfsetdash{}{0pt}%
\pgfpathmoveto{\pgfqpoint{4.794744in}{0.550000in}}%
\pgfpathlineto{\pgfqpoint{4.822266in}{0.550000in}}%
\pgfpathlineto{\pgfqpoint{4.822266in}{1.002986in}}%
\pgfpathlineto{\pgfqpoint{4.794744in}{1.002986in}}%
\pgfpathlineto{\pgfqpoint{4.794744in}{0.550000in}}%
\pgfpathclose%
\pgfusepath{fill}%
\end{pgfscope}%
\begin{pgfscope}%
\pgfpathrectangle{\pgfqpoint{1.250000in}{0.550000in}}{\pgfqpoint{7.750000in}{3.850000in}}%
\pgfusepath{clip}%
\pgfsetbuttcap%
\pgfsetmiterjoin%
\definecolor{currentfill}{rgb}{0.501961,0.501961,0.501961}%
\pgfsetfillcolor{currentfill}%
\pgfsetfillopacity{0.700000}%
\pgfsetlinewidth{0.000000pt}%
\definecolor{currentstroke}{rgb}{0.000000,0.000000,0.000000}%
\pgfsetstrokecolor{currentstroke}%
\pgfsetstrokeopacity{0.700000}%
\pgfsetdash{}{0pt}%
\pgfpathmoveto{\pgfqpoint{4.822266in}{0.550000in}}%
\pgfpathlineto{\pgfqpoint{4.849787in}{0.550000in}}%
\pgfpathlineto{\pgfqpoint{4.849787in}{0.550000in}}%
\pgfpathlineto{\pgfqpoint{4.822266in}{0.550000in}}%
\pgfpathlineto{\pgfqpoint{4.822266in}{0.550000in}}%
\pgfpathclose%
\pgfusepath{fill}%
\end{pgfscope}%
\begin{pgfscope}%
\pgfpathrectangle{\pgfqpoint{1.250000in}{0.550000in}}{\pgfqpoint{7.750000in}{3.850000in}}%
\pgfusepath{clip}%
\pgfsetbuttcap%
\pgfsetmiterjoin%
\definecolor{currentfill}{rgb}{0.501961,0.501961,0.501961}%
\pgfsetfillcolor{currentfill}%
\pgfsetfillopacity{0.700000}%
\pgfsetlinewidth{0.000000pt}%
\definecolor{currentstroke}{rgb}{0.000000,0.000000,0.000000}%
\pgfsetstrokecolor{currentstroke}%
\pgfsetstrokeopacity{0.700000}%
\pgfsetdash{}{0pt}%
\pgfpathmoveto{\pgfqpoint{4.849787in}{0.550000in}}%
\pgfpathlineto{\pgfqpoint{4.877308in}{0.550000in}}%
\pgfpathlineto{\pgfqpoint{4.877308in}{0.990516in}}%
\pgfpathlineto{\pgfqpoint{4.849787in}{0.990516in}}%
\pgfpathlineto{\pgfqpoint{4.849787in}{0.550000in}}%
\pgfpathclose%
\pgfusepath{fill}%
\end{pgfscope}%
\begin{pgfscope}%
\pgfpathrectangle{\pgfqpoint{1.250000in}{0.550000in}}{\pgfqpoint{7.750000in}{3.850000in}}%
\pgfusepath{clip}%
\pgfsetbuttcap%
\pgfsetmiterjoin%
\definecolor{currentfill}{rgb}{0.501961,0.501961,0.501961}%
\pgfsetfillcolor{currentfill}%
\pgfsetfillopacity{0.700000}%
\pgfsetlinewidth{0.000000pt}%
\definecolor{currentstroke}{rgb}{0.000000,0.000000,0.000000}%
\pgfsetstrokecolor{currentstroke}%
\pgfsetstrokeopacity{0.700000}%
\pgfsetdash{}{0pt}%
\pgfpathmoveto{\pgfqpoint{4.877308in}{0.550000in}}%
\pgfpathlineto{\pgfqpoint{4.904830in}{0.550000in}}%
\pgfpathlineto{\pgfqpoint{4.904830in}{0.550000in}}%
\pgfpathlineto{\pgfqpoint{4.877308in}{0.550000in}}%
\pgfpathlineto{\pgfqpoint{4.877308in}{0.550000in}}%
\pgfpathclose%
\pgfusepath{fill}%
\end{pgfscope}%
\begin{pgfscope}%
\pgfpathrectangle{\pgfqpoint{1.250000in}{0.550000in}}{\pgfqpoint{7.750000in}{3.850000in}}%
\pgfusepath{clip}%
\pgfsetbuttcap%
\pgfsetmiterjoin%
\definecolor{currentfill}{rgb}{0.501961,0.501961,0.501961}%
\pgfsetfillcolor{currentfill}%
\pgfsetfillopacity{0.700000}%
\pgfsetlinewidth{0.000000pt}%
\definecolor{currentstroke}{rgb}{0.000000,0.000000,0.000000}%
\pgfsetstrokecolor{currentstroke}%
\pgfsetstrokeopacity{0.700000}%
\pgfsetdash{}{0pt}%
\pgfpathmoveto{\pgfqpoint{4.904830in}{0.550000in}}%
\pgfpathlineto{\pgfqpoint{4.932351in}{0.550000in}}%
\pgfpathlineto{\pgfqpoint{4.932351in}{0.550000in}}%
\pgfpathlineto{\pgfqpoint{4.904830in}{0.550000in}}%
\pgfpathlineto{\pgfqpoint{4.904830in}{0.550000in}}%
\pgfpathclose%
\pgfusepath{fill}%
\end{pgfscope}%
\begin{pgfscope}%
\pgfpathrectangle{\pgfqpoint{1.250000in}{0.550000in}}{\pgfqpoint{7.750000in}{3.850000in}}%
\pgfusepath{clip}%
\pgfsetbuttcap%
\pgfsetmiterjoin%
\definecolor{currentfill}{rgb}{0.501961,0.501961,0.501961}%
\pgfsetfillcolor{currentfill}%
\pgfsetfillopacity{0.700000}%
\pgfsetlinewidth{0.000000pt}%
\definecolor{currentstroke}{rgb}{0.000000,0.000000,0.000000}%
\pgfsetstrokecolor{currentstroke}%
\pgfsetstrokeopacity{0.700000}%
\pgfsetdash{}{0pt}%
\pgfpathmoveto{\pgfqpoint{4.932351in}{0.550000in}}%
\pgfpathlineto{\pgfqpoint{4.959872in}{0.550000in}}%
\pgfpathlineto{\pgfqpoint{4.959872in}{0.968766in}}%
\pgfpathlineto{\pgfqpoint{4.932351in}{0.968766in}}%
\pgfpathlineto{\pgfqpoint{4.932351in}{0.550000in}}%
\pgfpathclose%
\pgfusepath{fill}%
\end{pgfscope}%
\begin{pgfscope}%
\pgfpathrectangle{\pgfqpoint{1.250000in}{0.550000in}}{\pgfqpoint{7.750000in}{3.850000in}}%
\pgfusepath{clip}%
\pgfsetbuttcap%
\pgfsetmiterjoin%
\definecolor{currentfill}{rgb}{0.501961,0.501961,0.501961}%
\pgfsetfillcolor{currentfill}%
\pgfsetfillopacity{0.700000}%
\pgfsetlinewidth{0.000000pt}%
\definecolor{currentstroke}{rgb}{0.000000,0.000000,0.000000}%
\pgfsetstrokecolor{currentstroke}%
\pgfsetstrokeopacity{0.700000}%
\pgfsetdash{}{0pt}%
\pgfpathmoveto{\pgfqpoint{4.959872in}{0.550000in}}%
\pgfpathlineto{\pgfqpoint{4.987393in}{0.550000in}}%
\pgfpathlineto{\pgfqpoint{4.987393in}{0.550000in}}%
\pgfpathlineto{\pgfqpoint{4.959872in}{0.550000in}}%
\pgfpathlineto{\pgfqpoint{4.959872in}{0.550000in}}%
\pgfpathclose%
\pgfusepath{fill}%
\end{pgfscope}%
\begin{pgfscope}%
\pgfpathrectangle{\pgfqpoint{1.250000in}{0.550000in}}{\pgfqpoint{7.750000in}{3.850000in}}%
\pgfusepath{clip}%
\pgfsetbuttcap%
\pgfsetmiterjoin%
\definecolor{currentfill}{rgb}{0.501961,0.501961,0.501961}%
\pgfsetfillcolor{currentfill}%
\pgfsetfillopacity{0.700000}%
\pgfsetlinewidth{0.000000pt}%
\definecolor{currentstroke}{rgb}{0.000000,0.000000,0.000000}%
\pgfsetstrokecolor{currentstroke}%
\pgfsetstrokeopacity{0.700000}%
\pgfsetdash{}{0pt}%
\pgfpathmoveto{\pgfqpoint{4.987393in}{0.550000in}}%
\pgfpathlineto{\pgfqpoint{5.014915in}{0.550000in}}%
\pgfpathlineto{\pgfqpoint{5.014915in}{0.550000in}}%
\pgfpathlineto{\pgfqpoint{4.987393in}{0.550000in}}%
\pgfpathlineto{\pgfqpoint{4.987393in}{0.550000in}}%
\pgfpathclose%
\pgfusepath{fill}%
\end{pgfscope}%
\begin{pgfscope}%
\pgfpathrectangle{\pgfqpoint{1.250000in}{0.550000in}}{\pgfqpoint{7.750000in}{3.850000in}}%
\pgfusepath{clip}%
\pgfsetbuttcap%
\pgfsetmiterjoin%
\definecolor{currentfill}{rgb}{0.501961,0.501961,0.501961}%
\pgfsetfillcolor{currentfill}%
\pgfsetfillopacity{0.700000}%
\pgfsetlinewidth{0.000000pt}%
\definecolor{currentstroke}{rgb}{0.000000,0.000000,0.000000}%
\pgfsetstrokecolor{currentstroke}%
\pgfsetstrokeopacity{0.700000}%
\pgfsetdash{}{0pt}%
\pgfpathmoveto{\pgfqpoint{5.014915in}{0.550000in}}%
\pgfpathlineto{\pgfqpoint{5.042436in}{0.550000in}}%
\pgfpathlineto{\pgfqpoint{5.042436in}{0.967461in}}%
\pgfpathlineto{\pgfqpoint{5.014915in}{0.967461in}}%
\pgfpathlineto{\pgfqpoint{5.014915in}{0.550000in}}%
\pgfpathclose%
\pgfusepath{fill}%
\end{pgfscope}%
\begin{pgfscope}%
\pgfpathrectangle{\pgfqpoint{1.250000in}{0.550000in}}{\pgfqpoint{7.750000in}{3.850000in}}%
\pgfusepath{clip}%
\pgfsetbuttcap%
\pgfsetmiterjoin%
\definecolor{currentfill}{rgb}{0.501961,0.501961,0.501961}%
\pgfsetfillcolor{currentfill}%
\pgfsetfillopacity{0.700000}%
\pgfsetlinewidth{0.000000pt}%
\definecolor{currentstroke}{rgb}{0.000000,0.000000,0.000000}%
\pgfsetstrokecolor{currentstroke}%
\pgfsetstrokeopacity{0.700000}%
\pgfsetdash{}{0pt}%
\pgfpathmoveto{\pgfqpoint{5.042436in}{0.550000in}}%
\pgfpathlineto{\pgfqpoint{5.069957in}{0.550000in}}%
\pgfpathlineto{\pgfqpoint{5.069957in}{0.550000in}}%
\pgfpathlineto{\pgfqpoint{5.042436in}{0.550000in}}%
\pgfpathlineto{\pgfqpoint{5.042436in}{0.550000in}}%
\pgfpathclose%
\pgfusepath{fill}%
\end{pgfscope}%
\begin{pgfscope}%
\pgfpathrectangle{\pgfqpoint{1.250000in}{0.550000in}}{\pgfqpoint{7.750000in}{3.850000in}}%
\pgfusepath{clip}%
\pgfsetbuttcap%
\pgfsetmiterjoin%
\definecolor{currentfill}{rgb}{0.501961,0.501961,0.501961}%
\pgfsetfillcolor{currentfill}%
\pgfsetfillopacity{0.700000}%
\pgfsetlinewidth{0.000000pt}%
\definecolor{currentstroke}{rgb}{0.000000,0.000000,0.000000}%
\pgfsetstrokecolor{currentstroke}%
\pgfsetstrokeopacity{0.700000}%
\pgfsetdash{}{0pt}%
\pgfpathmoveto{\pgfqpoint{5.069957in}{0.550000in}}%
\pgfpathlineto{\pgfqpoint{5.097479in}{0.550000in}}%
\pgfpathlineto{\pgfqpoint{5.097479in}{0.950206in}}%
\pgfpathlineto{\pgfqpoint{5.069957in}{0.950206in}}%
\pgfpathlineto{\pgfqpoint{5.069957in}{0.550000in}}%
\pgfpathclose%
\pgfusepath{fill}%
\end{pgfscope}%
\begin{pgfscope}%
\pgfpathrectangle{\pgfqpoint{1.250000in}{0.550000in}}{\pgfqpoint{7.750000in}{3.850000in}}%
\pgfusepath{clip}%
\pgfsetbuttcap%
\pgfsetmiterjoin%
\definecolor{currentfill}{rgb}{0.501961,0.501961,0.501961}%
\pgfsetfillcolor{currentfill}%
\pgfsetfillopacity{0.700000}%
\pgfsetlinewidth{0.000000pt}%
\definecolor{currentstroke}{rgb}{0.000000,0.000000,0.000000}%
\pgfsetstrokecolor{currentstroke}%
\pgfsetstrokeopacity{0.700000}%
\pgfsetdash{}{0pt}%
\pgfpathmoveto{\pgfqpoint{5.097479in}{0.550000in}}%
\pgfpathlineto{\pgfqpoint{5.125000in}{0.550000in}}%
\pgfpathlineto{\pgfqpoint{5.125000in}{0.550000in}}%
\pgfpathlineto{\pgfqpoint{5.097479in}{0.550000in}}%
\pgfpathlineto{\pgfqpoint{5.097479in}{0.550000in}}%
\pgfpathclose%
\pgfusepath{fill}%
\end{pgfscope}%
\begin{pgfscope}%
\pgfpathrectangle{\pgfqpoint{1.250000in}{0.550000in}}{\pgfqpoint{7.750000in}{3.850000in}}%
\pgfusepath{clip}%
\pgfsetbuttcap%
\pgfsetmiterjoin%
\definecolor{currentfill}{rgb}{0.501961,0.501961,0.501961}%
\pgfsetfillcolor{currentfill}%
\pgfsetfillopacity{0.700000}%
\pgfsetlinewidth{0.000000pt}%
\definecolor{currentstroke}{rgb}{0.000000,0.000000,0.000000}%
\pgfsetstrokecolor{currentstroke}%
\pgfsetstrokeopacity{0.700000}%
\pgfsetdash{}{0pt}%
\pgfpathmoveto{\pgfqpoint{5.125000in}{0.550000in}}%
\pgfpathlineto{\pgfqpoint{5.152521in}{0.550000in}}%
\pgfpathlineto{\pgfqpoint{5.152521in}{0.550000in}}%
\pgfpathlineto{\pgfqpoint{5.125000in}{0.550000in}}%
\pgfpathlineto{\pgfqpoint{5.125000in}{0.550000in}}%
\pgfpathclose%
\pgfusepath{fill}%
\end{pgfscope}%
\begin{pgfscope}%
\pgfpathrectangle{\pgfqpoint{1.250000in}{0.550000in}}{\pgfqpoint{7.750000in}{3.850000in}}%
\pgfusepath{clip}%
\pgfsetbuttcap%
\pgfsetmiterjoin%
\definecolor{currentfill}{rgb}{0.501961,0.501961,0.501961}%
\pgfsetfillcolor{currentfill}%
\pgfsetfillopacity{0.700000}%
\pgfsetlinewidth{0.000000pt}%
\definecolor{currentstroke}{rgb}{0.000000,0.000000,0.000000}%
\pgfsetstrokecolor{currentstroke}%
\pgfsetstrokeopacity{0.700000}%
\pgfsetdash{}{0pt}%
\pgfpathmoveto{\pgfqpoint{5.152521in}{0.550000in}}%
\pgfpathlineto{\pgfqpoint{5.180043in}{0.550000in}}%
\pgfpathlineto{\pgfqpoint{5.180043in}{0.950061in}}%
\pgfpathlineto{\pgfqpoint{5.152521in}{0.950061in}}%
\pgfpathlineto{\pgfqpoint{5.152521in}{0.550000in}}%
\pgfpathclose%
\pgfusepath{fill}%
\end{pgfscope}%
\begin{pgfscope}%
\pgfpathrectangle{\pgfqpoint{1.250000in}{0.550000in}}{\pgfqpoint{7.750000in}{3.850000in}}%
\pgfusepath{clip}%
\pgfsetbuttcap%
\pgfsetmiterjoin%
\definecolor{currentfill}{rgb}{0.501961,0.501961,0.501961}%
\pgfsetfillcolor{currentfill}%
\pgfsetfillopacity{0.700000}%
\pgfsetlinewidth{0.000000pt}%
\definecolor{currentstroke}{rgb}{0.000000,0.000000,0.000000}%
\pgfsetstrokecolor{currentstroke}%
\pgfsetstrokeopacity{0.700000}%
\pgfsetdash{}{0pt}%
\pgfpathmoveto{\pgfqpoint{5.180043in}{0.550000in}}%
\pgfpathlineto{\pgfqpoint{5.207564in}{0.550000in}}%
\pgfpathlineto{\pgfqpoint{5.207564in}{0.550000in}}%
\pgfpathlineto{\pgfqpoint{5.180043in}{0.550000in}}%
\pgfpathlineto{\pgfqpoint{5.180043in}{0.550000in}}%
\pgfpathclose%
\pgfusepath{fill}%
\end{pgfscope}%
\begin{pgfscope}%
\pgfpathrectangle{\pgfqpoint{1.250000in}{0.550000in}}{\pgfqpoint{7.750000in}{3.850000in}}%
\pgfusepath{clip}%
\pgfsetbuttcap%
\pgfsetmiterjoin%
\definecolor{currentfill}{rgb}{0.501961,0.501961,0.501961}%
\pgfsetfillcolor{currentfill}%
\pgfsetfillopacity{0.700000}%
\pgfsetlinewidth{0.000000pt}%
\definecolor{currentstroke}{rgb}{0.000000,0.000000,0.000000}%
\pgfsetstrokecolor{currentstroke}%
\pgfsetstrokeopacity{0.700000}%
\pgfsetdash{}{0pt}%
\pgfpathmoveto{\pgfqpoint{5.207564in}{0.550000in}}%
\pgfpathlineto{\pgfqpoint{5.235085in}{0.550000in}}%
\pgfpathlineto{\pgfqpoint{5.235085in}{0.932225in}}%
\pgfpathlineto{\pgfqpoint{5.207564in}{0.932225in}}%
\pgfpathlineto{\pgfqpoint{5.207564in}{0.550000in}}%
\pgfpathclose%
\pgfusepath{fill}%
\end{pgfscope}%
\begin{pgfscope}%
\pgfpathrectangle{\pgfqpoint{1.250000in}{0.550000in}}{\pgfqpoint{7.750000in}{3.850000in}}%
\pgfusepath{clip}%
\pgfsetbuttcap%
\pgfsetmiterjoin%
\definecolor{currentfill}{rgb}{0.501961,0.501961,0.501961}%
\pgfsetfillcolor{currentfill}%
\pgfsetfillopacity{0.700000}%
\pgfsetlinewidth{0.000000pt}%
\definecolor{currentstroke}{rgb}{0.000000,0.000000,0.000000}%
\pgfsetstrokecolor{currentstroke}%
\pgfsetstrokeopacity{0.700000}%
\pgfsetdash{}{0pt}%
\pgfpathmoveto{\pgfqpoint{5.235085in}{0.550000in}}%
\pgfpathlineto{\pgfqpoint{5.262607in}{0.550000in}}%
\pgfpathlineto{\pgfqpoint{5.262607in}{0.550000in}}%
\pgfpathlineto{\pgfqpoint{5.235085in}{0.550000in}}%
\pgfpathlineto{\pgfqpoint{5.235085in}{0.550000in}}%
\pgfpathclose%
\pgfusepath{fill}%
\end{pgfscope}%
\begin{pgfscope}%
\pgfpathrectangle{\pgfqpoint{1.250000in}{0.550000in}}{\pgfqpoint{7.750000in}{3.850000in}}%
\pgfusepath{clip}%
\pgfsetbuttcap%
\pgfsetmiterjoin%
\definecolor{currentfill}{rgb}{0.501961,0.501961,0.501961}%
\pgfsetfillcolor{currentfill}%
\pgfsetfillopacity{0.700000}%
\pgfsetlinewidth{0.000000pt}%
\definecolor{currentstroke}{rgb}{0.000000,0.000000,0.000000}%
\pgfsetstrokecolor{currentstroke}%
\pgfsetstrokeopacity{0.700000}%
\pgfsetdash{}{0pt}%
\pgfpathmoveto{\pgfqpoint{5.262607in}{0.550000in}}%
\pgfpathlineto{\pgfqpoint{5.290128in}{0.550000in}}%
\pgfpathlineto{\pgfqpoint{5.290128in}{0.928745in}}%
\pgfpathlineto{\pgfqpoint{5.262607in}{0.928745in}}%
\pgfpathlineto{\pgfqpoint{5.262607in}{0.550000in}}%
\pgfpathclose%
\pgfusepath{fill}%
\end{pgfscope}%
\begin{pgfscope}%
\pgfpathrectangle{\pgfqpoint{1.250000in}{0.550000in}}{\pgfqpoint{7.750000in}{3.850000in}}%
\pgfusepath{clip}%
\pgfsetbuttcap%
\pgfsetmiterjoin%
\definecolor{currentfill}{rgb}{0.501961,0.501961,0.501961}%
\pgfsetfillcolor{currentfill}%
\pgfsetfillopacity{0.700000}%
\pgfsetlinewidth{0.000000pt}%
\definecolor{currentstroke}{rgb}{0.000000,0.000000,0.000000}%
\pgfsetstrokecolor{currentstroke}%
\pgfsetstrokeopacity{0.700000}%
\pgfsetdash{}{0pt}%
\pgfpathmoveto{\pgfqpoint{5.290128in}{0.550000in}}%
\pgfpathlineto{\pgfqpoint{5.317649in}{0.550000in}}%
\pgfpathlineto{\pgfqpoint{5.317649in}{0.550000in}}%
\pgfpathlineto{\pgfqpoint{5.290128in}{0.550000in}}%
\pgfpathlineto{\pgfqpoint{5.290128in}{0.550000in}}%
\pgfpathclose%
\pgfusepath{fill}%
\end{pgfscope}%
\begin{pgfscope}%
\pgfpathrectangle{\pgfqpoint{1.250000in}{0.550000in}}{\pgfqpoint{7.750000in}{3.850000in}}%
\pgfusepath{clip}%
\pgfsetbuttcap%
\pgfsetmiterjoin%
\definecolor{currentfill}{rgb}{0.501961,0.501961,0.501961}%
\pgfsetfillcolor{currentfill}%
\pgfsetfillopacity{0.700000}%
\pgfsetlinewidth{0.000000pt}%
\definecolor{currentstroke}{rgb}{0.000000,0.000000,0.000000}%
\pgfsetstrokecolor{currentstroke}%
\pgfsetstrokeopacity{0.700000}%
\pgfsetdash{}{0pt}%
\pgfpathmoveto{\pgfqpoint{5.317649in}{0.550000in}}%
\pgfpathlineto{\pgfqpoint{5.345170in}{0.550000in}}%
\pgfpathlineto{\pgfqpoint{5.345170in}{0.550000in}}%
\pgfpathlineto{\pgfqpoint{5.317649in}{0.550000in}}%
\pgfpathlineto{\pgfqpoint{5.317649in}{0.550000in}}%
\pgfpathclose%
\pgfusepath{fill}%
\end{pgfscope}%
\begin{pgfscope}%
\pgfpathrectangle{\pgfqpoint{1.250000in}{0.550000in}}{\pgfqpoint{7.750000in}{3.850000in}}%
\pgfusepath{clip}%
\pgfsetbuttcap%
\pgfsetmiterjoin%
\definecolor{currentfill}{rgb}{0.501961,0.501961,0.501961}%
\pgfsetfillcolor{currentfill}%
\pgfsetfillopacity{0.700000}%
\pgfsetlinewidth{0.000000pt}%
\definecolor{currentstroke}{rgb}{0.000000,0.000000,0.000000}%
\pgfsetstrokecolor{currentstroke}%
\pgfsetstrokeopacity{0.700000}%
\pgfsetdash{}{0pt}%
\pgfpathmoveto{\pgfqpoint{5.345170in}{0.550000in}}%
\pgfpathlineto{\pgfqpoint{5.372692in}{0.550000in}}%
\pgfpathlineto{\pgfqpoint{5.372692in}{0.922365in}}%
\pgfpathlineto{\pgfqpoint{5.345170in}{0.922365in}}%
\pgfpathlineto{\pgfqpoint{5.345170in}{0.550000in}}%
\pgfpathclose%
\pgfusepath{fill}%
\end{pgfscope}%
\begin{pgfscope}%
\pgfpathrectangle{\pgfqpoint{1.250000in}{0.550000in}}{\pgfqpoint{7.750000in}{3.850000in}}%
\pgfusepath{clip}%
\pgfsetbuttcap%
\pgfsetmiterjoin%
\definecolor{currentfill}{rgb}{0.501961,0.501961,0.501961}%
\pgfsetfillcolor{currentfill}%
\pgfsetfillopacity{0.700000}%
\pgfsetlinewidth{0.000000pt}%
\definecolor{currentstroke}{rgb}{0.000000,0.000000,0.000000}%
\pgfsetstrokecolor{currentstroke}%
\pgfsetstrokeopacity{0.700000}%
\pgfsetdash{}{0pt}%
\pgfpathmoveto{\pgfqpoint{5.372692in}{0.550000in}}%
\pgfpathlineto{\pgfqpoint{5.400213in}{0.550000in}}%
\pgfpathlineto{\pgfqpoint{5.400213in}{0.550000in}}%
\pgfpathlineto{\pgfqpoint{5.372692in}{0.550000in}}%
\pgfpathlineto{\pgfqpoint{5.372692in}{0.550000in}}%
\pgfpathclose%
\pgfusepath{fill}%
\end{pgfscope}%
\begin{pgfscope}%
\pgfpathrectangle{\pgfqpoint{1.250000in}{0.550000in}}{\pgfqpoint{7.750000in}{3.850000in}}%
\pgfusepath{clip}%
\pgfsetbuttcap%
\pgfsetmiterjoin%
\definecolor{currentfill}{rgb}{0.501961,0.501961,0.501961}%
\pgfsetfillcolor{currentfill}%
\pgfsetfillopacity{0.700000}%
\pgfsetlinewidth{0.000000pt}%
\definecolor{currentstroke}{rgb}{0.000000,0.000000,0.000000}%
\pgfsetstrokecolor{currentstroke}%
\pgfsetstrokeopacity{0.700000}%
\pgfsetdash{}{0pt}%
\pgfpathmoveto{\pgfqpoint{5.400213in}{0.550000in}}%
\pgfpathlineto{\pgfqpoint{5.427734in}{0.550000in}}%
\pgfpathlineto{\pgfqpoint{5.427734in}{0.908445in}}%
\pgfpathlineto{\pgfqpoint{5.400213in}{0.908445in}}%
\pgfpathlineto{\pgfqpoint{5.400213in}{0.550000in}}%
\pgfpathclose%
\pgfusepath{fill}%
\end{pgfscope}%
\begin{pgfscope}%
\pgfpathrectangle{\pgfqpoint{1.250000in}{0.550000in}}{\pgfqpoint{7.750000in}{3.850000in}}%
\pgfusepath{clip}%
\pgfsetbuttcap%
\pgfsetmiterjoin%
\definecolor{currentfill}{rgb}{0.501961,0.501961,0.501961}%
\pgfsetfillcolor{currentfill}%
\pgfsetfillopacity{0.700000}%
\pgfsetlinewidth{0.000000pt}%
\definecolor{currentstroke}{rgb}{0.000000,0.000000,0.000000}%
\pgfsetstrokecolor{currentstroke}%
\pgfsetstrokeopacity{0.700000}%
\pgfsetdash{}{0pt}%
\pgfpathmoveto{\pgfqpoint{5.427734in}{0.550000in}}%
\pgfpathlineto{\pgfqpoint{5.455256in}{0.550000in}}%
\pgfpathlineto{\pgfqpoint{5.455256in}{0.550000in}}%
\pgfpathlineto{\pgfqpoint{5.427734in}{0.550000in}}%
\pgfpathlineto{\pgfqpoint{5.427734in}{0.550000in}}%
\pgfpathclose%
\pgfusepath{fill}%
\end{pgfscope}%
\begin{pgfscope}%
\pgfpathrectangle{\pgfqpoint{1.250000in}{0.550000in}}{\pgfqpoint{7.750000in}{3.850000in}}%
\pgfusepath{clip}%
\pgfsetbuttcap%
\pgfsetmiterjoin%
\definecolor{currentfill}{rgb}{0.501961,0.501961,0.501961}%
\pgfsetfillcolor{currentfill}%
\pgfsetfillopacity{0.700000}%
\pgfsetlinewidth{0.000000pt}%
\definecolor{currentstroke}{rgb}{0.000000,0.000000,0.000000}%
\pgfsetstrokecolor{currentstroke}%
\pgfsetstrokeopacity{0.700000}%
\pgfsetdash{}{0pt}%
\pgfpathmoveto{\pgfqpoint{5.455256in}{0.550000in}}%
\pgfpathlineto{\pgfqpoint{5.482777in}{0.550000in}}%
\pgfpathlineto{\pgfqpoint{5.482777in}{0.896700in}}%
\pgfpathlineto{\pgfqpoint{5.455256in}{0.896700in}}%
\pgfpathlineto{\pgfqpoint{5.455256in}{0.550000in}}%
\pgfpathclose%
\pgfusepath{fill}%
\end{pgfscope}%
\begin{pgfscope}%
\pgfpathrectangle{\pgfqpoint{1.250000in}{0.550000in}}{\pgfqpoint{7.750000in}{3.850000in}}%
\pgfusepath{clip}%
\pgfsetbuttcap%
\pgfsetmiterjoin%
\definecolor{currentfill}{rgb}{0.501961,0.501961,0.501961}%
\pgfsetfillcolor{currentfill}%
\pgfsetfillopacity{0.700000}%
\pgfsetlinewidth{0.000000pt}%
\definecolor{currentstroke}{rgb}{0.000000,0.000000,0.000000}%
\pgfsetstrokecolor{currentstroke}%
\pgfsetstrokeopacity{0.700000}%
\pgfsetdash{}{0pt}%
\pgfpathmoveto{\pgfqpoint{5.482777in}{0.550000in}}%
\pgfpathlineto{\pgfqpoint{5.510298in}{0.550000in}}%
\pgfpathlineto{\pgfqpoint{5.510298in}{0.550000in}}%
\pgfpathlineto{\pgfqpoint{5.482777in}{0.550000in}}%
\pgfpathlineto{\pgfqpoint{5.482777in}{0.550000in}}%
\pgfpathclose%
\pgfusepath{fill}%
\end{pgfscope}%
\begin{pgfscope}%
\pgfpathrectangle{\pgfqpoint{1.250000in}{0.550000in}}{\pgfqpoint{7.750000in}{3.850000in}}%
\pgfusepath{clip}%
\pgfsetbuttcap%
\pgfsetmiterjoin%
\definecolor{currentfill}{rgb}{0.501961,0.501961,0.501961}%
\pgfsetfillcolor{currentfill}%
\pgfsetfillopacity{0.700000}%
\pgfsetlinewidth{0.000000pt}%
\definecolor{currentstroke}{rgb}{0.000000,0.000000,0.000000}%
\pgfsetstrokecolor{currentstroke}%
\pgfsetstrokeopacity{0.700000}%
\pgfsetdash{}{0pt}%
\pgfpathmoveto{\pgfqpoint{5.510298in}{0.550000in}}%
\pgfpathlineto{\pgfqpoint{5.537820in}{0.550000in}}%
\pgfpathlineto{\pgfqpoint{5.537820in}{0.883360in}}%
\pgfpathlineto{\pgfqpoint{5.510298in}{0.883360in}}%
\pgfpathlineto{\pgfqpoint{5.510298in}{0.550000in}}%
\pgfpathclose%
\pgfusepath{fill}%
\end{pgfscope}%
\begin{pgfscope}%
\pgfpathrectangle{\pgfqpoint{1.250000in}{0.550000in}}{\pgfqpoint{7.750000in}{3.850000in}}%
\pgfusepath{clip}%
\pgfsetbuttcap%
\pgfsetmiterjoin%
\definecolor{currentfill}{rgb}{0.501961,0.501961,0.501961}%
\pgfsetfillcolor{currentfill}%
\pgfsetfillopacity{0.700000}%
\pgfsetlinewidth{0.000000pt}%
\definecolor{currentstroke}{rgb}{0.000000,0.000000,0.000000}%
\pgfsetstrokecolor{currentstroke}%
\pgfsetstrokeopacity{0.700000}%
\pgfsetdash{}{0pt}%
\pgfpathmoveto{\pgfqpoint{5.537820in}{0.550000in}}%
\pgfpathlineto{\pgfqpoint{5.565341in}{0.550000in}}%
\pgfpathlineto{\pgfqpoint{5.565341in}{0.550000in}}%
\pgfpathlineto{\pgfqpoint{5.537820in}{0.550000in}}%
\pgfpathlineto{\pgfqpoint{5.537820in}{0.550000in}}%
\pgfpathclose%
\pgfusepath{fill}%
\end{pgfscope}%
\begin{pgfscope}%
\pgfpathrectangle{\pgfqpoint{1.250000in}{0.550000in}}{\pgfqpoint{7.750000in}{3.850000in}}%
\pgfusepath{clip}%
\pgfsetbuttcap%
\pgfsetmiterjoin%
\definecolor{currentfill}{rgb}{0.501961,0.501961,0.501961}%
\pgfsetfillcolor{currentfill}%
\pgfsetfillopacity{0.700000}%
\pgfsetlinewidth{0.000000pt}%
\definecolor{currentstroke}{rgb}{0.000000,0.000000,0.000000}%
\pgfsetstrokecolor{currentstroke}%
\pgfsetstrokeopacity{0.700000}%
\pgfsetdash{}{0pt}%
\pgfpathmoveto{\pgfqpoint{5.565341in}{0.550000in}}%
\pgfpathlineto{\pgfqpoint{5.592862in}{0.550000in}}%
\pgfpathlineto{\pgfqpoint{5.592862in}{0.869149in}}%
\pgfpathlineto{\pgfqpoint{5.565341in}{0.869149in}}%
\pgfpathlineto{\pgfqpoint{5.565341in}{0.550000in}}%
\pgfpathclose%
\pgfusepath{fill}%
\end{pgfscope}%
\begin{pgfscope}%
\pgfpathrectangle{\pgfqpoint{1.250000in}{0.550000in}}{\pgfqpoint{7.750000in}{3.850000in}}%
\pgfusepath{clip}%
\pgfsetbuttcap%
\pgfsetmiterjoin%
\definecolor{currentfill}{rgb}{0.501961,0.501961,0.501961}%
\pgfsetfillcolor{currentfill}%
\pgfsetfillopacity{0.700000}%
\pgfsetlinewidth{0.000000pt}%
\definecolor{currentstroke}{rgb}{0.000000,0.000000,0.000000}%
\pgfsetstrokecolor{currentstroke}%
\pgfsetstrokeopacity{0.700000}%
\pgfsetdash{}{0pt}%
\pgfpathmoveto{\pgfqpoint{5.592862in}{0.550000in}}%
\pgfpathlineto{\pgfqpoint{5.620384in}{0.550000in}}%
\pgfpathlineto{\pgfqpoint{5.620384in}{0.550000in}}%
\pgfpathlineto{\pgfqpoint{5.592862in}{0.550000in}}%
\pgfpathlineto{\pgfqpoint{5.592862in}{0.550000in}}%
\pgfpathclose%
\pgfusepath{fill}%
\end{pgfscope}%
\begin{pgfscope}%
\pgfpathrectangle{\pgfqpoint{1.250000in}{0.550000in}}{\pgfqpoint{7.750000in}{3.850000in}}%
\pgfusepath{clip}%
\pgfsetbuttcap%
\pgfsetmiterjoin%
\definecolor{currentfill}{rgb}{0.501961,0.501961,0.501961}%
\pgfsetfillcolor{currentfill}%
\pgfsetfillopacity{0.700000}%
\pgfsetlinewidth{0.000000pt}%
\definecolor{currentstroke}{rgb}{0.000000,0.000000,0.000000}%
\pgfsetstrokecolor{currentstroke}%
\pgfsetstrokeopacity{0.700000}%
\pgfsetdash{}{0pt}%
\pgfpathmoveto{\pgfqpoint{5.620384in}{0.550000in}}%
\pgfpathlineto{\pgfqpoint{5.647905in}{0.550000in}}%
\pgfpathlineto{\pgfqpoint{5.647905in}{0.854794in}}%
\pgfpathlineto{\pgfqpoint{5.620384in}{0.854794in}}%
\pgfpathlineto{\pgfqpoint{5.620384in}{0.550000in}}%
\pgfpathclose%
\pgfusepath{fill}%
\end{pgfscope}%
\begin{pgfscope}%
\pgfpathrectangle{\pgfqpoint{1.250000in}{0.550000in}}{\pgfqpoint{7.750000in}{3.850000in}}%
\pgfusepath{clip}%
\pgfsetbuttcap%
\pgfsetmiterjoin%
\definecolor{currentfill}{rgb}{0.501961,0.501961,0.501961}%
\pgfsetfillcolor{currentfill}%
\pgfsetfillopacity{0.700000}%
\pgfsetlinewidth{0.000000pt}%
\definecolor{currentstroke}{rgb}{0.000000,0.000000,0.000000}%
\pgfsetstrokecolor{currentstroke}%
\pgfsetstrokeopacity{0.700000}%
\pgfsetdash{}{0pt}%
\pgfpathmoveto{\pgfqpoint{5.647905in}{0.550000in}}%
\pgfpathlineto{\pgfqpoint{5.675426in}{0.550000in}}%
\pgfpathlineto{\pgfqpoint{5.675426in}{0.550000in}}%
\pgfpathlineto{\pgfqpoint{5.647905in}{0.550000in}}%
\pgfpathlineto{\pgfqpoint{5.647905in}{0.550000in}}%
\pgfpathclose%
\pgfusepath{fill}%
\end{pgfscope}%
\begin{pgfscope}%
\pgfpathrectangle{\pgfqpoint{1.250000in}{0.550000in}}{\pgfqpoint{7.750000in}{3.850000in}}%
\pgfusepath{clip}%
\pgfsetbuttcap%
\pgfsetmiterjoin%
\definecolor{currentfill}{rgb}{0.501961,0.501961,0.501961}%
\pgfsetfillcolor{currentfill}%
\pgfsetfillopacity{0.700000}%
\pgfsetlinewidth{0.000000pt}%
\definecolor{currentstroke}{rgb}{0.000000,0.000000,0.000000}%
\pgfsetstrokecolor{currentstroke}%
\pgfsetstrokeopacity{0.700000}%
\pgfsetdash{}{0pt}%
\pgfpathmoveto{\pgfqpoint{5.675426in}{0.550000in}}%
\pgfpathlineto{\pgfqpoint{5.702947in}{0.550000in}}%
\pgfpathlineto{\pgfqpoint{5.702947in}{0.838264in}}%
\pgfpathlineto{\pgfqpoint{5.675426in}{0.838264in}}%
\pgfpathlineto{\pgfqpoint{5.675426in}{0.550000in}}%
\pgfpathclose%
\pgfusepath{fill}%
\end{pgfscope}%
\begin{pgfscope}%
\pgfpathrectangle{\pgfqpoint{1.250000in}{0.550000in}}{\pgfqpoint{7.750000in}{3.850000in}}%
\pgfusepath{clip}%
\pgfsetbuttcap%
\pgfsetmiterjoin%
\definecolor{currentfill}{rgb}{0.501961,0.501961,0.501961}%
\pgfsetfillcolor{currentfill}%
\pgfsetfillopacity{0.700000}%
\pgfsetlinewidth{0.000000pt}%
\definecolor{currentstroke}{rgb}{0.000000,0.000000,0.000000}%
\pgfsetstrokecolor{currentstroke}%
\pgfsetstrokeopacity{0.700000}%
\pgfsetdash{}{0pt}%
\pgfpathmoveto{\pgfqpoint{5.702947in}{0.550000in}}%
\pgfpathlineto{\pgfqpoint{5.730469in}{0.550000in}}%
\pgfpathlineto{\pgfqpoint{5.730469in}{0.550000in}}%
\pgfpathlineto{\pgfqpoint{5.702947in}{0.550000in}}%
\pgfpathlineto{\pgfqpoint{5.702947in}{0.550000in}}%
\pgfpathclose%
\pgfusepath{fill}%
\end{pgfscope}%
\begin{pgfscope}%
\pgfpathrectangle{\pgfqpoint{1.250000in}{0.550000in}}{\pgfqpoint{7.750000in}{3.850000in}}%
\pgfusepath{clip}%
\pgfsetbuttcap%
\pgfsetmiterjoin%
\definecolor{currentfill}{rgb}{0.501961,0.501961,0.501961}%
\pgfsetfillcolor{currentfill}%
\pgfsetfillopacity{0.700000}%
\pgfsetlinewidth{0.000000pt}%
\definecolor{currentstroke}{rgb}{0.000000,0.000000,0.000000}%
\pgfsetstrokecolor{currentstroke}%
\pgfsetstrokeopacity{0.700000}%
\pgfsetdash{}{0pt}%
\pgfpathmoveto{\pgfqpoint{5.730469in}{0.550000in}}%
\pgfpathlineto{\pgfqpoint{5.757990in}{0.550000in}}%
\pgfpathlineto{\pgfqpoint{5.757990in}{0.842469in}}%
\pgfpathlineto{\pgfqpoint{5.730469in}{0.842469in}}%
\pgfpathlineto{\pgfqpoint{5.730469in}{0.550000in}}%
\pgfpathclose%
\pgfusepath{fill}%
\end{pgfscope}%
\begin{pgfscope}%
\pgfpathrectangle{\pgfqpoint{1.250000in}{0.550000in}}{\pgfqpoint{7.750000in}{3.850000in}}%
\pgfusepath{clip}%
\pgfsetbuttcap%
\pgfsetmiterjoin%
\definecolor{currentfill}{rgb}{0.501961,0.501961,0.501961}%
\pgfsetfillcolor{currentfill}%
\pgfsetfillopacity{0.700000}%
\pgfsetlinewidth{0.000000pt}%
\definecolor{currentstroke}{rgb}{0.000000,0.000000,0.000000}%
\pgfsetstrokecolor{currentstroke}%
\pgfsetstrokeopacity{0.700000}%
\pgfsetdash{}{0pt}%
\pgfpathmoveto{\pgfqpoint{5.757990in}{0.550000in}}%
\pgfpathlineto{\pgfqpoint{5.785511in}{0.550000in}}%
\pgfpathlineto{\pgfqpoint{5.785511in}{0.823764in}}%
\pgfpathlineto{\pgfqpoint{5.757990in}{0.823764in}}%
\pgfpathlineto{\pgfqpoint{5.757990in}{0.550000in}}%
\pgfpathclose%
\pgfusepath{fill}%
\end{pgfscope}%
\begin{pgfscope}%
\pgfpathrectangle{\pgfqpoint{1.250000in}{0.550000in}}{\pgfqpoint{7.750000in}{3.850000in}}%
\pgfusepath{clip}%
\pgfsetbuttcap%
\pgfsetmiterjoin%
\definecolor{currentfill}{rgb}{0.501961,0.501961,0.501961}%
\pgfsetfillcolor{currentfill}%
\pgfsetfillopacity{0.700000}%
\pgfsetlinewidth{0.000000pt}%
\definecolor{currentstroke}{rgb}{0.000000,0.000000,0.000000}%
\pgfsetstrokecolor{currentstroke}%
\pgfsetstrokeopacity{0.700000}%
\pgfsetdash{}{0pt}%
\pgfpathmoveto{\pgfqpoint{5.785511in}{0.550000in}}%
\pgfpathlineto{\pgfqpoint{5.813033in}{0.550000in}}%
\pgfpathlineto{\pgfqpoint{5.813033in}{0.550000in}}%
\pgfpathlineto{\pgfqpoint{5.785511in}{0.550000in}}%
\pgfpathlineto{\pgfqpoint{5.785511in}{0.550000in}}%
\pgfpathclose%
\pgfusepath{fill}%
\end{pgfscope}%
\begin{pgfscope}%
\pgfpathrectangle{\pgfqpoint{1.250000in}{0.550000in}}{\pgfqpoint{7.750000in}{3.850000in}}%
\pgfusepath{clip}%
\pgfsetbuttcap%
\pgfsetmiterjoin%
\definecolor{currentfill}{rgb}{0.501961,0.501961,0.501961}%
\pgfsetfillcolor{currentfill}%
\pgfsetfillopacity{0.700000}%
\pgfsetlinewidth{0.000000pt}%
\definecolor{currentstroke}{rgb}{0.000000,0.000000,0.000000}%
\pgfsetstrokecolor{currentstroke}%
\pgfsetstrokeopacity{0.700000}%
\pgfsetdash{}{0pt}%
\pgfpathmoveto{\pgfqpoint{5.813033in}{0.550000in}}%
\pgfpathlineto{\pgfqpoint{5.840554in}{0.550000in}}%
\pgfpathlineto{\pgfqpoint{5.840554in}{0.816949in}}%
\pgfpathlineto{\pgfqpoint{5.813033in}{0.816949in}}%
\pgfpathlineto{\pgfqpoint{5.813033in}{0.550000in}}%
\pgfpathclose%
\pgfusepath{fill}%
\end{pgfscope}%
\begin{pgfscope}%
\pgfpathrectangle{\pgfqpoint{1.250000in}{0.550000in}}{\pgfqpoint{7.750000in}{3.850000in}}%
\pgfusepath{clip}%
\pgfsetbuttcap%
\pgfsetmiterjoin%
\definecolor{currentfill}{rgb}{0.501961,0.501961,0.501961}%
\pgfsetfillcolor{currentfill}%
\pgfsetfillopacity{0.700000}%
\pgfsetlinewidth{0.000000pt}%
\definecolor{currentstroke}{rgb}{0.000000,0.000000,0.000000}%
\pgfsetstrokecolor{currentstroke}%
\pgfsetstrokeopacity{0.700000}%
\pgfsetdash{}{0pt}%
\pgfpathmoveto{\pgfqpoint{5.840554in}{0.550000in}}%
\pgfpathlineto{\pgfqpoint{5.868075in}{0.550000in}}%
\pgfpathlineto{\pgfqpoint{5.868075in}{0.550000in}}%
\pgfpathlineto{\pgfqpoint{5.840554in}{0.550000in}}%
\pgfpathlineto{\pgfqpoint{5.840554in}{0.550000in}}%
\pgfpathclose%
\pgfusepath{fill}%
\end{pgfscope}%
\begin{pgfscope}%
\pgfpathrectangle{\pgfqpoint{1.250000in}{0.550000in}}{\pgfqpoint{7.750000in}{3.850000in}}%
\pgfusepath{clip}%
\pgfsetbuttcap%
\pgfsetmiterjoin%
\definecolor{currentfill}{rgb}{0.501961,0.501961,0.501961}%
\pgfsetfillcolor{currentfill}%
\pgfsetfillopacity{0.700000}%
\pgfsetlinewidth{0.000000pt}%
\definecolor{currentstroke}{rgb}{0.000000,0.000000,0.000000}%
\pgfsetstrokecolor{currentstroke}%
\pgfsetstrokeopacity{0.700000}%
\pgfsetdash{}{0pt}%
\pgfpathmoveto{\pgfqpoint{5.868075in}{0.550000in}}%
\pgfpathlineto{\pgfqpoint{5.895597in}{0.550000in}}%
\pgfpathlineto{\pgfqpoint{5.895597in}{0.804914in}}%
\pgfpathlineto{\pgfqpoint{5.868075in}{0.804914in}}%
\pgfpathlineto{\pgfqpoint{5.868075in}{0.550000in}}%
\pgfpathclose%
\pgfusepath{fill}%
\end{pgfscope}%
\begin{pgfscope}%
\pgfpathrectangle{\pgfqpoint{1.250000in}{0.550000in}}{\pgfqpoint{7.750000in}{3.850000in}}%
\pgfusepath{clip}%
\pgfsetbuttcap%
\pgfsetmiterjoin%
\definecolor{currentfill}{rgb}{0.501961,0.501961,0.501961}%
\pgfsetfillcolor{currentfill}%
\pgfsetfillopacity{0.700000}%
\pgfsetlinewidth{0.000000pt}%
\definecolor{currentstroke}{rgb}{0.000000,0.000000,0.000000}%
\pgfsetstrokecolor{currentstroke}%
\pgfsetstrokeopacity{0.700000}%
\pgfsetdash{}{0pt}%
\pgfpathmoveto{\pgfqpoint{5.895597in}{0.550000in}}%
\pgfpathlineto{\pgfqpoint{5.923118in}{0.550000in}}%
\pgfpathlineto{\pgfqpoint{5.923118in}{0.794763in}}%
\pgfpathlineto{\pgfqpoint{5.895597in}{0.794763in}}%
\pgfpathlineto{\pgfqpoint{5.895597in}{0.550000in}}%
\pgfpathclose%
\pgfusepath{fill}%
\end{pgfscope}%
\begin{pgfscope}%
\pgfpathrectangle{\pgfqpoint{1.250000in}{0.550000in}}{\pgfqpoint{7.750000in}{3.850000in}}%
\pgfusepath{clip}%
\pgfsetbuttcap%
\pgfsetmiterjoin%
\definecolor{currentfill}{rgb}{0.501961,0.501961,0.501961}%
\pgfsetfillcolor{currentfill}%
\pgfsetfillopacity{0.700000}%
\pgfsetlinewidth{0.000000pt}%
\definecolor{currentstroke}{rgb}{0.000000,0.000000,0.000000}%
\pgfsetstrokecolor{currentstroke}%
\pgfsetstrokeopacity{0.700000}%
\pgfsetdash{}{0pt}%
\pgfpathmoveto{\pgfqpoint{5.923118in}{0.550000in}}%
\pgfpathlineto{\pgfqpoint{5.950639in}{0.550000in}}%
\pgfpathlineto{\pgfqpoint{5.950639in}{0.550000in}}%
\pgfpathlineto{\pgfqpoint{5.923118in}{0.550000in}}%
\pgfpathlineto{\pgfqpoint{5.923118in}{0.550000in}}%
\pgfpathclose%
\pgfusepath{fill}%
\end{pgfscope}%
\begin{pgfscope}%
\pgfpathrectangle{\pgfqpoint{1.250000in}{0.550000in}}{\pgfqpoint{7.750000in}{3.850000in}}%
\pgfusepath{clip}%
\pgfsetbuttcap%
\pgfsetmiterjoin%
\definecolor{currentfill}{rgb}{0.501961,0.501961,0.501961}%
\pgfsetfillcolor{currentfill}%
\pgfsetfillopacity{0.700000}%
\pgfsetlinewidth{0.000000pt}%
\definecolor{currentstroke}{rgb}{0.000000,0.000000,0.000000}%
\pgfsetstrokecolor{currentstroke}%
\pgfsetstrokeopacity{0.700000}%
\pgfsetdash{}{0pt}%
\pgfpathmoveto{\pgfqpoint{5.950639in}{0.550000in}}%
\pgfpathlineto{\pgfqpoint{5.978161in}{0.550000in}}%
\pgfpathlineto{\pgfqpoint{5.978161in}{0.796503in}}%
\pgfpathlineto{\pgfqpoint{5.950639in}{0.796503in}}%
\pgfpathlineto{\pgfqpoint{5.950639in}{0.550000in}}%
\pgfpathclose%
\pgfusepath{fill}%
\end{pgfscope}%
\begin{pgfscope}%
\pgfpathrectangle{\pgfqpoint{1.250000in}{0.550000in}}{\pgfqpoint{7.750000in}{3.850000in}}%
\pgfusepath{clip}%
\pgfsetbuttcap%
\pgfsetmiterjoin%
\definecolor{currentfill}{rgb}{0.501961,0.501961,0.501961}%
\pgfsetfillcolor{currentfill}%
\pgfsetfillopacity{0.700000}%
\pgfsetlinewidth{0.000000pt}%
\definecolor{currentstroke}{rgb}{0.000000,0.000000,0.000000}%
\pgfsetstrokecolor{currentstroke}%
\pgfsetstrokeopacity{0.700000}%
\pgfsetdash{}{0pt}%
\pgfpathmoveto{\pgfqpoint{5.978161in}{0.550000in}}%
\pgfpathlineto{\pgfqpoint{6.005682in}{0.550000in}}%
\pgfpathlineto{\pgfqpoint{6.005682in}{0.782728in}}%
\pgfpathlineto{\pgfqpoint{5.978161in}{0.782728in}}%
\pgfpathlineto{\pgfqpoint{5.978161in}{0.550000in}}%
\pgfpathclose%
\pgfusepath{fill}%
\end{pgfscope}%
\begin{pgfscope}%
\pgfpathrectangle{\pgfqpoint{1.250000in}{0.550000in}}{\pgfqpoint{7.750000in}{3.850000in}}%
\pgfusepath{clip}%
\pgfsetbuttcap%
\pgfsetmiterjoin%
\definecolor{currentfill}{rgb}{0.501961,0.501961,0.501961}%
\pgfsetfillcolor{currentfill}%
\pgfsetfillopacity{0.700000}%
\pgfsetlinewidth{0.000000pt}%
\definecolor{currentstroke}{rgb}{0.000000,0.000000,0.000000}%
\pgfsetstrokecolor{currentstroke}%
\pgfsetstrokeopacity{0.700000}%
\pgfsetdash{}{0pt}%
\pgfpathmoveto{\pgfqpoint{6.005682in}{0.550000in}}%
\pgfpathlineto{\pgfqpoint{6.033203in}{0.550000in}}%
\pgfpathlineto{\pgfqpoint{6.033203in}{0.550000in}}%
\pgfpathlineto{\pgfqpoint{6.005682in}{0.550000in}}%
\pgfpathlineto{\pgfqpoint{6.005682in}{0.550000in}}%
\pgfpathclose%
\pgfusepath{fill}%
\end{pgfscope}%
\begin{pgfscope}%
\pgfpathrectangle{\pgfqpoint{1.250000in}{0.550000in}}{\pgfqpoint{7.750000in}{3.850000in}}%
\pgfusepath{clip}%
\pgfsetbuttcap%
\pgfsetmiterjoin%
\definecolor{currentfill}{rgb}{0.501961,0.501961,0.501961}%
\pgfsetfillcolor{currentfill}%
\pgfsetfillopacity{0.700000}%
\pgfsetlinewidth{0.000000pt}%
\definecolor{currentstroke}{rgb}{0.000000,0.000000,0.000000}%
\pgfsetstrokecolor{currentstroke}%
\pgfsetstrokeopacity{0.700000}%
\pgfsetdash{}{0pt}%
\pgfpathmoveto{\pgfqpoint{6.033203in}{0.550000in}}%
\pgfpathlineto{\pgfqpoint{6.060724in}{0.550000in}}%
\pgfpathlineto{\pgfqpoint{6.060724in}{0.779393in}}%
\pgfpathlineto{\pgfqpoint{6.033203in}{0.779393in}}%
\pgfpathlineto{\pgfqpoint{6.033203in}{0.550000in}}%
\pgfpathclose%
\pgfusepath{fill}%
\end{pgfscope}%
\begin{pgfscope}%
\pgfpathrectangle{\pgfqpoint{1.250000in}{0.550000in}}{\pgfqpoint{7.750000in}{3.850000in}}%
\pgfusepath{clip}%
\pgfsetbuttcap%
\pgfsetmiterjoin%
\definecolor{currentfill}{rgb}{0.501961,0.501961,0.501961}%
\pgfsetfillcolor{currentfill}%
\pgfsetfillopacity{0.700000}%
\pgfsetlinewidth{0.000000pt}%
\definecolor{currentstroke}{rgb}{0.000000,0.000000,0.000000}%
\pgfsetstrokecolor{currentstroke}%
\pgfsetstrokeopacity{0.700000}%
\pgfsetdash{}{0pt}%
\pgfpathmoveto{\pgfqpoint{6.060724in}{0.550000in}}%
\pgfpathlineto{\pgfqpoint{6.088246in}{0.550000in}}%
\pgfpathlineto{\pgfqpoint{6.088246in}{0.763733in}}%
\pgfpathlineto{\pgfqpoint{6.060724in}{0.763733in}}%
\pgfpathlineto{\pgfqpoint{6.060724in}{0.550000in}}%
\pgfpathclose%
\pgfusepath{fill}%
\end{pgfscope}%
\begin{pgfscope}%
\pgfpathrectangle{\pgfqpoint{1.250000in}{0.550000in}}{\pgfqpoint{7.750000in}{3.850000in}}%
\pgfusepath{clip}%
\pgfsetbuttcap%
\pgfsetmiterjoin%
\definecolor{currentfill}{rgb}{0.501961,0.501961,0.501961}%
\pgfsetfillcolor{currentfill}%
\pgfsetfillopacity{0.700000}%
\pgfsetlinewidth{0.000000pt}%
\definecolor{currentstroke}{rgb}{0.000000,0.000000,0.000000}%
\pgfsetstrokecolor{currentstroke}%
\pgfsetstrokeopacity{0.700000}%
\pgfsetdash{}{0pt}%
\pgfpathmoveto{\pgfqpoint{6.088246in}{0.550000in}}%
\pgfpathlineto{\pgfqpoint{6.115767in}{0.550000in}}%
\pgfpathlineto{\pgfqpoint{6.115767in}{0.762573in}}%
\pgfpathlineto{\pgfqpoint{6.088246in}{0.762573in}}%
\pgfpathlineto{\pgfqpoint{6.088246in}{0.550000in}}%
\pgfpathclose%
\pgfusepath{fill}%
\end{pgfscope}%
\begin{pgfscope}%
\pgfpathrectangle{\pgfqpoint{1.250000in}{0.550000in}}{\pgfqpoint{7.750000in}{3.850000in}}%
\pgfusepath{clip}%
\pgfsetbuttcap%
\pgfsetmiterjoin%
\definecolor{currentfill}{rgb}{0.501961,0.501961,0.501961}%
\pgfsetfillcolor{currentfill}%
\pgfsetfillopacity{0.700000}%
\pgfsetlinewidth{0.000000pt}%
\definecolor{currentstroke}{rgb}{0.000000,0.000000,0.000000}%
\pgfsetstrokecolor{currentstroke}%
\pgfsetstrokeopacity{0.700000}%
\pgfsetdash{}{0pt}%
\pgfpathmoveto{\pgfqpoint{6.115767in}{0.550000in}}%
\pgfpathlineto{\pgfqpoint{6.143288in}{0.550000in}}%
\pgfpathlineto{\pgfqpoint{6.143288in}{0.550000in}}%
\pgfpathlineto{\pgfqpoint{6.115767in}{0.550000in}}%
\pgfpathlineto{\pgfqpoint{6.115767in}{0.550000in}}%
\pgfpathclose%
\pgfusepath{fill}%
\end{pgfscope}%
\begin{pgfscope}%
\pgfpathrectangle{\pgfqpoint{1.250000in}{0.550000in}}{\pgfqpoint{7.750000in}{3.850000in}}%
\pgfusepath{clip}%
\pgfsetbuttcap%
\pgfsetmiterjoin%
\definecolor{currentfill}{rgb}{0.501961,0.501961,0.501961}%
\pgfsetfillcolor{currentfill}%
\pgfsetfillopacity{0.700000}%
\pgfsetlinewidth{0.000000pt}%
\definecolor{currentstroke}{rgb}{0.000000,0.000000,0.000000}%
\pgfsetstrokecolor{currentstroke}%
\pgfsetstrokeopacity{0.700000}%
\pgfsetdash{}{0pt}%
\pgfpathmoveto{\pgfqpoint{6.143288in}{0.550000in}}%
\pgfpathlineto{\pgfqpoint{6.170810in}{0.550000in}}%
\pgfpathlineto{\pgfqpoint{6.170810in}{0.749958in}}%
\pgfpathlineto{\pgfqpoint{6.143288in}{0.749958in}}%
\pgfpathlineto{\pgfqpoint{6.143288in}{0.550000in}}%
\pgfpathclose%
\pgfusepath{fill}%
\end{pgfscope}%
\begin{pgfscope}%
\pgfpathrectangle{\pgfqpoint{1.250000in}{0.550000in}}{\pgfqpoint{7.750000in}{3.850000in}}%
\pgfusepath{clip}%
\pgfsetbuttcap%
\pgfsetmiterjoin%
\definecolor{currentfill}{rgb}{0.501961,0.501961,0.501961}%
\pgfsetfillcolor{currentfill}%
\pgfsetfillopacity{0.700000}%
\pgfsetlinewidth{0.000000pt}%
\definecolor{currentstroke}{rgb}{0.000000,0.000000,0.000000}%
\pgfsetstrokecolor{currentstroke}%
\pgfsetstrokeopacity{0.700000}%
\pgfsetdash{}{0pt}%
\pgfpathmoveto{\pgfqpoint{6.170810in}{0.550000in}}%
\pgfpathlineto{\pgfqpoint{6.198331in}{0.550000in}}%
\pgfpathlineto{\pgfqpoint{6.198331in}{0.736328in}}%
\pgfpathlineto{\pgfqpoint{6.170810in}{0.736328in}}%
\pgfpathlineto{\pgfqpoint{6.170810in}{0.550000in}}%
\pgfpathclose%
\pgfusepath{fill}%
\end{pgfscope}%
\begin{pgfscope}%
\pgfpathrectangle{\pgfqpoint{1.250000in}{0.550000in}}{\pgfqpoint{7.750000in}{3.850000in}}%
\pgfusepath{clip}%
\pgfsetbuttcap%
\pgfsetmiterjoin%
\definecolor{currentfill}{rgb}{0.501961,0.501961,0.501961}%
\pgfsetfillcolor{currentfill}%
\pgfsetfillopacity{0.700000}%
\pgfsetlinewidth{0.000000pt}%
\definecolor{currentstroke}{rgb}{0.000000,0.000000,0.000000}%
\pgfsetstrokecolor{currentstroke}%
\pgfsetstrokeopacity{0.700000}%
\pgfsetdash{}{0pt}%
\pgfpathmoveto{\pgfqpoint{6.198331in}{0.550000in}}%
\pgfpathlineto{\pgfqpoint{6.225852in}{0.550000in}}%
\pgfpathlineto{\pgfqpoint{6.225852in}{0.726467in}}%
\pgfpathlineto{\pgfqpoint{6.198331in}{0.726467in}}%
\pgfpathlineto{\pgfqpoint{6.198331in}{0.550000in}}%
\pgfpathclose%
\pgfusepath{fill}%
\end{pgfscope}%
\begin{pgfscope}%
\pgfpathrectangle{\pgfqpoint{1.250000in}{0.550000in}}{\pgfqpoint{7.750000in}{3.850000in}}%
\pgfusepath{clip}%
\pgfsetbuttcap%
\pgfsetmiterjoin%
\definecolor{currentfill}{rgb}{0.501961,0.501961,0.501961}%
\pgfsetfillcolor{currentfill}%
\pgfsetfillopacity{0.700000}%
\pgfsetlinewidth{0.000000pt}%
\definecolor{currentstroke}{rgb}{0.000000,0.000000,0.000000}%
\pgfsetstrokecolor{currentstroke}%
\pgfsetstrokeopacity{0.700000}%
\pgfsetdash{}{0pt}%
\pgfpathmoveto{\pgfqpoint{6.225852in}{0.550000in}}%
\pgfpathlineto{\pgfqpoint{6.253374in}{0.550000in}}%
\pgfpathlineto{\pgfqpoint{6.253374in}{0.732413in}}%
\pgfpathlineto{\pgfqpoint{6.225852in}{0.732413in}}%
\pgfpathlineto{\pgfqpoint{6.225852in}{0.550000in}}%
\pgfpathclose%
\pgfusepath{fill}%
\end{pgfscope}%
\begin{pgfscope}%
\pgfpathrectangle{\pgfqpoint{1.250000in}{0.550000in}}{\pgfqpoint{7.750000in}{3.850000in}}%
\pgfusepath{clip}%
\pgfsetbuttcap%
\pgfsetmiterjoin%
\definecolor{currentfill}{rgb}{0.501961,0.501961,0.501961}%
\pgfsetfillcolor{currentfill}%
\pgfsetfillopacity{0.700000}%
\pgfsetlinewidth{0.000000pt}%
\definecolor{currentstroke}{rgb}{0.000000,0.000000,0.000000}%
\pgfsetstrokecolor{currentstroke}%
\pgfsetstrokeopacity{0.700000}%
\pgfsetdash{}{0pt}%
\pgfpathmoveto{\pgfqpoint{6.253374in}{0.550000in}}%
\pgfpathlineto{\pgfqpoint{6.280895in}{0.550000in}}%
\pgfpathlineto{\pgfqpoint{6.280895in}{0.721827in}}%
\pgfpathlineto{\pgfqpoint{6.253374in}{0.721827in}}%
\pgfpathlineto{\pgfqpoint{6.253374in}{0.550000in}}%
\pgfpathclose%
\pgfusepath{fill}%
\end{pgfscope}%
\begin{pgfscope}%
\pgfpathrectangle{\pgfqpoint{1.250000in}{0.550000in}}{\pgfqpoint{7.750000in}{3.850000in}}%
\pgfusepath{clip}%
\pgfsetbuttcap%
\pgfsetmiterjoin%
\definecolor{currentfill}{rgb}{0.501961,0.501961,0.501961}%
\pgfsetfillcolor{currentfill}%
\pgfsetfillopacity{0.700000}%
\pgfsetlinewidth{0.000000pt}%
\definecolor{currentstroke}{rgb}{0.000000,0.000000,0.000000}%
\pgfsetstrokecolor{currentstroke}%
\pgfsetstrokeopacity{0.700000}%
\pgfsetdash{}{0pt}%
\pgfpathmoveto{\pgfqpoint{6.280895in}{0.550000in}}%
\pgfpathlineto{\pgfqpoint{6.308416in}{0.550000in}}%
\pgfpathlineto{\pgfqpoint{6.308416in}{0.708487in}}%
\pgfpathlineto{\pgfqpoint{6.280895in}{0.708487in}}%
\pgfpathlineto{\pgfqpoint{6.280895in}{0.550000in}}%
\pgfpathclose%
\pgfusepath{fill}%
\end{pgfscope}%
\begin{pgfscope}%
\pgfpathrectangle{\pgfqpoint{1.250000in}{0.550000in}}{\pgfqpoint{7.750000in}{3.850000in}}%
\pgfusepath{clip}%
\pgfsetbuttcap%
\pgfsetmiterjoin%
\definecolor{currentfill}{rgb}{0.501961,0.501961,0.501961}%
\pgfsetfillcolor{currentfill}%
\pgfsetfillopacity{0.700000}%
\pgfsetlinewidth{0.000000pt}%
\definecolor{currentstroke}{rgb}{0.000000,0.000000,0.000000}%
\pgfsetstrokecolor{currentstroke}%
\pgfsetstrokeopacity{0.700000}%
\pgfsetdash{}{0pt}%
\pgfpathmoveto{\pgfqpoint{6.308416in}{0.550000in}}%
\pgfpathlineto{\pgfqpoint{6.335937in}{0.550000in}}%
\pgfpathlineto{\pgfqpoint{6.335937in}{0.704862in}}%
\pgfpathlineto{\pgfqpoint{6.308416in}{0.704862in}}%
\pgfpathlineto{\pgfqpoint{6.308416in}{0.550000in}}%
\pgfpathclose%
\pgfusepath{fill}%
\end{pgfscope}%
\begin{pgfscope}%
\pgfpathrectangle{\pgfqpoint{1.250000in}{0.550000in}}{\pgfqpoint{7.750000in}{3.850000in}}%
\pgfusepath{clip}%
\pgfsetbuttcap%
\pgfsetmiterjoin%
\definecolor{currentfill}{rgb}{0.501961,0.501961,0.501961}%
\pgfsetfillcolor{currentfill}%
\pgfsetfillopacity{0.700000}%
\pgfsetlinewidth{0.000000pt}%
\definecolor{currentstroke}{rgb}{0.000000,0.000000,0.000000}%
\pgfsetstrokecolor{currentstroke}%
\pgfsetstrokeopacity{0.700000}%
\pgfsetdash{}{0pt}%
\pgfpathmoveto{\pgfqpoint{6.335938in}{0.550000in}}%
\pgfpathlineto{\pgfqpoint{6.363459in}{0.550000in}}%
\pgfpathlineto{\pgfqpoint{6.363459in}{0.696017in}}%
\pgfpathlineto{\pgfqpoint{6.335938in}{0.696017in}}%
\pgfpathlineto{\pgfqpoint{6.335938in}{0.550000in}}%
\pgfpathclose%
\pgfusepath{fill}%
\end{pgfscope}%
\begin{pgfscope}%
\pgfpathrectangle{\pgfqpoint{1.250000in}{0.550000in}}{\pgfqpoint{7.750000in}{3.850000in}}%
\pgfusepath{clip}%
\pgfsetbuttcap%
\pgfsetmiterjoin%
\definecolor{currentfill}{rgb}{0.501961,0.501961,0.501961}%
\pgfsetfillcolor{currentfill}%
\pgfsetfillopacity{0.700000}%
\pgfsetlinewidth{0.000000pt}%
\definecolor{currentstroke}{rgb}{0.000000,0.000000,0.000000}%
\pgfsetstrokecolor{currentstroke}%
\pgfsetstrokeopacity{0.700000}%
\pgfsetdash{}{0pt}%
\pgfpathmoveto{\pgfqpoint{6.363459in}{0.550000in}}%
\pgfpathlineto{\pgfqpoint{6.390980in}{0.550000in}}%
\pgfpathlineto{\pgfqpoint{6.390980in}{0.699642in}}%
\pgfpathlineto{\pgfqpoint{6.363459in}{0.699642in}}%
\pgfpathlineto{\pgfqpoint{6.363459in}{0.550000in}}%
\pgfpathclose%
\pgfusepath{fill}%
\end{pgfscope}%
\begin{pgfscope}%
\pgfpathrectangle{\pgfqpoint{1.250000in}{0.550000in}}{\pgfqpoint{7.750000in}{3.850000in}}%
\pgfusepath{clip}%
\pgfsetbuttcap%
\pgfsetmiterjoin%
\definecolor{currentfill}{rgb}{0.501961,0.501961,0.501961}%
\pgfsetfillcolor{currentfill}%
\pgfsetfillopacity{0.700000}%
\pgfsetlinewidth{0.000000pt}%
\definecolor{currentstroke}{rgb}{0.000000,0.000000,0.000000}%
\pgfsetstrokecolor{currentstroke}%
\pgfsetstrokeopacity{0.700000}%
\pgfsetdash{}{0pt}%
\pgfpathmoveto{\pgfqpoint{6.390980in}{0.550000in}}%
\pgfpathlineto{\pgfqpoint{6.418501in}{0.550000in}}%
\pgfpathlineto{\pgfqpoint{6.418501in}{0.690797in}}%
\pgfpathlineto{\pgfqpoint{6.390980in}{0.690797in}}%
\pgfpathlineto{\pgfqpoint{6.390980in}{0.550000in}}%
\pgfpathclose%
\pgfusepath{fill}%
\end{pgfscope}%
\begin{pgfscope}%
\pgfpathrectangle{\pgfqpoint{1.250000in}{0.550000in}}{\pgfqpoint{7.750000in}{3.850000in}}%
\pgfusepath{clip}%
\pgfsetbuttcap%
\pgfsetmiterjoin%
\definecolor{currentfill}{rgb}{0.501961,0.501961,0.501961}%
\pgfsetfillcolor{currentfill}%
\pgfsetfillopacity{0.700000}%
\pgfsetlinewidth{0.000000pt}%
\definecolor{currentstroke}{rgb}{0.000000,0.000000,0.000000}%
\pgfsetstrokecolor{currentstroke}%
\pgfsetstrokeopacity{0.700000}%
\pgfsetdash{}{0pt}%
\pgfpathmoveto{\pgfqpoint{6.418501in}{0.550000in}}%
\pgfpathlineto{\pgfqpoint{6.446023in}{0.550000in}}%
\pgfpathlineto{\pgfqpoint{6.446023in}{0.822169in}}%
\pgfpathlineto{\pgfqpoint{6.418501in}{0.822169in}}%
\pgfpathlineto{\pgfqpoint{6.418501in}{0.550000in}}%
\pgfpathclose%
\pgfusepath{fill}%
\end{pgfscope}%
\begin{pgfscope}%
\pgfpathrectangle{\pgfqpoint{1.250000in}{0.550000in}}{\pgfqpoint{7.750000in}{3.850000in}}%
\pgfusepath{clip}%
\pgfsetbuttcap%
\pgfsetmiterjoin%
\definecolor{currentfill}{rgb}{0.501961,0.501961,0.501961}%
\pgfsetfillcolor{currentfill}%
\pgfsetfillopacity{0.700000}%
\pgfsetlinewidth{0.000000pt}%
\definecolor{currentstroke}{rgb}{0.000000,0.000000,0.000000}%
\pgfsetstrokecolor{currentstroke}%
\pgfsetstrokeopacity{0.700000}%
\pgfsetdash{}{0pt}%
\pgfpathmoveto{\pgfqpoint{6.446023in}{0.550000in}}%
\pgfpathlineto{\pgfqpoint{6.473544in}{0.550000in}}%
\pgfpathlineto{\pgfqpoint{6.473544in}{0.677602in}}%
\pgfpathlineto{\pgfqpoint{6.446023in}{0.677602in}}%
\pgfpathlineto{\pgfqpoint{6.446023in}{0.550000in}}%
\pgfpathclose%
\pgfusepath{fill}%
\end{pgfscope}%
\begin{pgfscope}%
\pgfpathrectangle{\pgfqpoint{1.250000in}{0.550000in}}{\pgfqpoint{7.750000in}{3.850000in}}%
\pgfusepath{clip}%
\pgfsetbuttcap%
\pgfsetmiterjoin%
\definecolor{currentfill}{rgb}{0.501961,0.501961,0.501961}%
\pgfsetfillcolor{currentfill}%
\pgfsetfillopacity{0.700000}%
\pgfsetlinewidth{0.000000pt}%
\definecolor{currentstroke}{rgb}{0.000000,0.000000,0.000000}%
\pgfsetstrokecolor{currentstroke}%
\pgfsetstrokeopacity{0.700000}%
\pgfsetdash{}{0pt}%
\pgfpathmoveto{\pgfqpoint{6.473544in}{0.550000in}}%
\pgfpathlineto{\pgfqpoint{6.501065in}{0.550000in}}%
\pgfpathlineto{\pgfqpoint{6.501065in}{0.678617in}}%
\pgfpathlineto{\pgfqpoint{6.473544in}{0.678617in}}%
\pgfpathlineto{\pgfqpoint{6.473544in}{0.550000in}}%
\pgfpathclose%
\pgfusepath{fill}%
\end{pgfscope}%
\begin{pgfscope}%
\pgfpathrectangle{\pgfqpoint{1.250000in}{0.550000in}}{\pgfqpoint{7.750000in}{3.850000in}}%
\pgfusepath{clip}%
\pgfsetbuttcap%
\pgfsetmiterjoin%
\definecolor{currentfill}{rgb}{0.501961,0.501961,0.501961}%
\pgfsetfillcolor{currentfill}%
\pgfsetfillopacity{0.700000}%
\pgfsetlinewidth{0.000000pt}%
\definecolor{currentstroke}{rgb}{0.000000,0.000000,0.000000}%
\pgfsetstrokecolor{currentstroke}%
\pgfsetstrokeopacity{0.700000}%
\pgfsetdash{}{0pt}%
\pgfpathmoveto{\pgfqpoint{6.501065in}{0.550000in}}%
\pgfpathlineto{\pgfqpoint{6.528587in}{0.550000in}}%
\pgfpathlineto{\pgfqpoint{6.528587in}{0.677312in}}%
\pgfpathlineto{\pgfqpoint{6.501065in}{0.677312in}}%
\pgfpathlineto{\pgfqpoint{6.501065in}{0.550000in}}%
\pgfpathclose%
\pgfusepath{fill}%
\end{pgfscope}%
\begin{pgfscope}%
\pgfpathrectangle{\pgfqpoint{1.250000in}{0.550000in}}{\pgfqpoint{7.750000in}{3.850000in}}%
\pgfusepath{clip}%
\pgfsetbuttcap%
\pgfsetmiterjoin%
\definecolor{currentfill}{rgb}{0.501961,0.501961,0.501961}%
\pgfsetfillcolor{currentfill}%
\pgfsetfillopacity{0.700000}%
\pgfsetlinewidth{0.000000pt}%
\definecolor{currentstroke}{rgb}{0.000000,0.000000,0.000000}%
\pgfsetstrokecolor{currentstroke}%
\pgfsetstrokeopacity{0.700000}%
\pgfsetdash{}{0pt}%
\pgfpathmoveto{\pgfqpoint{6.528587in}{0.550000in}}%
\pgfpathlineto{\pgfqpoint{6.556108in}{0.550000in}}%
\pgfpathlineto{\pgfqpoint{6.556108in}{0.799839in}}%
\pgfpathlineto{\pgfqpoint{6.528587in}{0.799839in}}%
\pgfpathlineto{\pgfqpoint{6.528587in}{0.550000in}}%
\pgfpathclose%
\pgfusepath{fill}%
\end{pgfscope}%
\begin{pgfscope}%
\pgfpathrectangle{\pgfqpoint{1.250000in}{0.550000in}}{\pgfqpoint{7.750000in}{3.850000in}}%
\pgfusepath{clip}%
\pgfsetbuttcap%
\pgfsetmiterjoin%
\definecolor{currentfill}{rgb}{0.501961,0.501961,0.501961}%
\pgfsetfillcolor{currentfill}%
\pgfsetfillopacity{0.700000}%
\pgfsetlinewidth{0.000000pt}%
\definecolor{currentstroke}{rgb}{0.000000,0.000000,0.000000}%
\pgfsetstrokecolor{currentstroke}%
\pgfsetstrokeopacity{0.700000}%
\pgfsetdash{}{0pt}%
\pgfpathmoveto{\pgfqpoint{6.556108in}{0.550000in}}%
\pgfpathlineto{\pgfqpoint{6.583629in}{0.550000in}}%
\pgfpathlineto{\pgfqpoint{6.583629in}{0.675282in}}%
\pgfpathlineto{\pgfqpoint{6.556108in}{0.675282in}}%
\pgfpathlineto{\pgfqpoint{6.556108in}{0.550000in}}%
\pgfpathclose%
\pgfusepath{fill}%
\end{pgfscope}%
\begin{pgfscope}%
\pgfpathrectangle{\pgfqpoint{1.250000in}{0.550000in}}{\pgfqpoint{7.750000in}{3.850000in}}%
\pgfusepath{clip}%
\pgfsetbuttcap%
\pgfsetmiterjoin%
\definecolor{currentfill}{rgb}{0.501961,0.501961,0.501961}%
\pgfsetfillcolor{currentfill}%
\pgfsetfillopacity{0.700000}%
\pgfsetlinewidth{0.000000pt}%
\definecolor{currentstroke}{rgb}{0.000000,0.000000,0.000000}%
\pgfsetstrokecolor{currentstroke}%
\pgfsetstrokeopacity{0.700000}%
\pgfsetdash{}{0pt}%
\pgfpathmoveto{\pgfqpoint{6.583629in}{0.550000in}}%
\pgfpathlineto{\pgfqpoint{6.611151in}{0.550000in}}%
\pgfpathlineto{\pgfqpoint{6.611151in}{0.663827in}}%
\pgfpathlineto{\pgfqpoint{6.583629in}{0.663827in}}%
\pgfpathlineto{\pgfqpoint{6.583629in}{0.550000in}}%
\pgfpathclose%
\pgfusepath{fill}%
\end{pgfscope}%
\begin{pgfscope}%
\pgfpathrectangle{\pgfqpoint{1.250000in}{0.550000in}}{\pgfqpoint{7.750000in}{3.850000in}}%
\pgfusepath{clip}%
\pgfsetbuttcap%
\pgfsetmiterjoin%
\definecolor{currentfill}{rgb}{0.501961,0.501961,0.501961}%
\pgfsetfillcolor{currentfill}%
\pgfsetfillopacity{0.700000}%
\pgfsetlinewidth{0.000000pt}%
\definecolor{currentstroke}{rgb}{0.000000,0.000000,0.000000}%
\pgfsetstrokecolor{currentstroke}%
\pgfsetstrokeopacity{0.700000}%
\pgfsetdash{}{0pt}%
\pgfpathmoveto{\pgfqpoint{6.611151in}{0.550000in}}%
\pgfpathlineto{\pgfqpoint{6.638672in}{0.550000in}}%
\pgfpathlineto{\pgfqpoint{6.638672in}{0.779828in}}%
\pgfpathlineto{\pgfqpoint{6.611151in}{0.779828in}}%
\pgfpathlineto{\pgfqpoint{6.611151in}{0.550000in}}%
\pgfpathclose%
\pgfusepath{fill}%
\end{pgfscope}%
\begin{pgfscope}%
\pgfpathrectangle{\pgfqpoint{1.250000in}{0.550000in}}{\pgfqpoint{7.750000in}{3.850000in}}%
\pgfusepath{clip}%
\pgfsetbuttcap%
\pgfsetmiterjoin%
\definecolor{currentfill}{rgb}{0.501961,0.501961,0.501961}%
\pgfsetfillcolor{currentfill}%
\pgfsetfillopacity{0.700000}%
\pgfsetlinewidth{0.000000pt}%
\definecolor{currentstroke}{rgb}{0.000000,0.000000,0.000000}%
\pgfsetstrokecolor{currentstroke}%
\pgfsetstrokeopacity{0.700000}%
\pgfsetdash{}{0pt}%
\pgfpathmoveto{\pgfqpoint{6.638672in}{0.550000in}}%
\pgfpathlineto{\pgfqpoint{6.666193in}{0.550000in}}%
\pgfpathlineto{\pgfqpoint{6.666193in}{0.664697in}}%
\pgfpathlineto{\pgfqpoint{6.638672in}{0.664697in}}%
\pgfpathlineto{\pgfqpoint{6.638672in}{0.550000in}}%
\pgfpathclose%
\pgfusepath{fill}%
\end{pgfscope}%
\begin{pgfscope}%
\pgfpathrectangle{\pgfqpoint{1.250000in}{0.550000in}}{\pgfqpoint{7.750000in}{3.850000in}}%
\pgfusepath{clip}%
\pgfsetbuttcap%
\pgfsetmiterjoin%
\definecolor{currentfill}{rgb}{0.501961,0.501961,0.501961}%
\pgfsetfillcolor{currentfill}%
\pgfsetfillopacity{0.700000}%
\pgfsetlinewidth{0.000000pt}%
\definecolor{currentstroke}{rgb}{0.000000,0.000000,0.000000}%
\pgfsetstrokecolor{currentstroke}%
\pgfsetstrokeopacity{0.700000}%
\pgfsetdash{}{0pt}%
\pgfpathmoveto{\pgfqpoint{6.666193in}{0.550000in}}%
\pgfpathlineto{\pgfqpoint{6.693714in}{0.550000in}}%
\pgfpathlineto{\pgfqpoint{6.693714in}{0.763153in}}%
\pgfpathlineto{\pgfqpoint{6.666193in}{0.763153in}}%
\pgfpathlineto{\pgfqpoint{6.666193in}{0.550000in}}%
\pgfpathclose%
\pgfusepath{fill}%
\end{pgfscope}%
\begin{pgfscope}%
\pgfpathrectangle{\pgfqpoint{1.250000in}{0.550000in}}{\pgfqpoint{7.750000in}{3.850000in}}%
\pgfusepath{clip}%
\pgfsetbuttcap%
\pgfsetmiterjoin%
\definecolor{currentfill}{rgb}{0.501961,0.501961,0.501961}%
\pgfsetfillcolor{currentfill}%
\pgfsetfillopacity{0.700000}%
\pgfsetlinewidth{0.000000pt}%
\definecolor{currentstroke}{rgb}{0.000000,0.000000,0.000000}%
\pgfsetstrokecolor{currentstroke}%
\pgfsetstrokeopacity{0.700000}%
\pgfsetdash{}{0pt}%
\pgfpathmoveto{\pgfqpoint{6.693714in}{0.550000in}}%
\pgfpathlineto{\pgfqpoint{6.721236in}{0.550000in}}%
\pgfpathlineto{\pgfqpoint{6.721236in}{0.658172in}}%
\pgfpathlineto{\pgfqpoint{6.693714in}{0.658172in}}%
\pgfpathlineto{\pgfqpoint{6.693714in}{0.550000in}}%
\pgfpathclose%
\pgfusepath{fill}%
\end{pgfscope}%
\begin{pgfscope}%
\pgfpathrectangle{\pgfqpoint{1.250000in}{0.550000in}}{\pgfqpoint{7.750000in}{3.850000in}}%
\pgfusepath{clip}%
\pgfsetbuttcap%
\pgfsetmiterjoin%
\definecolor{currentfill}{rgb}{0.501961,0.501961,0.501961}%
\pgfsetfillcolor{currentfill}%
\pgfsetfillopacity{0.700000}%
\pgfsetlinewidth{0.000000pt}%
\definecolor{currentstroke}{rgb}{0.000000,0.000000,0.000000}%
\pgfsetstrokecolor{currentstroke}%
\pgfsetstrokeopacity{0.700000}%
\pgfsetdash{}{0pt}%
\pgfpathmoveto{\pgfqpoint{6.721236in}{0.550000in}}%
\pgfpathlineto{\pgfqpoint{6.748757in}{0.550000in}}%
\pgfpathlineto{\pgfqpoint{6.748757in}{0.767503in}}%
\pgfpathlineto{\pgfqpoint{6.721236in}{0.767503in}}%
\pgfpathlineto{\pgfqpoint{6.721236in}{0.550000in}}%
\pgfpathclose%
\pgfusepath{fill}%
\end{pgfscope}%
\begin{pgfscope}%
\pgfpathrectangle{\pgfqpoint{1.250000in}{0.550000in}}{\pgfqpoint{7.750000in}{3.850000in}}%
\pgfusepath{clip}%
\pgfsetbuttcap%
\pgfsetmiterjoin%
\definecolor{currentfill}{rgb}{0.501961,0.501961,0.501961}%
\pgfsetfillcolor{currentfill}%
\pgfsetfillopacity{0.700000}%
\pgfsetlinewidth{0.000000pt}%
\definecolor{currentstroke}{rgb}{0.000000,0.000000,0.000000}%
\pgfsetstrokecolor{currentstroke}%
\pgfsetstrokeopacity{0.700000}%
\pgfsetdash{}{0pt}%
\pgfpathmoveto{\pgfqpoint{6.748757in}{0.550000in}}%
\pgfpathlineto{\pgfqpoint{6.776278in}{0.550000in}}%
\pgfpathlineto{\pgfqpoint{6.776278in}{0.651501in}}%
\pgfpathlineto{\pgfqpoint{6.748757in}{0.651501in}}%
\pgfpathlineto{\pgfqpoint{6.748757in}{0.550000in}}%
\pgfpathclose%
\pgfusepath{fill}%
\end{pgfscope}%
\begin{pgfscope}%
\pgfpathrectangle{\pgfqpoint{1.250000in}{0.550000in}}{\pgfqpoint{7.750000in}{3.850000in}}%
\pgfusepath{clip}%
\pgfsetbuttcap%
\pgfsetmiterjoin%
\definecolor{currentfill}{rgb}{0.501961,0.501961,0.501961}%
\pgfsetfillcolor{currentfill}%
\pgfsetfillopacity{0.700000}%
\pgfsetlinewidth{0.000000pt}%
\definecolor{currentstroke}{rgb}{0.000000,0.000000,0.000000}%
\pgfsetstrokecolor{currentstroke}%
\pgfsetstrokeopacity{0.700000}%
\pgfsetdash{}{0pt}%
\pgfpathmoveto{\pgfqpoint{6.776278in}{0.550000in}}%
\pgfpathlineto{\pgfqpoint{6.803800in}{0.550000in}}%
\pgfpathlineto{\pgfqpoint{6.803800in}{0.767503in}}%
\pgfpathlineto{\pgfqpoint{6.776278in}{0.767503in}}%
\pgfpathlineto{\pgfqpoint{6.776278in}{0.550000in}}%
\pgfpathclose%
\pgfusepath{fill}%
\end{pgfscope}%
\begin{pgfscope}%
\pgfpathrectangle{\pgfqpoint{1.250000in}{0.550000in}}{\pgfqpoint{7.750000in}{3.850000in}}%
\pgfusepath{clip}%
\pgfsetbuttcap%
\pgfsetmiterjoin%
\definecolor{currentfill}{rgb}{0.501961,0.501961,0.501961}%
\pgfsetfillcolor{currentfill}%
\pgfsetfillopacity{0.700000}%
\pgfsetlinewidth{0.000000pt}%
\definecolor{currentstroke}{rgb}{0.000000,0.000000,0.000000}%
\pgfsetstrokecolor{currentstroke}%
\pgfsetstrokeopacity{0.700000}%
\pgfsetdash{}{0pt}%
\pgfpathmoveto{\pgfqpoint{6.803800in}{0.550000in}}%
\pgfpathlineto{\pgfqpoint{6.831321in}{0.550000in}}%
\pgfpathlineto{\pgfqpoint{6.831321in}{0.649906in}}%
\pgfpathlineto{\pgfqpoint{6.803800in}{0.649906in}}%
\pgfpathlineto{\pgfqpoint{6.803800in}{0.550000in}}%
\pgfpathclose%
\pgfusepath{fill}%
\end{pgfscope}%
\begin{pgfscope}%
\pgfpathrectangle{\pgfqpoint{1.250000in}{0.550000in}}{\pgfqpoint{7.750000in}{3.850000in}}%
\pgfusepath{clip}%
\pgfsetbuttcap%
\pgfsetmiterjoin%
\definecolor{currentfill}{rgb}{0.501961,0.501961,0.501961}%
\pgfsetfillcolor{currentfill}%
\pgfsetfillopacity{0.700000}%
\pgfsetlinewidth{0.000000pt}%
\definecolor{currentstroke}{rgb}{0.000000,0.000000,0.000000}%
\pgfsetstrokecolor{currentstroke}%
\pgfsetstrokeopacity{0.700000}%
\pgfsetdash{}{0pt}%
\pgfpathmoveto{\pgfqpoint{6.831321in}{0.550000in}}%
\pgfpathlineto{\pgfqpoint{6.858842in}{0.550000in}}%
\pgfpathlineto{\pgfqpoint{6.858842in}{0.755033in}}%
\pgfpathlineto{\pgfqpoint{6.831321in}{0.755033in}}%
\pgfpathlineto{\pgfqpoint{6.831321in}{0.550000in}}%
\pgfpathclose%
\pgfusepath{fill}%
\end{pgfscope}%
\begin{pgfscope}%
\pgfpathrectangle{\pgfqpoint{1.250000in}{0.550000in}}{\pgfqpoint{7.750000in}{3.850000in}}%
\pgfusepath{clip}%
\pgfsetbuttcap%
\pgfsetmiterjoin%
\definecolor{currentfill}{rgb}{0.501961,0.501961,0.501961}%
\pgfsetfillcolor{currentfill}%
\pgfsetfillopacity{0.700000}%
\pgfsetlinewidth{0.000000pt}%
\definecolor{currentstroke}{rgb}{0.000000,0.000000,0.000000}%
\pgfsetstrokecolor{currentstroke}%
\pgfsetstrokeopacity{0.700000}%
\pgfsetdash{}{0pt}%
\pgfpathmoveto{\pgfqpoint{6.858842in}{0.550000in}}%
\pgfpathlineto{\pgfqpoint{6.886364in}{0.550000in}}%
\pgfpathlineto{\pgfqpoint{6.886364in}{0.650196in}}%
\pgfpathlineto{\pgfqpoint{6.858842in}{0.650196in}}%
\pgfpathlineto{\pgfqpoint{6.858842in}{0.550000in}}%
\pgfpathclose%
\pgfusepath{fill}%
\end{pgfscope}%
\begin{pgfscope}%
\pgfpathrectangle{\pgfqpoint{1.250000in}{0.550000in}}{\pgfqpoint{7.750000in}{3.850000in}}%
\pgfusepath{clip}%
\pgfsetbuttcap%
\pgfsetmiterjoin%
\definecolor{currentfill}{rgb}{0.501961,0.501961,0.501961}%
\pgfsetfillcolor{currentfill}%
\pgfsetfillopacity{0.700000}%
\pgfsetlinewidth{0.000000pt}%
\definecolor{currentstroke}{rgb}{0.000000,0.000000,0.000000}%
\pgfsetstrokecolor{currentstroke}%
\pgfsetstrokeopacity{0.700000}%
\pgfsetdash{}{0pt}%
\pgfpathmoveto{\pgfqpoint{6.886364in}{0.550000in}}%
\pgfpathlineto{\pgfqpoint{6.913885in}{0.550000in}}%
\pgfpathlineto{\pgfqpoint{6.913885in}{0.751408in}}%
\pgfpathlineto{\pgfqpoint{6.886364in}{0.751408in}}%
\pgfpathlineto{\pgfqpoint{6.886364in}{0.550000in}}%
\pgfpathclose%
\pgfusepath{fill}%
\end{pgfscope}%
\begin{pgfscope}%
\pgfpathrectangle{\pgfqpoint{1.250000in}{0.550000in}}{\pgfqpoint{7.750000in}{3.850000in}}%
\pgfusepath{clip}%
\pgfsetbuttcap%
\pgfsetmiterjoin%
\definecolor{currentfill}{rgb}{0.501961,0.501961,0.501961}%
\pgfsetfillcolor{currentfill}%
\pgfsetfillopacity{0.700000}%
\pgfsetlinewidth{0.000000pt}%
\definecolor{currentstroke}{rgb}{0.000000,0.000000,0.000000}%
\pgfsetstrokecolor{currentstroke}%
\pgfsetstrokeopacity{0.700000}%
\pgfsetdash{}{0pt}%
\pgfpathmoveto{\pgfqpoint{6.913885in}{0.550000in}}%
\pgfpathlineto{\pgfqpoint{6.941406in}{0.550000in}}%
\pgfpathlineto{\pgfqpoint{6.941406in}{0.746768in}}%
\pgfpathlineto{\pgfqpoint{6.913885in}{0.746768in}}%
\pgfpathlineto{\pgfqpoint{6.913885in}{0.550000in}}%
\pgfpathclose%
\pgfusepath{fill}%
\end{pgfscope}%
\begin{pgfscope}%
\pgfpathrectangle{\pgfqpoint{1.250000in}{0.550000in}}{\pgfqpoint{7.750000in}{3.850000in}}%
\pgfusepath{clip}%
\pgfsetbuttcap%
\pgfsetmiterjoin%
\definecolor{currentfill}{rgb}{0.501961,0.501961,0.501961}%
\pgfsetfillcolor{currentfill}%
\pgfsetfillopacity{0.700000}%
\pgfsetlinewidth{0.000000pt}%
\definecolor{currentstroke}{rgb}{0.000000,0.000000,0.000000}%
\pgfsetstrokecolor{currentstroke}%
\pgfsetstrokeopacity{0.700000}%
\pgfsetdash{}{0pt}%
\pgfpathmoveto{\pgfqpoint{6.941406in}{0.550000in}}%
\pgfpathlineto{\pgfqpoint{6.968928in}{0.550000in}}%
\pgfpathlineto{\pgfqpoint{6.968928in}{0.642221in}}%
\pgfpathlineto{\pgfqpoint{6.941406in}{0.642221in}}%
\pgfpathlineto{\pgfqpoint{6.941406in}{0.550000in}}%
\pgfpathclose%
\pgfusepath{fill}%
\end{pgfscope}%
\begin{pgfscope}%
\pgfpathrectangle{\pgfqpoint{1.250000in}{0.550000in}}{\pgfqpoint{7.750000in}{3.850000in}}%
\pgfusepath{clip}%
\pgfsetbuttcap%
\pgfsetmiterjoin%
\definecolor{currentfill}{rgb}{0.501961,0.501961,0.501961}%
\pgfsetfillcolor{currentfill}%
\pgfsetfillopacity{0.700000}%
\pgfsetlinewidth{0.000000pt}%
\definecolor{currentstroke}{rgb}{0.000000,0.000000,0.000000}%
\pgfsetstrokecolor{currentstroke}%
\pgfsetstrokeopacity{0.700000}%
\pgfsetdash{}{0pt}%
\pgfpathmoveto{\pgfqpoint{6.968928in}{0.550000in}}%
\pgfpathlineto{\pgfqpoint{6.996449in}{0.550000in}}%
\pgfpathlineto{\pgfqpoint{6.996449in}{0.751118in}}%
\pgfpathlineto{\pgfqpoint{6.968928in}{0.751118in}}%
\pgfpathlineto{\pgfqpoint{6.968928in}{0.550000in}}%
\pgfpathclose%
\pgfusepath{fill}%
\end{pgfscope}%
\begin{pgfscope}%
\pgfpathrectangle{\pgfqpoint{1.250000in}{0.550000in}}{\pgfqpoint{7.750000in}{3.850000in}}%
\pgfusepath{clip}%
\pgfsetbuttcap%
\pgfsetmiterjoin%
\definecolor{currentfill}{rgb}{0.501961,0.501961,0.501961}%
\pgfsetfillcolor{currentfill}%
\pgfsetfillopacity{0.700000}%
\pgfsetlinewidth{0.000000pt}%
\definecolor{currentstroke}{rgb}{0.000000,0.000000,0.000000}%
\pgfsetstrokecolor{currentstroke}%
\pgfsetstrokeopacity{0.700000}%
\pgfsetdash{}{0pt}%
\pgfpathmoveto{\pgfqpoint{6.996449in}{0.550000in}}%
\pgfpathlineto{\pgfqpoint{7.023970in}{0.550000in}}%
\pgfpathlineto{\pgfqpoint{7.023970in}{0.740533in}}%
\pgfpathlineto{\pgfqpoint{6.996449in}{0.740533in}}%
\pgfpathlineto{\pgfqpoint{6.996449in}{0.550000in}}%
\pgfpathclose%
\pgfusepath{fill}%
\end{pgfscope}%
\begin{pgfscope}%
\pgfpathrectangle{\pgfqpoint{1.250000in}{0.550000in}}{\pgfqpoint{7.750000in}{3.850000in}}%
\pgfusepath{clip}%
\pgfsetbuttcap%
\pgfsetmiterjoin%
\definecolor{currentfill}{rgb}{0.501961,0.501961,0.501961}%
\pgfsetfillcolor{currentfill}%
\pgfsetfillopacity{0.700000}%
\pgfsetlinewidth{0.000000pt}%
\definecolor{currentstroke}{rgb}{0.000000,0.000000,0.000000}%
\pgfsetstrokecolor{currentstroke}%
\pgfsetstrokeopacity{0.700000}%
\pgfsetdash{}{0pt}%
\pgfpathmoveto{\pgfqpoint{7.023970in}{0.550000in}}%
\pgfpathlineto{\pgfqpoint{7.051491in}{0.550000in}}%
\pgfpathlineto{\pgfqpoint{7.051491in}{0.647731in}}%
\pgfpathlineto{\pgfqpoint{7.023970in}{0.647731in}}%
\pgfpathlineto{\pgfqpoint{7.023970in}{0.550000in}}%
\pgfpathclose%
\pgfusepath{fill}%
\end{pgfscope}%
\begin{pgfscope}%
\pgfpathrectangle{\pgfqpoint{1.250000in}{0.550000in}}{\pgfqpoint{7.750000in}{3.850000in}}%
\pgfusepath{clip}%
\pgfsetbuttcap%
\pgfsetmiterjoin%
\definecolor{currentfill}{rgb}{0.501961,0.501961,0.501961}%
\pgfsetfillcolor{currentfill}%
\pgfsetfillopacity{0.700000}%
\pgfsetlinewidth{0.000000pt}%
\definecolor{currentstroke}{rgb}{0.000000,0.000000,0.000000}%
\pgfsetstrokecolor{currentstroke}%
\pgfsetstrokeopacity{0.700000}%
\pgfsetdash{}{0pt}%
\pgfpathmoveto{\pgfqpoint{7.051491in}{0.550000in}}%
\pgfpathlineto{\pgfqpoint{7.079013in}{0.550000in}}%
\pgfpathlineto{\pgfqpoint{7.079013in}{0.748508in}}%
\pgfpathlineto{\pgfqpoint{7.051491in}{0.748508in}}%
\pgfpathlineto{\pgfqpoint{7.051491in}{0.550000in}}%
\pgfpathclose%
\pgfusepath{fill}%
\end{pgfscope}%
\begin{pgfscope}%
\pgfpathrectangle{\pgfqpoint{1.250000in}{0.550000in}}{\pgfqpoint{7.750000in}{3.850000in}}%
\pgfusepath{clip}%
\pgfsetbuttcap%
\pgfsetmiterjoin%
\definecolor{currentfill}{rgb}{0.501961,0.501961,0.501961}%
\pgfsetfillcolor{currentfill}%
\pgfsetfillopacity{0.700000}%
\pgfsetlinewidth{0.000000pt}%
\definecolor{currentstroke}{rgb}{0.000000,0.000000,0.000000}%
\pgfsetstrokecolor{currentstroke}%
\pgfsetstrokeopacity{0.700000}%
\pgfsetdash{}{0pt}%
\pgfpathmoveto{\pgfqpoint{7.079013in}{0.550000in}}%
\pgfpathlineto{\pgfqpoint{7.106534in}{0.550000in}}%
\pgfpathlineto{\pgfqpoint{7.106534in}{0.749813in}}%
\pgfpathlineto{\pgfqpoint{7.079013in}{0.749813in}}%
\pgfpathlineto{\pgfqpoint{7.079013in}{0.550000in}}%
\pgfpathclose%
\pgfusepath{fill}%
\end{pgfscope}%
\begin{pgfscope}%
\pgfpathrectangle{\pgfqpoint{1.250000in}{0.550000in}}{\pgfqpoint{7.750000in}{3.850000in}}%
\pgfusepath{clip}%
\pgfsetbuttcap%
\pgfsetmiterjoin%
\definecolor{currentfill}{rgb}{0.501961,0.501961,0.501961}%
\pgfsetfillcolor{currentfill}%
\pgfsetfillopacity{0.700000}%
\pgfsetlinewidth{0.000000pt}%
\definecolor{currentstroke}{rgb}{0.000000,0.000000,0.000000}%
\pgfsetstrokecolor{currentstroke}%
\pgfsetstrokeopacity{0.700000}%
\pgfsetdash{}{0pt}%
\pgfpathmoveto{\pgfqpoint{7.106534in}{0.550000in}}%
\pgfpathlineto{\pgfqpoint{7.134055in}{0.550000in}}%
\pgfpathlineto{\pgfqpoint{7.134055in}{0.657012in}}%
\pgfpathlineto{\pgfqpoint{7.106534in}{0.657012in}}%
\pgfpathlineto{\pgfqpoint{7.106534in}{0.550000in}}%
\pgfpathclose%
\pgfusepath{fill}%
\end{pgfscope}%
\begin{pgfscope}%
\pgfpathrectangle{\pgfqpoint{1.250000in}{0.550000in}}{\pgfqpoint{7.750000in}{3.850000in}}%
\pgfusepath{clip}%
\pgfsetbuttcap%
\pgfsetmiterjoin%
\definecolor{currentfill}{rgb}{0.501961,0.501961,0.501961}%
\pgfsetfillcolor{currentfill}%
\pgfsetfillopacity{0.700000}%
\pgfsetlinewidth{0.000000pt}%
\definecolor{currentstroke}{rgb}{0.000000,0.000000,0.000000}%
\pgfsetstrokecolor{currentstroke}%
\pgfsetstrokeopacity{0.700000}%
\pgfsetdash{}{0pt}%
\pgfpathmoveto{\pgfqpoint{7.134055in}{0.550000in}}%
\pgfpathlineto{\pgfqpoint{7.161577in}{0.550000in}}%
\pgfpathlineto{\pgfqpoint{7.161577in}{0.749233in}}%
\pgfpathlineto{\pgfqpoint{7.134055in}{0.749233in}}%
\pgfpathlineto{\pgfqpoint{7.134055in}{0.550000in}}%
\pgfpathclose%
\pgfusepath{fill}%
\end{pgfscope}%
\begin{pgfscope}%
\pgfpathrectangle{\pgfqpoint{1.250000in}{0.550000in}}{\pgfqpoint{7.750000in}{3.850000in}}%
\pgfusepath{clip}%
\pgfsetbuttcap%
\pgfsetmiterjoin%
\definecolor{currentfill}{rgb}{0.501961,0.501961,0.501961}%
\pgfsetfillcolor{currentfill}%
\pgfsetfillopacity{0.700000}%
\pgfsetlinewidth{0.000000pt}%
\definecolor{currentstroke}{rgb}{0.000000,0.000000,0.000000}%
\pgfsetstrokecolor{currentstroke}%
\pgfsetstrokeopacity{0.700000}%
\pgfsetdash{}{0pt}%
\pgfpathmoveto{\pgfqpoint{7.161577in}{0.550000in}}%
\pgfpathlineto{\pgfqpoint{7.189098in}{0.550000in}}%
\pgfpathlineto{\pgfqpoint{7.189098in}{0.649906in}}%
\pgfpathlineto{\pgfqpoint{7.161577in}{0.649906in}}%
\pgfpathlineto{\pgfqpoint{7.161577in}{0.550000in}}%
\pgfpathclose%
\pgfusepath{fill}%
\end{pgfscope}%
\begin{pgfscope}%
\pgfpathrectangle{\pgfqpoint{1.250000in}{0.550000in}}{\pgfqpoint{7.750000in}{3.850000in}}%
\pgfusepath{clip}%
\pgfsetbuttcap%
\pgfsetmiterjoin%
\definecolor{currentfill}{rgb}{0.501961,0.501961,0.501961}%
\pgfsetfillcolor{currentfill}%
\pgfsetfillopacity{0.700000}%
\pgfsetlinewidth{0.000000pt}%
\definecolor{currentstroke}{rgb}{0.000000,0.000000,0.000000}%
\pgfsetstrokecolor{currentstroke}%
\pgfsetstrokeopacity{0.700000}%
\pgfsetdash{}{0pt}%
\pgfpathmoveto{\pgfqpoint{7.189098in}{0.550000in}}%
\pgfpathlineto{\pgfqpoint{7.216619in}{0.550000in}}%
\pgfpathlineto{\pgfqpoint{7.216619in}{0.752858in}}%
\pgfpathlineto{\pgfqpoint{7.189098in}{0.752858in}}%
\pgfpathlineto{\pgfqpoint{7.189098in}{0.550000in}}%
\pgfpathclose%
\pgfusepath{fill}%
\end{pgfscope}%
\begin{pgfscope}%
\pgfpathrectangle{\pgfqpoint{1.250000in}{0.550000in}}{\pgfqpoint{7.750000in}{3.850000in}}%
\pgfusepath{clip}%
\pgfsetbuttcap%
\pgfsetmiterjoin%
\definecolor{currentfill}{rgb}{0.501961,0.501961,0.501961}%
\pgfsetfillcolor{currentfill}%
\pgfsetfillopacity{0.700000}%
\pgfsetlinewidth{0.000000pt}%
\definecolor{currentstroke}{rgb}{0.000000,0.000000,0.000000}%
\pgfsetstrokecolor{currentstroke}%
\pgfsetstrokeopacity{0.700000}%
\pgfsetdash{}{0pt}%
\pgfpathmoveto{\pgfqpoint{7.216619in}{0.550000in}}%
\pgfpathlineto{\pgfqpoint{7.244141in}{0.550000in}}%
\pgfpathlineto{\pgfqpoint{7.244141in}{0.744158in}}%
\pgfpathlineto{\pgfqpoint{7.216619in}{0.744158in}}%
\pgfpathlineto{\pgfqpoint{7.216619in}{0.550000in}}%
\pgfpathclose%
\pgfusepath{fill}%
\end{pgfscope}%
\begin{pgfscope}%
\pgfpathrectangle{\pgfqpoint{1.250000in}{0.550000in}}{\pgfqpoint{7.750000in}{3.850000in}}%
\pgfusepath{clip}%
\pgfsetbuttcap%
\pgfsetmiterjoin%
\definecolor{currentfill}{rgb}{0.501961,0.501961,0.501961}%
\pgfsetfillcolor{currentfill}%
\pgfsetfillopacity{0.700000}%
\pgfsetlinewidth{0.000000pt}%
\definecolor{currentstroke}{rgb}{0.000000,0.000000,0.000000}%
\pgfsetstrokecolor{currentstroke}%
\pgfsetstrokeopacity{0.700000}%
\pgfsetdash{}{0pt}%
\pgfpathmoveto{\pgfqpoint{7.244141in}{0.550000in}}%
\pgfpathlineto{\pgfqpoint{7.271662in}{0.550000in}}%
\pgfpathlineto{\pgfqpoint{7.271662in}{0.646281in}}%
\pgfpathlineto{\pgfqpoint{7.244141in}{0.646281in}}%
\pgfpathlineto{\pgfqpoint{7.244141in}{0.550000in}}%
\pgfpathclose%
\pgfusepath{fill}%
\end{pgfscope}%
\begin{pgfscope}%
\pgfpathrectangle{\pgfqpoint{1.250000in}{0.550000in}}{\pgfqpoint{7.750000in}{3.850000in}}%
\pgfusepath{clip}%
\pgfsetbuttcap%
\pgfsetmiterjoin%
\definecolor{currentfill}{rgb}{0.501961,0.501961,0.501961}%
\pgfsetfillcolor{currentfill}%
\pgfsetfillopacity{0.700000}%
\pgfsetlinewidth{0.000000pt}%
\definecolor{currentstroke}{rgb}{0.000000,0.000000,0.000000}%
\pgfsetstrokecolor{currentstroke}%
\pgfsetstrokeopacity{0.700000}%
\pgfsetdash{}{0pt}%
\pgfpathmoveto{\pgfqpoint{7.271662in}{0.550000in}}%
\pgfpathlineto{\pgfqpoint{7.299183in}{0.550000in}}%
\pgfpathlineto{\pgfqpoint{7.299183in}{0.744738in}}%
\pgfpathlineto{\pgfqpoint{7.271662in}{0.744738in}}%
\pgfpathlineto{\pgfqpoint{7.271662in}{0.550000in}}%
\pgfpathclose%
\pgfusepath{fill}%
\end{pgfscope}%
\begin{pgfscope}%
\pgfpathrectangle{\pgfqpoint{1.250000in}{0.550000in}}{\pgfqpoint{7.750000in}{3.850000in}}%
\pgfusepath{clip}%
\pgfsetbuttcap%
\pgfsetmiterjoin%
\definecolor{currentfill}{rgb}{0.501961,0.501961,0.501961}%
\pgfsetfillcolor{currentfill}%
\pgfsetfillopacity{0.700000}%
\pgfsetlinewidth{0.000000pt}%
\definecolor{currentstroke}{rgb}{0.000000,0.000000,0.000000}%
\pgfsetstrokecolor{currentstroke}%
\pgfsetstrokeopacity{0.700000}%
\pgfsetdash{}{0pt}%
\pgfpathmoveto{\pgfqpoint{7.299183in}{0.550000in}}%
\pgfpathlineto{\pgfqpoint{7.326705in}{0.550000in}}%
\pgfpathlineto{\pgfqpoint{7.326705in}{0.656141in}}%
\pgfpathlineto{\pgfqpoint{7.299183in}{0.656141in}}%
\pgfpathlineto{\pgfqpoint{7.299183in}{0.550000in}}%
\pgfpathclose%
\pgfusepath{fill}%
\end{pgfscope}%
\begin{pgfscope}%
\pgfpathrectangle{\pgfqpoint{1.250000in}{0.550000in}}{\pgfqpoint{7.750000in}{3.850000in}}%
\pgfusepath{clip}%
\pgfsetbuttcap%
\pgfsetmiterjoin%
\definecolor{currentfill}{rgb}{0.501961,0.501961,0.501961}%
\pgfsetfillcolor{currentfill}%
\pgfsetfillopacity{0.700000}%
\pgfsetlinewidth{0.000000pt}%
\definecolor{currentstroke}{rgb}{0.000000,0.000000,0.000000}%
\pgfsetstrokecolor{currentstroke}%
\pgfsetstrokeopacity{0.700000}%
\pgfsetdash{}{0pt}%
\pgfpathmoveto{\pgfqpoint{7.326705in}{0.550000in}}%
\pgfpathlineto{\pgfqpoint{7.354226in}{0.550000in}}%
\pgfpathlineto{\pgfqpoint{7.354226in}{0.741693in}}%
\pgfpathlineto{\pgfqpoint{7.326705in}{0.741693in}}%
\pgfpathlineto{\pgfqpoint{7.326705in}{0.550000in}}%
\pgfpathclose%
\pgfusepath{fill}%
\end{pgfscope}%
\begin{pgfscope}%
\pgfpathrectangle{\pgfqpoint{1.250000in}{0.550000in}}{\pgfqpoint{7.750000in}{3.850000in}}%
\pgfusepath{clip}%
\pgfsetbuttcap%
\pgfsetmiterjoin%
\definecolor{currentfill}{rgb}{0.501961,0.501961,0.501961}%
\pgfsetfillcolor{currentfill}%
\pgfsetfillopacity{0.700000}%
\pgfsetlinewidth{0.000000pt}%
\definecolor{currentstroke}{rgb}{0.000000,0.000000,0.000000}%
\pgfsetstrokecolor{currentstroke}%
\pgfsetstrokeopacity{0.700000}%
\pgfsetdash{}{0pt}%
\pgfpathmoveto{\pgfqpoint{7.354226in}{0.550000in}}%
\pgfpathlineto{\pgfqpoint{7.381747in}{0.550000in}}%
\pgfpathlineto{\pgfqpoint{7.381747in}{0.737633in}}%
\pgfpathlineto{\pgfqpoint{7.354226in}{0.737633in}}%
\pgfpathlineto{\pgfqpoint{7.354226in}{0.550000in}}%
\pgfpathclose%
\pgfusepath{fill}%
\end{pgfscope}%
\begin{pgfscope}%
\pgfpathrectangle{\pgfqpoint{1.250000in}{0.550000in}}{\pgfqpoint{7.750000in}{3.850000in}}%
\pgfusepath{clip}%
\pgfsetbuttcap%
\pgfsetmiterjoin%
\definecolor{currentfill}{rgb}{0.501961,0.501961,0.501961}%
\pgfsetfillcolor{currentfill}%
\pgfsetfillopacity{0.700000}%
\pgfsetlinewidth{0.000000pt}%
\definecolor{currentstroke}{rgb}{0.000000,0.000000,0.000000}%
\pgfsetstrokecolor{currentstroke}%
\pgfsetstrokeopacity{0.700000}%
\pgfsetdash{}{0pt}%
\pgfpathmoveto{\pgfqpoint{7.381747in}{0.550000in}}%
\pgfpathlineto{\pgfqpoint{7.409268in}{0.550000in}}%
\pgfpathlineto{\pgfqpoint{7.409268in}{0.732848in}}%
\pgfpathlineto{\pgfqpoint{7.381747in}{0.732848in}}%
\pgfpathlineto{\pgfqpoint{7.381747in}{0.550000in}}%
\pgfpathclose%
\pgfusepath{fill}%
\end{pgfscope}%
\begin{pgfscope}%
\pgfpathrectangle{\pgfqpoint{1.250000in}{0.550000in}}{\pgfqpoint{7.750000in}{3.850000in}}%
\pgfusepath{clip}%
\pgfsetbuttcap%
\pgfsetmiterjoin%
\definecolor{currentfill}{rgb}{0.501961,0.501961,0.501961}%
\pgfsetfillcolor{currentfill}%
\pgfsetfillopacity{0.700000}%
\pgfsetlinewidth{0.000000pt}%
\definecolor{currentstroke}{rgb}{0.000000,0.000000,0.000000}%
\pgfsetstrokecolor{currentstroke}%
\pgfsetstrokeopacity{0.700000}%
\pgfsetdash{}{0pt}%
\pgfpathmoveto{\pgfqpoint{7.409268in}{0.550000in}}%
\pgfpathlineto{\pgfqpoint{7.436790in}{0.550000in}}%
\pgfpathlineto{\pgfqpoint{7.436790in}{0.635116in}}%
\pgfpathlineto{\pgfqpoint{7.409268in}{0.635116in}}%
\pgfpathlineto{\pgfqpoint{7.409268in}{0.550000in}}%
\pgfpathclose%
\pgfusepath{fill}%
\end{pgfscope}%
\begin{pgfscope}%
\pgfpathrectangle{\pgfqpoint{1.250000in}{0.550000in}}{\pgfqpoint{7.750000in}{3.850000in}}%
\pgfusepath{clip}%
\pgfsetbuttcap%
\pgfsetmiterjoin%
\definecolor{currentfill}{rgb}{0.501961,0.501961,0.501961}%
\pgfsetfillcolor{currentfill}%
\pgfsetfillopacity{0.700000}%
\pgfsetlinewidth{0.000000pt}%
\definecolor{currentstroke}{rgb}{0.000000,0.000000,0.000000}%
\pgfsetstrokecolor{currentstroke}%
\pgfsetstrokeopacity{0.700000}%
\pgfsetdash{}{0pt}%
\pgfpathmoveto{\pgfqpoint{7.436790in}{0.550000in}}%
\pgfpathlineto{\pgfqpoint{7.464311in}{0.550000in}}%
\pgfpathlineto{\pgfqpoint{7.464311in}{0.722697in}}%
\pgfpathlineto{\pgfqpoint{7.436790in}{0.722697in}}%
\pgfpathlineto{\pgfqpoint{7.436790in}{0.550000in}}%
\pgfpathclose%
\pgfusepath{fill}%
\end{pgfscope}%
\begin{pgfscope}%
\pgfpathrectangle{\pgfqpoint{1.250000in}{0.550000in}}{\pgfqpoint{7.750000in}{3.850000in}}%
\pgfusepath{clip}%
\pgfsetbuttcap%
\pgfsetmiterjoin%
\definecolor{currentfill}{rgb}{0.501961,0.501961,0.501961}%
\pgfsetfillcolor{currentfill}%
\pgfsetfillopacity{0.700000}%
\pgfsetlinewidth{0.000000pt}%
\definecolor{currentstroke}{rgb}{0.000000,0.000000,0.000000}%
\pgfsetstrokecolor{currentstroke}%
\pgfsetstrokeopacity{0.700000}%
\pgfsetdash{}{0pt}%
\pgfpathmoveto{\pgfqpoint{7.464311in}{0.550000in}}%
\pgfpathlineto{\pgfqpoint{7.491832in}{0.550000in}}%
\pgfpathlineto{\pgfqpoint{7.491832in}{0.712402in}}%
\pgfpathlineto{\pgfqpoint{7.464311in}{0.712402in}}%
\pgfpathlineto{\pgfqpoint{7.464311in}{0.550000in}}%
\pgfpathclose%
\pgfusepath{fill}%
\end{pgfscope}%
\begin{pgfscope}%
\pgfpathrectangle{\pgfqpoint{1.250000in}{0.550000in}}{\pgfqpoint{7.750000in}{3.850000in}}%
\pgfusepath{clip}%
\pgfsetbuttcap%
\pgfsetmiterjoin%
\definecolor{currentfill}{rgb}{0.501961,0.501961,0.501961}%
\pgfsetfillcolor{currentfill}%
\pgfsetfillopacity{0.700000}%
\pgfsetlinewidth{0.000000pt}%
\definecolor{currentstroke}{rgb}{0.000000,0.000000,0.000000}%
\pgfsetstrokecolor{currentstroke}%
\pgfsetstrokeopacity{0.700000}%
\pgfsetdash{}{0pt}%
\pgfpathmoveto{\pgfqpoint{7.491832in}{0.550000in}}%
\pgfpathlineto{\pgfqpoint{7.519354in}{0.550000in}}%
\pgfpathlineto{\pgfqpoint{7.519354in}{0.711967in}}%
\pgfpathlineto{\pgfqpoint{7.491832in}{0.711967in}}%
\pgfpathlineto{\pgfqpoint{7.491832in}{0.550000in}}%
\pgfpathclose%
\pgfusepath{fill}%
\end{pgfscope}%
\begin{pgfscope}%
\pgfpathrectangle{\pgfqpoint{1.250000in}{0.550000in}}{\pgfqpoint{7.750000in}{3.850000in}}%
\pgfusepath{clip}%
\pgfsetbuttcap%
\pgfsetmiterjoin%
\definecolor{currentfill}{rgb}{0.501961,0.501961,0.501961}%
\pgfsetfillcolor{currentfill}%
\pgfsetfillopacity{0.700000}%
\pgfsetlinewidth{0.000000pt}%
\definecolor{currentstroke}{rgb}{0.000000,0.000000,0.000000}%
\pgfsetstrokecolor{currentstroke}%
\pgfsetstrokeopacity{0.700000}%
\pgfsetdash{}{0pt}%
\pgfpathmoveto{\pgfqpoint{7.519354in}{0.550000in}}%
\pgfpathlineto{\pgfqpoint{7.546875in}{0.550000in}}%
\pgfpathlineto{\pgfqpoint{7.546875in}{0.767503in}}%
\pgfpathlineto{\pgfqpoint{7.519354in}{0.767503in}}%
\pgfpathlineto{\pgfqpoint{7.519354in}{0.550000in}}%
\pgfpathclose%
\pgfusepath{fill}%
\end{pgfscope}%
\begin{pgfscope}%
\pgfpathrectangle{\pgfqpoint{1.250000in}{0.550000in}}{\pgfqpoint{7.750000in}{3.850000in}}%
\pgfusepath{clip}%
\pgfsetbuttcap%
\pgfsetmiterjoin%
\definecolor{currentfill}{rgb}{0.501961,0.501961,0.501961}%
\pgfsetfillcolor{currentfill}%
\pgfsetfillopacity{0.700000}%
\pgfsetlinewidth{0.000000pt}%
\definecolor{currentstroke}{rgb}{0.000000,0.000000,0.000000}%
\pgfsetstrokecolor{currentstroke}%
\pgfsetstrokeopacity{0.700000}%
\pgfsetdash{}{0pt}%
\pgfpathmoveto{\pgfqpoint{7.546875in}{0.550000in}}%
\pgfpathlineto{\pgfqpoint{7.574396in}{0.550000in}}%
\pgfpathlineto{\pgfqpoint{7.574396in}{0.686012in}}%
\pgfpathlineto{\pgfqpoint{7.546875in}{0.686012in}}%
\pgfpathlineto{\pgfqpoint{7.546875in}{0.550000in}}%
\pgfpathclose%
\pgfusepath{fill}%
\end{pgfscope}%
\begin{pgfscope}%
\pgfpathrectangle{\pgfqpoint{1.250000in}{0.550000in}}{\pgfqpoint{7.750000in}{3.850000in}}%
\pgfusepath{clip}%
\pgfsetbuttcap%
\pgfsetmiterjoin%
\definecolor{currentfill}{rgb}{0.501961,0.501961,0.501961}%
\pgfsetfillcolor{currentfill}%
\pgfsetfillopacity{0.700000}%
\pgfsetlinewidth{0.000000pt}%
\definecolor{currentstroke}{rgb}{0.000000,0.000000,0.000000}%
\pgfsetstrokecolor{currentstroke}%
\pgfsetstrokeopacity{0.700000}%
\pgfsetdash{}{0pt}%
\pgfpathmoveto{\pgfqpoint{7.574396in}{0.550000in}}%
\pgfpathlineto{\pgfqpoint{7.601918in}{0.550000in}}%
\pgfpathlineto{\pgfqpoint{7.601918in}{0.686737in}}%
\pgfpathlineto{\pgfqpoint{7.574396in}{0.686737in}}%
\pgfpathlineto{\pgfqpoint{7.574396in}{0.550000in}}%
\pgfpathclose%
\pgfusepath{fill}%
\end{pgfscope}%
\begin{pgfscope}%
\pgfpathrectangle{\pgfqpoint{1.250000in}{0.550000in}}{\pgfqpoint{7.750000in}{3.850000in}}%
\pgfusepath{clip}%
\pgfsetbuttcap%
\pgfsetmiterjoin%
\definecolor{currentfill}{rgb}{0.501961,0.501961,0.501961}%
\pgfsetfillcolor{currentfill}%
\pgfsetfillopacity{0.700000}%
\pgfsetlinewidth{0.000000pt}%
\definecolor{currentstroke}{rgb}{0.000000,0.000000,0.000000}%
\pgfsetstrokecolor{currentstroke}%
\pgfsetstrokeopacity{0.700000}%
\pgfsetdash{}{0pt}%
\pgfpathmoveto{\pgfqpoint{7.601918in}{0.550000in}}%
\pgfpathlineto{\pgfqpoint{7.629439in}{0.550000in}}%
\pgfpathlineto{\pgfqpoint{7.629439in}{0.741983in}}%
\pgfpathlineto{\pgfqpoint{7.601918in}{0.741983in}}%
\pgfpathlineto{\pgfqpoint{7.601918in}{0.550000in}}%
\pgfpathclose%
\pgfusepath{fill}%
\end{pgfscope}%
\begin{pgfscope}%
\pgfpathrectangle{\pgfqpoint{1.250000in}{0.550000in}}{\pgfqpoint{7.750000in}{3.850000in}}%
\pgfusepath{clip}%
\pgfsetbuttcap%
\pgfsetmiterjoin%
\definecolor{currentfill}{rgb}{0.501961,0.501961,0.501961}%
\pgfsetfillcolor{currentfill}%
\pgfsetfillopacity{0.700000}%
\pgfsetlinewidth{0.000000pt}%
\definecolor{currentstroke}{rgb}{0.000000,0.000000,0.000000}%
\pgfsetstrokecolor{currentstroke}%
\pgfsetstrokeopacity{0.700000}%
\pgfsetdash{}{0pt}%
\pgfpathmoveto{\pgfqpoint{7.629439in}{0.550000in}}%
\pgfpathlineto{\pgfqpoint{7.656960in}{0.550000in}}%
\pgfpathlineto{\pgfqpoint{7.656960in}{0.679052in}}%
\pgfpathlineto{\pgfqpoint{7.629439in}{0.679052in}}%
\pgfpathlineto{\pgfqpoint{7.629439in}{0.550000in}}%
\pgfpathclose%
\pgfusepath{fill}%
\end{pgfscope}%
\begin{pgfscope}%
\pgfpathrectangle{\pgfqpoint{1.250000in}{0.550000in}}{\pgfqpoint{7.750000in}{3.850000in}}%
\pgfusepath{clip}%
\pgfsetbuttcap%
\pgfsetmiterjoin%
\definecolor{currentfill}{rgb}{0.501961,0.501961,0.501961}%
\pgfsetfillcolor{currentfill}%
\pgfsetfillopacity{0.700000}%
\pgfsetlinewidth{0.000000pt}%
\definecolor{currentstroke}{rgb}{0.000000,0.000000,0.000000}%
\pgfsetstrokecolor{currentstroke}%
\pgfsetstrokeopacity{0.700000}%
\pgfsetdash{}{0pt}%
\pgfpathmoveto{\pgfqpoint{7.656960in}{0.550000in}}%
\pgfpathlineto{\pgfqpoint{7.684482in}{0.550000in}}%
\pgfpathlineto{\pgfqpoint{7.684482in}{0.741113in}}%
\pgfpathlineto{\pgfqpoint{7.656960in}{0.741113in}}%
\pgfpathlineto{\pgfqpoint{7.656960in}{0.550000in}}%
\pgfpathclose%
\pgfusepath{fill}%
\end{pgfscope}%
\begin{pgfscope}%
\pgfpathrectangle{\pgfqpoint{1.250000in}{0.550000in}}{\pgfqpoint{7.750000in}{3.850000in}}%
\pgfusepath{clip}%
\pgfsetbuttcap%
\pgfsetmiterjoin%
\definecolor{currentfill}{rgb}{0.501961,0.501961,0.501961}%
\pgfsetfillcolor{currentfill}%
\pgfsetfillopacity{0.700000}%
\pgfsetlinewidth{0.000000pt}%
\definecolor{currentstroke}{rgb}{0.000000,0.000000,0.000000}%
\pgfsetstrokecolor{currentstroke}%
\pgfsetstrokeopacity{0.700000}%
\pgfsetdash{}{0pt}%
\pgfpathmoveto{\pgfqpoint{7.684482in}{0.550000in}}%
\pgfpathlineto{\pgfqpoint{7.712003in}{0.550000in}}%
\pgfpathlineto{\pgfqpoint{7.712003in}{0.732558in}}%
\pgfpathlineto{\pgfqpoint{7.684482in}{0.732558in}}%
\pgfpathlineto{\pgfqpoint{7.684482in}{0.550000in}}%
\pgfpathclose%
\pgfusepath{fill}%
\end{pgfscope}%
\begin{pgfscope}%
\pgfpathrectangle{\pgfqpoint{1.250000in}{0.550000in}}{\pgfqpoint{7.750000in}{3.850000in}}%
\pgfusepath{clip}%
\pgfsetbuttcap%
\pgfsetmiterjoin%
\definecolor{currentfill}{rgb}{0.501961,0.501961,0.501961}%
\pgfsetfillcolor{currentfill}%
\pgfsetfillopacity{0.700000}%
\pgfsetlinewidth{0.000000pt}%
\definecolor{currentstroke}{rgb}{0.000000,0.000000,0.000000}%
\pgfsetstrokecolor{currentstroke}%
\pgfsetstrokeopacity{0.700000}%
\pgfsetdash{}{0pt}%
\pgfpathmoveto{\pgfqpoint{7.712003in}{0.550000in}}%
\pgfpathlineto{\pgfqpoint{7.739524in}{0.550000in}}%
\pgfpathlineto{\pgfqpoint{7.739524in}{0.671367in}}%
\pgfpathlineto{\pgfqpoint{7.712003in}{0.671367in}}%
\pgfpathlineto{\pgfqpoint{7.712003in}{0.550000in}}%
\pgfpathclose%
\pgfusepath{fill}%
\end{pgfscope}%
\begin{pgfscope}%
\pgfpathrectangle{\pgfqpoint{1.250000in}{0.550000in}}{\pgfqpoint{7.750000in}{3.850000in}}%
\pgfusepath{clip}%
\pgfsetbuttcap%
\pgfsetmiterjoin%
\definecolor{currentfill}{rgb}{0.501961,0.501961,0.501961}%
\pgfsetfillcolor{currentfill}%
\pgfsetfillopacity{0.700000}%
\pgfsetlinewidth{0.000000pt}%
\definecolor{currentstroke}{rgb}{0.000000,0.000000,0.000000}%
\pgfsetstrokecolor{currentstroke}%
\pgfsetstrokeopacity{0.700000}%
\pgfsetdash{}{0pt}%
\pgfpathmoveto{\pgfqpoint{7.739524in}{0.550000in}}%
\pgfpathlineto{\pgfqpoint{7.767045in}{0.550000in}}%
\pgfpathlineto{\pgfqpoint{7.767045in}{0.734588in}}%
\pgfpathlineto{\pgfqpoint{7.739524in}{0.734588in}}%
\pgfpathlineto{\pgfqpoint{7.739524in}{0.550000in}}%
\pgfpathclose%
\pgfusepath{fill}%
\end{pgfscope}%
\begin{pgfscope}%
\pgfpathrectangle{\pgfqpoint{1.250000in}{0.550000in}}{\pgfqpoint{7.750000in}{3.850000in}}%
\pgfusepath{clip}%
\pgfsetbuttcap%
\pgfsetmiterjoin%
\definecolor{currentfill}{rgb}{0.501961,0.501961,0.501961}%
\pgfsetfillcolor{currentfill}%
\pgfsetfillopacity{0.700000}%
\pgfsetlinewidth{0.000000pt}%
\definecolor{currentstroke}{rgb}{0.000000,0.000000,0.000000}%
\pgfsetstrokecolor{currentstroke}%
\pgfsetstrokeopacity{0.700000}%
\pgfsetdash{}{0pt}%
\pgfpathmoveto{\pgfqpoint{7.767045in}{0.550000in}}%
\pgfpathlineto{\pgfqpoint{7.794567in}{0.550000in}}%
\pgfpathlineto{\pgfqpoint{7.794567in}{0.726612in}}%
\pgfpathlineto{\pgfqpoint{7.767045in}{0.726612in}}%
\pgfpathlineto{\pgfqpoint{7.767045in}{0.550000in}}%
\pgfpathclose%
\pgfusepath{fill}%
\end{pgfscope}%
\begin{pgfscope}%
\pgfpathrectangle{\pgfqpoint{1.250000in}{0.550000in}}{\pgfqpoint{7.750000in}{3.850000in}}%
\pgfusepath{clip}%
\pgfsetbuttcap%
\pgfsetmiterjoin%
\definecolor{currentfill}{rgb}{0.501961,0.501961,0.501961}%
\pgfsetfillcolor{currentfill}%
\pgfsetfillopacity{0.700000}%
\pgfsetlinewidth{0.000000pt}%
\definecolor{currentstroke}{rgb}{0.000000,0.000000,0.000000}%
\pgfsetstrokecolor{currentstroke}%
\pgfsetstrokeopacity{0.700000}%
\pgfsetdash{}{0pt}%
\pgfpathmoveto{\pgfqpoint{7.794567in}{0.550000in}}%
\pgfpathlineto{\pgfqpoint{7.822088in}{0.550000in}}%
\pgfpathlineto{\pgfqpoint{7.822088in}{0.673252in}}%
\pgfpathlineto{\pgfqpoint{7.794567in}{0.673252in}}%
\pgfpathlineto{\pgfqpoint{7.794567in}{0.550000in}}%
\pgfpathclose%
\pgfusepath{fill}%
\end{pgfscope}%
\begin{pgfscope}%
\pgfpathrectangle{\pgfqpoint{1.250000in}{0.550000in}}{\pgfqpoint{7.750000in}{3.850000in}}%
\pgfusepath{clip}%
\pgfsetbuttcap%
\pgfsetmiterjoin%
\definecolor{currentfill}{rgb}{0.501961,0.501961,0.501961}%
\pgfsetfillcolor{currentfill}%
\pgfsetfillopacity{0.700000}%
\pgfsetlinewidth{0.000000pt}%
\definecolor{currentstroke}{rgb}{0.000000,0.000000,0.000000}%
\pgfsetstrokecolor{currentstroke}%
\pgfsetstrokeopacity{0.700000}%
\pgfsetdash{}{0pt}%
\pgfpathmoveto{\pgfqpoint{7.822088in}{0.550000in}}%
\pgfpathlineto{\pgfqpoint{7.849609in}{0.550000in}}%
\pgfpathlineto{\pgfqpoint{7.849609in}{0.706892in}}%
\pgfpathlineto{\pgfqpoint{7.822088in}{0.706892in}}%
\pgfpathlineto{\pgfqpoint{7.822088in}{0.550000in}}%
\pgfpathclose%
\pgfusepath{fill}%
\end{pgfscope}%
\begin{pgfscope}%
\pgfpathrectangle{\pgfqpoint{1.250000in}{0.550000in}}{\pgfqpoint{7.750000in}{3.850000in}}%
\pgfusepath{clip}%
\pgfsetbuttcap%
\pgfsetmiterjoin%
\definecolor{currentfill}{rgb}{0.501961,0.501961,0.501961}%
\pgfsetfillcolor{currentfill}%
\pgfsetfillopacity{0.700000}%
\pgfsetlinewidth{0.000000pt}%
\definecolor{currentstroke}{rgb}{0.000000,0.000000,0.000000}%
\pgfsetstrokecolor{currentstroke}%
\pgfsetstrokeopacity{0.700000}%
\pgfsetdash{}{0pt}%
\pgfpathmoveto{\pgfqpoint{7.849609in}{0.550000in}}%
\pgfpathlineto{\pgfqpoint{7.877131in}{0.550000in}}%
\pgfpathlineto{\pgfqpoint{7.877131in}{0.745608in}}%
\pgfpathlineto{\pgfqpoint{7.849609in}{0.745608in}}%
\pgfpathlineto{\pgfqpoint{7.849609in}{0.550000in}}%
\pgfpathclose%
\pgfusepath{fill}%
\end{pgfscope}%
\begin{pgfscope}%
\pgfpathrectangle{\pgfqpoint{1.250000in}{0.550000in}}{\pgfqpoint{7.750000in}{3.850000in}}%
\pgfusepath{clip}%
\pgfsetbuttcap%
\pgfsetmiterjoin%
\definecolor{currentfill}{rgb}{0.501961,0.501961,0.501961}%
\pgfsetfillcolor{currentfill}%
\pgfsetfillopacity{0.700000}%
\pgfsetlinewidth{0.000000pt}%
\definecolor{currentstroke}{rgb}{0.000000,0.000000,0.000000}%
\pgfsetstrokecolor{currentstroke}%
\pgfsetstrokeopacity{0.700000}%
\pgfsetdash{}{0pt}%
\pgfpathmoveto{\pgfqpoint{7.877131in}{0.550000in}}%
\pgfpathlineto{\pgfqpoint{7.904652in}{0.550000in}}%
\pgfpathlineto{\pgfqpoint{7.904652in}{0.690072in}}%
\pgfpathlineto{\pgfqpoint{7.877131in}{0.690072in}}%
\pgfpathlineto{\pgfqpoint{7.877131in}{0.550000in}}%
\pgfpathclose%
\pgfusepath{fill}%
\end{pgfscope}%
\begin{pgfscope}%
\pgfpathrectangle{\pgfqpoint{1.250000in}{0.550000in}}{\pgfqpoint{7.750000in}{3.850000in}}%
\pgfusepath{clip}%
\pgfsetbuttcap%
\pgfsetmiterjoin%
\definecolor{currentfill}{rgb}{0.501961,0.501961,0.501961}%
\pgfsetfillcolor{currentfill}%
\pgfsetfillopacity{0.700000}%
\pgfsetlinewidth{0.000000pt}%
\definecolor{currentstroke}{rgb}{0.000000,0.000000,0.000000}%
\pgfsetstrokecolor{currentstroke}%
\pgfsetstrokeopacity{0.700000}%
\pgfsetdash{}{0pt}%
\pgfpathmoveto{\pgfqpoint{7.904652in}{0.550000in}}%
\pgfpathlineto{\pgfqpoint{7.932173in}{0.550000in}}%
\pgfpathlineto{\pgfqpoint{7.932173in}{0.722407in}}%
\pgfpathlineto{\pgfqpoint{7.904652in}{0.722407in}}%
\pgfpathlineto{\pgfqpoint{7.904652in}{0.550000in}}%
\pgfpathclose%
\pgfusepath{fill}%
\end{pgfscope}%
\begin{pgfscope}%
\pgfpathrectangle{\pgfqpoint{1.250000in}{0.550000in}}{\pgfqpoint{7.750000in}{3.850000in}}%
\pgfusepath{clip}%
\pgfsetbuttcap%
\pgfsetmiterjoin%
\definecolor{currentfill}{rgb}{0.501961,0.501961,0.501961}%
\pgfsetfillcolor{currentfill}%
\pgfsetfillopacity{0.700000}%
\pgfsetlinewidth{0.000000pt}%
\definecolor{currentstroke}{rgb}{0.000000,0.000000,0.000000}%
\pgfsetstrokecolor{currentstroke}%
\pgfsetstrokeopacity{0.700000}%
\pgfsetdash{}{0pt}%
\pgfpathmoveto{\pgfqpoint{7.932173in}{0.550000in}}%
\pgfpathlineto{\pgfqpoint{7.959695in}{0.550000in}}%
\pgfpathlineto{\pgfqpoint{7.959695in}{0.706892in}}%
\pgfpathlineto{\pgfqpoint{7.932173in}{0.706892in}}%
\pgfpathlineto{\pgfqpoint{7.932173in}{0.550000in}}%
\pgfpathclose%
\pgfusepath{fill}%
\end{pgfscope}%
\begin{pgfscope}%
\pgfpathrectangle{\pgfqpoint{1.250000in}{0.550000in}}{\pgfqpoint{7.750000in}{3.850000in}}%
\pgfusepath{clip}%
\pgfsetbuttcap%
\pgfsetmiterjoin%
\definecolor{currentfill}{rgb}{0.501961,0.501961,0.501961}%
\pgfsetfillcolor{currentfill}%
\pgfsetfillopacity{0.700000}%
\pgfsetlinewidth{0.000000pt}%
\definecolor{currentstroke}{rgb}{0.000000,0.000000,0.000000}%
\pgfsetstrokecolor{currentstroke}%
\pgfsetstrokeopacity{0.700000}%
\pgfsetdash{}{0pt}%
\pgfpathmoveto{\pgfqpoint{7.959695in}{0.550000in}}%
\pgfpathlineto{\pgfqpoint{7.987216in}{0.550000in}}%
\pgfpathlineto{\pgfqpoint{7.987216in}{0.703992in}}%
\pgfpathlineto{\pgfqpoint{7.959695in}{0.703992in}}%
\pgfpathlineto{\pgfqpoint{7.959695in}{0.550000in}}%
\pgfpathclose%
\pgfusepath{fill}%
\end{pgfscope}%
\begin{pgfscope}%
\pgfpathrectangle{\pgfqpoint{1.250000in}{0.550000in}}{\pgfqpoint{7.750000in}{3.850000in}}%
\pgfusepath{clip}%
\pgfsetbuttcap%
\pgfsetmiterjoin%
\definecolor{currentfill}{rgb}{0.501961,0.501961,0.501961}%
\pgfsetfillcolor{currentfill}%
\pgfsetfillopacity{0.700000}%
\pgfsetlinewidth{0.000000pt}%
\definecolor{currentstroke}{rgb}{0.000000,0.000000,0.000000}%
\pgfsetstrokecolor{currentstroke}%
\pgfsetstrokeopacity{0.700000}%
\pgfsetdash{}{0pt}%
\pgfpathmoveto{\pgfqpoint{7.987216in}{0.550000in}}%
\pgfpathlineto{\pgfqpoint{8.014737in}{0.550000in}}%
\pgfpathlineto{\pgfqpoint{8.014737in}{0.726902in}}%
\pgfpathlineto{\pgfqpoint{7.987216in}{0.726902in}}%
\pgfpathlineto{\pgfqpoint{7.987216in}{0.550000in}}%
\pgfpathclose%
\pgfusepath{fill}%
\end{pgfscope}%
\begin{pgfscope}%
\pgfpathrectangle{\pgfqpoint{1.250000in}{0.550000in}}{\pgfqpoint{7.750000in}{3.850000in}}%
\pgfusepath{clip}%
\pgfsetbuttcap%
\pgfsetmiterjoin%
\definecolor{currentfill}{rgb}{0.501961,0.501961,0.501961}%
\pgfsetfillcolor{currentfill}%
\pgfsetfillopacity{0.700000}%
\pgfsetlinewidth{0.000000pt}%
\definecolor{currentstroke}{rgb}{0.000000,0.000000,0.000000}%
\pgfsetstrokecolor{currentstroke}%
\pgfsetstrokeopacity{0.700000}%
\pgfsetdash{}{0pt}%
\pgfpathmoveto{\pgfqpoint{8.014737in}{0.550000in}}%
\pgfpathlineto{\pgfqpoint{8.042259in}{0.550000in}}%
\pgfpathlineto{\pgfqpoint{8.042259in}{0.692682in}}%
\pgfpathlineto{\pgfqpoint{8.014737in}{0.692682in}}%
\pgfpathlineto{\pgfqpoint{8.014737in}{0.550000in}}%
\pgfpathclose%
\pgfusepath{fill}%
\end{pgfscope}%
\begin{pgfscope}%
\pgfpathrectangle{\pgfqpoint{1.250000in}{0.550000in}}{\pgfqpoint{7.750000in}{3.850000in}}%
\pgfusepath{clip}%
\pgfsetbuttcap%
\pgfsetmiterjoin%
\definecolor{currentfill}{rgb}{0.501961,0.501961,0.501961}%
\pgfsetfillcolor{currentfill}%
\pgfsetfillopacity{0.700000}%
\pgfsetlinewidth{0.000000pt}%
\definecolor{currentstroke}{rgb}{0.000000,0.000000,0.000000}%
\pgfsetstrokecolor{currentstroke}%
\pgfsetstrokeopacity{0.700000}%
\pgfsetdash{}{0pt}%
\pgfpathmoveto{\pgfqpoint{8.042259in}{0.550000in}}%
\pgfpathlineto{\pgfqpoint{8.069780in}{0.550000in}}%
\pgfpathlineto{\pgfqpoint{8.069780in}{0.718637in}}%
\pgfpathlineto{\pgfqpoint{8.042259in}{0.718637in}}%
\pgfpathlineto{\pgfqpoint{8.042259in}{0.550000in}}%
\pgfpathclose%
\pgfusepath{fill}%
\end{pgfscope}%
\begin{pgfscope}%
\pgfpathrectangle{\pgfqpoint{1.250000in}{0.550000in}}{\pgfqpoint{7.750000in}{3.850000in}}%
\pgfusepath{clip}%
\pgfsetbuttcap%
\pgfsetmiterjoin%
\definecolor{currentfill}{rgb}{0.501961,0.501961,0.501961}%
\pgfsetfillcolor{currentfill}%
\pgfsetfillopacity{0.700000}%
\pgfsetlinewidth{0.000000pt}%
\definecolor{currentstroke}{rgb}{0.000000,0.000000,0.000000}%
\pgfsetstrokecolor{currentstroke}%
\pgfsetstrokeopacity{0.700000}%
\pgfsetdash{}{0pt}%
\pgfpathmoveto{\pgfqpoint{8.069780in}{0.550000in}}%
\pgfpathlineto{\pgfqpoint{8.097301in}{0.550000in}}%
\pgfpathlineto{\pgfqpoint{8.097301in}{0.702542in}}%
\pgfpathlineto{\pgfqpoint{8.069780in}{0.702542in}}%
\pgfpathlineto{\pgfqpoint{8.069780in}{0.550000in}}%
\pgfpathclose%
\pgfusepath{fill}%
\end{pgfscope}%
\begin{pgfscope}%
\pgfpathrectangle{\pgfqpoint{1.250000in}{0.550000in}}{\pgfqpoint{7.750000in}{3.850000in}}%
\pgfusepath{clip}%
\pgfsetbuttcap%
\pgfsetmiterjoin%
\definecolor{currentfill}{rgb}{0.501961,0.501961,0.501961}%
\pgfsetfillcolor{currentfill}%
\pgfsetfillopacity{0.700000}%
\pgfsetlinewidth{0.000000pt}%
\definecolor{currentstroke}{rgb}{0.000000,0.000000,0.000000}%
\pgfsetstrokecolor{currentstroke}%
\pgfsetstrokeopacity{0.700000}%
\pgfsetdash{}{0pt}%
\pgfpathmoveto{\pgfqpoint{8.097301in}{0.550000in}}%
\pgfpathlineto{\pgfqpoint{8.124822in}{0.550000in}}%
\pgfpathlineto{\pgfqpoint{8.124822in}{0.712547in}}%
\pgfpathlineto{\pgfqpoint{8.097301in}{0.712547in}}%
\pgfpathlineto{\pgfqpoint{8.097301in}{0.550000in}}%
\pgfpathclose%
\pgfusepath{fill}%
\end{pgfscope}%
\begin{pgfscope}%
\pgfpathrectangle{\pgfqpoint{1.250000in}{0.550000in}}{\pgfqpoint{7.750000in}{3.850000in}}%
\pgfusepath{clip}%
\pgfsetbuttcap%
\pgfsetmiterjoin%
\definecolor{currentfill}{rgb}{0.501961,0.501961,0.501961}%
\pgfsetfillcolor{currentfill}%
\pgfsetfillopacity{0.700000}%
\pgfsetlinewidth{0.000000pt}%
\definecolor{currentstroke}{rgb}{0.000000,0.000000,0.000000}%
\pgfsetstrokecolor{currentstroke}%
\pgfsetstrokeopacity{0.700000}%
\pgfsetdash{}{0pt}%
\pgfpathmoveto{\pgfqpoint{8.124822in}{0.550000in}}%
\pgfpathlineto{\pgfqpoint{8.152344in}{0.550000in}}%
\pgfpathlineto{\pgfqpoint{8.152344in}{0.700657in}}%
\pgfpathlineto{\pgfqpoint{8.124822in}{0.700657in}}%
\pgfpathlineto{\pgfqpoint{8.124822in}{0.550000in}}%
\pgfpathclose%
\pgfusepath{fill}%
\end{pgfscope}%
\begin{pgfscope}%
\pgfpathrectangle{\pgfqpoint{1.250000in}{0.550000in}}{\pgfqpoint{7.750000in}{3.850000in}}%
\pgfusepath{clip}%
\pgfsetbuttcap%
\pgfsetmiterjoin%
\definecolor{currentfill}{rgb}{0.501961,0.501961,0.501961}%
\pgfsetfillcolor{currentfill}%
\pgfsetfillopacity{0.700000}%
\pgfsetlinewidth{0.000000pt}%
\definecolor{currentstroke}{rgb}{0.000000,0.000000,0.000000}%
\pgfsetstrokecolor{currentstroke}%
\pgfsetstrokeopacity{0.700000}%
\pgfsetdash{}{0pt}%
\pgfpathmoveto{\pgfqpoint{8.152344in}{0.550000in}}%
\pgfpathlineto{\pgfqpoint{8.179865in}{0.550000in}}%
\pgfpathlineto{\pgfqpoint{8.179865in}{0.713272in}}%
\pgfpathlineto{\pgfqpoint{8.152344in}{0.713272in}}%
\pgfpathlineto{\pgfqpoint{8.152344in}{0.550000in}}%
\pgfpathclose%
\pgfusepath{fill}%
\end{pgfscope}%
\begin{pgfscope}%
\pgfpathrectangle{\pgfqpoint{1.250000in}{0.550000in}}{\pgfqpoint{7.750000in}{3.850000in}}%
\pgfusepath{clip}%
\pgfsetbuttcap%
\pgfsetmiterjoin%
\definecolor{currentfill}{rgb}{0.501961,0.501961,0.501961}%
\pgfsetfillcolor{currentfill}%
\pgfsetfillopacity{0.700000}%
\pgfsetlinewidth{0.000000pt}%
\definecolor{currentstroke}{rgb}{0.000000,0.000000,0.000000}%
\pgfsetstrokecolor{currentstroke}%
\pgfsetstrokeopacity{0.700000}%
\pgfsetdash{}{0pt}%
\pgfpathmoveto{\pgfqpoint{8.179865in}{0.550000in}}%
\pgfpathlineto{\pgfqpoint{8.207386in}{0.550000in}}%
\pgfpathlineto{\pgfqpoint{8.207386in}{0.722697in}}%
\pgfpathlineto{\pgfqpoint{8.179865in}{0.722697in}}%
\pgfpathlineto{\pgfqpoint{8.179865in}{0.550000in}}%
\pgfpathclose%
\pgfusepath{fill}%
\end{pgfscope}%
\begin{pgfscope}%
\pgfpathrectangle{\pgfqpoint{1.250000in}{0.550000in}}{\pgfqpoint{7.750000in}{3.850000in}}%
\pgfusepath{clip}%
\pgfsetbuttcap%
\pgfsetmiterjoin%
\definecolor{currentfill}{rgb}{0.501961,0.501961,0.501961}%
\pgfsetfillcolor{currentfill}%
\pgfsetfillopacity{0.700000}%
\pgfsetlinewidth{0.000000pt}%
\definecolor{currentstroke}{rgb}{0.000000,0.000000,0.000000}%
\pgfsetstrokecolor{currentstroke}%
\pgfsetstrokeopacity{0.700000}%
\pgfsetdash{}{0pt}%
\pgfpathmoveto{\pgfqpoint{8.207386in}{0.550000in}}%
\pgfpathlineto{\pgfqpoint{8.234908in}{0.550000in}}%
\pgfpathlineto{\pgfqpoint{8.234908in}{0.716462in}}%
\pgfpathlineto{\pgfqpoint{8.207386in}{0.716462in}}%
\pgfpathlineto{\pgfqpoint{8.207386in}{0.550000in}}%
\pgfpathclose%
\pgfusepath{fill}%
\end{pgfscope}%
\begin{pgfscope}%
\pgfpathrectangle{\pgfqpoint{1.250000in}{0.550000in}}{\pgfqpoint{7.750000in}{3.850000in}}%
\pgfusepath{clip}%
\pgfsetbuttcap%
\pgfsetmiterjoin%
\definecolor{currentfill}{rgb}{0.501961,0.501961,0.501961}%
\pgfsetfillcolor{currentfill}%
\pgfsetfillopacity{0.700000}%
\pgfsetlinewidth{0.000000pt}%
\definecolor{currentstroke}{rgb}{0.000000,0.000000,0.000000}%
\pgfsetstrokecolor{currentstroke}%
\pgfsetstrokeopacity{0.700000}%
\pgfsetdash{}{0pt}%
\pgfpathmoveto{\pgfqpoint{8.234908in}{0.550000in}}%
\pgfpathlineto{\pgfqpoint{8.262429in}{0.550000in}}%
\pgfpathlineto{\pgfqpoint{8.262429in}{0.694857in}}%
\pgfpathlineto{\pgfqpoint{8.234908in}{0.694857in}}%
\pgfpathlineto{\pgfqpoint{8.234908in}{0.550000in}}%
\pgfpathclose%
\pgfusepath{fill}%
\end{pgfscope}%
\begin{pgfscope}%
\pgfpathrectangle{\pgfqpoint{1.250000in}{0.550000in}}{\pgfqpoint{7.750000in}{3.850000in}}%
\pgfusepath{clip}%
\pgfsetbuttcap%
\pgfsetmiterjoin%
\definecolor{currentfill}{rgb}{0.501961,0.501961,0.501961}%
\pgfsetfillcolor{currentfill}%
\pgfsetfillopacity{0.700000}%
\pgfsetlinewidth{0.000000pt}%
\definecolor{currentstroke}{rgb}{0.000000,0.000000,0.000000}%
\pgfsetstrokecolor{currentstroke}%
\pgfsetstrokeopacity{0.700000}%
\pgfsetdash{}{0pt}%
\pgfpathmoveto{\pgfqpoint{8.262429in}{0.550000in}}%
\pgfpathlineto{\pgfqpoint{8.289950in}{0.550000in}}%
\pgfpathlineto{\pgfqpoint{8.289950in}{0.732268in}}%
\pgfpathlineto{\pgfqpoint{8.262429in}{0.732268in}}%
\pgfpathlineto{\pgfqpoint{8.262429in}{0.550000in}}%
\pgfpathclose%
\pgfusepath{fill}%
\end{pgfscope}%
\begin{pgfscope}%
\pgfpathrectangle{\pgfqpoint{1.250000in}{0.550000in}}{\pgfqpoint{7.750000in}{3.850000in}}%
\pgfusepath{clip}%
\pgfsetbuttcap%
\pgfsetmiterjoin%
\definecolor{currentfill}{rgb}{0.501961,0.501961,0.501961}%
\pgfsetfillcolor{currentfill}%
\pgfsetfillopacity{0.700000}%
\pgfsetlinewidth{0.000000pt}%
\definecolor{currentstroke}{rgb}{0.000000,0.000000,0.000000}%
\pgfsetstrokecolor{currentstroke}%
\pgfsetstrokeopacity{0.700000}%
\pgfsetdash{}{0pt}%
\pgfpathmoveto{\pgfqpoint{8.289950in}{0.550000in}}%
\pgfpathlineto{\pgfqpoint{8.317472in}{0.550000in}}%
\pgfpathlineto{\pgfqpoint{8.317472in}{0.709212in}}%
\pgfpathlineto{\pgfqpoint{8.289950in}{0.709212in}}%
\pgfpathlineto{\pgfqpoint{8.289950in}{0.550000in}}%
\pgfpathclose%
\pgfusepath{fill}%
\end{pgfscope}%
\begin{pgfscope}%
\pgfpathrectangle{\pgfqpoint{1.250000in}{0.550000in}}{\pgfqpoint{7.750000in}{3.850000in}}%
\pgfusepath{clip}%
\pgfsetbuttcap%
\pgfsetmiterjoin%
\definecolor{currentfill}{rgb}{0.501961,0.501961,0.501961}%
\pgfsetfillcolor{currentfill}%
\pgfsetfillopacity{0.700000}%
\pgfsetlinewidth{0.000000pt}%
\definecolor{currentstroke}{rgb}{0.000000,0.000000,0.000000}%
\pgfsetstrokecolor{currentstroke}%
\pgfsetstrokeopacity{0.700000}%
\pgfsetdash{}{0pt}%
\pgfpathmoveto{\pgfqpoint{8.317472in}{0.550000in}}%
\pgfpathlineto{\pgfqpoint{8.344993in}{0.550000in}}%
\pgfpathlineto{\pgfqpoint{8.344993in}{0.709647in}}%
\pgfpathlineto{\pgfqpoint{8.317472in}{0.709647in}}%
\pgfpathlineto{\pgfqpoint{8.317472in}{0.550000in}}%
\pgfpathclose%
\pgfusepath{fill}%
\end{pgfscope}%
\begin{pgfscope}%
\pgfpathrectangle{\pgfqpoint{1.250000in}{0.550000in}}{\pgfqpoint{7.750000in}{3.850000in}}%
\pgfusepath{clip}%
\pgfsetbuttcap%
\pgfsetmiterjoin%
\definecolor{currentfill}{rgb}{0.501961,0.501961,0.501961}%
\pgfsetfillcolor{currentfill}%
\pgfsetfillopacity{0.700000}%
\pgfsetlinewidth{0.000000pt}%
\definecolor{currentstroke}{rgb}{0.000000,0.000000,0.000000}%
\pgfsetstrokecolor{currentstroke}%
\pgfsetstrokeopacity{0.700000}%
\pgfsetdash{}{0pt}%
\pgfpathmoveto{\pgfqpoint{8.344993in}{0.550000in}}%
\pgfpathlineto{\pgfqpoint{8.372514in}{0.550000in}}%
\pgfpathlineto{\pgfqpoint{8.372514in}{0.711822in}}%
\pgfpathlineto{\pgfqpoint{8.344993in}{0.711822in}}%
\pgfpathlineto{\pgfqpoint{8.344993in}{0.550000in}}%
\pgfpathclose%
\pgfusepath{fill}%
\end{pgfscope}%
\begin{pgfscope}%
\pgfpathrectangle{\pgfqpoint{1.250000in}{0.550000in}}{\pgfqpoint{7.750000in}{3.850000in}}%
\pgfusepath{clip}%
\pgfsetbuttcap%
\pgfsetmiterjoin%
\definecolor{currentfill}{rgb}{0.501961,0.501961,0.501961}%
\pgfsetfillcolor{currentfill}%
\pgfsetfillopacity{0.700000}%
\pgfsetlinewidth{0.000000pt}%
\definecolor{currentstroke}{rgb}{0.000000,0.000000,0.000000}%
\pgfsetstrokecolor{currentstroke}%
\pgfsetstrokeopacity{0.700000}%
\pgfsetdash{}{0pt}%
\pgfpathmoveto{\pgfqpoint{8.372514in}{0.550000in}}%
\pgfpathlineto{\pgfqpoint{8.400036in}{0.550000in}}%
\pgfpathlineto{\pgfqpoint{8.400036in}{0.701962in}}%
\pgfpathlineto{\pgfqpoint{8.372514in}{0.701962in}}%
\pgfpathlineto{\pgfqpoint{8.372514in}{0.550000in}}%
\pgfpathclose%
\pgfusepath{fill}%
\end{pgfscope}%
\begin{pgfscope}%
\pgfpathrectangle{\pgfqpoint{1.250000in}{0.550000in}}{\pgfqpoint{7.750000in}{3.850000in}}%
\pgfusepath{clip}%
\pgfsetbuttcap%
\pgfsetmiterjoin%
\definecolor{currentfill}{rgb}{0.501961,0.501961,0.501961}%
\pgfsetfillcolor{currentfill}%
\pgfsetfillopacity{0.700000}%
\pgfsetlinewidth{0.000000pt}%
\definecolor{currentstroke}{rgb}{0.000000,0.000000,0.000000}%
\pgfsetstrokecolor{currentstroke}%
\pgfsetstrokeopacity{0.700000}%
\pgfsetdash{}{0pt}%
\pgfpathmoveto{\pgfqpoint{8.400036in}{0.550000in}}%
\pgfpathlineto{\pgfqpoint{8.427557in}{0.550000in}}%
\pgfpathlineto{\pgfqpoint{8.427557in}{0.706602in}}%
\pgfpathlineto{\pgfqpoint{8.400036in}{0.706602in}}%
\pgfpathlineto{\pgfqpoint{8.400036in}{0.550000in}}%
\pgfpathclose%
\pgfusepath{fill}%
\end{pgfscope}%
\begin{pgfscope}%
\pgfpathrectangle{\pgfqpoint{1.250000in}{0.550000in}}{\pgfqpoint{7.750000in}{3.850000in}}%
\pgfusepath{clip}%
\pgfsetbuttcap%
\pgfsetmiterjoin%
\definecolor{currentfill}{rgb}{0.501961,0.501961,0.501961}%
\pgfsetfillcolor{currentfill}%
\pgfsetfillopacity{0.700000}%
\pgfsetlinewidth{0.000000pt}%
\definecolor{currentstroke}{rgb}{0.000000,0.000000,0.000000}%
\pgfsetstrokecolor{currentstroke}%
\pgfsetstrokeopacity{0.700000}%
\pgfsetdash{}{0pt}%
\pgfpathmoveto{\pgfqpoint{8.427557in}{0.550000in}}%
\pgfpathlineto{\pgfqpoint{8.455078in}{0.550000in}}%
\pgfpathlineto{\pgfqpoint{8.455078in}{0.710082in}}%
\pgfpathlineto{\pgfqpoint{8.427557in}{0.710082in}}%
\pgfpathlineto{\pgfqpoint{8.427557in}{0.550000in}}%
\pgfpathclose%
\pgfusepath{fill}%
\end{pgfscope}%
\begin{pgfscope}%
\pgfpathrectangle{\pgfqpoint{1.250000in}{0.550000in}}{\pgfqpoint{7.750000in}{3.850000in}}%
\pgfusepath{clip}%
\pgfsetbuttcap%
\pgfsetmiterjoin%
\definecolor{currentfill}{rgb}{0.501961,0.501961,0.501961}%
\pgfsetfillcolor{currentfill}%
\pgfsetfillopacity{0.700000}%
\pgfsetlinewidth{0.000000pt}%
\definecolor{currentstroke}{rgb}{0.000000,0.000000,0.000000}%
\pgfsetstrokecolor{currentstroke}%
\pgfsetstrokeopacity{0.700000}%
\pgfsetdash{}{0pt}%
\pgfpathmoveto{\pgfqpoint{8.455078in}{0.550000in}}%
\pgfpathlineto{\pgfqpoint{8.482599in}{0.550000in}}%
\pgfpathlineto{\pgfqpoint{8.482599in}{0.692972in}}%
\pgfpathlineto{\pgfqpoint{8.455078in}{0.692972in}}%
\pgfpathlineto{\pgfqpoint{8.455078in}{0.550000in}}%
\pgfpathclose%
\pgfusepath{fill}%
\end{pgfscope}%
\begin{pgfscope}%
\pgfpathrectangle{\pgfqpoint{1.250000in}{0.550000in}}{\pgfqpoint{7.750000in}{3.850000in}}%
\pgfusepath{clip}%
\pgfsetbuttcap%
\pgfsetmiterjoin%
\definecolor{currentfill}{rgb}{0.501961,0.501961,0.501961}%
\pgfsetfillcolor{currentfill}%
\pgfsetfillopacity{0.700000}%
\pgfsetlinewidth{0.000000pt}%
\definecolor{currentstroke}{rgb}{0.000000,0.000000,0.000000}%
\pgfsetstrokecolor{currentstroke}%
\pgfsetstrokeopacity{0.700000}%
\pgfsetdash{}{0pt}%
\pgfpathmoveto{\pgfqpoint{8.482599in}{0.550000in}}%
\pgfpathlineto{\pgfqpoint{8.510121in}{0.550000in}}%
\pgfpathlineto{\pgfqpoint{8.510121in}{0.693842in}}%
\pgfpathlineto{\pgfqpoint{8.482599in}{0.693842in}}%
\pgfpathlineto{\pgfqpoint{8.482599in}{0.550000in}}%
\pgfpathclose%
\pgfusepath{fill}%
\end{pgfscope}%
\begin{pgfscope}%
\pgfpathrectangle{\pgfqpoint{1.250000in}{0.550000in}}{\pgfqpoint{7.750000in}{3.850000in}}%
\pgfusepath{clip}%
\pgfsetbuttcap%
\pgfsetmiterjoin%
\definecolor{currentfill}{rgb}{0.501961,0.501961,0.501961}%
\pgfsetfillcolor{currentfill}%
\pgfsetfillopacity{0.700000}%
\pgfsetlinewidth{0.000000pt}%
\definecolor{currentstroke}{rgb}{0.000000,0.000000,0.000000}%
\pgfsetstrokecolor{currentstroke}%
\pgfsetstrokeopacity{0.700000}%
\pgfsetdash{}{0pt}%
\pgfpathmoveto{\pgfqpoint{8.510121in}{0.550000in}}%
\pgfpathlineto{\pgfqpoint{8.537642in}{0.550000in}}%
\pgfpathlineto{\pgfqpoint{8.537642in}{0.655561in}}%
\pgfpathlineto{\pgfqpoint{8.510121in}{0.655561in}}%
\pgfpathlineto{\pgfqpoint{8.510121in}{0.550000in}}%
\pgfpathclose%
\pgfusepath{fill}%
\end{pgfscope}%
\begin{pgfscope}%
\pgfpathrectangle{\pgfqpoint{1.250000in}{0.550000in}}{\pgfqpoint{7.750000in}{3.850000in}}%
\pgfusepath{clip}%
\pgfsetbuttcap%
\pgfsetmiterjoin%
\definecolor{currentfill}{rgb}{0.501961,0.501961,0.501961}%
\pgfsetfillcolor{currentfill}%
\pgfsetfillopacity{0.700000}%
\pgfsetlinewidth{0.000000pt}%
\definecolor{currentstroke}{rgb}{0.000000,0.000000,0.000000}%
\pgfsetstrokecolor{currentstroke}%
\pgfsetstrokeopacity{0.700000}%
\pgfsetdash{}{0pt}%
\pgfpathmoveto{\pgfqpoint{8.537642in}{0.550000in}}%
\pgfpathlineto{\pgfqpoint{8.565163in}{0.550000in}}%
\pgfpathlineto{\pgfqpoint{8.565163in}{0.694857in}}%
\pgfpathlineto{\pgfqpoint{8.537642in}{0.694857in}}%
\pgfpathlineto{\pgfqpoint{8.537642in}{0.550000in}}%
\pgfpathclose%
\pgfusepath{fill}%
\end{pgfscope}%
\begin{pgfscope}%
\pgfpathrectangle{\pgfqpoint{1.250000in}{0.550000in}}{\pgfqpoint{7.750000in}{3.850000in}}%
\pgfusepath{clip}%
\pgfsetbuttcap%
\pgfsetmiterjoin%
\definecolor{currentfill}{rgb}{0.501961,0.501961,0.501961}%
\pgfsetfillcolor{currentfill}%
\pgfsetfillopacity{0.700000}%
\pgfsetlinewidth{0.000000pt}%
\definecolor{currentstroke}{rgb}{0.000000,0.000000,0.000000}%
\pgfsetstrokecolor{currentstroke}%
\pgfsetstrokeopacity{0.700000}%
\pgfsetdash{}{0pt}%
\pgfpathmoveto{\pgfqpoint{8.565163in}{0.550000in}}%
\pgfpathlineto{\pgfqpoint{8.592685in}{0.550000in}}%
\pgfpathlineto{\pgfqpoint{8.592685in}{0.741113in}}%
\pgfpathlineto{\pgfqpoint{8.565163in}{0.741113in}}%
\pgfpathlineto{\pgfqpoint{8.565163in}{0.550000in}}%
\pgfpathclose%
\pgfusepath{fill}%
\end{pgfscope}%
\begin{pgfscope}%
\pgfpathrectangle{\pgfqpoint{1.250000in}{0.550000in}}{\pgfqpoint{7.750000in}{3.850000in}}%
\pgfusepath{clip}%
\pgfsetbuttcap%
\pgfsetmiterjoin%
\definecolor{currentfill}{rgb}{0.501961,0.501961,0.501961}%
\pgfsetfillcolor{currentfill}%
\pgfsetfillopacity{0.700000}%
\pgfsetlinewidth{0.000000pt}%
\definecolor{currentstroke}{rgb}{0.000000,0.000000,0.000000}%
\pgfsetstrokecolor{currentstroke}%
\pgfsetstrokeopacity{0.700000}%
\pgfsetdash{}{0pt}%
\pgfpathmoveto{\pgfqpoint{8.592685in}{0.550000in}}%
\pgfpathlineto{\pgfqpoint{8.620206in}{0.550000in}}%
\pgfpathlineto{\pgfqpoint{8.620206in}{0.550000in}}%
\pgfpathlineto{\pgfqpoint{8.592685in}{0.550000in}}%
\pgfpathlineto{\pgfqpoint{8.592685in}{0.550000in}}%
\pgfpathclose%
\pgfusepath{fill}%
\end{pgfscope}%
\begin{pgfscope}%
\pgfpathrectangle{\pgfqpoint{1.250000in}{0.550000in}}{\pgfqpoint{7.750000in}{3.850000in}}%
\pgfusepath{clip}%
\pgfsetbuttcap%
\pgfsetmiterjoin%
\definecolor{currentfill}{rgb}{0.501961,0.501961,0.501961}%
\pgfsetfillcolor{currentfill}%
\pgfsetfillopacity{0.700000}%
\pgfsetlinewidth{0.000000pt}%
\definecolor{currentstroke}{rgb}{0.000000,0.000000,0.000000}%
\pgfsetstrokecolor{currentstroke}%
\pgfsetstrokeopacity{0.700000}%
\pgfsetdash{}{0pt}%
\pgfpathmoveto{\pgfqpoint{8.620206in}{0.550000in}}%
\pgfpathlineto{\pgfqpoint{8.647727in}{0.550000in}}%
\pgfpathlineto{\pgfqpoint{8.647727in}{0.880025in}}%
\pgfpathlineto{\pgfqpoint{8.620206in}{0.880025in}}%
\pgfpathlineto{\pgfqpoint{8.620206in}{0.550000in}}%
\pgfpathclose%
\pgfusepath{fill}%
\end{pgfscope}%
\begin{pgfscope}%
\pgfsetbuttcap%
\pgfsetroundjoin%
\definecolor{currentfill}{rgb}{0.000000,0.000000,0.000000}%
\pgfsetfillcolor{currentfill}%
\pgfsetlinewidth{0.803000pt}%
\definecolor{currentstroke}{rgb}{0.000000,0.000000,0.000000}%
\pgfsetstrokecolor{currentstroke}%
\pgfsetdash{}{0pt}%
\pgfsys@defobject{currentmarker}{\pgfqpoint{0.000000in}{-0.048611in}}{\pgfqpoint{0.000000in}{0.000000in}}{%
\pgfpathmoveto{\pgfqpoint{0.000000in}{0.000000in}}%
\pgfpathlineto{\pgfqpoint{0.000000in}{-0.048611in}}%
\pgfusepath{stroke,fill}%
}%
\begin{pgfscope}%
\pgfsys@transformshift{1.602273in}{0.550000in}%
\pgfsys@useobject{currentmarker}{}%
\end{pgfscope}%
\end{pgfscope}%
\begin{pgfscope}%
\definecolor{textcolor}{rgb}{0.000000,0.000000,0.000000}%
\pgfsetstrokecolor{textcolor}%
\pgfsetfillcolor{textcolor}%
\pgftext[x=1.602273in,y=0.452778in,,top]{\color{textcolor}{\sffamily\fontsize{10.000000}{12.000000}\selectfont\catcode`\^=\active\def^{\ifmmode\sp\else\^{}\fi}\catcode`\%=\active\def%{\%}0}}%
\end{pgfscope}%
\begin{pgfscope}%
\pgfsetbuttcap%
\pgfsetroundjoin%
\definecolor{currentfill}{rgb}{0.000000,0.000000,0.000000}%
\pgfsetfillcolor{currentfill}%
\pgfsetlinewidth{0.803000pt}%
\definecolor{currentstroke}{rgb}{0.000000,0.000000,0.000000}%
\pgfsetstrokecolor{currentstroke}%
\pgfsetdash{}{0pt}%
\pgfsys@defobject{currentmarker}{\pgfqpoint{0.000000in}{-0.048611in}}{\pgfqpoint{0.000000in}{0.000000in}}{%
\pgfpathmoveto{\pgfqpoint{0.000000in}{0.000000in}}%
\pgfpathlineto{\pgfqpoint{0.000000in}{-0.048611in}}%
\pgfusepath{stroke,fill}%
}%
\begin{pgfscope}%
\pgfsys@transformshift{2.978338in}{0.550000in}%
\pgfsys@useobject{currentmarker}{}%
\end{pgfscope}%
\end{pgfscope}%
\begin{pgfscope}%
\definecolor{textcolor}{rgb}{0.000000,0.000000,0.000000}%
\pgfsetstrokecolor{textcolor}%
\pgfsetfillcolor{textcolor}%
\pgftext[x=2.978338in,y=0.452778in,,top]{\color{textcolor}{\sffamily\fontsize{10.000000}{12.000000}\selectfont\catcode`\^=\active\def^{\ifmmode\sp\else\^{}\fi}\catcode`\%=\active\def%{\%}50}}%
\end{pgfscope}%
\begin{pgfscope}%
\pgfsetbuttcap%
\pgfsetroundjoin%
\definecolor{currentfill}{rgb}{0.000000,0.000000,0.000000}%
\pgfsetfillcolor{currentfill}%
\pgfsetlinewidth{0.803000pt}%
\definecolor{currentstroke}{rgb}{0.000000,0.000000,0.000000}%
\pgfsetstrokecolor{currentstroke}%
\pgfsetdash{}{0pt}%
\pgfsys@defobject{currentmarker}{\pgfqpoint{0.000000in}{-0.048611in}}{\pgfqpoint{0.000000in}{0.000000in}}{%
\pgfpathmoveto{\pgfqpoint{0.000000in}{0.000000in}}%
\pgfpathlineto{\pgfqpoint{0.000000in}{-0.048611in}}%
\pgfusepath{stroke,fill}%
}%
\begin{pgfscope}%
\pgfsys@transformshift{4.354403in}{0.550000in}%
\pgfsys@useobject{currentmarker}{}%
\end{pgfscope}%
\end{pgfscope}%
\begin{pgfscope}%
\definecolor{textcolor}{rgb}{0.000000,0.000000,0.000000}%
\pgfsetstrokecolor{textcolor}%
\pgfsetfillcolor{textcolor}%
\pgftext[x=4.354403in,y=0.452778in,,top]{\color{textcolor}{\sffamily\fontsize{10.000000}{12.000000}\selectfont\catcode`\^=\active\def^{\ifmmode\sp\else\^{}\fi}\catcode`\%=\active\def%{\%}100}}%
\end{pgfscope}%
\begin{pgfscope}%
\pgfsetbuttcap%
\pgfsetroundjoin%
\definecolor{currentfill}{rgb}{0.000000,0.000000,0.000000}%
\pgfsetfillcolor{currentfill}%
\pgfsetlinewidth{0.803000pt}%
\definecolor{currentstroke}{rgb}{0.000000,0.000000,0.000000}%
\pgfsetstrokecolor{currentstroke}%
\pgfsetdash{}{0pt}%
\pgfsys@defobject{currentmarker}{\pgfqpoint{0.000000in}{-0.048611in}}{\pgfqpoint{0.000000in}{0.000000in}}{%
\pgfpathmoveto{\pgfqpoint{0.000000in}{0.000000in}}%
\pgfpathlineto{\pgfqpoint{0.000000in}{-0.048611in}}%
\pgfusepath{stroke,fill}%
}%
\begin{pgfscope}%
\pgfsys@transformshift{5.730469in}{0.550000in}%
\pgfsys@useobject{currentmarker}{}%
\end{pgfscope}%
\end{pgfscope}%
\begin{pgfscope}%
\definecolor{textcolor}{rgb}{0.000000,0.000000,0.000000}%
\pgfsetstrokecolor{textcolor}%
\pgfsetfillcolor{textcolor}%
\pgftext[x=5.730469in,y=0.452778in,,top]{\color{textcolor}{\sffamily\fontsize{10.000000}{12.000000}\selectfont\catcode`\^=\active\def^{\ifmmode\sp\else\^{}\fi}\catcode`\%=\active\def%{\%}150}}%
\end{pgfscope}%
\begin{pgfscope}%
\pgfsetbuttcap%
\pgfsetroundjoin%
\definecolor{currentfill}{rgb}{0.000000,0.000000,0.000000}%
\pgfsetfillcolor{currentfill}%
\pgfsetlinewidth{0.803000pt}%
\definecolor{currentstroke}{rgb}{0.000000,0.000000,0.000000}%
\pgfsetstrokecolor{currentstroke}%
\pgfsetdash{}{0pt}%
\pgfsys@defobject{currentmarker}{\pgfqpoint{0.000000in}{-0.048611in}}{\pgfqpoint{0.000000in}{0.000000in}}{%
\pgfpathmoveto{\pgfqpoint{0.000000in}{0.000000in}}%
\pgfpathlineto{\pgfqpoint{0.000000in}{-0.048611in}}%
\pgfusepath{stroke,fill}%
}%
\begin{pgfscope}%
\pgfsys@transformshift{7.106534in}{0.550000in}%
\pgfsys@useobject{currentmarker}{}%
\end{pgfscope}%
\end{pgfscope}%
\begin{pgfscope}%
\definecolor{textcolor}{rgb}{0.000000,0.000000,0.000000}%
\pgfsetstrokecolor{textcolor}%
\pgfsetfillcolor{textcolor}%
\pgftext[x=7.106534in,y=0.452778in,,top]{\color{textcolor}{\sffamily\fontsize{10.000000}{12.000000}\selectfont\catcode`\^=\active\def^{\ifmmode\sp\else\^{}\fi}\catcode`\%=\active\def%{\%}200}}%
\end{pgfscope}%
\begin{pgfscope}%
\pgfsetbuttcap%
\pgfsetroundjoin%
\definecolor{currentfill}{rgb}{0.000000,0.000000,0.000000}%
\pgfsetfillcolor{currentfill}%
\pgfsetlinewidth{0.803000pt}%
\definecolor{currentstroke}{rgb}{0.000000,0.000000,0.000000}%
\pgfsetstrokecolor{currentstroke}%
\pgfsetdash{}{0pt}%
\pgfsys@defobject{currentmarker}{\pgfqpoint{0.000000in}{-0.048611in}}{\pgfqpoint{0.000000in}{0.000000in}}{%
\pgfpathmoveto{\pgfqpoint{0.000000in}{0.000000in}}%
\pgfpathlineto{\pgfqpoint{0.000000in}{-0.048611in}}%
\pgfusepath{stroke,fill}%
}%
\begin{pgfscope}%
\pgfsys@transformshift{8.482599in}{0.550000in}%
\pgfsys@useobject{currentmarker}{}%
\end{pgfscope}%
\end{pgfscope}%
\begin{pgfscope}%
\definecolor{textcolor}{rgb}{0.000000,0.000000,0.000000}%
\pgfsetstrokecolor{textcolor}%
\pgfsetfillcolor{textcolor}%
\pgftext[x=8.482599in,y=0.452778in,,top]{\color{textcolor}{\sffamily\fontsize{10.000000}{12.000000}\selectfont\catcode`\^=\active\def^{\ifmmode\sp\else\^{}\fi}\catcode`\%=\active\def%{\%}250}}%
\end{pgfscope}%
\begin{pgfscope}%
\definecolor{textcolor}{rgb}{0.000000,0.000000,0.000000}%
\pgfsetstrokecolor{textcolor}%
\pgfsetfillcolor{textcolor}%
\pgftext[x=5.125000in,y=0.262809in,,top]{\color{textcolor}{\sffamily\fontsize{10.000000}{12.000000}\selectfont\catcode`\^=\active\def^{\ifmmode\sp\else\^{}\fi}\catcode`\%=\active\def%{\%}Intensidad}}%
\end{pgfscope}%
\begin{pgfscope}%
\pgfsetbuttcap%
\pgfsetroundjoin%
\definecolor{currentfill}{rgb}{0.000000,0.000000,0.000000}%
\pgfsetfillcolor{currentfill}%
\pgfsetlinewidth{0.803000pt}%
\definecolor{currentstroke}{rgb}{0.000000,0.000000,0.000000}%
\pgfsetstrokecolor{currentstroke}%
\pgfsetdash{}{0pt}%
\pgfsys@defobject{currentmarker}{\pgfqpoint{-0.048611in}{0.000000in}}{\pgfqpoint{-0.000000in}{0.000000in}}{%
\pgfpathmoveto{\pgfqpoint{-0.000000in}{0.000000in}}%
\pgfpathlineto{\pgfqpoint{-0.048611in}{0.000000in}}%
\pgfusepath{stroke,fill}%
}%
\begin{pgfscope}%
\pgfsys@transformshift{1.250000in}{0.550000in}%
\pgfsys@useobject{currentmarker}{}%
\end{pgfscope}%
\end{pgfscope}%
\begin{pgfscope}%
\definecolor{textcolor}{rgb}{0.000000,0.000000,0.000000}%
\pgfsetstrokecolor{textcolor}%
\pgfsetfillcolor{textcolor}%
\pgftext[x=1.064412in, y=0.497238in, left, base]{\color{textcolor}{\sffamily\fontsize{10.000000}{12.000000}\selectfont\catcode`\^=\active\def^{\ifmmode\sp\else\^{}\fi}\catcode`\%=\active\def%{\%}0}}%
\end{pgfscope}%
\begin{pgfscope}%
\pgfsetbuttcap%
\pgfsetroundjoin%
\definecolor{currentfill}{rgb}{0.000000,0.000000,0.000000}%
\pgfsetfillcolor{currentfill}%
\pgfsetlinewidth{0.803000pt}%
\definecolor{currentstroke}{rgb}{0.000000,0.000000,0.000000}%
\pgfsetstrokecolor{currentstroke}%
\pgfsetdash{}{0pt}%
\pgfsys@defobject{currentmarker}{\pgfqpoint{-0.048611in}{0.000000in}}{\pgfqpoint{-0.000000in}{0.000000in}}{%
\pgfpathmoveto{\pgfqpoint{-0.000000in}{0.000000in}}%
\pgfpathlineto{\pgfqpoint{-0.048611in}{0.000000in}}%
\pgfusepath{stroke,fill}%
}%
\begin{pgfscope}%
\pgfsys@transformshift{1.250000in}{1.275010in}%
\pgfsys@useobject{currentmarker}{}%
\end{pgfscope}%
\end{pgfscope}%
\begin{pgfscope}%
\definecolor{textcolor}{rgb}{0.000000,0.000000,0.000000}%
\pgfsetstrokecolor{textcolor}%
\pgfsetfillcolor{textcolor}%
\pgftext[x=0.799316in, y=1.222249in, left, base]{\color{textcolor}{\sffamily\fontsize{10.000000}{12.000000}\selectfont\catcode`\^=\active\def^{\ifmmode\sp\else\^{}\fi}\catcode`\%=\active\def%{\%}5000}}%
\end{pgfscope}%
\begin{pgfscope}%
\pgfsetbuttcap%
\pgfsetroundjoin%
\definecolor{currentfill}{rgb}{0.000000,0.000000,0.000000}%
\pgfsetfillcolor{currentfill}%
\pgfsetlinewidth{0.803000pt}%
\definecolor{currentstroke}{rgb}{0.000000,0.000000,0.000000}%
\pgfsetstrokecolor{currentstroke}%
\pgfsetdash{}{0pt}%
\pgfsys@defobject{currentmarker}{\pgfqpoint{-0.048611in}{0.000000in}}{\pgfqpoint{-0.000000in}{0.000000in}}{%
\pgfpathmoveto{\pgfqpoint{-0.000000in}{0.000000in}}%
\pgfpathlineto{\pgfqpoint{-0.048611in}{0.000000in}}%
\pgfusepath{stroke,fill}%
}%
\begin{pgfscope}%
\pgfsys@transformshift{1.250000in}{2.000020in}%
\pgfsys@useobject{currentmarker}{}%
\end{pgfscope}%
\end{pgfscope}%
\begin{pgfscope}%
\definecolor{textcolor}{rgb}{0.000000,0.000000,0.000000}%
\pgfsetstrokecolor{textcolor}%
\pgfsetfillcolor{textcolor}%
\pgftext[x=0.710951in, y=1.947259in, left, base]{\color{textcolor}{\sffamily\fontsize{10.000000}{12.000000}\selectfont\catcode`\^=\active\def^{\ifmmode\sp\else\^{}\fi}\catcode`\%=\active\def%{\%}10000}}%
\end{pgfscope}%
\begin{pgfscope}%
\pgfsetbuttcap%
\pgfsetroundjoin%
\definecolor{currentfill}{rgb}{0.000000,0.000000,0.000000}%
\pgfsetfillcolor{currentfill}%
\pgfsetlinewidth{0.803000pt}%
\definecolor{currentstroke}{rgb}{0.000000,0.000000,0.000000}%
\pgfsetstrokecolor{currentstroke}%
\pgfsetdash{}{0pt}%
\pgfsys@defobject{currentmarker}{\pgfqpoint{-0.048611in}{0.000000in}}{\pgfqpoint{-0.000000in}{0.000000in}}{%
\pgfpathmoveto{\pgfqpoint{-0.000000in}{0.000000in}}%
\pgfpathlineto{\pgfqpoint{-0.048611in}{0.000000in}}%
\pgfusepath{stroke,fill}%
}%
\begin{pgfscope}%
\pgfsys@transformshift{1.250000in}{2.725031in}%
\pgfsys@useobject{currentmarker}{}%
\end{pgfscope}%
\end{pgfscope}%
\begin{pgfscope}%
\definecolor{textcolor}{rgb}{0.000000,0.000000,0.000000}%
\pgfsetstrokecolor{textcolor}%
\pgfsetfillcolor{textcolor}%
\pgftext[x=0.710951in, y=2.672269in, left, base]{\color{textcolor}{\sffamily\fontsize{10.000000}{12.000000}\selectfont\catcode`\^=\active\def^{\ifmmode\sp\else\^{}\fi}\catcode`\%=\active\def%{\%}15000}}%
\end{pgfscope}%
\begin{pgfscope}%
\pgfsetbuttcap%
\pgfsetroundjoin%
\definecolor{currentfill}{rgb}{0.000000,0.000000,0.000000}%
\pgfsetfillcolor{currentfill}%
\pgfsetlinewidth{0.803000pt}%
\definecolor{currentstroke}{rgb}{0.000000,0.000000,0.000000}%
\pgfsetstrokecolor{currentstroke}%
\pgfsetdash{}{0pt}%
\pgfsys@defobject{currentmarker}{\pgfqpoint{-0.048611in}{0.000000in}}{\pgfqpoint{-0.000000in}{0.000000in}}{%
\pgfpathmoveto{\pgfqpoint{-0.000000in}{0.000000in}}%
\pgfpathlineto{\pgfqpoint{-0.048611in}{0.000000in}}%
\pgfusepath{stroke,fill}%
}%
\begin{pgfscope}%
\pgfsys@transformshift{1.250000in}{3.450041in}%
\pgfsys@useobject{currentmarker}{}%
\end{pgfscope}%
\end{pgfscope}%
\begin{pgfscope}%
\definecolor{textcolor}{rgb}{0.000000,0.000000,0.000000}%
\pgfsetstrokecolor{textcolor}%
\pgfsetfillcolor{textcolor}%
\pgftext[x=0.710951in, y=3.397279in, left, base]{\color{textcolor}{\sffamily\fontsize{10.000000}{12.000000}\selectfont\catcode`\^=\active\def^{\ifmmode\sp\else\^{}\fi}\catcode`\%=\active\def%{\%}20000}}%
\end{pgfscope}%
\begin{pgfscope}%
\pgfsetbuttcap%
\pgfsetroundjoin%
\definecolor{currentfill}{rgb}{0.000000,0.000000,0.000000}%
\pgfsetfillcolor{currentfill}%
\pgfsetlinewidth{0.803000pt}%
\definecolor{currentstroke}{rgb}{0.000000,0.000000,0.000000}%
\pgfsetstrokecolor{currentstroke}%
\pgfsetdash{}{0pt}%
\pgfsys@defobject{currentmarker}{\pgfqpoint{-0.048611in}{0.000000in}}{\pgfqpoint{-0.000000in}{0.000000in}}{%
\pgfpathmoveto{\pgfqpoint{-0.000000in}{0.000000in}}%
\pgfpathlineto{\pgfqpoint{-0.048611in}{0.000000in}}%
\pgfusepath{stroke,fill}%
}%
\begin{pgfscope}%
\pgfsys@transformshift{1.250000in}{4.175051in}%
\pgfsys@useobject{currentmarker}{}%
\end{pgfscope}%
\end{pgfscope}%
\begin{pgfscope}%
\definecolor{textcolor}{rgb}{0.000000,0.000000,0.000000}%
\pgfsetstrokecolor{textcolor}%
\pgfsetfillcolor{textcolor}%
\pgftext[x=0.710951in, y=4.122290in, left, base]{\color{textcolor}{\sffamily\fontsize{10.000000}{12.000000}\selectfont\catcode`\^=\active\def^{\ifmmode\sp\else\^{}\fi}\catcode`\%=\active\def%{\%}25000}}%
\end{pgfscope}%
\begin{pgfscope}%
\definecolor{textcolor}{rgb}{0.000000,0.000000,0.000000}%
\pgfsetstrokecolor{textcolor}%
\pgfsetfillcolor{textcolor}%
\pgftext[x=0.655396in,y=2.475000in,,bottom,rotate=90.000000]{\color{textcolor}{\sffamily\fontsize{10.000000}{12.000000}\selectfont\catcode`\^=\active\def^{\ifmmode\sp\else\^{}\fi}\catcode`\%=\active\def%{\%}Frecuencia}}%
\end{pgfscope}%
\begin{pgfscope}%
\pgfsetrectcap%
\pgfsetmiterjoin%
\pgfsetlinewidth{0.803000pt}%
\definecolor{currentstroke}{rgb}{0.000000,0.000000,0.000000}%
\pgfsetstrokecolor{currentstroke}%
\pgfsetdash{}{0pt}%
\pgfpathmoveto{\pgfqpoint{1.250000in}{0.550000in}}%
\pgfpathlineto{\pgfqpoint{1.250000in}{4.400000in}}%
\pgfusepath{stroke}%
\end{pgfscope}%
\begin{pgfscope}%
\pgfsetrectcap%
\pgfsetmiterjoin%
\pgfsetlinewidth{0.803000pt}%
\definecolor{currentstroke}{rgb}{0.000000,0.000000,0.000000}%
\pgfsetstrokecolor{currentstroke}%
\pgfsetdash{}{0pt}%
\pgfpathmoveto{\pgfqpoint{9.000000in}{0.550000in}}%
\pgfpathlineto{\pgfqpoint{9.000000in}{4.400000in}}%
\pgfusepath{stroke}%
\end{pgfscope}%
\begin{pgfscope}%
\pgfsetrectcap%
\pgfsetmiterjoin%
\pgfsetlinewidth{0.803000pt}%
\definecolor{currentstroke}{rgb}{0.000000,0.000000,0.000000}%
\pgfsetstrokecolor{currentstroke}%
\pgfsetdash{}{0pt}%
\pgfpathmoveto{\pgfqpoint{1.250000in}{0.550000in}}%
\pgfpathlineto{\pgfqpoint{9.000000in}{0.550000in}}%
\pgfusepath{stroke}%
\end{pgfscope}%
\begin{pgfscope}%
\pgfsetrectcap%
\pgfsetmiterjoin%
\pgfsetlinewidth{0.803000pt}%
\definecolor{currentstroke}{rgb}{0.000000,0.000000,0.000000}%
\pgfsetstrokecolor{currentstroke}%
\pgfsetdash{}{0pt}%
\pgfpathmoveto{\pgfqpoint{1.250000in}{4.400000in}}%
\pgfpathlineto{\pgfqpoint{9.000000in}{4.400000in}}%
\pgfusepath{stroke}%
\end{pgfscope}%
\begin{pgfscope}%
\definecolor{textcolor}{rgb}{0.000000,0.000000,0.000000}%
\pgfsetstrokecolor{textcolor}%
\pgfsetfillcolor{textcolor}%
\pgftext[x=5.125000in,y=4.483333in,,base]{\color{textcolor}{\sffamily\fontsize{12.000000}{14.400000}\selectfont\catcode`\^=\active\def^{\ifmmode\sp\else\^{}\fi}\catcode`\%=\active\def%{\%}Histograma de HE}}%
\end{pgfscope}%
\end{pgfpicture}%
\makeatother%
\endgroup%
}
		\caption{HE}
		\label{fig:he_2}
	\end{subfigure}
	\hfill
	\begin{subfigure}[b]{0.45\textwidth}
		\resizebox{\linewidth}{!}{%% Creator: Matplotlib, PGF backend
%%
%% To include the figure in your LaTeX document, write
%%   \input{<filename>.pgf}
%%
%% Make sure the required packages are loaded in your preamble
%%   \usepackage{pgf}
%%
%% Also ensure that all the required font packages are loaded; for instance,
%% the lmodern package is sometimes necessary when using math font.
%%   \usepackage{lmodern}
%%
%% Figures using additional raster images can only be included by \input if
%% they are in the same directory as the main LaTeX file. For loading figures
%% from other directories you can use the `import` package
%%   \usepackage{import}
%%
%% and then include the figures with
%%   \import{<path to file>}{<filename>.pgf}
%%
%% Matplotlib used the following preamble
%%   \def\mathdefault#1{#1}
%%   \everymath=\expandafter{\the\everymath\displaystyle}
%%   \IfFileExists{scrextend.sty}{
%%     \usepackage[fontsize=10.000000pt]{scrextend}
%%   }{
%%     \renewcommand{\normalsize}{\fontsize{10.000000}{12.000000}\selectfont}
%%     \normalsize
%%   }
%%   
%%   \ifdefined\pdftexversion\else  % non-pdftex case.
%%     \usepackage{fontspec}
%%     \setmainfont{DejaVuSerif.ttf}[Path=\detokenize{/home/elias/.virtualenvs/src-pway/lib/python3.13/site-packages/matplotlib/mpl-data/fonts/ttf/}]
%%     \setsansfont{DejaVuSans.ttf}[Path=\detokenize{/home/elias/.virtualenvs/src-pway/lib/python3.13/site-packages/matplotlib/mpl-data/fonts/ttf/}]
%%     \setmonofont{DejaVuSansMono.ttf}[Path=\detokenize{/home/elias/.virtualenvs/src-pway/lib/python3.13/site-packages/matplotlib/mpl-data/fonts/ttf/}]
%%   \fi
%%   \makeatletter\@ifpackageloaded{underscore}{}{\usepackage[strings]{underscore}}\makeatother
%%
\begingroup%
\makeatletter%
\begin{pgfpicture}%
\pgfpathrectangle{\pgfpointorigin}{\pgfqpoint{10.000000in}{5.000000in}}%
\pgfusepath{use as bounding box, clip}%
\begin{pgfscope}%
\pgfsetbuttcap%
\pgfsetmiterjoin%
\definecolor{currentfill}{rgb}{1.000000,1.000000,1.000000}%
\pgfsetfillcolor{currentfill}%
\pgfsetlinewidth{0.000000pt}%
\definecolor{currentstroke}{rgb}{1.000000,1.000000,1.000000}%
\pgfsetstrokecolor{currentstroke}%
\pgfsetdash{}{0pt}%
\pgfpathmoveto{\pgfqpoint{0.000000in}{0.000000in}}%
\pgfpathlineto{\pgfqpoint{10.000000in}{0.000000in}}%
\pgfpathlineto{\pgfqpoint{10.000000in}{5.000000in}}%
\pgfpathlineto{\pgfqpoint{0.000000in}{5.000000in}}%
\pgfpathlineto{\pgfqpoint{0.000000in}{0.000000in}}%
\pgfpathclose%
\pgfusepath{fill}%
\end{pgfscope}%
\begin{pgfscope}%
\pgfsetbuttcap%
\pgfsetmiterjoin%
\definecolor{currentfill}{rgb}{1.000000,1.000000,1.000000}%
\pgfsetfillcolor{currentfill}%
\pgfsetlinewidth{0.000000pt}%
\definecolor{currentstroke}{rgb}{0.000000,0.000000,0.000000}%
\pgfsetstrokecolor{currentstroke}%
\pgfsetstrokeopacity{0.000000}%
\pgfsetdash{}{0pt}%
\pgfpathmoveto{\pgfqpoint{1.250000in}{0.550000in}}%
\pgfpathlineto{\pgfqpoint{9.000000in}{0.550000in}}%
\pgfpathlineto{\pgfqpoint{9.000000in}{4.400000in}}%
\pgfpathlineto{\pgfqpoint{1.250000in}{4.400000in}}%
\pgfpathlineto{\pgfqpoint{1.250000in}{0.550000in}}%
\pgfpathclose%
\pgfusepath{fill}%
\end{pgfscope}%
\begin{pgfscope}%
\pgfpathrectangle{\pgfqpoint{1.250000in}{0.550000in}}{\pgfqpoint{7.750000in}{3.850000in}}%
\pgfusepath{clip}%
\pgfsetbuttcap%
\pgfsetmiterjoin%
\definecolor{currentfill}{rgb}{0.501961,0.501961,0.501961}%
\pgfsetfillcolor{currentfill}%
\pgfsetfillopacity{0.700000}%
\pgfsetlinewidth{0.000000pt}%
\definecolor{currentstroke}{rgb}{0.000000,0.000000,0.000000}%
\pgfsetstrokecolor{currentstroke}%
\pgfsetstrokeopacity{0.700000}%
\pgfsetdash{}{0pt}%
\pgfpathmoveto{\pgfqpoint{1.602273in}{0.550000in}}%
\pgfpathlineto{\pgfqpoint{1.629794in}{0.550000in}}%
\pgfpathlineto{\pgfqpoint{1.629794in}{0.550000in}}%
\pgfpathlineto{\pgfqpoint{1.602273in}{0.550000in}}%
\pgfpathlineto{\pgfqpoint{1.602273in}{0.550000in}}%
\pgfpathclose%
\pgfusepath{fill}%
\end{pgfscope}%
\begin{pgfscope}%
\pgfpathrectangle{\pgfqpoint{1.250000in}{0.550000in}}{\pgfqpoint{7.750000in}{3.850000in}}%
\pgfusepath{clip}%
\pgfsetbuttcap%
\pgfsetmiterjoin%
\definecolor{currentfill}{rgb}{0.501961,0.501961,0.501961}%
\pgfsetfillcolor{currentfill}%
\pgfsetfillopacity{0.700000}%
\pgfsetlinewidth{0.000000pt}%
\definecolor{currentstroke}{rgb}{0.000000,0.000000,0.000000}%
\pgfsetstrokecolor{currentstroke}%
\pgfsetstrokeopacity{0.700000}%
\pgfsetdash{}{0pt}%
\pgfpathmoveto{\pgfqpoint{1.629794in}{0.550000in}}%
\pgfpathlineto{\pgfqpoint{1.657315in}{0.550000in}}%
\pgfpathlineto{\pgfqpoint{1.657315in}{0.555206in}}%
\pgfpathlineto{\pgfqpoint{1.629794in}{0.555206in}}%
\pgfpathlineto{\pgfqpoint{1.629794in}{0.550000in}}%
\pgfpathclose%
\pgfusepath{fill}%
\end{pgfscope}%
\begin{pgfscope}%
\pgfpathrectangle{\pgfqpoint{1.250000in}{0.550000in}}{\pgfqpoint{7.750000in}{3.850000in}}%
\pgfusepath{clip}%
\pgfsetbuttcap%
\pgfsetmiterjoin%
\definecolor{currentfill}{rgb}{0.501961,0.501961,0.501961}%
\pgfsetfillcolor{currentfill}%
\pgfsetfillopacity{0.700000}%
\pgfsetlinewidth{0.000000pt}%
\definecolor{currentstroke}{rgb}{0.000000,0.000000,0.000000}%
\pgfsetstrokecolor{currentstroke}%
\pgfsetstrokeopacity{0.700000}%
\pgfsetdash{}{0pt}%
\pgfpathmoveto{\pgfqpoint{1.657315in}{0.550000in}}%
\pgfpathlineto{\pgfqpoint{1.684837in}{0.550000in}}%
\pgfpathlineto{\pgfqpoint{1.684837in}{0.634516in}}%
\pgfpathlineto{\pgfqpoint{1.657315in}{0.634516in}}%
\pgfpathlineto{\pgfqpoint{1.657315in}{0.550000in}}%
\pgfpathclose%
\pgfusepath{fill}%
\end{pgfscope}%
\begin{pgfscope}%
\pgfpathrectangle{\pgfqpoint{1.250000in}{0.550000in}}{\pgfqpoint{7.750000in}{3.850000in}}%
\pgfusepath{clip}%
\pgfsetbuttcap%
\pgfsetmiterjoin%
\definecolor{currentfill}{rgb}{0.501961,0.501961,0.501961}%
\pgfsetfillcolor{currentfill}%
\pgfsetfillopacity{0.700000}%
\pgfsetlinewidth{0.000000pt}%
\definecolor{currentstroke}{rgb}{0.000000,0.000000,0.000000}%
\pgfsetstrokecolor{currentstroke}%
\pgfsetstrokeopacity{0.700000}%
\pgfsetdash{}{0pt}%
\pgfpathmoveto{\pgfqpoint{1.684837in}{0.550000in}}%
\pgfpathlineto{\pgfqpoint{1.712358in}{0.550000in}}%
\pgfpathlineto{\pgfqpoint{1.712358in}{4.216667in}}%
\pgfpathlineto{\pgfqpoint{1.684837in}{4.216667in}}%
\pgfpathlineto{\pgfqpoint{1.684837in}{0.550000in}}%
\pgfpathclose%
\pgfusepath{fill}%
\end{pgfscope}%
\begin{pgfscope}%
\pgfpathrectangle{\pgfqpoint{1.250000in}{0.550000in}}{\pgfqpoint{7.750000in}{3.850000in}}%
\pgfusepath{clip}%
\pgfsetbuttcap%
\pgfsetmiterjoin%
\definecolor{currentfill}{rgb}{0.501961,0.501961,0.501961}%
\pgfsetfillcolor{currentfill}%
\pgfsetfillopacity{0.700000}%
\pgfsetlinewidth{0.000000pt}%
\definecolor{currentstroke}{rgb}{0.000000,0.000000,0.000000}%
\pgfsetstrokecolor{currentstroke}%
\pgfsetstrokeopacity{0.700000}%
\pgfsetdash{}{0pt}%
\pgfpathmoveto{\pgfqpoint{1.712358in}{0.550000in}}%
\pgfpathlineto{\pgfqpoint{1.739879in}{0.550000in}}%
\pgfpathlineto{\pgfqpoint{1.739879in}{0.597158in}}%
\pgfpathlineto{\pgfqpoint{1.712358in}{0.597158in}}%
\pgfpathlineto{\pgfqpoint{1.712358in}{0.550000in}}%
\pgfpathclose%
\pgfusepath{fill}%
\end{pgfscope}%
\begin{pgfscope}%
\pgfpathrectangle{\pgfqpoint{1.250000in}{0.550000in}}{\pgfqpoint{7.750000in}{3.850000in}}%
\pgfusepath{clip}%
\pgfsetbuttcap%
\pgfsetmiterjoin%
\definecolor{currentfill}{rgb}{0.501961,0.501961,0.501961}%
\pgfsetfillcolor{currentfill}%
\pgfsetfillopacity{0.700000}%
\pgfsetlinewidth{0.000000pt}%
\definecolor{currentstroke}{rgb}{0.000000,0.000000,0.000000}%
\pgfsetstrokecolor{currentstroke}%
\pgfsetstrokeopacity{0.700000}%
\pgfsetdash{}{0pt}%
\pgfpathmoveto{\pgfqpoint{1.739879in}{0.550000in}}%
\pgfpathlineto{\pgfqpoint{1.767401in}{0.550000in}}%
\pgfpathlineto{\pgfqpoint{1.767401in}{1.176524in}}%
\pgfpathlineto{\pgfqpoint{1.739879in}{1.176524in}}%
\pgfpathlineto{\pgfqpoint{1.739879in}{0.550000in}}%
\pgfpathclose%
\pgfusepath{fill}%
\end{pgfscope}%
\begin{pgfscope}%
\pgfpathrectangle{\pgfqpoint{1.250000in}{0.550000in}}{\pgfqpoint{7.750000in}{3.850000in}}%
\pgfusepath{clip}%
\pgfsetbuttcap%
\pgfsetmiterjoin%
\definecolor{currentfill}{rgb}{0.501961,0.501961,0.501961}%
\pgfsetfillcolor{currentfill}%
\pgfsetfillopacity{0.700000}%
\pgfsetlinewidth{0.000000pt}%
\definecolor{currentstroke}{rgb}{0.000000,0.000000,0.000000}%
\pgfsetstrokecolor{currentstroke}%
\pgfsetstrokeopacity{0.700000}%
\pgfsetdash{}{0pt}%
\pgfpathmoveto{\pgfqpoint{1.767401in}{0.550000in}}%
\pgfpathlineto{\pgfqpoint{1.794922in}{0.550000in}}%
\pgfpathlineto{\pgfqpoint{1.794922in}{2.075430in}}%
\pgfpathlineto{\pgfqpoint{1.767401in}{2.075430in}}%
\pgfpathlineto{\pgfqpoint{1.767401in}{0.550000in}}%
\pgfpathclose%
\pgfusepath{fill}%
\end{pgfscope}%
\begin{pgfscope}%
\pgfpathrectangle{\pgfqpoint{1.250000in}{0.550000in}}{\pgfqpoint{7.750000in}{3.850000in}}%
\pgfusepath{clip}%
\pgfsetbuttcap%
\pgfsetmiterjoin%
\definecolor{currentfill}{rgb}{0.501961,0.501961,0.501961}%
\pgfsetfillcolor{currentfill}%
\pgfsetfillopacity{0.700000}%
\pgfsetlinewidth{0.000000pt}%
\definecolor{currentstroke}{rgb}{0.000000,0.000000,0.000000}%
\pgfsetstrokecolor{currentstroke}%
\pgfsetstrokeopacity{0.700000}%
\pgfsetdash{}{0pt}%
\pgfpathmoveto{\pgfqpoint{1.794922in}{0.550000in}}%
\pgfpathlineto{\pgfqpoint{1.822443in}{0.550000in}}%
\pgfpathlineto{\pgfqpoint{1.822443in}{0.638957in}}%
\pgfpathlineto{\pgfqpoint{1.794922in}{0.638957in}}%
\pgfpathlineto{\pgfqpoint{1.794922in}{0.550000in}}%
\pgfpathclose%
\pgfusepath{fill}%
\end{pgfscope}%
\begin{pgfscope}%
\pgfpathrectangle{\pgfqpoint{1.250000in}{0.550000in}}{\pgfqpoint{7.750000in}{3.850000in}}%
\pgfusepath{clip}%
\pgfsetbuttcap%
\pgfsetmiterjoin%
\definecolor{currentfill}{rgb}{0.501961,0.501961,0.501961}%
\pgfsetfillcolor{currentfill}%
\pgfsetfillopacity{0.700000}%
\pgfsetlinewidth{0.000000pt}%
\definecolor{currentstroke}{rgb}{0.000000,0.000000,0.000000}%
\pgfsetstrokecolor{currentstroke}%
\pgfsetstrokeopacity{0.700000}%
\pgfsetdash{}{0pt}%
\pgfpathmoveto{\pgfqpoint{1.822443in}{0.550000in}}%
\pgfpathlineto{\pgfqpoint{1.849964in}{0.550000in}}%
\pgfpathlineto{\pgfqpoint{1.849964in}{1.248026in}}%
\pgfpathlineto{\pgfqpoint{1.822443in}{1.248026in}}%
\pgfpathlineto{\pgfqpoint{1.822443in}{0.550000in}}%
\pgfpathclose%
\pgfusepath{fill}%
\end{pgfscope}%
\begin{pgfscope}%
\pgfpathrectangle{\pgfqpoint{1.250000in}{0.550000in}}{\pgfqpoint{7.750000in}{3.850000in}}%
\pgfusepath{clip}%
\pgfsetbuttcap%
\pgfsetmiterjoin%
\definecolor{currentfill}{rgb}{0.501961,0.501961,0.501961}%
\pgfsetfillcolor{currentfill}%
\pgfsetfillopacity{0.700000}%
\pgfsetlinewidth{0.000000pt}%
\definecolor{currentstroke}{rgb}{0.000000,0.000000,0.000000}%
\pgfsetstrokecolor{currentstroke}%
\pgfsetstrokeopacity{0.700000}%
\pgfsetdash{}{0pt}%
\pgfpathmoveto{\pgfqpoint{1.849964in}{0.550000in}}%
\pgfpathlineto{\pgfqpoint{1.877486in}{0.550000in}}%
\pgfpathlineto{\pgfqpoint{1.877486in}{1.045462in}}%
\pgfpathlineto{\pgfqpoint{1.849964in}{1.045462in}}%
\pgfpathlineto{\pgfqpoint{1.849964in}{0.550000in}}%
\pgfpathclose%
\pgfusepath{fill}%
\end{pgfscope}%
\begin{pgfscope}%
\pgfpathrectangle{\pgfqpoint{1.250000in}{0.550000in}}{\pgfqpoint{7.750000in}{3.850000in}}%
\pgfusepath{clip}%
\pgfsetbuttcap%
\pgfsetmiterjoin%
\definecolor{currentfill}{rgb}{0.501961,0.501961,0.501961}%
\pgfsetfillcolor{currentfill}%
\pgfsetfillopacity{0.700000}%
\pgfsetlinewidth{0.000000pt}%
\definecolor{currentstroke}{rgb}{0.000000,0.000000,0.000000}%
\pgfsetstrokecolor{currentstroke}%
\pgfsetstrokeopacity{0.700000}%
\pgfsetdash{}{0pt}%
\pgfpathmoveto{\pgfqpoint{1.877486in}{0.550000in}}%
\pgfpathlineto{\pgfqpoint{1.905007in}{0.550000in}}%
\pgfpathlineto{\pgfqpoint{1.905007in}{0.679684in}}%
\pgfpathlineto{\pgfqpoint{1.877486in}{0.679684in}}%
\pgfpathlineto{\pgfqpoint{1.877486in}{0.550000in}}%
\pgfpathclose%
\pgfusepath{fill}%
\end{pgfscope}%
\begin{pgfscope}%
\pgfpathrectangle{\pgfqpoint{1.250000in}{0.550000in}}{\pgfqpoint{7.750000in}{3.850000in}}%
\pgfusepath{clip}%
\pgfsetbuttcap%
\pgfsetmiterjoin%
\definecolor{currentfill}{rgb}{0.501961,0.501961,0.501961}%
\pgfsetfillcolor{currentfill}%
\pgfsetfillopacity{0.700000}%
\pgfsetlinewidth{0.000000pt}%
\definecolor{currentstroke}{rgb}{0.000000,0.000000,0.000000}%
\pgfsetstrokecolor{currentstroke}%
\pgfsetstrokeopacity{0.700000}%
\pgfsetdash{}{0pt}%
\pgfpathmoveto{\pgfqpoint{1.905007in}{0.550000in}}%
\pgfpathlineto{\pgfqpoint{1.932528in}{0.550000in}}%
\pgfpathlineto{\pgfqpoint{1.932528in}{1.116046in}}%
\pgfpathlineto{\pgfqpoint{1.905007in}{1.116046in}}%
\pgfpathlineto{\pgfqpoint{1.905007in}{0.550000in}}%
\pgfpathclose%
\pgfusepath{fill}%
\end{pgfscope}%
\begin{pgfscope}%
\pgfpathrectangle{\pgfqpoint{1.250000in}{0.550000in}}{\pgfqpoint{7.750000in}{3.850000in}}%
\pgfusepath{clip}%
\pgfsetbuttcap%
\pgfsetmiterjoin%
\definecolor{currentfill}{rgb}{0.501961,0.501961,0.501961}%
\pgfsetfillcolor{currentfill}%
\pgfsetfillopacity{0.700000}%
\pgfsetlinewidth{0.000000pt}%
\definecolor{currentstroke}{rgb}{0.000000,0.000000,0.000000}%
\pgfsetstrokecolor{currentstroke}%
\pgfsetstrokeopacity{0.700000}%
\pgfsetdash{}{0pt}%
\pgfpathmoveto{\pgfqpoint{1.932528in}{0.550000in}}%
\pgfpathlineto{\pgfqpoint{1.960050in}{0.550000in}}%
\pgfpathlineto{\pgfqpoint{1.960050in}{0.752870in}}%
\pgfpathlineto{\pgfqpoint{1.932528in}{0.752870in}}%
\pgfpathlineto{\pgfqpoint{1.932528in}{0.550000in}}%
\pgfpathclose%
\pgfusepath{fill}%
\end{pgfscope}%
\begin{pgfscope}%
\pgfpathrectangle{\pgfqpoint{1.250000in}{0.550000in}}{\pgfqpoint{7.750000in}{3.850000in}}%
\pgfusepath{clip}%
\pgfsetbuttcap%
\pgfsetmiterjoin%
\definecolor{currentfill}{rgb}{0.501961,0.501961,0.501961}%
\pgfsetfillcolor{currentfill}%
\pgfsetfillopacity{0.700000}%
\pgfsetlinewidth{0.000000pt}%
\definecolor{currentstroke}{rgb}{0.000000,0.000000,0.000000}%
\pgfsetstrokecolor{currentstroke}%
\pgfsetstrokeopacity{0.700000}%
\pgfsetdash{}{0pt}%
\pgfpathmoveto{\pgfqpoint{1.960050in}{0.550000in}}%
\pgfpathlineto{\pgfqpoint{1.987571in}{0.550000in}}%
\pgfpathlineto{\pgfqpoint{1.987571in}{0.732353in}}%
\pgfpathlineto{\pgfqpoint{1.960050in}{0.732353in}}%
\pgfpathlineto{\pgfqpoint{1.960050in}{0.550000in}}%
\pgfpathclose%
\pgfusepath{fill}%
\end{pgfscope}%
\begin{pgfscope}%
\pgfpathrectangle{\pgfqpoint{1.250000in}{0.550000in}}{\pgfqpoint{7.750000in}{3.850000in}}%
\pgfusepath{clip}%
\pgfsetbuttcap%
\pgfsetmiterjoin%
\definecolor{currentfill}{rgb}{0.501961,0.501961,0.501961}%
\pgfsetfillcolor{currentfill}%
\pgfsetfillopacity{0.700000}%
\pgfsetlinewidth{0.000000pt}%
\definecolor{currentstroke}{rgb}{0.000000,0.000000,0.000000}%
\pgfsetstrokecolor{currentstroke}%
\pgfsetstrokeopacity{0.700000}%
\pgfsetdash{}{0pt}%
\pgfpathmoveto{\pgfqpoint{1.987571in}{0.550000in}}%
\pgfpathlineto{\pgfqpoint{2.015092in}{0.550000in}}%
\pgfpathlineto{\pgfqpoint{2.015092in}{1.029998in}}%
\pgfpathlineto{\pgfqpoint{1.987571in}{1.029998in}}%
\pgfpathlineto{\pgfqpoint{1.987571in}{0.550000in}}%
\pgfpathclose%
\pgfusepath{fill}%
\end{pgfscope}%
\begin{pgfscope}%
\pgfpathrectangle{\pgfqpoint{1.250000in}{0.550000in}}{\pgfqpoint{7.750000in}{3.850000in}}%
\pgfusepath{clip}%
\pgfsetbuttcap%
\pgfsetmiterjoin%
\definecolor{currentfill}{rgb}{0.501961,0.501961,0.501961}%
\pgfsetfillcolor{currentfill}%
\pgfsetfillopacity{0.700000}%
\pgfsetlinewidth{0.000000pt}%
\definecolor{currentstroke}{rgb}{0.000000,0.000000,0.000000}%
\pgfsetstrokecolor{currentstroke}%
\pgfsetstrokeopacity{0.700000}%
\pgfsetdash{}{0pt}%
\pgfpathmoveto{\pgfqpoint{2.015092in}{0.550000in}}%
\pgfpathlineto{\pgfqpoint{2.042614in}{0.550000in}}%
\pgfpathlineto{\pgfqpoint{2.042614in}{0.722095in}}%
\pgfpathlineto{\pgfqpoint{2.015092in}{0.722095in}}%
\pgfpathlineto{\pgfqpoint{2.015092in}{0.550000in}}%
\pgfpathclose%
\pgfusepath{fill}%
\end{pgfscope}%
\begin{pgfscope}%
\pgfpathrectangle{\pgfqpoint{1.250000in}{0.550000in}}{\pgfqpoint{7.750000in}{3.850000in}}%
\pgfusepath{clip}%
\pgfsetbuttcap%
\pgfsetmiterjoin%
\definecolor{currentfill}{rgb}{0.501961,0.501961,0.501961}%
\pgfsetfillcolor{currentfill}%
\pgfsetfillopacity{0.700000}%
\pgfsetlinewidth{0.000000pt}%
\definecolor{currentstroke}{rgb}{0.000000,0.000000,0.000000}%
\pgfsetstrokecolor{currentstroke}%
\pgfsetstrokeopacity{0.700000}%
\pgfsetdash{}{0pt}%
\pgfpathmoveto{\pgfqpoint{2.042614in}{0.550000in}}%
\pgfpathlineto{\pgfqpoint{2.070135in}{0.550000in}}%
\pgfpathlineto{\pgfqpoint{2.070135in}{0.843358in}}%
\pgfpathlineto{\pgfqpoint{2.042614in}{0.843358in}}%
\pgfpathlineto{\pgfqpoint{2.042614in}{0.550000in}}%
\pgfpathclose%
\pgfusepath{fill}%
\end{pgfscope}%
\begin{pgfscope}%
\pgfpathrectangle{\pgfqpoint{1.250000in}{0.550000in}}{\pgfqpoint{7.750000in}{3.850000in}}%
\pgfusepath{clip}%
\pgfsetbuttcap%
\pgfsetmiterjoin%
\definecolor{currentfill}{rgb}{0.501961,0.501961,0.501961}%
\pgfsetfillcolor{currentfill}%
\pgfsetfillopacity{0.700000}%
\pgfsetlinewidth{0.000000pt}%
\definecolor{currentstroke}{rgb}{0.000000,0.000000,0.000000}%
\pgfsetstrokecolor{currentstroke}%
\pgfsetstrokeopacity{0.700000}%
\pgfsetdash{}{0pt}%
\pgfpathmoveto{\pgfqpoint{2.070135in}{0.550000in}}%
\pgfpathlineto{\pgfqpoint{2.097656in}{0.550000in}}%
\pgfpathlineto{\pgfqpoint{2.097656in}{0.925884in}}%
\pgfpathlineto{\pgfqpoint{2.070135in}{0.925884in}}%
\pgfpathlineto{\pgfqpoint{2.070135in}{0.550000in}}%
\pgfpathclose%
\pgfusepath{fill}%
\end{pgfscope}%
\begin{pgfscope}%
\pgfpathrectangle{\pgfqpoint{1.250000in}{0.550000in}}{\pgfqpoint{7.750000in}{3.850000in}}%
\pgfusepath{clip}%
\pgfsetbuttcap%
\pgfsetmiterjoin%
\definecolor{currentfill}{rgb}{0.501961,0.501961,0.501961}%
\pgfsetfillcolor{currentfill}%
\pgfsetfillopacity{0.700000}%
\pgfsetlinewidth{0.000000pt}%
\definecolor{currentstroke}{rgb}{0.000000,0.000000,0.000000}%
\pgfsetstrokecolor{currentstroke}%
\pgfsetstrokeopacity{0.700000}%
\pgfsetdash{}{0pt}%
\pgfpathmoveto{\pgfqpoint{2.097656in}{0.550000in}}%
\pgfpathlineto{\pgfqpoint{2.125178in}{0.550000in}}%
\pgfpathlineto{\pgfqpoint{2.125178in}{0.796659in}}%
\pgfpathlineto{\pgfqpoint{2.097656in}{0.796659in}}%
\pgfpathlineto{\pgfqpoint{2.097656in}{0.550000in}}%
\pgfpathclose%
\pgfusepath{fill}%
\end{pgfscope}%
\begin{pgfscope}%
\pgfpathrectangle{\pgfqpoint{1.250000in}{0.550000in}}{\pgfqpoint{7.750000in}{3.850000in}}%
\pgfusepath{clip}%
\pgfsetbuttcap%
\pgfsetmiterjoin%
\definecolor{currentfill}{rgb}{0.501961,0.501961,0.501961}%
\pgfsetfillcolor{currentfill}%
\pgfsetfillopacity{0.700000}%
\pgfsetlinewidth{0.000000pt}%
\definecolor{currentstroke}{rgb}{0.000000,0.000000,0.000000}%
\pgfsetstrokecolor{currentstroke}%
\pgfsetstrokeopacity{0.700000}%
\pgfsetdash{}{0pt}%
\pgfpathmoveto{\pgfqpoint{2.125178in}{0.550000in}}%
\pgfpathlineto{\pgfqpoint{2.152699in}{0.550000in}}%
\pgfpathlineto{\pgfqpoint{2.152699in}{1.010400in}}%
\pgfpathlineto{\pgfqpoint{2.125178in}{1.010400in}}%
\pgfpathlineto{\pgfqpoint{2.125178in}{0.550000in}}%
\pgfpathclose%
\pgfusepath{fill}%
\end{pgfscope}%
\begin{pgfscope}%
\pgfpathrectangle{\pgfqpoint{1.250000in}{0.550000in}}{\pgfqpoint{7.750000in}{3.850000in}}%
\pgfusepath{clip}%
\pgfsetbuttcap%
\pgfsetmiterjoin%
\definecolor{currentfill}{rgb}{0.501961,0.501961,0.501961}%
\pgfsetfillcolor{currentfill}%
\pgfsetfillopacity{0.700000}%
\pgfsetlinewidth{0.000000pt}%
\definecolor{currentstroke}{rgb}{0.000000,0.000000,0.000000}%
\pgfsetstrokecolor{currentstroke}%
\pgfsetstrokeopacity{0.700000}%
\pgfsetdash{}{0pt}%
\pgfpathmoveto{\pgfqpoint{2.152699in}{0.550000in}}%
\pgfpathlineto{\pgfqpoint{2.180220in}{0.550000in}}%
\pgfpathlineto{\pgfqpoint{2.180220in}{0.932927in}}%
\pgfpathlineto{\pgfqpoint{2.152699in}{0.932927in}}%
\pgfpathlineto{\pgfqpoint{2.152699in}{0.550000in}}%
\pgfpathclose%
\pgfusepath{fill}%
\end{pgfscope}%
\begin{pgfscope}%
\pgfpathrectangle{\pgfqpoint{1.250000in}{0.550000in}}{\pgfqpoint{7.750000in}{3.850000in}}%
\pgfusepath{clip}%
\pgfsetbuttcap%
\pgfsetmiterjoin%
\definecolor{currentfill}{rgb}{0.501961,0.501961,0.501961}%
\pgfsetfillcolor{currentfill}%
\pgfsetfillopacity{0.700000}%
\pgfsetlinewidth{0.000000pt}%
\definecolor{currentstroke}{rgb}{0.000000,0.000000,0.000000}%
\pgfsetstrokecolor{currentstroke}%
\pgfsetstrokeopacity{0.700000}%
\pgfsetdash{}{0pt}%
\pgfpathmoveto{\pgfqpoint{2.180220in}{0.550000in}}%
\pgfpathlineto{\pgfqpoint{2.207741in}{0.550000in}}%
\pgfpathlineto{\pgfqpoint{2.207741in}{0.906592in}}%
\pgfpathlineto{\pgfqpoint{2.180220in}{0.906592in}}%
\pgfpathlineto{\pgfqpoint{2.180220in}{0.550000in}}%
\pgfpathclose%
\pgfusepath{fill}%
\end{pgfscope}%
\begin{pgfscope}%
\pgfpathrectangle{\pgfqpoint{1.250000in}{0.550000in}}{\pgfqpoint{7.750000in}{3.850000in}}%
\pgfusepath{clip}%
\pgfsetbuttcap%
\pgfsetmiterjoin%
\definecolor{currentfill}{rgb}{0.501961,0.501961,0.501961}%
\pgfsetfillcolor{currentfill}%
\pgfsetfillopacity{0.700000}%
\pgfsetlinewidth{0.000000pt}%
\definecolor{currentstroke}{rgb}{0.000000,0.000000,0.000000}%
\pgfsetstrokecolor{currentstroke}%
\pgfsetstrokeopacity{0.700000}%
\pgfsetdash{}{0pt}%
\pgfpathmoveto{\pgfqpoint{2.207741in}{0.550000in}}%
\pgfpathlineto{\pgfqpoint{2.235263in}{0.550000in}}%
\pgfpathlineto{\pgfqpoint{2.235263in}{1.190304in}}%
\pgfpathlineto{\pgfqpoint{2.207741in}{1.190304in}}%
\pgfpathlineto{\pgfqpoint{2.207741in}{0.550000in}}%
\pgfpathclose%
\pgfusepath{fill}%
\end{pgfscope}%
\begin{pgfscope}%
\pgfpathrectangle{\pgfqpoint{1.250000in}{0.550000in}}{\pgfqpoint{7.750000in}{3.850000in}}%
\pgfusepath{clip}%
\pgfsetbuttcap%
\pgfsetmiterjoin%
\definecolor{currentfill}{rgb}{0.501961,0.501961,0.501961}%
\pgfsetfillcolor{currentfill}%
\pgfsetfillopacity{0.700000}%
\pgfsetlinewidth{0.000000pt}%
\definecolor{currentstroke}{rgb}{0.000000,0.000000,0.000000}%
\pgfsetstrokecolor{currentstroke}%
\pgfsetstrokeopacity{0.700000}%
\pgfsetdash{}{0pt}%
\pgfpathmoveto{\pgfqpoint{2.235263in}{0.550000in}}%
\pgfpathlineto{\pgfqpoint{2.262784in}{0.550000in}}%
\pgfpathlineto{\pgfqpoint{2.262784in}{1.226285in}}%
\pgfpathlineto{\pgfqpoint{2.235263in}{1.226285in}}%
\pgfpathlineto{\pgfqpoint{2.235263in}{0.550000in}}%
\pgfpathclose%
\pgfusepath{fill}%
\end{pgfscope}%
\begin{pgfscope}%
\pgfpathrectangle{\pgfqpoint{1.250000in}{0.550000in}}{\pgfqpoint{7.750000in}{3.850000in}}%
\pgfusepath{clip}%
\pgfsetbuttcap%
\pgfsetmiterjoin%
\definecolor{currentfill}{rgb}{0.501961,0.501961,0.501961}%
\pgfsetfillcolor{currentfill}%
\pgfsetfillopacity{0.700000}%
\pgfsetlinewidth{0.000000pt}%
\definecolor{currentstroke}{rgb}{0.000000,0.000000,0.000000}%
\pgfsetstrokecolor{currentstroke}%
\pgfsetstrokeopacity{0.700000}%
\pgfsetdash{}{0pt}%
\pgfpathmoveto{\pgfqpoint{2.262784in}{0.550000in}}%
\pgfpathlineto{\pgfqpoint{2.290305in}{0.550000in}}%
\pgfpathlineto{\pgfqpoint{2.290305in}{1.089405in}}%
\pgfpathlineto{\pgfqpoint{2.262784in}{1.089405in}}%
\pgfpathlineto{\pgfqpoint{2.262784in}{0.550000in}}%
\pgfpathclose%
\pgfusepath{fill}%
\end{pgfscope}%
\begin{pgfscope}%
\pgfpathrectangle{\pgfqpoint{1.250000in}{0.550000in}}{\pgfqpoint{7.750000in}{3.850000in}}%
\pgfusepath{clip}%
\pgfsetbuttcap%
\pgfsetmiterjoin%
\definecolor{currentfill}{rgb}{0.501961,0.501961,0.501961}%
\pgfsetfillcolor{currentfill}%
\pgfsetfillopacity{0.700000}%
\pgfsetlinewidth{0.000000pt}%
\definecolor{currentstroke}{rgb}{0.000000,0.000000,0.000000}%
\pgfsetstrokecolor{currentstroke}%
\pgfsetstrokeopacity{0.700000}%
\pgfsetdash{}{0pt}%
\pgfpathmoveto{\pgfqpoint{2.290305in}{0.550000in}}%
\pgfpathlineto{\pgfqpoint{2.317827in}{0.550000in}}%
\pgfpathlineto{\pgfqpoint{2.317827in}{1.623604in}}%
\pgfpathlineto{\pgfqpoint{2.290305in}{1.623604in}}%
\pgfpathlineto{\pgfqpoint{2.290305in}{0.550000in}}%
\pgfpathclose%
\pgfusepath{fill}%
\end{pgfscope}%
\begin{pgfscope}%
\pgfpathrectangle{\pgfqpoint{1.250000in}{0.550000in}}{\pgfqpoint{7.750000in}{3.850000in}}%
\pgfusepath{clip}%
\pgfsetbuttcap%
\pgfsetmiterjoin%
\definecolor{currentfill}{rgb}{0.501961,0.501961,0.501961}%
\pgfsetfillcolor{currentfill}%
\pgfsetfillopacity{0.700000}%
\pgfsetlinewidth{0.000000pt}%
\definecolor{currentstroke}{rgb}{0.000000,0.000000,0.000000}%
\pgfsetstrokecolor{currentstroke}%
\pgfsetstrokeopacity{0.700000}%
\pgfsetdash{}{0pt}%
\pgfpathmoveto{\pgfqpoint{2.317827in}{0.550000in}}%
\pgfpathlineto{\pgfqpoint{2.345348in}{0.550000in}}%
\pgfpathlineto{\pgfqpoint{2.345348in}{1.356122in}}%
\pgfpathlineto{\pgfqpoint{2.317827in}{1.356122in}}%
\pgfpathlineto{\pgfqpoint{2.317827in}{0.550000in}}%
\pgfpathclose%
\pgfusepath{fill}%
\end{pgfscope}%
\begin{pgfscope}%
\pgfpathrectangle{\pgfqpoint{1.250000in}{0.550000in}}{\pgfqpoint{7.750000in}{3.850000in}}%
\pgfusepath{clip}%
\pgfsetbuttcap%
\pgfsetmiterjoin%
\definecolor{currentfill}{rgb}{0.501961,0.501961,0.501961}%
\pgfsetfillcolor{currentfill}%
\pgfsetfillopacity{0.700000}%
\pgfsetlinewidth{0.000000pt}%
\definecolor{currentstroke}{rgb}{0.000000,0.000000,0.000000}%
\pgfsetstrokecolor{currentstroke}%
\pgfsetstrokeopacity{0.700000}%
\pgfsetdash{}{0pt}%
\pgfpathmoveto{\pgfqpoint{2.345348in}{0.550000in}}%
\pgfpathlineto{\pgfqpoint{2.372869in}{0.550000in}}%
\pgfpathlineto{\pgfqpoint{2.372869in}{1.344332in}}%
\pgfpathlineto{\pgfqpoint{2.345348in}{1.344332in}}%
\pgfpathlineto{\pgfqpoint{2.345348in}{0.550000in}}%
\pgfpathclose%
\pgfusepath{fill}%
\end{pgfscope}%
\begin{pgfscope}%
\pgfpathrectangle{\pgfqpoint{1.250000in}{0.550000in}}{\pgfqpoint{7.750000in}{3.850000in}}%
\pgfusepath{clip}%
\pgfsetbuttcap%
\pgfsetmiterjoin%
\definecolor{currentfill}{rgb}{0.501961,0.501961,0.501961}%
\pgfsetfillcolor{currentfill}%
\pgfsetfillopacity{0.700000}%
\pgfsetlinewidth{0.000000pt}%
\definecolor{currentstroke}{rgb}{0.000000,0.000000,0.000000}%
\pgfsetstrokecolor{currentstroke}%
\pgfsetstrokeopacity{0.700000}%
\pgfsetdash{}{0pt}%
\pgfpathmoveto{\pgfqpoint{2.372869in}{0.550000in}}%
\pgfpathlineto{\pgfqpoint{2.400391in}{0.550000in}}%
\pgfpathlineto{\pgfqpoint{2.400391in}{1.777785in}}%
\pgfpathlineto{\pgfqpoint{2.372869in}{1.777785in}}%
\pgfpathlineto{\pgfqpoint{2.372869in}{0.550000in}}%
\pgfpathclose%
\pgfusepath{fill}%
\end{pgfscope}%
\begin{pgfscope}%
\pgfpathrectangle{\pgfqpoint{1.250000in}{0.550000in}}{\pgfqpoint{7.750000in}{3.850000in}}%
\pgfusepath{clip}%
\pgfsetbuttcap%
\pgfsetmiterjoin%
\definecolor{currentfill}{rgb}{0.501961,0.501961,0.501961}%
\pgfsetfillcolor{currentfill}%
\pgfsetfillopacity{0.700000}%
\pgfsetlinewidth{0.000000pt}%
\definecolor{currentstroke}{rgb}{0.000000,0.000000,0.000000}%
\pgfsetstrokecolor{currentstroke}%
\pgfsetstrokeopacity{0.700000}%
\pgfsetdash{}{0pt}%
\pgfpathmoveto{\pgfqpoint{2.400391in}{0.550000in}}%
\pgfpathlineto{\pgfqpoint{2.427912in}{0.550000in}}%
\pgfpathlineto{\pgfqpoint{2.427912in}{1.537403in}}%
\pgfpathlineto{\pgfqpoint{2.400391in}{1.537403in}}%
\pgfpathlineto{\pgfqpoint{2.400391in}{0.550000in}}%
\pgfpathclose%
\pgfusepath{fill}%
\end{pgfscope}%
\begin{pgfscope}%
\pgfpathrectangle{\pgfqpoint{1.250000in}{0.550000in}}{\pgfqpoint{7.750000in}{3.850000in}}%
\pgfusepath{clip}%
\pgfsetbuttcap%
\pgfsetmiterjoin%
\definecolor{currentfill}{rgb}{0.501961,0.501961,0.501961}%
\pgfsetfillcolor{currentfill}%
\pgfsetfillopacity{0.700000}%
\pgfsetlinewidth{0.000000pt}%
\definecolor{currentstroke}{rgb}{0.000000,0.000000,0.000000}%
\pgfsetstrokecolor{currentstroke}%
\pgfsetstrokeopacity{0.700000}%
\pgfsetdash{}{0pt}%
\pgfpathmoveto{\pgfqpoint{2.427912in}{0.550000in}}%
\pgfpathlineto{\pgfqpoint{2.455433in}{0.550000in}}%
\pgfpathlineto{\pgfqpoint{2.455433in}{1.513978in}}%
\pgfpathlineto{\pgfqpoint{2.427912in}{1.513978in}}%
\pgfpathlineto{\pgfqpoint{2.427912in}{0.550000in}}%
\pgfpathclose%
\pgfusepath{fill}%
\end{pgfscope}%
\begin{pgfscope}%
\pgfpathrectangle{\pgfqpoint{1.250000in}{0.550000in}}{\pgfqpoint{7.750000in}{3.850000in}}%
\pgfusepath{clip}%
\pgfsetbuttcap%
\pgfsetmiterjoin%
\definecolor{currentfill}{rgb}{0.501961,0.501961,0.501961}%
\pgfsetfillcolor{currentfill}%
\pgfsetfillopacity{0.700000}%
\pgfsetlinewidth{0.000000pt}%
\definecolor{currentstroke}{rgb}{0.000000,0.000000,0.000000}%
\pgfsetstrokecolor{currentstroke}%
\pgfsetstrokeopacity{0.700000}%
\pgfsetdash{}{0pt}%
\pgfpathmoveto{\pgfqpoint{2.455433in}{0.550000in}}%
\pgfpathlineto{\pgfqpoint{2.482955in}{0.550000in}}%
\pgfpathlineto{\pgfqpoint{2.482955in}{1.840101in}}%
\pgfpathlineto{\pgfqpoint{2.455433in}{1.840101in}}%
\pgfpathlineto{\pgfqpoint{2.455433in}{0.550000in}}%
\pgfpathclose%
\pgfusepath{fill}%
\end{pgfscope}%
\begin{pgfscope}%
\pgfpathrectangle{\pgfqpoint{1.250000in}{0.550000in}}{\pgfqpoint{7.750000in}{3.850000in}}%
\pgfusepath{clip}%
\pgfsetbuttcap%
\pgfsetmiterjoin%
\definecolor{currentfill}{rgb}{0.501961,0.501961,0.501961}%
\pgfsetfillcolor{currentfill}%
\pgfsetfillopacity{0.700000}%
\pgfsetlinewidth{0.000000pt}%
\definecolor{currentstroke}{rgb}{0.000000,0.000000,0.000000}%
\pgfsetstrokecolor{currentstroke}%
\pgfsetstrokeopacity{0.700000}%
\pgfsetdash{}{0pt}%
\pgfpathmoveto{\pgfqpoint{2.482955in}{0.550000in}}%
\pgfpathlineto{\pgfqpoint{2.510476in}{0.550000in}}%
\pgfpathlineto{\pgfqpoint{2.510476in}{1.723125in}}%
\pgfpathlineto{\pgfqpoint{2.482955in}{1.723125in}}%
\pgfpathlineto{\pgfqpoint{2.482955in}{0.550000in}}%
\pgfpathclose%
\pgfusepath{fill}%
\end{pgfscope}%
\begin{pgfscope}%
\pgfpathrectangle{\pgfqpoint{1.250000in}{0.550000in}}{\pgfqpoint{7.750000in}{3.850000in}}%
\pgfusepath{clip}%
\pgfsetbuttcap%
\pgfsetmiterjoin%
\definecolor{currentfill}{rgb}{0.501961,0.501961,0.501961}%
\pgfsetfillcolor{currentfill}%
\pgfsetfillopacity{0.700000}%
\pgfsetlinewidth{0.000000pt}%
\definecolor{currentstroke}{rgb}{0.000000,0.000000,0.000000}%
\pgfsetstrokecolor{currentstroke}%
\pgfsetstrokeopacity{0.700000}%
\pgfsetdash{}{0pt}%
\pgfpathmoveto{\pgfqpoint{2.510476in}{0.550000in}}%
\pgfpathlineto{\pgfqpoint{2.537997in}{0.550000in}}%
\pgfpathlineto{\pgfqpoint{2.537997in}{1.680867in}}%
\pgfpathlineto{\pgfqpoint{2.510476in}{1.680867in}}%
\pgfpathlineto{\pgfqpoint{2.510476in}{0.550000in}}%
\pgfpathclose%
\pgfusepath{fill}%
\end{pgfscope}%
\begin{pgfscope}%
\pgfpathrectangle{\pgfqpoint{1.250000in}{0.550000in}}{\pgfqpoint{7.750000in}{3.850000in}}%
\pgfusepath{clip}%
\pgfsetbuttcap%
\pgfsetmiterjoin%
\definecolor{currentfill}{rgb}{0.501961,0.501961,0.501961}%
\pgfsetfillcolor{currentfill}%
\pgfsetfillopacity{0.700000}%
\pgfsetlinewidth{0.000000pt}%
\definecolor{currentstroke}{rgb}{0.000000,0.000000,0.000000}%
\pgfsetstrokecolor{currentstroke}%
\pgfsetstrokeopacity{0.700000}%
\pgfsetdash{}{0pt}%
\pgfpathmoveto{\pgfqpoint{2.537997in}{0.550000in}}%
\pgfpathlineto{\pgfqpoint{2.565518in}{0.550000in}}%
\pgfpathlineto{\pgfqpoint{2.565518in}{1.612580in}}%
\pgfpathlineto{\pgfqpoint{2.537997in}{1.612580in}}%
\pgfpathlineto{\pgfqpoint{2.537997in}{0.550000in}}%
\pgfpathclose%
\pgfusepath{fill}%
\end{pgfscope}%
\begin{pgfscope}%
\pgfpathrectangle{\pgfqpoint{1.250000in}{0.550000in}}{\pgfqpoint{7.750000in}{3.850000in}}%
\pgfusepath{clip}%
\pgfsetbuttcap%
\pgfsetmiterjoin%
\definecolor{currentfill}{rgb}{0.501961,0.501961,0.501961}%
\pgfsetfillcolor{currentfill}%
\pgfsetfillopacity{0.700000}%
\pgfsetlinewidth{0.000000pt}%
\definecolor{currentstroke}{rgb}{0.000000,0.000000,0.000000}%
\pgfsetstrokecolor{currentstroke}%
\pgfsetstrokeopacity{0.700000}%
\pgfsetdash{}{0pt}%
\pgfpathmoveto{\pgfqpoint{2.565518in}{0.550000in}}%
\pgfpathlineto{\pgfqpoint{2.593040in}{0.550000in}}%
\pgfpathlineto{\pgfqpoint{2.593040in}{1.534341in}}%
\pgfpathlineto{\pgfqpoint{2.565518in}{1.534341in}}%
\pgfpathlineto{\pgfqpoint{2.565518in}{0.550000in}}%
\pgfpathclose%
\pgfusepath{fill}%
\end{pgfscope}%
\begin{pgfscope}%
\pgfpathrectangle{\pgfqpoint{1.250000in}{0.550000in}}{\pgfqpoint{7.750000in}{3.850000in}}%
\pgfusepath{clip}%
\pgfsetbuttcap%
\pgfsetmiterjoin%
\definecolor{currentfill}{rgb}{0.501961,0.501961,0.501961}%
\pgfsetfillcolor{currentfill}%
\pgfsetfillopacity{0.700000}%
\pgfsetlinewidth{0.000000pt}%
\definecolor{currentstroke}{rgb}{0.000000,0.000000,0.000000}%
\pgfsetstrokecolor{currentstroke}%
\pgfsetstrokeopacity{0.700000}%
\pgfsetdash{}{0pt}%
\pgfpathmoveto{\pgfqpoint{2.593040in}{0.550000in}}%
\pgfpathlineto{\pgfqpoint{2.620561in}{0.550000in}}%
\pgfpathlineto{\pgfqpoint{2.620561in}{1.638609in}}%
\pgfpathlineto{\pgfqpoint{2.593040in}{1.638609in}}%
\pgfpathlineto{\pgfqpoint{2.593040in}{0.550000in}}%
\pgfpathclose%
\pgfusepath{fill}%
\end{pgfscope}%
\begin{pgfscope}%
\pgfpathrectangle{\pgfqpoint{1.250000in}{0.550000in}}{\pgfqpoint{7.750000in}{3.850000in}}%
\pgfusepath{clip}%
\pgfsetbuttcap%
\pgfsetmiterjoin%
\definecolor{currentfill}{rgb}{0.501961,0.501961,0.501961}%
\pgfsetfillcolor{currentfill}%
\pgfsetfillopacity{0.700000}%
\pgfsetlinewidth{0.000000pt}%
\definecolor{currentstroke}{rgb}{0.000000,0.000000,0.000000}%
\pgfsetstrokecolor{currentstroke}%
\pgfsetstrokeopacity{0.700000}%
\pgfsetdash{}{0pt}%
\pgfpathmoveto{\pgfqpoint{2.620561in}{0.550000in}}%
\pgfpathlineto{\pgfqpoint{2.648082in}{0.550000in}}%
\pgfpathlineto{\pgfqpoint{2.648082in}{1.574915in}}%
\pgfpathlineto{\pgfqpoint{2.620561in}{1.574915in}}%
\pgfpathlineto{\pgfqpoint{2.620561in}{0.550000in}}%
\pgfpathclose%
\pgfusepath{fill}%
\end{pgfscope}%
\begin{pgfscope}%
\pgfpathrectangle{\pgfqpoint{1.250000in}{0.550000in}}{\pgfqpoint{7.750000in}{3.850000in}}%
\pgfusepath{clip}%
\pgfsetbuttcap%
\pgfsetmiterjoin%
\definecolor{currentfill}{rgb}{0.501961,0.501961,0.501961}%
\pgfsetfillcolor{currentfill}%
\pgfsetfillopacity{0.700000}%
\pgfsetlinewidth{0.000000pt}%
\definecolor{currentstroke}{rgb}{0.000000,0.000000,0.000000}%
\pgfsetstrokecolor{currentstroke}%
\pgfsetstrokeopacity{0.700000}%
\pgfsetdash{}{0pt}%
\pgfpathmoveto{\pgfqpoint{2.648082in}{0.550000in}}%
\pgfpathlineto{\pgfqpoint{2.675604in}{0.550000in}}%
\pgfpathlineto{\pgfqpoint{2.675604in}{1.368677in}}%
\pgfpathlineto{\pgfqpoint{2.648082in}{1.368677in}}%
\pgfpathlineto{\pgfqpoint{2.648082in}{0.550000in}}%
\pgfpathclose%
\pgfusepath{fill}%
\end{pgfscope}%
\begin{pgfscope}%
\pgfpathrectangle{\pgfqpoint{1.250000in}{0.550000in}}{\pgfqpoint{7.750000in}{3.850000in}}%
\pgfusepath{clip}%
\pgfsetbuttcap%
\pgfsetmiterjoin%
\definecolor{currentfill}{rgb}{0.501961,0.501961,0.501961}%
\pgfsetfillcolor{currentfill}%
\pgfsetfillopacity{0.700000}%
\pgfsetlinewidth{0.000000pt}%
\definecolor{currentstroke}{rgb}{0.000000,0.000000,0.000000}%
\pgfsetstrokecolor{currentstroke}%
\pgfsetstrokeopacity{0.700000}%
\pgfsetdash{}{0pt}%
\pgfpathmoveto{\pgfqpoint{2.675604in}{0.550000in}}%
\pgfpathlineto{\pgfqpoint{2.703125in}{0.550000in}}%
\pgfpathlineto{\pgfqpoint{2.703125in}{1.402054in}}%
\pgfpathlineto{\pgfqpoint{2.675604in}{1.402054in}}%
\pgfpathlineto{\pgfqpoint{2.675604in}{0.550000in}}%
\pgfpathclose%
\pgfusepath{fill}%
\end{pgfscope}%
\begin{pgfscope}%
\pgfpathrectangle{\pgfqpoint{1.250000in}{0.550000in}}{\pgfqpoint{7.750000in}{3.850000in}}%
\pgfusepath{clip}%
\pgfsetbuttcap%
\pgfsetmiterjoin%
\definecolor{currentfill}{rgb}{0.501961,0.501961,0.501961}%
\pgfsetfillcolor{currentfill}%
\pgfsetfillopacity{0.700000}%
\pgfsetlinewidth{0.000000pt}%
\definecolor{currentstroke}{rgb}{0.000000,0.000000,0.000000}%
\pgfsetstrokecolor{currentstroke}%
\pgfsetstrokeopacity{0.700000}%
\pgfsetdash{}{0pt}%
\pgfpathmoveto{\pgfqpoint{2.703125in}{0.550000in}}%
\pgfpathlineto{\pgfqpoint{2.730646in}{0.550000in}}%
\pgfpathlineto{\pgfqpoint{2.730646in}{1.303605in}}%
\pgfpathlineto{\pgfqpoint{2.703125in}{1.303605in}}%
\pgfpathlineto{\pgfqpoint{2.703125in}{0.550000in}}%
\pgfpathclose%
\pgfusepath{fill}%
\end{pgfscope}%
\begin{pgfscope}%
\pgfpathrectangle{\pgfqpoint{1.250000in}{0.550000in}}{\pgfqpoint{7.750000in}{3.850000in}}%
\pgfusepath{clip}%
\pgfsetbuttcap%
\pgfsetmiterjoin%
\definecolor{currentfill}{rgb}{0.501961,0.501961,0.501961}%
\pgfsetfillcolor{currentfill}%
\pgfsetfillopacity{0.700000}%
\pgfsetlinewidth{0.000000pt}%
\definecolor{currentstroke}{rgb}{0.000000,0.000000,0.000000}%
\pgfsetstrokecolor{currentstroke}%
\pgfsetstrokeopacity{0.700000}%
\pgfsetdash{}{0pt}%
\pgfpathmoveto{\pgfqpoint{2.730646in}{0.550000in}}%
\pgfpathlineto{\pgfqpoint{2.758168in}{0.550000in}}%
\pgfpathlineto{\pgfqpoint{2.758168in}{1.292734in}}%
\pgfpathlineto{\pgfqpoint{2.730646in}{1.292734in}}%
\pgfpathlineto{\pgfqpoint{2.730646in}{0.550000in}}%
\pgfpathclose%
\pgfusepath{fill}%
\end{pgfscope}%
\begin{pgfscope}%
\pgfpathrectangle{\pgfqpoint{1.250000in}{0.550000in}}{\pgfqpoint{7.750000in}{3.850000in}}%
\pgfusepath{clip}%
\pgfsetbuttcap%
\pgfsetmiterjoin%
\definecolor{currentfill}{rgb}{0.501961,0.501961,0.501961}%
\pgfsetfillcolor{currentfill}%
\pgfsetfillopacity{0.700000}%
\pgfsetlinewidth{0.000000pt}%
\definecolor{currentstroke}{rgb}{0.000000,0.000000,0.000000}%
\pgfsetstrokecolor{currentstroke}%
\pgfsetstrokeopacity{0.700000}%
\pgfsetdash{}{0pt}%
\pgfpathmoveto{\pgfqpoint{2.758168in}{0.550000in}}%
\pgfpathlineto{\pgfqpoint{2.785689in}{0.550000in}}%
\pgfpathlineto{\pgfqpoint{2.785689in}{1.114974in}}%
\pgfpathlineto{\pgfqpoint{2.758168in}{1.114974in}}%
\pgfpathlineto{\pgfqpoint{2.758168in}{0.550000in}}%
\pgfpathclose%
\pgfusepath{fill}%
\end{pgfscope}%
\begin{pgfscope}%
\pgfpathrectangle{\pgfqpoint{1.250000in}{0.550000in}}{\pgfqpoint{7.750000in}{3.850000in}}%
\pgfusepath{clip}%
\pgfsetbuttcap%
\pgfsetmiterjoin%
\definecolor{currentfill}{rgb}{0.501961,0.501961,0.501961}%
\pgfsetfillcolor{currentfill}%
\pgfsetfillopacity{0.700000}%
\pgfsetlinewidth{0.000000pt}%
\definecolor{currentstroke}{rgb}{0.000000,0.000000,0.000000}%
\pgfsetstrokecolor{currentstroke}%
\pgfsetstrokeopacity{0.700000}%
\pgfsetdash{}{0pt}%
\pgfpathmoveto{\pgfqpoint{2.785689in}{0.550000in}}%
\pgfpathlineto{\pgfqpoint{2.813210in}{0.550000in}}%
\pgfpathlineto{\pgfqpoint{2.813210in}{1.037194in}}%
\pgfpathlineto{\pgfqpoint{2.785689in}{1.037194in}}%
\pgfpathlineto{\pgfqpoint{2.785689in}{0.550000in}}%
\pgfpathclose%
\pgfusepath{fill}%
\end{pgfscope}%
\begin{pgfscope}%
\pgfpathrectangle{\pgfqpoint{1.250000in}{0.550000in}}{\pgfqpoint{7.750000in}{3.850000in}}%
\pgfusepath{clip}%
\pgfsetbuttcap%
\pgfsetmiterjoin%
\definecolor{currentfill}{rgb}{0.501961,0.501961,0.501961}%
\pgfsetfillcolor{currentfill}%
\pgfsetfillopacity{0.700000}%
\pgfsetlinewidth{0.000000pt}%
\definecolor{currentstroke}{rgb}{0.000000,0.000000,0.000000}%
\pgfsetstrokecolor{currentstroke}%
\pgfsetstrokeopacity{0.700000}%
\pgfsetdash{}{0pt}%
\pgfpathmoveto{\pgfqpoint{2.813210in}{0.550000in}}%
\pgfpathlineto{\pgfqpoint{2.840732in}{0.550000in}}%
\pgfpathlineto{\pgfqpoint{2.840732in}{1.109462in}}%
\pgfpathlineto{\pgfqpoint{2.813210in}{1.109462in}}%
\pgfpathlineto{\pgfqpoint{2.813210in}{0.550000in}}%
\pgfpathclose%
\pgfusepath{fill}%
\end{pgfscope}%
\begin{pgfscope}%
\pgfpathrectangle{\pgfqpoint{1.250000in}{0.550000in}}{\pgfqpoint{7.750000in}{3.850000in}}%
\pgfusepath{clip}%
\pgfsetbuttcap%
\pgfsetmiterjoin%
\definecolor{currentfill}{rgb}{0.501961,0.501961,0.501961}%
\pgfsetfillcolor{currentfill}%
\pgfsetfillopacity{0.700000}%
\pgfsetlinewidth{0.000000pt}%
\definecolor{currentstroke}{rgb}{0.000000,0.000000,0.000000}%
\pgfsetstrokecolor{currentstroke}%
\pgfsetstrokeopacity{0.700000}%
\pgfsetdash{}{0pt}%
\pgfpathmoveto{\pgfqpoint{2.840732in}{0.550000in}}%
\pgfpathlineto{\pgfqpoint{2.868253in}{0.550000in}}%
\pgfpathlineto{\pgfqpoint{2.868253in}{0.884085in}}%
\pgfpathlineto{\pgfqpoint{2.840732in}{0.884085in}}%
\pgfpathlineto{\pgfqpoint{2.840732in}{0.550000in}}%
\pgfpathclose%
\pgfusepath{fill}%
\end{pgfscope}%
\begin{pgfscope}%
\pgfpathrectangle{\pgfqpoint{1.250000in}{0.550000in}}{\pgfqpoint{7.750000in}{3.850000in}}%
\pgfusepath{clip}%
\pgfsetbuttcap%
\pgfsetmiterjoin%
\definecolor{currentfill}{rgb}{0.501961,0.501961,0.501961}%
\pgfsetfillcolor{currentfill}%
\pgfsetfillopacity{0.700000}%
\pgfsetlinewidth{0.000000pt}%
\definecolor{currentstroke}{rgb}{0.000000,0.000000,0.000000}%
\pgfsetstrokecolor{currentstroke}%
\pgfsetstrokeopacity{0.700000}%
\pgfsetdash{}{0pt}%
\pgfpathmoveto{\pgfqpoint{2.868253in}{0.550000in}}%
\pgfpathlineto{\pgfqpoint{2.895774in}{0.550000in}}%
\pgfpathlineto{\pgfqpoint{2.895774in}{0.881023in}}%
\pgfpathlineto{\pgfqpoint{2.868253in}{0.881023in}}%
\pgfpathlineto{\pgfqpoint{2.868253in}{0.550000in}}%
\pgfpathclose%
\pgfusepath{fill}%
\end{pgfscope}%
\begin{pgfscope}%
\pgfpathrectangle{\pgfqpoint{1.250000in}{0.550000in}}{\pgfqpoint{7.750000in}{3.850000in}}%
\pgfusepath{clip}%
\pgfsetbuttcap%
\pgfsetmiterjoin%
\definecolor{currentfill}{rgb}{0.501961,0.501961,0.501961}%
\pgfsetfillcolor{currentfill}%
\pgfsetfillopacity{0.700000}%
\pgfsetlinewidth{0.000000pt}%
\definecolor{currentstroke}{rgb}{0.000000,0.000000,0.000000}%
\pgfsetstrokecolor{currentstroke}%
\pgfsetstrokeopacity{0.700000}%
\pgfsetdash{}{0pt}%
\pgfpathmoveto{\pgfqpoint{2.895774in}{0.550000in}}%
\pgfpathlineto{\pgfqpoint{2.923295in}{0.550000in}}%
\pgfpathlineto{\pgfqpoint{2.923295in}{0.901846in}}%
\pgfpathlineto{\pgfqpoint{2.895774in}{0.901846in}}%
\pgfpathlineto{\pgfqpoint{2.895774in}{0.550000in}}%
\pgfpathclose%
\pgfusepath{fill}%
\end{pgfscope}%
\begin{pgfscope}%
\pgfpathrectangle{\pgfqpoint{1.250000in}{0.550000in}}{\pgfqpoint{7.750000in}{3.850000in}}%
\pgfusepath{clip}%
\pgfsetbuttcap%
\pgfsetmiterjoin%
\definecolor{currentfill}{rgb}{0.501961,0.501961,0.501961}%
\pgfsetfillcolor{currentfill}%
\pgfsetfillopacity{0.700000}%
\pgfsetlinewidth{0.000000pt}%
\definecolor{currentstroke}{rgb}{0.000000,0.000000,0.000000}%
\pgfsetstrokecolor{currentstroke}%
\pgfsetstrokeopacity{0.700000}%
\pgfsetdash{}{0pt}%
\pgfpathmoveto{\pgfqpoint{2.923295in}{0.550000in}}%
\pgfpathlineto{\pgfqpoint{2.950817in}{0.550000in}}%
\pgfpathlineto{\pgfqpoint{2.950817in}{0.780430in}}%
\pgfpathlineto{\pgfqpoint{2.923295in}{0.780430in}}%
\pgfpathlineto{\pgfqpoint{2.923295in}{0.550000in}}%
\pgfpathclose%
\pgfusepath{fill}%
\end{pgfscope}%
\begin{pgfscope}%
\pgfpathrectangle{\pgfqpoint{1.250000in}{0.550000in}}{\pgfqpoint{7.750000in}{3.850000in}}%
\pgfusepath{clip}%
\pgfsetbuttcap%
\pgfsetmiterjoin%
\definecolor{currentfill}{rgb}{0.501961,0.501961,0.501961}%
\pgfsetfillcolor{currentfill}%
\pgfsetfillopacity{0.700000}%
\pgfsetlinewidth{0.000000pt}%
\definecolor{currentstroke}{rgb}{0.000000,0.000000,0.000000}%
\pgfsetstrokecolor{currentstroke}%
\pgfsetstrokeopacity{0.700000}%
\pgfsetdash{}{0pt}%
\pgfpathmoveto{\pgfqpoint{2.950817in}{0.550000in}}%
\pgfpathlineto{\pgfqpoint{2.978338in}{0.550000in}}%
\pgfpathlineto{\pgfqpoint{2.978338in}{0.798803in}}%
\pgfpathlineto{\pgfqpoint{2.950817in}{0.798803in}}%
\pgfpathlineto{\pgfqpoint{2.950817in}{0.550000in}}%
\pgfpathclose%
\pgfusepath{fill}%
\end{pgfscope}%
\begin{pgfscope}%
\pgfpathrectangle{\pgfqpoint{1.250000in}{0.550000in}}{\pgfqpoint{7.750000in}{3.850000in}}%
\pgfusepath{clip}%
\pgfsetbuttcap%
\pgfsetmiterjoin%
\definecolor{currentfill}{rgb}{0.501961,0.501961,0.501961}%
\pgfsetfillcolor{currentfill}%
\pgfsetfillopacity{0.700000}%
\pgfsetlinewidth{0.000000pt}%
\definecolor{currentstroke}{rgb}{0.000000,0.000000,0.000000}%
\pgfsetstrokecolor{currentstroke}%
\pgfsetstrokeopacity{0.700000}%
\pgfsetdash{}{0pt}%
\pgfpathmoveto{\pgfqpoint{2.978338in}{0.550000in}}%
\pgfpathlineto{\pgfqpoint{3.005859in}{0.550000in}}%
\pgfpathlineto{\pgfqpoint{3.005859in}{0.760219in}}%
\pgfpathlineto{\pgfqpoint{2.978338in}{0.760219in}}%
\pgfpathlineto{\pgfqpoint{2.978338in}{0.550000in}}%
\pgfpathclose%
\pgfusepath{fill}%
\end{pgfscope}%
\begin{pgfscope}%
\pgfpathrectangle{\pgfqpoint{1.250000in}{0.550000in}}{\pgfqpoint{7.750000in}{3.850000in}}%
\pgfusepath{clip}%
\pgfsetbuttcap%
\pgfsetmiterjoin%
\definecolor{currentfill}{rgb}{0.501961,0.501961,0.501961}%
\pgfsetfillcolor{currentfill}%
\pgfsetfillopacity{0.700000}%
\pgfsetlinewidth{0.000000pt}%
\definecolor{currentstroke}{rgb}{0.000000,0.000000,0.000000}%
\pgfsetstrokecolor{currentstroke}%
\pgfsetstrokeopacity{0.700000}%
\pgfsetdash{}{0pt}%
\pgfpathmoveto{\pgfqpoint{3.005859in}{0.550000in}}%
\pgfpathlineto{\pgfqpoint{3.033381in}{0.550000in}}%
\pgfpathlineto{\pgfqpoint{3.033381in}{0.721330in}}%
\pgfpathlineto{\pgfqpoint{3.005859in}{0.721330in}}%
\pgfpathlineto{\pgfqpoint{3.005859in}{0.550000in}}%
\pgfpathclose%
\pgfusepath{fill}%
\end{pgfscope}%
\begin{pgfscope}%
\pgfpathrectangle{\pgfqpoint{1.250000in}{0.550000in}}{\pgfqpoint{7.750000in}{3.850000in}}%
\pgfusepath{clip}%
\pgfsetbuttcap%
\pgfsetmiterjoin%
\definecolor{currentfill}{rgb}{0.501961,0.501961,0.501961}%
\pgfsetfillcolor{currentfill}%
\pgfsetfillopacity{0.700000}%
\pgfsetlinewidth{0.000000pt}%
\definecolor{currentstroke}{rgb}{0.000000,0.000000,0.000000}%
\pgfsetstrokecolor{currentstroke}%
\pgfsetstrokeopacity{0.700000}%
\pgfsetdash{}{0pt}%
\pgfpathmoveto{\pgfqpoint{3.033381in}{0.550000in}}%
\pgfpathlineto{\pgfqpoint{3.060902in}{0.550000in}}%
\pgfpathlineto{\pgfqpoint{3.060902in}{0.761291in}}%
\pgfpathlineto{\pgfqpoint{3.033381in}{0.761291in}}%
\pgfpathlineto{\pgfqpoint{3.033381in}{0.550000in}}%
\pgfpathclose%
\pgfusepath{fill}%
\end{pgfscope}%
\begin{pgfscope}%
\pgfpathrectangle{\pgfqpoint{1.250000in}{0.550000in}}{\pgfqpoint{7.750000in}{3.850000in}}%
\pgfusepath{clip}%
\pgfsetbuttcap%
\pgfsetmiterjoin%
\definecolor{currentfill}{rgb}{0.501961,0.501961,0.501961}%
\pgfsetfillcolor{currentfill}%
\pgfsetfillopacity{0.700000}%
\pgfsetlinewidth{0.000000pt}%
\definecolor{currentstroke}{rgb}{0.000000,0.000000,0.000000}%
\pgfsetstrokecolor{currentstroke}%
\pgfsetstrokeopacity{0.700000}%
\pgfsetdash{}{0pt}%
\pgfpathmoveto{\pgfqpoint{3.060902in}{0.550000in}}%
\pgfpathlineto{\pgfqpoint{3.088423in}{0.550000in}}%
\pgfpathlineto{\pgfqpoint{3.088423in}{0.709693in}}%
\pgfpathlineto{\pgfqpoint{3.060902in}{0.709693in}}%
\pgfpathlineto{\pgfqpoint{3.060902in}{0.550000in}}%
\pgfpathclose%
\pgfusepath{fill}%
\end{pgfscope}%
\begin{pgfscope}%
\pgfpathrectangle{\pgfqpoint{1.250000in}{0.550000in}}{\pgfqpoint{7.750000in}{3.850000in}}%
\pgfusepath{clip}%
\pgfsetbuttcap%
\pgfsetmiterjoin%
\definecolor{currentfill}{rgb}{0.501961,0.501961,0.501961}%
\pgfsetfillcolor{currentfill}%
\pgfsetfillopacity{0.700000}%
\pgfsetlinewidth{0.000000pt}%
\definecolor{currentstroke}{rgb}{0.000000,0.000000,0.000000}%
\pgfsetstrokecolor{currentstroke}%
\pgfsetstrokeopacity{0.700000}%
\pgfsetdash{}{0pt}%
\pgfpathmoveto{\pgfqpoint{3.088423in}{0.550000in}}%
\pgfpathlineto{\pgfqpoint{3.115945in}{0.550000in}}%
\pgfpathlineto{\pgfqpoint{3.115945in}{0.738631in}}%
\pgfpathlineto{\pgfqpoint{3.088423in}{0.738631in}}%
\pgfpathlineto{\pgfqpoint{3.088423in}{0.550000in}}%
\pgfpathclose%
\pgfusepath{fill}%
\end{pgfscope}%
\begin{pgfscope}%
\pgfpathrectangle{\pgfqpoint{1.250000in}{0.550000in}}{\pgfqpoint{7.750000in}{3.850000in}}%
\pgfusepath{clip}%
\pgfsetbuttcap%
\pgfsetmiterjoin%
\definecolor{currentfill}{rgb}{0.501961,0.501961,0.501961}%
\pgfsetfillcolor{currentfill}%
\pgfsetfillopacity{0.700000}%
\pgfsetlinewidth{0.000000pt}%
\definecolor{currentstroke}{rgb}{0.000000,0.000000,0.000000}%
\pgfsetstrokecolor{currentstroke}%
\pgfsetstrokeopacity{0.700000}%
\pgfsetdash{}{0pt}%
\pgfpathmoveto{\pgfqpoint{3.115945in}{0.550000in}}%
\pgfpathlineto{\pgfqpoint{3.143466in}{0.550000in}}%
\pgfpathlineto{\pgfqpoint{3.143466in}{0.707090in}}%
\pgfpathlineto{\pgfqpoint{3.115945in}{0.707090in}}%
\pgfpathlineto{\pgfqpoint{3.115945in}{0.550000in}}%
\pgfpathclose%
\pgfusepath{fill}%
\end{pgfscope}%
\begin{pgfscope}%
\pgfpathrectangle{\pgfqpoint{1.250000in}{0.550000in}}{\pgfqpoint{7.750000in}{3.850000in}}%
\pgfusepath{clip}%
\pgfsetbuttcap%
\pgfsetmiterjoin%
\definecolor{currentfill}{rgb}{0.501961,0.501961,0.501961}%
\pgfsetfillcolor{currentfill}%
\pgfsetfillopacity{0.700000}%
\pgfsetlinewidth{0.000000pt}%
\definecolor{currentstroke}{rgb}{0.000000,0.000000,0.000000}%
\pgfsetstrokecolor{currentstroke}%
\pgfsetstrokeopacity{0.700000}%
\pgfsetdash{}{0pt}%
\pgfpathmoveto{\pgfqpoint{3.143466in}{0.550000in}}%
\pgfpathlineto{\pgfqpoint{3.170987in}{0.550000in}}%
\pgfpathlineto{\pgfqpoint{3.170987in}{0.702650in}}%
\pgfpathlineto{\pgfqpoint{3.143466in}{0.702650in}}%
\pgfpathlineto{\pgfqpoint{3.143466in}{0.550000in}}%
\pgfpathclose%
\pgfusepath{fill}%
\end{pgfscope}%
\begin{pgfscope}%
\pgfpathrectangle{\pgfqpoint{1.250000in}{0.550000in}}{\pgfqpoint{7.750000in}{3.850000in}}%
\pgfusepath{clip}%
\pgfsetbuttcap%
\pgfsetmiterjoin%
\definecolor{currentfill}{rgb}{0.501961,0.501961,0.501961}%
\pgfsetfillcolor{currentfill}%
\pgfsetfillopacity{0.700000}%
\pgfsetlinewidth{0.000000pt}%
\definecolor{currentstroke}{rgb}{0.000000,0.000000,0.000000}%
\pgfsetstrokecolor{currentstroke}%
\pgfsetstrokeopacity{0.700000}%
\pgfsetdash{}{0pt}%
\pgfpathmoveto{\pgfqpoint{3.170987in}{0.550000in}}%
\pgfpathlineto{\pgfqpoint{3.198509in}{0.550000in}}%
\pgfpathlineto{\pgfqpoint{3.198509in}{0.733731in}}%
\pgfpathlineto{\pgfqpoint{3.170987in}{0.733731in}}%
\pgfpathlineto{\pgfqpoint{3.170987in}{0.550000in}}%
\pgfpathclose%
\pgfusepath{fill}%
\end{pgfscope}%
\begin{pgfscope}%
\pgfpathrectangle{\pgfqpoint{1.250000in}{0.550000in}}{\pgfqpoint{7.750000in}{3.850000in}}%
\pgfusepath{clip}%
\pgfsetbuttcap%
\pgfsetmiterjoin%
\definecolor{currentfill}{rgb}{0.501961,0.501961,0.501961}%
\pgfsetfillcolor{currentfill}%
\pgfsetfillopacity{0.700000}%
\pgfsetlinewidth{0.000000pt}%
\definecolor{currentstroke}{rgb}{0.000000,0.000000,0.000000}%
\pgfsetstrokecolor{currentstroke}%
\pgfsetstrokeopacity{0.700000}%
\pgfsetdash{}{0pt}%
\pgfpathmoveto{\pgfqpoint{3.198509in}{0.550000in}}%
\pgfpathlineto{\pgfqpoint{3.226030in}{0.550000in}}%
\pgfpathlineto{\pgfqpoint{3.226030in}{0.702191in}}%
\pgfpathlineto{\pgfqpoint{3.198509in}{0.702191in}}%
\pgfpathlineto{\pgfqpoint{3.198509in}{0.550000in}}%
\pgfpathclose%
\pgfusepath{fill}%
\end{pgfscope}%
\begin{pgfscope}%
\pgfpathrectangle{\pgfqpoint{1.250000in}{0.550000in}}{\pgfqpoint{7.750000in}{3.850000in}}%
\pgfusepath{clip}%
\pgfsetbuttcap%
\pgfsetmiterjoin%
\definecolor{currentfill}{rgb}{0.501961,0.501961,0.501961}%
\pgfsetfillcolor{currentfill}%
\pgfsetfillopacity{0.700000}%
\pgfsetlinewidth{0.000000pt}%
\definecolor{currentstroke}{rgb}{0.000000,0.000000,0.000000}%
\pgfsetstrokecolor{currentstroke}%
\pgfsetstrokeopacity{0.700000}%
\pgfsetdash{}{0pt}%
\pgfpathmoveto{\pgfqpoint{3.226030in}{0.550000in}}%
\pgfpathlineto{\pgfqpoint{3.253551in}{0.550000in}}%
\pgfpathlineto{\pgfqpoint{3.253551in}{0.725923in}}%
\pgfpathlineto{\pgfqpoint{3.226030in}{0.725923in}}%
\pgfpathlineto{\pgfqpoint{3.226030in}{0.550000in}}%
\pgfpathclose%
\pgfusepath{fill}%
\end{pgfscope}%
\begin{pgfscope}%
\pgfpathrectangle{\pgfqpoint{1.250000in}{0.550000in}}{\pgfqpoint{7.750000in}{3.850000in}}%
\pgfusepath{clip}%
\pgfsetbuttcap%
\pgfsetmiterjoin%
\definecolor{currentfill}{rgb}{0.501961,0.501961,0.501961}%
\pgfsetfillcolor{currentfill}%
\pgfsetfillopacity{0.700000}%
\pgfsetlinewidth{0.000000pt}%
\definecolor{currentstroke}{rgb}{0.000000,0.000000,0.000000}%
\pgfsetstrokecolor{currentstroke}%
\pgfsetstrokeopacity{0.700000}%
\pgfsetdash{}{0pt}%
\pgfpathmoveto{\pgfqpoint{3.253551in}{0.550000in}}%
\pgfpathlineto{\pgfqpoint{3.281072in}{0.550000in}}%
\pgfpathlineto{\pgfqpoint{3.281072in}{0.718727in}}%
\pgfpathlineto{\pgfqpoint{3.253551in}{0.718727in}}%
\pgfpathlineto{\pgfqpoint{3.253551in}{0.550000in}}%
\pgfpathclose%
\pgfusepath{fill}%
\end{pgfscope}%
\begin{pgfscope}%
\pgfpathrectangle{\pgfqpoint{1.250000in}{0.550000in}}{\pgfqpoint{7.750000in}{3.850000in}}%
\pgfusepath{clip}%
\pgfsetbuttcap%
\pgfsetmiterjoin%
\definecolor{currentfill}{rgb}{0.501961,0.501961,0.501961}%
\pgfsetfillcolor{currentfill}%
\pgfsetfillopacity{0.700000}%
\pgfsetlinewidth{0.000000pt}%
\definecolor{currentstroke}{rgb}{0.000000,0.000000,0.000000}%
\pgfsetstrokecolor{currentstroke}%
\pgfsetstrokeopacity{0.700000}%
\pgfsetdash{}{0pt}%
\pgfpathmoveto{\pgfqpoint{3.281072in}{0.550000in}}%
\pgfpathlineto{\pgfqpoint{3.308594in}{0.550000in}}%
\pgfpathlineto{\pgfqpoint{3.308594in}{0.711531in}}%
\pgfpathlineto{\pgfqpoint{3.281072in}{0.711531in}}%
\pgfpathlineto{\pgfqpoint{3.281072in}{0.550000in}}%
\pgfpathclose%
\pgfusepath{fill}%
\end{pgfscope}%
\begin{pgfscope}%
\pgfpathrectangle{\pgfqpoint{1.250000in}{0.550000in}}{\pgfqpoint{7.750000in}{3.850000in}}%
\pgfusepath{clip}%
\pgfsetbuttcap%
\pgfsetmiterjoin%
\definecolor{currentfill}{rgb}{0.501961,0.501961,0.501961}%
\pgfsetfillcolor{currentfill}%
\pgfsetfillopacity{0.700000}%
\pgfsetlinewidth{0.000000pt}%
\definecolor{currentstroke}{rgb}{0.000000,0.000000,0.000000}%
\pgfsetstrokecolor{currentstroke}%
\pgfsetstrokeopacity{0.700000}%
\pgfsetdash{}{0pt}%
\pgfpathmoveto{\pgfqpoint{3.308594in}{0.550000in}}%
\pgfpathlineto{\pgfqpoint{3.336115in}{0.550000in}}%
\pgfpathlineto{\pgfqpoint{3.336115in}{0.681674in}}%
\pgfpathlineto{\pgfqpoint{3.308594in}{0.681674in}}%
\pgfpathlineto{\pgfqpoint{3.308594in}{0.550000in}}%
\pgfpathclose%
\pgfusepath{fill}%
\end{pgfscope}%
\begin{pgfscope}%
\pgfpathrectangle{\pgfqpoint{1.250000in}{0.550000in}}{\pgfqpoint{7.750000in}{3.850000in}}%
\pgfusepath{clip}%
\pgfsetbuttcap%
\pgfsetmiterjoin%
\definecolor{currentfill}{rgb}{0.501961,0.501961,0.501961}%
\pgfsetfillcolor{currentfill}%
\pgfsetfillopacity{0.700000}%
\pgfsetlinewidth{0.000000pt}%
\definecolor{currentstroke}{rgb}{0.000000,0.000000,0.000000}%
\pgfsetstrokecolor{currentstroke}%
\pgfsetstrokeopacity{0.700000}%
\pgfsetdash{}{0pt}%
\pgfpathmoveto{\pgfqpoint{3.336115in}{0.550000in}}%
\pgfpathlineto{\pgfqpoint{3.363636in}{0.550000in}}%
\pgfpathlineto{\pgfqpoint{3.363636in}{0.704641in}}%
\pgfpathlineto{\pgfqpoint{3.336115in}{0.704641in}}%
\pgfpathlineto{\pgfqpoint{3.336115in}{0.550000in}}%
\pgfpathclose%
\pgfusepath{fill}%
\end{pgfscope}%
\begin{pgfscope}%
\pgfpathrectangle{\pgfqpoint{1.250000in}{0.550000in}}{\pgfqpoint{7.750000in}{3.850000in}}%
\pgfusepath{clip}%
\pgfsetbuttcap%
\pgfsetmiterjoin%
\definecolor{currentfill}{rgb}{0.501961,0.501961,0.501961}%
\pgfsetfillcolor{currentfill}%
\pgfsetfillopacity{0.700000}%
\pgfsetlinewidth{0.000000pt}%
\definecolor{currentstroke}{rgb}{0.000000,0.000000,0.000000}%
\pgfsetstrokecolor{currentstroke}%
\pgfsetstrokeopacity{0.700000}%
\pgfsetdash{}{0pt}%
\pgfpathmoveto{\pgfqpoint{3.363636in}{0.550000in}}%
\pgfpathlineto{\pgfqpoint{3.391158in}{0.550000in}}%
\pgfpathlineto{\pgfqpoint{3.391158in}{0.689483in}}%
\pgfpathlineto{\pgfqpoint{3.363636in}{0.689483in}}%
\pgfpathlineto{\pgfqpoint{3.363636in}{0.550000in}}%
\pgfpathclose%
\pgfusepath{fill}%
\end{pgfscope}%
\begin{pgfscope}%
\pgfpathrectangle{\pgfqpoint{1.250000in}{0.550000in}}{\pgfqpoint{7.750000in}{3.850000in}}%
\pgfusepath{clip}%
\pgfsetbuttcap%
\pgfsetmiterjoin%
\definecolor{currentfill}{rgb}{0.501961,0.501961,0.501961}%
\pgfsetfillcolor{currentfill}%
\pgfsetfillopacity{0.700000}%
\pgfsetlinewidth{0.000000pt}%
\definecolor{currentstroke}{rgb}{0.000000,0.000000,0.000000}%
\pgfsetstrokecolor{currentstroke}%
\pgfsetstrokeopacity{0.700000}%
\pgfsetdash{}{0pt}%
\pgfpathmoveto{\pgfqpoint{3.391158in}{0.550000in}}%
\pgfpathlineto{\pgfqpoint{3.418679in}{0.550000in}}%
\pgfpathlineto{\pgfqpoint{3.418679in}{0.674937in}}%
\pgfpathlineto{\pgfqpoint{3.391158in}{0.674937in}}%
\pgfpathlineto{\pgfqpoint{3.391158in}{0.550000in}}%
\pgfpathclose%
\pgfusepath{fill}%
\end{pgfscope}%
\begin{pgfscope}%
\pgfpathrectangle{\pgfqpoint{1.250000in}{0.550000in}}{\pgfqpoint{7.750000in}{3.850000in}}%
\pgfusepath{clip}%
\pgfsetbuttcap%
\pgfsetmiterjoin%
\definecolor{currentfill}{rgb}{0.501961,0.501961,0.501961}%
\pgfsetfillcolor{currentfill}%
\pgfsetfillopacity{0.700000}%
\pgfsetlinewidth{0.000000pt}%
\definecolor{currentstroke}{rgb}{0.000000,0.000000,0.000000}%
\pgfsetstrokecolor{currentstroke}%
\pgfsetstrokeopacity{0.700000}%
\pgfsetdash{}{0pt}%
\pgfpathmoveto{\pgfqpoint{3.418679in}{0.550000in}}%
\pgfpathlineto{\pgfqpoint{3.446200in}{0.550000in}}%
\pgfpathlineto{\pgfqpoint{3.446200in}{0.669272in}}%
\pgfpathlineto{\pgfqpoint{3.418679in}{0.669272in}}%
\pgfpathlineto{\pgfqpoint{3.418679in}{0.550000in}}%
\pgfpathclose%
\pgfusepath{fill}%
\end{pgfscope}%
\begin{pgfscope}%
\pgfpathrectangle{\pgfqpoint{1.250000in}{0.550000in}}{\pgfqpoint{7.750000in}{3.850000in}}%
\pgfusepath{clip}%
\pgfsetbuttcap%
\pgfsetmiterjoin%
\definecolor{currentfill}{rgb}{0.501961,0.501961,0.501961}%
\pgfsetfillcolor{currentfill}%
\pgfsetfillopacity{0.700000}%
\pgfsetlinewidth{0.000000pt}%
\definecolor{currentstroke}{rgb}{0.000000,0.000000,0.000000}%
\pgfsetstrokecolor{currentstroke}%
\pgfsetstrokeopacity{0.700000}%
\pgfsetdash{}{0pt}%
\pgfpathmoveto{\pgfqpoint{3.446200in}{0.550000in}}%
\pgfpathlineto{\pgfqpoint{3.473722in}{0.550000in}}%
\pgfpathlineto{\pgfqpoint{3.473722in}{0.671110in}}%
\pgfpathlineto{\pgfqpoint{3.446200in}{0.671110in}}%
\pgfpathlineto{\pgfqpoint{3.446200in}{0.550000in}}%
\pgfpathclose%
\pgfusepath{fill}%
\end{pgfscope}%
\begin{pgfscope}%
\pgfpathrectangle{\pgfqpoint{1.250000in}{0.550000in}}{\pgfqpoint{7.750000in}{3.850000in}}%
\pgfusepath{clip}%
\pgfsetbuttcap%
\pgfsetmiterjoin%
\definecolor{currentfill}{rgb}{0.501961,0.501961,0.501961}%
\pgfsetfillcolor{currentfill}%
\pgfsetfillopacity{0.700000}%
\pgfsetlinewidth{0.000000pt}%
\definecolor{currentstroke}{rgb}{0.000000,0.000000,0.000000}%
\pgfsetstrokecolor{currentstroke}%
\pgfsetstrokeopacity{0.700000}%
\pgfsetdash{}{0pt}%
\pgfpathmoveto{\pgfqpoint{3.473722in}{0.550000in}}%
\pgfpathlineto{\pgfqpoint{3.501243in}{0.550000in}}%
\pgfpathlineto{\pgfqpoint{3.501243in}{0.673100in}}%
\pgfpathlineto{\pgfqpoint{3.473722in}{0.673100in}}%
\pgfpathlineto{\pgfqpoint{3.473722in}{0.550000in}}%
\pgfpathclose%
\pgfusepath{fill}%
\end{pgfscope}%
\begin{pgfscope}%
\pgfpathrectangle{\pgfqpoint{1.250000in}{0.550000in}}{\pgfqpoint{7.750000in}{3.850000in}}%
\pgfusepath{clip}%
\pgfsetbuttcap%
\pgfsetmiterjoin%
\definecolor{currentfill}{rgb}{0.501961,0.501961,0.501961}%
\pgfsetfillcolor{currentfill}%
\pgfsetfillopacity{0.700000}%
\pgfsetlinewidth{0.000000pt}%
\definecolor{currentstroke}{rgb}{0.000000,0.000000,0.000000}%
\pgfsetstrokecolor{currentstroke}%
\pgfsetstrokeopacity{0.700000}%
\pgfsetdash{}{0pt}%
\pgfpathmoveto{\pgfqpoint{3.501243in}{0.550000in}}%
\pgfpathlineto{\pgfqpoint{3.528764in}{0.550000in}}%
\pgfpathlineto{\pgfqpoint{3.528764in}{0.683358in}}%
\pgfpathlineto{\pgfqpoint{3.501243in}{0.683358in}}%
\pgfpathlineto{\pgfqpoint{3.501243in}{0.550000in}}%
\pgfpathclose%
\pgfusepath{fill}%
\end{pgfscope}%
\begin{pgfscope}%
\pgfpathrectangle{\pgfqpoint{1.250000in}{0.550000in}}{\pgfqpoint{7.750000in}{3.850000in}}%
\pgfusepath{clip}%
\pgfsetbuttcap%
\pgfsetmiterjoin%
\definecolor{currentfill}{rgb}{0.501961,0.501961,0.501961}%
\pgfsetfillcolor{currentfill}%
\pgfsetfillopacity{0.700000}%
\pgfsetlinewidth{0.000000pt}%
\definecolor{currentstroke}{rgb}{0.000000,0.000000,0.000000}%
\pgfsetstrokecolor{currentstroke}%
\pgfsetstrokeopacity{0.700000}%
\pgfsetdash{}{0pt}%
\pgfpathmoveto{\pgfqpoint{3.528764in}{0.550000in}}%
\pgfpathlineto{\pgfqpoint{3.556286in}{0.550000in}}%
\pgfpathlineto{\pgfqpoint{3.556286in}{0.668201in}}%
\pgfpathlineto{\pgfqpoint{3.528764in}{0.668201in}}%
\pgfpathlineto{\pgfqpoint{3.528764in}{0.550000in}}%
\pgfpathclose%
\pgfusepath{fill}%
\end{pgfscope}%
\begin{pgfscope}%
\pgfpathrectangle{\pgfqpoint{1.250000in}{0.550000in}}{\pgfqpoint{7.750000in}{3.850000in}}%
\pgfusepath{clip}%
\pgfsetbuttcap%
\pgfsetmiterjoin%
\definecolor{currentfill}{rgb}{0.501961,0.501961,0.501961}%
\pgfsetfillcolor{currentfill}%
\pgfsetfillopacity{0.700000}%
\pgfsetlinewidth{0.000000pt}%
\definecolor{currentstroke}{rgb}{0.000000,0.000000,0.000000}%
\pgfsetstrokecolor{currentstroke}%
\pgfsetstrokeopacity{0.700000}%
\pgfsetdash{}{0pt}%
\pgfpathmoveto{\pgfqpoint{3.556286in}{0.550000in}}%
\pgfpathlineto{\pgfqpoint{3.583807in}{0.550000in}}%
\pgfpathlineto{\pgfqpoint{3.583807in}{0.653655in}}%
\pgfpathlineto{\pgfqpoint{3.556286in}{0.653655in}}%
\pgfpathlineto{\pgfqpoint{3.556286in}{0.550000in}}%
\pgfpathclose%
\pgfusepath{fill}%
\end{pgfscope}%
\begin{pgfscope}%
\pgfpathrectangle{\pgfqpoint{1.250000in}{0.550000in}}{\pgfqpoint{7.750000in}{3.850000in}}%
\pgfusepath{clip}%
\pgfsetbuttcap%
\pgfsetmiterjoin%
\definecolor{currentfill}{rgb}{0.501961,0.501961,0.501961}%
\pgfsetfillcolor{currentfill}%
\pgfsetfillopacity{0.700000}%
\pgfsetlinewidth{0.000000pt}%
\definecolor{currentstroke}{rgb}{0.000000,0.000000,0.000000}%
\pgfsetstrokecolor{currentstroke}%
\pgfsetstrokeopacity{0.700000}%
\pgfsetdash{}{0pt}%
\pgfpathmoveto{\pgfqpoint{3.583807in}{0.550000in}}%
\pgfpathlineto{\pgfqpoint{3.611328in}{0.550000in}}%
\pgfpathlineto{\pgfqpoint{3.611328in}{0.664067in}}%
\pgfpathlineto{\pgfqpoint{3.583807in}{0.664067in}}%
\pgfpathlineto{\pgfqpoint{3.583807in}{0.550000in}}%
\pgfpathclose%
\pgfusepath{fill}%
\end{pgfscope}%
\begin{pgfscope}%
\pgfpathrectangle{\pgfqpoint{1.250000in}{0.550000in}}{\pgfqpoint{7.750000in}{3.850000in}}%
\pgfusepath{clip}%
\pgfsetbuttcap%
\pgfsetmiterjoin%
\definecolor{currentfill}{rgb}{0.501961,0.501961,0.501961}%
\pgfsetfillcolor{currentfill}%
\pgfsetfillopacity{0.700000}%
\pgfsetlinewidth{0.000000pt}%
\definecolor{currentstroke}{rgb}{0.000000,0.000000,0.000000}%
\pgfsetstrokecolor{currentstroke}%
\pgfsetstrokeopacity{0.700000}%
\pgfsetdash{}{0pt}%
\pgfpathmoveto{\pgfqpoint{3.611328in}{0.550000in}}%
\pgfpathlineto{\pgfqpoint{3.638849in}{0.550000in}}%
\pgfpathlineto{\pgfqpoint{3.638849in}{0.652124in}}%
\pgfpathlineto{\pgfqpoint{3.611328in}{0.652124in}}%
\pgfpathlineto{\pgfqpoint{3.611328in}{0.550000in}}%
\pgfpathclose%
\pgfusepath{fill}%
\end{pgfscope}%
\begin{pgfscope}%
\pgfpathrectangle{\pgfqpoint{1.250000in}{0.550000in}}{\pgfqpoint{7.750000in}{3.850000in}}%
\pgfusepath{clip}%
\pgfsetbuttcap%
\pgfsetmiterjoin%
\definecolor{currentfill}{rgb}{0.501961,0.501961,0.501961}%
\pgfsetfillcolor{currentfill}%
\pgfsetfillopacity{0.700000}%
\pgfsetlinewidth{0.000000pt}%
\definecolor{currentstroke}{rgb}{0.000000,0.000000,0.000000}%
\pgfsetstrokecolor{currentstroke}%
\pgfsetstrokeopacity{0.700000}%
\pgfsetdash{}{0pt}%
\pgfpathmoveto{\pgfqpoint{3.638849in}{0.550000in}}%
\pgfpathlineto{\pgfqpoint{3.666371in}{0.550000in}}%
\pgfpathlineto{\pgfqpoint{3.666371in}{0.657942in}}%
\pgfpathlineto{\pgfqpoint{3.638849in}{0.657942in}}%
\pgfpathlineto{\pgfqpoint{3.638849in}{0.550000in}}%
\pgfpathclose%
\pgfusepath{fill}%
\end{pgfscope}%
\begin{pgfscope}%
\pgfpathrectangle{\pgfqpoint{1.250000in}{0.550000in}}{\pgfqpoint{7.750000in}{3.850000in}}%
\pgfusepath{clip}%
\pgfsetbuttcap%
\pgfsetmiterjoin%
\definecolor{currentfill}{rgb}{0.501961,0.501961,0.501961}%
\pgfsetfillcolor{currentfill}%
\pgfsetfillopacity{0.700000}%
\pgfsetlinewidth{0.000000pt}%
\definecolor{currentstroke}{rgb}{0.000000,0.000000,0.000000}%
\pgfsetstrokecolor{currentstroke}%
\pgfsetstrokeopacity{0.700000}%
\pgfsetdash{}{0pt}%
\pgfpathmoveto{\pgfqpoint{3.666371in}{0.550000in}}%
\pgfpathlineto{\pgfqpoint{3.693892in}{0.550000in}}%
\pgfpathlineto{\pgfqpoint{3.693892in}{0.644009in}}%
\pgfpathlineto{\pgfqpoint{3.666371in}{0.644009in}}%
\pgfpathlineto{\pgfqpoint{3.666371in}{0.550000in}}%
\pgfpathclose%
\pgfusepath{fill}%
\end{pgfscope}%
\begin{pgfscope}%
\pgfpathrectangle{\pgfqpoint{1.250000in}{0.550000in}}{\pgfqpoint{7.750000in}{3.850000in}}%
\pgfusepath{clip}%
\pgfsetbuttcap%
\pgfsetmiterjoin%
\definecolor{currentfill}{rgb}{0.501961,0.501961,0.501961}%
\pgfsetfillcolor{currentfill}%
\pgfsetfillopacity{0.700000}%
\pgfsetlinewidth{0.000000pt}%
\definecolor{currentstroke}{rgb}{0.000000,0.000000,0.000000}%
\pgfsetstrokecolor{currentstroke}%
\pgfsetstrokeopacity{0.700000}%
\pgfsetdash{}{0pt}%
\pgfpathmoveto{\pgfqpoint{3.693892in}{0.550000in}}%
\pgfpathlineto{\pgfqpoint{3.721413in}{0.550000in}}%
\pgfpathlineto{\pgfqpoint{3.721413in}{0.650593in}}%
\pgfpathlineto{\pgfqpoint{3.693892in}{0.650593in}}%
\pgfpathlineto{\pgfqpoint{3.693892in}{0.550000in}}%
\pgfpathclose%
\pgfusepath{fill}%
\end{pgfscope}%
\begin{pgfscope}%
\pgfpathrectangle{\pgfqpoint{1.250000in}{0.550000in}}{\pgfqpoint{7.750000in}{3.850000in}}%
\pgfusepath{clip}%
\pgfsetbuttcap%
\pgfsetmiterjoin%
\definecolor{currentfill}{rgb}{0.501961,0.501961,0.501961}%
\pgfsetfillcolor{currentfill}%
\pgfsetfillopacity{0.700000}%
\pgfsetlinewidth{0.000000pt}%
\definecolor{currentstroke}{rgb}{0.000000,0.000000,0.000000}%
\pgfsetstrokecolor{currentstroke}%
\pgfsetstrokeopacity{0.700000}%
\pgfsetdash{}{0pt}%
\pgfpathmoveto{\pgfqpoint{3.721413in}{0.550000in}}%
\pgfpathlineto{\pgfqpoint{3.748935in}{0.550000in}}%
\pgfpathlineto{\pgfqpoint{3.748935in}{0.640641in}}%
\pgfpathlineto{\pgfqpoint{3.721413in}{0.640641in}}%
\pgfpathlineto{\pgfqpoint{3.721413in}{0.550000in}}%
\pgfpathclose%
\pgfusepath{fill}%
\end{pgfscope}%
\begin{pgfscope}%
\pgfpathrectangle{\pgfqpoint{1.250000in}{0.550000in}}{\pgfqpoint{7.750000in}{3.850000in}}%
\pgfusepath{clip}%
\pgfsetbuttcap%
\pgfsetmiterjoin%
\definecolor{currentfill}{rgb}{0.501961,0.501961,0.501961}%
\pgfsetfillcolor{currentfill}%
\pgfsetfillopacity{0.700000}%
\pgfsetlinewidth{0.000000pt}%
\definecolor{currentstroke}{rgb}{0.000000,0.000000,0.000000}%
\pgfsetstrokecolor{currentstroke}%
\pgfsetstrokeopacity{0.700000}%
\pgfsetdash{}{0pt}%
\pgfpathmoveto{\pgfqpoint{3.748935in}{0.550000in}}%
\pgfpathlineto{\pgfqpoint{3.776456in}{0.550000in}}%
\pgfpathlineto{\pgfqpoint{3.776456in}{0.645387in}}%
\pgfpathlineto{\pgfqpoint{3.748935in}{0.645387in}}%
\pgfpathlineto{\pgfqpoint{3.748935in}{0.550000in}}%
\pgfpathclose%
\pgfusepath{fill}%
\end{pgfscope}%
\begin{pgfscope}%
\pgfpathrectangle{\pgfqpoint{1.250000in}{0.550000in}}{\pgfqpoint{7.750000in}{3.850000in}}%
\pgfusepath{clip}%
\pgfsetbuttcap%
\pgfsetmiterjoin%
\definecolor{currentfill}{rgb}{0.501961,0.501961,0.501961}%
\pgfsetfillcolor{currentfill}%
\pgfsetfillopacity{0.700000}%
\pgfsetlinewidth{0.000000pt}%
\definecolor{currentstroke}{rgb}{0.000000,0.000000,0.000000}%
\pgfsetstrokecolor{currentstroke}%
\pgfsetstrokeopacity{0.700000}%
\pgfsetdash{}{0pt}%
\pgfpathmoveto{\pgfqpoint{3.776456in}{0.550000in}}%
\pgfpathlineto{\pgfqpoint{3.803977in}{0.550000in}}%
\pgfpathlineto{\pgfqpoint{3.803977in}{0.632373in}}%
\pgfpathlineto{\pgfqpoint{3.776456in}{0.632373in}}%
\pgfpathlineto{\pgfqpoint{3.776456in}{0.550000in}}%
\pgfpathclose%
\pgfusepath{fill}%
\end{pgfscope}%
\begin{pgfscope}%
\pgfpathrectangle{\pgfqpoint{1.250000in}{0.550000in}}{\pgfqpoint{7.750000in}{3.850000in}}%
\pgfusepath{clip}%
\pgfsetbuttcap%
\pgfsetmiterjoin%
\definecolor{currentfill}{rgb}{0.501961,0.501961,0.501961}%
\pgfsetfillcolor{currentfill}%
\pgfsetfillopacity{0.700000}%
\pgfsetlinewidth{0.000000pt}%
\definecolor{currentstroke}{rgb}{0.000000,0.000000,0.000000}%
\pgfsetstrokecolor{currentstroke}%
\pgfsetstrokeopacity{0.700000}%
\pgfsetdash{}{0pt}%
\pgfpathmoveto{\pgfqpoint{3.803977in}{0.550000in}}%
\pgfpathlineto{\pgfqpoint{3.831499in}{0.550000in}}%
\pgfpathlineto{\pgfqpoint{3.831499in}{0.642019in}}%
\pgfpathlineto{\pgfqpoint{3.803977in}{0.642019in}}%
\pgfpathlineto{\pgfqpoint{3.803977in}{0.550000in}}%
\pgfpathclose%
\pgfusepath{fill}%
\end{pgfscope}%
\begin{pgfscope}%
\pgfpathrectangle{\pgfqpoint{1.250000in}{0.550000in}}{\pgfqpoint{7.750000in}{3.850000in}}%
\pgfusepath{clip}%
\pgfsetbuttcap%
\pgfsetmiterjoin%
\definecolor{currentfill}{rgb}{0.501961,0.501961,0.501961}%
\pgfsetfillcolor{currentfill}%
\pgfsetfillopacity{0.700000}%
\pgfsetlinewidth{0.000000pt}%
\definecolor{currentstroke}{rgb}{0.000000,0.000000,0.000000}%
\pgfsetstrokecolor{currentstroke}%
\pgfsetstrokeopacity{0.700000}%
\pgfsetdash{}{0pt}%
\pgfpathmoveto{\pgfqpoint{3.831499in}{0.550000in}}%
\pgfpathlineto{\pgfqpoint{3.859020in}{0.550000in}}%
\pgfpathlineto{\pgfqpoint{3.859020in}{0.634976in}}%
\pgfpathlineto{\pgfqpoint{3.831499in}{0.634976in}}%
\pgfpathlineto{\pgfqpoint{3.831499in}{0.550000in}}%
\pgfpathclose%
\pgfusepath{fill}%
\end{pgfscope}%
\begin{pgfscope}%
\pgfpathrectangle{\pgfqpoint{1.250000in}{0.550000in}}{\pgfqpoint{7.750000in}{3.850000in}}%
\pgfusepath{clip}%
\pgfsetbuttcap%
\pgfsetmiterjoin%
\definecolor{currentfill}{rgb}{0.501961,0.501961,0.501961}%
\pgfsetfillcolor{currentfill}%
\pgfsetfillopacity{0.700000}%
\pgfsetlinewidth{0.000000pt}%
\definecolor{currentstroke}{rgb}{0.000000,0.000000,0.000000}%
\pgfsetstrokecolor{currentstroke}%
\pgfsetstrokeopacity{0.700000}%
\pgfsetdash{}{0pt}%
\pgfpathmoveto{\pgfqpoint{3.859020in}{0.550000in}}%
\pgfpathlineto{\pgfqpoint{3.886541in}{0.550000in}}%
\pgfpathlineto{\pgfqpoint{3.886541in}{0.641559in}}%
\pgfpathlineto{\pgfqpoint{3.859020in}{0.641559in}}%
\pgfpathlineto{\pgfqpoint{3.859020in}{0.550000in}}%
\pgfpathclose%
\pgfusepath{fill}%
\end{pgfscope}%
\begin{pgfscope}%
\pgfpathrectangle{\pgfqpoint{1.250000in}{0.550000in}}{\pgfqpoint{7.750000in}{3.850000in}}%
\pgfusepath{clip}%
\pgfsetbuttcap%
\pgfsetmiterjoin%
\definecolor{currentfill}{rgb}{0.501961,0.501961,0.501961}%
\pgfsetfillcolor{currentfill}%
\pgfsetfillopacity{0.700000}%
\pgfsetlinewidth{0.000000pt}%
\definecolor{currentstroke}{rgb}{0.000000,0.000000,0.000000}%
\pgfsetstrokecolor{currentstroke}%
\pgfsetstrokeopacity{0.700000}%
\pgfsetdash{}{0pt}%
\pgfpathmoveto{\pgfqpoint{3.886541in}{0.550000in}}%
\pgfpathlineto{\pgfqpoint{3.914062in}{0.550000in}}%
\pgfpathlineto{\pgfqpoint{3.914062in}{0.634363in}}%
\pgfpathlineto{\pgfqpoint{3.886541in}{0.634363in}}%
\pgfpathlineto{\pgfqpoint{3.886541in}{0.550000in}}%
\pgfpathclose%
\pgfusepath{fill}%
\end{pgfscope}%
\begin{pgfscope}%
\pgfpathrectangle{\pgfqpoint{1.250000in}{0.550000in}}{\pgfqpoint{7.750000in}{3.850000in}}%
\pgfusepath{clip}%
\pgfsetbuttcap%
\pgfsetmiterjoin%
\definecolor{currentfill}{rgb}{0.501961,0.501961,0.501961}%
\pgfsetfillcolor{currentfill}%
\pgfsetfillopacity{0.700000}%
\pgfsetlinewidth{0.000000pt}%
\definecolor{currentstroke}{rgb}{0.000000,0.000000,0.000000}%
\pgfsetstrokecolor{currentstroke}%
\pgfsetstrokeopacity{0.700000}%
\pgfsetdash{}{0pt}%
\pgfpathmoveto{\pgfqpoint{3.914062in}{0.550000in}}%
\pgfpathlineto{\pgfqpoint{3.941584in}{0.550000in}}%
\pgfpathlineto{\pgfqpoint{3.941584in}{0.632985in}}%
\pgfpathlineto{\pgfqpoint{3.914062in}{0.632985in}}%
\pgfpathlineto{\pgfqpoint{3.914062in}{0.550000in}}%
\pgfpathclose%
\pgfusepath{fill}%
\end{pgfscope}%
\begin{pgfscope}%
\pgfpathrectangle{\pgfqpoint{1.250000in}{0.550000in}}{\pgfqpoint{7.750000in}{3.850000in}}%
\pgfusepath{clip}%
\pgfsetbuttcap%
\pgfsetmiterjoin%
\definecolor{currentfill}{rgb}{0.501961,0.501961,0.501961}%
\pgfsetfillcolor{currentfill}%
\pgfsetfillopacity{0.700000}%
\pgfsetlinewidth{0.000000pt}%
\definecolor{currentstroke}{rgb}{0.000000,0.000000,0.000000}%
\pgfsetstrokecolor{currentstroke}%
\pgfsetstrokeopacity{0.700000}%
\pgfsetdash{}{0pt}%
\pgfpathmoveto{\pgfqpoint{3.941584in}{0.550000in}}%
\pgfpathlineto{\pgfqpoint{3.969105in}{0.550000in}}%
\pgfpathlineto{\pgfqpoint{3.969105in}{0.628545in}}%
\pgfpathlineto{\pgfqpoint{3.941584in}{0.628545in}}%
\pgfpathlineto{\pgfqpoint{3.941584in}{0.550000in}}%
\pgfpathclose%
\pgfusepath{fill}%
\end{pgfscope}%
\begin{pgfscope}%
\pgfpathrectangle{\pgfqpoint{1.250000in}{0.550000in}}{\pgfqpoint{7.750000in}{3.850000in}}%
\pgfusepath{clip}%
\pgfsetbuttcap%
\pgfsetmiterjoin%
\definecolor{currentfill}{rgb}{0.501961,0.501961,0.501961}%
\pgfsetfillcolor{currentfill}%
\pgfsetfillopacity{0.700000}%
\pgfsetlinewidth{0.000000pt}%
\definecolor{currentstroke}{rgb}{0.000000,0.000000,0.000000}%
\pgfsetstrokecolor{currentstroke}%
\pgfsetstrokeopacity{0.700000}%
\pgfsetdash{}{0pt}%
\pgfpathmoveto{\pgfqpoint{3.969105in}{0.550000in}}%
\pgfpathlineto{\pgfqpoint{3.996626in}{0.550000in}}%
\pgfpathlineto{\pgfqpoint{3.996626in}{0.637426in}}%
\pgfpathlineto{\pgfqpoint{3.969105in}{0.637426in}}%
\pgfpathlineto{\pgfqpoint{3.969105in}{0.550000in}}%
\pgfpathclose%
\pgfusepath{fill}%
\end{pgfscope}%
\begin{pgfscope}%
\pgfpathrectangle{\pgfqpoint{1.250000in}{0.550000in}}{\pgfqpoint{7.750000in}{3.850000in}}%
\pgfusepath{clip}%
\pgfsetbuttcap%
\pgfsetmiterjoin%
\definecolor{currentfill}{rgb}{0.501961,0.501961,0.501961}%
\pgfsetfillcolor{currentfill}%
\pgfsetfillopacity{0.700000}%
\pgfsetlinewidth{0.000000pt}%
\definecolor{currentstroke}{rgb}{0.000000,0.000000,0.000000}%
\pgfsetstrokecolor{currentstroke}%
\pgfsetstrokeopacity{0.700000}%
\pgfsetdash{}{0pt}%
\pgfpathmoveto{\pgfqpoint{3.996626in}{0.550000in}}%
\pgfpathlineto{\pgfqpoint{4.024148in}{0.550000in}}%
\pgfpathlineto{\pgfqpoint{4.024148in}{0.624717in}}%
\pgfpathlineto{\pgfqpoint{3.996626in}{0.624717in}}%
\pgfpathlineto{\pgfqpoint{3.996626in}{0.550000in}}%
\pgfpathclose%
\pgfusepath{fill}%
\end{pgfscope}%
\begin{pgfscope}%
\pgfpathrectangle{\pgfqpoint{1.250000in}{0.550000in}}{\pgfqpoint{7.750000in}{3.850000in}}%
\pgfusepath{clip}%
\pgfsetbuttcap%
\pgfsetmiterjoin%
\definecolor{currentfill}{rgb}{0.501961,0.501961,0.501961}%
\pgfsetfillcolor{currentfill}%
\pgfsetfillopacity{0.700000}%
\pgfsetlinewidth{0.000000pt}%
\definecolor{currentstroke}{rgb}{0.000000,0.000000,0.000000}%
\pgfsetstrokecolor{currentstroke}%
\pgfsetstrokeopacity{0.700000}%
\pgfsetdash{}{0pt}%
\pgfpathmoveto{\pgfqpoint{4.024148in}{0.550000in}}%
\pgfpathlineto{\pgfqpoint{4.051669in}{0.550000in}}%
\pgfpathlineto{\pgfqpoint{4.051669in}{0.641866in}}%
\pgfpathlineto{\pgfqpoint{4.024148in}{0.641866in}}%
\pgfpathlineto{\pgfqpoint{4.024148in}{0.550000in}}%
\pgfpathclose%
\pgfusepath{fill}%
\end{pgfscope}%
\begin{pgfscope}%
\pgfpathrectangle{\pgfqpoint{1.250000in}{0.550000in}}{\pgfqpoint{7.750000in}{3.850000in}}%
\pgfusepath{clip}%
\pgfsetbuttcap%
\pgfsetmiterjoin%
\definecolor{currentfill}{rgb}{0.501961,0.501961,0.501961}%
\pgfsetfillcolor{currentfill}%
\pgfsetfillopacity{0.700000}%
\pgfsetlinewidth{0.000000pt}%
\definecolor{currentstroke}{rgb}{0.000000,0.000000,0.000000}%
\pgfsetstrokecolor{currentstroke}%
\pgfsetstrokeopacity{0.700000}%
\pgfsetdash{}{0pt}%
\pgfpathmoveto{\pgfqpoint{4.051669in}{0.550000in}}%
\pgfpathlineto{\pgfqpoint{4.079190in}{0.550000in}}%
\pgfpathlineto{\pgfqpoint{4.079190in}{0.629464in}}%
\pgfpathlineto{\pgfqpoint{4.051669in}{0.629464in}}%
\pgfpathlineto{\pgfqpoint{4.051669in}{0.550000in}}%
\pgfpathclose%
\pgfusepath{fill}%
\end{pgfscope}%
\begin{pgfscope}%
\pgfpathrectangle{\pgfqpoint{1.250000in}{0.550000in}}{\pgfqpoint{7.750000in}{3.850000in}}%
\pgfusepath{clip}%
\pgfsetbuttcap%
\pgfsetmiterjoin%
\definecolor{currentfill}{rgb}{0.501961,0.501961,0.501961}%
\pgfsetfillcolor{currentfill}%
\pgfsetfillopacity{0.700000}%
\pgfsetlinewidth{0.000000pt}%
\definecolor{currentstroke}{rgb}{0.000000,0.000000,0.000000}%
\pgfsetstrokecolor{currentstroke}%
\pgfsetstrokeopacity{0.700000}%
\pgfsetdash{}{0pt}%
\pgfpathmoveto{\pgfqpoint{4.079190in}{0.550000in}}%
\pgfpathlineto{\pgfqpoint{4.106712in}{0.550000in}}%
\pgfpathlineto{\pgfqpoint{4.106712in}{0.630382in}}%
\pgfpathlineto{\pgfqpoint{4.079190in}{0.630382in}}%
\pgfpathlineto{\pgfqpoint{4.079190in}{0.550000in}}%
\pgfpathclose%
\pgfusepath{fill}%
\end{pgfscope}%
\begin{pgfscope}%
\pgfpathrectangle{\pgfqpoint{1.250000in}{0.550000in}}{\pgfqpoint{7.750000in}{3.850000in}}%
\pgfusepath{clip}%
\pgfsetbuttcap%
\pgfsetmiterjoin%
\definecolor{currentfill}{rgb}{0.501961,0.501961,0.501961}%
\pgfsetfillcolor{currentfill}%
\pgfsetfillopacity{0.700000}%
\pgfsetlinewidth{0.000000pt}%
\definecolor{currentstroke}{rgb}{0.000000,0.000000,0.000000}%
\pgfsetstrokecolor{currentstroke}%
\pgfsetstrokeopacity{0.700000}%
\pgfsetdash{}{0pt}%
\pgfpathmoveto{\pgfqpoint{4.106712in}{0.550000in}}%
\pgfpathlineto{\pgfqpoint{4.134233in}{0.550000in}}%
\pgfpathlineto{\pgfqpoint{4.134233in}{0.628239in}}%
\pgfpathlineto{\pgfqpoint{4.106712in}{0.628239in}}%
\pgfpathlineto{\pgfqpoint{4.106712in}{0.550000in}}%
\pgfpathclose%
\pgfusepath{fill}%
\end{pgfscope}%
\begin{pgfscope}%
\pgfpathrectangle{\pgfqpoint{1.250000in}{0.550000in}}{\pgfqpoint{7.750000in}{3.850000in}}%
\pgfusepath{clip}%
\pgfsetbuttcap%
\pgfsetmiterjoin%
\definecolor{currentfill}{rgb}{0.501961,0.501961,0.501961}%
\pgfsetfillcolor{currentfill}%
\pgfsetfillopacity{0.700000}%
\pgfsetlinewidth{0.000000pt}%
\definecolor{currentstroke}{rgb}{0.000000,0.000000,0.000000}%
\pgfsetstrokecolor{currentstroke}%
\pgfsetstrokeopacity{0.700000}%
\pgfsetdash{}{0pt}%
\pgfpathmoveto{\pgfqpoint{4.134233in}{0.550000in}}%
\pgfpathlineto{\pgfqpoint{4.161754in}{0.550000in}}%
\pgfpathlineto{\pgfqpoint{4.161754in}{0.632985in}}%
\pgfpathlineto{\pgfqpoint{4.134233in}{0.632985in}}%
\pgfpathlineto{\pgfqpoint{4.134233in}{0.550000in}}%
\pgfpathclose%
\pgfusepath{fill}%
\end{pgfscope}%
\begin{pgfscope}%
\pgfpathrectangle{\pgfqpoint{1.250000in}{0.550000in}}{\pgfqpoint{7.750000in}{3.850000in}}%
\pgfusepath{clip}%
\pgfsetbuttcap%
\pgfsetmiterjoin%
\definecolor{currentfill}{rgb}{0.501961,0.501961,0.501961}%
\pgfsetfillcolor{currentfill}%
\pgfsetfillopacity{0.700000}%
\pgfsetlinewidth{0.000000pt}%
\definecolor{currentstroke}{rgb}{0.000000,0.000000,0.000000}%
\pgfsetstrokecolor{currentstroke}%
\pgfsetstrokeopacity{0.700000}%
\pgfsetdash{}{0pt}%
\pgfpathmoveto{\pgfqpoint{4.161754in}{0.550000in}}%
\pgfpathlineto{\pgfqpoint{4.189276in}{0.550000in}}%
\pgfpathlineto{\pgfqpoint{4.189276in}{0.634516in}}%
\pgfpathlineto{\pgfqpoint{4.161754in}{0.634516in}}%
\pgfpathlineto{\pgfqpoint{4.161754in}{0.550000in}}%
\pgfpathclose%
\pgfusepath{fill}%
\end{pgfscope}%
\begin{pgfscope}%
\pgfpathrectangle{\pgfqpoint{1.250000in}{0.550000in}}{\pgfqpoint{7.750000in}{3.850000in}}%
\pgfusepath{clip}%
\pgfsetbuttcap%
\pgfsetmiterjoin%
\definecolor{currentfill}{rgb}{0.501961,0.501961,0.501961}%
\pgfsetfillcolor{currentfill}%
\pgfsetfillopacity{0.700000}%
\pgfsetlinewidth{0.000000pt}%
\definecolor{currentstroke}{rgb}{0.000000,0.000000,0.000000}%
\pgfsetstrokecolor{currentstroke}%
\pgfsetstrokeopacity{0.700000}%
\pgfsetdash{}{0pt}%
\pgfpathmoveto{\pgfqpoint{4.189276in}{0.550000in}}%
\pgfpathlineto{\pgfqpoint{4.216797in}{0.550000in}}%
\pgfpathlineto{\pgfqpoint{4.216797in}{0.638038in}}%
\pgfpathlineto{\pgfqpoint{4.189276in}{0.638038in}}%
\pgfpathlineto{\pgfqpoint{4.189276in}{0.550000in}}%
\pgfpathclose%
\pgfusepath{fill}%
\end{pgfscope}%
\begin{pgfscope}%
\pgfpathrectangle{\pgfqpoint{1.250000in}{0.550000in}}{\pgfqpoint{7.750000in}{3.850000in}}%
\pgfusepath{clip}%
\pgfsetbuttcap%
\pgfsetmiterjoin%
\definecolor{currentfill}{rgb}{0.501961,0.501961,0.501961}%
\pgfsetfillcolor{currentfill}%
\pgfsetfillopacity{0.700000}%
\pgfsetlinewidth{0.000000pt}%
\definecolor{currentstroke}{rgb}{0.000000,0.000000,0.000000}%
\pgfsetstrokecolor{currentstroke}%
\pgfsetstrokeopacity{0.700000}%
\pgfsetdash{}{0pt}%
\pgfpathmoveto{\pgfqpoint{4.216797in}{0.550000in}}%
\pgfpathlineto{\pgfqpoint{4.244318in}{0.550000in}}%
\pgfpathlineto{\pgfqpoint{4.244318in}{0.629923in}}%
\pgfpathlineto{\pgfqpoint{4.216797in}{0.629923in}}%
\pgfpathlineto{\pgfqpoint{4.216797in}{0.550000in}}%
\pgfpathclose%
\pgfusepath{fill}%
\end{pgfscope}%
\begin{pgfscope}%
\pgfpathrectangle{\pgfqpoint{1.250000in}{0.550000in}}{\pgfqpoint{7.750000in}{3.850000in}}%
\pgfusepath{clip}%
\pgfsetbuttcap%
\pgfsetmiterjoin%
\definecolor{currentfill}{rgb}{0.501961,0.501961,0.501961}%
\pgfsetfillcolor{currentfill}%
\pgfsetfillopacity{0.700000}%
\pgfsetlinewidth{0.000000pt}%
\definecolor{currentstroke}{rgb}{0.000000,0.000000,0.000000}%
\pgfsetstrokecolor{currentstroke}%
\pgfsetstrokeopacity{0.700000}%
\pgfsetdash{}{0pt}%
\pgfpathmoveto{\pgfqpoint{4.244318in}{0.550000in}}%
\pgfpathlineto{\pgfqpoint{4.271839in}{0.550000in}}%
\pgfpathlineto{\pgfqpoint{4.271839in}{0.643397in}}%
\pgfpathlineto{\pgfqpoint{4.244318in}{0.643397in}}%
\pgfpathlineto{\pgfqpoint{4.244318in}{0.550000in}}%
\pgfpathclose%
\pgfusepath{fill}%
\end{pgfscope}%
\begin{pgfscope}%
\pgfpathrectangle{\pgfqpoint{1.250000in}{0.550000in}}{\pgfqpoint{7.750000in}{3.850000in}}%
\pgfusepath{clip}%
\pgfsetbuttcap%
\pgfsetmiterjoin%
\definecolor{currentfill}{rgb}{0.501961,0.501961,0.501961}%
\pgfsetfillcolor{currentfill}%
\pgfsetfillopacity{0.700000}%
\pgfsetlinewidth{0.000000pt}%
\definecolor{currentstroke}{rgb}{0.000000,0.000000,0.000000}%
\pgfsetstrokecolor{currentstroke}%
\pgfsetstrokeopacity{0.700000}%
\pgfsetdash{}{0pt}%
\pgfpathmoveto{\pgfqpoint{4.271839in}{0.550000in}}%
\pgfpathlineto{\pgfqpoint{4.299361in}{0.550000in}}%
\pgfpathlineto{\pgfqpoint{4.299361in}{0.633598in}}%
\pgfpathlineto{\pgfqpoint{4.271839in}{0.633598in}}%
\pgfpathlineto{\pgfqpoint{4.271839in}{0.550000in}}%
\pgfpathclose%
\pgfusepath{fill}%
\end{pgfscope}%
\begin{pgfscope}%
\pgfpathrectangle{\pgfqpoint{1.250000in}{0.550000in}}{\pgfqpoint{7.750000in}{3.850000in}}%
\pgfusepath{clip}%
\pgfsetbuttcap%
\pgfsetmiterjoin%
\definecolor{currentfill}{rgb}{0.501961,0.501961,0.501961}%
\pgfsetfillcolor{currentfill}%
\pgfsetfillopacity{0.700000}%
\pgfsetlinewidth{0.000000pt}%
\definecolor{currentstroke}{rgb}{0.000000,0.000000,0.000000}%
\pgfsetstrokecolor{currentstroke}%
\pgfsetstrokeopacity{0.700000}%
\pgfsetdash{}{0pt}%
\pgfpathmoveto{\pgfqpoint{4.299361in}{0.550000in}}%
\pgfpathlineto{\pgfqpoint{4.326882in}{0.550000in}}%
\pgfpathlineto{\pgfqpoint{4.326882in}{0.627473in}}%
\pgfpathlineto{\pgfqpoint{4.299361in}{0.627473in}}%
\pgfpathlineto{\pgfqpoint{4.299361in}{0.550000in}}%
\pgfpathclose%
\pgfusepath{fill}%
\end{pgfscope}%
\begin{pgfscope}%
\pgfpathrectangle{\pgfqpoint{1.250000in}{0.550000in}}{\pgfqpoint{7.750000in}{3.850000in}}%
\pgfusepath{clip}%
\pgfsetbuttcap%
\pgfsetmiterjoin%
\definecolor{currentfill}{rgb}{0.501961,0.501961,0.501961}%
\pgfsetfillcolor{currentfill}%
\pgfsetfillopacity{0.700000}%
\pgfsetlinewidth{0.000000pt}%
\definecolor{currentstroke}{rgb}{0.000000,0.000000,0.000000}%
\pgfsetstrokecolor{currentstroke}%
\pgfsetstrokeopacity{0.700000}%
\pgfsetdash{}{0pt}%
\pgfpathmoveto{\pgfqpoint{4.326882in}{0.550000in}}%
\pgfpathlineto{\pgfqpoint{4.354403in}{0.550000in}}%
\pgfpathlineto{\pgfqpoint{4.354403in}{0.644622in}}%
\pgfpathlineto{\pgfqpoint{4.326882in}{0.644622in}}%
\pgfpathlineto{\pgfqpoint{4.326882in}{0.550000in}}%
\pgfpathclose%
\pgfusepath{fill}%
\end{pgfscope}%
\begin{pgfscope}%
\pgfpathrectangle{\pgfqpoint{1.250000in}{0.550000in}}{\pgfqpoint{7.750000in}{3.850000in}}%
\pgfusepath{clip}%
\pgfsetbuttcap%
\pgfsetmiterjoin%
\definecolor{currentfill}{rgb}{0.501961,0.501961,0.501961}%
\pgfsetfillcolor{currentfill}%
\pgfsetfillopacity{0.700000}%
\pgfsetlinewidth{0.000000pt}%
\definecolor{currentstroke}{rgb}{0.000000,0.000000,0.000000}%
\pgfsetstrokecolor{currentstroke}%
\pgfsetstrokeopacity{0.700000}%
\pgfsetdash{}{0pt}%
\pgfpathmoveto{\pgfqpoint{4.354403in}{0.550000in}}%
\pgfpathlineto{\pgfqpoint{4.381925in}{0.550000in}}%
\pgfpathlineto{\pgfqpoint{4.381925in}{0.637426in}}%
\pgfpathlineto{\pgfqpoint{4.354403in}{0.637426in}}%
\pgfpathlineto{\pgfqpoint{4.354403in}{0.550000in}}%
\pgfpathclose%
\pgfusepath{fill}%
\end{pgfscope}%
\begin{pgfscope}%
\pgfpathrectangle{\pgfqpoint{1.250000in}{0.550000in}}{\pgfqpoint{7.750000in}{3.850000in}}%
\pgfusepath{clip}%
\pgfsetbuttcap%
\pgfsetmiterjoin%
\definecolor{currentfill}{rgb}{0.501961,0.501961,0.501961}%
\pgfsetfillcolor{currentfill}%
\pgfsetfillopacity{0.700000}%
\pgfsetlinewidth{0.000000pt}%
\definecolor{currentstroke}{rgb}{0.000000,0.000000,0.000000}%
\pgfsetstrokecolor{currentstroke}%
\pgfsetstrokeopacity{0.700000}%
\pgfsetdash{}{0pt}%
\pgfpathmoveto{\pgfqpoint{4.381925in}{0.550000in}}%
\pgfpathlineto{\pgfqpoint{4.409446in}{0.550000in}}%
\pgfpathlineto{\pgfqpoint{4.409446in}{0.628851in}}%
\pgfpathlineto{\pgfqpoint{4.381925in}{0.628851in}}%
\pgfpathlineto{\pgfqpoint{4.381925in}{0.550000in}}%
\pgfpathclose%
\pgfusepath{fill}%
\end{pgfscope}%
\begin{pgfscope}%
\pgfpathrectangle{\pgfqpoint{1.250000in}{0.550000in}}{\pgfqpoint{7.750000in}{3.850000in}}%
\pgfusepath{clip}%
\pgfsetbuttcap%
\pgfsetmiterjoin%
\definecolor{currentfill}{rgb}{0.501961,0.501961,0.501961}%
\pgfsetfillcolor{currentfill}%
\pgfsetfillopacity{0.700000}%
\pgfsetlinewidth{0.000000pt}%
\definecolor{currentstroke}{rgb}{0.000000,0.000000,0.000000}%
\pgfsetstrokecolor{currentstroke}%
\pgfsetstrokeopacity{0.700000}%
\pgfsetdash{}{0pt}%
\pgfpathmoveto{\pgfqpoint{4.409446in}{0.550000in}}%
\pgfpathlineto{\pgfqpoint{4.436967in}{0.550000in}}%
\pgfpathlineto{\pgfqpoint{4.436967in}{0.638957in}}%
\pgfpathlineto{\pgfqpoint{4.409446in}{0.638957in}}%
\pgfpathlineto{\pgfqpoint{4.409446in}{0.550000in}}%
\pgfpathclose%
\pgfusepath{fill}%
\end{pgfscope}%
\begin{pgfscope}%
\pgfpathrectangle{\pgfqpoint{1.250000in}{0.550000in}}{\pgfqpoint{7.750000in}{3.850000in}}%
\pgfusepath{clip}%
\pgfsetbuttcap%
\pgfsetmiterjoin%
\definecolor{currentfill}{rgb}{0.501961,0.501961,0.501961}%
\pgfsetfillcolor{currentfill}%
\pgfsetfillopacity{0.700000}%
\pgfsetlinewidth{0.000000pt}%
\definecolor{currentstroke}{rgb}{0.000000,0.000000,0.000000}%
\pgfsetstrokecolor{currentstroke}%
\pgfsetstrokeopacity{0.700000}%
\pgfsetdash{}{0pt}%
\pgfpathmoveto{\pgfqpoint{4.436967in}{0.550000in}}%
\pgfpathlineto{\pgfqpoint{4.464489in}{0.550000in}}%
\pgfpathlineto{\pgfqpoint{4.464489in}{0.632067in}}%
\pgfpathlineto{\pgfqpoint{4.436967in}{0.632067in}}%
\pgfpathlineto{\pgfqpoint{4.436967in}{0.550000in}}%
\pgfpathclose%
\pgfusepath{fill}%
\end{pgfscope}%
\begin{pgfscope}%
\pgfpathrectangle{\pgfqpoint{1.250000in}{0.550000in}}{\pgfqpoint{7.750000in}{3.850000in}}%
\pgfusepath{clip}%
\pgfsetbuttcap%
\pgfsetmiterjoin%
\definecolor{currentfill}{rgb}{0.501961,0.501961,0.501961}%
\pgfsetfillcolor{currentfill}%
\pgfsetfillopacity{0.700000}%
\pgfsetlinewidth{0.000000pt}%
\definecolor{currentstroke}{rgb}{0.000000,0.000000,0.000000}%
\pgfsetstrokecolor{currentstroke}%
\pgfsetstrokeopacity{0.700000}%
\pgfsetdash{}{0pt}%
\pgfpathmoveto{\pgfqpoint{4.464489in}{0.550000in}}%
\pgfpathlineto{\pgfqpoint{4.492010in}{0.550000in}}%
\pgfpathlineto{\pgfqpoint{4.492010in}{0.622727in}}%
\pgfpathlineto{\pgfqpoint{4.464489in}{0.622727in}}%
\pgfpathlineto{\pgfqpoint{4.464489in}{0.550000in}}%
\pgfpathclose%
\pgfusepath{fill}%
\end{pgfscope}%
\begin{pgfscope}%
\pgfpathrectangle{\pgfqpoint{1.250000in}{0.550000in}}{\pgfqpoint{7.750000in}{3.850000in}}%
\pgfusepath{clip}%
\pgfsetbuttcap%
\pgfsetmiterjoin%
\definecolor{currentfill}{rgb}{0.501961,0.501961,0.501961}%
\pgfsetfillcolor{currentfill}%
\pgfsetfillopacity{0.700000}%
\pgfsetlinewidth{0.000000pt}%
\definecolor{currentstroke}{rgb}{0.000000,0.000000,0.000000}%
\pgfsetstrokecolor{currentstroke}%
\pgfsetstrokeopacity{0.700000}%
\pgfsetdash{}{0pt}%
\pgfpathmoveto{\pgfqpoint{4.492010in}{0.550000in}}%
\pgfpathlineto{\pgfqpoint{4.519531in}{0.550000in}}%
\pgfpathlineto{\pgfqpoint{4.519531in}{0.641713in}}%
\pgfpathlineto{\pgfqpoint{4.492010in}{0.641713in}}%
\pgfpathlineto{\pgfqpoint{4.492010in}{0.550000in}}%
\pgfpathclose%
\pgfusepath{fill}%
\end{pgfscope}%
\begin{pgfscope}%
\pgfpathrectangle{\pgfqpoint{1.250000in}{0.550000in}}{\pgfqpoint{7.750000in}{3.850000in}}%
\pgfusepath{clip}%
\pgfsetbuttcap%
\pgfsetmiterjoin%
\definecolor{currentfill}{rgb}{0.501961,0.501961,0.501961}%
\pgfsetfillcolor{currentfill}%
\pgfsetfillopacity{0.700000}%
\pgfsetlinewidth{0.000000pt}%
\definecolor{currentstroke}{rgb}{0.000000,0.000000,0.000000}%
\pgfsetstrokecolor{currentstroke}%
\pgfsetstrokeopacity{0.700000}%
\pgfsetdash{}{0pt}%
\pgfpathmoveto{\pgfqpoint{4.519531in}{0.550000in}}%
\pgfpathlineto{\pgfqpoint{4.547053in}{0.550000in}}%
\pgfpathlineto{\pgfqpoint{4.547053in}{0.624411in}}%
\pgfpathlineto{\pgfqpoint{4.519531in}{0.624411in}}%
\pgfpathlineto{\pgfqpoint{4.519531in}{0.550000in}}%
\pgfpathclose%
\pgfusepath{fill}%
\end{pgfscope}%
\begin{pgfscope}%
\pgfpathrectangle{\pgfqpoint{1.250000in}{0.550000in}}{\pgfqpoint{7.750000in}{3.850000in}}%
\pgfusepath{clip}%
\pgfsetbuttcap%
\pgfsetmiterjoin%
\definecolor{currentfill}{rgb}{0.501961,0.501961,0.501961}%
\pgfsetfillcolor{currentfill}%
\pgfsetfillopacity{0.700000}%
\pgfsetlinewidth{0.000000pt}%
\definecolor{currentstroke}{rgb}{0.000000,0.000000,0.000000}%
\pgfsetstrokecolor{currentstroke}%
\pgfsetstrokeopacity{0.700000}%
\pgfsetdash{}{0pt}%
\pgfpathmoveto{\pgfqpoint{4.547053in}{0.550000in}}%
\pgfpathlineto{\pgfqpoint{4.574574in}{0.550000in}}%
\pgfpathlineto{\pgfqpoint{4.574574in}{0.637579in}}%
\pgfpathlineto{\pgfqpoint{4.547053in}{0.637579in}}%
\pgfpathlineto{\pgfqpoint{4.547053in}{0.550000in}}%
\pgfpathclose%
\pgfusepath{fill}%
\end{pgfscope}%
\begin{pgfscope}%
\pgfpathrectangle{\pgfqpoint{1.250000in}{0.550000in}}{\pgfqpoint{7.750000in}{3.850000in}}%
\pgfusepath{clip}%
\pgfsetbuttcap%
\pgfsetmiterjoin%
\definecolor{currentfill}{rgb}{0.501961,0.501961,0.501961}%
\pgfsetfillcolor{currentfill}%
\pgfsetfillopacity{0.700000}%
\pgfsetlinewidth{0.000000pt}%
\definecolor{currentstroke}{rgb}{0.000000,0.000000,0.000000}%
\pgfsetstrokecolor{currentstroke}%
\pgfsetstrokeopacity{0.700000}%
\pgfsetdash{}{0pt}%
\pgfpathmoveto{\pgfqpoint{4.574574in}{0.550000in}}%
\pgfpathlineto{\pgfqpoint{4.602095in}{0.550000in}}%
\pgfpathlineto{\pgfqpoint{4.602095in}{0.631760in}}%
\pgfpathlineto{\pgfqpoint{4.574574in}{0.631760in}}%
\pgfpathlineto{\pgfqpoint{4.574574in}{0.550000in}}%
\pgfpathclose%
\pgfusepath{fill}%
\end{pgfscope}%
\begin{pgfscope}%
\pgfpathrectangle{\pgfqpoint{1.250000in}{0.550000in}}{\pgfqpoint{7.750000in}{3.850000in}}%
\pgfusepath{clip}%
\pgfsetbuttcap%
\pgfsetmiterjoin%
\definecolor{currentfill}{rgb}{0.501961,0.501961,0.501961}%
\pgfsetfillcolor{currentfill}%
\pgfsetfillopacity{0.700000}%
\pgfsetlinewidth{0.000000pt}%
\definecolor{currentstroke}{rgb}{0.000000,0.000000,0.000000}%
\pgfsetstrokecolor{currentstroke}%
\pgfsetstrokeopacity{0.700000}%
\pgfsetdash{}{0pt}%
\pgfpathmoveto{\pgfqpoint{4.602095in}{0.550000in}}%
\pgfpathlineto{\pgfqpoint{4.629616in}{0.550000in}}%
\pgfpathlineto{\pgfqpoint{4.629616in}{0.633598in}}%
\pgfpathlineto{\pgfqpoint{4.602095in}{0.633598in}}%
\pgfpathlineto{\pgfqpoint{4.602095in}{0.550000in}}%
\pgfpathclose%
\pgfusepath{fill}%
\end{pgfscope}%
\begin{pgfscope}%
\pgfpathrectangle{\pgfqpoint{1.250000in}{0.550000in}}{\pgfqpoint{7.750000in}{3.850000in}}%
\pgfusepath{clip}%
\pgfsetbuttcap%
\pgfsetmiterjoin%
\definecolor{currentfill}{rgb}{0.501961,0.501961,0.501961}%
\pgfsetfillcolor{currentfill}%
\pgfsetfillopacity{0.700000}%
\pgfsetlinewidth{0.000000pt}%
\definecolor{currentstroke}{rgb}{0.000000,0.000000,0.000000}%
\pgfsetstrokecolor{currentstroke}%
\pgfsetstrokeopacity{0.700000}%
\pgfsetdash{}{0pt}%
\pgfpathmoveto{\pgfqpoint{4.629616in}{0.550000in}}%
\pgfpathlineto{\pgfqpoint{4.657138in}{0.550000in}}%
\pgfpathlineto{\pgfqpoint{4.657138in}{0.642937in}}%
\pgfpathlineto{\pgfqpoint{4.629616in}{0.642937in}}%
\pgfpathlineto{\pgfqpoint{4.629616in}{0.550000in}}%
\pgfpathclose%
\pgfusepath{fill}%
\end{pgfscope}%
\begin{pgfscope}%
\pgfpathrectangle{\pgfqpoint{1.250000in}{0.550000in}}{\pgfqpoint{7.750000in}{3.850000in}}%
\pgfusepath{clip}%
\pgfsetbuttcap%
\pgfsetmiterjoin%
\definecolor{currentfill}{rgb}{0.501961,0.501961,0.501961}%
\pgfsetfillcolor{currentfill}%
\pgfsetfillopacity{0.700000}%
\pgfsetlinewidth{0.000000pt}%
\definecolor{currentstroke}{rgb}{0.000000,0.000000,0.000000}%
\pgfsetstrokecolor{currentstroke}%
\pgfsetstrokeopacity{0.700000}%
\pgfsetdash{}{0pt}%
\pgfpathmoveto{\pgfqpoint{4.657138in}{0.550000in}}%
\pgfpathlineto{\pgfqpoint{4.684659in}{0.550000in}}%
\pgfpathlineto{\pgfqpoint{4.684659in}{0.624411in}}%
\pgfpathlineto{\pgfqpoint{4.657138in}{0.624411in}}%
\pgfpathlineto{\pgfqpoint{4.657138in}{0.550000in}}%
\pgfpathclose%
\pgfusepath{fill}%
\end{pgfscope}%
\begin{pgfscope}%
\pgfpathrectangle{\pgfqpoint{1.250000in}{0.550000in}}{\pgfqpoint{7.750000in}{3.850000in}}%
\pgfusepath{clip}%
\pgfsetbuttcap%
\pgfsetmiterjoin%
\definecolor{currentfill}{rgb}{0.501961,0.501961,0.501961}%
\pgfsetfillcolor{currentfill}%
\pgfsetfillopacity{0.700000}%
\pgfsetlinewidth{0.000000pt}%
\definecolor{currentstroke}{rgb}{0.000000,0.000000,0.000000}%
\pgfsetstrokecolor{currentstroke}%
\pgfsetstrokeopacity{0.700000}%
\pgfsetdash{}{0pt}%
\pgfpathmoveto{\pgfqpoint{4.684659in}{0.550000in}}%
\pgfpathlineto{\pgfqpoint{4.712180in}{0.550000in}}%
\pgfpathlineto{\pgfqpoint{4.712180in}{0.627627in}}%
\pgfpathlineto{\pgfqpoint{4.684659in}{0.627627in}}%
\pgfpathlineto{\pgfqpoint{4.684659in}{0.550000in}}%
\pgfpathclose%
\pgfusepath{fill}%
\end{pgfscope}%
\begin{pgfscope}%
\pgfpathrectangle{\pgfqpoint{1.250000in}{0.550000in}}{\pgfqpoint{7.750000in}{3.850000in}}%
\pgfusepath{clip}%
\pgfsetbuttcap%
\pgfsetmiterjoin%
\definecolor{currentfill}{rgb}{0.501961,0.501961,0.501961}%
\pgfsetfillcolor{currentfill}%
\pgfsetfillopacity{0.700000}%
\pgfsetlinewidth{0.000000pt}%
\definecolor{currentstroke}{rgb}{0.000000,0.000000,0.000000}%
\pgfsetstrokecolor{currentstroke}%
\pgfsetstrokeopacity{0.700000}%
\pgfsetdash{}{0pt}%
\pgfpathmoveto{\pgfqpoint{4.712180in}{0.550000in}}%
\pgfpathlineto{\pgfqpoint{4.739702in}{0.550000in}}%
\pgfpathlineto{\pgfqpoint{4.739702in}{0.636048in}}%
\pgfpathlineto{\pgfqpoint{4.712180in}{0.636048in}}%
\pgfpathlineto{\pgfqpoint{4.712180in}{0.550000in}}%
\pgfpathclose%
\pgfusepath{fill}%
\end{pgfscope}%
\begin{pgfscope}%
\pgfpathrectangle{\pgfqpoint{1.250000in}{0.550000in}}{\pgfqpoint{7.750000in}{3.850000in}}%
\pgfusepath{clip}%
\pgfsetbuttcap%
\pgfsetmiterjoin%
\definecolor{currentfill}{rgb}{0.501961,0.501961,0.501961}%
\pgfsetfillcolor{currentfill}%
\pgfsetfillopacity{0.700000}%
\pgfsetlinewidth{0.000000pt}%
\definecolor{currentstroke}{rgb}{0.000000,0.000000,0.000000}%
\pgfsetstrokecolor{currentstroke}%
\pgfsetstrokeopacity{0.700000}%
\pgfsetdash{}{0pt}%
\pgfpathmoveto{\pgfqpoint{4.739702in}{0.550000in}}%
\pgfpathlineto{\pgfqpoint{4.767223in}{0.550000in}}%
\pgfpathlineto{\pgfqpoint{4.767223in}{0.619052in}}%
\pgfpathlineto{\pgfqpoint{4.739702in}{0.619052in}}%
\pgfpathlineto{\pgfqpoint{4.739702in}{0.550000in}}%
\pgfpathclose%
\pgfusepath{fill}%
\end{pgfscope}%
\begin{pgfscope}%
\pgfpathrectangle{\pgfqpoint{1.250000in}{0.550000in}}{\pgfqpoint{7.750000in}{3.850000in}}%
\pgfusepath{clip}%
\pgfsetbuttcap%
\pgfsetmiterjoin%
\definecolor{currentfill}{rgb}{0.501961,0.501961,0.501961}%
\pgfsetfillcolor{currentfill}%
\pgfsetfillopacity{0.700000}%
\pgfsetlinewidth{0.000000pt}%
\definecolor{currentstroke}{rgb}{0.000000,0.000000,0.000000}%
\pgfsetstrokecolor{currentstroke}%
\pgfsetstrokeopacity{0.700000}%
\pgfsetdash{}{0pt}%
\pgfpathmoveto{\pgfqpoint{4.767223in}{0.550000in}}%
\pgfpathlineto{\pgfqpoint{4.794744in}{0.550000in}}%
\pgfpathlineto{\pgfqpoint{4.794744in}{0.637426in}}%
\pgfpathlineto{\pgfqpoint{4.767223in}{0.637426in}}%
\pgfpathlineto{\pgfqpoint{4.767223in}{0.550000in}}%
\pgfpathclose%
\pgfusepath{fill}%
\end{pgfscope}%
\begin{pgfscope}%
\pgfpathrectangle{\pgfqpoint{1.250000in}{0.550000in}}{\pgfqpoint{7.750000in}{3.850000in}}%
\pgfusepath{clip}%
\pgfsetbuttcap%
\pgfsetmiterjoin%
\definecolor{currentfill}{rgb}{0.501961,0.501961,0.501961}%
\pgfsetfillcolor{currentfill}%
\pgfsetfillopacity{0.700000}%
\pgfsetlinewidth{0.000000pt}%
\definecolor{currentstroke}{rgb}{0.000000,0.000000,0.000000}%
\pgfsetstrokecolor{currentstroke}%
\pgfsetstrokeopacity{0.700000}%
\pgfsetdash{}{0pt}%
\pgfpathmoveto{\pgfqpoint{4.794744in}{0.550000in}}%
\pgfpathlineto{\pgfqpoint{4.822266in}{0.550000in}}%
\pgfpathlineto{\pgfqpoint{4.822266in}{0.622421in}}%
\pgfpathlineto{\pgfqpoint{4.794744in}{0.622421in}}%
\pgfpathlineto{\pgfqpoint{4.794744in}{0.550000in}}%
\pgfpathclose%
\pgfusepath{fill}%
\end{pgfscope}%
\begin{pgfscope}%
\pgfpathrectangle{\pgfqpoint{1.250000in}{0.550000in}}{\pgfqpoint{7.750000in}{3.850000in}}%
\pgfusepath{clip}%
\pgfsetbuttcap%
\pgfsetmiterjoin%
\definecolor{currentfill}{rgb}{0.501961,0.501961,0.501961}%
\pgfsetfillcolor{currentfill}%
\pgfsetfillopacity{0.700000}%
\pgfsetlinewidth{0.000000pt}%
\definecolor{currentstroke}{rgb}{0.000000,0.000000,0.000000}%
\pgfsetstrokecolor{currentstroke}%
\pgfsetstrokeopacity{0.700000}%
\pgfsetdash{}{0pt}%
\pgfpathmoveto{\pgfqpoint{4.822266in}{0.550000in}}%
\pgfpathlineto{\pgfqpoint{4.849787in}{0.550000in}}%
\pgfpathlineto{\pgfqpoint{4.849787in}{0.633904in}}%
\pgfpathlineto{\pgfqpoint{4.822266in}{0.633904in}}%
\pgfpathlineto{\pgfqpoint{4.822266in}{0.550000in}}%
\pgfpathclose%
\pgfusepath{fill}%
\end{pgfscope}%
\begin{pgfscope}%
\pgfpathrectangle{\pgfqpoint{1.250000in}{0.550000in}}{\pgfqpoint{7.750000in}{3.850000in}}%
\pgfusepath{clip}%
\pgfsetbuttcap%
\pgfsetmiterjoin%
\definecolor{currentfill}{rgb}{0.501961,0.501961,0.501961}%
\pgfsetfillcolor{currentfill}%
\pgfsetfillopacity{0.700000}%
\pgfsetlinewidth{0.000000pt}%
\definecolor{currentstroke}{rgb}{0.000000,0.000000,0.000000}%
\pgfsetstrokecolor{currentstroke}%
\pgfsetstrokeopacity{0.700000}%
\pgfsetdash{}{0pt}%
\pgfpathmoveto{\pgfqpoint{4.849787in}{0.550000in}}%
\pgfpathlineto{\pgfqpoint{4.877308in}{0.550000in}}%
\pgfpathlineto{\pgfqpoint{4.877308in}{0.619052in}}%
\pgfpathlineto{\pgfqpoint{4.849787in}{0.619052in}}%
\pgfpathlineto{\pgfqpoint{4.849787in}{0.550000in}}%
\pgfpathclose%
\pgfusepath{fill}%
\end{pgfscope}%
\begin{pgfscope}%
\pgfpathrectangle{\pgfqpoint{1.250000in}{0.550000in}}{\pgfqpoint{7.750000in}{3.850000in}}%
\pgfusepath{clip}%
\pgfsetbuttcap%
\pgfsetmiterjoin%
\definecolor{currentfill}{rgb}{0.501961,0.501961,0.501961}%
\pgfsetfillcolor{currentfill}%
\pgfsetfillopacity{0.700000}%
\pgfsetlinewidth{0.000000pt}%
\definecolor{currentstroke}{rgb}{0.000000,0.000000,0.000000}%
\pgfsetstrokecolor{currentstroke}%
\pgfsetstrokeopacity{0.700000}%
\pgfsetdash{}{0pt}%
\pgfpathmoveto{\pgfqpoint{4.877308in}{0.550000in}}%
\pgfpathlineto{\pgfqpoint{4.904830in}{0.550000in}}%
\pgfpathlineto{\pgfqpoint{4.904830in}{0.622574in}}%
\pgfpathlineto{\pgfqpoint{4.877308in}{0.622574in}}%
\pgfpathlineto{\pgfqpoint{4.877308in}{0.550000in}}%
\pgfpathclose%
\pgfusepath{fill}%
\end{pgfscope}%
\begin{pgfscope}%
\pgfpathrectangle{\pgfqpoint{1.250000in}{0.550000in}}{\pgfqpoint{7.750000in}{3.850000in}}%
\pgfusepath{clip}%
\pgfsetbuttcap%
\pgfsetmiterjoin%
\definecolor{currentfill}{rgb}{0.501961,0.501961,0.501961}%
\pgfsetfillcolor{currentfill}%
\pgfsetfillopacity{0.700000}%
\pgfsetlinewidth{0.000000pt}%
\definecolor{currentstroke}{rgb}{0.000000,0.000000,0.000000}%
\pgfsetstrokecolor{currentstroke}%
\pgfsetstrokeopacity{0.700000}%
\pgfsetdash{}{0pt}%
\pgfpathmoveto{\pgfqpoint{4.904830in}{0.550000in}}%
\pgfpathlineto{\pgfqpoint{4.932351in}{0.550000in}}%
\pgfpathlineto{\pgfqpoint{4.932351in}{0.628545in}}%
\pgfpathlineto{\pgfqpoint{4.904830in}{0.628545in}}%
\pgfpathlineto{\pgfqpoint{4.904830in}{0.550000in}}%
\pgfpathclose%
\pgfusepath{fill}%
\end{pgfscope}%
\begin{pgfscope}%
\pgfpathrectangle{\pgfqpoint{1.250000in}{0.550000in}}{\pgfqpoint{7.750000in}{3.850000in}}%
\pgfusepath{clip}%
\pgfsetbuttcap%
\pgfsetmiterjoin%
\definecolor{currentfill}{rgb}{0.501961,0.501961,0.501961}%
\pgfsetfillcolor{currentfill}%
\pgfsetfillopacity{0.700000}%
\pgfsetlinewidth{0.000000pt}%
\definecolor{currentstroke}{rgb}{0.000000,0.000000,0.000000}%
\pgfsetstrokecolor{currentstroke}%
\pgfsetstrokeopacity{0.700000}%
\pgfsetdash{}{0pt}%
\pgfpathmoveto{\pgfqpoint{4.932351in}{0.550000in}}%
\pgfpathlineto{\pgfqpoint{4.959872in}{0.550000in}}%
\pgfpathlineto{\pgfqpoint{4.959872in}{0.616143in}}%
\pgfpathlineto{\pgfqpoint{4.932351in}{0.616143in}}%
\pgfpathlineto{\pgfqpoint{4.932351in}{0.550000in}}%
\pgfpathclose%
\pgfusepath{fill}%
\end{pgfscope}%
\begin{pgfscope}%
\pgfpathrectangle{\pgfqpoint{1.250000in}{0.550000in}}{\pgfqpoint{7.750000in}{3.850000in}}%
\pgfusepath{clip}%
\pgfsetbuttcap%
\pgfsetmiterjoin%
\definecolor{currentfill}{rgb}{0.501961,0.501961,0.501961}%
\pgfsetfillcolor{currentfill}%
\pgfsetfillopacity{0.700000}%
\pgfsetlinewidth{0.000000pt}%
\definecolor{currentstroke}{rgb}{0.000000,0.000000,0.000000}%
\pgfsetstrokecolor{currentstroke}%
\pgfsetstrokeopacity{0.700000}%
\pgfsetdash{}{0pt}%
\pgfpathmoveto{\pgfqpoint{4.959872in}{0.550000in}}%
\pgfpathlineto{\pgfqpoint{4.987393in}{0.550000in}}%
\pgfpathlineto{\pgfqpoint{4.987393in}{0.620583in}}%
\pgfpathlineto{\pgfqpoint{4.959872in}{0.620583in}}%
\pgfpathlineto{\pgfqpoint{4.959872in}{0.550000in}}%
\pgfpathclose%
\pgfusepath{fill}%
\end{pgfscope}%
\begin{pgfscope}%
\pgfpathrectangle{\pgfqpoint{1.250000in}{0.550000in}}{\pgfqpoint{7.750000in}{3.850000in}}%
\pgfusepath{clip}%
\pgfsetbuttcap%
\pgfsetmiterjoin%
\definecolor{currentfill}{rgb}{0.501961,0.501961,0.501961}%
\pgfsetfillcolor{currentfill}%
\pgfsetfillopacity{0.700000}%
\pgfsetlinewidth{0.000000pt}%
\definecolor{currentstroke}{rgb}{0.000000,0.000000,0.000000}%
\pgfsetstrokecolor{currentstroke}%
\pgfsetstrokeopacity{0.700000}%
\pgfsetdash{}{0pt}%
\pgfpathmoveto{\pgfqpoint{4.987393in}{0.550000in}}%
\pgfpathlineto{\pgfqpoint{5.014915in}{0.550000in}}%
\pgfpathlineto{\pgfqpoint{5.014915in}{0.619971in}}%
\pgfpathlineto{\pgfqpoint{4.987393in}{0.619971in}}%
\pgfpathlineto{\pgfqpoint{4.987393in}{0.550000in}}%
\pgfpathclose%
\pgfusepath{fill}%
\end{pgfscope}%
\begin{pgfscope}%
\pgfpathrectangle{\pgfqpoint{1.250000in}{0.550000in}}{\pgfqpoint{7.750000in}{3.850000in}}%
\pgfusepath{clip}%
\pgfsetbuttcap%
\pgfsetmiterjoin%
\definecolor{currentfill}{rgb}{0.501961,0.501961,0.501961}%
\pgfsetfillcolor{currentfill}%
\pgfsetfillopacity{0.700000}%
\pgfsetlinewidth{0.000000pt}%
\definecolor{currentstroke}{rgb}{0.000000,0.000000,0.000000}%
\pgfsetstrokecolor{currentstroke}%
\pgfsetstrokeopacity{0.700000}%
\pgfsetdash{}{0pt}%
\pgfpathmoveto{\pgfqpoint{5.014915in}{0.550000in}}%
\pgfpathlineto{\pgfqpoint{5.042436in}{0.550000in}}%
\pgfpathlineto{\pgfqpoint{5.042436in}{0.610784in}}%
\pgfpathlineto{\pgfqpoint{5.014915in}{0.610784in}}%
\pgfpathlineto{\pgfqpoint{5.014915in}{0.550000in}}%
\pgfpathclose%
\pgfusepath{fill}%
\end{pgfscope}%
\begin{pgfscope}%
\pgfpathrectangle{\pgfqpoint{1.250000in}{0.550000in}}{\pgfqpoint{7.750000in}{3.850000in}}%
\pgfusepath{clip}%
\pgfsetbuttcap%
\pgfsetmiterjoin%
\definecolor{currentfill}{rgb}{0.501961,0.501961,0.501961}%
\pgfsetfillcolor{currentfill}%
\pgfsetfillopacity{0.700000}%
\pgfsetlinewidth{0.000000pt}%
\definecolor{currentstroke}{rgb}{0.000000,0.000000,0.000000}%
\pgfsetstrokecolor{currentstroke}%
\pgfsetstrokeopacity{0.700000}%
\pgfsetdash{}{0pt}%
\pgfpathmoveto{\pgfqpoint{5.042436in}{0.550000in}}%
\pgfpathlineto{\pgfqpoint{5.069957in}{0.550000in}}%
\pgfpathlineto{\pgfqpoint{5.069957in}{0.604201in}}%
\pgfpathlineto{\pgfqpoint{5.042436in}{0.604201in}}%
\pgfpathlineto{\pgfqpoint{5.042436in}{0.550000in}}%
\pgfpathclose%
\pgfusepath{fill}%
\end{pgfscope}%
\begin{pgfscope}%
\pgfpathrectangle{\pgfqpoint{1.250000in}{0.550000in}}{\pgfqpoint{7.750000in}{3.850000in}}%
\pgfusepath{clip}%
\pgfsetbuttcap%
\pgfsetmiterjoin%
\definecolor{currentfill}{rgb}{0.501961,0.501961,0.501961}%
\pgfsetfillcolor{currentfill}%
\pgfsetfillopacity{0.700000}%
\pgfsetlinewidth{0.000000pt}%
\definecolor{currentstroke}{rgb}{0.000000,0.000000,0.000000}%
\pgfsetstrokecolor{currentstroke}%
\pgfsetstrokeopacity{0.700000}%
\pgfsetdash{}{0pt}%
\pgfpathmoveto{\pgfqpoint{5.069957in}{0.550000in}}%
\pgfpathlineto{\pgfqpoint{5.097479in}{0.550000in}}%
\pgfpathlineto{\pgfqpoint{5.097479in}{0.601751in}}%
\pgfpathlineto{\pgfqpoint{5.069957in}{0.601751in}}%
\pgfpathlineto{\pgfqpoint{5.069957in}{0.550000in}}%
\pgfpathclose%
\pgfusepath{fill}%
\end{pgfscope}%
\begin{pgfscope}%
\pgfpathrectangle{\pgfqpoint{1.250000in}{0.550000in}}{\pgfqpoint{7.750000in}{3.850000in}}%
\pgfusepath{clip}%
\pgfsetbuttcap%
\pgfsetmiterjoin%
\definecolor{currentfill}{rgb}{0.501961,0.501961,0.501961}%
\pgfsetfillcolor{currentfill}%
\pgfsetfillopacity{0.700000}%
\pgfsetlinewidth{0.000000pt}%
\definecolor{currentstroke}{rgb}{0.000000,0.000000,0.000000}%
\pgfsetstrokecolor{currentstroke}%
\pgfsetstrokeopacity{0.700000}%
\pgfsetdash{}{0pt}%
\pgfpathmoveto{\pgfqpoint{5.097479in}{0.550000in}}%
\pgfpathlineto{\pgfqpoint{5.125000in}{0.550000in}}%
\pgfpathlineto{\pgfqpoint{5.125000in}{0.600679in}}%
\pgfpathlineto{\pgfqpoint{5.097479in}{0.600679in}}%
\pgfpathlineto{\pgfqpoint{5.097479in}{0.550000in}}%
\pgfpathclose%
\pgfusepath{fill}%
\end{pgfscope}%
\begin{pgfscope}%
\pgfpathrectangle{\pgfqpoint{1.250000in}{0.550000in}}{\pgfqpoint{7.750000in}{3.850000in}}%
\pgfusepath{clip}%
\pgfsetbuttcap%
\pgfsetmiterjoin%
\definecolor{currentfill}{rgb}{0.501961,0.501961,0.501961}%
\pgfsetfillcolor{currentfill}%
\pgfsetfillopacity{0.700000}%
\pgfsetlinewidth{0.000000pt}%
\definecolor{currentstroke}{rgb}{0.000000,0.000000,0.000000}%
\pgfsetstrokecolor{currentstroke}%
\pgfsetstrokeopacity{0.700000}%
\pgfsetdash{}{0pt}%
\pgfpathmoveto{\pgfqpoint{5.125000in}{0.550000in}}%
\pgfpathlineto{\pgfqpoint{5.152521in}{0.550000in}}%
\pgfpathlineto{\pgfqpoint{5.152521in}{0.600526in}}%
\pgfpathlineto{\pgfqpoint{5.125000in}{0.600526in}}%
\pgfpathlineto{\pgfqpoint{5.125000in}{0.550000in}}%
\pgfpathclose%
\pgfusepath{fill}%
\end{pgfscope}%
\begin{pgfscope}%
\pgfpathrectangle{\pgfqpoint{1.250000in}{0.550000in}}{\pgfqpoint{7.750000in}{3.850000in}}%
\pgfusepath{clip}%
\pgfsetbuttcap%
\pgfsetmiterjoin%
\definecolor{currentfill}{rgb}{0.501961,0.501961,0.501961}%
\pgfsetfillcolor{currentfill}%
\pgfsetfillopacity{0.700000}%
\pgfsetlinewidth{0.000000pt}%
\definecolor{currentstroke}{rgb}{0.000000,0.000000,0.000000}%
\pgfsetstrokecolor{currentstroke}%
\pgfsetstrokeopacity{0.700000}%
\pgfsetdash{}{0pt}%
\pgfpathmoveto{\pgfqpoint{5.152521in}{0.550000in}}%
\pgfpathlineto{\pgfqpoint{5.180043in}{0.550000in}}%
\pgfpathlineto{\pgfqpoint{5.180043in}{0.589196in}}%
\pgfpathlineto{\pgfqpoint{5.152521in}{0.589196in}}%
\pgfpathlineto{\pgfqpoint{5.152521in}{0.550000in}}%
\pgfpathclose%
\pgfusepath{fill}%
\end{pgfscope}%
\begin{pgfscope}%
\pgfpathrectangle{\pgfqpoint{1.250000in}{0.550000in}}{\pgfqpoint{7.750000in}{3.850000in}}%
\pgfusepath{clip}%
\pgfsetbuttcap%
\pgfsetmiterjoin%
\definecolor{currentfill}{rgb}{0.501961,0.501961,0.501961}%
\pgfsetfillcolor{currentfill}%
\pgfsetfillopacity{0.700000}%
\pgfsetlinewidth{0.000000pt}%
\definecolor{currentstroke}{rgb}{0.000000,0.000000,0.000000}%
\pgfsetstrokecolor{currentstroke}%
\pgfsetstrokeopacity{0.700000}%
\pgfsetdash{}{0pt}%
\pgfpathmoveto{\pgfqpoint{5.180043in}{0.550000in}}%
\pgfpathlineto{\pgfqpoint{5.207564in}{0.550000in}}%
\pgfpathlineto{\pgfqpoint{5.207564in}{0.597464in}}%
\pgfpathlineto{\pgfqpoint{5.180043in}{0.597464in}}%
\pgfpathlineto{\pgfqpoint{5.180043in}{0.550000in}}%
\pgfpathclose%
\pgfusepath{fill}%
\end{pgfscope}%
\begin{pgfscope}%
\pgfpathrectangle{\pgfqpoint{1.250000in}{0.550000in}}{\pgfqpoint{7.750000in}{3.850000in}}%
\pgfusepath{clip}%
\pgfsetbuttcap%
\pgfsetmiterjoin%
\definecolor{currentfill}{rgb}{0.501961,0.501961,0.501961}%
\pgfsetfillcolor{currentfill}%
\pgfsetfillopacity{0.700000}%
\pgfsetlinewidth{0.000000pt}%
\definecolor{currentstroke}{rgb}{0.000000,0.000000,0.000000}%
\pgfsetstrokecolor{currentstroke}%
\pgfsetstrokeopacity{0.700000}%
\pgfsetdash{}{0pt}%
\pgfpathmoveto{\pgfqpoint{5.207564in}{0.550000in}}%
\pgfpathlineto{\pgfqpoint{5.235085in}{0.550000in}}%
\pgfpathlineto{\pgfqpoint{5.235085in}{0.587818in}}%
\pgfpathlineto{\pgfqpoint{5.207564in}{0.587818in}}%
\pgfpathlineto{\pgfqpoint{5.207564in}{0.550000in}}%
\pgfpathclose%
\pgfusepath{fill}%
\end{pgfscope}%
\begin{pgfscope}%
\pgfpathrectangle{\pgfqpoint{1.250000in}{0.550000in}}{\pgfqpoint{7.750000in}{3.850000in}}%
\pgfusepath{clip}%
\pgfsetbuttcap%
\pgfsetmiterjoin%
\definecolor{currentfill}{rgb}{0.501961,0.501961,0.501961}%
\pgfsetfillcolor{currentfill}%
\pgfsetfillopacity{0.700000}%
\pgfsetlinewidth{0.000000pt}%
\definecolor{currentstroke}{rgb}{0.000000,0.000000,0.000000}%
\pgfsetstrokecolor{currentstroke}%
\pgfsetstrokeopacity{0.700000}%
\pgfsetdash{}{0pt}%
\pgfpathmoveto{\pgfqpoint{5.235085in}{0.550000in}}%
\pgfpathlineto{\pgfqpoint{5.262607in}{0.550000in}}%
\pgfpathlineto{\pgfqpoint{5.262607in}{0.592564in}}%
\pgfpathlineto{\pgfqpoint{5.235085in}{0.592564in}}%
\pgfpathlineto{\pgfqpoint{5.235085in}{0.550000in}}%
\pgfpathclose%
\pgfusepath{fill}%
\end{pgfscope}%
\begin{pgfscope}%
\pgfpathrectangle{\pgfqpoint{1.250000in}{0.550000in}}{\pgfqpoint{7.750000in}{3.850000in}}%
\pgfusepath{clip}%
\pgfsetbuttcap%
\pgfsetmiterjoin%
\definecolor{currentfill}{rgb}{0.501961,0.501961,0.501961}%
\pgfsetfillcolor{currentfill}%
\pgfsetfillopacity{0.700000}%
\pgfsetlinewidth{0.000000pt}%
\definecolor{currentstroke}{rgb}{0.000000,0.000000,0.000000}%
\pgfsetstrokecolor{currentstroke}%
\pgfsetstrokeopacity{0.700000}%
\pgfsetdash{}{0pt}%
\pgfpathmoveto{\pgfqpoint{5.262607in}{0.550000in}}%
\pgfpathlineto{\pgfqpoint{5.290128in}{0.550000in}}%
\pgfpathlineto{\pgfqpoint{5.290128in}{0.584909in}}%
\pgfpathlineto{\pgfqpoint{5.262607in}{0.584909in}}%
\pgfpathlineto{\pgfqpoint{5.262607in}{0.550000in}}%
\pgfpathclose%
\pgfusepath{fill}%
\end{pgfscope}%
\begin{pgfscope}%
\pgfpathrectangle{\pgfqpoint{1.250000in}{0.550000in}}{\pgfqpoint{7.750000in}{3.850000in}}%
\pgfusepath{clip}%
\pgfsetbuttcap%
\pgfsetmiterjoin%
\definecolor{currentfill}{rgb}{0.501961,0.501961,0.501961}%
\pgfsetfillcolor{currentfill}%
\pgfsetfillopacity{0.700000}%
\pgfsetlinewidth{0.000000pt}%
\definecolor{currentstroke}{rgb}{0.000000,0.000000,0.000000}%
\pgfsetstrokecolor{currentstroke}%
\pgfsetstrokeopacity{0.700000}%
\pgfsetdash{}{0pt}%
\pgfpathmoveto{\pgfqpoint{5.290128in}{0.550000in}}%
\pgfpathlineto{\pgfqpoint{5.317649in}{0.550000in}}%
\pgfpathlineto{\pgfqpoint{5.317649in}{0.579091in}}%
\pgfpathlineto{\pgfqpoint{5.290128in}{0.579091in}}%
\pgfpathlineto{\pgfqpoint{5.290128in}{0.550000in}}%
\pgfpathclose%
\pgfusepath{fill}%
\end{pgfscope}%
\begin{pgfscope}%
\pgfpathrectangle{\pgfqpoint{1.250000in}{0.550000in}}{\pgfqpoint{7.750000in}{3.850000in}}%
\pgfusepath{clip}%
\pgfsetbuttcap%
\pgfsetmiterjoin%
\definecolor{currentfill}{rgb}{0.501961,0.501961,0.501961}%
\pgfsetfillcolor{currentfill}%
\pgfsetfillopacity{0.700000}%
\pgfsetlinewidth{0.000000pt}%
\definecolor{currentstroke}{rgb}{0.000000,0.000000,0.000000}%
\pgfsetstrokecolor{currentstroke}%
\pgfsetstrokeopacity{0.700000}%
\pgfsetdash{}{0pt}%
\pgfpathmoveto{\pgfqpoint{5.317649in}{0.550000in}}%
\pgfpathlineto{\pgfqpoint{5.345170in}{0.550000in}}%
\pgfpathlineto{\pgfqpoint{5.345170in}{0.579550in}}%
\pgfpathlineto{\pgfqpoint{5.317649in}{0.579550in}}%
\pgfpathlineto{\pgfqpoint{5.317649in}{0.550000in}}%
\pgfpathclose%
\pgfusepath{fill}%
\end{pgfscope}%
\begin{pgfscope}%
\pgfpathrectangle{\pgfqpoint{1.250000in}{0.550000in}}{\pgfqpoint{7.750000in}{3.850000in}}%
\pgfusepath{clip}%
\pgfsetbuttcap%
\pgfsetmiterjoin%
\definecolor{currentfill}{rgb}{0.501961,0.501961,0.501961}%
\pgfsetfillcolor{currentfill}%
\pgfsetfillopacity{0.700000}%
\pgfsetlinewidth{0.000000pt}%
\definecolor{currentstroke}{rgb}{0.000000,0.000000,0.000000}%
\pgfsetstrokecolor{currentstroke}%
\pgfsetstrokeopacity{0.700000}%
\pgfsetdash{}{0pt}%
\pgfpathmoveto{\pgfqpoint{5.345170in}{0.550000in}}%
\pgfpathlineto{\pgfqpoint{5.372692in}{0.550000in}}%
\pgfpathlineto{\pgfqpoint{5.372692in}{0.575263in}}%
\pgfpathlineto{\pgfqpoint{5.345170in}{0.575263in}}%
\pgfpathlineto{\pgfqpoint{5.345170in}{0.550000in}}%
\pgfpathclose%
\pgfusepath{fill}%
\end{pgfscope}%
\begin{pgfscope}%
\pgfpathrectangle{\pgfqpoint{1.250000in}{0.550000in}}{\pgfqpoint{7.750000in}{3.850000in}}%
\pgfusepath{clip}%
\pgfsetbuttcap%
\pgfsetmiterjoin%
\definecolor{currentfill}{rgb}{0.501961,0.501961,0.501961}%
\pgfsetfillcolor{currentfill}%
\pgfsetfillopacity{0.700000}%
\pgfsetlinewidth{0.000000pt}%
\definecolor{currentstroke}{rgb}{0.000000,0.000000,0.000000}%
\pgfsetstrokecolor{currentstroke}%
\pgfsetstrokeopacity{0.700000}%
\pgfsetdash{}{0pt}%
\pgfpathmoveto{\pgfqpoint{5.372692in}{0.550000in}}%
\pgfpathlineto{\pgfqpoint{5.400213in}{0.550000in}}%
\pgfpathlineto{\pgfqpoint{5.400213in}{0.574498in}}%
\pgfpathlineto{\pgfqpoint{5.372692in}{0.574498in}}%
\pgfpathlineto{\pgfqpoint{5.372692in}{0.550000in}}%
\pgfpathclose%
\pgfusepath{fill}%
\end{pgfscope}%
\begin{pgfscope}%
\pgfpathrectangle{\pgfqpoint{1.250000in}{0.550000in}}{\pgfqpoint{7.750000in}{3.850000in}}%
\pgfusepath{clip}%
\pgfsetbuttcap%
\pgfsetmiterjoin%
\definecolor{currentfill}{rgb}{0.501961,0.501961,0.501961}%
\pgfsetfillcolor{currentfill}%
\pgfsetfillopacity{0.700000}%
\pgfsetlinewidth{0.000000pt}%
\definecolor{currentstroke}{rgb}{0.000000,0.000000,0.000000}%
\pgfsetstrokecolor{currentstroke}%
\pgfsetstrokeopacity{0.700000}%
\pgfsetdash{}{0pt}%
\pgfpathmoveto{\pgfqpoint{5.400213in}{0.550000in}}%
\pgfpathlineto{\pgfqpoint{5.427734in}{0.550000in}}%
\pgfpathlineto{\pgfqpoint{5.427734in}{0.574804in}}%
\pgfpathlineto{\pgfqpoint{5.400213in}{0.574804in}}%
\pgfpathlineto{\pgfqpoint{5.400213in}{0.550000in}}%
\pgfpathclose%
\pgfusepath{fill}%
\end{pgfscope}%
\begin{pgfscope}%
\pgfpathrectangle{\pgfqpoint{1.250000in}{0.550000in}}{\pgfqpoint{7.750000in}{3.850000in}}%
\pgfusepath{clip}%
\pgfsetbuttcap%
\pgfsetmiterjoin%
\definecolor{currentfill}{rgb}{0.501961,0.501961,0.501961}%
\pgfsetfillcolor{currentfill}%
\pgfsetfillopacity{0.700000}%
\pgfsetlinewidth{0.000000pt}%
\definecolor{currentstroke}{rgb}{0.000000,0.000000,0.000000}%
\pgfsetstrokecolor{currentstroke}%
\pgfsetstrokeopacity{0.700000}%
\pgfsetdash{}{0pt}%
\pgfpathmoveto{\pgfqpoint{5.427734in}{0.550000in}}%
\pgfpathlineto{\pgfqpoint{5.455256in}{0.550000in}}%
\pgfpathlineto{\pgfqpoint{5.455256in}{0.573426in}}%
\pgfpathlineto{\pgfqpoint{5.427734in}{0.573426in}}%
\pgfpathlineto{\pgfqpoint{5.427734in}{0.550000in}}%
\pgfpathclose%
\pgfusepath{fill}%
\end{pgfscope}%
\begin{pgfscope}%
\pgfpathrectangle{\pgfqpoint{1.250000in}{0.550000in}}{\pgfqpoint{7.750000in}{3.850000in}}%
\pgfusepath{clip}%
\pgfsetbuttcap%
\pgfsetmiterjoin%
\definecolor{currentfill}{rgb}{0.501961,0.501961,0.501961}%
\pgfsetfillcolor{currentfill}%
\pgfsetfillopacity{0.700000}%
\pgfsetlinewidth{0.000000pt}%
\definecolor{currentstroke}{rgb}{0.000000,0.000000,0.000000}%
\pgfsetstrokecolor{currentstroke}%
\pgfsetstrokeopacity{0.700000}%
\pgfsetdash{}{0pt}%
\pgfpathmoveto{\pgfqpoint{5.455256in}{0.550000in}}%
\pgfpathlineto{\pgfqpoint{5.482777in}{0.550000in}}%
\pgfpathlineto{\pgfqpoint{5.482777in}{0.568832in}}%
\pgfpathlineto{\pgfqpoint{5.455256in}{0.568832in}}%
\pgfpathlineto{\pgfqpoint{5.455256in}{0.550000in}}%
\pgfpathclose%
\pgfusepath{fill}%
\end{pgfscope}%
\begin{pgfscope}%
\pgfpathrectangle{\pgfqpoint{1.250000in}{0.550000in}}{\pgfqpoint{7.750000in}{3.850000in}}%
\pgfusepath{clip}%
\pgfsetbuttcap%
\pgfsetmiterjoin%
\definecolor{currentfill}{rgb}{0.501961,0.501961,0.501961}%
\pgfsetfillcolor{currentfill}%
\pgfsetfillopacity{0.700000}%
\pgfsetlinewidth{0.000000pt}%
\definecolor{currentstroke}{rgb}{0.000000,0.000000,0.000000}%
\pgfsetstrokecolor{currentstroke}%
\pgfsetstrokeopacity{0.700000}%
\pgfsetdash{}{0pt}%
\pgfpathmoveto{\pgfqpoint{5.482777in}{0.550000in}}%
\pgfpathlineto{\pgfqpoint{5.510298in}{0.550000in}}%
\pgfpathlineto{\pgfqpoint{5.510298in}{0.566995in}}%
\pgfpathlineto{\pgfqpoint{5.482777in}{0.566995in}}%
\pgfpathlineto{\pgfqpoint{5.482777in}{0.550000in}}%
\pgfpathclose%
\pgfusepath{fill}%
\end{pgfscope}%
\begin{pgfscope}%
\pgfpathrectangle{\pgfqpoint{1.250000in}{0.550000in}}{\pgfqpoint{7.750000in}{3.850000in}}%
\pgfusepath{clip}%
\pgfsetbuttcap%
\pgfsetmiterjoin%
\definecolor{currentfill}{rgb}{0.501961,0.501961,0.501961}%
\pgfsetfillcolor{currentfill}%
\pgfsetfillopacity{0.700000}%
\pgfsetlinewidth{0.000000pt}%
\definecolor{currentstroke}{rgb}{0.000000,0.000000,0.000000}%
\pgfsetstrokecolor{currentstroke}%
\pgfsetstrokeopacity{0.700000}%
\pgfsetdash{}{0pt}%
\pgfpathmoveto{\pgfqpoint{5.510298in}{0.550000in}}%
\pgfpathlineto{\pgfqpoint{5.537820in}{0.550000in}}%
\pgfpathlineto{\pgfqpoint{5.537820in}{0.566383in}}%
\pgfpathlineto{\pgfqpoint{5.510298in}{0.566383in}}%
\pgfpathlineto{\pgfqpoint{5.510298in}{0.550000in}}%
\pgfpathclose%
\pgfusepath{fill}%
\end{pgfscope}%
\begin{pgfscope}%
\pgfpathrectangle{\pgfqpoint{1.250000in}{0.550000in}}{\pgfqpoint{7.750000in}{3.850000in}}%
\pgfusepath{clip}%
\pgfsetbuttcap%
\pgfsetmiterjoin%
\definecolor{currentfill}{rgb}{0.501961,0.501961,0.501961}%
\pgfsetfillcolor{currentfill}%
\pgfsetfillopacity{0.700000}%
\pgfsetlinewidth{0.000000pt}%
\definecolor{currentstroke}{rgb}{0.000000,0.000000,0.000000}%
\pgfsetstrokecolor{currentstroke}%
\pgfsetstrokeopacity{0.700000}%
\pgfsetdash{}{0pt}%
\pgfpathmoveto{\pgfqpoint{5.537820in}{0.550000in}}%
\pgfpathlineto{\pgfqpoint{5.565341in}{0.550000in}}%
\pgfpathlineto{\pgfqpoint{5.565341in}{0.561483in}}%
\pgfpathlineto{\pgfqpoint{5.537820in}{0.561483in}}%
\pgfpathlineto{\pgfqpoint{5.537820in}{0.550000in}}%
\pgfpathclose%
\pgfusepath{fill}%
\end{pgfscope}%
\begin{pgfscope}%
\pgfpathrectangle{\pgfqpoint{1.250000in}{0.550000in}}{\pgfqpoint{7.750000in}{3.850000in}}%
\pgfusepath{clip}%
\pgfsetbuttcap%
\pgfsetmiterjoin%
\definecolor{currentfill}{rgb}{0.501961,0.501961,0.501961}%
\pgfsetfillcolor{currentfill}%
\pgfsetfillopacity{0.700000}%
\pgfsetlinewidth{0.000000pt}%
\definecolor{currentstroke}{rgb}{0.000000,0.000000,0.000000}%
\pgfsetstrokecolor{currentstroke}%
\pgfsetstrokeopacity{0.700000}%
\pgfsetdash{}{0pt}%
\pgfpathmoveto{\pgfqpoint{5.565341in}{0.550000in}}%
\pgfpathlineto{\pgfqpoint{5.592862in}{0.550000in}}%
\pgfpathlineto{\pgfqpoint{5.592862in}{0.563167in}}%
\pgfpathlineto{\pgfqpoint{5.565341in}{0.563167in}}%
\pgfpathlineto{\pgfqpoint{5.565341in}{0.550000in}}%
\pgfpathclose%
\pgfusepath{fill}%
\end{pgfscope}%
\begin{pgfscope}%
\pgfpathrectangle{\pgfqpoint{1.250000in}{0.550000in}}{\pgfqpoint{7.750000in}{3.850000in}}%
\pgfusepath{clip}%
\pgfsetbuttcap%
\pgfsetmiterjoin%
\definecolor{currentfill}{rgb}{0.501961,0.501961,0.501961}%
\pgfsetfillcolor{currentfill}%
\pgfsetfillopacity{0.700000}%
\pgfsetlinewidth{0.000000pt}%
\definecolor{currentstroke}{rgb}{0.000000,0.000000,0.000000}%
\pgfsetstrokecolor{currentstroke}%
\pgfsetstrokeopacity{0.700000}%
\pgfsetdash{}{0pt}%
\pgfpathmoveto{\pgfqpoint{5.592862in}{0.550000in}}%
\pgfpathlineto{\pgfqpoint{5.620384in}{0.550000in}}%
\pgfpathlineto{\pgfqpoint{5.620384in}{0.561636in}}%
\pgfpathlineto{\pgfqpoint{5.592862in}{0.561636in}}%
\pgfpathlineto{\pgfqpoint{5.592862in}{0.550000in}}%
\pgfpathclose%
\pgfusepath{fill}%
\end{pgfscope}%
\begin{pgfscope}%
\pgfpathrectangle{\pgfqpoint{1.250000in}{0.550000in}}{\pgfqpoint{7.750000in}{3.850000in}}%
\pgfusepath{clip}%
\pgfsetbuttcap%
\pgfsetmiterjoin%
\definecolor{currentfill}{rgb}{0.501961,0.501961,0.501961}%
\pgfsetfillcolor{currentfill}%
\pgfsetfillopacity{0.700000}%
\pgfsetlinewidth{0.000000pt}%
\definecolor{currentstroke}{rgb}{0.000000,0.000000,0.000000}%
\pgfsetstrokecolor{currentstroke}%
\pgfsetstrokeopacity{0.700000}%
\pgfsetdash{}{0pt}%
\pgfpathmoveto{\pgfqpoint{5.620384in}{0.550000in}}%
\pgfpathlineto{\pgfqpoint{5.647905in}{0.550000in}}%
\pgfpathlineto{\pgfqpoint{5.647905in}{0.560258in}}%
\pgfpathlineto{\pgfqpoint{5.620384in}{0.560258in}}%
\pgfpathlineto{\pgfqpoint{5.620384in}{0.550000in}}%
\pgfpathclose%
\pgfusepath{fill}%
\end{pgfscope}%
\begin{pgfscope}%
\pgfpathrectangle{\pgfqpoint{1.250000in}{0.550000in}}{\pgfqpoint{7.750000in}{3.850000in}}%
\pgfusepath{clip}%
\pgfsetbuttcap%
\pgfsetmiterjoin%
\definecolor{currentfill}{rgb}{0.501961,0.501961,0.501961}%
\pgfsetfillcolor{currentfill}%
\pgfsetfillopacity{0.700000}%
\pgfsetlinewidth{0.000000pt}%
\definecolor{currentstroke}{rgb}{0.000000,0.000000,0.000000}%
\pgfsetstrokecolor{currentstroke}%
\pgfsetstrokeopacity{0.700000}%
\pgfsetdash{}{0pt}%
\pgfpathmoveto{\pgfqpoint{5.647905in}{0.550000in}}%
\pgfpathlineto{\pgfqpoint{5.675426in}{0.550000in}}%
\pgfpathlineto{\pgfqpoint{5.675426in}{0.559799in}}%
\pgfpathlineto{\pgfqpoint{5.647905in}{0.559799in}}%
\pgfpathlineto{\pgfqpoint{5.647905in}{0.550000in}}%
\pgfpathclose%
\pgfusepath{fill}%
\end{pgfscope}%
\begin{pgfscope}%
\pgfpathrectangle{\pgfqpoint{1.250000in}{0.550000in}}{\pgfqpoint{7.750000in}{3.850000in}}%
\pgfusepath{clip}%
\pgfsetbuttcap%
\pgfsetmiterjoin%
\definecolor{currentfill}{rgb}{0.501961,0.501961,0.501961}%
\pgfsetfillcolor{currentfill}%
\pgfsetfillopacity{0.700000}%
\pgfsetlinewidth{0.000000pt}%
\definecolor{currentstroke}{rgb}{0.000000,0.000000,0.000000}%
\pgfsetstrokecolor{currentstroke}%
\pgfsetstrokeopacity{0.700000}%
\pgfsetdash{}{0pt}%
\pgfpathmoveto{\pgfqpoint{5.675426in}{0.550000in}}%
\pgfpathlineto{\pgfqpoint{5.702947in}{0.550000in}}%
\pgfpathlineto{\pgfqpoint{5.702947in}{0.559493in}}%
\pgfpathlineto{\pgfqpoint{5.675426in}{0.559493in}}%
\pgfpathlineto{\pgfqpoint{5.675426in}{0.550000in}}%
\pgfpathclose%
\pgfusepath{fill}%
\end{pgfscope}%
\begin{pgfscope}%
\pgfpathrectangle{\pgfqpoint{1.250000in}{0.550000in}}{\pgfqpoint{7.750000in}{3.850000in}}%
\pgfusepath{clip}%
\pgfsetbuttcap%
\pgfsetmiterjoin%
\definecolor{currentfill}{rgb}{0.501961,0.501961,0.501961}%
\pgfsetfillcolor{currentfill}%
\pgfsetfillopacity{0.700000}%
\pgfsetlinewidth{0.000000pt}%
\definecolor{currentstroke}{rgb}{0.000000,0.000000,0.000000}%
\pgfsetstrokecolor{currentstroke}%
\pgfsetstrokeopacity{0.700000}%
\pgfsetdash{}{0pt}%
\pgfpathmoveto{\pgfqpoint{5.702947in}{0.550000in}}%
\pgfpathlineto{\pgfqpoint{5.730469in}{0.550000in}}%
\pgfpathlineto{\pgfqpoint{5.730469in}{0.559033in}}%
\pgfpathlineto{\pgfqpoint{5.702947in}{0.559033in}}%
\pgfpathlineto{\pgfqpoint{5.702947in}{0.550000in}}%
\pgfpathclose%
\pgfusepath{fill}%
\end{pgfscope}%
\begin{pgfscope}%
\pgfpathrectangle{\pgfqpoint{1.250000in}{0.550000in}}{\pgfqpoint{7.750000in}{3.850000in}}%
\pgfusepath{clip}%
\pgfsetbuttcap%
\pgfsetmiterjoin%
\definecolor{currentfill}{rgb}{0.501961,0.501961,0.501961}%
\pgfsetfillcolor{currentfill}%
\pgfsetfillopacity{0.700000}%
\pgfsetlinewidth{0.000000pt}%
\definecolor{currentstroke}{rgb}{0.000000,0.000000,0.000000}%
\pgfsetstrokecolor{currentstroke}%
\pgfsetstrokeopacity{0.700000}%
\pgfsetdash{}{0pt}%
\pgfpathmoveto{\pgfqpoint{5.730469in}{0.550000in}}%
\pgfpathlineto{\pgfqpoint{5.757990in}{0.550000in}}%
\pgfpathlineto{\pgfqpoint{5.757990in}{0.556124in}}%
\pgfpathlineto{\pgfqpoint{5.730469in}{0.556124in}}%
\pgfpathlineto{\pgfqpoint{5.730469in}{0.550000in}}%
\pgfpathclose%
\pgfusepath{fill}%
\end{pgfscope}%
\begin{pgfscope}%
\pgfpathrectangle{\pgfqpoint{1.250000in}{0.550000in}}{\pgfqpoint{7.750000in}{3.850000in}}%
\pgfusepath{clip}%
\pgfsetbuttcap%
\pgfsetmiterjoin%
\definecolor{currentfill}{rgb}{0.501961,0.501961,0.501961}%
\pgfsetfillcolor{currentfill}%
\pgfsetfillopacity{0.700000}%
\pgfsetlinewidth{0.000000pt}%
\definecolor{currentstroke}{rgb}{0.000000,0.000000,0.000000}%
\pgfsetstrokecolor{currentstroke}%
\pgfsetstrokeopacity{0.700000}%
\pgfsetdash{}{0pt}%
\pgfpathmoveto{\pgfqpoint{5.757990in}{0.550000in}}%
\pgfpathlineto{\pgfqpoint{5.785511in}{0.550000in}}%
\pgfpathlineto{\pgfqpoint{5.785511in}{0.557502in}}%
\pgfpathlineto{\pgfqpoint{5.757990in}{0.557502in}}%
\pgfpathlineto{\pgfqpoint{5.757990in}{0.550000in}}%
\pgfpathclose%
\pgfusepath{fill}%
\end{pgfscope}%
\begin{pgfscope}%
\pgfpathrectangle{\pgfqpoint{1.250000in}{0.550000in}}{\pgfqpoint{7.750000in}{3.850000in}}%
\pgfusepath{clip}%
\pgfsetbuttcap%
\pgfsetmiterjoin%
\definecolor{currentfill}{rgb}{0.501961,0.501961,0.501961}%
\pgfsetfillcolor{currentfill}%
\pgfsetfillopacity{0.700000}%
\pgfsetlinewidth{0.000000pt}%
\definecolor{currentstroke}{rgb}{0.000000,0.000000,0.000000}%
\pgfsetstrokecolor{currentstroke}%
\pgfsetstrokeopacity{0.700000}%
\pgfsetdash{}{0pt}%
\pgfpathmoveto{\pgfqpoint{5.785511in}{0.550000in}}%
\pgfpathlineto{\pgfqpoint{5.813033in}{0.550000in}}%
\pgfpathlineto{\pgfqpoint{5.813033in}{0.556890in}}%
\pgfpathlineto{\pgfqpoint{5.785511in}{0.556890in}}%
\pgfpathlineto{\pgfqpoint{5.785511in}{0.550000in}}%
\pgfpathclose%
\pgfusepath{fill}%
\end{pgfscope}%
\begin{pgfscope}%
\pgfpathrectangle{\pgfqpoint{1.250000in}{0.550000in}}{\pgfqpoint{7.750000in}{3.850000in}}%
\pgfusepath{clip}%
\pgfsetbuttcap%
\pgfsetmiterjoin%
\definecolor{currentfill}{rgb}{0.501961,0.501961,0.501961}%
\pgfsetfillcolor{currentfill}%
\pgfsetfillopacity{0.700000}%
\pgfsetlinewidth{0.000000pt}%
\definecolor{currentstroke}{rgb}{0.000000,0.000000,0.000000}%
\pgfsetstrokecolor{currentstroke}%
\pgfsetstrokeopacity{0.700000}%
\pgfsetdash{}{0pt}%
\pgfpathmoveto{\pgfqpoint{5.813033in}{0.550000in}}%
\pgfpathlineto{\pgfqpoint{5.840554in}{0.550000in}}%
\pgfpathlineto{\pgfqpoint{5.840554in}{0.558574in}}%
\pgfpathlineto{\pgfqpoint{5.813033in}{0.558574in}}%
\pgfpathlineto{\pgfqpoint{5.813033in}{0.550000in}}%
\pgfpathclose%
\pgfusepath{fill}%
\end{pgfscope}%
\begin{pgfscope}%
\pgfpathrectangle{\pgfqpoint{1.250000in}{0.550000in}}{\pgfqpoint{7.750000in}{3.850000in}}%
\pgfusepath{clip}%
\pgfsetbuttcap%
\pgfsetmiterjoin%
\definecolor{currentfill}{rgb}{0.501961,0.501961,0.501961}%
\pgfsetfillcolor{currentfill}%
\pgfsetfillopacity{0.700000}%
\pgfsetlinewidth{0.000000pt}%
\definecolor{currentstroke}{rgb}{0.000000,0.000000,0.000000}%
\pgfsetstrokecolor{currentstroke}%
\pgfsetstrokeopacity{0.700000}%
\pgfsetdash{}{0pt}%
\pgfpathmoveto{\pgfqpoint{5.840554in}{0.550000in}}%
\pgfpathlineto{\pgfqpoint{5.868075in}{0.550000in}}%
\pgfpathlineto{\pgfqpoint{5.868075in}{0.556431in}}%
\pgfpathlineto{\pgfqpoint{5.840554in}{0.556431in}}%
\pgfpathlineto{\pgfqpoint{5.840554in}{0.550000in}}%
\pgfpathclose%
\pgfusepath{fill}%
\end{pgfscope}%
\begin{pgfscope}%
\pgfpathrectangle{\pgfqpoint{1.250000in}{0.550000in}}{\pgfqpoint{7.750000in}{3.850000in}}%
\pgfusepath{clip}%
\pgfsetbuttcap%
\pgfsetmiterjoin%
\definecolor{currentfill}{rgb}{0.501961,0.501961,0.501961}%
\pgfsetfillcolor{currentfill}%
\pgfsetfillopacity{0.700000}%
\pgfsetlinewidth{0.000000pt}%
\definecolor{currentstroke}{rgb}{0.000000,0.000000,0.000000}%
\pgfsetstrokecolor{currentstroke}%
\pgfsetstrokeopacity{0.700000}%
\pgfsetdash{}{0pt}%
\pgfpathmoveto{\pgfqpoint{5.868075in}{0.550000in}}%
\pgfpathlineto{\pgfqpoint{5.895597in}{0.550000in}}%
\pgfpathlineto{\pgfqpoint{5.895597in}{0.558115in}}%
\pgfpathlineto{\pgfqpoint{5.868075in}{0.558115in}}%
\pgfpathlineto{\pgfqpoint{5.868075in}{0.550000in}}%
\pgfpathclose%
\pgfusepath{fill}%
\end{pgfscope}%
\begin{pgfscope}%
\pgfpathrectangle{\pgfqpoint{1.250000in}{0.550000in}}{\pgfqpoint{7.750000in}{3.850000in}}%
\pgfusepath{clip}%
\pgfsetbuttcap%
\pgfsetmiterjoin%
\definecolor{currentfill}{rgb}{0.501961,0.501961,0.501961}%
\pgfsetfillcolor{currentfill}%
\pgfsetfillopacity{0.700000}%
\pgfsetlinewidth{0.000000pt}%
\definecolor{currentstroke}{rgb}{0.000000,0.000000,0.000000}%
\pgfsetstrokecolor{currentstroke}%
\pgfsetstrokeopacity{0.700000}%
\pgfsetdash{}{0pt}%
\pgfpathmoveto{\pgfqpoint{5.895597in}{0.550000in}}%
\pgfpathlineto{\pgfqpoint{5.923118in}{0.550000in}}%
\pgfpathlineto{\pgfqpoint{5.923118in}{0.557502in}}%
\pgfpathlineto{\pgfqpoint{5.895597in}{0.557502in}}%
\pgfpathlineto{\pgfqpoint{5.895597in}{0.550000in}}%
\pgfpathclose%
\pgfusepath{fill}%
\end{pgfscope}%
\begin{pgfscope}%
\pgfpathrectangle{\pgfqpoint{1.250000in}{0.550000in}}{\pgfqpoint{7.750000in}{3.850000in}}%
\pgfusepath{clip}%
\pgfsetbuttcap%
\pgfsetmiterjoin%
\definecolor{currentfill}{rgb}{0.501961,0.501961,0.501961}%
\pgfsetfillcolor{currentfill}%
\pgfsetfillopacity{0.700000}%
\pgfsetlinewidth{0.000000pt}%
\definecolor{currentstroke}{rgb}{0.000000,0.000000,0.000000}%
\pgfsetstrokecolor{currentstroke}%
\pgfsetstrokeopacity{0.700000}%
\pgfsetdash{}{0pt}%
\pgfpathmoveto{\pgfqpoint{5.923118in}{0.550000in}}%
\pgfpathlineto{\pgfqpoint{5.950639in}{0.550000in}}%
\pgfpathlineto{\pgfqpoint{5.950639in}{0.556584in}}%
\pgfpathlineto{\pgfqpoint{5.923118in}{0.556584in}}%
\pgfpathlineto{\pgfqpoint{5.923118in}{0.550000in}}%
\pgfpathclose%
\pgfusepath{fill}%
\end{pgfscope}%
\begin{pgfscope}%
\pgfpathrectangle{\pgfqpoint{1.250000in}{0.550000in}}{\pgfqpoint{7.750000in}{3.850000in}}%
\pgfusepath{clip}%
\pgfsetbuttcap%
\pgfsetmiterjoin%
\definecolor{currentfill}{rgb}{0.501961,0.501961,0.501961}%
\pgfsetfillcolor{currentfill}%
\pgfsetfillopacity{0.700000}%
\pgfsetlinewidth{0.000000pt}%
\definecolor{currentstroke}{rgb}{0.000000,0.000000,0.000000}%
\pgfsetstrokecolor{currentstroke}%
\pgfsetstrokeopacity{0.700000}%
\pgfsetdash{}{0pt}%
\pgfpathmoveto{\pgfqpoint{5.950639in}{0.550000in}}%
\pgfpathlineto{\pgfqpoint{5.978161in}{0.550000in}}%
\pgfpathlineto{\pgfqpoint{5.978161in}{0.555818in}}%
\pgfpathlineto{\pgfqpoint{5.950639in}{0.555818in}}%
\pgfpathlineto{\pgfqpoint{5.950639in}{0.550000in}}%
\pgfpathclose%
\pgfusepath{fill}%
\end{pgfscope}%
\begin{pgfscope}%
\pgfpathrectangle{\pgfqpoint{1.250000in}{0.550000in}}{\pgfqpoint{7.750000in}{3.850000in}}%
\pgfusepath{clip}%
\pgfsetbuttcap%
\pgfsetmiterjoin%
\definecolor{currentfill}{rgb}{0.501961,0.501961,0.501961}%
\pgfsetfillcolor{currentfill}%
\pgfsetfillopacity{0.700000}%
\pgfsetlinewidth{0.000000pt}%
\definecolor{currentstroke}{rgb}{0.000000,0.000000,0.000000}%
\pgfsetstrokecolor{currentstroke}%
\pgfsetstrokeopacity{0.700000}%
\pgfsetdash{}{0pt}%
\pgfpathmoveto{\pgfqpoint{5.978161in}{0.550000in}}%
\pgfpathlineto{\pgfqpoint{6.005682in}{0.550000in}}%
\pgfpathlineto{\pgfqpoint{6.005682in}{0.556890in}}%
\pgfpathlineto{\pgfqpoint{5.978161in}{0.556890in}}%
\pgfpathlineto{\pgfqpoint{5.978161in}{0.550000in}}%
\pgfpathclose%
\pgfusepath{fill}%
\end{pgfscope}%
\begin{pgfscope}%
\pgfpathrectangle{\pgfqpoint{1.250000in}{0.550000in}}{\pgfqpoint{7.750000in}{3.850000in}}%
\pgfusepath{clip}%
\pgfsetbuttcap%
\pgfsetmiterjoin%
\definecolor{currentfill}{rgb}{0.501961,0.501961,0.501961}%
\pgfsetfillcolor{currentfill}%
\pgfsetfillopacity{0.700000}%
\pgfsetlinewidth{0.000000pt}%
\definecolor{currentstroke}{rgb}{0.000000,0.000000,0.000000}%
\pgfsetstrokecolor{currentstroke}%
\pgfsetstrokeopacity{0.700000}%
\pgfsetdash{}{0pt}%
\pgfpathmoveto{\pgfqpoint{6.005682in}{0.550000in}}%
\pgfpathlineto{\pgfqpoint{6.033203in}{0.550000in}}%
\pgfpathlineto{\pgfqpoint{6.033203in}{0.555206in}}%
\pgfpathlineto{\pgfqpoint{6.005682in}{0.555206in}}%
\pgfpathlineto{\pgfqpoint{6.005682in}{0.550000in}}%
\pgfpathclose%
\pgfusepath{fill}%
\end{pgfscope}%
\begin{pgfscope}%
\pgfpathrectangle{\pgfqpoint{1.250000in}{0.550000in}}{\pgfqpoint{7.750000in}{3.850000in}}%
\pgfusepath{clip}%
\pgfsetbuttcap%
\pgfsetmiterjoin%
\definecolor{currentfill}{rgb}{0.501961,0.501961,0.501961}%
\pgfsetfillcolor{currentfill}%
\pgfsetfillopacity{0.700000}%
\pgfsetlinewidth{0.000000pt}%
\definecolor{currentstroke}{rgb}{0.000000,0.000000,0.000000}%
\pgfsetstrokecolor{currentstroke}%
\pgfsetstrokeopacity{0.700000}%
\pgfsetdash{}{0pt}%
\pgfpathmoveto{\pgfqpoint{6.033203in}{0.550000in}}%
\pgfpathlineto{\pgfqpoint{6.060724in}{0.550000in}}%
\pgfpathlineto{\pgfqpoint{6.060724in}{0.556584in}}%
\pgfpathlineto{\pgfqpoint{6.033203in}{0.556584in}}%
\pgfpathlineto{\pgfqpoint{6.033203in}{0.550000in}}%
\pgfpathclose%
\pgfusepath{fill}%
\end{pgfscope}%
\begin{pgfscope}%
\pgfpathrectangle{\pgfqpoint{1.250000in}{0.550000in}}{\pgfqpoint{7.750000in}{3.850000in}}%
\pgfusepath{clip}%
\pgfsetbuttcap%
\pgfsetmiterjoin%
\definecolor{currentfill}{rgb}{0.501961,0.501961,0.501961}%
\pgfsetfillcolor{currentfill}%
\pgfsetfillopacity{0.700000}%
\pgfsetlinewidth{0.000000pt}%
\definecolor{currentstroke}{rgb}{0.000000,0.000000,0.000000}%
\pgfsetstrokecolor{currentstroke}%
\pgfsetstrokeopacity{0.700000}%
\pgfsetdash{}{0pt}%
\pgfpathmoveto{\pgfqpoint{6.060724in}{0.550000in}}%
\pgfpathlineto{\pgfqpoint{6.088246in}{0.550000in}}%
\pgfpathlineto{\pgfqpoint{6.088246in}{0.557502in}}%
\pgfpathlineto{\pgfqpoint{6.060724in}{0.557502in}}%
\pgfpathlineto{\pgfqpoint{6.060724in}{0.550000in}}%
\pgfpathclose%
\pgfusepath{fill}%
\end{pgfscope}%
\begin{pgfscope}%
\pgfpathrectangle{\pgfqpoint{1.250000in}{0.550000in}}{\pgfqpoint{7.750000in}{3.850000in}}%
\pgfusepath{clip}%
\pgfsetbuttcap%
\pgfsetmiterjoin%
\definecolor{currentfill}{rgb}{0.501961,0.501961,0.501961}%
\pgfsetfillcolor{currentfill}%
\pgfsetfillopacity{0.700000}%
\pgfsetlinewidth{0.000000pt}%
\definecolor{currentstroke}{rgb}{0.000000,0.000000,0.000000}%
\pgfsetstrokecolor{currentstroke}%
\pgfsetstrokeopacity{0.700000}%
\pgfsetdash{}{0pt}%
\pgfpathmoveto{\pgfqpoint{6.088246in}{0.550000in}}%
\pgfpathlineto{\pgfqpoint{6.115767in}{0.550000in}}%
\pgfpathlineto{\pgfqpoint{6.115767in}{0.555053in}}%
\pgfpathlineto{\pgfqpoint{6.088246in}{0.555053in}}%
\pgfpathlineto{\pgfqpoint{6.088246in}{0.550000in}}%
\pgfpathclose%
\pgfusepath{fill}%
\end{pgfscope}%
\begin{pgfscope}%
\pgfpathrectangle{\pgfqpoint{1.250000in}{0.550000in}}{\pgfqpoint{7.750000in}{3.850000in}}%
\pgfusepath{clip}%
\pgfsetbuttcap%
\pgfsetmiterjoin%
\definecolor{currentfill}{rgb}{0.501961,0.501961,0.501961}%
\pgfsetfillcolor{currentfill}%
\pgfsetfillopacity{0.700000}%
\pgfsetlinewidth{0.000000pt}%
\definecolor{currentstroke}{rgb}{0.000000,0.000000,0.000000}%
\pgfsetstrokecolor{currentstroke}%
\pgfsetstrokeopacity{0.700000}%
\pgfsetdash{}{0pt}%
\pgfpathmoveto{\pgfqpoint{6.115767in}{0.550000in}}%
\pgfpathlineto{\pgfqpoint{6.143288in}{0.550000in}}%
\pgfpathlineto{\pgfqpoint{6.143288in}{0.556890in}}%
\pgfpathlineto{\pgfqpoint{6.115767in}{0.556890in}}%
\pgfpathlineto{\pgfqpoint{6.115767in}{0.550000in}}%
\pgfpathclose%
\pgfusepath{fill}%
\end{pgfscope}%
\begin{pgfscope}%
\pgfpathrectangle{\pgfqpoint{1.250000in}{0.550000in}}{\pgfqpoint{7.750000in}{3.850000in}}%
\pgfusepath{clip}%
\pgfsetbuttcap%
\pgfsetmiterjoin%
\definecolor{currentfill}{rgb}{0.501961,0.501961,0.501961}%
\pgfsetfillcolor{currentfill}%
\pgfsetfillopacity{0.700000}%
\pgfsetlinewidth{0.000000pt}%
\definecolor{currentstroke}{rgb}{0.000000,0.000000,0.000000}%
\pgfsetstrokecolor{currentstroke}%
\pgfsetstrokeopacity{0.700000}%
\pgfsetdash{}{0pt}%
\pgfpathmoveto{\pgfqpoint{6.143288in}{0.550000in}}%
\pgfpathlineto{\pgfqpoint{6.170810in}{0.550000in}}%
\pgfpathlineto{\pgfqpoint{6.170810in}{0.553981in}}%
\pgfpathlineto{\pgfqpoint{6.143288in}{0.553981in}}%
\pgfpathlineto{\pgfqpoint{6.143288in}{0.550000in}}%
\pgfpathclose%
\pgfusepath{fill}%
\end{pgfscope}%
\begin{pgfscope}%
\pgfpathrectangle{\pgfqpoint{1.250000in}{0.550000in}}{\pgfqpoint{7.750000in}{3.850000in}}%
\pgfusepath{clip}%
\pgfsetbuttcap%
\pgfsetmiterjoin%
\definecolor{currentfill}{rgb}{0.501961,0.501961,0.501961}%
\pgfsetfillcolor{currentfill}%
\pgfsetfillopacity{0.700000}%
\pgfsetlinewidth{0.000000pt}%
\definecolor{currentstroke}{rgb}{0.000000,0.000000,0.000000}%
\pgfsetstrokecolor{currentstroke}%
\pgfsetstrokeopacity{0.700000}%
\pgfsetdash{}{0pt}%
\pgfpathmoveto{\pgfqpoint{6.170810in}{0.550000in}}%
\pgfpathlineto{\pgfqpoint{6.198331in}{0.550000in}}%
\pgfpathlineto{\pgfqpoint{6.198331in}{0.554900in}}%
\pgfpathlineto{\pgfqpoint{6.170810in}{0.554900in}}%
\pgfpathlineto{\pgfqpoint{6.170810in}{0.550000in}}%
\pgfpathclose%
\pgfusepath{fill}%
\end{pgfscope}%
\begin{pgfscope}%
\pgfpathrectangle{\pgfqpoint{1.250000in}{0.550000in}}{\pgfqpoint{7.750000in}{3.850000in}}%
\pgfusepath{clip}%
\pgfsetbuttcap%
\pgfsetmiterjoin%
\definecolor{currentfill}{rgb}{0.501961,0.501961,0.501961}%
\pgfsetfillcolor{currentfill}%
\pgfsetfillopacity{0.700000}%
\pgfsetlinewidth{0.000000pt}%
\definecolor{currentstroke}{rgb}{0.000000,0.000000,0.000000}%
\pgfsetstrokecolor{currentstroke}%
\pgfsetstrokeopacity{0.700000}%
\pgfsetdash{}{0pt}%
\pgfpathmoveto{\pgfqpoint{6.198331in}{0.550000in}}%
\pgfpathlineto{\pgfqpoint{6.225852in}{0.550000in}}%
\pgfpathlineto{\pgfqpoint{6.225852in}{0.555359in}}%
\pgfpathlineto{\pgfqpoint{6.198331in}{0.555359in}}%
\pgfpathlineto{\pgfqpoint{6.198331in}{0.550000in}}%
\pgfpathclose%
\pgfusepath{fill}%
\end{pgfscope}%
\begin{pgfscope}%
\pgfpathrectangle{\pgfqpoint{1.250000in}{0.550000in}}{\pgfqpoint{7.750000in}{3.850000in}}%
\pgfusepath{clip}%
\pgfsetbuttcap%
\pgfsetmiterjoin%
\definecolor{currentfill}{rgb}{0.501961,0.501961,0.501961}%
\pgfsetfillcolor{currentfill}%
\pgfsetfillopacity{0.700000}%
\pgfsetlinewidth{0.000000pt}%
\definecolor{currentstroke}{rgb}{0.000000,0.000000,0.000000}%
\pgfsetstrokecolor{currentstroke}%
\pgfsetstrokeopacity{0.700000}%
\pgfsetdash{}{0pt}%
\pgfpathmoveto{\pgfqpoint{6.225852in}{0.550000in}}%
\pgfpathlineto{\pgfqpoint{6.253374in}{0.550000in}}%
\pgfpathlineto{\pgfqpoint{6.253374in}{0.557043in}}%
\pgfpathlineto{\pgfqpoint{6.225852in}{0.557043in}}%
\pgfpathlineto{\pgfqpoint{6.225852in}{0.550000in}}%
\pgfpathclose%
\pgfusepath{fill}%
\end{pgfscope}%
\begin{pgfscope}%
\pgfpathrectangle{\pgfqpoint{1.250000in}{0.550000in}}{\pgfqpoint{7.750000in}{3.850000in}}%
\pgfusepath{clip}%
\pgfsetbuttcap%
\pgfsetmiterjoin%
\definecolor{currentfill}{rgb}{0.501961,0.501961,0.501961}%
\pgfsetfillcolor{currentfill}%
\pgfsetfillopacity{0.700000}%
\pgfsetlinewidth{0.000000pt}%
\definecolor{currentstroke}{rgb}{0.000000,0.000000,0.000000}%
\pgfsetstrokecolor{currentstroke}%
\pgfsetstrokeopacity{0.700000}%
\pgfsetdash{}{0pt}%
\pgfpathmoveto{\pgfqpoint{6.253374in}{0.550000in}}%
\pgfpathlineto{\pgfqpoint{6.280895in}{0.550000in}}%
\pgfpathlineto{\pgfqpoint{6.280895in}{0.555512in}}%
\pgfpathlineto{\pgfqpoint{6.253374in}{0.555512in}}%
\pgfpathlineto{\pgfqpoint{6.253374in}{0.550000in}}%
\pgfpathclose%
\pgfusepath{fill}%
\end{pgfscope}%
\begin{pgfscope}%
\pgfpathrectangle{\pgfqpoint{1.250000in}{0.550000in}}{\pgfqpoint{7.750000in}{3.850000in}}%
\pgfusepath{clip}%
\pgfsetbuttcap%
\pgfsetmiterjoin%
\definecolor{currentfill}{rgb}{0.501961,0.501961,0.501961}%
\pgfsetfillcolor{currentfill}%
\pgfsetfillopacity{0.700000}%
\pgfsetlinewidth{0.000000pt}%
\definecolor{currentstroke}{rgb}{0.000000,0.000000,0.000000}%
\pgfsetstrokecolor{currentstroke}%
\pgfsetstrokeopacity{0.700000}%
\pgfsetdash{}{0pt}%
\pgfpathmoveto{\pgfqpoint{6.280895in}{0.550000in}}%
\pgfpathlineto{\pgfqpoint{6.308416in}{0.550000in}}%
\pgfpathlineto{\pgfqpoint{6.308416in}{0.554746in}}%
\pgfpathlineto{\pgfqpoint{6.280895in}{0.554746in}}%
\pgfpathlineto{\pgfqpoint{6.280895in}{0.550000in}}%
\pgfpathclose%
\pgfusepath{fill}%
\end{pgfscope}%
\begin{pgfscope}%
\pgfpathrectangle{\pgfqpoint{1.250000in}{0.550000in}}{\pgfqpoint{7.750000in}{3.850000in}}%
\pgfusepath{clip}%
\pgfsetbuttcap%
\pgfsetmiterjoin%
\definecolor{currentfill}{rgb}{0.501961,0.501961,0.501961}%
\pgfsetfillcolor{currentfill}%
\pgfsetfillopacity{0.700000}%
\pgfsetlinewidth{0.000000pt}%
\definecolor{currentstroke}{rgb}{0.000000,0.000000,0.000000}%
\pgfsetstrokecolor{currentstroke}%
\pgfsetstrokeopacity{0.700000}%
\pgfsetdash{}{0pt}%
\pgfpathmoveto{\pgfqpoint{6.308416in}{0.550000in}}%
\pgfpathlineto{\pgfqpoint{6.335937in}{0.550000in}}%
\pgfpathlineto{\pgfqpoint{6.335937in}{0.553675in}}%
\pgfpathlineto{\pgfqpoint{6.308416in}{0.553675in}}%
\pgfpathlineto{\pgfqpoint{6.308416in}{0.550000in}}%
\pgfpathclose%
\pgfusepath{fill}%
\end{pgfscope}%
\begin{pgfscope}%
\pgfpathrectangle{\pgfqpoint{1.250000in}{0.550000in}}{\pgfqpoint{7.750000in}{3.850000in}}%
\pgfusepath{clip}%
\pgfsetbuttcap%
\pgfsetmiterjoin%
\definecolor{currentfill}{rgb}{0.501961,0.501961,0.501961}%
\pgfsetfillcolor{currentfill}%
\pgfsetfillopacity{0.700000}%
\pgfsetlinewidth{0.000000pt}%
\definecolor{currentstroke}{rgb}{0.000000,0.000000,0.000000}%
\pgfsetstrokecolor{currentstroke}%
\pgfsetstrokeopacity{0.700000}%
\pgfsetdash{}{0pt}%
\pgfpathmoveto{\pgfqpoint{6.335938in}{0.550000in}}%
\pgfpathlineto{\pgfqpoint{6.363459in}{0.550000in}}%
\pgfpathlineto{\pgfqpoint{6.363459in}{0.553368in}}%
\pgfpathlineto{\pgfqpoint{6.335938in}{0.553368in}}%
\pgfpathlineto{\pgfqpoint{6.335938in}{0.550000in}}%
\pgfpathclose%
\pgfusepath{fill}%
\end{pgfscope}%
\begin{pgfscope}%
\pgfpathrectangle{\pgfqpoint{1.250000in}{0.550000in}}{\pgfqpoint{7.750000in}{3.850000in}}%
\pgfusepath{clip}%
\pgfsetbuttcap%
\pgfsetmiterjoin%
\definecolor{currentfill}{rgb}{0.501961,0.501961,0.501961}%
\pgfsetfillcolor{currentfill}%
\pgfsetfillopacity{0.700000}%
\pgfsetlinewidth{0.000000pt}%
\definecolor{currentstroke}{rgb}{0.000000,0.000000,0.000000}%
\pgfsetstrokecolor{currentstroke}%
\pgfsetstrokeopacity{0.700000}%
\pgfsetdash{}{0pt}%
\pgfpathmoveto{\pgfqpoint{6.363459in}{0.550000in}}%
\pgfpathlineto{\pgfqpoint{6.390980in}{0.550000in}}%
\pgfpathlineto{\pgfqpoint{6.390980in}{0.554746in}}%
\pgfpathlineto{\pgfqpoint{6.363459in}{0.554746in}}%
\pgfpathlineto{\pgfqpoint{6.363459in}{0.550000in}}%
\pgfpathclose%
\pgfusepath{fill}%
\end{pgfscope}%
\begin{pgfscope}%
\pgfpathrectangle{\pgfqpoint{1.250000in}{0.550000in}}{\pgfqpoint{7.750000in}{3.850000in}}%
\pgfusepath{clip}%
\pgfsetbuttcap%
\pgfsetmiterjoin%
\definecolor{currentfill}{rgb}{0.501961,0.501961,0.501961}%
\pgfsetfillcolor{currentfill}%
\pgfsetfillopacity{0.700000}%
\pgfsetlinewidth{0.000000pt}%
\definecolor{currentstroke}{rgb}{0.000000,0.000000,0.000000}%
\pgfsetstrokecolor{currentstroke}%
\pgfsetstrokeopacity{0.700000}%
\pgfsetdash{}{0pt}%
\pgfpathmoveto{\pgfqpoint{6.390980in}{0.550000in}}%
\pgfpathlineto{\pgfqpoint{6.418501in}{0.550000in}}%
\pgfpathlineto{\pgfqpoint{6.418501in}{0.554746in}}%
\pgfpathlineto{\pgfqpoint{6.390980in}{0.554746in}}%
\pgfpathlineto{\pgfqpoint{6.390980in}{0.550000in}}%
\pgfpathclose%
\pgfusepath{fill}%
\end{pgfscope}%
\begin{pgfscope}%
\pgfpathrectangle{\pgfqpoint{1.250000in}{0.550000in}}{\pgfqpoint{7.750000in}{3.850000in}}%
\pgfusepath{clip}%
\pgfsetbuttcap%
\pgfsetmiterjoin%
\definecolor{currentfill}{rgb}{0.501961,0.501961,0.501961}%
\pgfsetfillcolor{currentfill}%
\pgfsetfillopacity{0.700000}%
\pgfsetlinewidth{0.000000pt}%
\definecolor{currentstroke}{rgb}{0.000000,0.000000,0.000000}%
\pgfsetstrokecolor{currentstroke}%
\pgfsetstrokeopacity{0.700000}%
\pgfsetdash{}{0pt}%
\pgfpathmoveto{\pgfqpoint{6.418501in}{0.550000in}}%
\pgfpathlineto{\pgfqpoint{6.446023in}{0.550000in}}%
\pgfpathlineto{\pgfqpoint{6.446023in}{0.555818in}}%
\pgfpathlineto{\pgfqpoint{6.418501in}{0.555818in}}%
\pgfpathlineto{\pgfqpoint{6.418501in}{0.550000in}}%
\pgfpathclose%
\pgfusepath{fill}%
\end{pgfscope}%
\begin{pgfscope}%
\pgfpathrectangle{\pgfqpoint{1.250000in}{0.550000in}}{\pgfqpoint{7.750000in}{3.850000in}}%
\pgfusepath{clip}%
\pgfsetbuttcap%
\pgfsetmiterjoin%
\definecolor{currentfill}{rgb}{0.501961,0.501961,0.501961}%
\pgfsetfillcolor{currentfill}%
\pgfsetfillopacity{0.700000}%
\pgfsetlinewidth{0.000000pt}%
\definecolor{currentstroke}{rgb}{0.000000,0.000000,0.000000}%
\pgfsetstrokecolor{currentstroke}%
\pgfsetstrokeopacity{0.700000}%
\pgfsetdash{}{0pt}%
\pgfpathmoveto{\pgfqpoint{6.446023in}{0.550000in}}%
\pgfpathlineto{\pgfqpoint{6.473544in}{0.550000in}}%
\pgfpathlineto{\pgfqpoint{6.473544in}{0.552756in}}%
\pgfpathlineto{\pgfqpoint{6.446023in}{0.552756in}}%
\pgfpathlineto{\pgfqpoint{6.446023in}{0.550000in}}%
\pgfpathclose%
\pgfusepath{fill}%
\end{pgfscope}%
\begin{pgfscope}%
\pgfpathrectangle{\pgfqpoint{1.250000in}{0.550000in}}{\pgfqpoint{7.750000in}{3.850000in}}%
\pgfusepath{clip}%
\pgfsetbuttcap%
\pgfsetmiterjoin%
\definecolor{currentfill}{rgb}{0.501961,0.501961,0.501961}%
\pgfsetfillcolor{currentfill}%
\pgfsetfillopacity{0.700000}%
\pgfsetlinewidth{0.000000pt}%
\definecolor{currentstroke}{rgb}{0.000000,0.000000,0.000000}%
\pgfsetstrokecolor{currentstroke}%
\pgfsetstrokeopacity{0.700000}%
\pgfsetdash{}{0pt}%
\pgfpathmoveto{\pgfqpoint{6.473544in}{0.550000in}}%
\pgfpathlineto{\pgfqpoint{6.501065in}{0.550000in}}%
\pgfpathlineto{\pgfqpoint{6.501065in}{0.553675in}}%
\pgfpathlineto{\pgfqpoint{6.473544in}{0.553675in}}%
\pgfpathlineto{\pgfqpoint{6.473544in}{0.550000in}}%
\pgfpathclose%
\pgfusepath{fill}%
\end{pgfscope}%
\begin{pgfscope}%
\pgfpathrectangle{\pgfqpoint{1.250000in}{0.550000in}}{\pgfqpoint{7.750000in}{3.850000in}}%
\pgfusepath{clip}%
\pgfsetbuttcap%
\pgfsetmiterjoin%
\definecolor{currentfill}{rgb}{0.501961,0.501961,0.501961}%
\pgfsetfillcolor{currentfill}%
\pgfsetfillopacity{0.700000}%
\pgfsetlinewidth{0.000000pt}%
\definecolor{currentstroke}{rgb}{0.000000,0.000000,0.000000}%
\pgfsetstrokecolor{currentstroke}%
\pgfsetstrokeopacity{0.700000}%
\pgfsetdash{}{0pt}%
\pgfpathmoveto{\pgfqpoint{6.501065in}{0.550000in}}%
\pgfpathlineto{\pgfqpoint{6.528587in}{0.550000in}}%
\pgfpathlineto{\pgfqpoint{6.528587in}{0.554746in}}%
\pgfpathlineto{\pgfqpoint{6.501065in}{0.554746in}}%
\pgfpathlineto{\pgfqpoint{6.501065in}{0.550000in}}%
\pgfpathclose%
\pgfusepath{fill}%
\end{pgfscope}%
\begin{pgfscope}%
\pgfpathrectangle{\pgfqpoint{1.250000in}{0.550000in}}{\pgfqpoint{7.750000in}{3.850000in}}%
\pgfusepath{clip}%
\pgfsetbuttcap%
\pgfsetmiterjoin%
\definecolor{currentfill}{rgb}{0.501961,0.501961,0.501961}%
\pgfsetfillcolor{currentfill}%
\pgfsetfillopacity{0.700000}%
\pgfsetlinewidth{0.000000pt}%
\definecolor{currentstroke}{rgb}{0.000000,0.000000,0.000000}%
\pgfsetstrokecolor{currentstroke}%
\pgfsetstrokeopacity{0.700000}%
\pgfsetdash{}{0pt}%
\pgfpathmoveto{\pgfqpoint{6.528587in}{0.550000in}}%
\pgfpathlineto{\pgfqpoint{6.556108in}{0.550000in}}%
\pgfpathlineto{\pgfqpoint{6.556108in}{0.553828in}}%
\pgfpathlineto{\pgfqpoint{6.528587in}{0.553828in}}%
\pgfpathlineto{\pgfqpoint{6.528587in}{0.550000in}}%
\pgfpathclose%
\pgfusepath{fill}%
\end{pgfscope}%
\begin{pgfscope}%
\pgfpathrectangle{\pgfqpoint{1.250000in}{0.550000in}}{\pgfqpoint{7.750000in}{3.850000in}}%
\pgfusepath{clip}%
\pgfsetbuttcap%
\pgfsetmiterjoin%
\definecolor{currentfill}{rgb}{0.501961,0.501961,0.501961}%
\pgfsetfillcolor{currentfill}%
\pgfsetfillopacity{0.700000}%
\pgfsetlinewidth{0.000000pt}%
\definecolor{currentstroke}{rgb}{0.000000,0.000000,0.000000}%
\pgfsetstrokecolor{currentstroke}%
\pgfsetstrokeopacity{0.700000}%
\pgfsetdash{}{0pt}%
\pgfpathmoveto{\pgfqpoint{6.556108in}{0.550000in}}%
\pgfpathlineto{\pgfqpoint{6.583629in}{0.550000in}}%
\pgfpathlineto{\pgfqpoint{6.583629in}{0.553215in}}%
\pgfpathlineto{\pgfqpoint{6.556108in}{0.553215in}}%
\pgfpathlineto{\pgfqpoint{6.556108in}{0.550000in}}%
\pgfpathclose%
\pgfusepath{fill}%
\end{pgfscope}%
\begin{pgfscope}%
\pgfpathrectangle{\pgfqpoint{1.250000in}{0.550000in}}{\pgfqpoint{7.750000in}{3.850000in}}%
\pgfusepath{clip}%
\pgfsetbuttcap%
\pgfsetmiterjoin%
\definecolor{currentfill}{rgb}{0.501961,0.501961,0.501961}%
\pgfsetfillcolor{currentfill}%
\pgfsetfillopacity{0.700000}%
\pgfsetlinewidth{0.000000pt}%
\definecolor{currentstroke}{rgb}{0.000000,0.000000,0.000000}%
\pgfsetstrokecolor{currentstroke}%
\pgfsetstrokeopacity{0.700000}%
\pgfsetdash{}{0pt}%
\pgfpathmoveto{\pgfqpoint{6.583629in}{0.550000in}}%
\pgfpathlineto{\pgfqpoint{6.611151in}{0.550000in}}%
\pgfpathlineto{\pgfqpoint{6.611151in}{0.552603in}}%
\pgfpathlineto{\pgfqpoint{6.583629in}{0.552603in}}%
\pgfpathlineto{\pgfqpoint{6.583629in}{0.550000in}}%
\pgfpathclose%
\pgfusepath{fill}%
\end{pgfscope}%
\begin{pgfscope}%
\pgfpathrectangle{\pgfqpoint{1.250000in}{0.550000in}}{\pgfqpoint{7.750000in}{3.850000in}}%
\pgfusepath{clip}%
\pgfsetbuttcap%
\pgfsetmiterjoin%
\definecolor{currentfill}{rgb}{0.501961,0.501961,0.501961}%
\pgfsetfillcolor{currentfill}%
\pgfsetfillopacity{0.700000}%
\pgfsetlinewidth{0.000000pt}%
\definecolor{currentstroke}{rgb}{0.000000,0.000000,0.000000}%
\pgfsetstrokecolor{currentstroke}%
\pgfsetstrokeopacity{0.700000}%
\pgfsetdash{}{0pt}%
\pgfpathmoveto{\pgfqpoint{6.611151in}{0.550000in}}%
\pgfpathlineto{\pgfqpoint{6.638672in}{0.550000in}}%
\pgfpathlineto{\pgfqpoint{6.638672in}{0.552297in}}%
\pgfpathlineto{\pgfqpoint{6.611151in}{0.552297in}}%
\pgfpathlineto{\pgfqpoint{6.611151in}{0.550000in}}%
\pgfpathclose%
\pgfusepath{fill}%
\end{pgfscope}%
\begin{pgfscope}%
\pgfpathrectangle{\pgfqpoint{1.250000in}{0.550000in}}{\pgfqpoint{7.750000in}{3.850000in}}%
\pgfusepath{clip}%
\pgfsetbuttcap%
\pgfsetmiterjoin%
\definecolor{currentfill}{rgb}{0.501961,0.501961,0.501961}%
\pgfsetfillcolor{currentfill}%
\pgfsetfillopacity{0.700000}%
\pgfsetlinewidth{0.000000pt}%
\definecolor{currentstroke}{rgb}{0.000000,0.000000,0.000000}%
\pgfsetstrokecolor{currentstroke}%
\pgfsetstrokeopacity{0.700000}%
\pgfsetdash{}{0pt}%
\pgfpathmoveto{\pgfqpoint{6.638672in}{0.550000in}}%
\pgfpathlineto{\pgfqpoint{6.666193in}{0.550000in}}%
\pgfpathlineto{\pgfqpoint{6.666193in}{0.552603in}}%
\pgfpathlineto{\pgfqpoint{6.638672in}{0.552603in}}%
\pgfpathlineto{\pgfqpoint{6.638672in}{0.550000in}}%
\pgfpathclose%
\pgfusepath{fill}%
\end{pgfscope}%
\begin{pgfscope}%
\pgfpathrectangle{\pgfqpoint{1.250000in}{0.550000in}}{\pgfqpoint{7.750000in}{3.850000in}}%
\pgfusepath{clip}%
\pgfsetbuttcap%
\pgfsetmiterjoin%
\definecolor{currentfill}{rgb}{0.501961,0.501961,0.501961}%
\pgfsetfillcolor{currentfill}%
\pgfsetfillopacity{0.700000}%
\pgfsetlinewidth{0.000000pt}%
\definecolor{currentstroke}{rgb}{0.000000,0.000000,0.000000}%
\pgfsetstrokecolor{currentstroke}%
\pgfsetstrokeopacity{0.700000}%
\pgfsetdash{}{0pt}%
\pgfpathmoveto{\pgfqpoint{6.666193in}{0.550000in}}%
\pgfpathlineto{\pgfqpoint{6.693714in}{0.550000in}}%
\pgfpathlineto{\pgfqpoint{6.693714in}{0.552450in}}%
\pgfpathlineto{\pgfqpoint{6.666193in}{0.552450in}}%
\pgfpathlineto{\pgfqpoint{6.666193in}{0.550000in}}%
\pgfpathclose%
\pgfusepath{fill}%
\end{pgfscope}%
\begin{pgfscope}%
\pgfpathrectangle{\pgfqpoint{1.250000in}{0.550000in}}{\pgfqpoint{7.750000in}{3.850000in}}%
\pgfusepath{clip}%
\pgfsetbuttcap%
\pgfsetmiterjoin%
\definecolor{currentfill}{rgb}{0.501961,0.501961,0.501961}%
\pgfsetfillcolor{currentfill}%
\pgfsetfillopacity{0.700000}%
\pgfsetlinewidth{0.000000pt}%
\definecolor{currentstroke}{rgb}{0.000000,0.000000,0.000000}%
\pgfsetstrokecolor{currentstroke}%
\pgfsetstrokeopacity{0.700000}%
\pgfsetdash{}{0pt}%
\pgfpathmoveto{\pgfqpoint{6.693714in}{0.550000in}}%
\pgfpathlineto{\pgfqpoint{6.721236in}{0.550000in}}%
\pgfpathlineto{\pgfqpoint{6.721236in}{0.553981in}}%
\pgfpathlineto{\pgfqpoint{6.693714in}{0.553981in}}%
\pgfpathlineto{\pgfqpoint{6.693714in}{0.550000in}}%
\pgfpathclose%
\pgfusepath{fill}%
\end{pgfscope}%
\begin{pgfscope}%
\pgfpathrectangle{\pgfqpoint{1.250000in}{0.550000in}}{\pgfqpoint{7.750000in}{3.850000in}}%
\pgfusepath{clip}%
\pgfsetbuttcap%
\pgfsetmiterjoin%
\definecolor{currentfill}{rgb}{0.501961,0.501961,0.501961}%
\pgfsetfillcolor{currentfill}%
\pgfsetfillopacity{0.700000}%
\pgfsetlinewidth{0.000000pt}%
\definecolor{currentstroke}{rgb}{0.000000,0.000000,0.000000}%
\pgfsetstrokecolor{currentstroke}%
\pgfsetstrokeopacity{0.700000}%
\pgfsetdash{}{0pt}%
\pgfpathmoveto{\pgfqpoint{6.721236in}{0.550000in}}%
\pgfpathlineto{\pgfqpoint{6.748757in}{0.550000in}}%
\pgfpathlineto{\pgfqpoint{6.748757in}{0.553368in}}%
\pgfpathlineto{\pgfqpoint{6.721236in}{0.553368in}}%
\pgfpathlineto{\pgfqpoint{6.721236in}{0.550000in}}%
\pgfpathclose%
\pgfusepath{fill}%
\end{pgfscope}%
\begin{pgfscope}%
\pgfpathrectangle{\pgfqpoint{1.250000in}{0.550000in}}{\pgfqpoint{7.750000in}{3.850000in}}%
\pgfusepath{clip}%
\pgfsetbuttcap%
\pgfsetmiterjoin%
\definecolor{currentfill}{rgb}{0.501961,0.501961,0.501961}%
\pgfsetfillcolor{currentfill}%
\pgfsetfillopacity{0.700000}%
\pgfsetlinewidth{0.000000pt}%
\definecolor{currentstroke}{rgb}{0.000000,0.000000,0.000000}%
\pgfsetstrokecolor{currentstroke}%
\pgfsetstrokeopacity{0.700000}%
\pgfsetdash{}{0pt}%
\pgfpathmoveto{\pgfqpoint{6.748757in}{0.550000in}}%
\pgfpathlineto{\pgfqpoint{6.776278in}{0.550000in}}%
\pgfpathlineto{\pgfqpoint{6.776278in}{0.553062in}}%
\pgfpathlineto{\pgfqpoint{6.748757in}{0.553062in}}%
\pgfpathlineto{\pgfqpoint{6.748757in}{0.550000in}}%
\pgfpathclose%
\pgfusepath{fill}%
\end{pgfscope}%
\begin{pgfscope}%
\pgfpathrectangle{\pgfqpoint{1.250000in}{0.550000in}}{\pgfqpoint{7.750000in}{3.850000in}}%
\pgfusepath{clip}%
\pgfsetbuttcap%
\pgfsetmiterjoin%
\definecolor{currentfill}{rgb}{0.501961,0.501961,0.501961}%
\pgfsetfillcolor{currentfill}%
\pgfsetfillopacity{0.700000}%
\pgfsetlinewidth{0.000000pt}%
\definecolor{currentstroke}{rgb}{0.000000,0.000000,0.000000}%
\pgfsetstrokecolor{currentstroke}%
\pgfsetstrokeopacity{0.700000}%
\pgfsetdash{}{0pt}%
\pgfpathmoveto{\pgfqpoint{6.776278in}{0.550000in}}%
\pgfpathlineto{\pgfqpoint{6.803800in}{0.550000in}}%
\pgfpathlineto{\pgfqpoint{6.803800in}{0.551684in}}%
\pgfpathlineto{\pgfqpoint{6.776278in}{0.551684in}}%
\pgfpathlineto{\pgfqpoint{6.776278in}{0.550000in}}%
\pgfpathclose%
\pgfusepath{fill}%
\end{pgfscope}%
\begin{pgfscope}%
\pgfpathrectangle{\pgfqpoint{1.250000in}{0.550000in}}{\pgfqpoint{7.750000in}{3.850000in}}%
\pgfusepath{clip}%
\pgfsetbuttcap%
\pgfsetmiterjoin%
\definecolor{currentfill}{rgb}{0.501961,0.501961,0.501961}%
\pgfsetfillcolor{currentfill}%
\pgfsetfillopacity{0.700000}%
\pgfsetlinewidth{0.000000pt}%
\definecolor{currentstroke}{rgb}{0.000000,0.000000,0.000000}%
\pgfsetstrokecolor{currentstroke}%
\pgfsetstrokeopacity{0.700000}%
\pgfsetdash{}{0pt}%
\pgfpathmoveto{\pgfqpoint{6.803800in}{0.550000in}}%
\pgfpathlineto{\pgfqpoint{6.831321in}{0.550000in}}%
\pgfpathlineto{\pgfqpoint{6.831321in}{0.552450in}}%
\pgfpathlineto{\pgfqpoint{6.803800in}{0.552450in}}%
\pgfpathlineto{\pgfqpoint{6.803800in}{0.550000in}}%
\pgfpathclose%
\pgfusepath{fill}%
\end{pgfscope}%
\begin{pgfscope}%
\pgfpathrectangle{\pgfqpoint{1.250000in}{0.550000in}}{\pgfqpoint{7.750000in}{3.850000in}}%
\pgfusepath{clip}%
\pgfsetbuttcap%
\pgfsetmiterjoin%
\definecolor{currentfill}{rgb}{0.501961,0.501961,0.501961}%
\pgfsetfillcolor{currentfill}%
\pgfsetfillopacity{0.700000}%
\pgfsetlinewidth{0.000000pt}%
\definecolor{currentstroke}{rgb}{0.000000,0.000000,0.000000}%
\pgfsetstrokecolor{currentstroke}%
\pgfsetstrokeopacity{0.700000}%
\pgfsetdash{}{0pt}%
\pgfpathmoveto{\pgfqpoint{6.831321in}{0.550000in}}%
\pgfpathlineto{\pgfqpoint{6.858842in}{0.550000in}}%
\pgfpathlineto{\pgfqpoint{6.858842in}{0.551378in}}%
\pgfpathlineto{\pgfqpoint{6.831321in}{0.551378in}}%
\pgfpathlineto{\pgfqpoint{6.831321in}{0.550000in}}%
\pgfpathclose%
\pgfusepath{fill}%
\end{pgfscope}%
\begin{pgfscope}%
\pgfpathrectangle{\pgfqpoint{1.250000in}{0.550000in}}{\pgfqpoint{7.750000in}{3.850000in}}%
\pgfusepath{clip}%
\pgfsetbuttcap%
\pgfsetmiterjoin%
\definecolor{currentfill}{rgb}{0.501961,0.501961,0.501961}%
\pgfsetfillcolor{currentfill}%
\pgfsetfillopacity{0.700000}%
\pgfsetlinewidth{0.000000pt}%
\definecolor{currentstroke}{rgb}{0.000000,0.000000,0.000000}%
\pgfsetstrokecolor{currentstroke}%
\pgfsetstrokeopacity{0.700000}%
\pgfsetdash{}{0pt}%
\pgfpathmoveto{\pgfqpoint{6.858842in}{0.550000in}}%
\pgfpathlineto{\pgfqpoint{6.886364in}{0.550000in}}%
\pgfpathlineto{\pgfqpoint{6.886364in}{0.551684in}}%
\pgfpathlineto{\pgfqpoint{6.858842in}{0.551684in}}%
\pgfpathlineto{\pgfqpoint{6.858842in}{0.550000in}}%
\pgfpathclose%
\pgfusepath{fill}%
\end{pgfscope}%
\begin{pgfscope}%
\pgfpathrectangle{\pgfqpoint{1.250000in}{0.550000in}}{\pgfqpoint{7.750000in}{3.850000in}}%
\pgfusepath{clip}%
\pgfsetbuttcap%
\pgfsetmiterjoin%
\definecolor{currentfill}{rgb}{0.501961,0.501961,0.501961}%
\pgfsetfillcolor{currentfill}%
\pgfsetfillopacity{0.700000}%
\pgfsetlinewidth{0.000000pt}%
\definecolor{currentstroke}{rgb}{0.000000,0.000000,0.000000}%
\pgfsetstrokecolor{currentstroke}%
\pgfsetstrokeopacity{0.700000}%
\pgfsetdash{}{0pt}%
\pgfpathmoveto{\pgfqpoint{6.886364in}{0.550000in}}%
\pgfpathlineto{\pgfqpoint{6.913885in}{0.550000in}}%
\pgfpathlineto{\pgfqpoint{6.913885in}{0.551990in}}%
\pgfpathlineto{\pgfqpoint{6.886364in}{0.551990in}}%
\pgfpathlineto{\pgfqpoint{6.886364in}{0.550000in}}%
\pgfpathclose%
\pgfusepath{fill}%
\end{pgfscope}%
\begin{pgfscope}%
\pgfpathrectangle{\pgfqpoint{1.250000in}{0.550000in}}{\pgfqpoint{7.750000in}{3.850000in}}%
\pgfusepath{clip}%
\pgfsetbuttcap%
\pgfsetmiterjoin%
\definecolor{currentfill}{rgb}{0.501961,0.501961,0.501961}%
\pgfsetfillcolor{currentfill}%
\pgfsetfillopacity{0.700000}%
\pgfsetlinewidth{0.000000pt}%
\definecolor{currentstroke}{rgb}{0.000000,0.000000,0.000000}%
\pgfsetstrokecolor{currentstroke}%
\pgfsetstrokeopacity{0.700000}%
\pgfsetdash{}{0pt}%
\pgfpathmoveto{\pgfqpoint{6.913885in}{0.550000in}}%
\pgfpathlineto{\pgfqpoint{6.941406in}{0.550000in}}%
\pgfpathlineto{\pgfqpoint{6.941406in}{0.552909in}}%
\pgfpathlineto{\pgfqpoint{6.913885in}{0.552909in}}%
\pgfpathlineto{\pgfqpoint{6.913885in}{0.550000in}}%
\pgfpathclose%
\pgfusepath{fill}%
\end{pgfscope}%
\begin{pgfscope}%
\pgfpathrectangle{\pgfqpoint{1.250000in}{0.550000in}}{\pgfqpoint{7.750000in}{3.850000in}}%
\pgfusepath{clip}%
\pgfsetbuttcap%
\pgfsetmiterjoin%
\definecolor{currentfill}{rgb}{0.501961,0.501961,0.501961}%
\pgfsetfillcolor{currentfill}%
\pgfsetfillopacity{0.700000}%
\pgfsetlinewidth{0.000000pt}%
\definecolor{currentstroke}{rgb}{0.000000,0.000000,0.000000}%
\pgfsetstrokecolor{currentstroke}%
\pgfsetstrokeopacity{0.700000}%
\pgfsetdash{}{0pt}%
\pgfpathmoveto{\pgfqpoint{6.941406in}{0.550000in}}%
\pgfpathlineto{\pgfqpoint{6.968928in}{0.550000in}}%
\pgfpathlineto{\pgfqpoint{6.968928in}{0.552450in}}%
\pgfpathlineto{\pgfqpoint{6.941406in}{0.552450in}}%
\pgfpathlineto{\pgfqpoint{6.941406in}{0.550000in}}%
\pgfpathclose%
\pgfusepath{fill}%
\end{pgfscope}%
\begin{pgfscope}%
\pgfpathrectangle{\pgfqpoint{1.250000in}{0.550000in}}{\pgfqpoint{7.750000in}{3.850000in}}%
\pgfusepath{clip}%
\pgfsetbuttcap%
\pgfsetmiterjoin%
\definecolor{currentfill}{rgb}{0.501961,0.501961,0.501961}%
\pgfsetfillcolor{currentfill}%
\pgfsetfillopacity{0.700000}%
\pgfsetlinewidth{0.000000pt}%
\definecolor{currentstroke}{rgb}{0.000000,0.000000,0.000000}%
\pgfsetstrokecolor{currentstroke}%
\pgfsetstrokeopacity{0.700000}%
\pgfsetdash{}{0pt}%
\pgfpathmoveto{\pgfqpoint{6.968928in}{0.550000in}}%
\pgfpathlineto{\pgfqpoint{6.996449in}{0.550000in}}%
\pgfpathlineto{\pgfqpoint{6.996449in}{0.551072in}}%
\pgfpathlineto{\pgfqpoint{6.968928in}{0.551072in}}%
\pgfpathlineto{\pgfqpoint{6.968928in}{0.550000in}}%
\pgfpathclose%
\pgfusepath{fill}%
\end{pgfscope}%
\begin{pgfscope}%
\pgfpathrectangle{\pgfqpoint{1.250000in}{0.550000in}}{\pgfqpoint{7.750000in}{3.850000in}}%
\pgfusepath{clip}%
\pgfsetbuttcap%
\pgfsetmiterjoin%
\definecolor{currentfill}{rgb}{0.501961,0.501961,0.501961}%
\pgfsetfillcolor{currentfill}%
\pgfsetfillopacity{0.700000}%
\pgfsetlinewidth{0.000000pt}%
\definecolor{currentstroke}{rgb}{0.000000,0.000000,0.000000}%
\pgfsetstrokecolor{currentstroke}%
\pgfsetstrokeopacity{0.700000}%
\pgfsetdash{}{0pt}%
\pgfpathmoveto{\pgfqpoint{6.996449in}{0.550000in}}%
\pgfpathlineto{\pgfqpoint{7.023970in}{0.550000in}}%
\pgfpathlineto{\pgfqpoint{7.023970in}{0.552297in}}%
\pgfpathlineto{\pgfqpoint{6.996449in}{0.552297in}}%
\pgfpathlineto{\pgfqpoint{6.996449in}{0.550000in}}%
\pgfpathclose%
\pgfusepath{fill}%
\end{pgfscope}%
\begin{pgfscope}%
\pgfpathrectangle{\pgfqpoint{1.250000in}{0.550000in}}{\pgfqpoint{7.750000in}{3.850000in}}%
\pgfusepath{clip}%
\pgfsetbuttcap%
\pgfsetmiterjoin%
\definecolor{currentfill}{rgb}{0.501961,0.501961,0.501961}%
\pgfsetfillcolor{currentfill}%
\pgfsetfillopacity{0.700000}%
\pgfsetlinewidth{0.000000pt}%
\definecolor{currentstroke}{rgb}{0.000000,0.000000,0.000000}%
\pgfsetstrokecolor{currentstroke}%
\pgfsetstrokeopacity{0.700000}%
\pgfsetdash{}{0pt}%
\pgfpathmoveto{\pgfqpoint{7.023970in}{0.550000in}}%
\pgfpathlineto{\pgfqpoint{7.051491in}{0.550000in}}%
\pgfpathlineto{\pgfqpoint{7.051491in}{0.552144in}}%
\pgfpathlineto{\pgfqpoint{7.023970in}{0.552144in}}%
\pgfpathlineto{\pgfqpoint{7.023970in}{0.550000in}}%
\pgfpathclose%
\pgfusepath{fill}%
\end{pgfscope}%
\begin{pgfscope}%
\pgfpathrectangle{\pgfqpoint{1.250000in}{0.550000in}}{\pgfqpoint{7.750000in}{3.850000in}}%
\pgfusepath{clip}%
\pgfsetbuttcap%
\pgfsetmiterjoin%
\definecolor{currentfill}{rgb}{0.501961,0.501961,0.501961}%
\pgfsetfillcolor{currentfill}%
\pgfsetfillopacity{0.700000}%
\pgfsetlinewidth{0.000000pt}%
\definecolor{currentstroke}{rgb}{0.000000,0.000000,0.000000}%
\pgfsetstrokecolor{currentstroke}%
\pgfsetstrokeopacity{0.700000}%
\pgfsetdash{}{0pt}%
\pgfpathmoveto{\pgfqpoint{7.051491in}{0.550000in}}%
\pgfpathlineto{\pgfqpoint{7.079013in}{0.550000in}}%
\pgfpathlineto{\pgfqpoint{7.079013in}{0.552756in}}%
\pgfpathlineto{\pgfqpoint{7.051491in}{0.552756in}}%
\pgfpathlineto{\pgfqpoint{7.051491in}{0.550000in}}%
\pgfpathclose%
\pgfusepath{fill}%
\end{pgfscope}%
\begin{pgfscope}%
\pgfpathrectangle{\pgfqpoint{1.250000in}{0.550000in}}{\pgfqpoint{7.750000in}{3.850000in}}%
\pgfusepath{clip}%
\pgfsetbuttcap%
\pgfsetmiterjoin%
\definecolor{currentfill}{rgb}{0.501961,0.501961,0.501961}%
\pgfsetfillcolor{currentfill}%
\pgfsetfillopacity{0.700000}%
\pgfsetlinewidth{0.000000pt}%
\definecolor{currentstroke}{rgb}{0.000000,0.000000,0.000000}%
\pgfsetstrokecolor{currentstroke}%
\pgfsetstrokeopacity{0.700000}%
\pgfsetdash{}{0pt}%
\pgfpathmoveto{\pgfqpoint{7.079013in}{0.550000in}}%
\pgfpathlineto{\pgfqpoint{7.106534in}{0.550000in}}%
\pgfpathlineto{\pgfqpoint{7.106534in}{0.551684in}}%
\pgfpathlineto{\pgfqpoint{7.079013in}{0.551684in}}%
\pgfpathlineto{\pgfqpoint{7.079013in}{0.550000in}}%
\pgfpathclose%
\pgfusepath{fill}%
\end{pgfscope}%
\begin{pgfscope}%
\pgfpathrectangle{\pgfqpoint{1.250000in}{0.550000in}}{\pgfqpoint{7.750000in}{3.850000in}}%
\pgfusepath{clip}%
\pgfsetbuttcap%
\pgfsetmiterjoin%
\definecolor{currentfill}{rgb}{0.501961,0.501961,0.501961}%
\pgfsetfillcolor{currentfill}%
\pgfsetfillopacity{0.700000}%
\pgfsetlinewidth{0.000000pt}%
\definecolor{currentstroke}{rgb}{0.000000,0.000000,0.000000}%
\pgfsetstrokecolor{currentstroke}%
\pgfsetstrokeopacity{0.700000}%
\pgfsetdash{}{0pt}%
\pgfpathmoveto{\pgfqpoint{7.106534in}{0.550000in}}%
\pgfpathlineto{\pgfqpoint{7.134055in}{0.550000in}}%
\pgfpathlineto{\pgfqpoint{7.134055in}{0.551837in}}%
\pgfpathlineto{\pgfqpoint{7.106534in}{0.551837in}}%
\pgfpathlineto{\pgfqpoint{7.106534in}{0.550000in}}%
\pgfpathclose%
\pgfusepath{fill}%
\end{pgfscope}%
\begin{pgfscope}%
\pgfpathrectangle{\pgfqpoint{1.250000in}{0.550000in}}{\pgfqpoint{7.750000in}{3.850000in}}%
\pgfusepath{clip}%
\pgfsetbuttcap%
\pgfsetmiterjoin%
\definecolor{currentfill}{rgb}{0.501961,0.501961,0.501961}%
\pgfsetfillcolor{currentfill}%
\pgfsetfillopacity{0.700000}%
\pgfsetlinewidth{0.000000pt}%
\definecolor{currentstroke}{rgb}{0.000000,0.000000,0.000000}%
\pgfsetstrokecolor{currentstroke}%
\pgfsetstrokeopacity{0.700000}%
\pgfsetdash{}{0pt}%
\pgfpathmoveto{\pgfqpoint{7.134055in}{0.550000in}}%
\pgfpathlineto{\pgfqpoint{7.161577in}{0.550000in}}%
\pgfpathlineto{\pgfqpoint{7.161577in}{0.551072in}}%
\pgfpathlineto{\pgfqpoint{7.134055in}{0.551072in}}%
\pgfpathlineto{\pgfqpoint{7.134055in}{0.550000in}}%
\pgfpathclose%
\pgfusepath{fill}%
\end{pgfscope}%
\begin{pgfscope}%
\pgfpathrectangle{\pgfqpoint{1.250000in}{0.550000in}}{\pgfqpoint{7.750000in}{3.850000in}}%
\pgfusepath{clip}%
\pgfsetbuttcap%
\pgfsetmiterjoin%
\definecolor{currentfill}{rgb}{0.501961,0.501961,0.501961}%
\pgfsetfillcolor{currentfill}%
\pgfsetfillopacity{0.700000}%
\pgfsetlinewidth{0.000000pt}%
\definecolor{currentstroke}{rgb}{0.000000,0.000000,0.000000}%
\pgfsetstrokecolor{currentstroke}%
\pgfsetstrokeopacity{0.700000}%
\pgfsetdash{}{0pt}%
\pgfpathmoveto{\pgfqpoint{7.161577in}{0.550000in}}%
\pgfpathlineto{\pgfqpoint{7.189098in}{0.550000in}}%
\pgfpathlineto{\pgfqpoint{7.189098in}{0.551225in}}%
\pgfpathlineto{\pgfqpoint{7.161577in}{0.551225in}}%
\pgfpathlineto{\pgfqpoint{7.161577in}{0.550000in}}%
\pgfpathclose%
\pgfusepath{fill}%
\end{pgfscope}%
\begin{pgfscope}%
\pgfpathrectangle{\pgfqpoint{1.250000in}{0.550000in}}{\pgfqpoint{7.750000in}{3.850000in}}%
\pgfusepath{clip}%
\pgfsetbuttcap%
\pgfsetmiterjoin%
\definecolor{currentfill}{rgb}{0.501961,0.501961,0.501961}%
\pgfsetfillcolor{currentfill}%
\pgfsetfillopacity{0.700000}%
\pgfsetlinewidth{0.000000pt}%
\definecolor{currentstroke}{rgb}{0.000000,0.000000,0.000000}%
\pgfsetstrokecolor{currentstroke}%
\pgfsetstrokeopacity{0.700000}%
\pgfsetdash{}{0pt}%
\pgfpathmoveto{\pgfqpoint{7.189098in}{0.550000in}}%
\pgfpathlineto{\pgfqpoint{7.216619in}{0.550000in}}%
\pgfpathlineto{\pgfqpoint{7.216619in}{0.550459in}}%
\pgfpathlineto{\pgfqpoint{7.189098in}{0.550459in}}%
\pgfpathlineto{\pgfqpoint{7.189098in}{0.550000in}}%
\pgfpathclose%
\pgfusepath{fill}%
\end{pgfscope}%
\begin{pgfscope}%
\pgfpathrectangle{\pgfqpoint{1.250000in}{0.550000in}}{\pgfqpoint{7.750000in}{3.850000in}}%
\pgfusepath{clip}%
\pgfsetbuttcap%
\pgfsetmiterjoin%
\definecolor{currentfill}{rgb}{0.501961,0.501961,0.501961}%
\pgfsetfillcolor{currentfill}%
\pgfsetfillopacity{0.700000}%
\pgfsetlinewidth{0.000000pt}%
\definecolor{currentstroke}{rgb}{0.000000,0.000000,0.000000}%
\pgfsetstrokecolor{currentstroke}%
\pgfsetstrokeopacity{0.700000}%
\pgfsetdash{}{0pt}%
\pgfpathmoveto{\pgfqpoint{7.216619in}{0.550000in}}%
\pgfpathlineto{\pgfqpoint{7.244141in}{0.550000in}}%
\pgfpathlineto{\pgfqpoint{7.244141in}{0.550919in}}%
\pgfpathlineto{\pgfqpoint{7.216619in}{0.550919in}}%
\pgfpathlineto{\pgfqpoint{7.216619in}{0.550000in}}%
\pgfpathclose%
\pgfusepath{fill}%
\end{pgfscope}%
\begin{pgfscope}%
\pgfpathrectangle{\pgfqpoint{1.250000in}{0.550000in}}{\pgfqpoint{7.750000in}{3.850000in}}%
\pgfusepath{clip}%
\pgfsetbuttcap%
\pgfsetmiterjoin%
\definecolor{currentfill}{rgb}{0.501961,0.501961,0.501961}%
\pgfsetfillcolor{currentfill}%
\pgfsetfillopacity{0.700000}%
\pgfsetlinewidth{0.000000pt}%
\definecolor{currentstroke}{rgb}{0.000000,0.000000,0.000000}%
\pgfsetstrokecolor{currentstroke}%
\pgfsetstrokeopacity{0.700000}%
\pgfsetdash{}{0pt}%
\pgfpathmoveto{\pgfqpoint{7.244141in}{0.550000in}}%
\pgfpathlineto{\pgfqpoint{7.271662in}{0.550000in}}%
\pgfpathlineto{\pgfqpoint{7.271662in}{0.550000in}}%
\pgfpathlineto{\pgfqpoint{7.244141in}{0.550000in}}%
\pgfpathlineto{\pgfqpoint{7.244141in}{0.550000in}}%
\pgfpathclose%
\pgfusepath{fill}%
\end{pgfscope}%
\begin{pgfscope}%
\pgfpathrectangle{\pgfqpoint{1.250000in}{0.550000in}}{\pgfqpoint{7.750000in}{3.850000in}}%
\pgfusepath{clip}%
\pgfsetbuttcap%
\pgfsetmiterjoin%
\definecolor{currentfill}{rgb}{0.501961,0.501961,0.501961}%
\pgfsetfillcolor{currentfill}%
\pgfsetfillopacity{0.700000}%
\pgfsetlinewidth{0.000000pt}%
\definecolor{currentstroke}{rgb}{0.000000,0.000000,0.000000}%
\pgfsetstrokecolor{currentstroke}%
\pgfsetstrokeopacity{0.700000}%
\pgfsetdash{}{0pt}%
\pgfpathmoveto{\pgfqpoint{7.271662in}{0.550000in}}%
\pgfpathlineto{\pgfqpoint{7.299183in}{0.550000in}}%
\pgfpathlineto{\pgfqpoint{7.299183in}{0.550153in}}%
\pgfpathlineto{\pgfqpoint{7.271662in}{0.550153in}}%
\pgfpathlineto{\pgfqpoint{7.271662in}{0.550000in}}%
\pgfpathclose%
\pgfusepath{fill}%
\end{pgfscope}%
\begin{pgfscope}%
\pgfpathrectangle{\pgfqpoint{1.250000in}{0.550000in}}{\pgfqpoint{7.750000in}{3.850000in}}%
\pgfusepath{clip}%
\pgfsetbuttcap%
\pgfsetmiterjoin%
\definecolor{currentfill}{rgb}{0.501961,0.501961,0.501961}%
\pgfsetfillcolor{currentfill}%
\pgfsetfillopacity{0.700000}%
\pgfsetlinewidth{0.000000pt}%
\definecolor{currentstroke}{rgb}{0.000000,0.000000,0.000000}%
\pgfsetstrokecolor{currentstroke}%
\pgfsetstrokeopacity{0.700000}%
\pgfsetdash{}{0pt}%
\pgfpathmoveto{\pgfqpoint{7.299183in}{0.550000in}}%
\pgfpathlineto{\pgfqpoint{7.326705in}{0.550000in}}%
\pgfpathlineto{\pgfqpoint{7.326705in}{0.550306in}}%
\pgfpathlineto{\pgfqpoint{7.299183in}{0.550306in}}%
\pgfpathlineto{\pgfqpoint{7.299183in}{0.550000in}}%
\pgfpathclose%
\pgfusepath{fill}%
\end{pgfscope}%
\begin{pgfscope}%
\pgfpathrectangle{\pgfqpoint{1.250000in}{0.550000in}}{\pgfqpoint{7.750000in}{3.850000in}}%
\pgfusepath{clip}%
\pgfsetbuttcap%
\pgfsetmiterjoin%
\definecolor{currentfill}{rgb}{0.501961,0.501961,0.501961}%
\pgfsetfillcolor{currentfill}%
\pgfsetfillopacity{0.700000}%
\pgfsetlinewidth{0.000000pt}%
\definecolor{currentstroke}{rgb}{0.000000,0.000000,0.000000}%
\pgfsetstrokecolor{currentstroke}%
\pgfsetstrokeopacity{0.700000}%
\pgfsetdash{}{0pt}%
\pgfpathmoveto{\pgfqpoint{7.326705in}{0.550000in}}%
\pgfpathlineto{\pgfqpoint{7.354226in}{0.550000in}}%
\pgfpathlineto{\pgfqpoint{7.354226in}{0.550000in}}%
\pgfpathlineto{\pgfqpoint{7.326705in}{0.550000in}}%
\pgfpathlineto{\pgfqpoint{7.326705in}{0.550000in}}%
\pgfpathclose%
\pgfusepath{fill}%
\end{pgfscope}%
\begin{pgfscope}%
\pgfpathrectangle{\pgfqpoint{1.250000in}{0.550000in}}{\pgfqpoint{7.750000in}{3.850000in}}%
\pgfusepath{clip}%
\pgfsetbuttcap%
\pgfsetmiterjoin%
\definecolor{currentfill}{rgb}{0.501961,0.501961,0.501961}%
\pgfsetfillcolor{currentfill}%
\pgfsetfillopacity{0.700000}%
\pgfsetlinewidth{0.000000pt}%
\definecolor{currentstroke}{rgb}{0.000000,0.000000,0.000000}%
\pgfsetstrokecolor{currentstroke}%
\pgfsetstrokeopacity{0.700000}%
\pgfsetdash{}{0pt}%
\pgfpathmoveto{\pgfqpoint{7.354226in}{0.550000in}}%
\pgfpathlineto{\pgfqpoint{7.381747in}{0.550000in}}%
\pgfpathlineto{\pgfqpoint{7.381747in}{0.550000in}}%
\pgfpathlineto{\pgfqpoint{7.354226in}{0.550000in}}%
\pgfpathlineto{\pgfqpoint{7.354226in}{0.550000in}}%
\pgfpathclose%
\pgfusepath{fill}%
\end{pgfscope}%
\begin{pgfscope}%
\pgfpathrectangle{\pgfqpoint{1.250000in}{0.550000in}}{\pgfqpoint{7.750000in}{3.850000in}}%
\pgfusepath{clip}%
\pgfsetbuttcap%
\pgfsetmiterjoin%
\definecolor{currentfill}{rgb}{0.501961,0.501961,0.501961}%
\pgfsetfillcolor{currentfill}%
\pgfsetfillopacity{0.700000}%
\pgfsetlinewidth{0.000000pt}%
\definecolor{currentstroke}{rgb}{0.000000,0.000000,0.000000}%
\pgfsetstrokecolor{currentstroke}%
\pgfsetstrokeopacity{0.700000}%
\pgfsetdash{}{0pt}%
\pgfpathmoveto{\pgfqpoint{7.381747in}{0.550000in}}%
\pgfpathlineto{\pgfqpoint{7.409268in}{0.550000in}}%
\pgfpathlineto{\pgfqpoint{7.409268in}{0.550000in}}%
\pgfpathlineto{\pgfqpoint{7.381747in}{0.550000in}}%
\pgfpathlineto{\pgfqpoint{7.381747in}{0.550000in}}%
\pgfpathclose%
\pgfusepath{fill}%
\end{pgfscope}%
\begin{pgfscope}%
\pgfpathrectangle{\pgfqpoint{1.250000in}{0.550000in}}{\pgfqpoint{7.750000in}{3.850000in}}%
\pgfusepath{clip}%
\pgfsetbuttcap%
\pgfsetmiterjoin%
\definecolor{currentfill}{rgb}{0.501961,0.501961,0.501961}%
\pgfsetfillcolor{currentfill}%
\pgfsetfillopacity{0.700000}%
\pgfsetlinewidth{0.000000pt}%
\definecolor{currentstroke}{rgb}{0.000000,0.000000,0.000000}%
\pgfsetstrokecolor{currentstroke}%
\pgfsetstrokeopacity{0.700000}%
\pgfsetdash{}{0pt}%
\pgfpathmoveto{\pgfqpoint{7.409268in}{0.550000in}}%
\pgfpathlineto{\pgfqpoint{7.436790in}{0.550000in}}%
\pgfpathlineto{\pgfqpoint{7.436790in}{0.550153in}}%
\pgfpathlineto{\pgfqpoint{7.409268in}{0.550153in}}%
\pgfpathlineto{\pgfqpoint{7.409268in}{0.550000in}}%
\pgfpathclose%
\pgfusepath{fill}%
\end{pgfscope}%
\begin{pgfscope}%
\pgfpathrectangle{\pgfqpoint{1.250000in}{0.550000in}}{\pgfqpoint{7.750000in}{3.850000in}}%
\pgfusepath{clip}%
\pgfsetbuttcap%
\pgfsetmiterjoin%
\definecolor{currentfill}{rgb}{0.501961,0.501961,0.501961}%
\pgfsetfillcolor{currentfill}%
\pgfsetfillopacity{0.700000}%
\pgfsetlinewidth{0.000000pt}%
\definecolor{currentstroke}{rgb}{0.000000,0.000000,0.000000}%
\pgfsetstrokecolor{currentstroke}%
\pgfsetstrokeopacity{0.700000}%
\pgfsetdash{}{0pt}%
\pgfpathmoveto{\pgfqpoint{7.436790in}{0.550000in}}%
\pgfpathlineto{\pgfqpoint{7.464311in}{0.550000in}}%
\pgfpathlineto{\pgfqpoint{7.464311in}{0.550000in}}%
\pgfpathlineto{\pgfqpoint{7.436790in}{0.550000in}}%
\pgfpathlineto{\pgfqpoint{7.436790in}{0.550000in}}%
\pgfpathclose%
\pgfusepath{fill}%
\end{pgfscope}%
\begin{pgfscope}%
\pgfpathrectangle{\pgfqpoint{1.250000in}{0.550000in}}{\pgfqpoint{7.750000in}{3.850000in}}%
\pgfusepath{clip}%
\pgfsetbuttcap%
\pgfsetmiterjoin%
\definecolor{currentfill}{rgb}{0.501961,0.501961,0.501961}%
\pgfsetfillcolor{currentfill}%
\pgfsetfillopacity{0.700000}%
\pgfsetlinewidth{0.000000pt}%
\definecolor{currentstroke}{rgb}{0.000000,0.000000,0.000000}%
\pgfsetstrokecolor{currentstroke}%
\pgfsetstrokeopacity{0.700000}%
\pgfsetdash{}{0pt}%
\pgfpathmoveto{\pgfqpoint{7.464311in}{0.550000in}}%
\pgfpathlineto{\pgfqpoint{7.491832in}{0.550000in}}%
\pgfpathlineto{\pgfqpoint{7.491832in}{0.550153in}}%
\pgfpathlineto{\pgfqpoint{7.464311in}{0.550153in}}%
\pgfpathlineto{\pgfqpoint{7.464311in}{0.550000in}}%
\pgfpathclose%
\pgfusepath{fill}%
\end{pgfscope}%
\begin{pgfscope}%
\pgfpathrectangle{\pgfqpoint{1.250000in}{0.550000in}}{\pgfqpoint{7.750000in}{3.850000in}}%
\pgfusepath{clip}%
\pgfsetbuttcap%
\pgfsetmiterjoin%
\definecolor{currentfill}{rgb}{0.501961,0.501961,0.501961}%
\pgfsetfillcolor{currentfill}%
\pgfsetfillopacity{0.700000}%
\pgfsetlinewidth{0.000000pt}%
\definecolor{currentstroke}{rgb}{0.000000,0.000000,0.000000}%
\pgfsetstrokecolor{currentstroke}%
\pgfsetstrokeopacity{0.700000}%
\pgfsetdash{}{0pt}%
\pgfpathmoveto{\pgfqpoint{7.491832in}{0.550000in}}%
\pgfpathlineto{\pgfqpoint{7.519354in}{0.550000in}}%
\pgfpathlineto{\pgfqpoint{7.519354in}{0.550000in}}%
\pgfpathlineto{\pgfqpoint{7.491832in}{0.550000in}}%
\pgfpathlineto{\pgfqpoint{7.491832in}{0.550000in}}%
\pgfpathclose%
\pgfusepath{fill}%
\end{pgfscope}%
\begin{pgfscope}%
\pgfpathrectangle{\pgfqpoint{1.250000in}{0.550000in}}{\pgfqpoint{7.750000in}{3.850000in}}%
\pgfusepath{clip}%
\pgfsetbuttcap%
\pgfsetmiterjoin%
\definecolor{currentfill}{rgb}{0.501961,0.501961,0.501961}%
\pgfsetfillcolor{currentfill}%
\pgfsetfillopacity{0.700000}%
\pgfsetlinewidth{0.000000pt}%
\definecolor{currentstroke}{rgb}{0.000000,0.000000,0.000000}%
\pgfsetstrokecolor{currentstroke}%
\pgfsetstrokeopacity{0.700000}%
\pgfsetdash{}{0pt}%
\pgfpathmoveto{\pgfqpoint{7.519354in}{0.550000in}}%
\pgfpathlineto{\pgfqpoint{7.546875in}{0.550000in}}%
\pgfpathlineto{\pgfqpoint{7.546875in}{0.550000in}}%
\pgfpathlineto{\pgfqpoint{7.519354in}{0.550000in}}%
\pgfpathlineto{\pgfqpoint{7.519354in}{0.550000in}}%
\pgfpathclose%
\pgfusepath{fill}%
\end{pgfscope}%
\begin{pgfscope}%
\pgfpathrectangle{\pgfqpoint{1.250000in}{0.550000in}}{\pgfqpoint{7.750000in}{3.850000in}}%
\pgfusepath{clip}%
\pgfsetbuttcap%
\pgfsetmiterjoin%
\definecolor{currentfill}{rgb}{0.501961,0.501961,0.501961}%
\pgfsetfillcolor{currentfill}%
\pgfsetfillopacity{0.700000}%
\pgfsetlinewidth{0.000000pt}%
\definecolor{currentstroke}{rgb}{0.000000,0.000000,0.000000}%
\pgfsetstrokecolor{currentstroke}%
\pgfsetstrokeopacity{0.700000}%
\pgfsetdash{}{0pt}%
\pgfpathmoveto{\pgfqpoint{7.546875in}{0.550000in}}%
\pgfpathlineto{\pgfqpoint{7.574396in}{0.550000in}}%
\pgfpathlineto{\pgfqpoint{7.574396in}{0.550000in}}%
\pgfpathlineto{\pgfqpoint{7.546875in}{0.550000in}}%
\pgfpathlineto{\pgfqpoint{7.546875in}{0.550000in}}%
\pgfpathclose%
\pgfusepath{fill}%
\end{pgfscope}%
\begin{pgfscope}%
\pgfpathrectangle{\pgfqpoint{1.250000in}{0.550000in}}{\pgfqpoint{7.750000in}{3.850000in}}%
\pgfusepath{clip}%
\pgfsetbuttcap%
\pgfsetmiterjoin%
\definecolor{currentfill}{rgb}{0.501961,0.501961,0.501961}%
\pgfsetfillcolor{currentfill}%
\pgfsetfillopacity{0.700000}%
\pgfsetlinewidth{0.000000pt}%
\definecolor{currentstroke}{rgb}{0.000000,0.000000,0.000000}%
\pgfsetstrokecolor{currentstroke}%
\pgfsetstrokeopacity{0.700000}%
\pgfsetdash{}{0pt}%
\pgfpathmoveto{\pgfqpoint{7.574396in}{0.550000in}}%
\pgfpathlineto{\pgfqpoint{7.601918in}{0.550000in}}%
\pgfpathlineto{\pgfqpoint{7.601918in}{0.550000in}}%
\pgfpathlineto{\pgfqpoint{7.574396in}{0.550000in}}%
\pgfpathlineto{\pgfqpoint{7.574396in}{0.550000in}}%
\pgfpathclose%
\pgfusepath{fill}%
\end{pgfscope}%
\begin{pgfscope}%
\pgfpathrectangle{\pgfqpoint{1.250000in}{0.550000in}}{\pgfqpoint{7.750000in}{3.850000in}}%
\pgfusepath{clip}%
\pgfsetbuttcap%
\pgfsetmiterjoin%
\definecolor{currentfill}{rgb}{0.501961,0.501961,0.501961}%
\pgfsetfillcolor{currentfill}%
\pgfsetfillopacity{0.700000}%
\pgfsetlinewidth{0.000000pt}%
\definecolor{currentstroke}{rgb}{0.000000,0.000000,0.000000}%
\pgfsetstrokecolor{currentstroke}%
\pgfsetstrokeopacity{0.700000}%
\pgfsetdash{}{0pt}%
\pgfpathmoveto{\pgfqpoint{7.601918in}{0.550000in}}%
\pgfpathlineto{\pgfqpoint{7.629439in}{0.550000in}}%
\pgfpathlineto{\pgfqpoint{7.629439in}{0.550000in}}%
\pgfpathlineto{\pgfqpoint{7.601918in}{0.550000in}}%
\pgfpathlineto{\pgfqpoint{7.601918in}{0.550000in}}%
\pgfpathclose%
\pgfusepath{fill}%
\end{pgfscope}%
\begin{pgfscope}%
\pgfpathrectangle{\pgfqpoint{1.250000in}{0.550000in}}{\pgfqpoint{7.750000in}{3.850000in}}%
\pgfusepath{clip}%
\pgfsetbuttcap%
\pgfsetmiterjoin%
\definecolor{currentfill}{rgb}{0.501961,0.501961,0.501961}%
\pgfsetfillcolor{currentfill}%
\pgfsetfillopacity{0.700000}%
\pgfsetlinewidth{0.000000pt}%
\definecolor{currentstroke}{rgb}{0.000000,0.000000,0.000000}%
\pgfsetstrokecolor{currentstroke}%
\pgfsetstrokeopacity{0.700000}%
\pgfsetdash{}{0pt}%
\pgfpathmoveto{\pgfqpoint{7.629439in}{0.550000in}}%
\pgfpathlineto{\pgfqpoint{7.656960in}{0.550000in}}%
\pgfpathlineto{\pgfqpoint{7.656960in}{0.550000in}}%
\pgfpathlineto{\pgfqpoint{7.629439in}{0.550000in}}%
\pgfpathlineto{\pgfqpoint{7.629439in}{0.550000in}}%
\pgfpathclose%
\pgfusepath{fill}%
\end{pgfscope}%
\begin{pgfscope}%
\pgfpathrectangle{\pgfqpoint{1.250000in}{0.550000in}}{\pgfqpoint{7.750000in}{3.850000in}}%
\pgfusepath{clip}%
\pgfsetbuttcap%
\pgfsetmiterjoin%
\definecolor{currentfill}{rgb}{0.501961,0.501961,0.501961}%
\pgfsetfillcolor{currentfill}%
\pgfsetfillopacity{0.700000}%
\pgfsetlinewidth{0.000000pt}%
\definecolor{currentstroke}{rgb}{0.000000,0.000000,0.000000}%
\pgfsetstrokecolor{currentstroke}%
\pgfsetstrokeopacity{0.700000}%
\pgfsetdash{}{0pt}%
\pgfpathmoveto{\pgfqpoint{7.656960in}{0.550000in}}%
\pgfpathlineto{\pgfqpoint{7.684482in}{0.550000in}}%
\pgfpathlineto{\pgfqpoint{7.684482in}{0.550306in}}%
\pgfpathlineto{\pgfqpoint{7.656960in}{0.550306in}}%
\pgfpathlineto{\pgfqpoint{7.656960in}{0.550000in}}%
\pgfpathclose%
\pgfusepath{fill}%
\end{pgfscope}%
\begin{pgfscope}%
\pgfpathrectangle{\pgfqpoint{1.250000in}{0.550000in}}{\pgfqpoint{7.750000in}{3.850000in}}%
\pgfusepath{clip}%
\pgfsetbuttcap%
\pgfsetmiterjoin%
\definecolor{currentfill}{rgb}{0.501961,0.501961,0.501961}%
\pgfsetfillcolor{currentfill}%
\pgfsetfillopacity{0.700000}%
\pgfsetlinewidth{0.000000pt}%
\definecolor{currentstroke}{rgb}{0.000000,0.000000,0.000000}%
\pgfsetstrokecolor{currentstroke}%
\pgfsetstrokeopacity{0.700000}%
\pgfsetdash{}{0pt}%
\pgfpathmoveto{\pgfqpoint{7.684482in}{0.550000in}}%
\pgfpathlineto{\pgfqpoint{7.712003in}{0.550000in}}%
\pgfpathlineto{\pgfqpoint{7.712003in}{0.550000in}}%
\pgfpathlineto{\pgfqpoint{7.684482in}{0.550000in}}%
\pgfpathlineto{\pgfqpoint{7.684482in}{0.550000in}}%
\pgfpathclose%
\pgfusepath{fill}%
\end{pgfscope}%
\begin{pgfscope}%
\pgfpathrectangle{\pgfqpoint{1.250000in}{0.550000in}}{\pgfqpoint{7.750000in}{3.850000in}}%
\pgfusepath{clip}%
\pgfsetbuttcap%
\pgfsetmiterjoin%
\definecolor{currentfill}{rgb}{0.501961,0.501961,0.501961}%
\pgfsetfillcolor{currentfill}%
\pgfsetfillopacity{0.700000}%
\pgfsetlinewidth{0.000000pt}%
\definecolor{currentstroke}{rgb}{0.000000,0.000000,0.000000}%
\pgfsetstrokecolor{currentstroke}%
\pgfsetstrokeopacity{0.700000}%
\pgfsetdash{}{0pt}%
\pgfpathmoveto{\pgfqpoint{7.712003in}{0.550000in}}%
\pgfpathlineto{\pgfqpoint{7.739524in}{0.550000in}}%
\pgfpathlineto{\pgfqpoint{7.739524in}{0.550153in}}%
\pgfpathlineto{\pgfqpoint{7.712003in}{0.550153in}}%
\pgfpathlineto{\pgfqpoint{7.712003in}{0.550000in}}%
\pgfpathclose%
\pgfusepath{fill}%
\end{pgfscope}%
\begin{pgfscope}%
\pgfpathrectangle{\pgfqpoint{1.250000in}{0.550000in}}{\pgfqpoint{7.750000in}{3.850000in}}%
\pgfusepath{clip}%
\pgfsetbuttcap%
\pgfsetmiterjoin%
\definecolor{currentfill}{rgb}{0.501961,0.501961,0.501961}%
\pgfsetfillcolor{currentfill}%
\pgfsetfillopacity{0.700000}%
\pgfsetlinewidth{0.000000pt}%
\definecolor{currentstroke}{rgb}{0.000000,0.000000,0.000000}%
\pgfsetstrokecolor{currentstroke}%
\pgfsetstrokeopacity{0.700000}%
\pgfsetdash{}{0pt}%
\pgfpathmoveto{\pgfqpoint{7.739524in}{0.550000in}}%
\pgfpathlineto{\pgfqpoint{7.767045in}{0.550000in}}%
\pgfpathlineto{\pgfqpoint{7.767045in}{0.550000in}}%
\pgfpathlineto{\pgfqpoint{7.739524in}{0.550000in}}%
\pgfpathlineto{\pgfqpoint{7.739524in}{0.550000in}}%
\pgfpathclose%
\pgfusepath{fill}%
\end{pgfscope}%
\begin{pgfscope}%
\pgfpathrectangle{\pgfqpoint{1.250000in}{0.550000in}}{\pgfqpoint{7.750000in}{3.850000in}}%
\pgfusepath{clip}%
\pgfsetbuttcap%
\pgfsetmiterjoin%
\definecolor{currentfill}{rgb}{0.501961,0.501961,0.501961}%
\pgfsetfillcolor{currentfill}%
\pgfsetfillopacity{0.700000}%
\pgfsetlinewidth{0.000000pt}%
\definecolor{currentstroke}{rgb}{0.000000,0.000000,0.000000}%
\pgfsetstrokecolor{currentstroke}%
\pgfsetstrokeopacity{0.700000}%
\pgfsetdash{}{0pt}%
\pgfpathmoveto{\pgfqpoint{7.767045in}{0.550000in}}%
\pgfpathlineto{\pgfqpoint{7.794567in}{0.550000in}}%
\pgfpathlineto{\pgfqpoint{7.794567in}{0.550153in}}%
\pgfpathlineto{\pgfqpoint{7.767045in}{0.550153in}}%
\pgfpathlineto{\pgfqpoint{7.767045in}{0.550000in}}%
\pgfpathclose%
\pgfusepath{fill}%
\end{pgfscope}%
\begin{pgfscope}%
\pgfpathrectangle{\pgfqpoint{1.250000in}{0.550000in}}{\pgfqpoint{7.750000in}{3.850000in}}%
\pgfusepath{clip}%
\pgfsetbuttcap%
\pgfsetmiterjoin%
\definecolor{currentfill}{rgb}{0.501961,0.501961,0.501961}%
\pgfsetfillcolor{currentfill}%
\pgfsetfillopacity{0.700000}%
\pgfsetlinewidth{0.000000pt}%
\definecolor{currentstroke}{rgb}{0.000000,0.000000,0.000000}%
\pgfsetstrokecolor{currentstroke}%
\pgfsetstrokeopacity{0.700000}%
\pgfsetdash{}{0pt}%
\pgfpathmoveto{\pgfqpoint{7.794567in}{0.550000in}}%
\pgfpathlineto{\pgfqpoint{7.822088in}{0.550000in}}%
\pgfpathlineto{\pgfqpoint{7.822088in}{0.550153in}}%
\pgfpathlineto{\pgfqpoint{7.794567in}{0.550153in}}%
\pgfpathlineto{\pgfqpoint{7.794567in}{0.550000in}}%
\pgfpathclose%
\pgfusepath{fill}%
\end{pgfscope}%
\begin{pgfscope}%
\pgfpathrectangle{\pgfqpoint{1.250000in}{0.550000in}}{\pgfqpoint{7.750000in}{3.850000in}}%
\pgfusepath{clip}%
\pgfsetbuttcap%
\pgfsetmiterjoin%
\definecolor{currentfill}{rgb}{0.501961,0.501961,0.501961}%
\pgfsetfillcolor{currentfill}%
\pgfsetfillopacity{0.700000}%
\pgfsetlinewidth{0.000000pt}%
\definecolor{currentstroke}{rgb}{0.000000,0.000000,0.000000}%
\pgfsetstrokecolor{currentstroke}%
\pgfsetstrokeopacity{0.700000}%
\pgfsetdash{}{0pt}%
\pgfpathmoveto{\pgfqpoint{7.822088in}{0.550000in}}%
\pgfpathlineto{\pgfqpoint{7.849609in}{0.550000in}}%
\pgfpathlineto{\pgfqpoint{7.849609in}{0.550153in}}%
\pgfpathlineto{\pgfqpoint{7.822088in}{0.550153in}}%
\pgfpathlineto{\pgfqpoint{7.822088in}{0.550000in}}%
\pgfpathclose%
\pgfusepath{fill}%
\end{pgfscope}%
\begin{pgfscope}%
\pgfpathrectangle{\pgfqpoint{1.250000in}{0.550000in}}{\pgfqpoint{7.750000in}{3.850000in}}%
\pgfusepath{clip}%
\pgfsetbuttcap%
\pgfsetmiterjoin%
\definecolor{currentfill}{rgb}{0.501961,0.501961,0.501961}%
\pgfsetfillcolor{currentfill}%
\pgfsetfillopacity{0.700000}%
\pgfsetlinewidth{0.000000pt}%
\definecolor{currentstroke}{rgb}{0.000000,0.000000,0.000000}%
\pgfsetstrokecolor{currentstroke}%
\pgfsetstrokeopacity{0.700000}%
\pgfsetdash{}{0pt}%
\pgfpathmoveto{\pgfqpoint{7.849609in}{0.550000in}}%
\pgfpathlineto{\pgfqpoint{7.877131in}{0.550000in}}%
\pgfpathlineto{\pgfqpoint{7.877131in}{0.550000in}}%
\pgfpathlineto{\pgfqpoint{7.849609in}{0.550000in}}%
\pgfpathlineto{\pgfqpoint{7.849609in}{0.550000in}}%
\pgfpathclose%
\pgfusepath{fill}%
\end{pgfscope}%
\begin{pgfscope}%
\pgfpathrectangle{\pgfqpoint{1.250000in}{0.550000in}}{\pgfqpoint{7.750000in}{3.850000in}}%
\pgfusepath{clip}%
\pgfsetbuttcap%
\pgfsetmiterjoin%
\definecolor{currentfill}{rgb}{0.501961,0.501961,0.501961}%
\pgfsetfillcolor{currentfill}%
\pgfsetfillopacity{0.700000}%
\pgfsetlinewidth{0.000000pt}%
\definecolor{currentstroke}{rgb}{0.000000,0.000000,0.000000}%
\pgfsetstrokecolor{currentstroke}%
\pgfsetstrokeopacity{0.700000}%
\pgfsetdash{}{0pt}%
\pgfpathmoveto{\pgfqpoint{7.877131in}{0.550000in}}%
\pgfpathlineto{\pgfqpoint{7.904652in}{0.550000in}}%
\pgfpathlineto{\pgfqpoint{7.904652in}{0.550000in}}%
\pgfpathlineto{\pgfqpoint{7.877131in}{0.550000in}}%
\pgfpathlineto{\pgfqpoint{7.877131in}{0.550000in}}%
\pgfpathclose%
\pgfusepath{fill}%
\end{pgfscope}%
\begin{pgfscope}%
\pgfpathrectangle{\pgfqpoint{1.250000in}{0.550000in}}{\pgfqpoint{7.750000in}{3.850000in}}%
\pgfusepath{clip}%
\pgfsetbuttcap%
\pgfsetmiterjoin%
\definecolor{currentfill}{rgb}{0.501961,0.501961,0.501961}%
\pgfsetfillcolor{currentfill}%
\pgfsetfillopacity{0.700000}%
\pgfsetlinewidth{0.000000pt}%
\definecolor{currentstroke}{rgb}{0.000000,0.000000,0.000000}%
\pgfsetstrokecolor{currentstroke}%
\pgfsetstrokeopacity{0.700000}%
\pgfsetdash{}{0pt}%
\pgfpathmoveto{\pgfqpoint{7.904652in}{0.550000in}}%
\pgfpathlineto{\pgfqpoint{7.932173in}{0.550000in}}%
\pgfpathlineto{\pgfqpoint{7.932173in}{0.550153in}}%
\pgfpathlineto{\pgfqpoint{7.904652in}{0.550153in}}%
\pgfpathlineto{\pgfqpoint{7.904652in}{0.550000in}}%
\pgfpathclose%
\pgfusepath{fill}%
\end{pgfscope}%
\begin{pgfscope}%
\pgfpathrectangle{\pgfqpoint{1.250000in}{0.550000in}}{\pgfqpoint{7.750000in}{3.850000in}}%
\pgfusepath{clip}%
\pgfsetbuttcap%
\pgfsetmiterjoin%
\definecolor{currentfill}{rgb}{0.501961,0.501961,0.501961}%
\pgfsetfillcolor{currentfill}%
\pgfsetfillopacity{0.700000}%
\pgfsetlinewidth{0.000000pt}%
\definecolor{currentstroke}{rgb}{0.000000,0.000000,0.000000}%
\pgfsetstrokecolor{currentstroke}%
\pgfsetstrokeopacity{0.700000}%
\pgfsetdash{}{0pt}%
\pgfpathmoveto{\pgfqpoint{7.932173in}{0.550000in}}%
\pgfpathlineto{\pgfqpoint{7.959695in}{0.550000in}}%
\pgfpathlineto{\pgfqpoint{7.959695in}{0.550306in}}%
\pgfpathlineto{\pgfqpoint{7.932173in}{0.550306in}}%
\pgfpathlineto{\pgfqpoint{7.932173in}{0.550000in}}%
\pgfpathclose%
\pgfusepath{fill}%
\end{pgfscope}%
\begin{pgfscope}%
\pgfpathrectangle{\pgfqpoint{1.250000in}{0.550000in}}{\pgfqpoint{7.750000in}{3.850000in}}%
\pgfusepath{clip}%
\pgfsetbuttcap%
\pgfsetmiterjoin%
\definecolor{currentfill}{rgb}{0.501961,0.501961,0.501961}%
\pgfsetfillcolor{currentfill}%
\pgfsetfillopacity{0.700000}%
\pgfsetlinewidth{0.000000pt}%
\definecolor{currentstroke}{rgb}{0.000000,0.000000,0.000000}%
\pgfsetstrokecolor{currentstroke}%
\pgfsetstrokeopacity{0.700000}%
\pgfsetdash{}{0pt}%
\pgfpathmoveto{\pgfqpoint{7.959695in}{0.550000in}}%
\pgfpathlineto{\pgfqpoint{7.987216in}{0.550000in}}%
\pgfpathlineto{\pgfqpoint{7.987216in}{0.550000in}}%
\pgfpathlineto{\pgfqpoint{7.959695in}{0.550000in}}%
\pgfpathlineto{\pgfqpoint{7.959695in}{0.550000in}}%
\pgfpathclose%
\pgfusepath{fill}%
\end{pgfscope}%
\begin{pgfscope}%
\pgfpathrectangle{\pgfqpoint{1.250000in}{0.550000in}}{\pgfqpoint{7.750000in}{3.850000in}}%
\pgfusepath{clip}%
\pgfsetbuttcap%
\pgfsetmiterjoin%
\definecolor{currentfill}{rgb}{0.501961,0.501961,0.501961}%
\pgfsetfillcolor{currentfill}%
\pgfsetfillopacity{0.700000}%
\pgfsetlinewidth{0.000000pt}%
\definecolor{currentstroke}{rgb}{0.000000,0.000000,0.000000}%
\pgfsetstrokecolor{currentstroke}%
\pgfsetstrokeopacity{0.700000}%
\pgfsetdash{}{0pt}%
\pgfpathmoveto{\pgfqpoint{7.987216in}{0.550000in}}%
\pgfpathlineto{\pgfqpoint{8.014737in}{0.550000in}}%
\pgfpathlineto{\pgfqpoint{8.014737in}{0.550153in}}%
\pgfpathlineto{\pgfqpoint{7.987216in}{0.550153in}}%
\pgfpathlineto{\pgfqpoint{7.987216in}{0.550000in}}%
\pgfpathclose%
\pgfusepath{fill}%
\end{pgfscope}%
\begin{pgfscope}%
\pgfpathrectangle{\pgfqpoint{1.250000in}{0.550000in}}{\pgfqpoint{7.750000in}{3.850000in}}%
\pgfusepath{clip}%
\pgfsetbuttcap%
\pgfsetmiterjoin%
\definecolor{currentfill}{rgb}{0.501961,0.501961,0.501961}%
\pgfsetfillcolor{currentfill}%
\pgfsetfillopacity{0.700000}%
\pgfsetlinewidth{0.000000pt}%
\definecolor{currentstroke}{rgb}{0.000000,0.000000,0.000000}%
\pgfsetstrokecolor{currentstroke}%
\pgfsetstrokeopacity{0.700000}%
\pgfsetdash{}{0pt}%
\pgfpathmoveto{\pgfqpoint{8.014737in}{0.550000in}}%
\pgfpathlineto{\pgfqpoint{8.042259in}{0.550000in}}%
\pgfpathlineto{\pgfqpoint{8.042259in}{0.550153in}}%
\pgfpathlineto{\pgfqpoint{8.014737in}{0.550153in}}%
\pgfpathlineto{\pgfqpoint{8.014737in}{0.550000in}}%
\pgfpathclose%
\pgfusepath{fill}%
\end{pgfscope}%
\begin{pgfscope}%
\pgfpathrectangle{\pgfqpoint{1.250000in}{0.550000in}}{\pgfqpoint{7.750000in}{3.850000in}}%
\pgfusepath{clip}%
\pgfsetbuttcap%
\pgfsetmiterjoin%
\definecolor{currentfill}{rgb}{0.501961,0.501961,0.501961}%
\pgfsetfillcolor{currentfill}%
\pgfsetfillopacity{0.700000}%
\pgfsetlinewidth{0.000000pt}%
\definecolor{currentstroke}{rgb}{0.000000,0.000000,0.000000}%
\pgfsetstrokecolor{currentstroke}%
\pgfsetstrokeopacity{0.700000}%
\pgfsetdash{}{0pt}%
\pgfpathmoveto{\pgfqpoint{8.042259in}{0.550000in}}%
\pgfpathlineto{\pgfqpoint{8.069780in}{0.550000in}}%
\pgfpathlineto{\pgfqpoint{8.069780in}{0.550000in}}%
\pgfpathlineto{\pgfqpoint{8.042259in}{0.550000in}}%
\pgfpathlineto{\pgfqpoint{8.042259in}{0.550000in}}%
\pgfpathclose%
\pgfusepath{fill}%
\end{pgfscope}%
\begin{pgfscope}%
\pgfpathrectangle{\pgfqpoint{1.250000in}{0.550000in}}{\pgfqpoint{7.750000in}{3.850000in}}%
\pgfusepath{clip}%
\pgfsetbuttcap%
\pgfsetmiterjoin%
\definecolor{currentfill}{rgb}{0.501961,0.501961,0.501961}%
\pgfsetfillcolor{currentfill}%
\pgfsetfillopacity{0.700000}%
\pgfsetlinewidth{0.000000pt}%
\definecolor{currentstroke}{rgb}{0.000000,0.000000,0.000000}%
\pgfsetstrokecolor{currentstroke}%
\pgfsetstrokeopacity{0.700000}%
\pgfsetdash{}{0pt}%
\pgfpathmoveto{\pgfqpoint{8.069780in}{0.550000in}}%
\pgfpathlineto{\pgfqpoint{8.097301in}{0.550000in}}%
\pgfpathlineto{\pgfqpoint{8.097301in}{0.550000in}}%
\pgfpathlineto{\pgfqpoint{8.069780in}{0.550000in}}%
\pgfpathlineto{\pgfqpoint{8.069780in}{0.550000in}}%
\pgfpathclose%
\pgfusepath{fill}%
\end{pgfscope}%
\begin{pgfscope}%
\pgfpathrectangle{\pgfqpoint{1.250000in}{0.550000in}}{\pgfqpoint{7.750000in}{3.850000in}}%
\pgfusepath{clip}%
\pgfsetbuttcap%
\pgfsetmiterjoin%
\definecolor{currentfill}{rgb}{0.501961,0.501961,0.501961}%
\pgfsetfillcolor{currentfill}%
\pgfsetfillopacity{0.700000}%
\pgfsetlinewidth{0.000000pt}%
\definecolor{currentstroke}{rgb}{0.000000,0.000000,0.000000}%
\pgfsetstrokecolor{currentstroke}%
\pgfsetstrokeopacity{0.700000}%
\pgfsetdash{}{0pt}%
\pgfpathmoveto{\pgfqpoint{8.097301in}{0.550000in}}%
\pgfpathlineto{\pgfqpoint{8.124822in}{0.550000in}}%
\pgfpathlineto{\pgfqpoint{8.124822in}{0.550153in}}%
\pgfpathlineto{\pgfqpoint{8.097301in}{0.550153in}}%
\pgfpathlineto{\pgfqpoint{8.097301in}{0.550000in}}%
\pgfpathclose%
\pgfusepath{fill}%
\end{pgfscope}%
\begin{pgfscope}%
\pgfpathrectangle{\pgfqpoint{1.250000in}{0.550000in}}{\pgfqpoint{7.750000in}{3.850000in}}%
\pgfusepath{clip}%
\pgfsetbuttcap%
\pgfsetmiterjoin%
\definecolor{currentfill}{rgb}{0.501961,0.501961,0.501961}%
\pgfsetfillcolor{currentfill}%
\pgfsetfillopacity{0.700000}%
\pgfsetlinewidth{0.000000pt}%
\definecolor{currentstroke}{rgb}{0.000000,0.000000,0.000000}%
\pgfsetstrokecolor{currentstroke}%
\pgfsetstrokeopacity{0.700000}%
\pgfsetdash{}{0pt}%
\pgfpathmoveto{\pgfqpoint{8.124822in}{0.550000in}}%
\pgfpathlineto{\pgfqpoint{8.152344in}{0.550000in}}%
\pgfpathlineto{\pgfqpoint{8.152344in}{0.550306in}}%
\pgfpathlineto{\pgfqpoint{8.124822in}{0.550306in}}%
\pgfpathlineto{\pgfqpoint{8.124822in}{0.550000in}}%
\pgfpathclose%
\pgfusepath{fill}%
\end{pgfscope}%
\begin{pgfscope}%
\pgfpathrectangle{\pgfqpoint{1.250000in}{0.550000in}}{\pgfqpoint{7.750000in}{3.850000in}}%
\pgfusepath{clip}%
\pgfsetbuttcap%
\pgfsetmiterjoin%
\definecolor{currentfill}{rgb}{0.501961,0.501961,0.501961}%
\pgfsetfillcolor{currentfill}%
\pgfsetfillopacity{0.700000}%
\pgfsetlinewidth{0.000000pt}%
\definecolor{currentstroke}{rgb}{0.000000,0.000000,0.000000}%
\pgfsetstrokecolor{currentstroke}%
\pgfsetstrokeopacity{0.700000}%
\pgfsetdash{}{0pt}%
\pgfpathmoveto{\pgfqpoint{8.152344in}{0.550000in}}%
\pgfpathlineto{\pgfqpoint{8.179865in}{0.550000in}}%
\pgfpathlineto{\pgfqpoint{8.179865in}{0.550000in}}%
\pgfpathlineto{\pgfqpoint{8.152344in}{0.550000in}}%
\pgfpathlineto{\pgfqpoint{8.152344in}{0.550000in}}%
\pgfpathclose%
\pgfusepath{fill}%
\end{pgfscope}%
\begin{pgfscope}%
\pgfpathrectangle{\pgfqpoint{1.250000in}{0.550000in}}{\pgfqpoint{7.750000in}{3.850000in}}%
\pgfusepath{clip}%
\pgfsetbuttcap%
\pgfsetmiterjoin%
\definecolor{currentfill}{rgb}{0.501961,0.501961,0.501961}%
\pgfsetfillcolor{currentfill}%
\pgfsetfillopacity{0.700000}%
\pgfsetlinewidth{0.000000pt}%
\definecolor{currentstroke}{rgb}{0.000000,0.000000,0.000000}%
\pgfsetstrokecolor{currentstroke}%
\pgfsetstrokeopacity{0.700000}%
\pgfsetdash{}{0pt}%
\pgfpathmoveto{\pgfqpoint{8.179865in}{0.550000in}}%
\pgfpathlineto{\pgfqpoint{8.207386in}{0.550000in}}%
\pgfpathlineto{\pgfqpoint{8.207386in}{0.550000in}}%
\pgfpathlineto{\pgfqpoint{8.179865in}{0.550000in}}%
\pgfpathlineto{\pgfqpoint{8.179865in}{0.550000in}}%
\pgfpathclose%
\pgfusepath{fill}%
\end{pgfscope}%
\begin{pgfscope}%
\pgfpathrectangle{\pgfqpoint{1.250000in}{0.550000in}}{\pgfqpoint{7.750000in}{3.850000in}}%
\pgfusepath{clip}%
\pgfsetbuttcap%
\pgfsetmiterjoin%
\definecolor{currentfill}{rgb}{0.501961,0.501961,0.501961}%
\pgfsetfillcolor{currentfill}%
\pgfsetfillopacity{0.700000}%
\pgfsetlinewidth{0.000000pt}%
\definecolor{currentstroke}{rgb}{0.000000,0.000000,0.000000}%
\pgfsetstrokecolor{currentstroke}%
\pgfsetstrokeopacity{0.700000}%
\pgfsetdash{}{0pt}%
\pgfpathmoveto{\pgfqpoint{8.207386in}{0.550000in}}%
\pgfpathlineto{\pgfqpoint{8.234908in}{0.550000in}}%
\pgfpathlineto{\pgfqpoint{8.234908in}{0.550000in}}%
\pgfpathlineto{\pgfqpoint{8.207386in}{0.550000in}}%
\pgfpathlineto{\pgfqpoint{8.207386in}{0.550000in}}%
\pgfpathclose%
\pgfusepath{fill}%
\end{pgfscope}%
\begin{pgfscope}%
\pgfpathrectangle{\pgfqpoint{1.250000in}{0.550000in}}{\pgfqpoint{7.750000in}{3.850000in}}%
\pgfusepath{clip}%
\pgfsetbuttcap%
\pgfsetmiterjoin%
\definecolor{currentfill}{rgb}{0.501961,0.501961,0.501961}%
\pgfsetfillcolor{currentfill}%
\pgfsetfillopacity{0.700000}%
\pgfsetlinewidth{0.000000pt}%
\definecolor{currentstroke}{rgb}{0.000000,0.000000,0.000000}%
\pgfsetstrokecolor{currentstroke}%
\pgfsetstrokeopacity{0.700000}%
\pgfsetdash{}{0pt}%
\pgfpathmoveto{\pgfqpoint{8.234908in}{0.550000in}}%
\pgfpathlineto{\pgfqpoint{8.262429in}{0.550000in}}%
\pgfpathlineto{\pgfqpoint{8.262429in}{0.550000in}}%
\pgfpathlineto{\pgfqpoint{8.234908in}{0.550000in}}%
\pgfpathlineto{\pgfqpoint{8.234908in}{0.550000in}}%
\pgfpathclose%
\pgfusepath{fill}%
\end{pgfscope}%
\begin{pgfscope}%
\pgfpathrectangle{\pgfqpoint{1.250000in}{0.550000in}}{\pgfqpoint{7.750000in}{3.850000in}}%
\pgfusepath{clip}%
\pgfsetbuttcap%
\pgfsetmiterjoin%
\definecolor{currentfill}{rgb}{0.501961,0.501961,0.501961}%
\pgfsetfillcolor{currentfill}%
\pgfsetfillopacity{0.700000}%
\pgfsetlinewidth{0.000000pt}%
\definecolor{currentstroke}{rgb}{0.000000,0.000000,0.000000}%
\pgfsetstrokecolor{currentstroke}%
\pgfsetstrokeopacity{0.700000}%
\pgfsetdash{}{0pt}%
\pgfpathmoveto{\pgfqpoint{8.262429in}{0.550000in}}%
\pgfpathlineto{\pgfqpoint{8.289950in}{0.550000in}}%
\pgfpathlineto{\pgfqpoint{8.289950in}{0.550153in}}%
\pgfpathlineto{\pgfqpoint{8.262429in}{0.550153in}}%
\pgfpathlineto{\pgfqpoint{8.262429in}{0.550000in}}%
\pgfpathclose%
\pgfusepath{fill}%
\end{pgfscope}%
\begin{pgfscope}%
\pgfpathrectangle{\pgfqpoint{1.250000in}{0.550000in}}{\pgfqpoint{7.750000in}{3.850000in}}%
\pgfusepath{clip}%
\pgfsetbuttcap%
\pgfsetmiterjoin%
\definecolor{currentfill}{rgb}{0.501961,0.501961,0.501961}%
\pgfsetfillcolor{currentfill}%
\pgfsetfillopacity{0.700000}%
\pgfsetlinewidth{0.000000pt}%
\definecolor{currentstroke}{rgb}{0.000000,0.000000,0.000000}%
\pgfsetstrokecolor{currentstroke}%
\pgfsetstrokeopacity{0.700000}%
\pgfsetdash{}{0pt}%
\pgfpathmoveto{\pgfqpoint{8.289950in}{0.550000in}}%
\pgfpathlineto{\pgfqpoint{8.317472in}{0.550000in}}%
\pgfpathlineto{\pgfqpoint{8.317472in}{0.550000in}}%
\pgfpathlineto{\pgfqpoint{8.289950in}{0.550000in}}%
\pgfpathlineto{\pgfqpoint{8.289950in}{0.550000in}}%
\pgfpathclose%
\pgfusepath{fill}%
\end{pgfscope}%
\begin{pgfscope}%
\pgfpathrectangle{\pgfqpoint{1.250000in}{0.550000in}}{\pgfqpoint{7.750000in}{3.850000in}}%
\pgfusepath{clip}%
\pgfsetbuttcap%
\pgfsetmiterjoin%
\definecolor{currentfill}{rgb}{0.501961,0.501961,0.501961}%
\pgfsetfillcolor{currentfill}%
\pgfsetfillopacity{0.700000}%
\pgfsetlinewidth{0.000000pt}%
\definecolor{currentstroke}{rgb}{0.000000,0.000000,0.000000}%
\pgfsetstrokecolor{currentstroke}%
\pgfsetstrokeopacity{0.700000}%
\pgfsetdash{}{0pt}%
\pgfpathmoveto{\pgfqpoint{8.317472in}{0.550000in}}%
\pgfpathlineto{\pgfqpoint{8.344993in}{0.550000in}}%
\pgfpathlineto{\pgfqpoint{8.344993in}{0.550000in}}%
\pgfpathlineto{\pgfqpoint{8.317472in}{0.550000in}}%
\pgfpathlineto{\pgfqpoint{8.317472in}{0.550000in}}%
\pgfpathclose%
\pgfusepath{fill}%
\end{pgfscope}%
\begin{pgfscope}%
\pgfpathrectangle{\pgfqpoint{1.250000in}{0.550000in}}{\pgfqpoint{7.750000in}{3.850000in}}%
\pgfusepath{clip}%
\pgfsetbuttcap%
\pgfsetmiterjoin%
\definecolor{currentfill}{rgb}{0.501961,0.501961,0.501961}%
\pgfsetfillcolor{currentfill}%
\pgfsetfillopacity{0.700000}%
\pgfsetlinewidth{0.000000pt}%
\definecolor{currentstroke}{rgb}{0.000000,0.000000,0.000000}%
\pgfsetstrokecolor{currentstroke}%
\pgfsetstrokeopacity{0.700000}%
\pgfsetdash{}{0pt}%
\pgfpathmoveto{\pgfqpoint{8.344993in}{0.550000in}}%
\pgfpathlineto{\pgfqpoint{8.372514in}{0.550000in}}%
\pgfpathlineto{\pgfqpoint{8.372514in}{0.550000in}}%
\pgfpathlineto{\pgfqpoint{8.344993in}{0.550000in}}%
\pgfpathlineto{\pgfqpoint{8.344993in}{0.550000in}}%
\pgfpathclose%
\pgfusepath{fill}%
\end{pgfscope}%
\begin{pgfscope}%
\pgfpathrectangle{\pgfqpoint{1.250000in}{0.550000in}}{\pgfqpoint{7.750000in}{3.850000in}}%
\pgfusepath{clip}%
\pgfsetbuttcap%
\pgfsetmiterjoin%
\definecolor{currentfill}{rgb}{0.501961,0.501961,0.501961}%
\pgfsetfillcolor{currentfill}%
\pgfsetfillopacity{0.700000}%
\pgfsetlinewidth{0.000000pt}%
\definecolor{currentstroke}{rgb}{0.000000,0.000000,0.000000}%
\pgfsetstrokecolor{currentstroke}%
\pgfsetstrokeopacity{0.700000}%
\pgfsetdash{}{0pt}%
\pgfpathmoveto{\pgfqpoint{8.372514in}{0.550000in}}%
\pgfpathlineto{\pgfqpoint{8.400036in}{0.550000in}}%
\pgfpathlineto{\pgfqpoint{8.400036in}{0.550000in}}%
\pgfpathlineto{\pgfqpoint{8.372514in}{0.550000in}}%
\pgfpathlineto{\pgfqpoint{8.372514in}{0.550000in}}%
\pgfpathclose%
\pgfusepath{fill}%
\end{pgfscope}%
\begin{pgfscope}%
\pgfpathrectangle{\pgfqpoint{1.250000in}{0.550000in}}{\pgfqpoint{7.750000in}{3.850000in}}%
\pgfusepath{clip}%
\pgfsetbuttcap%
\pgfsetmiterjoin%
\definecolor{currentfill}{rgb}{0.501961,0.501961,0.501961}%
\pgfsetfillcolor{currentfill}%
\pgfsetfillopacity{0.700000}%
\pgfsetlinewidth{0.000000pt}%
\definecolor{currentstroke}{rgb}{0.000000,0.000000,0.000000}%
\pgfsetstrokecolor{currentstroke}%
\pgfsetstrokeopacity{0.700000}%
\pgfsetdash{}{0pt}%
\pgfpathmoveto{\pgfqpoint{8.400036in}{0.550000in}}%
\pgfpathlineto{\pgfqpoint{8.427557in}{0.550000in}}%
\pgfpathlineto{\pgfqpoint{8.427557in}{0.550000in}}%
\pgfpathlineto{\pgfqpoint{8.400036in}{0.550000in}}%
\pgfpathlineto{\pgfqpoint{8.400036in}{0.550000in}}%
\pgfpathclose%
\pgfusepath{fill}%
\end{pgfscope}%
\begin{pgfscope}%
\pgfpathrectangle{\pgfqpoint{1.250000in}{0.550000in}}{\pgfqpoint{7.750000in}{3.850000in}}%
\pgfusepath{clip}%
\pgfsetbuttcap%
\pgfsetmiterjoin%
\definecolor{currentfill}{rgb}{0.501961,0.501961,0.501961}%
\pgfsetfillcolor{currentfill}%
\pgfsetfillopacity{0.700000}%
\pgfsetlinewidth{0.000000pt}%
\definecolor{currentstroke}{rgb}{0.000000,0.000000,0.000000}%
\pgfsetstrokecolor{currentstroke}%
\pgfsetstrokeopacity{0.700000}%
\pgfsetdash{}{0pt}%
\pgfpathmoveto{\pgfqpoint{8.427557in}{0.550000in}}%
\pgfpathlineto{\pgfqpoint{8.455078in}{0.550000in}}%
\pgfpathlineto{\pgfqpoint{8.455078in}{0.550000in}}%
\pgfpathlineto{\pgfqpoint{8.427557in}{0.550000in}}%
\pgfpathlineto{\pgfqpoint{8.427557in}{0.550000in}}%
\pgfpathclose%
\pgfusepath{fill}%
\end{pgfscope}%
\begin{pgfscope}%
\pgfpathrectangle{\pgfqpoint{1.250000in}{0.550000in}}{\pgfqpoint{7.750000in}{3.850000in}}%
\pgfusepath{clip}%
\pgfsetbuttcap%
\pgfsetmiterjoin%
\definecolor{currentfill}{rgb}{0.501961,0.501961,0.501961}%
\pgfsetfillcolor{currentfill}%
\pgfsetfillopacity{0.700000}%
\pgfsetlinewidth{0.000000pt}%
\definecolor{currentstroke}{rgb}{0.000000,0.000000,0.000000}%
\pgfsetstrokecolor{currentstroke}%
\pgfsetstrokeopacity{0.700000}%
\pgfsetdash{}{0pt}%
\pgfpathmoveto{\pgfqpoint{8.455078in}{0.550000in}}%
\pgfpathlineto{\pgfqpoint{8.482599in}{0.550000in}}%
\pgfpathlineto{\pgfqpoint{8.482599in}{0.550000in}}%
\pgfpathlineto{\pgfqpoint{8.455078in}{0.550000in}}%
\pgfpathlineto{\pgfqpoint{8.455078in}{0.550000in}}%
\pgfpathclose%
\pgfusepath{fill}%
\end{pgfscope}%
\begin{pgfscope}%
\pgfpathrectangle{\pgfqpoint{1.250000in}{0.550000in}}{\pgfqpoint{7.750000in}{3.850000in}}%
\pgfusepath{clip}%
\pgfsetbuttcap%
\pgfsetmiterjoin%
\definecolor{currentfill}{rgb}{0.501961,0.501961,0.501961}%
\pgfsetfillcolor{currentfill}%
\pgfsetfillopacity{0.700000}%
\pgfsetlinewidth{0.000000pt}%
\definecolor{currentstroke}{rgb}{0.000000,0.000000,0.000000}%
\pgfsetstrokecolor{currentstroke}%
\pgfsetstrokeopacity{0.700000}%
\pgfsetdash{}{0pt}%
\pgfpathmoveto{\pgfqpoint{8.482599in}{0.550000in}}%
\pgfpathlineto{\pgfqpoint{8.510121in}{0.550000in}}%
\pgfpathlineto{\pgfqpoint{8.510121in}{0.550000in}}%
\pgfpathlineto{\pgfqpoint{8.482599in}{0.550000in}}%
\pgfpathlineto{\pgfqpoint{8.482599in}{0.550000in}}%
\pgfpathclose%
\pgfusepath{fill}%
\end{pgfscope}%
\begin{pgfscope}%
\pgfpathrectangle{\pgfqpoint{1.250000in}{0.550000in}}{\pgfqpoint{7.750000in}{3.850000in}}%
\pgfusepath{clip}%
\pgfsetbuttcap%
\pgfsetmiterjoin%
\definecolor{currentfill}{rgb}{0.501961,0.501961,0.501961}%
\pgfsetfillcolor{currentfill}%
\pgfsetfillopacity{0.700000}%
\pgfsetlinewidth{0.000000pt}%
\definecolor{currentstroke}{rgb}{0.000000,0.000000,0.000000}%
\pgfsetstrokecolor{currentstroke}%
\pgfsetstrokeopacity{0.700000}%
\pgfsetdash{}{0pt}%
\pgfpathmoveto{\pgfqpoint{8.510121in}{0.550000in}}%
\pgfpathlineto{\pgfqpoint{8.537642in}{0.550000in}}%
\pgfpathlineto{\pgfqpoint{8.537642in}{0.550000in}}%
\pgfpathlineto{\pgfqpoint{8.510121in}{0.550000in}}%
\pgfpathlineto{\pgfqpoint{8.510121in}{0.550000in}}%
\pgfpathclose%
\pgfusepath{fill}%
\end{pgfscope}%
\begin{pgfscope}%
\pgfpathrectangle{\pgfqpoint{1.250000in}{0.550000in}}{\pgfqpoint{7.750000in}{3.850000in}}%
\pgfusepath{clip}%
\pgfsetbuttcap%
\pgfsetmiterjoin%
\definecolor{currentfill}{rgb}{0.501961,0.501961,0.501961}%
\pgfsetfillcolor{currentfill}%
\pgfsetfillopacity{0.700000}%
\pgfsetlinewidth{0.000000pt}%
\definecolor{currentstroke}{rgb}{0.000000,0.000000,0.000000}%
\pgfsetstrokecolor{currentstroke}%
\pgfsetstrokeopacity{0.700000}%
\pgfsetdash{}{0pt}%
\pgfpathmoveto{\pgfqpoint{8.537642in}{0.550000in}}%
\pgfpathlineto{\pgfqpoint{8.565163in}{0.550000in}}%
\pgfpathlineto{\pgfqpoint{8.565163in}{0.550000in}}%
\pgfpathlineto{\pgfqpoint{8.537642in}{0.550000in}}%
\pgfpathlineto{\pgfqpoint{8.537642in}{0.550000in}}%
\pgfpathclose%
\pgfusepath{fill}%
\end{pgfscope}%
\begin{pgfscope}%
\pgfpathrectangle{\pgfqpoint{1.250000in}{0.550000in}}{\pgfqpoint{7.750000in}{3.850000in}}%
\pgfusepath{clip}%
\pgfsetbuttcap%
\pgfsetmiterjoin%
\definecolor{currentfill}{rgb}{0.501961,0.501961,0.501961}%
\pgfsetfillcolor{currentfill}%
\pgfsetfillopacity{0.700000}%
\pgfsetlinewidth{0.000000pt}%
\definecolor{currentstroke}{rgb}{0.000000,0.000000,0.000000}%
\pgfsetstrokecolor{currentstroke}%
\pgfsetstrokeopacity{0.700000}%
\pgfsetdash{}{0pt}%
\pgfpathmoveto{\pgfqpoint{8.565163in}{0.550000in}}%
\pgfpathlineto{\pgfqpoint{8.592685in}{0.550000in}}%
\pgfpathlineto{\pgfqpoint{8.592685in}{0.550306in}}%
\pgfpathlineto{\pgfqpoint{8.565163in}{0.550306in}}%
\pgfpathlineto{\pgfqpoint{8.565163in}{0.550000in}}%
\pgfpathclose%
\pgfusepath{fill}%
\end{pgfscope}%
\begin{pgfscope}%
\pgfpathrectangle{\pgfqpoint{1.250000in}{0.550000in}}{\pgfqpoint{7.750000in}{3.850000in}}%
\pgfusepath{clip}%
\pgfsetbuttcap%
\pgfsetmiterjoin%
\definecolor{currentfill}{rgb}{0.501961,0.501961,0.501961}%
\pgfsetfillcolor{currentfill}%
\pgfsetfillopacity{0.700000}%
\pgfsetlinewidth{0.000000pt}%
\definecolor{currentstroke}{rgb}{0.000000,0.000000,0.000000}%
\pgfsetstrokecolor{currentstroke}%
\pgfsetstrokeopacity{0.700000}%
\pgfsetdash{}{0pt}%
\pgfpathmoveto{\pgfqpoint{8.592685in}{0.550000in}}%
\pgfpathlineto{\pgfqpoint{8.620206in}{0.550000in}}%
\pgfpathlineto{\pgfqpoint{8.620206in}{0.550153in}}%
\pgfpathlineto{\pgfqpoint{8.592685in}{0.550153in}}%
\pgfpathlineto{\pgfqpoint{8.592685in}{0.550000in}}%
\pgfpathclose%
\pgfusepath{fill}%
\end{pgfscope}%
\begin{pgfscope}%
\pgfpathrectangle{\pgfqpoint{1.250000in}{0.550000in}}{\pgfqpoint{7.750000in}{3.850000in}}%
\pgfusepath{clip}%
\pgfsetbuttcap%
\pgfsetmiterjoin%
\definecolor{currentfill}{rgb}{0.501961,0.501961,0.501961}%
\pgfsetfillcolor{currentfill}%
\pgfsetfillopacity{0.700000}%
\pgfsetlinewidth{0.000000pt}%
\definecolor{currentstroke}{rgb}{0.000000,0.000000,0.000000}%
\pgfsetstrokecolor{currentstroke}%
\pgfsetstrokeopacity{0.700000}%
\pgfsetdash{}{0pt}%
\pgfpathmoveto{\pgfqpoint{8.620206in}{0.550000in}}%
\pgfpathlineto{\pgfqpoint{8.647727in}{0.550000in}}%
\pgfpathlineto{\pgfqpoint{8.647727in}{0.550612in}}%
\pgfpathlineto{\pgfqpoint{8.620206in}{0.550612in}}%
\pgfpathlineto{\pgfqpoint{8.620206in}{0.550000in}}%
\pgfpathclose%
\pgfusepath{fill}%
\end{pgfscope}%
\begin{pgfscope}%
\pgfsetbuttcap%
\pgfsetroundjoin%
\definecolor{currentfill}{rgb}{0.000000,0.000000,0.000000}%
\pgfsetfillcolor{currentfill}%
\pgfsetlinewidth{0.803000pt}%
\definecolor{currentstroke}{rgb}{0.000000,0.000000,0.000000}%
\pgfsetstrokecolor{currentstroke}%
\pgfsetdash{}{0pt}%
\pgfsys@defobject{currentmarker}{\pgfqpoint{0.000000in}{-0.048611in}}{\pgfqpoint{0.000000in}{0.000000in}}{%
\pgfpathmoveto{\pgfqpoint{0.000000in}{0.000000in}}%
\pgfpathlineto{\pgfqpoint{0.000000in}{-0.048611in}}%
\pgfusepath{stroke,fill}%
}%
\begin{pgfscope}%
\pgfsys@transformshift{1.602273in}{0.550000in}%
\pgfsys@useobject{currentmarker}{}%
\end{pgfscope}%
\end{pgfscope}%
\begin{pgfscope}%
\definecolor{textcolor}{rgb}{0.000000,0.000000,0.000000}%
\pgfsetstrokecolor{textcolor}%
\pgfsetfillcolor{textcolor}%
\pgftext[x=1.602273in,y=0.452778in,,top]{\color{textcolor}{\sffamily\fontsize{10.000000}{12.000000}\selectfont\catcode`\^=\active\def^{\ifmmode\sp\else\^{}\fi}\catcode`\%=\active\def%{\%}0}}%
\end{pgfscope}%
\begin{pgfscope}%
\pgfsetbuttcap%
\pgfsetroundjoin%
\definecolor{currentfill}{rgb}{0.000000,0.000000,0.000000}%
\pgfsetfillcolor{currentfill}%
\pgfsetlinewidth{0.803000pt}%
\definecolor{currentstroke}{rgb}{0.000000,0.000000,0.000000}%
\pgfsetstrokecolor{currentstroke}%
\pgfsetdash{}{0pt}%
\pgfsys@defobject{currentmarker}{\pgfqpoint{0.000000in}{-0.048611in}}{\pgfqpoint{0.000000in}{0.000000in}}{%
\pgfpathmoveto{\pgfqpoint{0.000000in}{0.000000in}}%
\pgfpathlineto{\pgfqpoint{0.000000in}{-0.048611in}}%
\pgfusepath{stroke,fill}%
}%
\begin{pgfscope}%
\pgfsys@transformshift{2.978338in}{0.550000in}%
\pgfsys@useobject{currentmarker}{}%
\end{pgfscope}%
\end{pgfscope}%
\begin{pgfscope}%
\definecolor{textcolor}{rgb}{0.000000,0.000000,0.000000}%
\pgfsetstrokecolor{textcolor}%
\pgfsetfillcolor{textcolor}%
\pgftext[x=2.978338in,y=0.452778in,,top]{\color{textcolor}{\sffamily\fontsize{10.000000}{12.000000}\selectfont\catcode`\^=\active\def^{\ifmmode\sp\else\^{}\fi}\catcode`\%=\active\def%{\%}50}}%
\end{pgfscope}%
\begin{pgfscope}%
\pgfsetbuttcap%
\pgfsetroundjoin%
\definecolor{currentfill}{rgb}{0.000000,0.000000,0.000000}%
\pgfsetfillcolor{currentfill}%
\pgfsetlinewidth{0.803000pt}%
\definecolor{currentstroke}{rgb}{0.000000,0.000000,0.000000}%
\pgfsetstrokecolor{currentstroke}%
\pgfsetdash{}{0pt}%
\pgfsys@defobject{currentmarker}{\pgfqpoint{0.000000in}{-0.048611in}}{\pgfqpoint{0.000000in}{0.000000in}}{%
\pgfpathmoveto{\pgfqpoint{0.000000in}{0.000000in}}%
\pgfpathlineto{\pgfqpoint{0.000000in}{-0.048611in}}%
\pgfusepath{stroke,fill}%
}%
\begin{pgfscope}%
\pgfsys@transformshift{4.354403in}{0.550000in}%
\pgfsys@useobject{currentmarker}{}%
\end{pgfscope}%
\end{pgfscope}%
\begin{pgfscope}%
\definecolor{textcolor}{rgb}{0.000000,0.000000,0.000000}%
\pgfsetstrokecolor{textcolor}%
\pgfsetfillcolor{textcolor}%
\pgftext[x=4.354403in,y=0.452778in,,top]{\color{textcolor}{\sffamily\fontsize{10.000000}{12.000000}\selectfont\catcode`\^=\active\def^{\ifmmode\sp\else\^{}\fi}\catcode`\%=\active\def%{\%}100}}%
\end{pgfscope}%
\begin{pgfscope}%
\pgfsetbuttcap%
\pgfsetroundjoin%
\definecolor{currentfill}{rgb}{0.000000,0.000000,0.000000}%
\pgfsetfillcolor{currentfill}%
\pgfsetlinewidth{0.803000pt}%
\definecolor{currentstroke}{rgb}{0.000000,0.000000,0.000000}%
\pgfsetstrokecolor{currentstroke}%
\pgfsetdash{}{0pt}%
\pgfsys@defobject{currentmarker}{\pgfqpoint{0.000000in}{-0.048611in}}{\pgfqpoint{0.000000in}{0.000000in}}{%
\pgfpathmoveto{\pgfqpoint{0.000000in}{0.000000in}}%
\pgfpathlineto{\pgfqpoint{0.000000in}{-0.048611in}}%
\pgfusepath{stroke,fill}%
}%
\begin{pgfscope}%
\pgfsys@transformshift{5.730469in}{0.550000in}%
\pgfsys@useobject{currentmarker}{}%
\end{pgfscope}%
\end{pgfscope}%
\begin{pgfscope}%
\definecolor{textcolor}{rgb}{0.000000,0.000000,0.000000}%
\pgfsetstrokecolor{textcolor}%
\pgfsetfillcolor{textcolor}%
\pgftext[x=5.730469in,y=0.452778in,,top]{\color{textcolor}{\sffamily\fontsize{10.000000}{12.000000}\selectfont\catcode`\^=\active\def^{\ifmmode\sp\else\^{}\fi}\catcode`\%=\active\def%{\%}150}}%
\end{pgfscope}%
\begin{pgfscope}%
\pgfsetbuttcap%
\pgfsetroundjoin%
\definecolor{currentfill}{rgb}{0.000000,0.000000,0.000000}%
\pgfsetfillcolor{currentfill}%
\pgfsetlinewidth{0.803000pt}%
\definecolor{currentstroke}{rgb}{0.000000,0.000000,0.000000}%
\pgfsetstrokecolor{currentstroke}%
\pgfsetdash{}{0pt}%
\pgfsys@defobject{currentmarker}{\pgfqpoint{0.000000in}{-0.048611in}}{\pgfqpoint{0.000000in}{0.000000in}}{%
\pgfpathmoveto{\pgfqpoint{0.000000in}{0.000000in}}%
\pgfpathlineto{\pgfqpoint{0.000000in}{-0.048611in}}%
\pgfusepath{stroke,fill}%
}%
\begin{pgfscope}%
\pgfsys@transformshift{7.106534in}{0.550000in}%
\pgfsys@useobject{currentmarker}{}%
\end{pgfscope}%
\end{pgfscope}%
\begin{pgfscope}%
\definecolor{textcolor}{rgb}{0.000000,0.000000,0.000000}%
\pgfsetstrokecolor{textcolor}%
\pgfsetfillcolor{textcolor}%
\pgftext[x=7.106534in,y=0.452778in,,top]{\color{textcolor}{\sffamily\fontsize{10.000000}{12.000000}\selectfont\catcode`\^=\active\def^{\ifmmode\sp\else\^{}\fi}\catcode`\%=\active\def%{\%}200}}%
\end{pgfscope}%
\begin{pgfscope}%
\pgfsetbuttcap%
\pgfsetroundjoin%
\definecolor{currentfill}{rgb}{0.000000,0.000000,0.000000}%
\pgfsetfillcolor{currentfill}%
\pgfsetlinewidth{0.803000pt}%
\definecolor{currentstroke}{rgb}{0.000000,0.000000,0.000000}%
\pgfsetstrokecolor{currentstroke}%
\pgfsetdash{}{0pt}%
\pgfsys@defobject{currentmarker}{\pgfqpoint{0.000000in}{-0.048611in}}{\pgfqpoint{0.000000in}{0.000000in}}{%
\pgfpathmoveto{\pgfqpoint{0.000000in}{0.000000in}}%
\pgfpathlineto{\pgfqpoint{0.000000in}{-0.048611in}}%
\pgfusepath{stroke,fill}%
}%
\begin{pgfscope}%
\pgfsys@transformshift{8.482599in}{0.550000in}%
\pgfsys@useobject{currentmarker}{}%
\end{pgfscope}%
\end{pgfscope}%
\begin{pgfscope}%
\definecolor{textcolor}{rgb}{0.000000,0.000000,0.000000}%
\pgfsetstrokecolor{textcolor}%
\pgfsetfillcolor{textcolor}%
\pgftext[x=8.482599in,y=0.452778in,,top]{\color{textcolor}{\sffamily\fontsize{10.000000}{12.000000}\selectfont\catcode`\^=\active\def^{\ifmmode\sp\else\^{}\fi}\catcode`\%=\active\def%{\%}250}}%
\end{pgfscope}%
\begin{pgfscope}%
\definecolor{textcolor}{rgb}{0.000000,0.000000,0.000000}%
\pgfsetstrokecolor{textcolor}%
\pgfsetfillcolor{textcolor}%
\pgftext[x=5.125000in,y=0.262809in,,top]{\color{textcolor}{\sffamily\fontsize{10.000000}{12.000000}\selectfont\catcode`\^=\active\def^{\ifmmode\sp\else\^{}\fi}\catcode`\%=\active\def%{\%}Intensidad}}%
\end{pgfscope}%
\begin{pgfscope}%
\pgfsetbuttcap%
\pgfsetroundjoin%
\definecolor{currentfill}{rgb}{0.000000,0.000000,0.000000}%
\pgfsetfillcolor{currentfill}%
\pgfsetlinewidth{0.803000pt}%
\definecolor{currentstroke}{rgb}{0.000000,0.000000,0.000000}%
\pgfsetstrokecolor{currentstroke}%
\pgfsetdash{}{0pt}%
\pgfsys@defobject{currentmarker}{\pgfqpoint{-0.048611in}{0.000000in}}{\pgfqpoint{-0.000000in}{0.000000in}}{%
\pgfpathmoveto{\pgfqpoint{-0.000000in}{0.000000in}}%
\pgfpathlineto{\pgfqpoint{-0.048611in}{0.000000in}}%
\pgfusepath{stroke,fill}%
}%
\begin{pgfscope}%
\pgfsys@transformshift{1.250000in}{0.550000in}%
\pgfsys@useobject{currentmarker}{}%
\end{pgfscope}%
\end{pgfscope}%
\begin{pgfscope}%
\definecolor{textcolor}{rgb}{0.000000,0.000000,0.000000}%
\pgfsetstrokecolor{textcolor}%
\pgfsetfillcolor{textcolor}%
\pgftext[x=1.064412in, y=0.497238in, left, base]{\color{textcolor}{\sffamily\fontsize{10.000000}{12.000000}\selectfont\catcode`\^=\active\def^{\ifmmode\sp\else\^{}\fi}\catcode`\%=\active\def%{\%}0}}%
\end{pgfscope}%
\begin{pgfscope}%
\pgfsetbuttcap%
\pgfsetroundjoin%
\definecolor{currentfill}{rgb}{0.000000,0.000000,0.000000}%
\pgfsetfillcolor{currentfill}%
\pgfsetlinewidth{0.803000pt}%
\definecolor{currentstroke}{rgb}{0.000000,0.000000,0.000000}%
\pgfsetstrokecolor{currentstroke}%
\pgfsetdash{}{0pt}%
\pgfsys@defobject{currentmarker}{\pgfqpoint{-0.048611in}{0.000000in}}{\pgfqpoint{-0.000000in}{0.000000in}}{%
\pgfpathmoveto{\pgfqpoint{-0.000000in}{0.000000in}}%
\pgfpathlineto{\pgfqpoint{-0.048611in}{0.000000in}}%
\pgfusepath{stroke,fill}%
}%
\begin{pgfscope}%
\pgfsys@transformshift{1.250000in}{1.315548in}%
\pgfsys@useobject{currentmarker}{}%
\end{pgfscope}%
\end{pgfscope}%
\begin{pgfscope}%
\definecolor{textcolor}{rgb}{0.000000,0.000000,0.000000}%
\pgfsetstrokecolor{textcolor}%
\pgfsetfillcolor{textcolor}%
\pgftext[x=0.799316in, y=1.262786in, left, base]{\color{textcolor}{\sffamily\fontsize{10.000000}{12.000000}\selectfont\catcode`\^=\active\def^{\ifmmode\sp\else\^{}\fi}\catcode`\%=\active\def%{\%}5000}}%
\end{pgfscope}%
\begin{pgfscope}%
\pgfsetbuttcap%
\pgfsetroundjoin%
\definecolor{currentfill}{rgb}{0.000000,0.000000,0.000000}%
\pgfsetfillcolor{currentfill}%
\pgfsetlinewidth{0.803000pt}%
\definecolor{currentstroke}{rgb}{0.000000,0.000000,0.000000}%
\pgfsetstrokecolor{currentstroke}%
\pgfsetdash{}{0pt}%
\pgfsys@defobject{currentmarker}{\pgfqpoint{-0.048611in}{0.000000in}}{\pgfqpoint{-0.000000in}{0.000000in}}{%
\pgfpathmoveto{\pgfqpoint{-0.000000in}{0.000000in}}%
\pgfpathlineto{\pgfqpoint{-0.048611in}{0.000000in}}%
\pgfusepath{stroke,fill}%
}%
\begin{pgfscope}%
\pgfsys@transformshift{1.250000in}{2.081095in}%
\pgfsys@useobject{currentmarker}{}%
\end{pgfscope}%
\end{pgfscope}%
\begin{pgfscope}%
\definecolor{textcolor}{rgb}{0.000000,0.000000,0.000000}%
\pgfsetstrokecolor{textcolor}%
\pgfsetfillcolor{textcolor}%
\pgftext[x=0.710951in, y=2.028334in, left, base]{\color{textcolor}{\sffamily\fontsize{10.000000}{12.000000}\selectfont\catcode`\^=\active\def^{\ifmmode\sp\else\^{}\fi}\catcode`\%=\active\def%{\%}10000}}%
\end{pgfscope}%
\begin{pgfscope}%
\pgfsetbuttcap%
\pgfsetroundjoin%
\definecolor{currentfill}{rgb}{0.000000,0.000000,0.000000}%
\pgfsetfillcolor{currentfill}%
\pgfsetlinewidth{0.803000pt}%
\definecolor{currentstroke}{rgb}{0.000000,0.000000,0.000000}%
\pgfsetstrokecolor{currentstroke}%
\pgfsetdash{}{0pt}%
\pgfsys@defobject{currentmarker}{\pgfqpoint{-0.048611in}{0.000000in}}{\pgfqpoint{-0.000000in}{0.000000in}}{%
\pgfpathmoveto{\pgfqpoint{-0.000000in}{0.000000in}}%
\pgfpathlineto{\pgfqpoint{-0.048611in}{0.000000in}}%
\pgfusepath{stroke,fill}%
}%
\begin{pgfscope}%
\pgfsys@transformshift{1.250000in}{2.846643in}%
\pgfsys@useobject{currentmarker}{}%
\end{pgfscope}%
\end{pgfscope}%
\begin{pgfscope}%
\definecolor{textcolor}{rgb}{0.000000,0.000000,0.000000}%
\pgfsetstrokecolor{textcolor}%
\pgfsetfillcolor{textcolor}%
\pgftext[x=0.710951in, y=2.793881in, left, base]{\color{textcolor}{\sffamily\fontsize{10.000000}{12.000000}\selectfont\catcode`\^=\active\def^{\ifmmode\sp\else\^{}\fi}\catcode`\%=\active\def%{\%}15000}}%
\end{pgfscope}%
\begin{pgfscope}%
\pgfsetbuttcap%
\pgfsetroundjoin%
\definecolor{currentfill}{rgb}{0.000000,0.000000,0.000000}%
\pgfsetfillcolor{currentfill}%
\pgfsetlinewidth{0.803000pt}%
\definecolor{currentstroke}{rgb}{0.000000,0.000000,0.000000}%
\pgfsetstrokecolor{currentstroke}%
\pgfsetdash{}{0pt}%
\pgfsys@defobject{currentmarker}{\pgfqpoint{-0.048611in}{0.000000in}}{\pgfqpoint{-0.000000in}{0.000000in}}{%
\pgfpathmoveto{\pgfqpoint{-0.000000in}{0.000000in}}%
\pgfpathlineto{\pgfqpoint{-0.048611in}{0.000000in}}%
\pgfusepath{stroke,fill}%
}%
\begin{pgfscope}%
\pgfsys@transformshift{1.250000in}{3.612190in}%
\pgfsys@useobject{currentmarker}{}%
\end{pgfscope}%
\end{pgfscope}%
\begin{pgfscope}%
\definecolor{textcolor}{rgb}{0.000000,0.000000,0.000000}%
\pgfsetstrokecolor{textcolor}%
\pgfsetfillcolor{textcolor}%
\pgftext[x=0.710951in, y=3.559429in, left, base]{\color{textcolor}{\sffamily\fontsize{10.000000}{12.000000}\selectfont\catcode`\^=\active\def^{\ifmmode\sp\else\^{}\fi}\catcode`\%=\active\def%{\%}20000}}%
\end{pgfscope}%
\begin{pgfscope}%
\pgfsetbuttcap%
\pgfsetroundjoin%
\definecolor{currentfill}{rgb}{0.000000,0.000000,0.000000}%
\pgfsetfillcolor{currentfill}%
\pgfsetlinewidth{0.803000pt}%
\definecolor{currentstroke}{rgb}{0.000000,0.000000,0.000000}%
\pgfsetstrokecolor{currentstroke}%
\pgfsetdash{}{0pt}%
\pgfsys@defobject{currentmarker}{\pgfqpoint{-0.048611in}{0.000000in}}{\pgfqpoint{-0.000000in}{0.000000in}}{%
\pgfpathmoveto{\pgfqpoint{-0.000000in}{0.000000in}}%
\pgfpathlineto{\pgfqpoint{-0.048611in}{0.000000in}}%
\pgfusepath{stroke,fill}%
}%
\begin{pgfscope}%
\pgfsys@transformshift{1.250000in}{4.377738in}%
\pgfsys@useobject{currentmarker}{}%
\end{pgfscope}%
\end{pgfscope}%
\begin{pgfscope}%
\definecolor{textcolor}{rgb}{0.000000,0.000000,0.000000}%
\pgfsetstrokecolor{textcolor}%
\pgfsetfillcolor{textcolor}%
\pgftext[x=0.710951in, y=4.324976in, left, base]{\color{textcolor}{\sffamily\fontsize{10.000000}{12.000000}\selectfont\catcode`\^=\active\def^{\ifmmode\sp\else\^{}\fi}\catcode`\%=\active\def%{\%}25000}}%
\end{pgfscope}%
\begin{pgfscope}%
\definecolor{textcolor}{rgb}{0.000000,0.000000,0.000000}%
\pgfsetstrokecolor{textcolor}%
\pgfsetfillcolor{textcolor}%
\pgftext[x=0.655396in,y=2.475000in,,bottom,rotate=90.000000]{\color{textcolor}{\sffamily\fontsize{10.000000}{12.000000}\selectfont\catcode`\^=\active\def^{\ifmmode\sp\else\^{}\fi}\catcode`\%=\active\def%{\%}Frecuencia}}%
\end{pgfscope}%
\begin{pgfscope}%
\pgfsetrectcap%
\pgfsetmiterjoin%
\pgfsetlinewidth{0.803000pt}%
\definecolor{currentstroke}{rgb}{0.000000,0.000000,0.000000}%
\pgfsetstrokecolor{currentstroke}%
\pgfsetdash{}{0pt}%
\pgfpathmoveto{\pgfqpoint{1.250000in}{0.550000in}}%
\pgfpathlineto{\pgfqpoint{1.250000in}{4.400000in}}%
\pgfusepath{stroke}%
\end{pgfscope}%
\begin{pgfscope}%
\pgfsetrectcap%
\pgfsetmiterjoin%
\pgfsetlinewidth{0.803000pt}%
\definecolor{currentstroke}{rgb}{0.000000,0.000000,0.000000}%
\pgfsetstrokecolor{currentstroke}%
\pgfsetdash{}{0pt}%
\pgfpathmoveto{\pgfqpoint{9.000000in}{0.550000in}}%
\pgfpathlineto{\pgfqpoint{9.000000in}{4.400000in}}%
\pgfusepath{stroke}%
\end{pgfscope}%
\begin{pgfscope}%
\pgfsetrectcap%
\pgfsetmiterjoin%
\pgfsetlinewidth{0.803000pt}%
\definecolor{currentstroke}{rgb}{0.000000,0.000000,0.000000}%
\pgfsetstrokecolor{currentstroke}%
\pgfsetdash{}{0pt}%
\pgfpathmoveto{\pgfqpoint{1.250000in}{0.550000in}}%
\pgfpathlineto{\pgfqpoint{9.000000in}{0.550000in}}%
\pgfusepath{stroke}%
\end{pgfscope}%
\begin{pgfscope}%
\pgfsetrectcap%
\pgfsetmiterjoin%
\pgfsetlinewidth{0.803000pt}%
\definecolor{currentstroke}{rgb}{0.000000,0.000000,0.000000}%
\pgfsetstrokecolor{currentstroke}%
\pgfsetdash{}{0pt}%
\pgfpathmoveto{\pgfqpoint{1.250000in}{4.400000in}}%
\pgfpathlineto{\pgfqpoint{9.000000in}{4.400000in}}%
\pgfusepath{stroke}%
\end{pgfscope}%
\begin{pgfscope}%
\definecolor{textcolor}{rgb}{0.000000,0.000000,0.000000}%
\pgfsetstrokecolor{textcolor}%
\pgfsetfillcolor{textcolor}%
\pgftext[x=5.125000in,y=4.483333in,,base]{\color{textcolor}{\sffamily\fontsize{12.000000}{14.400000}\selectfont\catcode`\^=\active\def^{\ifmmode\sp\else\^{}\fi}\catcode`\%=\active\def%{\%}Histograma de CLAHE}}%
\end{pgfscope}%
\end{pgfpicture}%
\makeatother%
\endgroup%
}
		\caption{CLAHE}
		\label{fig:clahe_2}
	\end{subfigure}

	\vspace{5mm}

	\centering
	\begin{subfigure}[b]{0.45\textwidth}
		\resizebox{\linewidth}{!}{%% Creator: Matplotlib, PGF backend
%%
%% To include the figure in your LaTeX document, write
%%   \input{<filename>.pgf}
%%
%% Make sure the required packages are loaded in your preamble
%%   \usepackage{pgf}
%%
%% Also ensure that all the required font packages are loaded; for instance,
%% the lmodern package is sometimes necessary when using math font.
%%   \usepackage{lmodern}
%%
%% Figures using additional raster images can only be included by \input if
%% they are in the same directory as the main LaTeX file. For loading figures
%% from other directories you can use the `import` package
%%   \usepackage{import}
%%
%% and then include the figures with
%%   \import{<path to file>}{<filename>.pgf}
%%
%% Matplotlib used the following preamble
%%   \def\mathdefault#1{#1}
%%   \everymath=\expandafter{\the\everymath\displaystyle}
%%   \IfFileExists{scrextend.sty}{
%%     \usepackage[fontsize=10.000000pt]{scrextend}
%%   }{
%%     \renewcommand{\normalsize}{\fontsize{10.000000}{12.000000}\selectfont}
%%     \normalsize
%%   }
%%   
%%   \ifdefined\pdftexversion\else  % non-pdftex case.
%%     \usepackage{fontspec}
%%     \setmainfont{DejaVuSerif.ttf}[Path=\detokenize{/home/elias/.virtualenvs/src-pway/lib/python3.13/site-packages/matplotlib/mpl-data/fonts/ttf/}]
%%     \setsansfont{DejaVuSans.ttf}[Path=\detokenize{/home/elias/.virtualenvs/src-pway/lib/python3.13/site-packages/matplotlib/mpl-data/fonts/ttf/}]
%%     \setmonofont{DejaVuSansMono.ttf}[Path=\detokenize{/home/elias/.virtualenvs/src-pway/lib/python3.13/site-packages/matplotlib/mpl-data/fonts/ttf/}]
%%   \fi
%%   \makeatletter\@ifpackageloaded{underscore}{}{\usepackage[strings]{underscore}}\makeatother
%%
\begingroup%
\makeatletter%
\begin{pgfpicture}%
\pgfpathrectangle{\pgfpointorigin}{\pgfqpoint{10.000000in}{5.000000in}}%
\pgfusepath{use as bounding box, clip}%
\begin{pgfscope}%
\pgfsetbuttcap%
\pgfsetmiterjoin%
\definecolor{currentfill}{rgb}{1.000000,1.000000,1.000000}%
\pgfsetfillcolor{currentfill}%
\pgfsetlinewidth{0.000000pt}%
\definecolor{currentstroke}{rgb}{1.000000,1.000000,1.000000}%
\pgfsetstrokecolor{currentstroke}%
\pgfsetdash{}{0pt}%
\pgfpathmoveto{\pgfqpoint{0.000000in}{0.000000in}}%
\pgfpathlineto{\pgfqpoint{10.000000in}{0.000000in}}%
\pgfpathlineto{\pgfqpoint{10.000000in}{5.000000in}}%
\pgfpathlineto{\pgfqpoint{0.000000in}{5.000000in}}%
\pgfpathlineto{\pgfqpoint{0.000000in}{0.000000in}}%
\pgfpathclose%
\pgfusepath{fill}%
\end{pgfscope}%
\begin{pgfscope}%
\pgfsetbuttcap%
\pgfsetmiterjoin%
\definecolor{currentfill}{rgb}{1.000000,1.000000,1.000000}%
\pgfsetfillcolor{currentfill}%
\pgfsetlinewidth{0.000000pt}%
\definecolor{currentstroke}{rgb}{0.000000,0.000000,0.000000}%
\pgfsetstrokecolor{currentstroke}%
\pgfsetstrokeopacity{0.000000}%
\pgfsetdash{}{0pt}%
\pgfpathmoveto{\pgfqpoint{1.250000in}{0.550000in}}%
\pgfpathlineto{\pgfqpoint{9.000000in}{0.550000in}}%
\pgfpathlineto{\pgfqpoint{9.000000in}{4.400000in}}%
\pgfpathlineto{\pgfqpoint{1.250000in}{4.400000in}}%
\pgfpathlineto{\pgfqpoint{1.250000in}{0.550000in}}%
\pgfpathclose%
\pgfusepath{fill}%
\end{pgfscope}%
\begin{pgfscope}%
\pgfpathrectangle{\pgfqpoint{1.250000in}{0.550000in}}{\pgfqpoint{7.750000in}{3.850000in}}%
\pgfusepath{clip}%
\pgfsetbuttcap%
\pgfsetmiterjoin%
\definecolor{currentfill}{rgb}{0.501961,0.501961,0.501961}%
\pgfsetfillcolor{currentfill}%
\pgfsetfillopacity{0.700000}%
\pgfsetlinewidth{0.000000pt}%
\definecolor{currentstroke}{rgb}{0.000000,0.000000,0.000000}%
\pgfsetstrokecolor{currentstroke}%
\pgfsetstrokeopacity{0.700000}%
\pgfsetdash{}{0pt}%
\pgfpathmoveto{\pgfqpoint{1.602273in}{0.550000in}}%
\pgfpathlineto{\pgfqpoint{1.629794in}{0.550000in}}%
\pgfpathlineto{\pgfqpoint{1.629794in}{0.550000in}}%
\pgfpathlineto{\pgfqpoint{1.602273in}{0.550000in}}%
\pgfpathlineto{\pgfqpoint{1.602273in}{0.550000in}}%
\pgfpathclose%
\pgfusepath{fill}%
\end{pgfscope}%
\begin{pgfscope}%
\pgfpathrectangle{\pgfqpoint{1.250000in}{0.550000in}}{\pgfqpoint{7.750000in}{3.850000in}}%
\pgfusepath{clip}%
\pgfsetbuttcap%
\pgfsetmiterjoin%
\definecolor{currentfill}{rgb}{0.501961,0.501961,0.501961}%
\pgfsetfillcolor{currentfill}%
\pgfsetfillopacity{0.700000}%
\pgfsetlinewidth{0.000000pt}%
\definecolor{currentstroke}{rgb}{0.000000,0.000000,0.000000}%
\pgfsetstrokecolor{currentstroke}%
\pgfsetstrokeopacity{0.700000}%
\pgfsetdash{}{0pt}%
\pgfpathmoveto{\pgfqpoint{1.629794in}{0.550000in}}%
\pgfpathlineto{\pgfqpoint{1.657315in}{0.550000in}}%
\pgfpathlineto{\pgfqpoint{1.657315in}{1.579420in}}%
\pgfpathlineto{\pgfqpoint{1.629794in}{1.579420in}}%
\pgfpathlineto{\pgfqpoint{1.629794in}{0.550000in}}%
\pgfpathclose%
\pgfusepath{fill}%
\end{pgfscope}%
\begin{pgfscope}%
\pgfpathrectangle{\pgfqpoint{1.250000in}{0.550000in}}{\pgfqpoint{7.750000in}{3.850000in}}%
\pgfusepath{clip}%
\pgfsetbuttcap%
\pgfsetmiterjoin%
\definecolor{currentfill}{rgb}{0.501961,0.501961,0.501961}%
\pgfsetfillcolor{currentfill}%
\pgfsetfillopacity{0.700000}%
\pgfsetlinewidth{0.000000pt}%
\definecolor{currentstroke}{rgb}{0.000000,0.000000,0.000000}%
\pgfsetstrokecolor{currentstroke}%
\pgfsetstrokeopacity{0.700000}%
\pgfsetdash{}{0pt}%
\pgfpathmoveto{\pgfqpoint{1.657315in}{0.550000in}}%
\pgfpathlineto{\pgfqpoint{1.684837in}{0.550000in}}%
\pgfpathlineto{\pgfqpoint{1.684837in}{4.216667in}}%
\pgfpathlineto{\pgfqpoint{1.657315in}{4.216667in}}%
\pgfpathlineto{\pgfqpoint{1.657315in}{0.550000in}}%
\pgfpathclose%
\pgfusepath{fill}%
\end{pgfscope}%
\begin{pgfscope}%
\pgfpathrectangle{\pgfqpoint{1.250000in}{0.550000in}}{\pgfqpoint{7.750000in}{3.850000in}}%
\pgfusepath{clip}%
\pgfsetbuttcap%
\pgfsetmiterjoin%
\definecolor{currentfill}{rgb}{0.501961,0.501961,0.501961}%
\pgfsetfillcolor{currentfill}%
\pgfsetfillopacity{0.700000}%
\pgfsetlinewidth{0.000000pt}%
\definecolor{currentstroke}{rgb}{0.000000,0.000000,0.000000}%
\pgfsetstrokecolor{currentstroke}%
\pgfsetstrokeopacity{0.700000}%
\pgfsetdash{}{0pt}%
\pgfpathmoveto{\pgfqpoint{1.684837in}{0.550000in}}%
\pgfpathlineto{\pgfqpoint{1.712358in}{0.550000in}}%
\pgfpathlineto{\pgfqpoint{1.712358in}{2.976377in}}%
\pgfpathlineto{\pgfqpoint{1.684837in}{2.976377in}}%
\pgfpathlineto{\pgfqpoint{1.684837in}{0.550000in}}%
\pgfpathclose%
\pgfusepath{fill}%
\end{pgfscope}%
\begin{pgfscope}%
\pgfpathrectangle{\pgfqpoint{1.250000in}{0.550000in}}{\pgfqpoint{7.750000in}{3.850000in}}%
\pgfusepath{clip}%
\pgfsetbuttcap%
\pgfsetmiterjoin%
\definecolor{currentfill}{rgb}{0.501961,0.501961,0.501961}%
\pgfsetfillcolor{currentfill}%
\pgfsetfillopacity{0.700000}%
\pgfsetlinewidth{0.000000pt}%
\definecolor{currentstroke}{rgb}{0.000000,0.000000,0.000000}%
\pgfsetstrokecolor{currentstroke}%
\pgfsetstrokeopacity{0.700000}%
\pgfsetdash{}{0pt}%
\pgfpathmoveto{\pgfqpoint{1.712358in}{0.550000in}}%
\pgfpathlineto{\pgfqpoint{1.739879in}{0.550000in}}%
\pgfpathlineto{\pgfqpoint{1.739879in}{1.805323in}}%
\pgfpathlineto{\pgfqpoint{1.712358in}{1.805323in}}%
\pgfpathlineto{\pgfqpoint{1.712358in}{0.550000in}}%
\pgfpathclose%
\pgfusepath{fill}%
\end{pgfscope}%
\begin{pgfscope}%
\pgfpathrectangle{\pgfqpoint{1.250000in}{0.550000in}}{\pgfqpoint{7.750000in}{3.850000in}}%
\pgfusepath{clip}%
\pgfsetbuttcap%
\pgfsetmiterjoin%
\definecolor{currentfill}{rgb}{0.501961,0.501961,0.501961}%
\pgfsetfillcolor{currentfill}%
\pgfsetfillopacity{0.700000}%
\pgfsetlinewidth{0.000000pt}%
\definecolor{currentstroke}{rgb}{0.000000,0.000000,0.000000}%
\pgfsetstrokecolor{currentstroke}%
\pgfsetstrokeopacity{0.700000}%
\pgfsetdash{}{0pt}%
\pgfpathmoveto{\pgfqpoint{1.739879in}{0.550000in}}%
\pgfpathlineto{\pgfqpoint{1.767401in}{0.550000in}}%
\pgfpathlineto{\pgfqpoint{1.767401in}{1.526011in}}%
\pgfpathlineto{\pgfqpoint{1.739879in}{1.526011in}}%
\pgfpathlineto{\pgfqpoint{1.739879in}{0.550000in}}%
\pgfpathclose%
\pgfusepath{fill}%
\end{pgfscope}%
\begin{pgfscope}%
\pgfpathrectangle{\pgfqpoint{1.250000in}{0.550000in}}{\pgfqpoint{7.750000in}{3.850000in}}%
\pgfusepath{clip}%
\pgfsetbuttcap%
\pgfsetmiterjoin%
\definecolor{currentfill}{rgb}{0.501961,0.501961,0.501961}%
\pgfsetfillcolor{currentfill}%
\pgfsetfillopacity{0.700000}%
\pgfsetlinewidth{0.000000pt}%
\definecolor{currentstroke}{rgb}{0.000000,0.000000,0.000000}%
\pgfsetstrokecolor{currentstroke}%
\pgfsetstrokeopacity{0.700000}%
\pgfsetdash{}{0pt}%
\pgfpathmoveto{\pgfqpoint{1.767401in}{0.550000in}}%
\pgfpathlineto{\pgfqpoint{1.794922in}{0.550000in}}%
\pgfpathlineto{\pgfqpoint{1.794922in}{1.324637in}}%
\pgfpathlineto{\pgfqpoint{1.767401in}{1.324637in}}%
\pgfpathlineto{\pgfqpoint{1.767401in}{0.550000in}}%
\pgfpathclose%
\pgfusepath{fill}%
\end{pgfscope}%
\begin{pgfscope}%
\pgfpathrectangle{\pgfqpoint{1.250000in}{0.550000in}}{\pgfqpoint{7.750000in}{3.850000in}}%
\pgfusepath{clip}%
\pgfsetbuttcap%
\pgfsetmiterjoin%
\definecolor{currentfill}{rgb}{0.501961,0.501961,0.501961}%
\pgfsetfillcolor{currentfill}%
\pgfsetfillopacity{0.700000}%
\pgfsetlinewidth{0.000000pt}%
\definecolor{currentstroke}{rgb}{0.000000,0.000000,0.000000}%
\pgfsetstrokecolor{currentstroke}%
\pgfsetstrokeopacity{0.700000}%
\pgfsetdash{}{0pt}%
\pgfpathmoveto{\pgfqpoint{1.794922in}{0.550000in}}%
\pgfpathlineto{\pgfqpoint{1.822443in}{0.550000in}}%
\pgfpathlineto{\pgfqpoint{1.822443in}{1.264106in}}%
\pgfpathlineto{\pgfqpoint{1.794922in}{1.264106in}}%
\pgfpathlineto{\pgfqpoint{1.794922in}{0.550000in}}%
\pgfpathclose%
\pgfusepath{fill}%
\end{pgfscope}%
\begin{pgfscope}%
\pgfpathrectangle{\pgfqpoint{1.250000in}{0.550000in}}{\pgfqpoint{7.750000in}{3.850000in}}%
\pgfusepath{clip}%
\pgfsetbuttcap%
\pgfsetmiterjoin%
\definecolor{currentfill}{rgb}{0.501961,0.501961,0.501961}%
\pgfsetfillcolor{currentfill}%
\pgfsetfillopacity{0.700000}%
\pgfsetlinewidth{0.000000pt}%
\definecolor{currentstroke}{rgb}{0.000000,0.000000,0.000000}%
\pgfsetstrokecolor{currentstroke}%
\pgfsetstrokeopacity{0.700000}%
\pgfsetdash{}{0pt}%
\pgfpathmoveto{\pgfqpoint{1.822443in}{0.550000in}}%
\pgfpathlineto{\pgfqpoint{1.849964in}{0.550000in}}%
\pgfpathlineto{\pgfqpoint{1.849964in}{1.248281in}}%
\pgfpathlineto{\pgfqpoint{1.822443in}{1.248281in}}%
\pgfpathlineto{\pgfqpoint{1.822443in}{0.550000in}}%
\pgfpathclose%
\pgfusepath{fill}%
\end{pgfscope}%
\begin{pgfscope}%
\pgfpathrectangle{\pgfqpoint{1.250000in}{0.550000in}}{\pgfqpoint{7.750000in}{3.850000in}}%
\pgfusepath{clip}%
\pgfsetbuttcap%
\pgfsetmiterjoin%
\definecolor{currentfill}{rgb}{0.501961,0.501961,0.501961}%
\pgfsetfillcolor{currentfill}%
\pgfsetfillopacity{0.700000}%
\pgfsetlinewidth{0.000000pt}%
\definecolor{currentstroke}{rgb}{0.000000,0.000000,0.000000}%
\pgfsetstrokecolor{currentstroke}%
\pgfsetstrokeopacity{0.700000}%
\pgfsetdash{}{0pt}%
\pgfpathmoveto{\pgfqpoint{1.849964in}{0.550000in}}%
\pgfpathlineto{\pgfqpoint{1.877486in}{0.550000in}}%
\pgfpathlineto{\pgfqpoint{1.877486in}{1.290217in}}%
\pgfpathlineto{\pgfqpoint{1.849964in}{1.290217in}}%
\pgfpathlineto{\pgfqpoint{1.849964in}{0.550000in}}%
\pgfpathclose%
\pgfusepath{fill}%
\end{pgfscope}%
\begin{pgfscope}%
\pgfpathrectangle{\pgfqpoint{1.250000in}{0.550000in}}{\pgfqpoint{7.750000in}{3.850000in}}%
\pgfusepath{clip}%
\pgfsetbuttcap%
\pgfsetmiterjoin%
\definecolor{currentfill}{rgb}{0.501961,0.501961,0.501961}%
\pgfsetfillcolor{currentfill}%
\pgfsetfillopacity{0.700000}%
\pgfsetlinewidth{0.000000pt}%
\definecolor{currentstroke}{rgb}{0.000000,0.000000,0.000000}%
\pgfsetstrokecolor{currentstroke}%
\pgfsetstrokeopacity{0.700000}%
\pgfsetdash{}{0pt}%
\pgfpathmoveto{\pgfqpoint{1.877486in}{0.550000in}}%
\pgfpathlineto{\pgfqpoint{1.905007in}{0.550000in}}%
\pgfpathlineto{\pgfqpoint{1.905007in}{0.550000in}}%
\pgfpathlineto{\pgfqpoint{1.877486in}{0.550000in}}%
\pgfpathlineto{\pgfqpoint{1.877486in}{0.550000in}}%
\pgfpathclose%
\pgfusepath{fill}%
\end{pgfscope}%
\begin{pgfscope}%
\pgfpathrectangle{\pgfqpoint{1.250000in}{0.550000in}}{\pgfqpoint{7.750000in}{3.850000in}}%
\pgfusepath{clip}%
\pgfsetbuttcap%
\pgfsetmiterjoin%
\definecolor{currentfill}{rgb}{0.501961,0.501961,0.501961}%
\pgfsetfillcolor{currentfill}%
\pgfsetfillopacity{0.700000}%
\pgfsetlinewidth{0.000000pt}%
\definecolor{currentstroke}{rgb}{0.000000,0.000000,0.000000}%
\pgfsetstrokecolor{currentstroke}%
\pgfsetstrokeopacity{0.700000}%
\pgfsetdash{}{0pt}%
\pgfpathmoveto{\pgfqpoint{1.905007in}{0.550000in}}%
\pgfpathlineto{\pgfqpoint{1.932528in}{0.550000in}}%
\pgfpathlineto{\pgfqpoint{1.932528in}{1.256193in}}%
\pgfpathlineto{\pgfqpoint{1.905007in}{1.256193in}}%
\pgfpathlineto{\pgfqpoint{1.905007in}{0.550000in}}%
\pgfpathclose%
\pgfusepath{fill}%
\end{pgfscope}%
\begin{pgfscope}%
\pgfpathrectangle{\pgfqpoint{1.250000in}{0.550000in}}{\pgfqpoint{7.750000in}{3.850000in}}%
\pgfusepath{clip}%
\pgfsetbuttcap%
\pgfsetmiterjoin%
\definecolor{currentfill}{rgb}{0.501961,0.501961,0.501961}%
\pgfsetfillcolor{currentfill}%
\pgfsetfillopacity{0.700000}%
\pgfsetlinewidth{0.000000pt}%
\definecolor{currentstroke}{rgb}{0.000000,0.000000,0.000000}%
\pgfsetstrokecolor{currentstroke}%
\pgfsetstrokeopacity{0.700000}%
\pgfsetdash{}{0pt}%
\pgfpathmoveto{\pgfqpoint{1.932528in}{0.550000in}}%
\pgfpathlineto{\pgfqpoint{1.960050in}{0.550000in}}%
\pgfpathlineto{\pgfqpoint{1.960050in}{1.258963in}}%
\pgfpathlineto{\pgfqpoint{1.932528in}{1.258963in}}%
\pgfpathlineto{\pgfqpoint{1.932528in}{0.550000in}}%
\pgfpathclose%
\pgfusepath{fill}%
\end{pgfscope}%
\begin{pgfscope}%
\pgfpathrectangle{\pgfqpoint{1.250000in}{0.550000in}}{\pgfqpoint{7.750000in}{3.850000in}}%
\pgfusepath{clip}%
\pgfsetbuttcap%
\pgfsetmiterjoin%
\definecolor{currentfill}{rgb}{0.501961,0.501961,0.501961}%
\pgfsetfillcolor{currentfill}%
\pgfsetfillopacity{0.700000}%
\pgfsetlinewidth{0.000000pt}%
\definecolor{currentstroke}{rgb}{0.000000,0.000000,0.000000}%
\pgfsetstrokecolor{currentstroke}%
\pgfsetstrokeopacity{0.700000}%
\pgfsetdash{}{0pt}%
\pgfpathmoveto{\pgfqpoint{1.960050in}{0.550000in}}%
\pgfpathlineto{\pgfqpoint{1.987571in}{0.550000in}}%
\pgfpathlineto{\pgfqpoint{1.987571in}{1.819961in}}%
\pgfpathlineto{\pgfqpoint{1.960050in}{1.819961in}}%
\pgfpathlineto{\pgfqpoint{1.960050in}{0.550000in}}%
\pgfpathclose%
\pgfusepath{fill}%
\end{pgfscope}%
\begin{pgfscope}%
\pgfpathrectangle{\pgfqpoint{1.250000in}{0.550000in}}{\pgfqpoint{7.750000in}{3.850000in}}%
\pgfusepath{clip}%
\pgfsetbuttcap%
\pgfsetmiterjoin%
\definecolor{currentfill}{rgb}{0.501961,0.501961,0.501961}%
\pgfsetfillcolor{currentfill}%
\pgfsetfillopacity{0.700000}%
\pgfsetlinewidth{0.000000pt}%
\definecolor{currentstroke}{rgb}{0.000000,0.000000,0.000000}%
\pgfsetstrokecolor{currentstroke}%
\pgfsetstrokeopacity{0.700000}%
\pgfsetdash{}{0pt}%
\pgfpathmoveto{\pgfqpoint{1.987571in}{0.550000in}}%
\pgfpathlineto{\pgfqpoint{2.015092in}{0.550000in}}%
\pgfpathlineto{\pgfqpoint{2.015092in}{1.143044in}}%
\pgfpathlineto{\pgfqpoint{1.987571in}{1.143044in}}%
\pgfpathlineto{\pgfqpoint{1.987571in}{0.550000in}}%
\pgfpathclose%
\pgfusepath{fill}%
\end{pgfscope}%
\begin{pgfscope}%
\pgfpathrectangle{\pgfqpoint{1.250000in}{0.550000in}}{\pgfqpoint{7.750000in}{3.850000in}}%
\pgfusepath{clip}%
\pgfsetbuttcap%
\pgfsetmiterjoin%
\definecolor{currentfill}{rgb}{0.501961,0.501961,0.501961}%
\pgfsetfillcolor{currentfill}%
\pgfsetfillopacity{0.700000}%
\pgfsetlinewidth{0.000000pt}%
\definecolor{currentstroke}{rgb}{0.000000,0.000000,0.000000}%
\pgfsetstrokecolor{currentstroke}%
\pgfsetstrokeopacity{0.700000}%
\pgfsetdash{}{0pt}%
\pgfpathmoveto{\pgfqpoint{2.015092in}{0.550000in}}%
\pgfpathlineto{\pgfqpoint{2.042614in}{0.550000in}}%
\pgfpathlineto{\pgfqpoint{2.042614in}{1.129593in}}%
\pgfpathlineto{\pgfqpoint{2.015092in}{1.129593in}}%
\pgfpathlineto{\pgfqpoint{2.015092in}{0.550000in}}%
\pgfpathclose%
\pgfusepath{fill}%
\end{pgfscope}%
\begin{pgfscope}%
\pgfpathrectangle{\pgfqpoint{1.250000in}{0.550000in}}{\pgfqpoint{7.750000in}{3.850000in}}%
\pgfusepath{clip}%
\pgfsetbuttcap%
\pgfsetmiterjoin%
\definecolor{currentfill}{rgb}{0.501961,0.501961,0.501961}%
\pgfsetfillcolor{currentfill}%
\pgfsetfillopacity{0.700000}%
\pgfsetlinewidth{0.000000pt}%
\definecolor{currentstroke}{rgb}{0.000000,0.000000,0.000000}%
\pgfsetstrokecolor{currentstroke}%
\pgfsetstrokeopacity{0.700000}%
\pgfsetdash{}{0pt}%
\pgfpathmoveto{\pgfqpoint{2.042614in}{0.550000in}}%
\pgfpathlineto{\pgfqpoint{2.070135in}{0.550000in}}%
\pgfpathlineto{\pgfqpoint{2.070135in}{1.086074in}}%
\pgfpathlineto{\pgfqpoint{2.042614in}{1.086074in}}%
\pgfpathlineto{\pgfqpoint{2.042614in}{0.550000in}}%
\pgfpathclose%
\pgfusepath{fill}%
\end{pgfscope}%
\begin{pgfscope}%
\pgfpathrectangle{\pgfqpoint{1.250000in}{0.550000in}}{\pgfqpoint{7.750000in}{3.850000in}}%
\pgfusepath{clip}%
\pgfsetbuttcap%
\pgfsetmiterjoin%
\definecolor{currentfill}{rgb}{0.501961,0.501961,0.501961}%
\pgfsetfillcolor{currentfill}%
\pgfsetfillopacity{0.700000}%
\pgfsetlinewidth{0.000000pt}%
\definecolor{currentstroke}{rgb}{0.000000,0.000000,0.000000}%
\pgfsetstrokecolor{currentstroke}%
\pgfsetstrokeopacity{0.700000}%
\pgfsetdash{}{0pt}%
\pgfpathmoveto{\pgfqpoint{2.070135in}{0.550000in}}%
\pgfpathlineto{\pgfqpoint{2.097656in}{0.550000in}}%
\pgfpathlineto{\pgfqpoint{2.097656in}{1.580211in}}%
\pgfpathlineto{\pgfqpoint{2.070135in}{1.580211in}}%
\pgfpathlineto{\pgfqpoint{2.070135in}{0.550000in}}%
\pgfpathclose%
\pgfusepath{fill}%
\end{pgfscope}%
\begin{pgfscope}%
\pgfpathrectangle{\pgfqpoint{1.250000in}{0.550000in}}{\pgfqpoint{7.750000in}{3.850000in}}%
\pgfusepath{clip}%
\pgfsetbuttcap%
\pgfsetmiterjoin%
\definecolor{currentfill}{rgb}{0.501961,0.501961,0.501961}%
\pgfsetfillcolor{currentfill}%
\pgfsetfillopacity{0.700000}%
\pgfsetlinewidth{0.000000pt}%
\definecolor{currentstroke}{rgb}{0.000000,0.000000,0.000000}%
\pgfsetstrokecolor{currentstroke}%
\pgfsetstrokeopacity{0.700000}%
\pgfsetdash{}{0pt}%
\pgfpathmoveto{\pgfqpoint{2.097656in}{0.550000in}}%
\pgfpathlineto{\pgfqpoint{2.125178in}{0.550000in}}%
\pgfpathlineto{\pgfqpoint{2.125178in}{1.048885in}}%
\pgfpathlineto{\pgfqpoint{2.097656in}{1.048885in}}%
\pgfpathlineto{\pgfqpoint{2.097656in}{0.550000in}}%
\pgfpathclose%
\pgfusepath{fill}%
\end{pgfscope}%
\begin{pgfscope}%
\pgfpathrectangle{\pgfqpoint{1.250000in}{0.550000in}}{\pgfqpoint{7.750000in}{3.850000in}}%
\pgfusepath{clip}%
\pgfsetbuttcap%
\pgfsetmiterjoin%
\definecolor{currentfill}{rgb}{0.501961,0.501961,0.501961}%
\pgfsetfillcolor{currentfill}%
\pgfsetfillopacity{0.700000}%
\pgfsetlinewidth{0.000000pt}%
\definecolor{currentstroke}{rgb}{0.000000,0.000000,0.000000}%
\pgfsetstrokecolor{currentstroke}%
\pgfsetstrokeopacity{0.700000}%
\pgfsetdash{}{0pt}%
\pgfpathmoveto{\pgfqpoint{2.125178in}{0.550000in}}%
\pgfpathlineto{\pgfqpoint{2.152699in}{0.550000in}}%
\pgfpathlineto{\pgfqpoint{2.152699in}{1.068666in}}%
\pgfpathlineto{\pgfqpoint{2.125178in}{1.068666in}}%
\pgfpathlineto{\pgfqpoint{2.125178in}{0.550000in}}%
\pgfpathclose%
\pgfusepath{fill}%
\end{pgfscope}%
\begin{pgfscope}%
\pgfpathrectangle{\pgfqpoint{1.250000in}{0.550000in}}{\pgfqpoint{7.750000in}{3.850000in}}%
\pgfusepath{clip}%
\pgfsetbuttcap%
\pgfsetmiterjoin%
\definecolor{currentfill}{rgb}{0.501961,0.501961,0.501961}%
\pgfsetfillcolor{currentfill}%
\pgfsetfillopacity{0.700000}%
\pgfsetlinewidth{0.000000pt}%
\definecolor{currentstroke}{rgb}{0.000000,0.000000,0.000000}%
\pgfsetstrokecolor{currentstroke}%
\pgfsetstrokeopacity{0.700000}%
\pgfsetdash{}{0pt}%
\pgfpathmoveto{\pgfqpoint{2.152699in}{0.550000in}}%
\pgfpathlineto{\pgfqpoint{2.180220in}{0.550000in}}%
\pgfpathlineto{\pgfqpoint{2.180220in}{1.102690in}}%
\pgfpathlineto{\pgfqpoint{2.152699in}{1.102690in}}%
\pgfpathlineto{\pgfqpoint{2.152699in}{0.550000in}}%
\pgfpathclose%
\pgfusepath{fill}%
\end{pgfscope}%
\begin{pgfscope}%
\pgfpathrectangle{\pgfqpoint{1.250000in}{0.550000in}}{\pgfqpoint{7.750000in}{3.850000in}}%
\pgfusepath{clip}%
\pgfsetbuttcap%
\pgfsetmiterjoin%
\definecolor{currentfill}{rgb}{0.501961,0.501961,0.501961}%
\pgfsetfillcolor{currentfill}%
\pgfsetfillopacity{0.700000}%
\pgfsetlinewidth{0.000000pt}%
\definecolor{currentstroke}{rgb}{0.000000,0.000000,0.000000}%
\pgfsetstrokecolor{currentstroke}%
\pgfsetstrokeopacity{0.700000}%
\pgfsetdash{}{0pt}%
\pgfpathmoveto{\pgfqpoint{2.180220in}{0.550000in}}%
\pgfpathlineto{\pgfqpoint{2.207741in}{0.550000in}}%
\pgfpathlineto{\pgfqpoint{2.207741in}{1.743210in}}%
\pgfpathlineto{\pgfqpoint{2.180220in}{1.743210in}}%
\pgfpathlineto{\pgfqpoint{2.180220in}{0.550000in}}%
\pgfpathclose%
\pgfusepath{fill}%
\end{pgfscope}%
\begin{pgfscope}%
\pgfpathrectangle{\pgfqpoint{1.250000in}{0.550000in}}{\pgfqpoint{7.750000in}{3.850000in}}%
\pgfusepath{clip}%
\pgfsetbuttcap%
\pgfsetmiterjoin%
\definecolor{currentfill}{rgb}{0.501961,0.501961,0.501961}%
\pgfsetfillcolor{currentfill}%
\pgfsetfillopacity{0.700000}%
\pgfsetlinewidth{0.000000pt}%
\definecolor{currentstroke}{rgb}{0.000000,0.000000,0.000000}%
\pgfsetstrokecolor{currentstroke}%
\pgfsetstrokeopacity{0.700000}%
\pgfsetdash{}{0pt}%
\pgfpathmoveto{\pgfqpoint{2.207741in}{0.550000in}}%
\pgfpathlineto{\pgfqpoint{2.235263in}{0.550000in}}%
\pgfpathlineto{\pgfqpoint{2.235263in}{1.135132in}}%
\pgfpathlineto{\pgfqpoint{2.207741in}{1.135132in}}%
\pgfpathlineto{\pgfqpoint{2.207741in}{0.550000in}}%
\pgfpathclose%
\pgfusepath{fill}%
\end{pgfscope}%
\begin{pgfscope}%
\pgfpathrectangle{\pgfqpoint{1.250000in}{0.550000in}}{\pgfqpoint{7.750000in}{3.850000in}}%
\pgfusepath{clip}%
\pgfsetbuttcap%
\pgfsetmiterjoin%
\definecolor{currentfill}{rgb}{0.501961,0.501961,0.501961}%
\pgfsetfillcolor{currentfill}%
\pgfsetfillopacity{0.700000}%
\pgfsetlinewidth{0.000000pt}%
\definecolor{currentstroke}{rgb}{0.000000,0.000000,0.000000}%
\pgfsetstrokecolor{currentstroke}%
\pgfsetstrokeopacity{0.700000}%
\pgfsetdash{}{0pt}%
\pgfpathmoveto{\pgfqpoint{2.235263in}{0.550000in}}%
\pgfpathlineto{\pgfqpoint{2.262784in}{0.550000in}}%
\pgfpathlineto{\pgfqpoint{2.262784in}{1.118120in}}%
\pgfpathlineto{\pgfqpoint{2.235263in}{1.118120in}}%
\pgfpathlineto{\pgfqpoint{2.235263in}{0.550000in}}%
\pgfpathclose%
\pgfusepath{fill}%
\end{pgfscope}%
\begin{pgfscope}%
\pgfpathrectangle{\pgfqpoint{1.250000in}{0.550000in}}{\pgfqpoint{7.750000in}{3.850000in}}%
\pgfusepath{clip}%
\pgfsetbuttcap%
\pgfsetmiterjoin%
\definecolor{currentfill}{rgb}{0.501961,0.501961,0.501961}%
\pgfsetfillcolor{currentfill}%
\pgfsetfillopacity{0.700000}%
\pgfsetlinewidth{0.000000pt}%
\definecolor{currentstroke}{rgb}{0.000000,0.000000,0.000000}%
\pgfsetstrokecolor{currentstroke}%
\pgfsetstrokeopacity{0.700000}%
\pgfsetdash{}{0pt}%
\pgfpathmoveto{\pgfqpoint{2.262784in}{0.550000in}}%
\pgfpathlineto{\pgfqpoint{2.290305in}{0.550000in}}%
\pgfpathlineto{\pgfqpoint{2.290305in}{1.145418in}}%
\pgfpathlineto{\pgfqpoint{2.262784in}{1.145418in}}%
\pgfpathlineto{\pgfqpoint{2.262784in}{0.550000in}}%
\pgfpathclose%
\pgfusepath{fill}%
\end{pgfscope}%
\begin{pgfscope}%
\pgfpathrectangle{\pgfqpoint{1.250000in}{0.550000in}}{\pgfqpoint{7.750000in}{3.850000in}}%
\pgfusepath{clip}%
\pgfsetbuttcap%
\pgfsetmiterjoin%
\definecolor{currentfill}{rgb}{0.501961,0.501961,0.501961}%
\pgfsetfillcolor{currentfill}%
\pgfsetfillopacity{0.700000}%
\pgfsetlinewidth{0.000000pt}%
\definecolor{currentstroke}{rgb}{0.000000,0.000000,0.000000}%
\pgfsetstrokecolor{currentstroke}%
\pgfsetstrokeopacity{0.700000}%
\pgfsetdash{}{0pt}%
\pgfpathmoveto{\pgfqpoint{2.290305in}{0.550000in}}%
\pgfpathlineto{\pgfqpoint{2.317827in}{0.550000in}}%
\pgfpathlineto{\pgfqpoint{2.317827in}{1.164012in}}%
\pgfpathlineto{\pgfqpoint{2.290305in}{1.164012in}}%
\pgfpathlineto{\pgfqpoint{2.290305in}{0.550000in}}%
\pgfpathclose%
\pgfusepath{fill}%
\end{pgfscope}%
\begin{pgfscope}%
\pgfpathrectangle{\pgfqpoint{1.250000in}{0.550000in}}{\pgfqpoint{7.750000in}{3.850000in}}%
\pgfusepath{clip}%
\pgfsetbuttcap%
\pgfsetmiterjoin%
\definecolor{currentfill}{rgb}{0.501961,0.501961,0.501961}%
\pgfsetfillcolor{currentfill}%
\pgfsetfillopacity{0.700000}%
\pgfsetlinewidth{0.000000pt}%
\definecolor{currentstroke}{rgb}{0.000000,0.000000,0.000000}%
\pgfsetstrokecolor{currentstroke}%
\pgfsetstrokeopacity{0.700000}%
\pgfsetdash{}{0pt}%
\pgfpathmoveto{\pgfqpoint{2.317827in}{0.550000in}}%
\pgfpathlineto{\pgfqpoint{2.345348in}{0.550000in}}%
\pgfpathlineto{\pgfqpoint{2.345348in}{1.191311in}}%
\pgfpathlineto{\pgfqpoint{2.317827in}{1.191311in}}%
\pgfpathlineto{\pgfqpoint{2.317827in}{0.550000in}}%
\pgfpathclose%
\pgfusepath{fill}%
\end{pgfscope}%
\begin{pgfscope}%
\pgfpathrectangle{\pgfqpoint{1.250000in}{0.550000in}}{\pgfqpoint{7.750000in}{3.850000in}}%
\pgfusepath{clip}%
\pgfsetbuttcap%
\pgfsetmiterjoin%
\definecolor{currentfill}{rgb}{0.501961,0.501961,0.501961}%
\pgfsetfillcolor{currentfill}%
\pgfsetfillopacity{0.700000}%
\pgfsetlinewidth{0.000000pt}%
\definecolor{currentstroke}{rgb}{0.000000,0.000000,0.000000}%
\pgfsetstrokecolor{currentstroke}%
\pgfsetstrokeopacity{0.700000}%
\pgfsetdash{}{0pt}%
\pgfpathmoveto{\pgfqpoint{2.345348in}{0.550000in}}%
\pgfpathlineto{\pgfqpoint{2.372869in}{0.550000in}}%
\pgfpathlineto{\pgfqpoint{2.372869in}{1.294569in}}%
\pgfpathlineto{\pgfqpoint{2.345348in}{1.294569in}}%
\pgfpathlineto{\pgfqpoint{2.345348in}{0.550000in}}%
\pgfpathclose%
\pgfusepath{fill}%
\end{pgfscope}%
\begin{pgfscope}%
\pgfpathrectangle{\pgfqpoint{1.250000in}{0.550000in}}{\pgfqpoint{7.750000in}{3.850000in}}%
\pgfusepath{clip}%
\pgfsetbuttcap%
\pgfsetmiterjoin%
\definecolor{currentfill}{rgb}{0.501961,0.501961,0.501961}%
\pgfsetfillcolor{currentfill}%
\pgfsetfillopacity{0.700000}%
\pgfsetlinewidth{0.000000pt}%
\definecolor{currentstroke}{rgb}{0.000000,0.000000,0.000000}%
\pgfsetstrokecolor{currentstroke}%
\pgfsetstrokeopacity{0.700000}%
\pgfsetdash{}{0pt}%
\pgfpathmoveto{\pgfqpoint{2.372869in}{0.550000in}}%
\pgfpathlineto{\pgfqpoint{2.400391in}{0.550000in}}%
\pgfpathlineto{\pgfqpoint{2.400391in}{1.368156in}}%
\pgfpathlineto{\pgfqpoint{2.372869in}{1.368156in}}%
\pgfpathlineto{\pgfqpoint{2.372869in}{0.550000in}}%
\pgfpathclose%
\pgfusepath{fill}%
\end{pgfscope}%
\begin{pgfscope}%
\pgfpathrectangle{\pgfqpoint{1.250000in}{0.550000in}}{\pgfqpoint{7.750000in}{3.850000in}}%
\pgfusepath{clip}%
\pgfsetbuttcap%
\pgfsetmiterjoin%
\definecolor{currentfill}{rgb}{0.501961,0.501961,0.501961}%
\pgfsetfillcolor{currentfill}%
\pgfsetfillopacity{0.700000}%
\pgfsetlinewidth{0.000000pt}%
\definecolor{currentstroke}{rgb}{0.000000,0.000000,0.000000}%
\pgfsetstrokecolor{currentstroke}%
\pgfsetstrokeopacity{0.700000}%
\pgfsetdash{}{0pt}%
\pgfpathmoveto{\pgfqpoint{2.400391in}{0.550000in}}%
\pgfpathlineto{\pgfqpoint{2.427912in}{0.550000in}}%
\pgfpathlineto{\pgfqpoint{2.427912in}{1.476557in}}%
\pgfpathlineto{\pgfqpoint{2.400391in}{1.476557in}}%
\pgfpathlineto{\pgfqpoint{2.400391in}{0.550000in}}%
\pgfpathclose%
\pgfusepath{fill}%
\end{pgfscope}%
\begin{pgfscope}%
\pgfpathrectangle{\pgfqpoint{1.250000in}{0.550000in}}{\pgfqpoint{7.750000in}{3.850000in}}%
\pgfusepath{clip}%
\pgfsetbuttcap%
\pgfsetmiterjoin%
\definecolor{currentfill}{rgb}{0.501961,0.501961,0.501961}%
\pgfsetfillcolor{currentfill}%
\pgfsetfillopacity{0.700000}%
\pgfsetlinewidth{0.000000pt}%
\definecolor{currentstroke}{rgb}{0.000000,0.000000,0.000000}%
\pgfsetstrokecolor{currentstroke}%
\pgfsetstrokeopacity{0.700000}%
\pgfsetdash{}{0pt}%
\pgfpathmoveto{\pgfqpoint{2.427912in}{0.550000in}}%
\pgfpathlineto{\pgfqpoint{2.455433in}{0.550000in}}%
\pgfpathlineto{\pgfqpoint{2.455433in}{1.641929in}}%
\pgfpathlineto{\pgfqpoint{2.427912in}{1.641929in}}%
\pgfpathlineto{\pgfqpoint{2.427912in}{0.550000in}}%
\pgfpathclose%
\pgfusepath{fill}%
\end{pgfscope}%
\begin{pgfscope}%
\pgfpathrectangle{\pgfqpoint{1.250000in}{0.550000in}}{\pgfqpoint{7.750000in}{3.850000in}}%
\pgfusepath{clip}%
\pgfsetbuttcap%
\pgfsetmiterjoin%
\definecolor{currentfill}{rgb}{0.501961,0.501961,0.501961}%
\pgfsetfillcolor{currentfill}%
\pgfsetfillopacity{0.700000}%
\pgfsetlinewidth{0.000000pt}%
\definecolor{currentstroke}{rgb}{0.000000,0.000000,0.000000}%
\pgfsetstrokecolor{currentstroke}%
\pgfsetstrokeopacity{0.700000}%
\pgfsetdash{}{0pt}%
\pgfpathmoveto{\pgfqpoint{2.455433in}{0.550000in}}%
\pgfpathlineto{\pgfqpoint{2.482955in}{0.550000in}}%
\pgfpathlineto{\pgfqpoint{2.482955in}{1.565573in}}%
\pgfpathlineto{\pgfqpoint{2.455433in}{1.565573in}}%
\pgfpathlineto{\pgfqpoint{2.455433in}{0.550000in}}%
\pgfpathclose%
\pgfusepath{fill}%
\end{pgfscope}%
\begin{pgfscope}%
\pgfpathrectangle{\pgfqpoint{1.250000in}{0.550000in}}{\pgfqpoint{7.750000in}{3.850000in}}%
\pgfusepath{clip}%
\pgfsetbuttcap%
\pgfsetmiterjoin%
\definecolor{currentfill}{rgb}{0.501961,0.501961,0.501961}%
\pgfsetfillcolor{currentfill}%
\pgfsetfillopacity{0.700000}%
\pgfsetlinewidth{0.000000pt}%
\definecolor{currentstroke}{rgb}{0.000000,0.000000,0.000000}%
\pgfsetstrokecolor{currentstroke}%
\pgfsetstrokeopacity{0.700000}%
\pgfsetdash{}{0pt}%
\pgfpathmoveto{\pgfqpoint{2.482955in}{0.550000in}}%
\pgfpathlineto{\pgfqpoint{2.510476in}{0.550000in}}%
\pgfpathlineto{\pgfqpoint{2.510476in}{1.613048in}}%
\pgfpathlineto{\pgfqpoint{2.482955in}{1.613048in}}%
\pgfpathlineto{\pgfqpoint{2.482955in}{0.550000in}}%
\pgfpathclose%
\pgfusepath{fill}%
\end{pgfscope}%
\begin{pgfscope}%
\pgfpathrectangle{\pgfqpoint{1.250000in}{0.550000in}}{\pgfqpoint{7.750000in}{3.850000in}}%
\pgfusepath{clip}%
\pgfsetbuttcap%
\pgfsetmiterjoin%
\definecolor{currentfill}{rgb}{0.501961,0.501961,0.501961}%
\pgfsetfillcolor{currentfill}%
\pgfsetfillopacity{0.700000}%
\pgfsetlinewidth{0.000000pt}%
\definecolor{currentstroke}{rgb}{0.000000,0.000000,0.000000}%
\pgfsetstrokecolor{currentstroke}%
\pgfsetstrokeopacity{0.700000}%
\pgfsetdash{}{0pt}%
\pgfpathmoveto{\pgfqpoint{2.510476in}{0.550000in}}%
\pgfpathlineto{\pgfqpoint{2.537997in}{0.550000in}}%
\pgfpathlineto{\pgfqpoint{2.537997in}{1.790289in}}%
\pgfpathlineto{\pgfqpoint{2.510476in}{1.790289in}}%
\pgfpathlineto{\pgfqpoint{2.510476in}{0.550000in}}%
\pgfpathclose%
\pgfusepath{fill}%
\end{pgfscope}%
\begin{pgfscope}%
\pgfpathrectangle{\pgfqpoint{1.250000in}{0.550000in}}{\pgfqpoint{7.750000in}{3.850000in}}%
\pgfusepath{clip}%
\pgfsetbuttcap%
\pgfsetmiterjoin%
\definecolor{currentfill}{rgb}{0.501961,0.501961,0.501961}%
\pgfsetfillcolor{currentfill}%
\pgfsetfillopacity{0.700000}%
\pgfsetlinewidth{0.000000pt}%
\definecolor{currentstroke}{rgb}{0.000000,0.000000,0.000000}%
\pgfsetstrokecolor{currentstroke}%
\pgfsetstrokeopacity{0.700000}%
\pgfsetdash{}{0pt}%
\pgfpathmoveto{\pgfqpoint{2.537997in}{0.550000in}}%
\pgfpathlineto{\pgfqpoint{2.565518in}{0.550000in}}%
\pgfpathlineto{\pgfqpoint{2.565518in}{1.673975in}}%
\pgfpathlineto{\pgfqpoint{2.537997in}{1.673975in}}%
\pgfpathlineto{\pgfqpoint{2.537997in}{0.550000in}}%
\pgfpathclose%
\pgfusepath{fill}%
\end{pgfscope}%
\begin{pgfscope}%
\pgfpathrectangle{\pgfqpoint{1.250000in}{0.550000in}}{\pgfqpoint{7.750000in}{3.850000in}}%
\pgfusepath{clip}%
\pgfsetbuttcap%
\pgfsetmiterjoin%
\definecolor{currentfill}{rgb}{0.501961,0.501961,0.501961}%
\pgfsetfillcolor{currentfill}%
\pgfsetfillopacity{0.700000}%
\pgfsetlinewidth{0.000000pt}%
\definecolor{currentstroke}{rgb}{0.000000,0.000000,0.000000}%
\pgfsetstrokecolor{currentstroke}%
\pgfsetstrokeopacity{0.700000}%
\pgfsetdash{}{0pt}%
\pgfpathmoveto{\pgfqpoint{2.565518in}{0.550000in}}%
\pgfpathlineto{\pgfqpoint{2.593040in}{0.550000in}}%
\pgfpathlineto{\pgfqpoint{2.593040in}{1.753892in}}%
\pgfpathlineto{\pgfqpoint{2.565518in}{1.753892in}}%
\pgfpathlineto{\pgfqpoint{2.565518in}{0.550000in}}%
\pgfpathclose%
\pgfusepath{fill}%
\end{pgfscope}%
\begin{pgfscope}%
\pgfpathrectangle{\pgfqpoint{1.250000in}{0.550000in}}{\pgfqpoint{7.750000in}{3.850000in}}%
\pgfusepath{clip}%
\pgfsetbuttcap%
\pgfsetmiterjoin%
\definecolor{currentfill}{rgb}{0.501961,0.501961,0.501961}%
\pgfsetfillcolor{currentfill}%
\pgfsetfillopacity{0.700000}%
\pgfsetlinewidth{0.000000pt}%
\definecolor{currentstroke}{rgb}{0.000000,0.000000,0.000000}%
\pgfsetstrokecolor{currentstroke}%
\pgfsetstrokeopacity{0.700000}%
\pgfsetdash{}{0pt}%
\pgfpathmoveto{\pgfqpoint{2.593040in}{0.550000in}}%
\pgfpathlineto{\pgfqpoint{2.620561in}{0.550000in}}%
\pgfpathlineto{\pgfqpoint{2.620561in}{1.768134in}}%
\pgfpathlineto{\pgfqpoint{2.593040in}{1.768134in}}%
\pgfpathlineto{\pgfqpoint{2.593040in}{0.550000in}}%
\pgfpathclose%
\pgfusepath{fill}%
\end{pgfscope}%
\begin{pgfscope}%
\pgfpathrectangle{\pgfqpoint{1.250000in}{0.550000in}}{\pgfqpoint{7.750000in}{3.850000in}}%
\pgfusepath{clip}%
\pgfsetbuttcap%
\pgfsetmiterjoin%
\definecolor{currentfill}{rgb}{0.501961,0.501961,0.501961}%
\pgfsetfillcolor{currentfill}%
\pgfsetfillopacity{0.700000}%
\pgfsetlinewidth{0.000000pt}%
\definecolor{currentstroke}{rgb}{0.000000,0.000000,0.000000}%
\pgfsetstrokecolor{currentstroke}%
\pgfsetstrokeopacity{0.700000}%
\pgfsetdash{}{0pt}%
\pgfpathmoveto{\pgfqpoint{2.620561in}{0.550000in}}%
\pgfpathlineto{\pgfqpoint{2.648082in}{0.550000in}}%
\pgfpathlineto{\pgfqpoint{2.648082in}{0.550000in}}%
\pgfpathlineto{\pgfqpoint{2.620561in}{0.550000in}}%
\pgfpathlineto{\pgfqpoint{2.620561in}{0.550000in}}%
\pgfpathclose%
\pgfusepath{fill}%
\end{pgfscope}%
\begin{pgfscope}%
\pgfpathrectangle{\pgfqpoint{1.250000in}{0.550000in}}{\pgfqpoint{7.750000in}{3.850000in}}%
\pgfusepath{clip}%
\pgfsetbuttcap%
\pgfsetmiterjoin%
\definecolor{currentfill}{rgb}{0.501961,0.501961,0.501961}%
\pgfsetfillcolor{currentfill}%
\pgfsetfillopacity{0.700000}%
\pgfsetlinewidth{0.000000pt}%
\definecolor{currentstroke}{rgb}{0.000000,0.000000,0.000000}%
\pgfsetstrokecolor{currentstroke}%
\pgfsetstrokeopacity{0.700000}%
\pgfsetdash{}{0pt}%
\pgfpathmoveto{\pgfqpoint{2.648082in}{0.550000in}}%
\pgfpathlineto{\pgfqpoint{2.675604in}{0.550000in}}%
\pgfpathlineto{\pgfqpoint{2.675604in}{1.801367in}}%
\pgfpathlineto{\pgfqpoint{2.648082in}{1.801367in}}%
\pgfpathlineto{\pgfqpoint{2.648082in}{0.550000in}}%
\pgfpathclose%
\pgfusepath{fill}%
\end{pgfscope}%
\begin{pgfscope}%
\pgfpathrectangle{\pgfqpoint{1.250000in}{0.550000in}}{\pgfqpoint{7.750000in}{3.850000in}}%
\pgfusepath{clip}%
\pgfsetbuttcap%
\pgfsetmiterjoin%
\definecolor{currentfill}{rgb}{0.501961,0.501961,0.501961}%
\pgfsetfillcolor{currentfill}%
\pgfsetfillopacity{0.700000}%
\pgfsetlinewidth{0.000000pt}%
\definecolor{currentstroke}{rgb}{0.000000,0.000000,0.000000}%
\pgfsetstrokecolor{currentstroke}%
\pgfsetstrokeopacity{0.700000}%
\pgfsetdash{}{0pt}%
\pgfpathmoveto{\pgfqpoint{2.675604in}{0.550000in}}%
\pgfpathlineto{\pgfqpoint{2.703125in}{0.550000in}}%
\pgfpathlineto{\pgfqpoint{2.703125in}{1.788707in}}%
\pgfpathlineto{\pgfqpoint{2.675604in}{1.788707in}}%
\pgfpathlineto{\pgfqpoint{2.675604in}{0.550000in}}%
\pgfpathclose%
\pgfusepath{fill}%
\end{pgfscope}%
\begin{pgfscope}%
\pgfpathrectangle{\pgfqpoint{1.250000in}{0.550000in}}{\pgfqpoint{7.750000in}{3.850000in}}%
\pgfusepath{clip}%
\pgfsetbuttcap%
\pgfsetmiterjoin%
\definecolor{currentfill}{rgb}{0.501961,0.501961,0.501961}%
\pgfsetfillcolor{currentfill}%
\pgfsetfillopacity{0.700000}%
\pgfsetlinewidth{0.000000pt}%
\definecolor{currentstroke}{rgb}{0.000000,0.000000,0.000000}%
\pgfsetstrokecolor{currentstroke}%
\pgfsetstrokeopacity{0.700000}%
\pgfsetdash{}{0pt}%
\pgfpathmoveto{\pgfqpoint{2.703125in}{0.550000in}}%
\pgfpathlineto{\pgfqpoint{2.730646in}{0.550000in}}%
\pgfpathlineto{\pgfqpoint{2.730646in}{1.716703in}}%
\pgfpathlineto{\pgfqpoint{2.703125in}{1.716703in}}%
\pgfpathlineto{\pgfqpoint{2.703125in}{0.550000in}}%
\pgfpathclose%
\pgfusepath{fill}%
\end{pgfscope}%
\begin{pgfscope}%
\pgfpathrectangle{\pgfqpoint{1.250000in}{0.550000in}}{\pgfqpoint{7.750000in}{3.850000in}}%
\pgfusepath{clip}%
\pgfsetbuttcap%
\pgfsetmiterjoin%
\definecolor{currentfill}{rgb}{0.501961,0.501961,0.501961}%
\pgfsetfillcolor{currentfill}%
\pgfsetfillopacity{0.700000}%
\pgfsetlinewidth{0.000000pt}%
\definecolor{currentstroke}{rgb}{0.000000,0.000000,0.000000}%
\pgfsetstrokecolor{currentstroke}%
\pgfsetstrokeopacity{0.700000}%
\pgfsetdash{}{0pt}%
\pgfpathmoveto{\pgfqpoint{2.730646in}{0.550000in}}%
\pgfpathlineto{\pgfqpoint{2.758168in}{0.550000in}}%
\pgfpathlineto{\pgfqpoint{2.758168in}{1.641534in}}%
\pgfpathlineto{\pgfqpoint{2.730646in}{1.641534in}}%
\pgfpathlineto{\pgfqpoint{2.730646in}{0.550000in}}%
\pgfpathclose%
\pgfusepath{fill}%
\end{pgfscope}%
\begin{pgfscope}%
\pgfpathrectangle{\pgfqpoint{1.250000in}{0.550000in}}{\pgfqpoint{7.750000in}{3.850000in}}%
\pgfusepath{clip}%
\pgfsetbuttcap%
\pgfsetmiterjoin%
\definecolor{currentfill}{rgb}{0.501961,0.501961,0.501961}%
\pgfsetfillcolor{currentfill}%
\pgfsetfillopacity{0.700000}%
\pgfsetlinewidth{0.000000pt}%
\definecolor{currentstroke}{rgb}{0.000000,0.000000,0.000000}%
\pgfsetstrokecolor{currentstroke}%
\pgfsetstrokeopacity{0.700000}%
\pgfsetdash{}{0pt}%
\pgfpathmoveto{\pgfqpoint{2.758168in}{0.550000in}}%
\pgfpathlineto{\pgfqpoint{2.785689in}{0.550000in}}%
\pgfpathlineto{\pgfqpoint{2.785689in}{1.477349in}}%
\pgfpathlineto{\pgfqpoint{2.758168in}{1.477349in}}%
\pgfpathlineto{\pgfqpoint{2.758168in}{0.550000in}}%
\pgfpathclose%
\pgfusepath{fill}%
\end{pgfscope}%
\begin{pgfscope}%
\pgfpathrectangle{\pgfqpoint{1.250000in}{0.550000in}}{\pgfqpoint{7.750000in}{3.850000in}}%
\pgfusepath{clip}%
\pgfsetbuttcap%
\pgfsetmiterjoin%
\definecolor{currentfill}{rgb}{0.501961,0.501961,0.501961}%
\pgfsetfillcolor{currentfill}%
\pgfsetfillopacity{0.700000}%
\pgfsetlinewidth{0.000000pt}%
\definecolor{currentstroke}{rgb}{0.000000,0.000000,0.000000}%
\pgfsetstrokecolor{currentstroke}%
\pgfsetstrokeopacity{0.700000}%
\pgfsetdash{}{0pt}%
\pgfpathmoveto{\pgfqpoint{2.785689in}{0.550000in}}%
\pgfpathlineto{\pgfqpoint{2.813210in}{0.550000in}}%
\pgfpathlineto{\pgfqpoint{2.813210in}{1.431060in}}%
\pgfpathlineto{\pgfqpoint{2.785689in}{1.431060in}}%
\pgfpathlineto{\pgfqpoint{2.785689in}{0.550000in}}%
\pgfpathclose%
\pgfusepath{fill}%
\end{pgfscope}%
\begin{pgfscope}%
\pgfpathrectangle{\pgfqpoint{1.250000in}{0.550000in}}{\pgfqpoint{7.750000in}{3.850000in}}%
\pgfusepath{clip}%
\pgfsetbuttcap%
\pgfsetmiterjoin%
\definecolor{currentfill}{rgb}{0.501961,0.501961,0.501961}%
\pgfsetfillcolor{currentfill}%
\pgfsetfillopacity{0.700000}%
\pgfsetlinewidth{0.000000pt}%
\definecolor{currentstroke}{rgb}{0.000000,0.000000,0.000000}%
\pgfsetstrokecolor{currentstroke}%
\pgfsetstrokeopacity{0.700000}%
\pgfsetdash{}{0pt}%
\pgfpathmoveto{\pgfqpoint{2.813210in}{0.550000in}}%
\pgfpathlineto{\pgfqpoint{2.840732in}{0.550000in}}%
\pgfpathlineto{\pgfqpoint{2.840732in}{1.289426in}}%
\pgfpathlineto{\pgfqpoint{2.813210in}{1.289426in}}%
\pgfpathlineto{\pgfqpoint{2.813210in}{0.550000in}}%
\pgfpathclose%
\pgfusepath{fill}%
\end{pgfscope}%
\begin{pgfscope}%
\pgfpathrectangle{\pgfqpoint{1.250000in}{0.550000in}}{\pgfqpoint{7.750000in}{3.850000in}}%
\pgfusepath{clip}%
\pgfsetbuttcap%
\pgfsetmiterjoin%
\definecolor{currentfill}{rgb}{0.501961,0.501961,0.501961}%
\pgfsetfillcolor{currentfill}%
\pgfsetfillopacity{0.700000}%
\pgfsetlinewidth{0.000000pt}%
\definecolor{currentstroke}{rgb}{0.000000,0.000000,0.000000}%
\pgfsetstrokecolor{currentstroke}%
\pgfsetstrokeopacity{0.700000}%
\pgfsetdash{}{0pt}%
\pgfpathmoveto{\pgfqpoint{2.840732in}{0.550000in}}%
\pgfpathlineto{\pgfqpoint{2.868253in}{0.550000in}}%
\pgfpathlineto{\pgfqpoint{2.868253in}{1.277162in}}%
\pgfpathlineto{\pgfqpoint{2.840732in}{1.277162in}}%
\pgfpathlineto{\pgfqpoint{2.840732in}{0.550000in}}%
\pgfpathclose%
\pgfusepath{fill}%
\end{pgfscope}%
\begin{pgfscope}%
\pgfpathrectangle{\pgfqpoint{1.250000in}{0.550000in}}{\pgfqpoint{7.750000in}{3.850000in}}%
\pgfusepath{clip}%
\pgfsetbuttcap%
\pgfsetmiterjoin%
\definecolor{currentfill}{rgb}{0.501961,0.501961,0.501961}%
\pgfsetfillcolor{currentfill}%
\pgfsetfillopacity{0.700000}%
\pgfsetlinewidth{0.000000pt}%
\definecolor{currentstroke}{rgb}{0.000000,0.000000,0.000000}%
\pgfsetstrokecolor{currentstroke}%
\pgfsetstrokeopacity{0.700000}%
\pgfsetdash{}{0pt}%
\pgfpathmoveto{\pgfqpoint{2.868253in}{0.550000in}}%
\pgfpathlineto{\pgfqpoint{2.895774in}{0.550000in}}%
\pgfpathlineto{\pgfqpoint{2.895774in}{1.270040in}}%
\pgfpathlineto{\pgfqpoint{2.868253in}{1.270040in}}%
\pgfpathlineto{\pgfqpoint{2.868253in}{0.550000in}}%
\pgfpathclose%
\pgfusepath{fill}%
\end{pgfscope}%
\begin{pgfscope}%
\pgfpathrectangle{\pgfqpoint{1.250000in}{0.550000in}}{\pgfqpoint{7.750000in}{3.850000in}}%
\pgfusepath{clip}%
\pgfsetbuttcap%
\pgfsetmiterjoin%
\definecolor{currentfill}{rgb}{0.501961,0.501961,0.501961}%
\pgfsetfillcolor{currentfill}%
\pgfsetfillopacity{0.700000}%
\pgfsetlinewidth{0.000000pt}%
\definecolor{currentstroke}{rgb}{0.000000,0.000000,0.000000}%
\pgfsetstrokecolor{currentstroke}%
\pgfsetstrokeopacity{0.700000}%
\pgfsetdash{}{0pt}%
\pgfpathmoveto{\pgfqpoint{2.895774in}{0.550000in}}%
\pgfpathlineto{\pgfqpoint{2.923295in}{0.550000in}}%
\pgfpathlineto{\pgfqpoint{2.923295in}{1.246303in}}%
\pgfpathlineto{\pgfqpoint{2.895774in}{1.246303in}}%
\pgfpathlineto{\pgfqpoint{2.895774in}{0.550000in}}%
\pgfpathclose%
\pgfusepath{fill}%
\end{pgfscope}%
\begin{pgfscope}%
\pgfpathrectangle{\pgfqpoint{1.250000in}{0.550000in}}{\pgfqpoint{7.750000in}{3.850000in}}%
\pgfusepath{clip}%
\pgfsetbuttcap%
\pgfsetmiterjoin%
\definecolor{currentfill}{rgb}{0.501961,0.501961,0.501961}%
\pgfsetfillcolor{currentfill}%
\pgfsetfillopacity{0.700000}%
\pgfsetlinewidth{0.000000pt}%
\definecolor{currentstroke}{rgb}{0.000000,0.000000,0.000000}%
\pgfsetstrokecolor{currentstroke}%
\pgfsetstrokeopacity{0.700000}%
\pgfsetdash{}{0pt}%
\pgfpathmoveto{\pgfqpoint{2.923295in}{0.550000in}}%
\pgfpathlineto{\pgfqpoint{2.950817in}{0.550000in}}%
\pgfpathlineto{\pgfqpoint{2.950817in}{1.208718in}}%
\pgfpathlineto{\pgfqpoint{2.923295in}{1.208718in}}%
\pgfpathlineto{\pgfqpoint{2.923295in}{0.550000in}}%
\pgfpathclose%
\pgfusepath{fill}%
\end{pgfscope}%
\begin{pgfscope}%
\pgfpathrectangle{\pgfqpoint{1.250000in}{0.550000in}}{\pgfqpoint{7.750000in}{3.850000in}}%
\pgfusepath{clip}%
\pgfsetbuttcap%
\pgfsetmiterjoin%
\definecolor{currentfill}{rgb}{0.501961,0.501961,0.501961}%
\pgfsetfillcolor{currentfill}%
\pgfsetfillopacity{0.700000}%
\pgfsetlinewidth{0.000000pt}%
\definecolor{currentstroke}{rgb}{0.000000,0.000000,0.000000}%
\pgfsetstrokecolor{currentstroke}%
\pgfsetstrokeopacity{0.700000}%
\pgfsetdash{}{0pt}%
\pgfpathmoveto{\pgfqpoint{2.950817in}{0.550000in}}%
\pgfpathlineto{\pgfqpoint{2.978338in}{0.550000in}}%
\pgfpathlineto{\pgfqpoint{2.978338in}{0.550000in}}%
\pgfpathlineto{\pgfqpoint{2.950817in}{0.550000in}}%
\pgfpathlineto{\pgfqpoint{2.950817in}{0.550000in}}%
\pgfpathclose%
\pgfusepath{fill}%
\end{pgfscope}%
\begin{pgfscope}%
\pgfpathrectangle{\pgfqpoint{1.250000in}{0.550000in}}{\pgfqpoint{7.750000in}{3.850000in}}%
\pgfusepath{clip}%
\pgfsetbuttcap%
\pgfsetmiterjoin%
\definecolor{currentfill}{rgb}{0.501961,0.501961,0.501961}%
\pgfsetfillcolor{currentfill}%
\pgfsetfillopacity{0.700000}%
\pgfsetlinewidth{0.000000pt}%
\definecolor{currentstroke}{rgb}{0.000000,0.000000,0.000000}%
\pgfsetstrokecolor{currentstroke}%
\pgfsetstrokeopacity{0.700000}%
\pgfsetdash{}{0pt}%
\pgfpathmoveto{\pgfqpoint{2.978338in}{0.550000in}}%
\pgfpathlineto{\pgfqpoint{3.005859in}{0.550000in}}%
\pgfpathlineto{\pgfqpoint{3.005859in}{1.212674in}}%
\pgfpathlineto{\pgfqpoint{2.978338in}{1.212674in}}%
\pgfpathlineto{\pgfqpoint{2.978338in}{0.550000in}}%
\pgfpathclose%
\pgfusepath{fill}%
\end{pgfscope}%
\begin{pgfscope}%
\pgfpathrectangle{\pgfqpoint{1.250000in}{0.550000in}}{\pgfqpoint{7.750000in}{3.850000in}}%
\pgfusepath{clip}%
\pgfsetbuttcap%
\pgfsetmiterjoin%
\definecolor{currentfill}{rgb}{0.501961,0.501961,0.501961}%
\pgfsetfillcolor{currentfill}%
\pgfsetfillopacity{0.700000}%
\pgfsetlinewidth{0.000000pt}%
\definecolor{currentstroke}{rgb}{0.000000,0.000000,0.000000}%
\pgfsetstrokecolor{currentstroke}%
\pgfsetstrokeopacity{0.700000}%
\pgfsetdash{}{0pt}%
\pgfpathmoveto{\pgfqpoint{3.005859in}{0.550000in}}%
\pgfpathlineto{\pgfqpoint{3.033381in}{0.550000in}}%
\pgfpathlineto{\pgfqpoint{3.033381in}{0.550000in}}%
\pgfpathlineto{\pgfqpoint{3.005859in}{0.550000in}}%
\pgfpathlineto{\pgfqpoint{3.005859in}{0.550000in}}%
\pgfpathclose%
\pgfusepath{fill}%
\end{pgfscope}%
\begin{pgfscope}%
\pgfpathrectangle{\pgfqpoint{1.250000in}{0.550000in}}{\pgfqpoint{7.750000in}{3.850000in}}%
\pgfusepath{clip}%
\pgfsetbuttcap%
\pgfsetmiterjoin%
\definecolor{currentfill}{rgb}{0.501961,0.501961,0.501961}%
\pgfsetfillcolor{currentfill}%
\pgfsetfillopacity{0.700000}%
\pgfsetlinewidth{0.000000pt}%
\definecolor{currentstroke}{rgb}{0.000000,0.000000,0.000000}%
\pgfsetstrokecolor{currentstroke}%
\pgfsetstrokeopacity{0.700000}%
\pgfsetdash{}{0pt}%
\pgfpathmoveto{\pgfqpoint{3.033381in}{0.550000in}}%
\pgfpathlineto{\pgfqpoint{3.060902in}{0.550000in}}%
\pgfpathlineto{\pgfqpoint{3.060902in}{1.162034in}}%
\pgfpathlineto{\pgfqpoint{3.033381in}{1.162034in}}%
\pgfpathlineto{\pgfqpoint{3.033381in}{0.550000in}}%
\pgfpathclose%
\pgfusepath{fill}%
\end{pgfscope}%
\begin{pgfscope}%
\pgfpathrectangle{\pgfqpoint{1.250000in}{0.550000in}}{\pgfqpoint{7.750000in}{3.850000in}}%
\pgfusepath{clip}%
\pgfsetbuttcap%
\pgfsetmiterjoin%
\definecolor{currentfill}{rgb}{0.501961,0.501961,0.501961}%
\pgfsetfillcolor{currentfill}%
\pgfsetfillopacity{0.700000}%
\pgfsetlinewidth{0.000000pt}%
\definecolor{currentstroke}{rgb}{0.000000,0.000000,0.000000}%
\pgfsetstrokecolor{currentstroke}%
\pgfsetstrokeopacity{0.700000}%
\pgfsetdash{}{0pt}%
\pgfpathmoveto{\pgfqpoint{3.060902in}{0.550000in}}%
\pgfpathlineto{\pgfqpoint{3.088423in}{0.550000in}}%
\pgfpathlineto{\pgfqpoint{3.088423in}{1.161243in}}%
\pgfpathlineto{\pgfqpoint{3.060902in}{1.161243in}}%
\pgfpathlineto{\pgfqpoint{3.060902in}{0.550000in}}%
\pgfpathclose%
\pgfusepath{fill}%
\end{pgfscope}%
\begin{pgfscope}%
\pgfpathrectangle{\pgfqpoint{1.250000in}{0.550000in}}{\pgfqpoint{7.750000in}{3.850000in}}%
\pgfusepath{clip}%
\pgfsetbuttcap%
\pgfsetmiterjoin%
\definecolor{currentfill}{rgb}{0.501961,0.501961,0.501961}%
\pgfsetfillcolor{currentfill}%
\pgfsetfillopacity{0.700000}%
\pgfsetlinewidth{0.000000pt}%
\definecolor{currentstroke}{rgb}{0.000000,0.000000,0.000000}%
\pgfsetstrokecolor{currentstroke}%
\pgfsetstrokeopacity{0.700000}%
\pgfsetdash{}{0pt}%
\pgfpathmoveto{\pgfqpoint{3.088423in}{0.550000in}}%
\pgfpathlineto{\pgfqpoint{3.115945in}{0.550000in}}%
\pgfpathlineto{\pgfqpoint{3.115945in}{1.144231in}}%
\pgfpathlineto{\pgfqpoint{3.088423in}{1.144231in}}%
\pgfpathlineto{\pgfqpoint{3.088423in}{0.550000in}}%
\pgfpathclose%
\pgfusepath{fill}%
\end{pgfscope}%
\begin{pgfscope}%
\pgfpathrectangle{\pgfqpoint{1.250000in}{0.550000in}}{\pgfqpoint{7.750000in}{3.850000in}}%
\pgfusepath{clip}%
\pgfsetbuttcap%
\pgfsetmiterjoin%
\definecolor{currentfill}{rgb}{0.501961,0.501961,0.501961}%
\pgfsetfillcolor{currentfill}%
\pgfsetfillopacity{0.700000}%
\pgfsetlinewidth{0.000000pt}%
\definecolor{currentstroke}{rgb}{0.000000,0.000000,0.000000}%
\pgfsetstrokecolor{currentstroke}%
\pgfsetstrokeopacity{0.700000}%
\pgfsetdash{}{0pt}%
\pgfpathmoveto{\pgfqpoint{3.115945in}{0.550000in}}%
\pgfpathlineto{\pgfqpoint{3.143466in}{0.550000in}}%
\pgfpathlineto{\pgfqpoint{3.143466in}{1.059171in}}%
\pgfpathlineto{\pgfqpoint{3.115945in}{1.059171in}}%
\pgfpathlineto{\pgfqpoint{3.115945in}{0.550000in}}%
\pgfpathclose%
\pgfusepath{fill}%
\end{pgfscope}%
\begin{pgfscope}%
\pgfpathrectangle{\pgfqpoint{1.250000in}{0.550000in}}{\pgfqpoint{7.750000in}{3.850000in}}%
\pgfusepath{clip}%
\pgfsetbuttcap%
\pgfsetmiterjoin%
\definecolor{currentfill}{rgb}{0.501961,0.501961,0.501961}%
\pgfsetfillcolor{currentfill}%
\pgfsetfillopacity{0.700000}%
\pgfsetlinewidth{0.000000pt}%
\definecolor{currentstroke}{rgb}{0.000000,0.000000,0.000000}%
\pgfsetstrokecolor{currentstroke}%
\pgfsetstrokeopacity{0.700000}%
\pgfsetdash{}{0pt}%
\pgfpathmoveto{\pgfqpoint{3.143466in}{0.550000in}}%
\pgfpathlineto{\pgfqpoint{3.170987in}{0.550000in}}%
\pgfpathlineto{\pgfqpoint{3.170987in}{0.550000in}}%
\pgfpathlineto{\pgfqpoint{3.143466in}{0.550000in}}%
\pgfpathlineto{\pgfqpoint{3.143466in}{0.550000in}}%
\pgfpathclose%
\pgfusepath{fill}%
\end{pgfscope}%
\begin{pgfscope}%
\pgfpathrectangle{\pgfqpoint{1.250000in}{0.550000in}}{\pgfqpoint{7.750000in}{3.850000in}}%
\pgfusepath{clip}%
\pgfsetbuttcap%
\pgfsetmiterjoin%
\definecolor{currentfill}{rgb}{0.501961,0.501961,0.501961}%
\pgfsetfillcolor{currentfill}%
\pgfsetfillopacity{0.700000}%
\pgfsetlinewidth{0.000000pt}%
\definecolor{currentstroke}{rgb}{0.000000,0.000000,0.000000}%
\pgfsetstrokecolor{currentstroke}%
\pgfsetstrokeopacity{0.700000}%
\pgfsetdash{}{0pt}%
\pgfpathmoveto{\pgfqpoint{3.170987in}{0.550000in}}%
\pgfpathlineto{\pgfqpoint{3.198509in}{0.550000in}}%
\pgfpathlineto{\pgfqpoint{3.198509in}{1.025147in}}%
\pgfpathlineto{\pgfqpoint{3.170987in}{1.025147in}}%
\pgfpathlineto{\pgfqpoint{3.170987in}{0.550000in}}%
\pgfpathclose%
\pgfusepath{fill}%
\end{pgfscope}%
\begin{pgfscope}%
\pgfpathrectangle{\pgfqpoint{1.250000in}{0.550000in}}{\pgfqpoint{7.750000in}{3.850000in}}%
\pgfusepath{clip}%
\pgfsetbuttcap%
\pgfsetmiterjoin%
\definecolor{currentfill}{rgb}{0.501961,0.501961,0.501961}%
\pgfsetfillcolor{currentfill}%
\pgfsetfillopacity{0.700000}%
\pgfsetlinewidth{0.000000pt}%
\definecolor{currentstroke}{rgb}{0.000000,0.000000,0.000000}%
\pgfsetstrokecolor{currentstroke}%
\pgfsetstrokeopacity{0.700000}%
\pgfsetdash{}{0pt}%
\pgfpathmoveto{\pgfqpoint{3.198509in}{0.550000in}}%
\pgfpathlineto{\pgfqpoint{3.226030in}{0.550000in}}%
\pgfpathlineto{\pgfqpoint{3.226030in}{1.054819in}}%
\pgfpathlineto{\pgfqpoint{3.198509in}{1.054819in}}%
\pgfpathlineto{\pgfqpoint{3.198509in}{0.550000in}}%
\pgfpathclose%
\pgfusepath{fill}%
\end{pgfscope}%
\begin{pgfscope}%
\pgfpathrectangle{\pgfqpoint{1.250000in}{0.550000in}}{\pgfqpoint{7.750000in}{3.850000in}}%
\pgfusepath{clip}%
\pgfsetbuttcap%
\pgfsetmiterjoin%
\definecolor{currentfill}{rgb}{0.501961,0.501961,0.501961}%
\pgfsetfillcolor{currentfill}%
\pgfsetfillopacity{0.700000}%
\pgfsetlinewidth{0.000000pt}%
\definecolor{currentstroke}{rgb}{0.000000,0.000000,0.000000}%
\pgfsetstrokecolor{currentstroke}%
\pgfsetstrokeopacity{0.700000}%
\pgfsetdash{}{0pt}%
\pgfpathmoveto{\pgfqpoint{3.226030in}{0.550000in}}%
\pgfpathlineto{\pgfqpoint{3.253551in}{0.550000in}}%
\pgfpathlineto{\pgfqpoint{3.253551in}{1.013279in}}%
\pgfpathlineto{\pgfqpoint{3.226030in}{1.013279in}}%
\pgfpathlineto{\pgfqpoint{3.226030in}{0.550000in}}%
\pgfpathclose%
\pgfusepath{fill}%
\end{pgfscope}%
\begin{pgfscope}%
\pgfpathrectangle{\pgfqpoint{1.250000in}{0.550000in}}{\pgfqpoint{7.750000in}{3.850000in}}%
\pgfusepath{clip}%
\pgfsetbuttcap%
\pgfsetmiterjoin%
\definecolor{currentfill}{rgb}{0.501961,0.501961,0.501961}%
\pgfsetfillcolor{currentfill}%
\pgfsetfillopacity{0.700000}%
\pgfsetlinewidth{0.000000pt}%
\definecolor{currentstroke}{rgb}{0.000000,0.000000,0.000000}%
\pgfsetstrokecolor{currentstroke}%
\pgfsetstrokeopacity{0.700000}%
\pgfsetdash{}{0pt}%
\pgfpathmoveto{\pgfqpoint{3.253551in}{0.550000in}}%
\pgfpathlineto{\pgfqpoint{3.281072in}{0.550000in}}%
\pgfpathlineto{\pgfqpoint{3.281072in}{0.986772in}}%
\pgfpathlineto{\pgfqpoint{3.253551in}{0.986772in}}%
\pgfpathlineto{\pgfqpoint{3.253551in}{0.550000in}}%
\pgfpathclose%
\pgfusepath{fill}%
\end{pgfscope}%
\begin{pgfscope}%
\pgfpathrectangle{\pgfqpoint{1.250000in}{0.550000in}}{\pgfqpoint{7.750000in}{3.850000in}}%
\pgfusepath{clip}%
\pgfsetbuttcap%
\pgfsetmiterjoin%
\definecolor{currentfill}{rgb}{0.501961,0.501961,0.501961}%
\pgfsetfillcolor{currentfill}%
\pgfsetfillopacity{0.700000}%
\pgfsetlinewidth{0.000000pt}%
\definecolor{currentstroke}{rgb}{0.000000,0.000000,0.000000}%
\pgfsetstrokecolor{currentstroke}%
\pgfsetstrokeopacity{0.700000}%
\pgfsetdash{}{0pt}%
\pgfpathmoveto{\pgfqpoint{3.281072in}{0.550000in}}%
\pgfpathlineto{\pgfqpoint{3.308594in}{0.550000in}}%
\pgfpathlineto{\pgfqpoint{3.308594in}{0.550000in}}%
\pgfpathlineto{\pgfqpoint{3.281072in}{0.550000in}}%
\pgfpathlineto{\pgfqpoint{3.281072in}{0.550000in}}%
\pgfpathclose%
\pgfusepath{fill}%
\end{pgfscope}%
\begin{pgfscope}%
\pgfpathrectangle{\pgfqpoint{1.250000in}{0.550000in}}{\pgfqpoint{7.750000in}{3.850000in}}%
\pgfusepath{clip}%
\pgfsetbuttcap%
\pgfsetmiterjoin%
\definecolor{currentfill}{rgb}{0.501961,0.501961,0.501961}%
\pgfsetfillcolor{currentfill}%
\pgfsetfillopacity{0.700000}%
\pgfsetlinewidth{0.000000pt}%
\definecolor{currentstroke}{rgb}{0.000000,0.000000,0.000000}%
\pgfsetstrokecolor{currentstroke}%
\pgfsetstrokeopacity{0.700000}%
\pgfsetdash{}{0pt}%
\pgfpathmoveto{\pgfqpoint{3.308594in}{0.550000in}}%
\pgfpathlineto{\pgfqpoint{3.336115in}{0.550000in}}%
\pgfpathlineto{\pgfqpoint{3.336115in}{0.974903in}}%
\pgfpathlineto{\pgfqpoint{3.308594in}{0.974903in}}%
\pgfpathlineto{\pgfqpoint{3.308594in}{0.550000in}}%
\pgfpathclose%
\pgfusepath{fill}%
\end{pgfscope}%
\begin{pgfscope}%
\pgfpathrectangle{\pgfqpoint{1.250000in}{0.550000in}}{\pgfqpoint{7.750000in}{3.850000in}}%
\pgfusepath{clip}%
\pgfsetbuttcap%
\pgfsetmiterjoin%
\definecolor{currentfill}{rgb}{0.501961,0.501961,0.501961}%
\pgfsetfillcolor{currentfill}%
\pgfsetfillopacity{0.700000}%
\pgfsetlinewidth{0.000000pt}%
\definecolor{currentstroke}{rgb}{0.000000,0.000000,0.000000}%
\pgfsetstrokecolor{currentstroke}%
\pgfsetstrokeopacity{0.700000}%
\pgfsetdash{}{0pt}%
\pgfpathmoveto{\pgfqpoint{3.336115in}{0.550000in}}%
\pgfpathlineto{\pgfqpoint{3.363636in}{0.550000in}}%
\pgfpathlineto{\pgfqpoint{3.363636in}{0.948396in}}%
\pgfpathlineto{\pgfqpoint{3.336115in}{0.948396in}}%
\pgfpathlineto{\pgfqpoint{3.336115in}{0.550000in}}%
\pgfpathclose%
\pgfusepath{fill}%
\end{pgfscope}%
\begin{pgfscope}%
\pgfpathrectangle{\pgfqpoint{1.250000in}{0.550000in}}{\pgfqpoint{7.750000in}{3.850000in}}%
\pgfusepath{clip}%
\pgfsetbuttcap%
\pgfsetmiterjoin%
\definecolor{currentfill}{rgb}{0.501961,0.501961,0.501961}%
\pgfsetfillcolor{currentfill}%
\pgfsetfillopacity{0.700000}%
\pgfsetlinewidth{0.000000pt}%
\definecolor{currentstroke}{rgb}{0.000000,0.000000,0.000000}%
\pgfsetstrokecolor{currentstroke}%
\pgfsetstrokeopacity{0.700000}%
\pgfsetdash{}{0pt}%
\pgfpathmoveto{\pgfqpoint{3.363636in}{0.550000in}}%
\pgfpathlineto{\pgfqpoint{3.391158in}{0.550000in}}%
\pgfpathlineto{\pgfqpoint{3.391158in}{0.914768in}}%
\pgfpathlineto{\pgfqpoint{3.363636in}{0.914768in}}%
\pgfpathlineto{\pgfqpoint{3.363636in}{0.550000in}}%
\pgfpathclose%
\pgfusepath{fill}%
\end{pgfscope}%
\begin{pgfscope}%
\pgfpathrectangle{\pgfqpoint{1.250000in}{0.550000in}}{\pgfqpoint{7.750000in}{3.850000in}}%
\pgfusepath{clip}%
\pgfsetbuttcap%
\pgfsetmiterjoin%
\definecolor{currentfill}{rgb}{0.501961,0.501961,0.501961}%
\pgfsetfillcolor{currentfill}%
\pgfsetfillopacity{0.700000}%
\pgfsetlinewidth{0.000000pt}%
\definecolor{currentstroke}{rgb}{0.000000,0.000000,0.000000}%
\pgfsetstrokecolor{currentstroke}%
\pgfsetstrokeopacity{0.700000}%
\pgfsetdash{}{0pt}%
\pgfpathmoveto{\pgfqpoint{3.391158in}{0.550000in}}%
\pgfpathlineto{\pgfqpoint{3.418679in}{0.550000in}}%
\pgfpathlineto{\pgfqpoint{3.418679in}{0.925845in}}%
\pgfpathlineto{\pgfqpoint{3.391158in}{0.925845in}}%
\pgfpathlineto{\pgfqpoint{3.391158in}{0.550000in}}%
\pgfpathclose%
\pgfusepath{fill}%
\end{pgfscope}%
\begin{pgfscope}%
\pgfpathrectangle{\pgfqpoint{1.250000in}{0.550000in}}{\pgfqpoint{7.750000in}{3.850000in}}%
\pgfusepath{clip}%
\pgfsetbuttcap%
\pgfsetmiterjoin%
\definecolor{currentfill}{rgb}{0.501961,0.501961,0.501961}%
\pgfsetfillcolor{currentfill}%
\pgfsetfillopacity{0.700000}%
\pgfsetlinewidth{0.000000pt}%
\definecolor{currentstroke}{rgb}{0.000000,0.000000,0.000000}%
\pgfsetstrokecolor{currentstroke}%
\pgfsetstrokeopacity{0.700000}%
\pgfsetdash{}{0pt}%
\pgfpathmoveto{\pgfqpoint{3.418679in}{0.550000in}}%
\pgfpathlineto{\pgfqpoint{3.446200in}{0.550000in}}%
\pgfpathlineto{\pgfqpoint{3.446200in}{0.550000in}}%
\pgfpathlineto{\pgfqpoint{3.418679in}{0.550000in}}%
\pgfpathlineto{\pgfqpoint{3.418679in}{0.550000in}}%
\pgfpathclose%
\pgfusepath{fill}%
\end{pgfscope}%
\begin{pgfscope}%
\pgfpathrectangle{\pgfqpoint{1.250000in}{0.550000in}}{\pgfqpoint{7.750000in}{3.850000in}}%
\pgfusepath{clip}%
\pgfsetbuttcap%
\pgfsetmiterjoin%
\definecolor{currentfill}{rgb}{0.501961,0.501961,0.501961}%
\pgfsetfillcolor{currentfill}%
\pgfsetfillopacity{0.700000}%
\pgfsetlinewidth{0.000000pt}%
\definecolor{currentstroke}{rgb}{0.000000,0.000000,0.000000}%
\pgfsetstrokecolor{currentstroke}%
\pgfsetstrokeopacity{0.700000}%
\pgfsetdash{}{0pt}%
\pgfpathmoveto{\pgfqpoint{3.446200in}{0.550000in}}%
\pgfpathlineto{\pgfqpoint{3.473722in}{0.550000in}}%
\pgfpathlineto{\pgfqpoint{3.473722in}{0.927032in}}%
\pgfpathlineto{\pgfqpoint{3.446200in}{0.927032in}}%
\pgfpathlineto{\pgfqpoint{3.446200in}{0.550000in}}%
\pgfpathclose%
\pgfusepath{fill}%
\end{pgfscope}%
\begin{pgfscope}%
\pgfpathrectangle{\pgfqpoint{1.250000in}{0.550000in}}{\pgfqpoint{7.750000in}{3.850000in}}%
\pgfusepath{clip}%
\pgfsetbuttcap%
\pgfsetmiterjoin%
\definecolor{currentfill}{rgb}{0.501961,0.501961,0.501961}%
\pgfsetfillcolor{currentfill}%
\pgfsetfillopacity{0.700000}%
\pgfsetlinewidth{0.000000pt}%
\definecolor{currentstroke}{rgb}{0.000000,0.000000,0.000000}%
\pgfsetstrokecolor{currentstroke}%
\pgfsetstrokeopacity{0.700000}%
\pgfsetdash{}{0pt}%
\pgfpathmoveto{\pgfqpoint{3.473722in}{0.550000in}}%
\pgfpathlineto{\pgfqpoint{3.501243in}{0.550000in}}%
\pgfpathlineto{\pgfqpoint{3.501243in}{0.900921in}}%
\pgfpathlineto{\pgfqpoint{3.473722in}{0.900921in}}%
\pgfpathlineto{\pgfqpoint{3.473722in}{0.550000in}}%
\pgfpathclose%
\pgfusepath{fill}%
\end{pgfscope}%
\begin{pgfscope}%
\pgfpathrectangle{\pgfqpoint{1.250000in}{0.550000in}}{\pgfqpoint{7.750000in}{3.850000in}}%
\pgfusepath{clip}%
\pgfsetbuttcap%
\pgfsetmiterjoin%
\definecolor{currentfill}{rgb}{0.501961,0.501961,0.501961}%
\pgfsetfillcolor{currentfill}%
\pgfsetfillopacity{0.700000}%
\pgfsetlinewidth{0.000000pt}%
\definecolor{currentstroke}{rgb}{0.000000,0.000000,0.000000}%
\pgfsetstrokecolor{currentstroke}%
\pgfsetstrokeopacity{0.700000}%
\pgfsetdash{}{0pt}%
\pgfpathmoveto{\pgfqpoint{3.501243in}{0.550000in}}%
\pgfpathlineto{\pgfqpoint{3.528764in}{0.550000in}}%
\pgfpathlineto{\pgfqpoint{3.528764in}{0.904086in}}%
\pgfpathlineto{\pgfqpoint{3.501243in}{0.904086in}}%
\pgfpathlineto{\pgfqpoint{3.501243in}{0.550000in}}%
\pgfpathclose%
\pgfusepath{fill}%
\end{pgfscope}%
\begin{pgfscope}%
\pgfpathrectangle{\pgfqpoint{1.250000in}{0.550000in}}{\pgfqpoint{7.750000in}{3.850000in}}%
\pgfusepath{clip}%
\pgfsetbuttcap%
\pgfsetmiterjoin%
\definecolor{currentfill}{rgb}{0.501961,0.501961,0.501961}%
\pgfsetfillcolor{currentfill}%
\pgfsetfillopacity{0.700000}%
\pgfsetlinewidth{0.000000pt}%
\definecolor{currentstroke}{rgb}{0.000000,0.000000,0.000000}%
\pgfsetstrokecolor{currentstroke}%
\pgfsetstrokeopacity{0.700000}%
\pgfsetdash{}{0pt}%
\pgfpathmoveto{\pgfqpoint{3.528764in}{0.550000in}}%
\pgfpathlineto{\pgfqpoint{3.556286in}{0.550000in}}%
\pgfpathlineto{\pgfqpoint{3.556286in}{0.911207in}}%
\pgfpathlineto{\pgfqpoint{3.528764in}{0.911207in}}%
\pgfpathlineto{\pgfqpoint{3.528764in}{0.550000in}}%
\pgfpathclose%
\pgfusepath{fill}%
\end{pgfscope}%
\begin{pgfscope}%
\pgfpathrectangle{\pgfqpoint{1.250000in}{0.550000in}}{\pgfqpoint{7.750000in}{3.850000in}}%
\pgfusepath{clip}%
\pgfsetbuttcap%
\pgfsetmiterjoin%
\definecolor{currentfill}{rgb}{0.501961,0.501961,0.501961}%
\pgfsetfillcolor{currentfill}%
\pgfsetfillopacity{0.700000}%
\pgfsetlinewidth{0.000000pt}%
\definecolor{currentstroke}{rgb}{0.000000,0.000000,0.000000}%
\pgfsetstrokecolor{currentstroke}%
\pgfsetstrokeopacity{0.700000}%
\pgfsetdash{}{0pt}%
\pgfpathmoveto{\pgfqpoint{3.556286in}{0.550000in}}%
\pgfpathlineto{\pgfqpoint{3.583807in}{0.550000in}}%
\pgfpathlineto{\pgfqpoint{3.583807in}{0.550000in}}%
\pgfpathlineto{\pgfqpoint{3.556286in}{0.550000in}}%
\pgfpathlineto{\pgfqpoint{3.556286in}{0.550000in}}%
\pgfpathclose%
\pgfusepath{fill}%
\end{pgfscope}%
\begin{pgfscope}%
\pgfpathrectangle{\pgfqpoint{1.250000in}{0.550000in}}{\pgfqpoint{7.750000in}{3.850000in}}%
\pgfusepath{clip}%
\pgfsetbuttcap%
\pgfsetmiterjoin%
\definecolor{currentfill}{rgb}{0.501961,0.501961,0.501961}%
\pgfsetfillcolor{currentfill}%
\pgfsetfillopacity{0.700000}%
\pgfsetlinewidth{0.000000pt}%
\definecolor{currentstroke}{rgb}{0.000000,0.000000,0.000000}%
\pgfsetstrokecolor{currentstroke}%
\pgfsetstrokeopacity{0.700000}%
\pgfsetdash{}{0pt}%
\pgfpathmoveto{\pgfqpoint{3.583807in}{0.550000in}}%
\pgfpathlineto{\pgfqpoint{3.611328in}{0.550000in}}%
\pgfpathlineto{\pgfqpoint{3.611328in}{0.910416in}}%
\pgfpathlineto{\pgfqpoint{3.583807in}{0.910416in}}%
\pgfpathlineto{\pgfqpoint{3.583807in}{0.550000in}}%
\pgfpathclose%
\pgfusepath{fill}%
\end{pgfscope}%
\begin{pgfscope}%
\pgfpathrectangle{\pgfqpoint{1.250000in}{0.550000in}}{\pgfqpoint{7.750000in}{3.850000in}}%
\pgfusepath{clip}%
\pgfsetbuttcap%
\pgfsetmiterjoin%
\definecolor{currentfill}{rgb}{0.501961,0.501961,0.501961}%
\pgfsetfillcolor{currentfill}%
\pgfsetfillopacity{0.700000}%
\pgfsetlinewidth{0.000000pt}%
\definecolor{currentstroke}{rgb}{0.000000,0.000000,0.000000}%
\pgfsetstrokecolor{currentstroke}%
\pgfsetstrokeopacity{0.700000}%
\pgfsetdash{}{0pt}%
\pgfpathmoveto{\pgfqpoint{3.611328in}{0.550000in}}%
\pgfpathlineto{\pgfqpoint{3.638849in}{0.550000in}}%
\pgfpathlineto{\pgfqpoint{3.638849in}{0.893008in}}%
\pgfpathlineto{\pgfqpoint{3.611328in}{0.893008in}}%
\pgfpathlineto{\pgfqpoint{3.611328in}{0.550000in}}%
\pgfpathclose%
\pgfusepath{fill}%
\end{pgfscope}%
\begin{pgfscope}%
\pgfpathrectangle{\pgfqpoint{1.250000in}{0.550000in}}{\pgfqpoint{7.750000in}{3.850000in}}%
\pgfusepath{clip}%
\pgfsetbuttcap%
\pgfsetmiterjoin%
\definecolor{currentfill}{rgb}{0.501961,0.501961,0.501961}%
\pgfsetfillcolor{currentfill}%
\pgfsetfillopacity{0.700000}%
\pgfsetlinewidth{0.000000pt}%
\definecolor{currentstroke}{rgb}{0.000000,0.000000,0.000000}%
\pgfsetstrokecolor{currentstroke}%
\pgfsetstrokeopacity{0.700000}%
\pgfsetdash{}{0pt}%
\pgfpathmoveto{\pgfqpoint{3.638849in}{0.550000in}}%
\pgfpathlineto{\pgfqpoint{3.666371in}{0.550000in}}%
\pgfpathlineto{\pgfqpoint{3.666371in}{0.896964in}}%
\pgfpathlineto{\pgfqpoint{3.638849in}{0.896964in}}%
\pgfpathlineto{\pgfqpoint{3.638849in}{0.550000in}}%
\pgfpathclose%
\pgfusepath{fill}%
\end{pgfscope}%
\begin{pgfscope}%
\pgfpathrectangle{\pgfqpoint{1.250000in}{0.550000in}}{\pgfqpoint{7.750000in}{3.850000in}}%
\pgfusepath{clip}%
\pgfsetbuttcap%
\pgfsetmiterjoin%
\definecolor{currentfill}{rgb}{0.501961,0.501961,0.501961}%
\pgfsetfillcolor{currentfill}%
\pgfsetfillopacity{0.700000}%
\pgfsetlinewidth{0.000000pt}%
\definecolor{currentstroke}{rgb}{0.000000,0.000000,0.000000}%
\pgfsetstrokecolor{currentstroke}%
\pgfsetstrokeopacity{0.700000}%
\pgfsetdash{}{0pt}%
\pgfpathmoveto{\pgfqpoint{3.666371in}{0.550000in}}%
\pgfpathlineto{\pgfqpoint{3.693892in}{0.550000in}}%
\pgfpathlineto{\pgfqpoint{3.693892in}{0.893799in}}%
\pgfpathlineto{\pgfqpoint{3.666371in}{0.893799in}}%
\pgfpathlineto{\pgfqpoint{3.666371in}{0.550000in}}%
\pgfpathclose%
\pgfusepath{fill}%
\end{pgfscope}%
\begin{pgfscope}%
\pgfpathrectangle{\pgfqpoint{1.250000in}{0.550000in}}{\pgfqpoint{7.750000in}{3.850000in}}%
\pgfusepath{clip}%
\pgfsetbuttcap%
\pgfsetmiterjoin%
\definecolor{currentfill}{rgb}{0.501961,0.501961,0.501961}%
\pgfsetfillcolor{currentfill}%
\pgfsetfillopacity{0.700000}%
\pgfsetlinewidth{0.000000pt}%
\definecolor{currentstroke}{rgb}{0.000000,0.000000,0.000000}%
\pgfsetstrokecolor{currentstroke}%
\pgfsetstrokeopacity{0.700000}%
\pgfsetdash{}{0pt}%
\pgfpathmoveto{\pgfqpoint{3.693892in}{0.550000in}}%
\pgfpathlineto{\pgfqpoint{3.721413in}{0.550000in}}%
\pgfpathlineto{\pgfqpoint{3.721413in}{0.550000in}}%
\pgfpathlineto{\pgfqpoint{3.693892in}{0.550000in}}%
\pgfpathlineto{\pgfqpoint{3.693892in}{0.550000in}}%
\pgfpathclose%
\pgfusepath{fill}%
\end{pgfscope}%
\begin{pgfscope}%
\pgfpathrectangle{\pgfqpoint{1.250000in}{0.550000in}}{\pgfqpoint{7.750000in}{3.850000in}}%
\pgfusepath{clip}%
\pgfsetbuttcap%
\pgfsetmiterjoin%
\definecolor{currentfill}{rgb}{0.501961,0.501961,0.501961}%
\pgfsetfillcolor{currentfill}%
\pgfsetfillopacity{0.700000}%
\pgfsetlinewidth{0.000000pt}%
\definecolor{currentstroke}{rgb}{0.000000,0.000000,0.000000}%
\pgfsetstrokecolor{currentstroke}%
\pgfsetstrokeopacity{0.700000}%
\pgfsetdash{}{0pt}%
\pgfpathmoveto{\pgfqpoint{3.721413in}{0.550000in}}%
\pgfpathlineto{\pgfqpoint{3.748935in}{0.550000in}}%
\pgfpathlineto{\pgfqpoint{3.748935in}{0.875996in}}%
\pgfpathlineto{\pgfqpoint{3.721413in}{0.875996in}}%
\pgfpathlineto{\pgfqpoint{3.721413in}{0.550000in}}%
\pgfpathclose%
\pgfusepath{fill}%
\end{pgfscope}%
\begin{pgfscope}%
\pgfpathrectangle{\pgfqpoint{1.250000in}{0.550000in}}{\pgfqpoint{7.750000in}{3.850000in}}%
\pgfusepath{clip}%
\pgfsetbuttcap%
\pgfsetmiterjoin%
\definecolor{currentfill}{rgb}{0.501961,0.501961,0.501961}%
\pgfsetfillcolor{currentfill}%
\pgfsetfillopacity{0.700000}%
\pgfsetlinewidth{0.000000pt}%
\definecolor{currentstroke}{rgb}{0.000000,0.000000,0.000000}%
\pgfsetstrokecolor{currentstroke}%
\pgfsetstrokeopacity{0.700000}%
\pgfsetdash{}{0pt}%
\pgfpathmoveto{\pgfqpoint{3.748935in}{0.550000in}}%
\pgfpathlineto{\pgfqpoint{3.776456in}{0.550000in}}%
\pgfpathlineto{\pgfqpoint{3.776456in}{0.877579in}}%
\pgfpathlineto{\pgfqpoint{3.748935in}{0.877579in}}%
\pgfpathlineto{\pgfqpoint{3.748935in}{0.550000in}}%
\pgfpathclose%
\pgfusepath{fill}%
\end{pgfscope}%
\begin{pgfscope}%
\pgfpathrectangle{\pgfqpoint{1.250000in}{0.550000in}}{\pgfqpoint{7.750000in}{3.850000in}}%
\pgfusepath{clip}%
\pgfsetbuttcap%
\pgfsetmiterjoin%
\definecolor{currentfill}{rgb}{0.501961,0.501961,0.501961}%
\pgfsetfillcolor{currentfill}%
\pgfsetfillopacity{0.700000}%
\pgfsetlinewidth{0.000000pt}%
\definecolor{currentstroke}{rgb}{0.000000,0.000000,0.000000}%
\pgfsetstrokecolor{currentstroke}%
\pgfsetstrokeopacity{0.700000}%
\pgfsetdash{}{0pt}%
\pgfpathmoveto{\pgfqpoint{3.776456in}{0.550000in}}%
\pgfpathlineto{\pgfqpoint{3.803977in}{0.550000in}}%
\pgfpathlineto{\pgfqpoint{3.803977in}{0.868084in}}%
\pgfpathlineto{\pgfqpoint{3.776456in}{0.868084in}}%
\pgfpathlineto{\pgfqpoint{3.776456in}{0.550000in}}%
\pgfpathclose%
\pgfusepath{fill}%
\end{pgfscope}%
\begin{pgfscope}%
\pgfpathrectangle{\pgfqpoint{1.250000in}{0.550000in}}{\pgfqpoint{7.750000in}{3.850000in}}%
\pgfusepath{clip}%
\pgfsetbuttcap%
\pgfsetmiterjoin%
\definecolor{currentfill}{rgb}{0.501961,0.501961,0.501961}%
\pgfsetfillcolor{currentfill}%
\pgfsetfillopacity{0.700000}%
\pgfsetlinewidth{0.000000pt}%
\definecolor{currentstroke}{rgb}{0.000000,0.000000,0.000000}%
\pgfsetstrokecolor{currentstroke}%
\pgfsetstrokeopacity{0.700000}%
\pgfsetdash{}{0pt}%
\pgfpathmoveto{\pgfqpoint{3.803977in}{0.550000in}}%
\pgfpathlineto{\pgfqpoint{3.831499in}{0.550000in}}%
\pgfpathlineto{\pgfqpoint{3.831499in}{0.866501in}}%
\pgfpathlineto{\pgfqpoint{3.803977in}{0.866501in}}%
\pgfpathlineto{\pgfqpoint{3.803977in}{0.550000in}}%
\pgfpathclose%
\pgfusepath{fill}%
\end{pgfscope}%
\begin{pgfscope}%
\pgfpathrectangle{\pgfqpoint{1.250000in}{0.550000in}}{\pgfqpoint{7.750000in}{3.850000in}}%
\pgfusepath{clip}%
\pgfsetbuttcap%
\pgfsetmiterjoin%
\definecolor{currentfill}{rgb}{0.501961,0.501961,0.501961}%
\pgfsetfillcolor{currentfill}%
\pgfsetfillopacity{0.700000}%
\pgfsetlinewidth{0.000000pt}%
\definecolor{currentstroke}{rgb}{0.000000,0.000000,0.000000}%
\pgfsetstrokecolor{currentstroke}%
\pgfsetstrokeopacity{0.700000}%
\pgfsetdash{}{0pt}%
\pgfpathmoveto{\pgfqpoint{3.831499in}{0.550000in}}%
\pgfpathlineto{\pgfqpoint{3.859020in}{0.550000in}}%
\pgfpathlineto{\pgfqpoint{3.859020in}{0.550000in}}%
\pgfpathlineto{\pgfqpoint{3.831499in}{0.550000in}}%
\pgfpathlineto{\pgfqpoint{3.831499in}{0.550000in}}%
\pgfpathclose%
\pgfusepath{fill}%
\end{pgfscope}%
\begin{pgfscope}%
\pgfpathrectangle{\pgfqpoint{1.250000in}{0.550000in}}{\pgfqpoint{7.750000in}{3.850000in}}%
\pgfusepath{clip}%
\pgfsetbuttcap%
\pgfsetmiterjoin%
\definecolor{currentfill}{rgb}{0.501961,0.501961,0.501961}%
\pgfsetfillcolor{currentfill}%
\pgfsetfillopacity{0.700000}%
\pgfsetlinewidth{0.000000pt}%
\definecolor{currentstroke}{rgb}{0.000000,0.000000,0.000000}%
\pgfsetstrokecolor{currentstroke}%
\pgfsetstrokeopacity{0.700000}%
\pgfsetdash{}{0pt}%
\pgfpathmoveto{\pgfqpoint{3.859020in}{0.550000in}}%
\pgfpathlineto{\pgfqpoint{3.886541in}{0.550000in}}%
\pgfpathlineto{\pgfqpoint{3.886541in}{0.841577in}}%
\pgfpathlineto{\pgfqpoint{3.859020in}{0.841577in}}%
\pgfpathlineto{\pgfqpoint{3.859020in}{0.550000in}}%
\pgfpathclose%
\pgfusepath{fill}%
\end{pgfscope}%
\begin{pgfscope}%
\pgfpathrectangle{\pgfqpoint{1.250000in}{0.550000in}}{\pgfqpoint{7.750000in}{3.850000in}}%
\pgfusepath{clip}%
\pgfsetbuttcap%
\pgfsetmiterjoin%
\definecolor{currentfill}{rgb}{0.501961,0.501961,0.501961}%
\pgfsetfillcolor{currentfill}%
\pgfsetfillopacity{0.700000}%
\pgfsetlinewidth{0.000000pt}%
\definecolor{currentstroke}{rgb}{0.000000,0.000000,0.000000}%
\pgfsetstrokecolor{currentstroke}%
\pgfsetstrokeopacity{0.700000}%
\pgfsetdash{}{0pt}%
\pgfpathmoveto{\pgfqpoint{3.886541in}{0.550000in}}%
\pgfpathlineto{\pgfqpoint{3.914062in}{0.550000in}}%
\pgfpathlineto{\pgfqpoint{3.914062in}{0.837620in}}%
\pgfpathlineto{\pgfqpoint{3.886541in}{0.837620in}}%
\pgfpathlineto{\pgfqpoint{3.886541in}{0.550000in}}%
\pgfpathclose%
\pgfusepath{fill}%
\end{pgfscope}%
\begin{pgfscope}%
\pgfpathrectangle{\pgfqpoint{1.250000in}{0.550000in}}{\pgfqpoint{7.750000in}{3.850000in}}%
\pgfusepath{clip}%
\pgfsetbuttcap%
\pgfsetmiterjoin%
\definecolor{currentfill}{rgb}{0.501961,0.501961,0.501961}%
\pgfsetfillcolor{currentfill}%
\pgfsetfillopacity{0.700000}%
\pgfsetlinewidth{0.000000pt}%
\definecolor{currentstroke}{rgb}{0.000000,0.000000,0.000000}%
\pgfsetstrokecolor{currentstroke}%
\pgfsetstrokeopacity{0.700000}%
\pgfsetdash{}{0pt}%
\pgfpathmoveto{\pgfqpoint{3.914062in}{0.550000in}}%
\pgfpathlineto{\pgfqpoint{3.941584in}{0.550000in}}%
\pgfpathlineto{\pgfqpoint{3.941584in}{0.827730in}}%
\pgfpathlineto{\pgfqpoint{3.914062in}{0.827730in}}%
\pgfpathlineto{\pgfqpoint{3.914062in}{0.550000in}}%
\pgfpathclose%
\pgfusepath{fill}%
\end{pgfscope}%
\begin{pgfscope}%
\pgfpathrectangle{\pgfqpoint{1.250000in}{0.550000in}}{\pgfqpoint{7.750000in}{3.850000in}}%
\pgfusepath{clip}%
\pgfsetbuttcap%
\pgfsetmiterjoin%
\definecolor{currentfill}{rgb}{0.501961,0.501961,0.501961}%
\pgfsetfillcolor{currentfill}%
\pgfsetfillopacity{0.700000}%
\pgfsetlinewidth{0.000000pt}%
\definecolor{currentstroke}{rgb}{0.000000,0.000000,0.000000}%
\pgfsetstrokecolor{currentstroke}%
\pgfsetstrokeopacity{0.700000}%
\pgfsetdash{}{0pt}%
\pgfpathmoveto{\pgfqpoint{3.941584in}{0.550000in}}%
\pgfpathlineto{\pgfqpoint{3.969105in}{0.550000in}}%
\pgfpathlineto{\pgfqpoint{3.969105in}{0.550000in}}%
\pgfpathlineto{\pgfqpoint{3.941584in}{0.550000in}}%
\pgfpathlineto{\pgfqpoint{3.941584in}{0.550000in}}%
\pgfpathclose%
\pgfusepath{fill}%
\end{pgfscope}%
\begin{pgfscope}%
\pgfpathrectangle{\pgfqpoint{1.250000in}{0.550000in}}{\pgfqpoint{7.750000in}{3.850000in}}%
\pgfusepath{clip}%
\pgfsetbuttcap%
\pgfsetmiterjoin%
\definecolor{currentfill}{rgb}{0.501961,0.501961,0.501961}%
\pgfsetfillcolor{currentfill}%
\pgfsetfillopacity{0.700000}%
\pgfsetlinewidth{0.000000pt}%
\definecolor{currentstroke}{rgb}{0.000000,0.000000,0.000000}%
\pgfsetstrokecolor{currentstroke}%
\pgfsetstrokeopacity{0.700000}%
\pgfsetdash{}{0pt}%
\pgfpathmoveto{\pgfqpoint{3.969105in}{0.550000in}}%
\pgfpathlineto{\pgfqpoint{3.996626in}{0.550000in}}%
\pgfpathlineto{\pgfqpoint{3.996626in}{0.813092in}}%
\pgfpathlineto{\pgfqpoint{3.969105in}{0.813092in}}%
\pgfpathlineto{\pgfqpoint{3.969105in}{0.550000in}}%
\pgfpathclose%
\pgfusepath{fill}%
\end{pgfscope}%
\begin{pgfscope}%
\pgfpathrectangle{\pgfqpoint{1.250000in}{0.550000in}}{\pgfqpoint{7.750000in}{3.850000in}}%
\pgfusepath{clip}%
\pgfsetbuttcap%
\pgfsetmiterjoin%
\definecolor{currentfill}{rgb}{0.501961,0.501961,0.501961}%
\pgfsetfillcolor{currentfill}%
\pgfsetfillopacity{0.700000}%
\pgfsetlinewidth{0.000000pt}%
\definecolor{currentstroke}{rgb}{0.000000,0.000000,0.000000}%
\pgfsetstrokecolor{currentstroke}%
\pgfsetstrokeopacity{0.700000}%
\pgfsetdash{}{0pt}%
\pgfpathmoveto{\pgfqpoint{3.996626in}{0.550000in}}%
\pgfpathlineto{\pgfqpoint{4.024148in}{0.550000in}}%
\pgfpathlineto{\pgfqpoint{4.024148in}{0.815070in}}%
\pgfpathlineto{\pgfqpoint{3.996626in}{0.815070in}}%
\pgfpathlineto{\pgfqpoint{3.996626in}{0.550000in}}%
\pgfpathclose%
\pgfusepath{fill}%
\end{pgfscope}%
\begin{pgfscope}%
\pgfpathrectangle{\pgfqpoint{1.250000in}{0.550000in}}{\pgfqpoint{7.750000in}{3.850000in}}%
\pgfusepath{clip}%
\pgfsetbuttcap%
\pgfsetmiterjoin%
\definecolor{currentfill}{rgb}{0.501961,0.501961,0.501961}%
\pgfsetfillcolor{currentfill}%
\pgfsetfillopacity{0.700000}%
\pgfsetlinewidth{0.000000pt}%
\definecolor{currentstroke}{rgb}{0.000000,0.000000,0.000000}%
\pgfsetstrokecolor{currentstroke}%
\pgfsetstrokeopacity{0.700000}%
\pgfsetdash{}{0pt}%
\pgfpathmoveto{\pgfqpoint{4.024148in}{0.550000in}}%
\pgfpathlineto{\pgfqpoint{4.051669in}{0.550000in}}%
\pgfpathlineto{\pgfqpoint{4.051669in}{0.804388in}}%
\pgfpathlineto{\pgfqpoint{4.024148in}{0.804388in}}%
\pgfpathlineto{\pgfqpoint{4.024148in}{0.550000in}}%
\pgfpathclose%
\pgfusepath{fill}%
\end{pgfscope}%
\begin{pgfscope}%
\pgfpathrectangle{\pgfqpoint{1.250000in}{0.550000in}}{\pgfqpoint{7.750000in}{3.850000in}}%
\pgfusepath{clip}%
\pgfsetbuttcap%
\pgfsetmiterjoin%
\definecolor{currentfill}{rgb}{0.501961,0.501961,0.501961}%
\pgfsetfillcolor{currentfill}%
\pgfsetfillopacity{0.700000}%
\pgfsetlinewidth{0.000000pt}%
\definecolor{currentstroke}{rgb}{0.000000,0.000000,0.000000}%
\pgfsetstrokecolor{currentstroke}%
\pgfsetstrokeopacity{0.700000}%
\pgfsetdash{}{0pt}%
\pgfpathmoveto{\pgfqpoint{4.051669in}{0.550000in}}%
\pgfpathlineto{\pgfqpoint{4.079190in}{0.550000in}}%
\pgfpathlineto{\pgfqpoint{4.079190in}{0.794893in}}%
\pgfpathlineto{\pgfqpoint{4.051669in}{0.794893in}}%
\pgfpathlineto{\pgfqpoint{4.051669in}{0.550000in}}%
\pgfpathclose%
\pgfusepath{fill}%
\end{pgfscope}%
\begin{pgfscope}%
\pgfpathrectangle{\pgfqpoint{1.250000in}{0.550000in}}{\pgfqpoint{7.750000in}{3.850000in}}%
\pgfusepath{clip}%
\pgfsetbuttcap%
\pgfsetmiterjoin%
\definecolor{currentfill}{rgb}{0.501961,0.501961,0.501961}%
\pgfsetfillcolor{currentfill}%
\pgfsetfillopacity{0.700000}%
\pgfsetlinewidth{0.000000pt}%
\definecolor{currentstroke}{rgb}{0.000000,0.000000,0.000000}%
\pgfsetstrokecolor{currentstroke}%
\pgfsetstrokeopacity{0.700000}%
\pgfsetdash{}{0pt}%
\pgfpathmoveto{\pgfqpoint{4.079190in}{0.550000in}}%
\pgfpathlineto{\pgfqpoint{4.106712in}{0.550000in}}%
\pgfpathlineto{\pgfqpoint{4.106712in}{0.550000in}}%
\pgfpathlineto{\pgfqpoint{4.079190in}{0.550000in}}%
\pgfpathlineto{\pgfqpoint{4.079190in}{0.550000in}}%
\pgfpathclose%
\pgfusepath{fill}%
\end{pgfscope}%
\begin{pgfscope}%
\pgfpathrectangle{\pgfqpoint{1.250000in}{0.550000in}}{\pgfqpoint{7.750000in}{3.850000in}}%
\pgfusepath{clip}%
\pgfsetbuttcap%
\pgfsetmiterjoin%
\definecolor{currentfill}{rgb}{0.501961,0.501961,0.501961}%
\pgfsetfillcolor{currentfill}%
\pgfsetfillopacity{0.700000}%
\pgfsetlinewidth{0.000000pt}%
\definecolor{currentstroke}{rgb}{0.000000,0.000000,0.000000}%
\pgfsetstrokecolor{currentstroke}%
\pgfsetstrokeopacity{0.700000}%
\pgfsetdash{}{0pt}%
\pgfpathmoveto{\pgfqpoint{4.106712in}{0.550000in}}%
\pgfpathlineto{\pgfqpoint{4.134233in}{0.550000in}}%
\pgfpathlineto{\pgfqpoint{4.134233in}{0.799640in}}%
\pgfpathlineto{\pgfqpoint{4.106712in}{0.799640in}}%
\pgfpathlineto{\pgfqpoint{4.106712in}{0.550000in}}%
\pgfpathclose%
\pgfusepath{fill}%
\end{pgfscope}%
\begin{pgfscope}%
\pgfpathrectangle{\pgfqpoint{1.250000in}{0.550000in}}{\pgfqpoint{7.750000in}{3.850000in}}%
\pgfusepath{clip}%
\pgfsetbuttcap%
\pgfsetmiterjoin%
\definecolor{currentfill}{rgb}{0.501961,0.501961,0.501961}%
\pgfsetfillcolor{currentfill}%
\pgfsetfillopacity{0.700000}%
\pgfsetlinewidth{0.000000pt}%
\definecolor{currentstroke}{rgb}{0.000000,0.000000,0.000000}%
\pgfsetstrokecolor{currentstroke}%
\pgfsetstrokeopacity{0.700000}%
\pgfsetdash{}{0pt}%
\pgfpathmoveto{\pgfqpoint{4.134233in}{0.550000in}}%
\pgfpathlineto{\pgfqpoint{4.161754in}{0.550000in}}%
\pgfpathlineto{\pgfqpoint{4.161754in}{0.813487in}}%
\pgfpathlineto{\pgfqpoint{4.134233in}{0.813487in}}%
\pgfpathlineto{\pgfqpoint{4.134233in}{0.550000in}}%
\pgfpathclose%
\pgfusepath{fill}%
\end{pgfscope}%
\begin{pgfscope}%
\pgfpathrectangle{\pgfqpoint{1.250000in}{0.550000in}}{\pgfqpoint{7.750000in}{3.850000in}}%
\pgfusepath{clip}%
\pgfsetbuttcap%
\pgfsetmiterjoin%
\definecolor{currentfill}{rgb}{0.501961,0.501961,0.501961}%
\pgfsetfillcolor{currentfill}%
\pgfsetfillopacity{0.700000}%
\pgfsetlinewidth{0.000000pt}%
\definecolor{currentstroke}{rgb}{0.000000,0.000000,0.000000}%
\pgfsetstrokecolor{currentstroke}%
\pgfsetstrokeopacity{0.700000}%
\pgfsetdash{}{0pt}%
\pgfpathmoveto{\pgfqpoint{4.161754in}{0.550000in}}%
\pgfpathlineto{\pgfqpoint{4.189276in}{0.550000in}}%
\pgfpathlineto{\pgfqpoint{4.189276in}{0.819422in}}%
\pgfpathlineto{\pgfqpoint{4.161754in}{0.819422in}}%
\pgfpathlineto{\pgfqpoint{4.161754in}{0.550000in}}%
\pgfpathclose%
\pgfusepath{fill}%
\end{pgfscope}%
\begin{pgfscope}%
\pgfpathrectangle{\pgfqpoint{1.250000in}{0.550000in}}{\pgfqpoint{7.750000in}{3.850000in}}%
\pgfusepath{clip}%
\pgfsetbuttcap%
\pgfsetmiterjoin%
\definecolor{currentfill}{rgb}{0.501961,0.501961,0.501961}%
\pgfsetfillcolor{currentfill}%
\pgfsetfillopacity{0.700000}%
\pgfsetlinewidth{0.000000pt}%
\definecolor{currentstroke}{rgb}{0.000000,0.000000,0.000000}%
\pgfsetstrokecolor{currentstroke}%
\pgfsetstrokeopacity{0.700000}%
\pgfsetdash{}{0pt}%
\pgfpathmoveto{\pgfqpoint{4.189276in}{0.550000in}}%
\pgfpathlineto{\pgfqpoint{4.216797in}{0.550000in}}%
\pgfpathlineto{\pgfqpoint{4.216797in}{0.790937in}}%
\pgfpathlineto{\pgfqpoint{4.189276in}{0.790937in}}%
\pgfpathlineto{\pgfqpoint{4.189276in}{0.550000in}}%
\pgfpathclose%
\pgfusepath{fill}%
\end{pgfscope}%
\begin{pgfscope}%
\pgfpathrectangle{\pgfqpoint{1.250000in}{0.550000in}}{\pgfqpoint{7.750000in}{3.850000in}}%
\pgfusepath{clip}%
\pgfsetbuttcap%
\pgfsetmiterjoin%
\definecolor{currentfill}{rgb}{0.501961,0.501961,0.501961}%
\pgfsetfillcolor{currentfill}%
\pgfsetfillopacity{0.700000}%
\pgfsetlinewidth{0.000000pt}%
\definecolor{currentstroke}{rgb}{0.000000,0.000000,0.000000}%
\pgfsetstrokecolor{currentstroke}%
\pgfsetstrokeopacity{0.700000}%
\pgfsetdash{}{0pt}%
\pgfpathmoveto{\pgfqpoint{4.216797in}{0.550000in}}%
\pgfpathlineto{\pgfqpoint{4.244318in}{0.550000in}}%
\pgfpathlineto{\pgfqpoint{4.244318in}{0.550000in}}%
\pgfpathlineto{\pgfqpoint{4.216797in}{0.550000in}}%
\pgfpathlineto{\pgfqpoint{4.216797in}{0.550000in}}%
\pgfpathclose%
\pgfusepath{fill}%
\end{pgfscope}%
\begin{pgfscope}%
\pgfpathrectangle{\pgfqpoint{1.250000in}{0.550000in}}{\pgfqpoint{7.750000in}{3.850000in}}%
\pgfusepath{clip}%
\pgfsetbuttcap%
\pgfsetmiterjoin%
\definecolor{currentfill}{rgb}{0.501961,0.501961,0.501961}%
\pgfsetfillcolor{currentfill}%
\pgfsetfillopacity{0.700000}%
\pgfsetlinewidth{0.000000pt}%
\definecolor{currentstroke}{rgb}{0.000000,0.000000,0.000000}%
\pgfsetstrokecolor{currentstroke}%
\pgfsetstrokeopacity{0.700000}%
\pgfsetdash{}{0pt}%
\pgfpathmoveto{\pgfqpoint{4.244318in}{0.550000in}}%
\pgfpathlineto{\pgfqpoint{4.271839in}{0.550000in}}%
\pgfpathlineto{\pgfqpoint{4.271839in}{0.788958in}}%
\pgfpathlineto{\pgfqpoint{4.244318in}{0.788958in}}%
\pgfpathlineto{\pgfqpoint{4.244318in}{0.550000in}}%
\pgfpathclose%
\pgfusepath{fill}%
\end{pgfscope}%
\begin{pgfscope}%
\pgfpathrectangle{\pgfqpoint{1.250000in}{0.550000in}}{\pgfqpoint{7.750000in}{3.850000in}}%
\pgfusepath{clip}%
\pgfsetbuttcap%
\pgfsetmiterjoin%
\definecolor{currentfill}{rgb}{0.501961,0.501961,0.501961}%
\pgfsetfillcolor{currentfill}%
\pgfsetfillopacity{0.700000}%
\pgfsetlinewidth{0.000000pt}%
\definecolor{currentstroke}{rgb}{0.000000,0.000000,0.000000}%
\pgfsetstrokecolor{currentstroke}%
\pgfsetstrokeopacity{0.700000}%
\pgfsetdash{}{0pt}%
\pgfpathmoveto{\pgfqpoint{4.271839in}{0.550000in}}%
\pgfpathlineto{\pgfqpoint{4.299361in}{0.550000in}}%
\pgfpathlineto{\pgfqpoint{4.299361in}{0.779859in}}%
\pgfpathlineto{\pgfqpoint{4.271839in}{0.779859in}}%
\pgfpathlineto{\pgfqpoint{4.271839in}{0.550000in}}%
\pgfpathclose%
\pgfusepath{fill}%
\end{pgfscope}%
\begin{pgfscope}%
\pgfpathrectangle{\pgfqpoint{1.250000in}{0.550000in}}{\pgfqpoint{7.750000in}{3.850000in}}%
\pgfusepath{clip}%
\pgfsetbuttcap%
\pgfsetmiterjoin%
\definecolor{currentfill}{rgb}{0.501961,0.501961,0.501961}%
\pgfsetfillcolor{currentfill}%
\pgfsetfillopacity{0.700000}%
\pgfsetlinewidth{0.000000pt}%
\definecolor{currentstroke}{rgb}{0.000000,0.000000,0.000000}%
\pgfsetstrokecolor{currentstroke}%
\pgfsetstrokeopacity{0.700000}%
\pgfsetdash{}{0pt}%
\pgfpathmoveto{\pgfqpoint{4.299361in}{0.550000in}}%
\pgfpathlineto{\pgfqpoint{4.326882in}{0.550000in}}%
\pgfpathlineto{\pgfqpoint{4.326882in}{0.763243in}}%
\pgfpathlineto{\pgfqpoint{4.299361in}{0.763243in}}%
\pgfpathlineto{\pgfqpoint{4.299361in}{0.550000in}}%
\pgfpathclose%
\pgfusepath{fill}%
\end{pgfscope}%
\begin{pgfscope}%
\pgfpathrectangle{\pgfqpoint{1.250000in}{0.550000in}}{\pgfqpoint{7.750000in}{3.850000in}}%
\pgfusepath{clip}%
\pgfsetbuttcap%
\pgfsetmiterjoin%
\definecolor{currentfill}{rgb}{0.501961,0.501961,0.501961}%
\pgfsetfillcolor{currentfill}%
\pgfsetfillopacity{0.700000}%
\pgfsetlinewidth{0.000000pt}%
\definecolor{currentstroke}{rgb}{0.000000,0.000000,0.000000}%
\pgfsetstrokecolor{currentstroke}%
\pgfsetstrokeopacity{0.700000}%
\pgfsetdash{}{0pt}%
\pgfpathmoveto{\pgfqpoint{4.326882in}{0.550000in}}%
\pgfpathlineto{\pgfqpoint{4.354403in}{0.550000in}}%
\pgfpathlineto{\pgfqpoint{4.354403in}{0.756517in}}%
\pgfpathlineto{\pgfqpoint{4.326882in}{0.756517in}}%
\pgfpathlineto{\pgfqpoint{4.326882in}{0.550000in}}%
\pgfpathclose%
\pgfusepath{fill}%
\end{pgfscope}%
\begin{pgfscope}%
\pgfpathrectangle{\pgfqpoint{1.250000in}{0.550000in}}{\pgfqpoint{7.750000in}{3.850000in}}%
\pgfusepath{clip}%
\pgfsetbuttcap%
\pgfsetmiterjoin%
\definecolor{currentfill}{rgb}{0.501961,0.501961,0.501961}%
\pgfsetfillcolor{currentfill}%
\pgfsetfillopacity{0.700000}%
\pgfsetlinewidth{0.000000pt}%
\definecolor{currentstroke}{rgb}{0.000000,0.000000,0.000000}%
\pgfsetstrokecolor{currentstroke}%
\pgfsetstrokeopacity{0.700000}%
\pgfsetdash{}{0pt}%
\pgfpathmoveto{\pgfqpoint{4.354403in}{0.550000in}}%
\pgfpathlineto{\pgfqpoint{4.381925in}{0.550000in}}%
\pgfpathlineto{\pgfqpoint{4.381925in}{0.550000in}}%
\pgfpathlineto{\pgfqpoint{4.354403in}{0.550000in}}%
\pgfpathlineto{\pgfqpoint{4.354403in}{0.550000in}}%
\pgfpathclose%
\pgfusepath{fill}%
\end{pgfscope}%
\begin{pgfscope}%
\pgfpathrectangle{\pgfqpoint{1.250000in}{0.550000in}}{\pgfqpoint{7.750000in}{3.850000in}}%
\pgfusepath{clip}%
\pgfsetbuttcap%
\pgfsetmiterjoin%
\definecolor{currentfill}{rgb}{0.501961,0.501961,0.501961}%
\pgfsetfillcolor{currentfill}%
\pgfsetfillopacity{0.700000}%
\pgfsetlinewidth{0.000000pt}%
\definecolor{currentstroke}{rgb}{0.000000,0.000000,0.000000}%
\pgfsetstrokecolor{currentstroke}%
\pgfsetstrokeopacity{0.700000}%
\pgfsetdash{}{0pt}%
\pgfpathmoveto{\pgfqpoint{4.381925in}{0.550000in}}%
\pgfpathlineto{\pgfqpoint{4.409446in}{0.550000in}}%
\pgfpathlineto{\pgfqpoint{4.409446in}{0.744253in}}%
\pgfpathlineto{\pgfqpoint{4.381925in}{0.744253in}}%
\pgfpathlineto{\pgfqpoint{4.381925in}{0.550000in}}%
\pgfpathclose%
\pgfusepath{fill}%
\end{pgfscope}%
\begin{pgfscope}%
\pgfpathrectangle{\pgfqpoint{1.250000in}{0.550000in}}{\pgfqpoint{7.750000in}{3.850000in}}%
\pgfusepath{clip}%
\pgfsetbuttcap%
\pgfsetmiterjoin%
\definecolor{currentfill}{rgb}{0.501961,0.501961,0.501961}%
\pgfsetfillcolor{currentfill}%
\pgfsetfillopacity{0.700000}%
\pgfsetlinewidth{0.000000pt}%
\definecolor{currentstroke}{rgb}{0.000000,0.000000,0.000000}%
\pgfsetstrokecolor{currentstroke}%
\pgfsetstrokeopacity{0.700000}%
\pgfsetdash{}{0pt}%
\pgfpathmoveto{\pgfqpoint{4.409446in}{0.550000in}}%
\pgfpathlineto{\pgfqpoint{4.436967in}{0.550000in}}%
\pgfpathlineto{\pgfqpoint{4.436967in}{0.737923in}}%
\pgfpathlineto{\pgfqpoint{4.409446in}{0.737923in}}%
\pgfpathlineto{\pgfqpoint{4.409446in}{0.550000in}}%
\pgfpathclose%
\pgfusepath{fill}%
\end{pgfscope}%
\begin{pgfscope}%
\pgfpathrectangle{\pgfqpoint{1.250000in}{0.550000in}}{\pgfqpoint{7.750000in}{3.850000in}}%
\pgfusepath{clip}%
\pgfsetbuttcap%
\pgfsetmiterjoin%
\definecolor{currentfill}{rgb}{0.501961,0.501961,0.501961}%
\pgfsetfillcolor{currentfill}%
\pgfsetfillopacity{0.700000}%
\pgfsetlinewidth{0.000000pt}%
\definecolor{currentstroke}{rgb}{0.000000,0.000000,0.000000}%
\pgfsetstrokecolor{currentstroke}%
\pgfsetstrokeopacity{0.700000}%
\pgfsetdash{}{0pt}%
\pgfpathmoveto{\pgfqpoint{4.436967in}{0.550000in}}%
\pgfpathlineto{\pgfqpoint{4.464489in}{0.550000in}}%
\pgfpathlineto{\pgfqpoint{4.464489in}{0.731593in}}%
\pgfpathlineto{\pgfqpoint{4.436967in}{0.731593in}}%
\pgfpathlineto{\pgfqpoint{4.436967in}{0.550000in}}%
\pgfpathclose%
\pgfusepath{fill}%
\end{pgfscope}%
\begin{pgfscope}%
\pgfpathrectangle{\pgfqpoint{1.250000in}{0.550000in}}{\pgfqpoint{7.750000in}{3.850000in}}%
\pgfusepath{clip}%
\pgfsetbuttcap%
\pgfsetmiterjoin%
\definecolor{currentfill}{rgb}{0.501961,0.501961,0.501961}%
\pgfsetfillcolor{currentfill}%
\pgfsetfillopacity{0.700000}%
\pgfsetlinewidth{0.000000pt}%
\definecolor{currentstroke}{rgb}{0.000000,0.000000,0.000000}%
\pgfsetstrokecolor{currentstroke}%
\pgfsetstrokeopacity{0.700000}%
\pgfsetdash{}{0pt}%
\pgfpathmoveto{\pgfqpoint{4.464489in}{0.550000in}}%
\pgfpathlineto{\pgfqpoint{4.492010in}{0.550000in}}%
\pgfpathlineto{\pgfqpoint{4.492010in}{0.723284in}}%
\pgfpathlineto{\pgfqpoint{4.464489in}{0.723284in}}%
\pgfpathlineto{\pgfqpoint{4.464489in}{0.550000in}}%
\pgfpathclose%
\pgfusepath{fill}%
\end{pgfscope}%
\begin{pgfscope}%
\pgfpathrectangle{\pgfqpoint{1.250000in}{0.550000in}}{\pgfqpoint{7.750000in}{3.850000in}}%
\pgfusepath{clip}%
\pgfsetbuttcap%
\pgfsetmiterjoin%
\definecolor{currentfill}{rgb}{0.501961,0.501961,0.501961}%
\pgfsetfillcolor{currentfill}%
\pgfsetfillopacity{0.700000}%
\pgfsetlinewidth{0.000000pt}%
\definecolor{currentstroke}{rgb}{0.000000,0.000000,0.000000}%
\pgfsetstrokecolor{currentstroke}%
\pgfsetstrokeopacity{0.700000}%
\pgfsetdash{}{0pt}%
\pgfpathmoveto{\pgfqpoint{4.492010in}{0.550000in}}%
\pgfpathlineto{\pgfqpoint{4.519531in}{0.550000in}}%
\pgfpathlineto{\pgfqpoint{4.519531in}{0.550000in}}%
\pgfpathlineto{\pgfqpoint{4.492010in}{0.550000in}}%
\pgfpathlineto{\pgfqpoint{4.492010in}{0.550000in}}%
\pgfpathclose%
\pgfusepath{fill}%
\end{pgfscope}%
\begin{pgfscope}%
\pgfpathrectangle{\pgfqpoint{1.250000in}{0.550000in}}{\pgfqpoint{7.750000in}{3.850000in}}%
\pgfusepath{clip}%
\pgfsetbuttcap%
\pgfsetmiterjoin%
\definecolor{currentfill}{rgb}{0.501961,0.501961,0.501961}%
\pgfsetfillcolor{currentfill}%
\pgfsetfillopacity{0.700000}%
\pgfsetlinewidth{0.000000pt}%
\definecolor{currentstroke}{rgb}{0.000000,0.000000,0.000000}%
\pgfsetstrokecolor{currentstroke}%
\pgfsetstrokeopacity{0.700000}%
\pgfsetdash{}{0pt}%
\pgfpathmoveto{\pgfqpoint{4.519531in}{0.550000in}}%
\pgfpathlineto{\pgfqpoint{4.547053in}{0.550000in}}%
\pgfpathlineto{\pgfqpoint{4.547053in}{0.726449in}}%
\pgfpathlineto{\pgfqpoint{4.519531in}{0.726449in}}%
\pgfpathlineto{\pgfqpoint{4.519531in}{0.550000in}}%
\pgfpathclose%
\pgfusepath{fill}%
\end{pgfscope}%
\begin{pgfscope}%
\pgfpathrectangle{\pgfqpoint{1.250000in}{0.550000in}}{\pgfqpoint{7.750000in}{3.850000in}}%
\pgfusepath{clip}%
\pgfsetbuttcap%
\pgfsetmiterjoin%
\definecolor{currentfill}{rgb}{0.501961,0.501961,0.501961}%
\pgfsetfillcolor{currentfill}%
\pgfsetfillopacity{0.700000}%
\pgfsetlinewidth{0.000000pt}%
\definecolor{currentstroke}{rgb}{0.000000,0.000000,0.000000}%
\pgfsetstrokecolor{currentstroke}%
\pgfsetstrokeopacity{0.700000}%
\pgfsetdash{}{0pt}%
\pgfpathmoveto{\pgfqpoint{4.547053in}{0.550000in}}%
\pgfpathlineto{\pgfqpoint{4.574574in}{0.550000in}}%
\pgfpathlineto{\pgfqpoint{4.574574in}{0.727636in}}%
\pgfpathlineto{\pgfqpoint{4.547053in}{0.727636in}}%
\pgfpathlineto{\pgfqpoint{4.547053in}{0.550000in}}%
\pgfpathclose%
\pgfusepath{fill}%
\end{pgfscope}%
\begin{pgfscope}%
\pgfpathrectangle{\pgfqpoint{1.250000in}{0.550000in}}{\pgfqpoint{7.750000in}{3.850000in}}%
\pgfusepath{clip}%
\pgfsetbuttcap%
\pgfsetmiterjoin%
\definecolor{currentfill}{rgb}{0.501961,0.501961,0.501961}%
\pgfsetfillcolor{currentfill}%
\pgfsetfillopacity{0.700000}%
\pgfsetlinewidth{0.000000pt}%
\definecolor{currentstroke}{rgb}{0.000000,0.000000,0.000000}%
\pgfsetstrokecolor{currentstroke}%
\pgfsetstrokeopacity{0.700000}%
\pgfsetdash{}{0pt}%
\pgfpathmoveto{\pgfqpoint{4.574574in}{0.550000in}}%
\pgfpathlineto{\pgfqpoint{4.602095in}{0.550000in}}%
\pgfpathlineto{\pgfqpoint{4.602095in}{0.726054in}}%
\pgfpathlineto{\pgfqpoint{4.574574in}{0.726054in}}%
\pgfpathlineto{\pgfqpoint{4.574574in}{0.550000in}}%
\pgfpathclose%
\pgfusepath{fill}%
\end{pgfscope}%
\begin{pgfscope}%
\pgfpathrectangle{\pgfqpoint{1.250000in}{0.550000in}}{\pgfqpoint{7.750000in}{3.850000in}}%
\pgfusepath{clip}%
\pgfsetbuttcap%
\pgfsetmiterjoin%
\definecolor{currentfill}{rgb}{0.501961,0.501961,0.501961}%
\pgfsetfillcolor{currentfill}%
\pgfsetfillopacity{0.700000}%
\pgfsetlinewidth{0.000000pt}%
\definecolor{currentstroke}{rgb}{0.000000,0.000000,0.000000}%
\pgfsetstrokecolor{currentstroke}%
\pgfsetstrokeopacity{0.700000}%
\pgfsetdash{}{0pt}%
\pgfpathmoveto{\pgfqpoint{4.602095in}{0.550000in}}%
\pgfpathlineto{\pgfqpoint{4.629616in}{0.550000in}}%
\pgfpathlineto{\pgfqpoint{4.629616in}{0.729219in}}%
\pgfpathlineto{\pgfqpoint{4.602095in}{0.729219in}}%
\pgfpathlineto{\pgfqpoint{4.602095in}{0.550000in}}%
\pgfpathclose%
\pgfusepath{fill}%
\end{pgfscope}%
\begin{pgfscope}%
\pgfpathrectangle{\pgfqpoint{1.250000in}{0.550000in}}{\pgfqpoint{7.750000in}{3.850000in}}%
\pgfusepath{clip}%
\pgfsetbuttcap%
\pgfsetmiterjoin%
\definecolor{currentfill}{rgb}{0.501961,0.501961,0.501961}%
\pgfsetfillcolor{currentfill}%
\pgfsetfillopacity{0.700000}%
\pgfsetlinewidth{0.000000pt}%
\definecolor{currentstroke}{rgb}{0.000000,0.000000,0.000000}%
\pgfsetstrokecolor{currentstroke}%
\pgfsetstrokeopacity{0.700000}%
\pgfsetdash{}{0pt}%
\pgfpathmoveto{\pgfqpoint{4.629616in}{0.550000in}}%
\pgfpathlineto{\pgfqpoint{4.657138in}{0.550000in}}%
\pgfpathlineto{\pgfqpoint{4.657138in}{0.550000in}}%
\pgfpathlineto{\pgfqpoint{4.629616in}{0.550000in}}%
\pgfpathlineto{\pgfqpoint{4.629616in}{0.550000in}}%
\pgfpathclose%
\pgfusepath{fill}%
\end{pgfscope}%
\begin{pgfscope}%
\pgfpathrectangle{\pgfqpoint{1.250000in}{0.550000in}}{\pgfqpoint{7.750000in}{3.850000in}}%
\pgfusepath{clip}%
\pgfsetbuttcap%
\pgfsetmiterjoin%
\definecolor{currentfill}{rgb}{0.501961,0.501961,0.501961}%
\pgfsetfillcolor{currentfill}%
\pgfsetfillopacity{0.700000}%
\pgfsetlinewidth{0.000000pt}%
\definecolor{currentstroke}{rgb}{0.000000,0.000000,0.000000}%
\pgfsetstrokecolor{currentstroke}%
\pgfsetstrokeopacity{0.700000}%
\pgfsetdash{}{0pt}%
\pgfpathmoveto{\pgfqpoint{4.657138in}{0.550000in}}%
\pgfpathlineto{\pgfqpoint{4.684659in}{0.550000in}}%
\pgfpathlineto{\pgfqpoint{4.684659in}{0.726845in}}%
\pgfpathlineto{\pgfqpoint{4.657138in}{0.726845in}}%
\pgfpathlineto{\pgfqpoint{4.657138in}{0.550000in}}%
\pgfpathclose%
\pgfusepath{fill}%
\end{pgfscope}%
\begin{pgfscope}%
\pgfpathrectangle{\pgfqpoint{1.250000in}{0.550000in}}{\pgfqpoint{7.750000in}{3.850000in}}%
\pgfusepath{clip}%
\pgfsetbuttcap%
\pgfsetmiterjoin%
\definecolor{currentfill}{rgb}{0.501961,0.501961,0.501961}%
\pgfsetfillcolor{currentfill}%
\pgfsetfillopacity{0.700000}%
\pgfsetlinewidth{0.000000pt}%
\definecolor{currentstroke}{rgb}{0.000000,0.000000,0.000000}%
\pgfsetstrokecolor{currentstroke}%
\pgfsetstrokeopacity{0.700000}%
\pgfsetdash{}{0pt}%
\pgfpathmoveto{\pgfqpoint{4.684659in}{0.550000in}}%
\pgfpathlineto{\pgfqpoint{4.712180in}{0.550000in}}%
\pgfpathlineto{\pgfqpoint{4.712180in}{0.707855in}}%
\pgfpathlineto{\pgfqpoint{4.684659in}{0.707855in}}%
\pgfpathlineto{\pgfqpoint{4.684659in}{0.550000in}}%
\pgfpathclose%
\pgfusepath{fill}%
\end{pgfscope}%
\begin{pgfscope}%
\pgfpathrectangle{\pgfqpoint{1.250000in}{0.550000in}}{\pgfqpoint{7.750000in}{3.850000in}}%
\pgfusepath{clip}%
\pgfsetbuttcap%
\pgfsetmiterjoin%
\definecolor{currentfill}{rgb}{0.501961,0.501961,0.501961}%
\pgfsetfillcolor{currentfill}%
\pgfsetfillopacity{0.700000}%
\pgfsetlinewidth{0.000000pt}%
\definecolor{currentstroke}{rgb}{0.000000,0.000000,0.000000}%
\pgfsetstrokecolor{currentstroke}%
\pgfsetstrokeopacity{0.700000}%
\pgfsetdash{}{0pt}%
\pgfpathmoveto{\pgfqpoint{4.712180in}{0.550000in}}%
\pgfpathlineto{\pgfqpoint{4.739702in}{0.550000in}}%
\pgfpathlineto{\pgfqpoint{4.739702in}{0.705877in}}%
\pgfpathlineto{\pgfqpoint{4.712180in}{0.705877in}}%
\pgfpathlineto{\pgfqpoint{4.712180in}{0.550000in}}%
\pgfpathclose%
\pgfusepath{fill}%
\end{pgfscope}%
\begin{pgfscope}%
\pgfpathrectangle{\pgfqpoint{1.250000in}{0.550000in}}{\pgfqpoint{7.750000in}{3.850000in}}%
\pgfusepath{clip}%
\pgfsetbuttcap%
\pgfsetmiterjoin%
\definecolor{currentfill}{rgb}{0.501961,0.501961,0.501961}%
\pgfsetfillcolor{currentfill}%
\pgfsetfillopacity{0.700000}%
\pgfsetlinewidth{0.000000pt}%
\definecolor{currentstroke}{rgb}{0.000000,0.000000,0.000000}%
\pgfsetstrokecolor{currentstroke}%
\pgfsetstrokeopacity{0.700000}%
\pgfsetdash{}{0pt}%
\pgfpathmoveto{\pgfqpoint{4.739702in}{0.550000in}}%
\pgfpathlineto{\pgfqpoint{4.767223in}{0.550000in}}%
\pgfpathlineto{\pgfqpoint{4.767223in}{0.709833in}}%
\pgfpathlineto{\pgfqpoint{4.739702in}{0.709833in}}%
\pgfpathlineto{\pgfqpoint{4.739702in}{0.550000in}}%
\pgfpathclose%
\pgfusepath{fill}%
\end{pgfscope}%
\begin{pgfscope}%
\pgfpathrectangle{\pgfqpoint{1.250000in}{0.550000in}}{\pgfqpoint{7.750000in}{3.850000in}}%
\pgfusepath{clip}%
\pgfsetbuttcap%
\pgfsetmiterjoin%
\definecolor{currentfill}{rgb}{0.501961,0.501961,0.501961}%
\pgfsetfillcolor{currentfill}%
\pgfsetfillopacity{0.700000}%
\pgfsetlinewidth{0.000000pt}%
\definecolor{currentstroke}{rgb}{0.000000,0.000000,0.000000}%
\pgfsetstrokecolor{currentstroke}%
\pgfsetstrokeopacity{0.700000}%
\pgfsetdash{}{0pt}%
\pgfpathmoveto{\pgfqpoint{4.767223in}{0.550000in}}%
\pgfpathlineto{\pgfqpoint{4.794744in}{0.550000in}}%
\pgfpathlineto{\pgfqpoint{4.794744in}{0.550000in}}%
\pgfpathlineto{\pgfqpoint{4.767223in}{0.550000in}}%
\pgfpathlineto{\pgfqpoint{4.767223in}{0.550000in}}%
\pgfpathclose%
\pgfusepath{fill}%
\end{pgfscope}%
\begin{pgfscope}%
\pgfpathrectangle{\pgfqpoint{1.250000in}{0.550000in}}{\pgfqpoint{7.750000in}{3.850000in}}%
\pgfusepath{clip}%
\pgfsetbuttcap%
\pgfsetmiterjoin%
\definecolor{currentfill}{rgb}{0.501961,0.501961,0.501961}%
\pgfsetfillcolor{currentfill}%
\pgfsetfillopacity{0.700000}%
\pgfsetlinewidth{0.000000pt}%
\definecolor{currentstroke}{rgb}{0.000000,0.000000,0.000000}%
\pgfsetstrokecolor{currentstroke}%
\pgfsetstrokeopacity{0.700000}%
\pgfsetdash{}{0pt}%
\pgfpathmoveto{\pgfqpoint{4.794744in}{0.550000in}}%
\pgfpathlineto{\pgfqpoint{4.822266in}{0.550000in}}%
\pgfpathlineto{\pgfqpoint{4.822266in}{0.699942in}}%
\pgfpathlineto{\pgfqpoint{4.794744in}{0.699942in}}%
\pgfpathlineto{\pgfqpoint{4.794744in}{0.550000in}}%
\pgfpathclose%
\pgfusepath{fill}%
\end{pgfscope}%
\begin{pgfscope}%
\pgfpathrectangle{\pgfqpoint{1.250000in}{0.550000in}}{\pgfqpoint{7.750000in}{3.850000in}}%
\pgfusepath{clip}%
\pgfsetbuttcap%
\pgfsetmiterjoin%
\definecolor{currentfill}{rgb}{0.501961,0.501961,0.501961}%
\pgfsetfillcolor{currentfill}%
\pgfsetfillopacity{0.700000}%
\pgfsetlinewidth{0.000000pt}%
\definecolor{currentstroke}{rgb}{0.000000,0.000000,0.000000}%
\pgfsetstrokecolor{currentstroke}%
\pgfsetstrokeopacity{0.700000}%
\pgfsetdash{}{0pt}%
\pgfpathmoveto{\pgfqpoint{4.822266in}{0.550000in}}%
\pgfpathlineto{\pgfqpoint{4.849787in}{0.550000in}}%
\pgfpathlineto{\pgfqpoint{4.849787in}{0.689261in}}%
\pgfpathlineto{\pgfqpoint{4.822266in}{0.689261in}}%
\pgfpathlineto{\pgfqpoint{4.822266in}{0.550000in}}%
\pgfpathclose%
\pgfusepath{fill}%
\end{pgfscope}%
\begin{pgfscope}%
\pgfpathrectangle{\pgfqpoint{1.250000in}{0.550000in}}{\pgfqpoint{7.750000in}{3.850000in}}%
\pgfusepath{clip}%
\pgfsetbuttcap%
\pgfsetmiterjoin%
\definecolor{currentfill}{rgb}{0.501961,0.501961,0.501961}%
\pgfsetfillcolor{currentfill}%
\pgfsetfillopacity{0.700000}%
\pgfsetlinewidth{0.000000pt}%
\definecolor{currentstroke}{rgb}{0.000000,0.000000,0.000000}%
\pgfsetstrokecolor{currentstroke}%
\pgfsetstrokeopacity{0.700000}%
\pgfsetdash{}{0pt}%
\pgfpathmoveto{\pgfqpoint{4.849787in}{0.550000in}}%
\pgfpathlineto{\pgfqpoint{4.877308in}{0.550000in}}%
\pgfpathlineto{\pgfqpoint{4.877308in}{0.690843in}}%
\pgfpathlineto{\pgfqpoint{4.849787in}{0.690843in}}%
\pgfpathlineto{\pgfqpoint{4.849787in}{0.550000in}}%
\pgfpathclose%
\pgfusepath{fill}%
\end{pgfscope}%
\begin{pgfscope}%
\pgfpathrectangle{\pgfqpoint{1.250000in}{0.550000in}}{\pgfqpoint{7.750000in}{3.850000in}}%
\pgfusepath{clip}%
\pgfsetbuttcap%
\pgfsetmiterjoin%
\definecolor{currentfill}{rgb}{0.501961,0.501961,0.501961}%
\pgfsetfillcolor{currentfill}%
\pgfsetfillopacity{0.700000}%
\pgfsetlinewidth{0.000000pt}%
\definecolor{currentstroke}{rgb}{0.000000,0.000000,0.000000}%
\pgfsetstrokecolor{currentstroke}%
\pgfsetstrokeopacity{0.700000}%
\pgfsetdash{}{0pt}%
\pgfpathmoveto{\pgfqpoint{4.877308in}{0.550000in}}%
\pgfpathlineto{\pgfqpoint{4.904830in}{0.550000in}}%
\pgfpathlineto{\pgfqpoint{4.904830in}{0.550000in}}%
\pgfpathlineto{\pgfqpoint{4.877308in}{0.550000in}}%
\pgfpathlineto{\pgfqpoint{4.877308in}{0.550000in}}%
\pgfpathclose%
\pgfusepath{fill}%
\end{pgfscope}%
\begin{pgfscope}%
\pgfpathrectangle{\pgfqpoint{1.250000in}{0.550000in}}{\pgfqpoint{7.750000in}{3.850000in}}%
\pgfusepath{clip}%
\pgfsetbuttcap%
\pgfsetmiterjoin%
\definecolor{currentfill}{rgb}{0.501961,0.501961,0.501961}%
\pgfsetfillcolor{currentfill}%
\pgfsetfillopacity{0.700000}%
\pgfsetlinewidth{0.000000pt}%
\definecolor{currentstroke}{rgb}{0.000000,0.000000,0.000000}%
\pgfsetstrokecolor{currentstroke}%
\pgfsetstrokeopacity{0.700000}%
\pgfsetdash{}{0pt}%
\pgfpathmoveto{\pgfqpoint{4.904830in}{0.550000in}}%
\pgfpathlineto{\pgfqpoint{4.932351in}{0.550000in}}%
\pgfpathlineto{\pgfqpoint{4.932351in}{0.663149in}}%
\pgfpathlineto{\pgfqpoint{4.904830in}{0.663149in}}%
\pgfpathlineto{\pgfqpoint{4.904830in}{0.550000in}}%
\pgfpathclose%
\pgfusepath{fill}%
\end{pgfscope}%
\begin{pgfscope}%
\pgfpathrectangle{\pgfqpoint{1.250000in}{0.550000in}}{\pgfqpoint{7.750000in}{3.850000in}}%
\pgfusepath{clip}%
\pgfsetbuttcap%
\pgfsetmiterjoin%
\definecolor{currentfill}{rgb}{0.501961,0.501961,0.501961}%
\pgfsetfillcolor{currentfill}%
\pgfsetfillopacity{0.700000}%
\pgfsetlinewidth{0.000000pt}%
\definecolor{currentstroke}{rgb}{0.000000,0.000000,0.000000}%
\pgfsetstrokecolor{currentstroke}%
\pgfsetstrokeopacity{0.700000}%
\pgfsetdash{}{0pt}%
\pgfpathmoveto{\pgfqpoint{4.932351in}{0.550000in}}%
\pgfpathlineto{\pgfqpoint{4.959872in}{0.550000in}}%
\pgfpathlineto{\pgfqpoint{4.959872in}{0.674622in}}%
\pgfpathlineto{\pgfqpoint{4.932351in}{0.674622in}}%
\pgfpathlineto{\pgfqpoint{4.932351in}{0.550000in}}%
\pgfpathclose%
\pgfusepath{fill}%
\end{pgfscope}%
\begin{pgfscope}%
\pgfpathrectangle{\pgfqpoint{1.250000in}{0.550000in}}{\pgfqpoint{7.750000in}{3.850000in}}%
\pgfusepath{clip}%
\pgfsetbuttcap%
\pgfsetmiterjoin%
\definecolor{currentfill}{rgb}{0.501961,0.501961,0.501961}%
\pgfsetfillcolor{currentfill}%
\pgfsetfillopacity{0.700000}%
\pgfsetlinewidth{0.000000pt}%
\definecolor{currentstroke}{rgb}{0.000000,0.000000,0.000000}%
\pgfsetstrokecolor{currentstroke}%
\pgfsetstrokeopacity{0.700000}%
\pgfsetdash{}{0pt}%
\pgfpathmoveto{\pgfqpoint{4.959872in}{0.550000in}}%
\pgfpathlineto{\pgfqpoint{4.987393in}{0.550000in}}%
\pgfpathlineto{\pgfqpoint{4.987393in}{0.670270in}}%
\pgfpathlineto{\pgfqpoint{4.959872in}{0.670270in}}%
\pgfpathlineto{\pgfqpoint{4.959872in}{0.550000in}}%
\pgfpathclose%
\pgfusepath{fill}%
\end{pgfscope}%
\begin{pgfscope}%
\pgfpathrectangle{\pgfqpoint{1.250000in}{0.550000in}}{\pgfqpoint{7.750000in}{3.850000in}}%
\pgfusepath{clip}%
\pgfsetbuttcap%
\pgfsetmiterjoin%
\definecolor{currentfill}{rgb}{0.501961,0.501961,0.501961}%
\pgfsetfillcolor{currentfill}%
\pgfsetfillopacity{0.700000}%
\pgfsetlinewidth{0.000000pt}%
\definecolor{currentstroke}{rgb}{0.000000,0.000000,0.000000}%
\pgfsetstrokecolor{currentstroke}%
\pgfsetstrokeopacity{0.700000}%
\pgfsetdash{}{0pt}%
\pgfpathmoveto{\pgfqpoint{4.987393in}{0.550000in}}%
\pgfpathlineto{\pgfqpoint{5.014915in}{0.550000in}}%
\pgfpathlineto{\pgfqpoint{5.014915in}{0.688865in}}%
\pgfpathlineto{\pgfqpoint{4.987393in}{0.688865in}}%
\pgfpathlineto{\pgfqpoint{4.987393in}{0.550000in}}%
\pgfpathclose%
\pgfusepath{fill}%
\end{pgfscope}%
\begin{pgfscope}%
\pgfpathrectangle{\pgfqpoint{1.250000in}{0.550000in}}{\pgfqpoint{7.750000in}{3.850000in}}%
\pgfusepath{clip}%
\pgfsetbuttcap%
\pgfsetmiterjoin%
\definecolor{currentfill}{rgb}{0.501961,0.501961,0.501961}%
\pgfsetfillcolor{currentfill}%
\pgfsetfillopacity{0.700000}%
\pgfsetlinewidth{0.000000pt}%
\definecolor{currentstroke}{rgb}{0.000000,0.000000,0.000000}%
\pgfsetstrokecolor{currentstroke}%
\pgfsetstrokeopacity{0.700000}%
\pgfsetdash{}{0pt}%
\pgfpathmoveto{\pgfqpoint{5.014915in}{0.550000in}}%
\pgfpathlineto{\pgfqpoint{5.042436in}{0.550000in}}%
\pgfpathlineto{\pgfqpoint{5.042436in}{0.550000in}}%
\pgfpathlineto{\pgfqpoint{5.014915in}{0.550000in}}%
\pgfpathlineto{\pgfqpoint{5.014915in}{0.550000in}}%
\pgfpathclose%
\pgfusepath{fill}%
\end{pgfscope}%
\begin{pgfscope}%
\pgfpathrectangle{\pgfqpoint{1.250000in}{0.550000in}}{\pgfqpoint{7.750000in}{3.850000in}}%
\pgfusepath{clip}%
\pgfsetbuttcap%
\pgfsetmiterjoin%
\definecolor{currentfill}{rgb}{0.501961,0.501961,0.501961}%
\pgfsetfillcolor{currentfill}%
\pgfsetfillopacity{0.700000}%
\pgfsetlinewidth{0.000000pt}%
\definecolor{currentstroke}{rgb}{0.000000,0.000000,0.000000}%
\pgfsetstrokecolor{currentstroke}%
\pgfsetstrokeopacity{0.700000}%
\pgfsetdash{}{0pt}%
\pgfpathmoveto{\pgfqpoint{5.042436in}{0.550000in}}%
\pgfpathlineto{\pgfqpoint{5.069957in}{0.550000in}}%
\pgfpathlineto{\pgfqpoint{5.069957in}{0.677787in}}%
\pgfpathlineto{\pgfqpoint{5.042436in}{0.677787in}}%
\pgfpathlineto{\pgfqpoint{5.042436in}{0.550000in}}%
\pgfpathclose%
\pgfusepath{fill}%
\end{pgfscope}%
\begin{pgfscope}%
\pgfpathrectangle{\pgfqpoint{1.250000in}{0.550000in}}{\pgfqpoint{7.750000in}{3.850000in}}%
\pgfusepath{clip}%
\pgfsetbuttcap%
\pgfsetmiterjoin%
\definecolor{currentfill}{rgb}{0.501961,0.501961,0.501961}%
\pgfsetfillcolor{currentfill}%
\pgfsetfillopacity{0.700000}%
\pgfsetlinewidth{0.000000pt}%
\definecolor{currentstroke}{rgb}{0.000000,0.000000,0.000000}%
\pgfsetstrokecolor{currentstroke}%
\pgfsetstrokeopacity{0.700000}%
\pgfsetdash{}{0pt}%
\pgfpathmoveto{\pgfqpoint{5.069957in}{0.550000in}}%
\pgfpathlineto{\pgfqpoint{5.097479in}{0.550000in}}%
\pgfpathlineto{\pgfqpoint{5.097479in}{0.670270in}}%
\pgfpathlineto{\pgfqpoint{5.069957in}{0.670270in}}%
\pgfpathlineto{\pgfqpoint{5.069957in}{0.550000in}}%
\pgfpathclose%
\pgfusepath{fill}%
\end{pgfscope}%
\begin{pgfscope}%
\pgfpathrectangle{\pgfqpoint{1.250000in}{0.550000in}}{\pgfqpoint{7.750000in}{3.850000in}}%
\pgfusepath{clip}%
\pgfsetbuttcap%
\pgfsetmiterjoin%
\definecolor{currentfill}{rgb}{0.501961,0.501961,0.501961}%
\pgfsetfillcolor{currentfill}%
\pgfsetfillopacity{0.700000}%
\pgfsetlinewidth{0.000000pt}%
\definecolor{currentstroke}{rgb}{0.000000,0.000000,0.000000}%
\pgfsetstrokecolor{currentstroke}%
\pgfsetstrokeopacity{0.700000}%
\pgfsetdash{}{0pt}%
\pgfpathmoveto{\pgfqpoint{5.097479in}{0.550000in}}%
\pgfpathlineto{\pgfqpoint{5.125000in}{0.550000in}}%
\pgfpathlineto{\pgfqpoint{5.125000in}{0.658402in}}%
\pgfpathlineto{\pgfqpoint{5.097479in}{0.658402in}}%
\pgfpathlineto{\pgfqpoint{5.097479in}{0.550000in}}%
\pgfpathclose%
\pgfusepath{fill}%
\end{pgfscope}%
\begin{pgfscope}%
\pgfpathrectangle{\pgfqpoint{1.250000in}{0.550000in}}{\pgfqpoint{7.750000in}{3.850000in}}%
\pgfusepath{clip}%
\pgfsetbuttcap%
\pgfsetmiterjoin%
\definecolor{currentfill}{rgb}{0.501961,0.501961,0.501961}%
\pgfsetfillcolor{currentfill}%
\pgfsetfillopacity{0.700000}%
\pgfsetlinewidth{0.000000pt}%
\definecolor{currentstroke}{rgb}{0.000000,0.000000,0.000000}%
\pgfsetstrokecolor{currentstroke}%
\pgfsetstrokeopacity{0.700000}%
\pgfsetdash{}{0pt}%
\pgfpathmoveto{\pgfqpoint{5.125000in}{0.550000in}}%
\pgfpathlineto{\pgfqpoint{5.152521in}{0.550000in}}%
\pgfpathlineto{\pgfqpoint{5.152521in}{0.647324in}}%
\pgfpathlineto{\pgfqpoint{5.125000in}{0.647324in}}%
\pgfpathlineto{\pgfqpoint{5.125000in}{0.550000in}}%
\pgfpathclose%
\pgfusepath{fill}%
\end{pgfscope}%
\begin{pgfscope}%
\pgfpathrectangle{\pgfqpoint{1.250000in}{0.550000in}}{\pgfqpoint{7.750000in}{3.850000in}}%
\pgfusepath{clip}%
\pgfsetbuttcap%
\pgfsetmiterjoin%
\definecolor{currentfill}{rgb}{0.501961,0.501961,0.501961}%
\pgfsetfillcolor{currentfill}%
\pgfsetfillopacity{0.700000}%
\pgfsetlinewidth{0.000000pt}%
\definecolor{currentstroke}{rgb}{0.000000,0.000000,0.000000}%
\pgfsetstrokecolor{currentstroke}%
\pgfsetstrokeopacity{0.700000}%
\pgfsetdash{}{0pt}%
\pgfpathmoveto{\pgfqpoint{5.152521in}{0.550000in}}%
\pgfpathlineto{\pgfqpoint{5.180043in}{0.550000in}}%
\pgfpathlineto{\pgfqpoint{5.180043in}{0.550000in}}%
\pgfpathlineto{\pgfqpoint{5.152521in}{0.550000in}}%
\pgfpathlineto{\pgfqpoint{5.152521in}{0.550000in}}%
\pgfpathclose%
\pgfusepath{fill}%
\end{pgfscope}%
\begin{pgfscope}%
\pgfpathrectangle{\pgfqpoint{1.250000in}{0.550000in}}{\pgfqpoint{7.750000in}{3.850000in}}%
\pgfusepath{clip}%
\pgfsetbuttcap%
\pgfsetmiterjoin%
\definecolor{currentfill}{rgb}{0.501961,0.501961,0.501961}%
\pgfsetfillcolor{currentfill}%
\pgfsetfillopacity{0.700000}%
\pgfsetlinewidth{0.000000pt}%
\definecolor{currentstroke}{rgb}{0.000000,0.000000,0.000000}%
\pgfsetstrokecolor{currentstroke}%
\pgfsetstrokeopacity{0.700000}%
\pgfsetdash{}{0pt}%
\pgfpathmoveto{\pgfqpoint{5.180043in}{0.550000in}}%
\pgfpathlineto{\pgfqpoint{5.207564in}{0.550000in}}%
\pgfpathlineto{\pgfqpoint{5.207564in}{0.658006in}}%
\pgfpathlineto{\pgfqpoint{5.180043in}{0.658006in}}%
\pgfpathlineto{\pgfqpoint{5.180043in}{0.550000in}}%
\pgfpathclose%
\pgfusepath{fill}%
\end{pgfscope}%
\begin{pgfscope}%
\pgfpathrectangle{\pgfqpoint{1.250000in}{0.550000in}}{\pgfqpoint{7.750000in}{3.850000in}}%
\pgfusepath{clip}%
\pgfsetbuttcap%
\pgfsetmiterjoin%
\definecolor{currentfill}{rgb}{0.501961,0.501961,0.501961}%
\pgfsetfillcolor{currentfill}%
\pgfsetfillopacity{0.700000}%
\pgfsetlinewidth{0.000000pt}%
\definecolor{currentstroke}{rgb}{0.000000,0.000000,0.000000}%
\pgfsetstrokecolor{currentstroke}%
\pgfsetstrokeopacity{0.700000}%
\pgfsetdash{}{0pt}%
\pgfpathmoveto{\pgfqpoint{5.207564in}{0.550000in}}%
\pgfpathlineto{\pgfqpoint{5.235085in}{0.550000in}}%
\pgfpathlineto{\pgfqpoint{5.235085in}{0.681744in}}%
\pgfpathlineto{\pgfqpoint{5.207564in}{0.681744in}}%
\pgfpathlineto{\pgfqpoint{5.207564in}{0.550000in}}%
\pgfpathclose%
\pgfusepath{fill}%
\end{pgfscope}%
\begin{pgfscope}%
\pgfpathrectangle{\pgfqpoint{1.250000in}{0.550000in}}{\pgfqpoint{7.750000in}{3.850000in}}%
\pgfusepath{clip}%
\pgfsetbuttcap%
\pgfsetmiterjoin%
\definecolor{currentfill}{rgb}{0.501961,0.501961,0.501961}%
\pgfsetfillcolor{currentfill}%
\pgfsetfillopacity{0.700000}%
\pgfsetlinewidth{0.000000pt}%
\definecolor{currentstroke}{rgb}{0.000000,0.000000,0.000000}%
\pgfsetstrokecolor{currentstroke}%
\pgfsetstrokeopacity{0.700000}%
\pgfsetdash{}{0pt}%
\pgfpathmoveto{\pgfqpoint{5.235085in}{0.550000in}}%
\pgfpathlineto{\pgfqpoint{5.262607in}{0.550000in}}%
\pgfpathlineto{\pgfqpoint{5.262607in}{0.665523in}}%
\pgfpathlineto{\pgfqpoint{5.235085in}{0.665523in}}%
\pgfpathlineto{\pgfqpoint{5.235085in}{0.550000in}}%
\pgfpathclose%
\pgfusepath{fill}%
\end{pgfscope}%
\begin{pgfscope}%
\pgfpathrectangle{\pgfqpoint{1.250000in}{0.550000in}}{\pgfqpoint{7.750000in}{3.850000in}}%
\pgfusepath{clip}%
\pgfsetbuttcap%
\pgfsetmiterjoin%
\definecolor{currentfill}{rgb}{0.501961,0.501961,0.501961}%
\pgfsetfillcolor{currentfill}%
\pgfsetfillopacity{0.700000}%
\pgfsetlinewidth{0.000000pt}%
\definecolor{currentstroke}{rgb}{0.000000,0.000000,0.000000}%
\pgfsetstrokecolor{currentstroke}%
\pgfsetstrokeopacity{0.700000}%
\pgfsetdash{}{0pt}%
\pgfpathmoveto{\pgfqpoint{5.262607in}{0.550000in}}%
\pgfpathlineto{\pgfqpoint{5.290128in}{0.550000in}}%
\pgfpathlineto{\pgfqpoint{5.290128in}{0.654050in}}%
\pgfpathlineto{\pgfqpoint{5.262607in}{0.654050in}}%
\pgfpathlineto{\pgfqpoint{5.262607in}{0.550000in}}%
\pgfpathclose%
\pgfusepath{fill}%
\end{pgfscope}%
\begin{pgfscope}%
\pgfpathrectangle{\pgfqpoint{1.250000in}{0.550000in}}{\pgfqpoint{7.750000in}{3.850000in}}%
\pgfusepath{clip}%
\pgfsetbuttcap%
\pgfsetmiterjoin%
\definecolor{currentfill}{rgb}{0.501961,0.501961,0.501961}%
\pgfsetfillcolor{currentfill}%
\pgfsetfillopacity{0.700000}%
\pgfsetlinewidth{0.000000pt}%
\definecolor{currentstroke}{rgb}{0.000000,0.000000,0.000000}%
\pgfsetstrokecolor{currentstroke}%
\pgfsetstrokeopacity{0.700000}%
\pgfsetdash{}{0pt}%
\pgfpathmoveto{\pgfqpoint{5.290128in}{0.550000in}}%
\pgfpathlineto{\pgfqpoint{5.317649in}{0.550000in}}%
\pgfpathlineto{\pgfqpoint{5.317649in}{0.550000in}}%
\pgfpathlineto{\pgfqpoint{5.290128in}{0.550000in}}%
\pgfpathlineto{\pgfqpoint{5.290128in}{0.550000in}}%
\pgfpathclose%
\pgfusepath{fill}%
\end{pgfscope}%
\begin{pgfscope}%
\pgfpathrectangle{\pgfqpoint{1.250000in}{0.550000in}}{\pgfqpoint{7.750000in}{3.850000in}}%
\pgfusepath{clip}%
\pgfsetbuttcap%
\pgfsetmiterjoin%
\definecolor{currentfill}{rgb}{0.501961,0.501961,0.501961}%
\pgfsetfillcolor{currentfill}%
\pgfsetfillopacity{0.700000}%
\pgfsetlinewidth{0.000000pt}%
\definecolor{currentstroke}{rgb}{0.000000,0.000000,0.000000}%
\pgfsetstrokecolor{currentstroke}%
\pgfsetstrokeopacity{0.700000}%
\pgfsetdash{}{0pt}%
\pgfpathmoveto{\pgfqpoint{5.317649in}{0.550000in}}%
\pgfpathlineto{\pgfqpoint{5.345170in}{0.550000in}}%
\pgfpathlineto{\pgfqpoint{5.345170in}{0.648511in}}%
\pgfpathlineto{\pgfqpoint{5.317649in}{0.648511in}}%
\pgfpathlineto{\pgfqpoint{5.317649in}{0.550000in}}%
\pgfpathclose%
\pgfusepath{fill}%
\end{pgfscope}%
\begin{pgfscope}%
\pgfpathrectangle{\pgfqpoint{1.250000in}{0.550000in}}{\pgfqpoint{7.750000in}{3.850000in}}%
\pgfusepath{clip}%
\pgfsetbuttcap%
\pgfsetmiterjoin%
\definecolor{currentfill}{rgb}{0.501961,0.501961,0.501961}%
\pgfsetfillcolor{currentfill}%
\pgfsetfillopacity{0.700000}%
\pgfsetlinewidth{0.000000pt}%
\definecolor{currentstroke}{rgb}{0.000000,0.000000,0.000000}%
\pgfsetstrokecolor{currentstroke}%
\pgfsetstrokeopacity{0.700000}%
\pgfsetdash{}{0pt}%
\pgfpathmoveto{\pgfqpoint{5.345170in}{0.550000in}}%
\pgfpathlineto{\pgfqpoint{5.372692in}{0.550000in}}%
\pgfpathlineto{\pgfqpoint{5.372692in}{0.645346in}}%
\pgfpathlineto{\pgfqpoint{5.345170in}{0.645346in}}%
\pgfpathlineto{\pgfqpoint{5.345170in}{0.550000in}}%
\pgfpathclose%
\pgfusepath{fill}%
\end{pgfscope}%
\begin{pgfscope}%
\pgfpathrectangle{\pgfqpoint{1.250000in}{0.550000in}}{\pgfqpoint{7.750000in}{3.850000in}}%
\pgfusepath{clip}%
\pgfsetbuttcap%
\pgfsetmiterjoin%
\definecolor{currentfill}{rgb}{0.501961,0.501961,0.501961}%
\pgfsetfillcolor{currentfill}%
\pgfsetfillopacity{0.700000}%
\pgfsetlinewidth{0.000000pt}%
\definecolor{currentstroke}{rgb}{0.000000,0.000000,0.000000}%
\pgfsetstrokecolor{currentstroke}%
\pgfsetstrokeopacity{0.700000}%
\pgfsetdash{}{0pt}%
\pgfpathmoveto{\pgfqpoint{5.372692in}{0.550000in}}%
\pgfpathlineto{\pgfqpoint{5.400213in}{0.550000in}}%
\pgfpathlineto{\pgfqpoint{5.400213in}{0.639807in}}%
\pgfpathlineto{\pgfqpoint{5.372692in}{0.639807in}}%
\pgfpathlineto{\pgfqpoint{5.372692in}{0.550000in}}%
\pgfpathclose%
\pgfusepath{fill}%
\end{pgfscope}%
\begin{pgfscope}%
\pgfpathrectangle{\pgfqpoint{1.250000in}{0.550000in}}{\pgfqpoint{7.750000in}{3.850000in}}%
\pgfusepath{clip}%
\pgfsetbuttcap%
\pgfsetmiterjoin%
\definecolor{currentfill}{rgb}{0.501961,0.501961,0.501961}%
\pgfsetfillcolor{currentfill}%
\pgfsetfillopacity{0.700000}%
\pgfsetlinewidth{0.000000pt}%
\definecolor{currentstroke}{rgb}{0.000000,0.000000,0.000000}%
\pgfsetstrokecolor{currentstroke}%
\pgfsetstrokeopacity{0.700000}%
\pgfsetdash{}{0pt}%
\pgfpathmoveto{\pgfqpoint{5.400213in}{0.550000in}}%
\pgfpathlineto{\pgfqpoint{5.427734in}{0.550000in}}%
\pgfpathlineto{\pgfqpoint{5.427734in}{0.639016in}}%
\pgfpathlineto{\pgfqpoint{5.400213in}{0.639016in}}%
\pgfpathlineto{\pgfqpoint{5.400213in}{0.550000in}}%
\pgfpathclose%
\pgfusepath{fill}%
\end{pgfscope}%
\begin{pgfscope}%
\pgfpathrectangle{\pgfqpoint{1.250000in}{0.550000in}}{\pgfqpoint{7.750000in}{3.850000in}}%
\pgfusepath{clip}%
\pgfsetbuttcap%
\pgfsetmiterjoin%
\definecolor{currentfill}{rgb}{0.501961,0.501961,0.501961}%
\pgfsetfillcolor{currentfill}%
\pgfsetfillopacity{0.700000}%
\pgfsetlinewidth{0.000000pt}%
\definecolor{currentstroke}{rgb}{0.000000,0.000000,0.000000}%
\pgfsetstrokecolor{currentstroke}%
\pgfsetstrokeopacity{0.700000}%
\pgfsetdash{}{0pt}%
\pgfpathmoveto{\pgfqpoint{5.427734in}{0.550000in}}%
\pgfpathlineto{\pgfqpoint{5.455256in}{0.550000in}}%
\pgfpathlineto{\pgfqpoint{5.455256in}{0.550000in}}%
\pgfpathlineto{\pgfqpoint{5.427734in}{0.550000in}}%
\pgfpathlineto{\pgfqpoint{5.427734in}{0.550000in}}%
\pgfpathclose%
\pgfusepath{fill}%
\end{pgfscope}%
\begin{pgfscope}%
\pgfpathrectangle{\pgfqpoint{1.250000in}{0.550000in}}{\pgfqpoint{7.750000in}{3.850000in}}%
\pgfusepath{clip}%
\pgfsetbuttcap%
\pgfsetmiterjoin%
\definecolor{currentfill}{rgb}{0.501961,0.501961,0.501961}%
\pgfsetfillcolor{currentfill}%
\pgfsetfillopacity{0.700000}%
\pgfsetlinewidth{0.000000pt}%
\definecolor{currentstroke}{rgb}{0.000000,0.000000,0.000000}%
\pgfsetstrokecolor{currentstroke}%
\pgfsetstrokeopacity{0.700000}%
\pgfsetdash{}{0pt}%
\pgfpathmoveto{\pgfqpoint{5.455256in}{0.550000in}}%
\pgfpathlineto{\pgfqpoint{5.482777in}{0.550000in}}%
\pgfpathlineto{\pgfqpoint{5.482777in}{0.640598in}}%
\pgfpathlineto{\pgfqpoint{5.455256in}{0.640598in}}%
\pgfpathlineto{\pgfqpoint{5.455256in}{0.550000in}}%
\pgfpathclose%
\pgfusepath{fill}%
\end{pgfscope}%
\begin{pgfscope}%
\pgfpathrectangle{\pgfqpoint{1.250000in}{0.550000in}}{\pgfqpoint{7.750000in}{3.850000in}}%
\pgfusepath{clip}%
\pgfsetbuttcap%
\pgfsetmiterjoin%
\definecolor{currentfill}{rgb}{0.501961,0.501961,0.501961}%
\pgfsetfillcolor{currentfill}%
\pgfsetfillopacity{0.700000}%
\pgfsetlinewidth{0.000000pt}%
\definecolor{currentstroke}{rgb}{0.000000,0.000000,0.000000}%
\pgfsetstrokecolor{currentstroke}%
\pgfsetstrokeopacity{0.700000}%
\pgfsetdash{}{0pt}%
\pgfpathmoveto{\pgfqpoint{5.482777in}{0.550000in}}%
\pgfpathlineto{\pgfqpoint{5.510298in}{0.550000in}}%
\pgfpathlineto{\pgfqpoint{5.510298in}{0.631499in}}%
\pgfpathlineto{\pgfqpoint{5.482777in}{0.631499in}}%
\pgfpathlineto{\pgfqpoint{5.482777in}{0.550000in}}%
\pgfpathclose%
\pgfusepath{fill}%
\end{pgfscope}%
\begin{pgfscope}%
\pgfpathrectangle{\pgfqpoint{1.250000in}{0.550000in}}{\pgfqpoint{7.750000in}{3.850000in}}%
\pgfusepath{clip}%
\pgfsetbuttcap%
\pgfsetmiterjoin%
\definecolor{currentfill}{rgb}{0.501961,0.501961,0.501961}%
\pgfsetfillcolor{currentfill}%
\pgfsetfillopacity{0.700000}%
\pgfsetlinewidth{0.000000pt}%
\definecolor{currentstroke}{rgb}{0.000000,0.000000,0.000000}%
\pgfsetstrokecolor{currentstroke}%
\pgfsetstrokeopacity{0.700000}%
\pgfsetdash{}{0pt}%
\pgfpathmoveto{\pgfqpoint{5.510298in}{0.550000in}}%
\pgfpathlineto{\pgfqpoint{5.537820in}{0.550000in}}%
\pgfpathlineto{\pgfqpoint{5.537820in}{0.647720in}}%
\pgfpathlineto{\pgfqpoint{5.510298in}{0.647720in}}%
\pgfpathlineto{\pgfqpoint{5.510298in}{0.550000in}}%
\pgfpathclose%
\pgfusepath{fill}%
\end{pgfscope}%
\begin{pgfscope}%
\pgfpathrectangle{\pgfqpoint{1.250000in}{0.550000in}}{\pgfqpoint{7.750000in}{3.850000in}}%
\pgfusepath{clip}%
\pgfsetbuttcap%
\pgfsetmiterjoin%
\definecolor{currentfill}{rgb}{0.501961,0.501961,0.501961}%
\pgfsetfillcolor{currentfill}%
\pgfsetfillopacity{0.700000}%
\pgfsetlinewidth{0.000000pt}%
\definecolor{currentstroke}{rgb}{0.000000,0.000000,0.000000}%
\pgfsetstrokecolor{currentstroke}%
\pgfsetstrokeopacity{0.700000}%
\pgfsetdash{}{0pt}%
\pgfpathmoveto{\pgfqpoint{5.537820in}{0.550000in}}%
\pgfpathlineto{\pgfqpoint{5.565341in}{0.550000in}}%
\pgfpathlineto{\pgfqpoint{5.565341in}{0.635851in}}%
\pgfpathlineto{\pgfqpoint{5.537820in}{0.635851in}}%
\pgfpathlineto{\pgfqpoint{5.537820in}{0.550000in}}%
\pgfpathclose%
\pgfusepath{fill}%
\end{pgfscope}%
\begin{pgfscope}%
\pgfpathrectangle{\pgfqpoint{1.250000in}{0.550000in}}{\pgfqpoint{7.750000in}{3.850000in}}%
\pgfusepath{clip}%
\pgfsetbuttcap%
\pgfsetmiterjoin%
\definecolor{currentfill}{rgb}{0.501961,0.501961,0.501961}%
\pgfsetfillcolor{currentfill}%
\pgfsetfillopacity{0.700000}%
\pgfsetlinewidth{0.000000pt}%
\definecolor{currentstroke}{rgb}{0.000000,0.000000,0.000000}%
\pgfsetstrokecolor{currentstroke}%
\pgfsetstrokeopacity{0.700000}%
\pgfsetdash{}{0pt}%
\pgfpathmoveto{\pgfqpoint{5.565341in}{0.550000in}}%
\pgfpathlineto{\pgfqpoint{5.592862in}{0.550000in}}%
\pgfpathlineto{\pgfqpoint{5.592862in}{0.550000in}}%
\pgfpathlineto{\pgfqpoint{5.565341in}{0.550000in}}%
\pgfpathlineto{\pgfqpoint{5.565341in}{0.550000in}}%
\pgfpathclose%
\pgfusepath{fill}%
\end{pgfscope}%
\begin{pgfscope}%
\pgfpathrectangle{\pgfqpoint{1.250000in}{0.550000in}}{\pgfqpoint{7.750000in}{3.850000in}}%
\pgfusepath{clip}%
\pgfsetbuttcap%
\pgfsetmiterjoin%
\definecolor{currentfill}{rgb}{0.501961,0.501961,0.501961}%
\pgfsetfillcolor{currentfill}%
\pgfsetfillopacity{0.700000}%
\pgfsetlinewidth{0.000000pt}%
\definecolor{currentstroke}{rgb}{0.000000,0.000000,0.000000}%
\pgfsetstrokecolor{currentstroke}%
\pgfsetstrokeopacity{0.700000}%
\pgfsetdash{}{0pt}%
\pgfpathmoveto{\pgfqpoint{5.592862in}{0.550000in}}%
\pgfpathlineto{\pgfqpoint{5.620384in}{0.550000in}}%
\pgfpathlineto{\pgfqpoint{5.620384in}{0.633082in}}%
\pgfpathlineto{\pgfqpoint{5.592862in}{0.633082in}}%
\pgfpathlineto{\pgfqpoint{5.592862in}{0.550000in}}%
\pgfpathclose%
\pgfusepath{fill}%
\end{pgfscope}%
\begin{pgfscope}%
\pgfpathrectangle{\pgfqpoint{1.250000in}{0.550000in}}{\pgfqpoint{7.750000in}{3.850000in}}%
\pgfusepath{clip}%
\pgfsetbuttcap%
\pgfsetmiterjoin%
\definecolor{currentfill}{rgb}{0.501961,0.501961,0.501961}%
\pgfsetfillcolor{currentfill}%
\pgfsetfillopacity{0.700000}%
\pgfsetlinewidth{0.000000pt}%
\definecolor{currentstroke}{rgb}{0.000000,0.000000,0.000000}%
\pgfsetstrokecolor{currentstroke}%
\pgfsetstrokeopacity{0.700000}%
\pgfsetdash{}{0pt}%
\pgfpathmoveto{\pgfqpoint{5.620384in}{0.550000in}}%
\pgfpathlineto{\pgfqpoint{5.647905in}{0.550000in}}%
\pgfpathlineto{\pgfqpoint{5.647905in}{0.631103in}}%
\pgfpathlineto{\pgfqpoint{5.620384in}{0.631103in}}%
\pgfpathlineto{\pgfqpoint{5.620384in}{0.550000in}}%
\pgfpathclose%
\pgfusepath{fill}%
\end{pgfscope}%
\begin{pgfscope}%
\pgfpathrectangle{\pgfqpoint{1.250000in}{0.550000in}}{\pgfqpoint{7.750000in}{3.850000in}}%
\pgfusepath{clip}%
\pgfsetbuttcap%
\pgfsetmiterjoin%
\definecolor{currentfill}{rgb}{0.501961,0.501961,0.501961}%
\pgfsetfillcolor{currentfill}%
\pgfsetfillopacity{0.700000}%
\pgfsetlinewidth{0.000000pt}%
\definecolor{currentstroke}{rgb}{0.000000,0.000000,0.000000}%
\pgfsetstrokecolor{currentstroke}%
\pgfsetstrokeopacity{0.700000}%
\pgfsetdash{}{0pt}%
\pgfpathmoveto{\pgfqpoint{5.647905in}{0.550000in}}%
\pgfpathlineto{\pgfqpoint{5.675426in}{0.550000in}}%
\pgfpathlineto{\pgfqpoint{5.675426in}{0.626752in}}%
\pgfpathlineto{\pgfqpoint{5.647905in}{0.626752in}}%
\pgfpathlineto{\pgfqpoint{5.647905in}{0.550000in}}%
\pgfpathclose%
\pgfusepath{fill}%
\end{pgfscope}%
\begin{pgfscope}%
\pgfpathrectangle{\pgfqpoint{1.250000in}{0.550000in}}{\pgfqpoint{7.750000in}{3.850000in}}%
\pgfusepath{clip}%
\pgfsetbuttcap%
\pgfsetmiterjoin%
\definecolor{currentfill}{rgb}{0.501961,0.501961,0.501961}%
\pgfsetfillcolor{currentfill}%
\pgfsetfillopacity{0.700000}%
\pgfsetlinewidth{0.000000pt}%
\definecolor{currentstroke}{rgb}{0.000000,0.000000,0.000000}%
\pgfsetstrokecolor{currentstroke}%
\pgfsetstrokeopacity{0.700000}%
\pgfsetdash{}{0pt}%
\pgfpathmoveto{\pgfqpoint{5.675426in}{0.550000in}}%
\pgfpathlineto{\pgfqpoint{5.702947in}{0.550000in}}%
\pgfpathlineto{\pgfqpoint{5.702947in}{0.621608in}}%
\pgfpathlineto{\pgfqpoint{5.675426in}{0.621608in}}%
\pgfpathlineto{\pgfqpoint{5.675426in}{0.550000in}}%
\pgfpathclose%
\pgfusepath{fill}%
\end{pgfscope}%
\begin{pgfscope}%
\pgfpathrectangle{\pgfqpoint{1.250000in}{0.550000in}}{\pgfqpoint{7.750000in}{3.850000in}}%
\pgfusepath{clip}%
\pgfsetbuttcap%
\pgfsetmiterjoin%
\definecolor{currentfill}{rgb}{0.501961,0.501961,0.501961}%
\pgfsetfillcolor{currentfill}%
\pgfsetfillopacity{0.700000}%
\pgfsetlinewidth{0.000000pt}%
\definecolor{currentstroke}{rgb}{0.000000,0.000000,0.000000}%
\pgfsetstrokecolor{currentstroke}%
\pgfsetstrokeopacity{0.700000}%
\pgfsetdash{}{0pt}%
\pgfpathmoveto{\pgfqpoint{5.702947in}{0.550000in}}%
\pgfpathlineto{\pgfqpoint{5.730469in}{0.550000in}}%
\pgfpathlineto{\pgfqpoint{5.730469in}{0.550000in}}%
\pgfpathlineto{\pgfqpoint{5.702947in}{0.550000in}}%
\pgfpathlineto{\pgfqpoint{5.702947in}{0.550000in}}%
\pgfpathclose%
\pgfusepath{fill}%
\end{pgfscope}%
\begin{pgfscope}%
\pgfpathrectangle{\pgfqpoint{1.250000in}{0.550000in}}{\pgfqpoint{7.750000in}{3.850000in}}%
\pgfusepath{clip}%
\pgfsetbuttcap%
\pgfsetmiterjoin%
\definecolor{currentfill}{rgb}{0.501961,0.501961,0.501961}%
\pgfsetfillcolor{currentfill}%
\pgfsetfillopacity{0.700000}%
\pgfsetlinewidth{0.000000pt}%
\definecolor{currentstroke}{rgb}{0.000000,0.000000,0.000000}%
\pgfsetstrokecolor{currentstroke}%
\pgfsetstrokeopacity{0.700000}%
\pgfsetdash{}{0pt}%
\pgfpathmoveto{\pgfqpoint{5.730469in}{0.550000in}}%
\pgfpathlineto{\pgfqpoint{5.757990in}{0.550000in}}%
\pgfpathlineto{\pgfqpoint{5.757990in}{0.627543in}}%
\pgfpathlineto{\pgfqpoint{5.730469in}{0.627543in}}%
\pgfpathlineto{\pgfqpoint{5.730469in}{0.550000in}}%
\pgfpathclose%
\pgfusepath{fill}%
\end{pgfscope}%
\begin{pgfscope}%
\pgfpathrectangle{\pgfqpoint{1.250000in}{0.550000in}}{\pgfqpoint{7.750000in}{3.850000in}}%
\pgfusepath{clip}%
\pgfsetbuttcap%
\pgfsetmiterjoin%
\definecolor{currentfill}{rgb}{0.501961,0.501961,0.501961}%
\pgfsetfillcolor{currentfill}%
\pgfsetfillopacity{0.700000}%
\pgfsetlinewidth{0.000000pt}%
\definecolor{currentstroke}{rgb}{0.000000,0.000000,0.000000}%
\pgfsetstrokecolor{currentstroke}%
\pgfsetstrokeopacity{0.700000}%
\pgfsetdash{}{0pt}%
\pgfpathmoveto{\pgfqpoint{5.757990in}{0.550000in}}%
\pgfpathlineto{\pgfqpoint{5.785511in}{0.550000in}}%
\pgfpathlineto{\pgfqpoint{5.785511in}{0.619235in}}%
\pgfpathlineto{\pgfqpoint{5.757990in}{0.619235in}}%
\pgfpathlineto{\pgfqpoint{5.757990in}{0.550000in}}%
\pgfpathclose%
\pgfusepath{fill}%
\end{pgfscope}%
\begin{pgfscope}%
\pgfpathrectangle{\pgfqpoint{1.250000in}{0.550000in}}{\pgfqpoint{7.750000in}{3.850000in}}%
\pgfusepath{clip}%
\pgfsetbuttcap%
\pgfsetmiterjoin%
\definecolor{currentfill}{rgb}{0.501961,0.501961,0.501961}%
\pgfsetfillcolor{currentfill}%
\pgfsetfillopacity{0.700000}%
\pgfsetlinewidth{0.000000pt}%
\definecolor{currentstroke}{rgb}{0.000000,0.000000,0.000000}%
\pgfsetstrokecolor{currentstroke}%
\pgfsetstrokeopacity{0.700000}%
\pgfsetdash{}{0pt}%
\pgfpathmoveto{\pgfqpoint{5.785511in}{0.550000in}}%
\pgfpathlineto{\pgfqpoint{5.813033in}{0.550000in}}%
\pgfpathlineto{\pgfqpoint{5.813033in}{0.620422in}}%
\pgfpathlineto{\pgfqpoint{5.785511in}{0.620422in}}%
\pgfpathlineto{\pgfqpoint{5.785511in}{0.550000in}}%
\pgfpathclose%
\pgfusepath{fill}%
\end{pgfscope}%
\begin{pgfscope}%
\pgfpathrectangle{\pgfqpoint{1.250000in}{0.550000in}}{\pgfqpoint{7.750000in}{3.850000in}}%
\pgfusepath{clip}%
\pgfsetbuttcap%
\pgfsetmiterjoin%
\definecolor{currentfill}{rgb}{0.501961,0.501961,0.501961}%
\pgfsetfillcolor{currentfill}%
\pgfsetfillopacity{0.700000}%
\pgfsetlinewidth{0.000000pt}%
\definecolor{currentstroke}{rgb}{0.000000,0.000000,0.000000}%
\pgfsetstrokecolor{currentstroke}%
\pgfsetstrokeopacity{0.700000}%
\pgfsetdash{}{0pt}%
\pgfpathmoveto{\pgfqpoint{5.813033in}{0.550000in}}%
\pgfpathlineto{\pgfqpoint{5.840554in}{0.550000in}}%
\pgfpathlineto{\pgfqpoint{5.840554in}{0.550000in}}%
\pgfpathlineto{\pgfqpoint{5.813033in}{0.550000in}}%
\pgfpathlineto{\pgfqpoint{5.813033in}{0.550000in}}%
\pgfpathclose%
\pgfusepath{fill}%
\end{pgfscope}%
\begin{pgfscope}%
\pgfpathrectangle{\pgfqpoint{1.250000in}{0.550000in}}{\pgfqpoint{7.750000in}{3.850000in}}%
\pgfusepath{clip}%
\pgfsetbuttcap%
\pgfsetmiterjoin%
\definecolor{currentfill}{rgb}{0.501961,0.501961,0.501961}%
\pgfsetfillcolor{currentfill}%
\pgfsetfillopacity{0.700000}%
\pgfsetlinewidth{0.000000pt}%
\definecolor{currentstroke}{rgb}{0.000000,0.000000,0.000000}%
\pgfsetstrokecolor{currentstroke}%
\pgfsetstrokeopacity{0.700000}%
\pgfsetdash{}{0pt}%
\pgfpathmoveto{\pgfqpoint{5.840554in}{0.550000in}}%
\pgfpathlineto{\pgfqpoint{5.868075in}{0.550000in}}%
\pgfpathlineto{\pgfqpoint{5.868075in}{0.610135in}}%
\pgfpathlineto{\pgfqpoint{5.840554in}{0.610135in}}%
\pgfpathlineto{\pgfqpoint{5.840554in}{0.550000in}}%
\pgfpathclose%
\pgfusepath{fill}%
\end{pgfscope}%
\begin{pgfscope}%
\pgfpathrectangle{\pgfqpoint{1.250000in}{0.550000in}}{\pgfqpoint{7.750000in}{3.850000in}}%
\pgfusepath{clip}%
\pgfsetbuttcap%
\pgfsetmiterjoin%
\definecolor{currentfill}{rgb}{0.501961,0.501961,0.501961}%
\pgfsetfillcolor{currentfill}%
\pgfsetfillopacity{0.700000}%
\pgfsetlinewidth{0.000000pt}%
\definecolor{currentstroke}{rgb}{0.000000,0.000000,0.000000}%
\pgfsetstrokecolor{currentstroke}%
\pgfsetstrokeopacity{0.700000}%
\pgfsetdash{}{0pt}%
\pgfpathmoveto{\pgfqpoint{5.868075in}{0.550000in}}%
\pgfpathlineto{\pgfqpoint{5.895597in}{0.550000in}}%
\pgfpathlineto{\pgfqpoint{5.895597in}{0.635060in}}%
\pgfpathlineto{\pgfqpoint{5.868075in}{0.635060in}}%
\pgfpathlineto{\pgfqpoint{5.868075in}{0.550000in}}%
\pgfpathclose%
\pgfusepath{fill}%
\end{pgfscope}%
\begin{pgfscope}%
\pgfpathrectangle{\pgfqpoint{1.250000in}{0.550000in}}{\pgfqpoint{7.750000in}{3.850000in}}%
\pgfusepath{clip}%
\pgfsetbuttcap%
\pgfsetmiterjoin%
\definecolor{currentfill}{rgb}{0.501961,0.501961,0.501961}%
\pgfsetfillcolor{currentfill}%
\pgfsetfillopacity{0.700000}%
\pgfsetlinewidth{0.000000pt}%
\definecolor{currentstroke}{rgb}{0.000000,0.000000,0.000000}%
\pgfsetstrokecolor{currentstroke}%
\pgfsetstrokeopacity{0.700000}%
\pgfsetdash{}{0pt}%
\pgfpathmoveto{\pgfqpoint{5.895597in}{0.550000in}}%
\pgfpathlineto{\pgfqpoint{5.923118in}{0.550000in}}%
\pgfpathlineto{\pgfqpoint{5.923118in}{0.621213in}}%
\pgfpathlineto{\pgfqpoint{5.895597in}{0.621213in}}%
\pgfpathlineto{\pgfqpoint{5.895597in}{0.550000in}}%
\pgfpathclose%
\pgfusepath{fill}%
\end{pgfscope}%
\begin{pgfscope}%
\pgfpathrectangle{\pgfqpoint{1.250000in}{0.550000in}}{\pgfqpoint{7.750000in}{3.850000in}}%
\pgfusepath{clip}%
\pgfsetbuttcap%
\pgfsetmiterjoin%
\definecolor{currentfill}{rgb}{0.501961,0.501961,0.501961}%
\pgfsetfillcolor{currentfill}%
\pgfsetfillopacity{0.700000}%
\pgfsetlinewidth{0.000000pt}%
\definecolor{currentstroke}{rgb}{0.000000,0.000000,0.000000}%
\pgfsetstrokecolor{currentstroke}%
\pgfsetstrokeopacity{0.700000}%
\pgfsetdash{}{0pt}%
\pgfpathmoveto{\pgfqpoint{5.923118in}{0.550000in}}%
\pgfpathlineto{\pgfqpoint{5.950639in}{0.550000in}}%
\pgfpathlineto{\pgfqpoint{5.950639in}{0.607366in}}%
\pgfpathlineto{\pgfqpoint{5.923118in}{0.607366in}}%
\pgfpathlineto{\pgfqpoint{5.923118in}{0.550000in}}%
\pgfpathclose%
\pgfusepath{fill}%
\end{pgfscope}%
\begin{pgfscope}%
\pgfpathrectangle{\pgfqpoint{1.250000in}{0.550000in}}{\pgfqpoint{7.750000in}{3.850000in}}%
\pgfusepath{clip}%
\pgfsetbuttcap%
\pgfsetmiterjoin%
\definecolor{currentfill}{rgb}{0.501961,0.501961,0.501961}%
\pgfsetfillcolor{currentfill}%
\pgfsetfillopacity{0.700000}%
\pgfsetlinewidth{0.000000pt}%
\definecolor{currentstroke}{rgb}{0.000000,0.000000,0.000000}%
\pgfsetstrokecolor{currentstroke}%
\pgfsetstrokeopacity{0.700000}%
\pgfsetdash{}{0pt}%
\pgfpathmoveto{\pgfqpoint{5.950639in}{0.550000in}}%
\pgfpathlineto{\pgfqpoint{5.978161in}{0.550000in}}%
\pgfpathlineto{\pgfqpoint{5.978161in}{0.550000in}}%
\pgfpathlineto{\pgfqpoint{5.950639in}{0.550000in}}%
\pgfpathlineto{\pgfqpoint{5.950639in}{0.550000in}}%
\pgfpathclose%
\pgfusepath{fill}%
\end{pgfscope}%
\begin{pgfscope}%
\pgfpathrectangle{\pgfqpoint{1.250000in}{0.550000in}}{\pgfqpoint{7.750000in}{3.850000in}}%
\pgfusepath{clip}%
\pgfsetbuttcap%
\pgfsetmiterjoin%
\definecolor{currentfill}{rgb}{0.501961,0.501961,0.501961}%
\pgfsetfillcolor{currentfill}%
\pgfsetfillopacity{0.700000}%
\pgfsetlinewidth{0.000000pt}%
\definecolor{currentstroke}{rgb}{0.000000,0.000000,0.000000}%
\pgfsetstrokecolor{currentstroke}%
\pgfsetstrokeopacity{0.700000}%
\pgfsetdash{}{0pt}%
\pgfpathmoveto{\pgfqpoint{5.978161in}{0.550000in}}%
\pgfpathlineto{\pgfqpoint{6.005682in}{0.550000in}}%
\pgfpathlineto{\pgfqpoint{6.005682in}{0.616070in}}%
\pgfpathlineto{\pgfqpoint{5.978161in}{0.616070in}}%
\pgfpathlineto{\pgfqpoint{5.978161in}{0.550000in}}%
\pgfpathclose%
\pgfusepath{fill}%
\end{pgfscope}%
\begin{pgfscope}%
\pgfpathrectangle{\pgfqpoint{1.250000in}{0.550000in}}{\pgfqpoint{7.750000in}{3.850000in}}%
\pgfusepath{clip}%
\pgfsetbuttcap%
\pgfsetmiterjoin%
\definecolor{currentfill}{rgb}{0.501961,0.501961,0.501961}%
\pgfsetfillcolor{currentfill}%
\pgfsetfillopacity{0.700000}%
\pgfsetlinewidth{0.000000pt}%
\definecolor{currentstroke}{rgb}{0.000000,0.000000,0.000000}%
\pgfsetstrokecolor{currentstroke}%
\pgfsetstrokeopacity{0.700000}%
\pgfsetdash{}{0pt}%
\pgfpathmoveto{\pgfqpoint{6.005682in}{0.550000in}}%
\pgfpathlineto{\pgfqpoint{6.033203in}{0.550000in}}%
\pgfpathlineto{\pgfqpoint{6.033203in}{0.615674in}}%
\pgfpathlineto{\pgfqpoint{6.005682in}{0.615674in}}%
\pgfpathlineto{\pgfqpoint{6.005682in}{0.550000in}}%
\pgfpathclose%
\pgfusepath{fill}%
\end{pgfscope}%
\begin{pgfscope}%
\pgfpathrectangle{\pgfqpoint{1.250000in}{0.550000in}}{\pgfqpoint{7.750000in}{3.850000in}}%
\pgfusepath{clip}%
\pgfsetbuttcap%
\pgfsetmiterjoin%
\definecolor{currentfill}{rgb}{0.501961,0.501961,0.501961}%
\pgfsetfillcolor{currentfill}%
\pgfsetfillopacity{0.700000}%
\pgfsetlinewidth{0.000000pt}%
\definecolor{currentstroke}{rgb}{0.000000,0.000000,0.000000}%
\pgfsetstrokecolor{currentstroke}%
\pgfsetstrokeopacity{0.700000}%
\pgfsetdash{}{0pt}%
\pgfpathmoveto{\pgfqpoint{6.033203in}{0.550000in}}%
\pgfpathlineto{\pgfqpoint{6.060724in}{0.550000in}}%
\pgfpathlineto{\pgfqpoint{6.060724in}{0.606179in}}%
\pgfpathlineto{\pgfqpoint{6.033203in}{0.606179in}}%
\pgfpathlineto{\pgfqpoint{6.033203in}{0.550000in}}%
\pgfpathclose%
\pgfusepath{fill}%
\end{pgfscope}%
\begin{pgfscope}%
\pgfpathrectangle{\pgfqpoint{1.250000in}{0.550000in}}{\pgfqpoint{7.750000in}{3.850000in}}%
\pgfusepath{clip}%
\pgfsetbuttcap%
\pgfsetmiterjoin%
\definecolor{currentfill}{rgb}{0.501961,0.501961,0.501961}%
\pgfsetfillcolor{currentfill}%
\pgfsetfillopacity{0.700000}%
\pgfsetlinewidth{0.000000pt}%
\definecolor{currentstroke}{rgb}{0.000000,0.000000,0.000000}%
\pgfsetstrokecolor{currentstroke}%
\pgfsetstrokeopacity{0.700000}%
\pgfsetdash{}{0pt}%
\pgfpathmoveto{\pgfqpoint{6.060724in}{0.550000in}}%
\pgfpathlineto{\pgfqpoint{6.088246in}{0.550000in}}%
\pgfpathlineto{\pgfqpoint{6.088246in}{0.602618in}}%
\pgfpathlineto{\pgfqpoint{6.060724in}{0.602618in}}%
\pgfpathlineto{\pgfqpoint{6.060724in}{0.550000in}}%
\pgfpathclose%
\pgfusepath{fill}%
\end{pgfscope}%
\begin{pgfscope}%
\pgfpathrectangle{\pgfqpoint{1.250000in}{0.550000in}}{\pgfqpoint{7.750000in}{3.850000in}}%
\pgfusepath{clip}%
\pgfsetbuttcap%
\pgfsetmiterjoin%
\definecolor{currentfill}{rgb}{0.501961,0.501961,0.501961}%
\pgfsetfillcolor{currentfill}%
\pgfsetfillopacity{0.700000}%
\pgfsetlinewidth{0.000000pt}%
\definecolor{currentstroke}{rgb}{0.000000,0.000000,0.000000}%
\pgfsetstrokecolor{currentstroke}%
\pgfsetstrokeopacity{0.700000}%
\pgfsetdash{}{0pt}%
\pgfpathmoveto{\pgfqpoint{6.088246in}{0.550000in}}%
\pgfpathlineto{\pgfqpoint{6.115767in}{0.550000in}}%
\pgfpathlineto{\pgfqpoint{6.115767in}{0.550000in}}%
\pgfpathlineto{\pgfqpoint{6.088246in}{0.550000in}}%
\pgfpathlineto{\pgfqpoint{6.088246in}{0.550000in}}%
\pgfpathclose%
\pgfusepath{fill}%
\end{pgfscope}%
\begin{pgfscope}%
\pgfpathrectangle{\pgfqpoint{1.250000in}{0.550000in}}{\pgfqpoint{7.750000in}{3.850000in}}%
\pgfusepath{clip}%
\pgfsetbuttcap%
\pgfsetmiterjoin%
\definecolor{currentfill}{rgb}{0.501961,0.501961,0.501961}%
\pgfsetfillcolor{currentfill}%
\pgfsetfillopacity{0.700000}%
\pgfsetlinewidth{0.000000pt}%
\definecolor{currentstroke}{rgb}{0.000000,0.000000,0.000000}%
\pgfsetstrokecolor{currentstroke}%
\pgfsetstrokeopacity{0.700000}%
\pgfsetdash{}{0pt}%
\pgfpathmoveto{\pgfqpoint{6.115767in}{0.550000in}}%
\pgfpathlineto{\pgfqpoint{6.143288in}{0.550000in}}%
\pgfpathlineto{\pgfqpoint{6.143288in}{0.606970in}}%
\pgfpathlineto{\pgfqpoint{6.115767in}{0.606970in}}%
\pgfpathlineto{\pgfqpoint{6.115767in}{0.550000in}}%
\pgfpathclose%
\pgfusepath{fill}%
\end{pgfscope}%
\begin{pgfscope}%
\pgfpathrectangle{\pgfqpoint{1.250000in}{0.550000in}}{\pgfqpoint{7.750000in}{3.850000in}}%
\pgfusepath{clip}%
\pgfsetbuttcap%
\pgfsetmiterjoin%
\definecolor{currentfill}{rgb}{0.501961,0.501961,0.501961}%
\pgfsetfillcolor{currentfill}%
\pgfsetfillopacity{0.700000}%
\pgfsetlinewidth{0.000000pt}%
\definecolor{currentstroke}{rgb}{0.000000,0.000000,0.000000}%
\pgfsetstrokecolor{currentstroke}%
\pgfsetstrokeopacity{0.700000}%
\pgfsetdash{}{0pt}%
\pgfpathmoveto{\pgfqpoint{6.143288in}{0.550000in}}%
\pgfpathlineto{\pgfqpoint{6.170810in}{0.550000in}}%
\pgfpathlineto{\pgfqpoint{6.170810in}{0.617652in}}%
\pgfpathlineto{\pgfqpoint{6.143288in}{0.617652in}}%
\pgfpathlineto{\pgfqpoint{6.143288in}{0.550000in}}%
\pgfpathclose%
\pgfusepath{fill}%
\end{pgfscope}%
\begin{pgfscope}%
\pgfpathrectangle{\pgfqpoint{1.250000in}{0.550000in}}{\pgfqpoint{7.750000in}{3.850000in}}%
\pgfusepath{clip}%
\pgfsetbuttcap%
\pgfsetmiterjoin%
\definecolor{currentfill}{rgb}{0.501961,0.501961,0.501961}%
\pgfsetfillcolor{currentfill}%
\pgfsetfillopacity{0.700000}%
\pgfsetlinewidth{0.000000pt}%
\definecolor{currentstroke}{rgb}{0.000000,0.000000,0.000000}%
\pgfsetstrokecolor{currentstroke}%
\pgfsetstrokeopacity{0.700000}%
\pgfsetdash{}{0pt}%
\pgfpathmoveto{\pgfqpoint{6.170810in}{0.550000in}}%
\pgfpathlineto{\pgfqpoint{6.198331in}{0.550000in}}%
\pgfpathlineto{\pgfqpoint{6.198331in}{0.610135in}}%
\pgfpathlineto{\pgfqpoint{6.170810in}{0.610135in}}%
\pgfpathlineto{\pgfqpoint{6.170810in}{0.550000in}}%
\pgfpathclose%
\pgfusepath{fill}%
\end{pgfscope}%
\begin{pgfscope}%
\pgfpathrectangle{\pgfqpoint{1.250000in}{0.550000in}}{\pgfqpoint{7.750000in}{3.850000in}}%
\pgfusepath{clip}%
\pgfsetbuttcap%
\pgfsetmiterjoin%
\definecolor{currentfill}{rgb}{0.501961,0.501961,0.501961}%
\pgfsetfillcolor{currentfill}%
\pgfsetfillopacity{0.700000}%
\pgfsetlinewidth{0.000000pt}%
\definecolor{currentstroke}{rgb}{0.000000,0.000000,0.000000}%
\pgfsetstrokecolor{currentstroke}%
\pgfsetstrokeopacity{0.700000}%
\pgfsetdash{}{0pt}%
\pgfpathmoveto{\pgfqpoint{6.198331in}{0.550000in}}%
\pgfpathlineto{\pgfqpoint{6.225852in}{0.550000in}}%
\pgfpathlineto{\pgfqpoint{6.225852in}{0.617257in}}%
\pgfpathlineto{\pgfqpoint{6.198331in}{0.617257in}}%
\pgfpathlineto{\pgfqpoint{6.198331in}{0.550000in}}%
\pgfpathclose%
\pgfusepath{fill}%
\end{pgfscope}%
\begin{pgfscope}%
\pgfpathrectangle{\pgfqpoint{1.250000in}{0.550000in}}{\pgfqpoint{7.750000in}{3.850000in}}%
\pgfusepath{clip}%
\pgfsetbuttcap%
\pgfsetmiterjoin%
\definecolor{currentfill}{rgb}{0.501961,0.501961,0.501961}%
\pgfsetfillcolor{currentfill}%
\pgfsetfillopacity{0.700000}%
\pgfsetlinewidth{0.000000pt}%
\definecolor{currentstroke}{rgb}{0.000000,0.000000,0.000000}%
\pgfsetstrokecolor{currentstroke}%
\pgfsetstrokeopacity{0.700000}%
\pgfsetdash{}{0pt}%
\pgfpathmoveto{\pgfqpoint{6.225852in}{0.550000in}}%
\pgfpathlineto{\pgfqpoint{6.253374in}{0.550000in}}%
\pgfpathlineto{\pgfqpoint{6.253374in}{0.550000in}}%
\pgfpathlineto{\pgfqpoint{6.225852in}{0.550000in}}%
\pgfpathlineto{\pgfqpoint{6.225852in}{0.550000in}}%
\pgfpathclose%
\pgfusepath{fill}%
\end{pgfscope}%
\begin{pgfscope}%
\pgfpathrectangle{\pgfqpoint{1.250000in}{0.550000in}}{\pgfqpoint{7.750000in}{3.850000in}}%
\pgfusepath{clip}%
\pgfsetbuttcap%
\pgfsetmiterjoin%
\definecolor{currentfill}{rgb}{0.501961,0.501961,0.501961}%
\pgfsetfillcolor{currentfill}%
\pgfsetfillopacity{0.700000}%
\pgfsetlinewidth{0.000000pt}%
\definecolor{currentstroke}{rgb}{0.000000,0.000000,0.000000}%
\pgfsetstrokecolor{currentstroke}%
\pgfsetstrokeopacity{0.700000}%
\pgfsetdash{}{0pt}%
\pgfpathmoveto{\pgfqpoint{6.253374in}{0.550000in}}%
\pgfpathlineto{\pgfqpoint{6.280895in}{0.550000in}}%
\pgfpathlineto{\pgfqpoint{6.280895in}{0.608948in}}%
\pgfpathlineto{\pgfqpoint{6.253374in}{0.608948in}}%
\pgfpathlineto{\pgfqpoint{6.253374in}{0.550000in}}%
\pgfpathclose%
\pgfusepath{fill}%
\end{pgfscope}%
\begin{pgfscope}%
\pgfpathrectangle{\pgfqpoint{1.250000in}{0.550000in}}{\pgfqpoint{7.750000in}{3.850000in}}%
\pgfusepath{clip}%
\pgfsetbuttcap%
\pgfsetmiterjoin%
\definecolor{currentfill}{rgb}{0.501961,0.501961,0.501961}%
\pgfsetfillcolor{currentfill}%
\pgfsetfillopacity{0.700000}%
\pgfsetlinewidth{0.000000pt}%
\definecolor{currentstroke}{rgb}{0.000000,0.000000,0.000000}%
\pgfsetstrokecolor{currentstroke}%
\pgfsetstrokeopacity{0.700000}%
\pgfsetdash{}{0pt}%
\pgfpathmoveto{\pgfqpoint{6.280895in}{0.550000in}}%
\pgfpathlineto{\pgfqpoint{6.308416in}{0.550000in}}%
\pgfpathlineto{\pgfqpoint{6.308416in}{0.603410in}}%
\pgfpathlineto{\pgfqpoint{6.280895in}{0.603410in}}%
\pgfpathlineto{\pgfqpoint{6.280895in}{0.550000in}}%
\pgfpathclose%
\pgfusepath{fill}%
\end{pgfscope}%
\begin{pgfscope}%
\pgfpathrectangle{\pgfqpoint{1.250000in}{0.550000in}}{\pgfqpoint{7.750000in}{3.850000in}}%
\pgfusepath{clip}%
\pgfsetbuttcap%
\pgfsetmiterjoin%
\definecolor{currentfill}{rgb}{0.501961,0.501961,0.501961}%
\pgfsetfillcolor{currentfill}%
\pgfsetfillopacity{0.700000}%
\pgfsetlinewidth{0.000000pt}%
\definecolor{currentstroke}{rgb}{0.000000,0.000000,0.000000}%
\pgfsetstrokecolor{currentstroke}%
\pgfsetstrokeopacity{0.700000}%
\pgfsetdash{}{0pt}%
\pgfpathmoveto{\pgfqpoint{6.308416in}{0.550000in}}%
\pgfpathlineto{\pgfqpoint{6.335937in}{0.550000in}}%
\pgfpathlineto{\pgfqpoint{6.335937in}{0.597080in}}%
\pgfpathlineto{\pgfqpoint{6.308416in}{0.597080in}}%
\pgfpathlineto{\pgfqpoint{6.308416in}{0.550000in}}%
\pgfpathclose%
\pgfusepath{fill}%
\end{pgfscope}%
\begin{pgfscope}%
\pgfpathrectangle{\pgfqpoint{1.250000in}{0.550000in}}{\pgfqpoint{7.750000in}{3.850000in}}%
\pgfusepath{clip}%
\pgfsetbuttcap%
\pgfsetmiterjoin%
\definecolor{currentfill}{rgb}{0.501961,0.501961,0.501961}%
\pgfsetfillcolor{currentfill}%
\pgfsetfillopacity{0.700000}%
\pgfsetlinewidth{0.000000pt}%
\definecolor{currentstroke}{rgb}{0.000000,0.000000,0.000000}%
\pgfsetstrokecolor{currentstroke}%
\pgfsetstrokeopacity{0.700000}%
\pgfsetdash{}{0pt}%
\pgfpathmoveto{\pgfqpoint{6.335938in}{0.550000in}}%
\pgfpathlineto{\pgfqpoint{6.363459in}{0.550000in}}%
\pgfpathlineto{\pgfqpoint{6.363459in}{0.593519in}}%
\pgfpathlineto{\pgfqpoint{6.335938in}{0.593519in}}%
\pgfpathlineto{\pgfqpoint{6.335938in}{0.550000in}}%
\pgfpathclose%
\pgfusepath{fill}%
\end{pgfscope}%
\begin{pgfscope}%
\pgfpathrectangle{\pgfqpoint{1.250000in}{0.550000in}}{\pgfqpoint{7.750000in}{3.850000in}}%
\pgfusepath{clip}%
\pgfsetbuttcap%
\pgfsetmiterjoin%
\definecolor{currentfill}{rgb}{0.501961,0.501961,0.501961}%
\pgfsetfillcolor{currentfill}%
\pgfsetfillopacity{0.700000}%
\pgfsetlinewidth{0.000000pt}%
\definecolor{currentstroke}{rgb}{0.000000,0.000000,0.000000}%
\pgfsetstrokecolor{currentstroke}%
\pgfsetstrokeopacity{0.700000}%
\pgfsetdash{}{0pt}%
\pgfpathmoveto{\pgfqpoint{6.363459in}{0.550000in}}%
\pgfpathlineto{\pgfqpoint{6.390980in}{0.550000in}}%
\pgfpathlineto{\pgfqpoint{6.390980in}{0.597080in}}%
\pgfpathlineto{\pgfqpoint{6.363459in}{0.597080in}}%
\pgfpathlineto{\pgfqpoint{6.363459in}{0.550000in}}%
\pgfpathclose%
\pgfusepath{fill}%
\end{pgfscope}%
\begin{pgfscope}%
\pgfpathrectangle{\pgfqpoint{1.250000in}{0.550000in}}{\pgfqpoint{7.750000in}{3.850000in}}%
\pgfusepath{clip}%
\pgfsetbuttcap%
\pgfsetmiterjoin%
\definecolor{currentfill}{rgb}{0.501961,0.501961,0.501961}%
\pgfsetfillcolor{currentfill}%
\pgfsetfillopacity{0.700000}%
\pgfsetlinewidth{0.000000pt}%
\definecolor{currentstroke}{rgb}{0.000000,0.000000,0.000000}%
\pgfsetstrokecolor{currentstroke}%
\pgfsetstrokeopacity{0.700000}%
\pgfsetdash{}{0pt}%
\pgfpathmoveto{\pgfqpoint{6.390980in}{0.550000in}}%
\pgfpathlineto{\pgfqpoint{6.418501in}{0.550000in}}%
\pgfpathlineto{\pgfqpoint{6.418501in}{0.590750in}}%
\pgfpathlineto{\pgfqpoint{6.390980in}{0.590750in}}%
\pgfpathlineto{\pgfqpoint{6.390980in}{0.550000in}}%
\pgfpathclose%
\pgfusepath{fill}%
\end{pgfscope}%
\begin{pgfscope}%
\pgfpathrectangle{\pgfqpoint{1.250000in}{0.550000in}}{\pgfqpoint{7.750000in}{3.850000in}}%
\pgfusepath{clip}%
\pgfsetbuttcap%
\pgfsetmiterjoin%
\definecolor{currentfill}{rgb}{0.501961,0.501961,0.501961}%
\pgfsetfillcolor{currentfill}%
\pgfsetfillopacity{0.700000}%
\pgfsetlinewidth{0.000000pt}%
\definecolor{currentstroke}{rgb}{0.000000,0.000000,0.000000}%
\pgfsetstrokecolor{currentstroke}%
\pgfsetstrokeopacity{0.700000}%
\pgfsetdash{}{0pt}%
\pgfpathmoveto{\pgfqpoint{6.418501in}{0.550000in}}%
\pgfpathlineto{\pgfqpoint{6.446023in}{0.550000in}}%
\pgfpathlineto{\pgfqpoint{6.446023in}{0.550000in}}%
\pgfpathlineto{\pgfqpoint{6.418501in}{0.550000in}}%
\pgfpathlineto{\pgfqpoint{6.418501in}{0.550000in}}%
\pgfpathclose%
\pgfusepath{fill}%
\end{pgfscope}%
\begin{pgfscope}%
\pgfpathrectangle{\pgfqpoint{1.250000in}{0.550000in}}{\pgfqpoint{7.750000in}{3.850000in}}%
\pgfusepath{clip}%
\pgfsetbuttcap%
\pgfsetmiterjoin%
\definecolor{currentfill}{rgb}{0.501961,0.501961,0.501961}%
\pgfsetfillcolor{currentfill}%
\pgfsetfillopacity{0.700000}%
\pgfsetlinewidth{0.000000pt}%
\definecolor{currentstroke}{rgb}{0.000000,0.000000,0.000000}%
\pgfsetstrokecolor{currentstroke}%
\pgfsetstrokeopacity{0.700000}%
\pgfsetdash{}{0pt}%
\pgfpathmoveto{\pgfqpoint{6.446023in}{0.550000in}}%
\pgfpathlineto{\pgfqpoint{6.473544in}{0.550000in}}%
\pgfpathlineto{\pgfqpoint{6.473544in}{0.596288in}}%
\pgfpathlineto{\pgfqpoint{6.446023in}{0.596288in}}%
\pgfpathlineto{\pgfqpoint{6.446023in}{0.550000in}}%
\pgfpathclose%
\pgfusepath{fill}%
\end{pgfscope}%
\begin{pgfscope}%
\pgfpathrectangle{\pgfqpoint{1.250000in}{0.550000in}}{\pgfqpoint{7.750000in}{3.850000in}}%
\pgfusepath{clip}%
\pgfsetbuttcap%
\pgfsetmiterjoin%
\definecolor{currentfill}{rgb}{0.501961,0.501961,0.501961}%
\pgfsetfillcolor{currentfill}%
\pgfsetfillopacity{0.700000}%
\pgfsetlinewidth{0.000000pt}%
\definecolor{currentstroke}{rgb}{0.000000,0.000000,0.000000}%
\pgfsetstrokecolor{currentstroke}%
\pgfsetstrokeopacity{0.700000}%
\pgfsetdash{}{0pt}%
\pgfpathmoveto{\pgfqpoint{6.473544in}{0.550000in}}%
\pgfpathlineto{\pgfqpoint{6.501065in}{0.550000in}}%
\pgfpathlineto{\pgfqpoint{6.501065in}{0.594310in}}%
\pgfpathlineto{\pgfqpoint{6.473544in}{0.594310in}}%
\pgfpathlineto{\pgfqpoint{6.473544in}{0.550000in}}%
\pgfpathclose%
\pgfusepath{fill}%
\end{pgfscope}%
\begin{pgfscope}%
\pgfpathrectangle{\pgfqpoint{1.250000in}{0.550000in}}{\pgfqpoint{7.750000in}{3.850000in}}%
\pgfusepath{clip}%
\pgfsetbuttcap%
\pgfsetmiterjoin%
\definecolor{currentfill}{rgb}{0.501961,0.501961,0.501961}%
\pgfsetfillcolor{currentfill}%
\pgfsetfillopacity{0.700000}%
\pgfsetlinewidth{0.000000pt}%
\definecolor{currentstroke}{rgb}{0.000000,0.000000,0.000000}%
\pgfsetstrokecolor{currentstroke}%
\pgfsetstrokeopacity{0.700000}%
\pgfsetdash{}{0pt}%
\pgfpathmoveto{\pgfqpoint{6.501065in}{0.550000in}}%
\pgfpathlineto{\pgfqpoint{6.528587in}{0.550000in}}%
\pgfpathlineto{\pgfqpoint{6.528587in}{0.597080in}}%
\pgfpathlineto{\pgfqpoint{6.501065in}{0.597080in}}%
\pgfpathlineto{\pgfqpoint{6.501065in}{0.550000in}}%
\pgfpathclose%
\pgfusepath{fill}%
\end{pgfscope}%
\begin{pgfscope}%
\pgfpathrectangle{\pgfqpoint{1.250000in}{0.550000in}}{\pgfqpoint{7.750000in}{3.850000in}}%
\pgfusepath{clip}%
\pgfsetbuttcap%
\pgfsetmiterjoin%
\definecolor{currentfill}{rgb}{0.501961,0.501961,0.501961}%
\pgfsetfillcolor{currentfill}%
\pgfsetfillopacity{0.700000}%
\pgfsetlinewidth{0.000000pt}%
\definecolor{currentstroke}{rgb}{0.000000,0.000000,0.000000}%
\pgfsetstrokecolor{currentstroke}%
\pgfsetstrokeopacity{0.700000}%
\pgfsetdash{}{0pt}%
\pgfpathmoveto{\pgfqpoint{6.528587in}{0.550000in}}%
\pgfpathlineto{\pgfqpoint{6.556108in}{0.550000in}}%
\pgfpathlineto{\pgfqpoint{6.556108in}{0.591936in}}%
\pgfpathlineto{\pgfqpoint{6.528587in}{0.591936in}}%
\pgfpathlineto{\pgfqpoint{6.528587in}{0.550000in}}%
\pgfpathclose%
\pgfusepath{fill}%
\end{pgfscope}%
\begin{pgfscope}%
\pgfpathrectangle{\pgfqpoint{1.250000in}{0.550000in}}{\pgfqpoint{7.750000in}{3.850000in}}%
\pgfusepath{clip}%
\pgfsetbuttcap%
\pgfsetmiterjoin%
\definecolor{currentfill}{rgb}{0.501961,0.501961,0.501961}%
\pgfsetfillcolor{currentfill}%
\pgfsetfillopacity{0.700000}%
\pgfsetlinewidth{0.000000pt}%
\definecolor{currentstroke}{rgb}{0.000000,0.000000,0.000000}%
\pgfsetstrokecolor{currentstroke}%
\pgfsetstrokeopacity{0.700000}%
\pgfsetdash{}{0pt}%
\pgfpathmoveto{\pgfqpoint{6.556108in}{0.550000in}}%
\pgfpathlineto{\pgfqpoint{6.583629in}{0.550000in}}%
\pgfpathlineto{\pgfqpoint{6.583629in}{0.592332in}}%
\pgfpathlineto{\pgfqpoint{6.556108in}{0.592332in}}%
\pgfpathlineto{\pgfqpoint{6.556108in}{0.550000in}}%
\pgfpathclose%
\pgfusepath{fill}%
\end{pgfscope}%
\begin{pgfscope}%
\pgfpathrectangle{\pgfqpoint{1.250000in}{0.550000in}}{\pgfqpoint{7.750000in}{3.850000in}}%
\pgfusepath{clip}%
\pgfsetbuttcap%
\pgfsetmiterjoin%
\definecolor{currentfill}{rgb}{0.501961,0.501961,0.501961}%
\pgfsetfillcolor{currentfill}%
\pgfsetfillopacity{0.700000}%
\pgfsetlinewidth{0.000000pt}%
\definecolor{currentstroke}{rgb}{0.000000,0.000000,0.000000}%
\pgfsetstrokecolor{currentstroke}%
\pgfsetstrokeopacity{0.700000}%
\pgfsetdash{}{0pt}%
\pgfpathmoveto{\pgfqpoint{6.583629in}{0.550000in}}%
\pgfpathlineto{\pgfqpoint{6.611151in}{0.550000in}}%
\pgfpathlineto{\pgfqpoint{6.611151in}{0.595497in}}%
\pgfpathlineto{\pgfqpoint{6.583629in}{0.595497in}}%
\pgfpathlineto{\pgfqpoint{6.583629in}{0.550000in}}%
\pgfpathclose%
\pgfusepath{fill}%
\end{pgfscope}%
\begin{pgfscope}%
\pgfpathrectangle{\pgfqpoint{1.250000in}{0.550000in}}{\pgfqpoint{7.750000in}{3.850000in}}%
\pgfusepath{clip}%
\pgfsetbuttcap%
\pgfsetmiterjoin%
\definecolor{currentfill}{rgb}{0.501961,0.501961,0.501961}%
\pgfsetfillcolor{currentfill}%
\pgfsetfillopacity{0.700000}%
\pgfsetlinewidth{0.000000pt}%
\definecolor{currentstroke}{rgb}{0.000000,0.000000,0.000000}%
\pgfsetstrokecolor{currentstroke}%
\pgfsetstrokeopacity{0.700000}%
\pgfsetdash{}{0pt}%
\pgfpathmoveto{\pgfqpoint{6.611151in}{0.550000in}}%
\pgfpathlineto{\pgfqpoint{6.638672in}{0.550000in}}%
\pgfpathlineto{\pgfqpoint{6.638672in}{0.593915in}}%
\pgfpathlineto{\pgfqpoint{6.611151in}{0.593915in}}%
\pgfpathlineto{\pgfqpoint{6.611151in}{0.550000in}}%
\pgfpathclose%
\pgfusepath{fill}%
\end{pgfscope}%
\begin{pgfscope}%
\pgfpathrectangle{\pgfqpoint{1.250000in}{0.550000in}}{\pgfqpoint{7.750000in}{3.850000in}}%
\pgfusepath{clip}%
\pgfsetbuttcap%
\pgfsetmiterjoin%
\definecolor{currentfill}{rgb}{0.501961,0.501961,0.501961}%
\pgfsetfillcolor{currentfill}%
\pgfsetfillopacity{0.700000}%
\pgfsetlinewidth{0.000000pt}%
\definecolor{currentstroke}{rgb}{0.000000,0.000000,0.000000}%
\pgfsetstrokecolor{currentstroke}%
\pgfsetstrokeopacity{0.700000}%
\pgfsetdash{}{0pt}%
\pgfpathmoveto{\pgfqpoint{6.638672in}{0.550000in}}%
\pgfpathlineto{\pgfqpoint{6.666193in}{0.550000in}}%
\pgfpathlineto{\pgfqpoint{6.666193in}{0.586398in}}%
\pgfpathlineto{\pgfqpoint{6.638672in}{0.586398in}}%
\pgfpathlineto{\pgfqpoint{6.638672in}{0.550000in}}%
\pgfpathclose%
\pgfusepath{fill}%
\end{pgfscope}%
\begin{pgfscope}%
\pgfpathrectangle{\pgfqpoint{1.250000in}{0.550000in}}{\pgfqpoint{7.750000in}{3.850000in}}%
\pgfusepath{clip}%
\pgfsetbuttcap%
\pgfsetmiterjoin%
\definecolor{currentfill}{rgb}{0.501961,0.501961,0.501961}%
\pgfsetfillcolor{currentfill}%
\pgfsetfillopacity{0.700000}%
\pgfsetlinewidth{0.000000pt}%
\definecolor{currentstroke}{rgb}{0.000000,0.000000,0.000000}%
\pgfsetstrokecolor{currentstroke}%
\pgfsetstrokeopacity{0.700000}%
\pgfsetdash{}{0pt}%
\pgfpathmoveto{\pgfqpoint{6.666193in}{0.550000in}}%
\pgfpathlineto{\pgfqpoint{6.693714in}{0.550000in}}%
\pgfpathlineto{\pgfqpoint{6.693714in}{0.585606in}}%
\pgfpathlineto{\pgfqpoint{6.666193in}{0.585606in}}%
\pgfpathlineto{\pgfqpoint{6.666193in}{0.550000in}}%
\pgfpathclose%
\pgfusepath{fill}%
\end{pgfscope}%
\begin{pgfscope}%
\pgfpathrectangle{\pgfqpoint{1.250000in}{0.550000in}}{\pgfqpoint{7.750000in}{3.850000in}}%
\pgfusepath{clip}%
\pgfsetbuttcap%
\pgfsetmiterjoin%
\definecolor{currentfill}{rgb}{0.501961,0.501961,0.501961}%
\pgfsetfillcolor{currentfill}%
\pgfsetfillopacity{0.700000}%
\pgfsetlinewidth{0.000000pt}%
\definecolor{currentstroke}{rgb}{0.000000,0.000000,0.000000}%
\pgfsetstrokecolor{currentstroke}%
\pgfsetstrokeopacity{0.700000}%
\pgfsetdash{}{0pt}%
\pgfpathmoveto{\pgfqpoint{6.693714in}{0.550000in}}%
\pgfpathlineto{\pgfqpoint{6.721236in}{0.550000in}}%
\pgfpathlineto{\pgfqpoint{6.721236in}{0.590354in}}%
\pgfpathlineto{\pgfqpoint{6.693714in}{0.590354in}}%
\pgfpathlineto{\pgfqpoint{6.693714in}{0.550000in}}%
\pgfpathclose%
\pgfusepath{fill}%
\end{pgfscope}%
\begin{pgfscope}%
\pgfpathrectangle{\pgfqpoint{1.250000in}{0.550000in}}{\pgfqpoint{7.750000in}{3.850000in}}%
\pgfusepath{clip}%
\pgfsetbuttcap%
\pgfsetmiterjoin%
\definecolor{currentfill}{rgb}{0.501961,0.501961,0.501961}%
\pgfsetfillcolor{currentfill}%
\pgfsetfillopacity{0.700000}%
\pgfsetlinewidth{0.000000pt}%
\definecolor{currentstroke}{rgb}{0.000000,0.000000,0.000000}%
\pgfsetstrokecolor{currentstroke}%
\pgfsetstrokeopacity{0.700000}%
\pgfsetdash{}{0pt}%
\pgfpathmoveto{\pgfqpoint{6.721236in}{0.550000in}}%
\pgfpathlineto{\pgfqpoint{6.748757in}{0.550000in}}%
\pgfpathlineto{\pgfqpoint{6.748757in}{0.591936in}}%
\pgfpathlineto{\pgfqpoint{6.721236in}{0.591936in}}%
\pgfpathlineto{\pgfqpoint{6.721236in}{0.550000in}}%
\pgfpathclose%
\pgfusepath{fill}%
\end{pgfscope}%
\begin{pgfscope}%
\pgfpathrectangle{\pgfqpoint{1.250000in}{0.550000in}}{\pgfqpoint{7.750000in}{3.850000in}}%
\pgfusepath{clip}%
\pgfsetbuttcap%
\pgfsetmiterjoin%
\definecolor{currentfill}{rgb}{0.501961,0.501961,0.501961}%
\pgfsetfillcolor{currentfill}%
\pgfsetfillopacity{0.700000}%
\pgfsetlinewidth{0.000000pt}%
\definecolor{currentstroke}{rgb}{0.000000,0.000000,0.000000}%
\pgfsetstrokecolor{currentstroke}%
\pgfsetstrokeopacity{0.700000}%
\pgfsetdash{}{0pt}%
\pgfpathmoveto{\pgfqpoint{6.748757in}{0.550000in}}%
\pgfpathlineto{\pgfqpoint{6.776278in}{0.550000in}}%
\pgfpathlineto{\pgfqpoint{6.776278in}{0.550000in}}%
\pgfpathlineto{\pgfqpoint{6.748757in}{0.550000in}}%
\pgfpathlineto{\pgfqpoint{6.748757in}{0.550000in}}%
\pgfpathclose%
\pgfusepath{fill}%
\end{pgfscope}%
\begin{pgfscope}%
\pgfpathrectangle{\pgfqpoint{1.250000in}{0.550000in}}{\pgfqpoint{7.750000in}{3.850000in}}%
\pgfusepath{clip}%
\pgfsetbuttcap%
\pgfsetmiterjoin%
\definecolor{currentfill}{rgb}{0.501961,0.501961,0.501961}%
\pgfsetfillcolor{currentfill}%
\pgfsetfillopacity{0.700000}%
\pgfsetlinewidth{0.000000pt}%
\definecolor{currentstroke}{rgb}{0.000000,0.000000,0.000000}%
\pgfsetstrokecolor{currentstroke}%
\pgfsetstrokeopacity{0.700000}%
\pgfsetdash{}{0pt}%
\pgfpathmoveto{\pgfqpoint{6.776278in}{0.550000in}}%
\pgfpathlineto{\pgfqpoint{6.803800in}{0.550000in}}%
\pgfpathlineto{\pgfqpoint{6.803800in}{0.595101in}}%
\pgfpathlineto{\pgfqpoint{6.776278in}{0.595101in}}%
\pgfpathlineto{\pgfqpoint{6.776278in}{0.550000in}}%
\pgfpathclose%
\pgfusepath{fill}%
\end{pgfscope}%
\begin{pgfscope}%
\pgfpathrectangle{\pgfqpoint{1.250000in}{0.550000in}}{\pgfqpoint{7.750000in}{3.850000in}}%
\pgfusepath{clip}%
\pgfsetbuttcap%
\pgfsetmiterjoin%
\definecolor{currentfill}{rgb}{0.501961,0.501961,0.501961}%
\pgfsetfillcolor{currentfill}%
\pgfsetfillopacity{0.700000}%
\pgfsetlinewidth{0.000000pt}%
\definecolor{currentstroke}{rgb}{0.000000,0.000000,0.000000}%
\pgfsetstrokecolor{currentstroke}%
\pgfsetstrokeopacity{0.700000}%
\pgfsetdash{}{0pt}%
\pgfpathmoveto{\pgfqpoint{6.803800in}{0.550000in}}%
\pgfpathlineto{\pgfqpoint{6.831321in}{0.550000in}}%
\pgfpathlineto{\pgfqpoint{6.831321in}{0.592728in}}%
\pgfpathlineto{\pgfqpoint{6.803800in}{0.592728in}}%
\pgfpathlineto{\pgfqpoint{6.803800in}{0.550000in}}%
\pgfpathclose%
\pgfusepath{fill}%
\end{pgfscope}%
\begin{pgfscope}%
\pgfpathrectangle{\pgfqpoint{1.250000in}{0.550000in}}{\pgfqpoint{7.750000in}{3.850000in}}%
\pgfusepath{clip}%
\pgfsetbuttcap%
\pgfsetmiterjoin%
\definecolor{currentfill}{rgb}{0.501961,0.501961,0.501961}%
\pgfsetfillcolor{currentfill}%
\pgfsetfillopacity{0.700000}%
\pgfsetlinewidth{0.000000pt}%
\definecolor{currentstroke}{rgb}{0.000000,0.000000,0.000000}%
\pgfsetstrokecolor{currentstroke}%
\pgfsetstrokeopacity{0.700000}%
\pgfsetdash{}{0pt}%
\pgfpathmoveto{\pgfqpoint{6.831321in}{0.550000in}}%
\pgfpathlineto{\pgfqpoint{6.858842in}{0.550000in}}%
\pgfpathlineto{\pgfqpoint{6.858842in}{0.595101in}}%
\pgfpathlineto{\pgfqpoint{6.831321in}{0.595101in}}%
\pgfpathlineto{\pgfqpoint{6.831321in}{0.550000in}}%
\pgfpathclose%
\pgfusepath{fill}%
\end{pgfscope}%
\begin{pgfscope}%
\pgfpathrectangle{\pgfqpoint{1.250000in}{0.550000in}}{\pgfqpoint{7.750000in}{3.850000in}}%
\pgfusepath{clip}%
\pgfsetbuttcap%
\pgfsetmiterjoin%
\definecolor{currentfill}{rgb}{0.501961,0.501961,0.501961}%
\pgfsetfillcolor{currentfill}%
\pgfsetfillopacity{0.700000}%
\pgfsetlinewidth{0.000000pt}%
\definecolor{currentstroke}{rgb}{0.000000,0.000000,0.000000}%
\pgfsetstrokecolor{currentstroke}%
\pgfsetstrokeopacity{0.700000}%
\pgfsetdash{}{0pt}%
\pgfpathmoveto{\pgfqpoint{6.858842in}{0.550000in}}%
\pgfpathlineto{\pgfqpoint{6.886364in}{0.550000in}}%
\pgfpathlineto{\pgfqpoint{6.886364in}{0.591145in}}%
\pgfpathlineto{\pgfqpoint{6.858842in}{0.591145in}}%
\pgfpathlineto{\pgfqpoint{6.858842in}{0.550000in}}%
\pgfpathclose%
\pgfusepath{fill}%
\end{pgfscope}%
\begin{pgfscope}%
\pgfpathrectangle{\pgfqpoint{1.250000in}{0.550000in}}{\pgfqpoint{7.750000in}{3.850000in}}%
\pgfusepath{clip}%
\pgfsetbuttcap%
\pgfsetmiterjoin%
\definecolor{currentfill}{rgb}{0.501961,0.501961,0.501961}%
\pgfsetfillcolor{currentfill}%
\pgfsetfillopacity{0.700000}%
\pgfsetlinewidth{0.000000pt}%
\definecolor{currentstroke}{rgb}{0.000000,0.000000,0.000000}%
\pgfsetstrokecolor{currentstroke}%
\pgfsetstrokeopacity{0.700000}%
\pgfsetdash{}{0pt}%
\pgfpathmoveto{\pgfqpoint{6.886364in}{0.550000in}}%
\pgfpathlineto{\pgfqpoint{6.913885in}{0.550000in}}%
\pgfpathlineto{\pgfqpoint{6.913885in}{0.590354in}}%
\pgfpathlineto{\pgfqpoint{6.886364in}{0.590354in}}%
\pgfpathlineto{\pgfqpoint{6.886364in}{0.550000in}}%
\pgfpathclose%
\pgfusepath{fill}%
\end{pgfscope}%
\begin{pgfscope}%
\pgfpathrectangle{\pgfqpoint{1.250000in}{0.550000in}}{\pgfqpoint{7.750000in}{3.850000in}}%
\pgfusepath{clip}%
\pgfsetbuttcap%
\pgfsetmiterjoin%
\definecolor{currentfill}{rgb}{0.501961,0.501961,0.501961}%
\pgfsetfillcolor{currentfill}%
\pgfsetfillopacity{0.700000}%
\pgfsetlinewidth{0.000000pt}%
\definecolor{currentstroke}{rgb}{0.000000,0.000000,0.000000}%
\pgfsetstrokecolor{currentstroke}%
\pgfsetstrokeopacity{0.700000}%
\pgfsetdash{}{0pt}%
\pgfpathmoveto{\pgfqpoint{6.913885in}{0.550000in}}%
\pgfpathlineto{\pgfqpoint{6.941406in}{0.550000in}}%
\pgfpathlineto{\pgfqpoint{6.941406in}{0.587980in}}%
\pgfpathlineto{\pgfqpoint{6.913885in}{0.587980in}}%
\pgfpathlineto{\pgfqpoint{6.913885in}{0.550000in}}%
\pgfpathclose%
\pgfusepath{fill}%
\end{pgfscope}%
\begin{pgfscope}%
\pgfpathrectangle{\pgfqpoint{1.250000in}{0.550000in}}{\pgfqpoint{7.750000in}{3.850000in}}%
\pgfusepath{clip}%
\pgfsetbuttcap%
\pgfsetmiterjoin%
\definecolor{currentfill}{rgb}{0.501961,0.501961,0.501961}%
\pgfsetfillcolor{currentfill}%
\pgfsetfillopacity{0.700000}%
\pgfsetlinewidth{0.000000pt}%
\definecolor{currentstroke}{rgb}{0.000000,0.000000,0.000000}%
\pgfsetstrokecolor{currentstroke}%
\pgfsetstrokeopacity{0.700000}%
\pgfsetdash{}{0pt}%
\pgfpathmoveto{\pgfqpoint{6.941406in}{0.550000in}}%
\pgfpathlineto{\pgfqpoint{6.968928in}{0.550000in}}%
\pgfpathlineto{\pgfqpoint{6.968928in}{0.585211in}}%
\pgfpathlineto{\pgfqpoint{6.941406in}{0.585211in}}%
\pgfpathlineto{\pgfqpoint{6.941406in}{0.550000in}}%
\pgfpathclose%
\pgfusepath{fill}%
\end{pgfscope}%
\begin{pgfscope}%
\pgfpathrectangle{\pgfqpoint{1.250000in}{0.550000in}}{\pgfqpoint{7.750000in}{3.850000in}}%
\pgfusepath{clip}%
\pgfsetbuttcap%
\pgfsetmiterjoin%
\definecolor{currentfill}{rgb}{0.501961,0.501961,0.501961}%
\pgfsetfillcolor{currentfill}%
\pgfsetfillopacity{0.700000}%
\pgfsetlinewidth{0.000000pt}%
\definecolor{currentstroke}{rgb}{0.000000,0.000000,0.000000}%
\pgfsetstrokecolor{currentstroke}%
\pgfsetstrokeopacity{0.700000}%
\pgfsetdash{}{0pt}%
\pgfpathmoveto{\pgfqpoint{6.968928in}{0.550000in}}%
\pgfpathlineto{\pgfqpoint{6.996449in}{0.550000in}}%
\pgfpathlineto{\pgfqpoint{6.996449in}{0.587189in}}%
\pgfpathlineto{\pgfqpoint{6.968928in}{0.587189in}}%
\pgfpathlineto{\pgfqpoint{6.968928in}{0.550000in}}%
\pgfpathclose%
\pgfusepath{fill}%
\end{pgfscope}%
\begin{pgfscope}%
\pgfpathrectangle{\pgfqpoint{1.250000in}{0.550000in}}{\pgfqpoint{7.750000in}{3.850000in}}%
\pgfusepath{clip}%
\pgfsetbuttcap%
\pgfsetmiterjoin%
\definecolor{currentfill}{rgb}{0.501961,0.501961,0.501961}%
\pgfsetfillcolor{currentfill}%
\pgfsetfillopacity{0.700000}%
\pgfsetlinewidth{0.000000pt}%
\definecolor{currentstroke}{rgb}{0.000000,0.000000,0.000000}%
\pgfsetstrokecolor{currentstroke}%
\pgfsetstrokeopacity{0.700000}%
\pgfsetdash{}{0pt}%
\pgfpathmoveto{\pgfqpoint{6.996449in}{0.550000in}}%
\pgfpathlineto{\pgfqpoint{7.023970in}{0.550000in}}%
\pgfpathlineto{\pgfqpoint{7.023970in}{0.591936in}}%
\pgfpathlineto{\pgfqpoint{6.996449in}{0.591936in}}%
\pgfpathlineto{\pgfqpoint{6.996449in}{0.550000in}}%
\pgfpathclose%
\pgfusepath{fill}%
\end{pgfscope}%
\begin{pgfscope}%
\pgfpathrectangle{\pgfqpoint{1.250000in}{0.550000in}}{\pgfqpoint{7.750000in}{3.850000in}}%
\pgfusepath{clip}%
\pgfsetbuttcap%
\pgfsetmiterjoin%
\definecolor{currentfill}{rgb}{0.501961,0.501961,0.501961}%
\pgfsetfillcolor{currentfill}%
\pgfsetfillopacity{0.700000}%
\pgfsetlinewidth{0.000000pt}%
\definecolor{currentstroke}{rgb}{0.000000,0.000000,0.000000}%
\pgfsetstrokecolor{currentstroke}%
\pgfsetstrokeopacity{0.700000}%
\pgfsetdash{}{0pt}%
\pgfpathmoveto{\pgfqpoint{7.023970in}{0.550000in}}%
\pgfpathlineto{\pgfqpoint{7.051491in}{0.550000in}}%
\pgfpathlineto{\pgfqpoint{7.051491in}{0.586398in}}%
\pgfpathlineto{\pgfqpoint{7.023970in}{0.586398in}}%
\pgfpathlineto{\pgfqpoint{7.023970in}{0.550000in}}%
\pgfpathclose%
\pgfusepath{fill}%
\end{pgfscope}%
\begin{pgfscope}%
\pgfpathrectangle{\pgfqpoint{1.250000in}{0.550000in}}{\pgfqpoint{7.750000in}{3.850000in}}%
\pgfusepath{clip}%
\pgfsetbuttcap%
\pgfsetmiterjoin%
\definecolor{currentfill}{rgb}{0.501961,0.501961,0.501961}%
\pgfsetfillcolor{currentfill}%
\pgfsetfillopacity{0.700000}%
\pgfsetlinewidth{0.000000pt}%
\definecolor{currentstroke}{rgb}{0.000000,0.000000,0.000000}%
\pgfsetstrokecolor{currentstroke}%
\pgfsetstrokeopacity{0.700000}%
\pgfsetdash{}{0pt}%
\pgfpathmoveto{\pgfqpoint{7.051491in}{0.550000in}}%
\pgfpathlineto{\pgfqpoint{7.079013in}{0.550000in}}%
\pgfpathlineto{\pgfqpoint{7.079013in}{0.593915in}}%
\pgfpathlineto{\pgfqpoint{7.051491in}{0.593915in}}%
\pgfpathlineto{\pgfqpoint{7.051491in}{0.550000in}}%
\pgfpathclose%
\pgfusepath{fill}%
\end{pgfscope}%
\begin{pgfscope}%
\pgfpathrectangle{\pgfqpoint{1.250000in}{0.550000in}}{\pgfqpoint{7.750000in}{3.850000in}}%
\pgfusepath{clip}%
\pgfsetbuttcap%
\pgfsetmiterjoin%
\definecolor{currentfill}{rgb}{0.501961,0.501961,0.501961}%
\pgfsetfillcolor{currentfill}%
\pgfsetfillopacity{0.700000}%
\pgfsetlinewidth{0.000000pt}%
\definecolor{currentstroke}{rgb}{0.000000,0.000000,0.000000}%
\pgfsetstrokecolor{currentstroke}%
\pgfsetstrokeopacity{0.700000}%
\pgfsetdash{}{0pt}%
\pgfpathmoveto{\pgfqpoint{7.079013in}{0.550000in}}%
\pgfpathlineto{\pgfqpoint{7.106534in}{0.550000in}}%
\pgfpathlineto{\pgfqpoint{7.106534in}{0.601431in}}%
\pgfpathlineto{\pgfqpoint{7.079013in}{0.601431in}}%
\pgfpathlineto{\pgfqpoint{7.079013in}{0.550000in}}%
\pgfpathclose%
\pgfusepath{fill}%
\end{pgfscope}%
\begin{pgfscope}%
\pgfpathrectangle{\pgfqpoint{1.250000in}{0.550000in}}{\pgfqpoint{7.750000in}{3.850000in}}%
\pgfusepath{clip}%
\pgfsetbuttcap%
\pgfsetmiterjoin%
\definecolor{currentfill}{rgb}{0.501961,0.501961,0.501961}%
\pgfsetfillcolor{currentfill}%
\pgfsetfillopacity{0.700000}%
\pgfsetlinewidth{0.000000pt}%
\definecolor{currentstroke}{rgb}{0.000000,0.000000,0.000000}%
\pgfsetstrokecolor{currentstroke}%
\pgfsetstrokeopacity{0.700000}%
\pgfsetdash{}{0pt}%
\pgfpathmoveto{\pgfqpoint{7.106534in}{0.550000in}}%
\pgfpathlineto{\pgfqpoint{7.134055in}{0.550000in}}%
\pgfpathlineto{\pgfqpoint{7.134055in}{0.594310in}}%
\pgfpathlineto{\pgfqpoint{7.106534in}{0.594310in}}%
\pgfpathlineto{\pgfqpoint{7.106534in}{0.550000in}}%
\pgfpathclose%
\pgfusepath{fill}%
\end{pgfscope}%
\begin{pgfscope}%
\pgfpathrectangle{\pgfqpoint{1.250000in}{0.550000in}}{\pgfqpoint{7.750000in}{3.850000in}}%
\pgfusepath{clip}%
\pgfsetbuttcap%
\pgfsetmiterjoin%
\definecolor{currentfill}{rgb}{0.501961,0.501961,0.501961}%
\pgfsetfillcolor{currentfill}%
\pgfsetfillopacity{0.700000}%
\pgfsetlinewidth{0.000000pt}%
\definecolor{currentstroke}{rgb}{0.000000,0.000000,0.000000}%
\pgfsetstrokecolor{currentstroke}%
\pgfsetstrokeopacity{0.700000}%
\pgfsetdash{}{0pt}%
\pgfpathmoveto{\pgfqpoint{7.134055in}{0.550000in}}%
\pgfpathlineto{\pgfqpoint{7.161577in}{0.550000in}}%
\pgfpathlineto{\pgfqpoint{7.161577in}{0.550000in}}%
\pgfpathlineto{\pgfqpoint{7.134055in}{0.550000in}}%
\pgfpathlineto{\pgfqpoint{7.134055in}{0.550000in}}%
\pgfpathclose%
\pgfusepath{fill}%
\end{pgfscope}%
\begin{pgfscope}%
\pgfpathrectangle{\pgfqpoint{1.250000in}{0.550000in}}{\pgfqpoint{7.750000in}{3.850000in}}%
\pgfusepath{clip}%
\pgfsetbuttcap%
\pgfsetmiterjoin%
\definecolor{currentfill}{rgb}{0.501961,0.501961,0.501961}%
\pgfsetfillcolor{currentfill}%
\pgfsetfillopacity{0.700000}%
\pgfsetlinewidth{0.000000pt}%
\definecolor{currentstroke}{rgb}{0.000000,0.000000,0.000000}%
\pgfsetstrokecolor{currentstroke}%
\pgfsetstrokeopacity{0.700000}%
\pgfsetdash{}{0pt}%
\pgfpathmoveto{\pgfqpoint{7.161577in}{0.550000in}}%
\pgfpathlineto{\pgfqpoint{7.189098in}{0.550000in}}%
\pgfpathlineto{\pgfqpoint{7.189098in}{0.600245in}}%
\pgfpathlineto{\pgfqpoint{7.161577in}{0.600245in}}%
\pgfpathlineto{\pgfqpoint{7.161577in}{0.550000in}}%
\pgfpathclose%
\pgfusepath{fill}%
\end{pgfscope}%
\begin{pgfscope}%
\pgfpathrectangle{\pgfqpoint{1.250000in}{0.550000in}}{\pgfqpoint{7.750000in}{3.850000in}}%
\pgfusepath{clip}%
\pgfsetbuttcap%
\pgfsetmiterjoin%
\definecolor{currentfill}{rgb}{0.501961,0.501961,0.501961}%
\pgfsetfillcolor{currentfill}%
\pgfsetfillopacity{0.700000}%
\pgfsetlinewidth{0.000000pt}%
\definecolor{currentstroke}{rgb}{0.000000,0.000000,0.000000}%
\pgfsetstrokecolor{currentstroke}%
\pgfsetstrokeopacity{0.700000}%
\pgfsetdash{}{0pt}%
\pgfpathmoveto{\pgfqpoint{7.189098in}{0.550000in}}%
\pgfpathlineto{\pgfqpoint{7.216619in}{0.550000in}}%
\pgfpathlineto{\pgfqpoint{7.216619in}{0.596288in}}%
\pgfpathlineto{\pgfqpoint{7.189098in}{0.596288in}}%
\pgfpathlineto{\pgfqpoint{7.189098in}{0.550000in}}%
\pgfpathclose%
\pgfusepath{fill}%
\end{pgfscope}%
\begin{pgfscope}%
\pgfpathrectangle{\pgfqpoint{1.250000in}{0.550000in}}{\pgfqpoint{7.750000in}{3.850000in}}%
\pgfusepath{clip}%
\pgfsetbuttcap%
\pgfsetmiterjoin%
\definecolor{currentfill}{rgb}{0.501961,0.501961,0.501961}%
\pgfsetfillcolor{currentfill}%
\pgfsetfillopacity{0.700000}%
\pgfsetlinewidth{0.000000pt}%
\definecolor{currentstroke}{rgb}{0.000000,0.000000,0.000000}%
\pgfsetstrokecolor{currentstroke}%
\pgfsetstrokeopacity{0.700000}%
\pgfsetdash{}{0pt}%
\pgfpathmoveto{\pgfqpoint{7.216619in}{0.550000in}}%
\pgfpathlineto{\pgfqpoint{7.244141in}{0.550000in}}%
\pgfpathlineto{\pgfqpoint{7.244141in}{0.591936in}}%
\pgfpathlineto{\pgfqpoint{7.216619in}{0.591936in}}%
\pgfpathlineto{\pgfqpoint{7.216619in}{0.550000in}}%
\pgfpathclose%
\pgfusepath{fill}%
\end{pgfscope}%
\begin{pgfscope}%
\pgfpathrectangle{\pgfqpoint{1.250000in}{0.550000in}}{\pgfqpoint{7.750000in}{3.850000in}}%
\pgfusepath{clip}%
\pgfsetbuttcap%
\pgfsetmiterjoin%
\definecolor{currentfill}{rgb}{0.501961,0.501961,0.501961}%
\pgfsetfillcolor{currentfill}%
\pgfsetfillopacity{0.700000}%
\pgfsetlinewidth{0.000000pt}%
\definecolor{currentstroke}{rgb}{0.000000,0.000000,0.000000}%
\pgfsetstrokecolor{currentstroke}%
\pgfsetstrokeopacity{0.700000}%
\pgfsetdash{}{0pt}%
\pgfpathmoveto{\pgfqpoint{7.244141in}{0.550000in}}%
\pgfpathlineto{\pgfqpoint{7.271662in}{0.550000in}}%
\pgfpathlineto{\pgfqpoint{7.271662in}{0.593519in}}%
\pgfpathlineto{\pgfqpoint{7.244141in}{0.593519in}}%
\pgfpathlineto{\pgfqpoint{7.244141in}{0.550000in}}%
\pgfpathclose%
\pgfusepath{fill}%
\end{pgfscope}%
\begin{pgfscope}%
\pgfpathrectangle{\pgfqpoint{1.250000in}{0.550000in}}{\pgfqpoint{7.750000in}{3.850000in}}%
\pgfusepath{clip}%
\pgfsetbuttcap%
\pgfsetmiterjoin%
\definecolor{currentfill}{rgb}{0.501961,0.501961,0.501961}%
\pgfsetfillcolor{currentfill}%
\pgfsetfillopacity{0.700000}%
\pgfsetlinewidth{0.000000pt}%
\definecolor{currentstroke}{rgb}{0.000000,0.000000,0.000000}%
\pgfsetstrokecolor{currentstroke}%
\pgfsetstrokeopacity{0.700000}%
\pgfsetdash{}{0pt}%
\pgfpathmoveto{\pgfqpoint{7.271662in}{0.550000in}}%
\pgfpathlineto{\pgfqpoint{7.299183in}{0.550000in}}%
\pgfpathlineto{\pgfqpoint{7.299183in}{0.590750in}}%
\pgfpathlineto{\pgfqpoint{7.271662in}{0.590750in}}%
\pgfpathlineto{\pgfqpoint{7.271662in}{0.550000in}}%
\pgfpathclose%
\pgfusepath{fill}%
\end{pgfscope}%
\begin{pgfscope}%
\pgfpathrectangle{\pgfqpoint{1.250000in}{0.550000in}}{\pgfqpoint{7.750000in}{3.850000in}}%
\pgfusepath{clip}%
\pgfsetbuttcap%
\pgfsetmiterjoin%
\definecolor{currentfill}{rgb}{0.501961,0.501961,0.501961}%
\pgfsetfillcolor{currentfill}%
\pgfsetfillopacity{0.700000}%
\pgfsetlinewidth{0.000000pt}%
\definecolor{currentstroke}{rgb}{0.000000,0.000000,0.000000}%
\pgfsetstrokecolor{currentstroke}%
\pgfsetstrokeopacity{0.700000}%
\pgfsetdash{}{0pt}%
\pgfpathmoveto{\pgfqpoint{7.299183in}{0.550000in}}%
\pgfpathlineto{\pgfqpoint{7.326705in}{0.550000in}}%
\pgfpathlineto{\pgfqpoint{7.326705in}{0.586793in}}%
\pgfpathlineto{\pgfqpoint{7.299183in}{0.586793in}}%
\pgfpathlineto{\pgfqpoint{7.299183in}{0.550000in}}%
\pgfpathclose%
\pgfusepath{fill}%
\end{pgfscope}%
\begin{pgfscope}%
\pgfpathrectangle{\pgfqpoint{1.250000in}{0.550000in}}{\pgfqpoint{7.750000in}{3.850000in}}%
\pgfusepath{clip}%
\pgfsetbuttcap%
\pgfsetmiterjoin%
\definecolor{currentfill}{rgb}{0.501961,0.501961,0.501961}%
\pgfsetfillcolor{currentfill}%
\pgfsetfillopacity{0.700000}%
\pgfsetlinewidth{0.000000pt}%
\definecolor{currentstroke}{rgb}{0.000000,0.000000,0.000000}%
\pgfsetstrokecolor{currentstroke}%
\pgfsetstrokeopacity{0.700000}%
\pgfsetdash{}{0pt}%
\pgfpathmoveto{\pgfqpoint{7.326705in}{0.550000in}}%
\pgfpathlineto{\pgfqpoint{7.354226in}{0.550000in}}%
\pgfpathlineto{\pgfqpoint{7.354226in}{0.592332in}}%
\pgfpathlineto{\pgfqpoint{7.326705in}{0.592332in}}%
\pgfpathlineto{\pgfqpoint{7.326705in}{0.550000in}}%
\pgfpathclose%
\pgfusepath{fill}%
\end{pgfscope}%
\begin{pgfscope}%
\pgfpathrectangle{\pgfqpoint{1.250000in}{0.550000in}}{\pgfqpoint{7.750000in}{3.850000in}}%
\pgfusepath{clip}%
\pgfsetbuttcap%
\pgfsetmiterjoin%
\definecolor{currentfill}{rgb}{0.501961,0.501961,0.501961}%
\pgfsetfillcolor{currentfill}%
\pgfsetfillopacity{0.700000}%
\pgfsetlinewidth{0.000000pt}%
\definecolor{currentstroke}{rgb}{0.000000,0.000000,0.000000}%
\pgfsetstrokecolor{currentstroke}%
\pgfsetstrokeopacity{0.700000}%
\pgfsetdash{}{0pt}%
\pgfpathmoveto{\pgfqpoint{7.354226in}{0.550000in}}%
\pgfpathlineto{\pgfqpoint{7.381747in}{0.550000in}}%
\pgfpathlineto{\pgfqpoint{7.381747in}{0.592728in}}%
\pgfpathlineto{\pgfqpoint{7.354226in}{0.592728in}}%
\pgfpathlineto{\pgfqpoint{7.354226in}{0.550000in}}%
\pgfpathclose%
\pgfusepath{fill}%
\end{pgfscope}%
\begin{pgfscope}%
\pgfpathrectangle{\pgfqpoint{1.250000in}{0.550000in}}{\pgfqpoint{7.750000in}{3.850000in}}%
\pgfusepath{clip}%
\pgfsetbuttcap%
\pgfsetmiterjoin%
\definecolor{currentfill}{rgb}{0.501961,0.501961,0.501961}%
\pgfsetfillcolor{currentfill}%
\pgfsetfillopacity{0.700000}%
\pgfsetlinewidth{0.000000pt}%
\definecolor{currentstroke}{rgb}{0.000000,0.000000,0.000000}%
\pgfsetstrokecolor{currentstroke}%
\pgfsetstrokeopacity{0.700000}%
\pgfsetdash{}{0pt}%
\pgfpathmoveto{\pgfqpoint{7.381747in}{0.550000in}}%
\pgfpathlineto{\pgfqpoint{7.409268in}{0.550000in}}%
\pgfpathlineto{\pgfqpoint{7.409268in}{0.593915in}}%
\pgfpathlineto{\pgfqpoint{7.381747in}{0.593915in}}%
\pgfpathlineto{\pgfqpoint{7.381747in}{0.550000in}}%
\pgfpathclose%
\pgfusepath{fill}%
\end{pgfscope}%
\begin{pgfscope}%
\pgfpathrectangle{\pgfqpoint{1.250000in}{0.550000in}}{\pgfqpoint{7.750000in}{3.850000in}}%
\pgfusepath{clip}%
\pgfsetbuttcap%
\pgfsetmiterjoin%
\definecolor{currentfill}{rgb}{0.501961,0.501961,0.501961}%
\pgfsetfillcolor{currentfill}%
\pgfsetfillopacity{0.700000}%
\pgfsetlinewidth{0.000000pt}%
\definecolor{currentstroke}{rgb}{0.000000,0.000000,0.000000}%
\pgfsetstrokecolor{currentstroke}%
\pgfsetstrokeopacity{0.700000}%
\pgfsetdash{}{0pt}%
\pgfpathmoveto{\pgfqpoint{7.409268in}{0.550000in}}%
\pgfpathlineto{\pgfqpoint{7.436790in}{0.550000in}}%
\pgfpathlineto{\pgfqpoint{7.436790in}{0.598662in}}%
\pgfpathlineto{\pgfqpoint{7.409268in}{0.598662in}}%
\pgfpathlineto{\pgfqpoint{7.409268in}{0.550000in}}%
\pgfpathclose%
\pgfusepath{fill}%
\end{pgfscope}%
\begin{pgfscope}%
\pgfpathrectangle{\pgfqpoint{1.250000in}{0.550000in}}{\pgfqpoint{7.750000in}{3.850000in}}%
\pgfusepath{clip}%
\pgfsetbuttcap%
\pgfsetmiterjoin%
\definecolor{currentfill}{rgb}{0.501961,0.501961,0.501961}%
\pgfsetfillcolor{currentfill}%
\pgfsetfillopacity{0.700000}%
\pgfsetlinewidth{0.000000pt}%
\definecolor{currentstroke}{rgb}{0.000000,0.000000,0.000000}%
\pgfsetstrokecolor{currentstroke}%
\pgfsetstrokeopacity{0.700000}%
\pgfsetdash{}{0pt}%
\pgfpathmoveto{\pgfqpoint{7.436790in}{0.550000in}}%
\pgfpathlineto{\pgfqpoint{7.464311in}{0.550000in}}%
\pgfpathlineto{\pgfqpoint{7.464311in}{0.550000in}}%
\pgfpathlineto{\pgfqpoint{7.436790in}{0.550000in}}%
\pgfpathlineto{\pgfqpoint{7.436790in}{0.550000in}}%
\pgfpathclose%
\pgfusepath{fill}%
\end{pgfscope}%
\begin{pgfscope}%
\pgfpathrectangle{\pgfqpoint{1.250000in}{0.550000in}}{\pgfqpoint{7.750000in}{3.850000in}}%
\pgfusepath{clip}%
\pgfsetbuttcap%
\pgfsetmiterjoin%
\definecolor{currentfill}{rgb}{0.501961,0.501961,0.501961}%
\pgfsetfillcolor{currentfill}%
\pgfsetfillopacity{0.700000}%
\pgfsetlinewidth{0.000000pt}%
\definecolor{currentstroke}{rgb}{0.000000,0.000000,0.000000}%
\pgfsetstrokecolor{currentstroke}%
\pgfsetstrokeopacity{0.700000}%
\pgfsetdash{}{0pt}%
\pgfpathmoveto{\pgfqpoint{7.464311in}{0.550000in}}%
\pgfpathlineto{\pgfqpoint{7.491832in}{0.550000in}}%
\pgfpathlineto{\pgfqpoint{7.491832in}{0.598266in}}%
\pgfpathlineto{\pgfqpoint{7.464311in}{0.598266in}}%
\pgfpathlineto{\pgfqpoint{7.464311in}{0.550000in}}%
\pgfpathclose%
\pgfusepath{fill}%
\end{pgfscope}%
\begin{pgfscope}%
\pgfpathrectangle{\pgfqpoint{1.250000in}{0.550000in}}{\pgfqpoint{7.750000in}{3.850000in}}%
\pgfusepath{clip}%
\pgfsetbuttcap%
\pgfsetmiterjoin%
\definecolor{currentfill}{rgb}{0.501961,0.501961,0.501961}%
\pgfsetfillcolor{currentfill}%
\pgfsetfillopacity{0.700000}%
\pgfsetlinewidth{0.000000pt}%
\definecolor{currentstroke}{rgb}{0.000000,0.000000,0.000000}%
\pgfsetstrokecolor{currentstroke}%
\pgfsetstrokeopacity{0.700000}%
\pgfsetdash{}{0pt}%
\pgfpathmoveto{\pgfqpoint{7.491832in}{0.550000in}}%
\pgfpathlineto{\pgfqpoint{7.519354in}{0.550000in}}%
\pgfpathlineto{\pgfqpoint{7.519354in}{0.596684in}}%
\pgfpathlineto{\pgfqpoint{7.491832in}{0.596684in}}%
\pgfpathlineto{\pgfqpoint{7.491832in}{0.550000in}}%
\pgfpathclose%
\pgfusepath{fill}%
\end{pgfscope}%
\begin{pgfscope}%
\pgfpathrectangle{\pgfqpoint{1.250000in}{0.550000in}}{\pgfqpoint{7.750000in}{3.850000in}}%
\pgfusepath{clip}%
\pgfsetbuttcap%
\pgfsetmiterjoin%
\definecolor{currentfill}{rgb}{0.501961,0.501961,0.501961}%
\pgfsetfillcolor{currentfill}%
\pgfsetfillopacity{0.700000}%
\pgfsetlinewidth{0.000000pt}%
\definecolor{currentstroke}{rgb}{0.000000,0.000000,0.000000}%
\pgfsetstrokecolor{currentstroke}%
\pgfsetstrokeopacity{0.700000}%
\pgfsetdash{}{0pt}%
\pgfpathmoveto{\pgfqpoint{7.519354in}{0.550000in}}%
\pgfpathlineto{\pgfqpoint{7.546875in}{0.550000in}}%
\pgfpathlineto{\pgfqpoint{7.546875in}{0.589563in}}%
\pgfpathlineto{\pgfqpoint{7.519354in}{0.589563in}}%
\pgfpathlineto{\pgfqpoint{7.519354in}{0.550000in}}%
\pgfpathclose%
\pgfusepath{fill}%
\end{pgfscope}%
\begin{pgfscope}%
\pgfpathrectangle{\pgfqpoint{1.250000in}{0.550000in}}{\pgfqpoint{7.750000in}{3.850000in}}%
\pgfusepath{clip}%
\pgfsetbuttcap%
\pgfsetmiterjoin%
\definecolor{currentfill}{rgb}{0.501961,0.501961,0.501961}%
\pgfsetfillcolor{currentfill}%
\pgfsetfillopacity{0.700000}%
\pgfsetlinewidth{0.000000pt}%
\definecolor{currentstroke}{rgb}{0.000000,0.000000,0.000000}%
\pgfsetstrokecolor{currentstroke}%
\pgfsetstrokeopacity{0.700000}%
\pgfsetdash{}{0pt}%
\pgfpathmoveto{\pgfqpoint{7.546875in}{0.550000in}}%
\pgfpathlineto{\pgfqpoint{7.574396in}{0.550000in}}%
\pgfpathlineto{\pgfqpoint{7.574396in}{0.594706in}}%
\pgfpathlineto{\pgfqpoint{7.546875in}{0.594706in}}%
\pgfpathlineto{\pgfqpoint{7.546875in}{0.550000in}}%
\pgfpathclose%
\pgfusepath{fill}%
\end{pgfscope}%
\begin{pgfscope}%
\pgfpathrectangle{\pgfqpoint{1.250000in}{0.550000in}}{\pgfqpoint{7.750000in}{3.850000in}}%
\pgfusepath{clip}%
\pgfsetbuttcap%
\pgfsetmiterjoin%
\definecolor{currentfill}{rgb}{0.501961,0.501961,0.501961}%
\pgfsetfillcolor{currentfill}%
\pgfsetfillopacity{0.700000}%
\pgfsetlinewidth{0.000000pt}%
\definecolor{currentstroke}{rgb}{0.000000,0.000000,0.000000}%
\pgfsetstrokecolor{currentstroke}%
\pgfsetstrokeopacity{0.700000}%
\pgfsetdash{}{0pt}%
\pgfpathmoveto{\pgfqpoint{7.574396in}{0.550000in}}%
\pgfpathlineto{\pgfqpoint{7.601918in}{0.550000in}}%
\pgfpathlineto{\pgfqpoint{7.601918in}{0.595893in}}%
\pgfpathlineto{\pgfqpoint{7.574396in}{0.595893in}}%
\pgfpathlineto{\pgfqpoint{7.574396in}{0.550000in}}%
\pgfpathclose%
\pgfusepath{fill}%
\end{pgfscope}%
\begin{pgfscope}%
\pgfpathrectangle{\pgfqpoint{1.250000in}{0.550000in}}{\pgfqpoint{7.750000in}{3.850000in}}%
\pgfusepath{clip}%
\pgfsetbuttcap%
\pgfsetmiterjoin%
\definecolor{currentfill}{rgb}{0.501961,0.501961,0.501961}%
\pgfsetfillcolor{currentfill}%
\pgfsetfillopacity{0.700000}%
\pgfsetlinewidth{0.000000pt}%
\definecolor{currentstroke}{rgb}{0.000000,0.000000,0.000000}%
\pgfsetstrokecolor{currentstroke}%
\pgfsetstrokeopacity{0.700000}%
\pgfsetdash{}{0pt}%
\pgfpathmoveto{\pgfqpoint{7.601918in}{0.550000in}}%
\pgfpathlineto{\pgfqpoint{7.629439in}{0.550000in}}%
\pgfpathlineto{\pgfqpoint{7.629439in}{0.602223in}}%
\pgfpathlineto{\pgfqpoint{7.601918in}{0.602223in}}%
\pgfpathlineto{\pgfqpoint{7.601918in}{0.550000in}}%
\pgfpathclose%
\pgfusepath{fill}%
\end{pgfscope}%
\begin{pgfscope}%
\pgfpathrectangle{\pgfqpoint{1.250000in}{0.550000in}}{\pgfqpoint{7.750000in}{3.850000in}}%
\pgfusepath{clip}%
\pgfsetbuttcap%
\pgfsetmiterjoin%
\definecolor{currentfill}{rgb}{0.501961,0.501961,0.501961}%
\pgfsetfillcolor{currentfill}%
\pgfsetfillopacity{0.700000}%
\pgfsetlinewidth{0.000000pt}%
\definecolor{currentstroke}{rgb}{0.000000,0.000000,0.000000}%
\pgfsetstrokecolor{currentstroke}%
\pgfsetstrokeopacity{0.700000}%
\pgfsetdash{}{0pt}%
\pgfpathmoveto{\pgfqpoint{7.629439in}{0.550000in}}%
\pgfpathlineto{\pgfqpoint{7.656960in}{0.550000in}}%
\pgfpathlineto{\pgfqpoint{7.656960in}{0.550000in}}%
\pgfpathlineto{\pgfqpoint{7.629439in}{0.550000in}}%
\pgfpathlineto{\pgfqpoint{7.629439in}{0.550000in}}%
\pgfpathclose%
\pgfusepath{fill}%
\end{pgfscope}%
\begin{pgfscope}%
\pgfpathrectangle{\pgfqpoint{1.250000in}{0.550000in}}{\pgfqpoint{7.750000in}{3.850000in}}%
\pgfusepath{clip}%
\pgfsetbuttcap%
\pgfsetmiterjoin%
\definecolor{currentfill}{rgb}{0.501961,0.501961,0.501961}%
\pgfsetfillcolor{currentfill}%
\pgfsetfillopacity{0.700000}%
\pgfsetlinewidth{0.000000pt}%
\definecolor{currentstroke}{rgb}{0.000000,0.000000,0.000000}%
\pgfsetstrokecolor{currentstroke}%
\pgfsetstrokeopacity{0.700000}%
\pgfsetdash{}{0pt}%
\pgfpathmoveto{\pgfqpoint{7.656960in}{0.550000in}}%
\pgfpathlineto{\pgfqpoint{7.684482in}{0.550000in}}%
\pgfpathlineto{\pgfqpoint{7.684482in}{0.599453in}}%
\pgfpathlineto{\pgfqpoint{7.656960in}{0.599453in}}%
\pgfpathlineto{\pgfqpoint{7.656960in}{0.550000in}}%
\pgfpathclose%
\pgfusepath{fill}%
\end{pgfscope}%
\begin{pgfscope}%
\pgfpathrectangle{\pgfqpoint{1.250000in}{0.550000in}}{\pgfqpoint{7.750000in}{3.850000in}}%
\pgfusepath{clip}%
\pgfsetbuttcap%
\pgfsetmiterjoin%
\definecolor{currentfill}{rgb}{0.501961,0.501961,0.501961}%
\pgfsetfillcolor{currentfill}%
\pgfsetfillopacity{0.700000}%
\pgfsetlinewidth{0.000000pt}%
\definecolor{currentstroke}{rgb}{0.000000,0.000000,0.000000}%
\pgfsetstrokecolor{currentstroke}%
\pgfsetstrokeopacity{0.700000}%
\pgfsetdash{}{0pt}%
\pgfpathmoveto{\pgfqpoint{7.684482in}{0.550000in}}%
\pgfpathlineto{\pgfqpoint{7.712003in}{0.550000in}}%
\pgfpathlineto{\pgfqpoint{7.712003in}{0.602223in}}%
\pgfpathlineto{\pgfqpoint{7.684482in}{0.602223in}}%
\pgfpathlineto{\pgfqpoint{7.684482in}{0.550000in}}%
\pgfpathclose%
\pgfusepath{fill}%
\end{pgfscope}%
\begin{pgfscope}%
\pgfpathrectangle{\pgfqpoint{1.250000in}{0.550000in}}{\pgfqpoint{7.750000in}{3.850000in}}%
\pgfusepath{clip}%
\pgfsetbuttcap%
\pgfsetmiterjoin%
\definecolor{currentfill}{rgb}{0.501961,0.501961,0.501961}%
\pgfsetfillcolor{currentfill}%
\pgfsetfillopacity{0.700000}%
\pgfsetlinewidth{0.000000pt}%
\definecolor{currentstroke}{rgb}{0.000000,0.000000,0.000000}%
\pgfsetstrokecolor{currentstroke}%
\pgfsetstrokeopacity{0.700000}%
\pgfsetdash{}{0pt}%
\pgfpathmoveto{\pgfqpoint{7.712003in}{0.550000in}}%
\pgfpathlineto{\pgfqpoint{7.739524in}{0.550000in}}%
\pgfpathlineto{\pgfqpoint{7.739524in}{0.614883in}}%
\pgfpathlineto{\pgfqpoint{7.712003in}{0.614883in}}%
\pgfpathlineto{\pgfqpoint{7.712003in}{0.550000in}}%
\pgfpathclose%
\pgfusepath{fill}%
\end{pgfscope}%
\begin{pgfscope}%
\pgfpathrectangle{\pgfqpoint{1.250000in}{0.550000in}}{\pgfqpoint{7.750000in}{3.850000in}}%
\pgfusepath{clip}%
\pgfsetbuttcap%
\pgfsetmiterjoin%
\definecolor{currentfill}{rgb}{0.501961,0.501961,0.501961}%
\pgfsetfillcolor{currentfill}%
\pgfsetfillopacity{0.700000}%
\pgfsetlinewidth{0.000000pt}%
\definecolor{currentstroke}{rgb}{0.000000,0.000000,0.000000}%
\pgfsetstrokecolor{currentstroke}%
\pgfsetstrokeopacity{0.700000}%
\pgfsetdash{}{0pt}%
\pgfpathmoveto{\pgfqpoint{7.739524in}{0.550000in}}%
\pgfpathlineto{\pgfqpoint{7.767045in}{0.550000in}}%
\pgfpathlineto{\pgfqpoint{7.767045in}{0.550000in}}%
\pgfpathlineto{\pgfqpoint{7.739524in}{0.550000in}}%
\pgfpathlineto{\pgfqpoint{7.739524in}{0.550000in}}%
\pgfpathclose%
\pgfusepath{fill}%
\end{pgfscope}%
\begin{pgfscope}%
\pgfpathrectangle{\pgfqpoint{1.250000in}{0.550000in}}{\pgfqpoint{7.750000in}{3.850000in}}%
\pgfusepath{clip}%
\pgfsetbuttcap%
\pgfsetmiterjoin%
\definecolor{currentfill}{rgb}{0.501961,0.501961,0.501961}%
\pgfsetfillcolor{currentfill}%
\pgfsetfillopacity{0.700000}%
\pgfsetlinewidth{0.000000pt}%
\definecolor{currentstroke}{rgb}{0.000000,0.000000,0.000000}%
\pgfsetstrokecolor{currentstroke}%
\pgfsetstrokeopacity{0.700000}%
\pgfsetdash{}{0pt}%
\pgfpathmoveto{\pgfqpoint{7.767045in}{0.550000in}}%
\pgfpathlineto{\pgfqpoint{7.794567in}{0.550000in}}%
\pgfpathlineto{\pgfqpoint{7.794567in}{0.606970in}}%
\pgfpathlineto{\pgfqpoint{7.767045in}{0.606970in}}%
\pgfpathlineto{\pgfqpoint{7.767045in}{0.550000in}}%
\pgfpathclose%
\pgfusepath{fill}%
\end{pgfscope}%
\begin{pgfscope}%
\pgfpathrectangle{\pgfqpoint{1.250000in}{0.550000in}}{\pgfqpoint{7.750000in}{3.850000in}}%
\pgfusepath{clip}%
\pgfsetbuttcap%
\pgfsetmiterjoin%
\definecolor{currentfill}{rgb}{0.501961,0.501961,0.501961}%
\pgfsetfillcolor{currentfill}%
\pgfsetfillopacity{0.700000}%
\pgfsetlinewidth{0.000000pt}%
\definecolor{currentstroke}{rgb}{0.000000,0.000000,0.000000}%
\pgfsetstrokecolor{currentstroke}%
\pgfsetstrokeopacity{0.700000}%
\pgfsetdash{}{0pt}%
\pgfpathmoveto{\pgfqpoint{7.794567in}{0.550000in}}%
\pgfpathlineto{\pgfqpoint{7.822088in}{0.550000in}}%
\pgfpathlineto{\pgfqpoint{7.822088in}{0.599058in}}%
\pgfpathlineto{\pgfqpoint{7.794567in}{0.599058in}}%
\pgfpathlineto{\pgfqpoint{7.794567in}{0.550000in}}%
\pgfpathclose%
\pgfusepath{fill}%
\end{pgfscope}%
\begin{pgfscope}%
\pgfpathrectangle{\pgfqpoint{1.250000in}{0.550000in}}{\pgfqpoint{7.750000in}{3.850000in}}%
\pgfusepath{clip}%
\pgfsetbuttcap%
\pgfsetmiterjoin%
\definecolor{currentfill}{rgb}{0.501961,0.501961,0.501961}%
\pgfsetfillcolor{currentfill}%
\pgfsetfillopacity{0.700000}%
\pgfsetlinewidth{0.000000pt}%
\definecolor{currentstroke}{rgb}{0.000000,0.000000,0.000000}%
\pgfsetstrokecolor{currentstroke}%
\pgfsetstrokeopacity{0.700000}%
\pgfsetdash{}{0pt}%
\pgfpathmoveto{\pgfqpoint{7.822088in}{0.550000in}}%
\pgfpathlineto{\pgfqpoint{7.849609in}{0.550000in}}%
\pgfpathlineto{\pgfqpoint{7.849609in}{0.595893in}}%
\pgfpathlineto{\pgfqpoint{7.822088in}{0.595893in}}%
\pgfpathlineto{\pgfqpoint{7.822088in}{0.550000in}}%
\pgfpathclose%
\pgfusepath{fill}%
\end{pgfscope}%
\begin{pgfscope}%
\pgfpathrectangle{\pgfqpoint{1.250000in}{0.550000in}}{\pgfqpoint{7.750000in}{3.850000in}}%
\pgfusepath{clip}%
\pgfsetbuttcap%
\pgfsetmiterjoin%
\definecolor{currentfill}{rgb}{0.501961,0.501961,0.501961}%
\pgfsetfillcolor{currentfill}%
\pgfsetfillopacity{0.700000}%
\pgfsetlinewidth{0.000000pt}%
\definecolor{currentstroke}{rgb}{0.000000,0.000000,0.000000}%
\pgfsetstrokecolor{currentstroke}%
\pgfsetstrokeopacity{0.700000}%
\pgfsetdash{}{0pt}%
\pgfpathmoveto{\pgfqpoint{7.849609in}{0.550000in}}%
\pgfpathlineto{\pgfqpoint{7.877131in}{0.550000in}}%
\pgfpathlineto{\pgfqpoint{7.877131in}{0.592332in}}%
\pgfpathlineto{\pgfqpoint{7.849609in}{0.592332in}}%
\pgfpathlineto{\pgfqpoint{7.849609in}{0.550000in}}%
\pgfpathclose%
\pgfusepath{fill}%
\end{pgfscope}%
\begin{pgfscope}%
\pgfpathrectangle{\pgfqpoint{1.250000in}{0.550000in}}{\pgfqpoint{7.750000in}{3.850000in}}%
\pgfusepath{clip}%
\pgfsetbuttcap%
\pgfsetmiterjoin%
\definecolor{currentfill}{rgb}{0.501961,0.501961,0.501961}%
\pgfsetfillcolor{currentfill}%
\pgfsetfillopacity{0.700000}%
\pgfsetlinewidth{0.000000pt}%
\definecolor{currentstroke}{rgb}{0.000000,0.000000,0.000000}%
\pgfsetstrokecolor{currentstroke}%
\pgfsetstrokeopacity{0.700000}%
\pgfsetdash{}{0pt}%
\pgfpathmoveto{\pgfqpoint{7.877131in}{0.550000in}}%
\pgfpathlineto{\pgfqpoint{7.904652in}{0.550000in}}%
\pgfpathlineto{\pgfqpoint{7.904652in}{0.593519in}}%
\pgfpathlineto{\pgfqpoint{7.877131in}{0.593519in}}%
\pgfpathlineto{\pgfqpoint{7.877131in}{0.550000in}}%
\pgfpathclose%
\pgfusepath{fill}%
\end{pgfscope}%
\begin{pgfscope}%
\pgfpathrectangle{\pgfqpoint{1.250000in}{0.550000in}}{\pgfqpoint{7.750000in}{3.850000in}}%
\pgfusepath{clip}%
\pgfsetbuttcap%
\pgfsetmiterjoin%
\definecolor{currentfill}{rgb}{0.501961,0.501961,0.501961}%
\pgfsetfillcolor{currentfill}%
\pgfsetfillopacity{0.700000}%
\pgfsetlinewidth{0.000000pt}%
\definecolor{currentstroke}{rgb}{0.000000,0.000000,0.000000}%
\pgfsetstrokecolor{currentstroke}%
\pgfsetstrokeopacity{0.700000}%
\pgfsetdash{}{0pt}%
\pgfpathmoveto{\pgfqpoint{7.904652in}{0.550000in}}%
\pgfpathlineto{\pgfqpoint{7.932173in}{0.550000in}}%
\pgfpathlineto{\pgfqpoint{7.932173in}{0.588376in}}%
\pgfpathlineto{\pgfqpoint{7.904652in}{0.588376in}}%
\pgfpathlineto{\pgfqpoint{7.904652in}{0.550000in}}%
\pgfpathclose%
\pgfusepath{fill}%
\end{pgfscope}%
\begin{pgfscope}%
\pgfpathrectangle{\pgfqpoint{1.250000in}{0.550000in}}{\pgfqpoint{7.750000in}{3.850000in}}%
\pgfusepath{clip}%
\pgfsetbuttcap%
\pgfsetmiterjoin%
\definecolor{currentfill}{rgb}{0.501961,0.501961,0.501961}%
\pgfsetfillcolor{currentfill}%
\pgfsetfillopacity{0.700000}%
\pgfsetlinewidth{0.000000pt}%
\definecolor{currentstroke}{rgb}{0.000000,0.000000,0.000000}%
\pgfsetstrokecolor{currentstroke}%
\pgfsetstrokeopacity{0.700000}%
\pgfsetdash{}{0pt}%
\pgfpathmoveto{\pgfqpoint{7.932173in}{0.550000in}}%
\pgfpathlineto{\pgfqpoint{7.959695in}{0.550000in}}%
\pgfpathlineto{\pgfqpoint{7.959695in}{0.593519in}}%
\pgfpathlineto{\pgfqpoint{7.932173in}{0.593519in}}%
\pgfpathlineto{\pgfqpoint{7.932173in}{0.550000in}}%
\pgfpathclose%
\pgfusepath{fill}%
\end{pgfscope}%
\begin{pgfscope}%
\pgfpathrectangle{\pgfqpoint{1.250000in}{0.550000in}}{\pgfqpoint{7.750000in}{3.850000in}}%
\pgfusepath{clip}%
\pgfsetbuttcap%
\pgfsetmiterjoin%
\definecolor{currentfill}{rgb}{0.501961,0.501961,0.501961}%
\pgfsetfillcolor{currentfill}%
\pgfsetfillopacity{0.700000}%
\pgfsetlinewidth{0.000000pt}%
\definecolor{currentstroke}{rgb}{0.000000,0.000000,0.000000}%
\pgfsetstrokecolor{currentstroke}%
\pgfsetstrokeopacity{0.700000}%
\pgfsetdash{}{0pt}%
\pgfpathmoveto{\pgfqpoint{7.959695in}{0.550000in}}%
\pgfpathlineto{\pgfqpoint{7.987216in}{0.550000in}}%
\pgfpathlineto{\pgfqpoint{7.987216in}{0.597080in}}%
\pgfpathlineto{\pgfqpoint{7.959695in}{0.597080in}}%
\pgfpathlineto{\pgfqpoint{7.959695in}{0.550000in}}%
\pgfpathclose%
\pgfusepath{fill}%
\end{pgfscope}%
\begin{pgfscope}%
\pgfpathrectangle{\pgfqpoint{1.250000in}{0.550000in}}{\pgfqpoint{7.750000in}{3.850000in}}%
\pgfusepath{clip}%
\pgfsetbuttcap%
\pgfsetmiterjoin%
\definecolor{currentfill}{rgb}{0.501961,0.501961,0.501961}%
\pgfsetfillcolor{currentfill}%
\pgfsetfillopacity{0.700000}%
\pgfsetlinewidth{0.000000pt}%
\definecolor{currentstroke}{rgb}{0.000000,0.000000,0.000000}%
\pgfsetstrokecolor{currentstroke}%
\pgfsetstrokeopacity{0.700000}%
\pgfsetdash{}{0pt}%
\pgfpathmoveto{\pgfqpoint{7.987216in}{0.550000in}}%
\pgfpathlineto{\pgfqpoint{8.014737in}{0.550000in}}%
\pgfpathlineto{\pgfqpoint{8.014737in}{0.550000in}}%
\pgfpathlineto{\pgfqpoint{7.987216in}{0.550000in}}%
\pgfpathlineto{\pgfqpoint{7.987216in}{0.550000in}}%
\pgfpathclose%
\pgfusepath{fill}%
\end{pgfscope}%
\begin{pgfscope}%
\pgfpathrectangle{\pgfqpoint{1.250000in}{0.550000in}}{\pgfqpoint{7.750000in}{3.850000in}}%
\pgfusepath{clip}%
\pgfsetbuttcap%
\pgfsetmiterjoin%
\definecolor{currentfill}{rgb}{0.501961,0.501961,0.501961}%
\pgfsetfillcolor{currentfill}%
\pgfsetfillopacity{0.700000}%
\pgfsetlinewidth{0.000000pt}%
\definecolor{currentstroke}{rgb}{0.000000,0.000000,0.000000}%
\pgfsetstrokecolor{currentstroke}%
\pgfsetstrokeopacity{0.700000}%
\pgfsetdash{}{0pt}%
\pgfpathmoveto{\pgfqpoint{8.014737in}{0.550000in}}%
\pgfpathlineto{\pgfqpoint{8.042259in}{0.550000in}}%
\pgfpathlineto{\pgfqpoint{8.042259in}{0.597871in}}%
\pgfpathlineto{\pgfqpoint{8.014737in}{0.597871in}}%
\pgfpathlineto{\pgfqpoint{8.014737in}{0.550000in}}%
\pgfpathclose%
\pgfusepath{fill}%
\end{pgfscope}%
\begin{pgfscope}%
\pgfpathrectangle{\pgfqpoint{1.250000in}{0.550000in}}{\pgfqpoint{7.750000in}{3.850000in}}%
\pgfusepath{clip}%
\pgfsetbuttcap%
\pgfsetmiterjoin%
\definecolor{currentfill}{rgb}{0.501961,0.501961,0.501961}%
\pgfsetfillcolor{currentfill}%
\pgfsetfillopacity{0.700000}%
\pgfsetlinewidth{0.000000pt}%
\definecolor{currentstroke}{rgb}{0.000000,0.000000,0.000000}%
\pgfsetstrokecolor{currentstroke}%
\pgfsetstrokeopacity{0.700000}%
\pgfsetdash{}{0pt}%
\pgfpathmoveto{\pgfqpoint{8.042259in}{0.550000in}}%
\pgfpathlineto{\pgfqpoint{8.069780in}{0.550000in}}%
\pgfpathlineto{\pgfqpoint{8.069780in}{0.594706in}}%
\pgfpathlineto{\pgfqpoint{8.042259in}{0.594706in}}%
\pgfpathlineto{\pgfqpoint{8.042259in}{0.550000in}}%
\pgfpathclose%
\pgfusepath{fill}%
\end{pgfscope}%
\begin{pgfscope}%
\pgfpathrectangle{\pgfqpoint{1.250000in}{0.550000in}}{\pgfqpoint{7.750000in}{3.850000in}}%
\pgfusepath{clip}%
\pgfsetbuttcap%
\pgfsetmiterjoin%
\definecolor{currentfill}{rgb}{0.501961,0.501961,0.501961}%
\pgfsetfillcolor{currentfill}%
\pgfsetfillopacity{0.700000}%
\pgfsetlinewidth{0.000000pt}%
\definecolor{currentstroke}{rgb}{0.000000,0.000000,0.000000}%
\pgfsetstrokecolor{currentstroke}%
\pgfsetstrokeopacity{0.700000}%
\pgfsetdash{}{0pt}%
\pgfpathmoveto{\pgfqpoint{8.069780in}{0.550000in}}%
\pgfpathlineto{\pgfqpoint{8.097301in}{0.550000in}}%
\pgfpathlineto{\pgfqpoint{8.097301in}{0.589958in}}%
\pgfpathlineto{\pgfqpoint{8.069780in}{0.589958in}}%
\pgfpathlineto{\pgfqpoint{8.069780in}{0.550000in}}%
\pgfpathclose%
\pgfusepath{fill}%
\end{pgfscope}%
\begin{pgfscope}%
\pgfpathrectangle{\pgfqpoint{1.250000in}{0.550000in}}{\pgfqpoint{7.750000in}{3.850000in}}%
\pgfusepath{clip}%
\pgfsetbuttcap%
\pgfsetmiterjoin%
\definecolor{currentfill}{rgb}{0.501961,0.501961,0.501961}%
\pgfsetfillcolor{currentfill}%
\pgfsetfillopacity{0.700000}%
\pgfsetlinewidth{0.000000pt}%
\definecolor{currentstroke}{rgb}{0.000000,0.000000,0.000000}%
\pgfsetstrokecolor{currentstroke}%
\pgfsetstrokeopacity{0.700000}%
\pgfsetdash{}{0pt}%
\pgfpathmoveto{\pgfqpoint{8.097301in}{0.550000in}}%
\pgfpathlineto{\pgfqpoint{8.124822in}{0.550000in}}%
\pgfpathlineto{\pgfqpoint{8.124822in}{0.597475in}}%
\pgfpathlineto{\pgfqpoint{8.097301in}{0.597475in}}%
\pgfpathlineto{\pgfqpoint{8.097301in}{0.550000in}}%
\pgfpathclose%
\pgfusepath{fill}%
\end{pgfscope}%
\begin{pgfscope}%
\pgfpathrectangle{\pgfqpoint{1.250000in}{0.550000in}}{\pgfqpoint{7.750000in}{3.850000in}}%
\pgfusepath{clip}%
\pgfsetbuttcap%
\pgfsetmiterjoin%
\definecolor{currentfill}{rgb}{0.501961,0.501961,0.501961}%
\pgfsetfillcolor{currentfill}%
\pgfsetfillopacity{0.700000}%
\pgfsetlinewidth{0.000000pt}%
\definecolor{currentstroke}{rgb}{0.000000,0.000000,0.000000}%
\pgfsetstrokecolor{currentstroke}%
\pgfsetstrokeopacity{0.700000}%
\pgfsetdash{}{0pt}%
\pgfpathmoveto{\pgfqpoint{8.124822in}{0.550000in}}%
\pgfpathlineto{\pgfqpoint{8.152344in}{0.550000in}}%
\pgfpathlineto{\pgfqpoint{8.152344in}{0.589563in}}%
\pgfpathlineto{\pgfqpoint{8.124822in}{0.589563in}}%
\pgfpathlineto{\pgfqpoint{8.124822in}{0.550000in}}%
\pgfpathclose%
\pgfusepath{fill}%
\end{pgfscope}%
\begin{pgfscope}%
\pgfpathrectangle{\pgfqpoint{1.250000in}{0.550000in}}{\pgfqpoint{7.750000in}{3.850000in}}%
\pgfusepath{clip}%
\pgfsetbuttcap%
\pgfsetmiterjoin%
\definecolor{currentfill}{rgb}{0.501961,0.501961,0.501961}%
\pgfsetfillcolor{currentfill}%
\pgfsetfillopacity{0.700000}%
\pgfsetlinewidth{0.000000pt}%
\definecolor{currentstroke}{rgb}{0.000000,0.000000,0.000000}%
\pgfsetstrokecolor{currentstroke}%
\pgfsetstrokeopacity{0.700000}%
\pgfsetdash{}{0pt}%
\pgfpathmoveto{\pgfqpoint{8.152344in}{0.550000in}}%
\pgfpathlineto{\pgfqpoint{8.179865in}{0.550000in}}%
\pgfpathlineto{\pgfqpoint{8.179865in}{0.590750in}}%
\pgfpathlineto{\pgfqpoint{8.152344in}{0.590750in}}%
\pgfpathlineto{\pgfqpoint{8.152344in}{0.550000in}}%
\pgfpathclose%
\pgfusepath{fill}%
\end{pgfscope}%
\begin{pgfscope}%
\pgfpathrectangle{\pgfqpoint{1.250000in}{0.550000in}}{\pgfqpoint{7.750000in}{3.850000in}}%
\pgfusepath{clip}%
\pgfsetbuttcap%
\pgfsetmiterjoin%
\definecolor{currentfill}{rgb}{0.501961,0.501961,0.501961}%
\pgfsetfillcolor{currentfill}%
\pgfsetfillopacity{0.700000}%
\pgfsetlinewidth{0.000000pt}%
\definecolor{currentstroke}{rgb}{0.000000,0.000000,0.000000}%
\pgfsetstrokecolor{currentstroke}%
\pgfsetstrokeopacity{0.700000}%
\pgfsetdash{}{0pt}%
\pgfpathmoveto{\pgfqpoint{8.179865in}{0.550000in}}%
\pgfpathlineto{\pgfqpoint{8.207386in}{0.550000in}}%
\pgfpathlineto{\pgfqpoint{8.207386in}{0.582046in}}%
\pgfpathlineto{\pgfqpoint{8.179865in}{0.582046in}}%
\pgfpathlineto{\pgfqpoint{8.179865in}{0.550000in}}%
\pgfpathclose%
\pgfusepath{fill}%
\end{pgfscope}%
\begin{pgfscope}%
\pgfpathrectangle{\pgfqpoint{1.250000in}{0.550000in}}{\pgfqpoint{7.750000in}{3.850000in}}%
\pgfusepath{clip}%
\pgfsetbuttcap%
\pgfsetmiterjoin%
\definecolor{currentfill}{rgb}{0.501961,0.501961,0.501961}%
\pgfsetfillcolor{currentfill}%
\pgfsetfillopacity{0.700000}%
\pgfsetlinewidth{0.000000pt}%
\definecolor{currentstroke}{rgb}{0.000000,0.000000,0.000000}%
\pgfsetstrokecolor{currentstroke}%
\pgfsetstrokeopacity{0.700000}%
\pgfsetdash{}{0pt}%
\pgfpathmoveto{\pgfqpoint{8.207386in}{0.550000in}}%
\pgfpathlineto{\pgfqpoint{8.234908in}{0.550000in}}%
\pgfpathlineto{\pgfqpoint{8.234908in}{0.585606in}}%
\pgfpathlineto{\pgfqpoint{8.207386in}{0.585606in}}%
\pgfpathlineto{\pgfqpoint{8.207386in}{0.550000in}}%
\pgfpathclose%
\pgfusepath{fill}%
\end{pgfscope}%
\begin{pgfscope}%
\pgfpathrectangle{\pgfqpoint{1.250000in}{0.550000in}}{\pgfqpoint{7.750000in}{3.850000in}}%
\pgfusepath{clip}%
\pgfsetbuttcap%
\pgfsetmiterjoin%
\definecolor{currentfill}{rgb}{0.501961,0.501961,0.501961}%
\pgfsetfillcolor{currentfill}%
\pgfsetfillopacity{0.700000}%
\pgfsetlinewidth{0.000000pt}%
\definecolor{currentstroke}{rgb}{0.000000,0.000000,0.000000}%
\pgfsetstrokecolor{currentstroke}%
\pgfsetstrokeopacity{0.700000}%
\pgfsetdash{}{0pt}%
\pgfpathmoveto{\pgfqpoint{8.234908in}{0.550000in}}%
\pgfpathlineto{\pgfqpoint{8.262429in}{0.550000in}}%
\pgfpathlineto{\pgfqpoint{8.262429in}{0.605388in}}%
\pgfpathlineto{\pgfqpoint{8.234908in}{0.605388in}}%
\pgfpathlineto{\pgfqpoint{8.234908in}{0.550000in}}%
\pgfpathclose%
\pgfusepath{fill}%
\end{pgfscope}%
\begin{pgfscope}%
\pgfpathrectangle{\pgfqpoint{1.250000in}{0.550000in}}{\pgfqpoint{7.750000in}{3.850000in}}%
\pgfusepath{clip}%
\pgfsetbuttcap%
\pgfsetmiterjoin%
\definecolor{currentfill}{rgb}{0.501961,0.501961,0.501961}%
\pgfsetfillcolor{currentfill}%
\pgfsetfillopacity{0.700000}%
\pgfsetlinewidth{0.000000pt}%
\definecolor{currentstroke}{rgb}{0.000000,0.000000,0.000000}%
\pgfsetstrokecolor{currentstroke}%
\pgfsetstrokeopacity{0.700000}%
\pgfsetdash{}{0pt}%
\pgfpathmoveto{\pgfqpoint{8.262429in}{0.550000in}}%
\pgfpathlineto{\pgfqpoint{8.289950in}{0.550000in}}%
\pgfpathlineto{\pgfqpoint{8.289950in}{0.581650in}}%
\pgfpathlineto{\pgfqpoint{8.262429in}{0.581650in}}%
\pgfpathlineto{\pgfqpoint{8.262429in}{0.550000in}}%
\pgfpathclose%
\pgfusepath{fill}%
\end{pgfscope}%
\begin{pgfscope}%
\pgfpathrectangle{\pgfqpoint{1.250000in}{0.550000in}}{\pgfqpoint{7.750000in}{3.850000in}}%
\pgfusepath{clip}%
\pgfsetbuttcap%
\pgfsetmiterjoin%
\definecolor{currentfill}{rgb}{0.501961,0.501961,0.501961}%
\pgfsetfillcolor{currentfill}%
\pgfsetfillopacity{0.700000}%
\pgfsetlinewidth{0.000000pt}%
\definecolor{currentstroke}{rgb}{0.000000,0.000000,0.000000}%
\pgfsetstrokecolor{currentstroke}%
\pgfsetstrokeopacity{0.700000}%
\pgfsetdash{}{0pt}%
\pgfpathmoveto{\pgfqpoint{8.289950in}{0.550000in}}%
\pgfpathlineto{\pgfqpoint{8.317472in}{0.550000in}}%
\pgfpathlineto{\pgfqpoint{8.317472in}{0.579276in}}%
\pgfpathlineto{\pgfqpoint{8.289950in}{0.579276in}}%
\pgfpathlineto{\pgfqpoint{8.289950in}{0.550000in}}%
\pgfpathclose%
\pgfusepath{fill}%
\end{pgfscope}%
\begin{pgfscope}%
\pgfpathrectangle{\pgfqpoint{1.250000in}{0.550000in}}{\pgfqpoint{7.750000in}{3.850000in}}%
\pgfusepath{clip}%
\pgfsetbuttcap%
\pgfsetmiterjoin%
\definecolor{currentfill}{rgb}{0.501961,0.501961,0.501961}%
\pgfsetfillcolor{currentfill}%
\pgfsetfillopacity{0.700000}%
\pgfsetlinewidth{0.000000pt}%
\definecolor{currentstroke}{rgb}{0.000000,0.000000,0.000000}%
\pgfsetstrokecolor{currentstroke}%
\pgfsetstrokeopacity{0.700000}%
\pgfsetdash{}{0pt}%
\pgfpathmoveto{\pgfqpoint{8.317472in}{0.550000in}}%
\pgfpathlineto{\pgfqpoint{8.344993in}{0.550000in}}%
\pgfpathlineto{\pgfqpoint{8.344993in}{0.597871in}}%
\pgfpathlineto{\pgfqpoint{8.317472in}{0.597871in}}%
\pgfpathlineto{\pgfqpoint{8.317472in}{0.550000in}}%
\pgfpathclose%
\pgfusepath{fill}%
\end{pgfscope}%
\begin{pgfscope}%
\pgfpathrectangle{\pgfqpoint{1.250000in}{0.550000in}}{\pgfqpoint{7.750000in}{3.850000in}}%
\pgfusepath{clip}%
\pgfsetbuttcap%
\pgfsetmiterjoin%
\definecolor{currentfill}{rgb}{0.501961,0.501961,0.501961}%
\pgfsetfillcolor{currentfill}%
\pgfsetfillopacity{0.700000}%
\pgfsetlinewidth{0.000000pt}%
\definecolor{currentstroke}{rgb}{0.000000,0.000000,0.000000}%
\pgfsetstrokecolor{currentstroke}%
\pgfsetstrokeopacity{0.700000}%
\pgfsetdash{}{0pt}%
\pgfpathmoveto{\pgfqpoint{8.344993in}{0.550000in}}%
\pgfpathlineto{\pgfqpoint{8.372514in}{0.550000in}}%
\pgfpathlineto{\pgfqpoint{8.372514in}{0.590354in}}%
\pgfpathlineto{\pgfqpoint{8.344993in}{0.590354in}}%
\pgfpathlineto{\pgfqpoint{8.344993in}{0.550000in}}%
\pgfpathclose%
\pgfusepath{fill}%
\end{pgfscope}%
\begin{pgfscope}%
\pgfpathrectangle{\pgfqpoint{1.250000in}{0.550000in}}{\pgfqpoint{7.750000in}{3.850000in}}%
\pgfusepath{clip}%
\pgfsetbuttcap%
\pgfsetmiterjoin%
\definecolor{currentfill}{rgb}{0.501961,0.501961,0.501961}%
\pgfsetfillcolor{currentfill}%
\pgfsetfillopacity{0.700000}%
\pgfsetlinewidth{0.000000pt}%
\definecolor{currentstroke}{rgb}{0.000000,0.000000,0.000000}%
\pgfsetstrokecolor{currentstroke}%
\pgfsetstrokeopacity{0.700000}%
\pgfsetdash{}{0pt}%
\pgfpathmoveto{\pgfqpoint{8.372514in}{0.550000in}}%
\pgfpathlineto{\pgfqpoint{8.400036in}{0.550000in}}%
\pgfpathlineto{\pgfqpoint{8.400036in}{0.586398in}}%
\pgfpathlineto{\pgfqpoint{8.372514in}{0.586398in}}%
\pgfpathlineto{\pgfqpoint{8.372514in}{0.550000in}}%
\pgfpathclose%
\pgfusepath{fill}%
\end{pgfscope}%
\begin{pgfscope}%
\pgfpathrectangle{\pgfqpoint{1.250000in}{0.550000in}}{\pgfqpoint{7.750000in}{3.850000in}}%
\pgfusepath{clip}%
\pgfsetbuttcap%
\pgfsetmiterjoin%
\definecolor{currentfill}{rgb}{0.501961,0.501961,0.501961}%
\pgfsetfillcolor{currentfill}%
\pgfsetfillopacity{0.700000}%
\pgfsetlinewidth{0.000000pt}%
\definecolor{currentstroke}{rgb}{0.000000,0.000000,0.000000}%
\pgfsetstrokecolor{currentstroke}%
\pgfsetstrokeopacity{0.700000}%
\pgfsetdash{}{0pt}%
\pgfpathmoveto{\pgfqpoint{8.400036in}{0.550000in}}%
\pgfpathlineto{\pgfqpoint{8.427557in}{0.550000in}}%
\pgfpathlineto{\pgfqpoint{8.427557in}{0.595497in}}%
\pgfpathlineto{\pgfqpoint{8.400036in}{0.595497in}}%
\pgfpathlineto{\pgfqpoint{8.400036in}{0.550000in}}%
\pgfpathclose%
\pgfusepath{fill}%
\end{pgfscope}%
\begin{pgfscope}%
\pgfpathrectangle{\pgfqpoint{1.250000in}{0.550000in}}{\pgfqpoint{7.750000in}{3.850000in}}%
\pgfusepath{clip}%
\pgfsetbuttcap%
\pgfsetmiterjoin%
\definecolor{currentfill}{rgb}{0.501961,0.501961,0.501961}%
\pgfsetfillcolor{currentfill}%
\pgfsetfillopacity{0.700000}%
\pgfsetlinewidth{0.000000pt}%
\definecolor{currentstroke}{rgb}{0.000000,0.000000,0.000000}%
\pgfsetstrokecolor{currentstroke}%
\pgfsetstrokeopacity{0.700000}%
\pgfsetdash{}{0pt}%
\pgfpathmoveto{\pgfqpoint{8.427557in}{0.550000in}}%
\pgfpathlineto{\pgfqpoint{8.455078in}{0.550000in}}%
\pgfpathlineto{\pgfqpoint{8.455078in}{0.586398in}}%
\pgfpathlineto{\pgfqpoint{8.427557in}{0.586398in}}%
\pgfpathlineto{\pgfqpoint{8.427557in}{0.550000in}}%
\pgfpathclose%
\pgfusepath{fill}%
\end{pgfscope}%
\begin{pgfscope}%
\pgfpathrectangle{\pgfqpoint{1.250000in}{0.550000in}}{\pgfqpoint{7.750000in}{3.850000in}}%
\pgfusepath{clip}%
\pgfsetbuttcap%
\pgfsetmiterjoin%
\definecolor{currentfill}{rgb}{0.501961,0.501961,0.501961}%
\pgfsetfillcolor{currentfill}%
\pgfsetfillopacity{0.700000}%
\pgfsetlinewidth{0.000000pt}%
\definecolor{currentstroke}{rgb}{0.000000,0.000000,0.000000}%
\pgfsetstrokecolor{currentstroke}%
\pgfsetstrokeopacity{0.700000}%
\pgfsetdash{}{0pt}%
\pgfpathmoveto{\pgfqpoint{8.455078in}{0.550000in}}%
\pgfpathlineto{\pgfqpoint{8.482599in}{0.550000in}}%
\pgfpathlineto{\pgfqpoint{8.482599in}{0.593519in}}%
\pgfpathlineto{\pgfqpoint{8.455078in}{0.593519in}}%
\pgfpathlineto{\pgfqpoint{8.455078in}{0.550000in}}%
\pgfpathclose%
\pgfusepath{fill}%
\end{pgfscope}%
\begin{pgfscope}%
\pgfpathrectangle{\pgfqpoint{1.250000in}{0.550000in}}{\pgfqpoint{7.750000in}{3.850000in}}%
\pgfusepath{clip}%
\pgfsetbuttcap%
\pgfsetmiterjoin%
\definecolor{currentfill}{rgb}{0.501961,0.501961,0.501961}%
\pgfsetfillcolor{currentfill}%
\pgfsetfillopacity{0.700000}%
\pgfsetlinewidth{0.000000pt}%
\definecolor{currentstroke}{rgb}{0.000000,0.000000,0.000000}%
\pgfsetstrokecolor{currentstroke}%
\pgfsetstrokeopacity{0.700000}%
\pgfsetdash{}{0pt}%
\pgfpathmoveto{\pgfqpoint{8.482599in}{0.550000in}}%
\pgfpathlineto{\pgfqpoint{8.510121in}{0.550000in}}%
\pgfpathlineto{\pgfqpoint{8.510121in}{0.595497in}}%
\pgfpathlineto{\pgfqpoint{8.482599in}{0.595497in}}%
\pgfpathlineto{\pgfqpoint{8.482599in}{0.550000in}}%
\pgfpathclose%
\pgfusepath{fill}%
\end{pgfscope}%
\begin{pgfscope}%
\pgfpathrectangle{\pgfqpoint{1.250000in}{0.550000in}}{\pgfqpoint{7.750000in}{3.850000in}}%
\pgfusepath{clip}%
\pgfsetbuttcap%
\pgfsetmiterjoin%
\definecolor{currentfill}{rgb}{0.501961,0.501961,0.501961}%
\pgfsetfillcolor{currentfill}%
\pgfsetfillopacity{0.700000}%
\pgfsetlinewidth{0.000000pt}%
\definecolor{currentstroke}{rgb}{0.000000,0.000000,0.000000}%
\pgfsetstrokecolor{currentstroke}%
\pgfsetstrokeopacity{0.700000}%
\pgfsetdash{}{0pt}%
\pgfpathmoveto{\pgfqpoint{8.510121in}{0.550000in}}%
\pgfpathlineto{\pgfqpoint{8.537642in}{0.550000in}}%
\pgfpathlineto{\pgfqpoint{8.537642in}{0.584420in}}%
\pgfpathlineto{\pgfqpoint{8.510121in}{0.584420in}}%
\pgfpathlineto{\pgfqpoint{8.510121in}{0.550000in}}%
\pgfpathclose%
\pgfusepath{fill}%
\end{pgfscope}%
\begin{pgfscope}%
\pgfpathrectangle{\pgfqpoint{1.250000in}{0.550000in}}{\pgfqpoint{7.750000in}{3.850000in}}%
\pgfusepath{clip}%
\pgfsetbuttcap%
\pgfsetmiterjoin%
\definecolor{currentfill}{rgb}{0.501961,0.501961,0.501961}%
\pgfsetfillcolor{currentfill}%
\pgfsetfillopacity{0.700000}%
\pgfsetlinewidth{0.000000pt}%
\definecolor{currentstroke}{rgb}{0.000000,0.000000,0.000000}%
\pgfsetstrokecolor{currentstroke}%
\pgfsetstrokeopacity{0.700000}%
\pgfsetdash{}{0pt}%
\pgfpathmoveto{\pgfqpoint{8.537642in}{0.550000in}}%
\pgfpathlineto{\pgfqpoint{8.565163in}{0.550000in}}%
\pgfpathlineto{\pgfqpoint{8.565163in}{0.586398in}}%
\pgfpathlineto{\pgfqpoint{8.537642in}{0.586398in}}%
\pgfpathlineto{\pgfqpoint{8.537642in}{0.550000in}}%
\pgfpathclose%
\pgfusepath{fill}%
\end{pgfscope}%
\begin{pgfscope}%
\pgfpathrectangle{\pgfqpoint{1.250000in}{0.550000in}}{\pgfqpoint{7.750000in}{3.850000in}}%
\pgfusepath{clip}%
\pgfsetbuttcap%
\pgfsetmiterjoin%
\definecolor{currentfill}{rgb}{0.501961,0.501961,0.501961}%
\pgfsetfillcolor{currentfill}%
\pgfsetfillopacity{0.700000}%
\pgfsetlinewidth{0.000000pt}%
\definecolor{currentstroke}{rgb}{0.000000,0.000000,0.000000}%
\pgfsetstrokecolor{currentstroke}%
\pgfsetstrokeopacity{0.700000}%
\pgfsetdash{}{0pt}%
\pgfpathmoveto{\pgfqpoint{8.565163in}{0.550000in}}%
\pgfpathlineto{\pgfqpoint{8.592685in}{0.550000in}}%
\pgfpathlineto{\pgfqpoint{8.592685in}{0.589958in}}%
\pgfpathlineto{\pgfqpoint{8.565163in}{0.589958in}}%
\pgfpathlineto{\pgfqpoint{8.565163in}{0.550000in}}%
\pgfpathclose%
\pgfusepath{fill}%
\end{pgfscope}%
\begin{pgfscope}%
\pgfpathrectangle{\pgfqpoint{1.250000in}{0.550000in}}{\pgfqpoint{7.750000in}{3.850000in}}%
\pgfusepath{clip}%
\pgfsetbuttcap%
\pgfsetmiterjoin%
\definecolor{currentfill}{rgb}{0.501961,0.501961,0.501961}%
\pgfsetfillcolor{currentfill}%
\pgfsetfillopacity{0.700000}%
\pgfsetlinewidth{0.000000pt}%
\definecolor{currentstroke}{rgb}{0.000000,0.000000,0.000000}%
\pgfsetstrokecolor{currentstroke}%
\pgfsetstrokeopacity{0.700000}%
\pgfsetdash{}{0pt}%
\pgfpathmoveto{\pgfqpoint{8.592685in}{0.550000in}}%
\pgfpathlineto{\pgfqpoint{8.620206in}{0.550000in}}%
\pgfpathlineto{\pgfqpoint{8.620206in}{0.592332in}}%
\pgfpathlineto{\pgfqpoint{8.592685in}{0.592332in}}%
\pgfpathlineto{\pgfqpoint{8.592685in}{0.550000in}}%
\pgfpathclose%
\pgfusepath{fill}%
\end{pgfscope}%
\begin{pgfscope}%
\pgfpathrectangle{\pgfqpoint{1.250000in}{0.550000in}}{\pgfqpoint{7.750000in}{3.850000in}}%
\pgfusepath{clip}%
\pgfsetbuttcap%
\pgfsetmiterjoin%
\definecolor{currentfill}{rgb}{0.501961,0.501961,0.501961}%
\pgfsetfillcolor{currentfill}%
\pgfsetfillopacity{0.700000}%
\pgfsetlinewidth{0.000000pt}%
\definecolor{currentstroke}{rgb}{0.000000,0.000000,0.000000}%
\pgfsetstrokecolor{currentstroke}%
\pgfsetstrokeopacity{0.700000}%
\pgfsetdash{}{0pt}%
\pgfpathmoveto{\pgfqpoint{8.620206in}{0.550000in}}%
\pgfpathlineto{\pgfqpoint{8.647727in}{0.550000in}}%
\pgfpathlineto{\pgfqpoint{8.647727in}{0.573342in}}%
\pgfpathlineto{\pgfqpoint{8.620206in}{0.573342in}}%
\pgfpathlineto{\pgfqpoint{8.620206in}{0.550000in}}%
\pgfpathclose%
\pgfusepath{fill}%
\end{pgfscope}%
\begin{pgfscope}%
\pgfsetbuttcap%
\pgfsetroundjoin%
\definecolor{currentfill}{rgb}{0.000000,0.000000,0.000000}%
\pgfsetfillcolor{currentfill}%
\pgfsetlinewidth{0.803000pt}%
\definecolor{currentstroke}{rgb}{0.000000,0.000000,0.000000}%
\pgfsetstrokecolor{currentstroke}%
\pgfsetdash{}{0pt}%
\pgfsys@defobject{currentmarker}{\pgfqpoint{0.000000in}{-0.048611in}}{\pgfqpoint{0.000000in}{0.000000in}}{%
\pgfpathmoveto{\pgfqpoint{0.000000in}{0.000000in}}%
\pgfpathlineto{\pgfqpoint{0.000000in}{-0.048611in}}%
\pgfusepath{stroke,fill}%
}%
\begin{pgfscope}%
\pgfsys@transformshift{1.602273in}{0.550000in}%
\pgfsys@useobject{currentmarker}{}%
\end{pgfscope}%
\end{pgfscope}%
\begin{pgfscope}%
\definecolor{textcolor}{rgb}{0.000000,0.000000,0.000000}%
\pgfsetstrokecolor{textcolor}%
\pgfsetfillcolor{textcolor}%
\pgftext[x=1.602273in,y=0.452778in,,top]{\color{textcolor}{\sffamily\fontsize{10.000000}{12.000000}\selectfont\catcode`\^=\active\def^{\ifmmode\sp\else\^{}\fi}\catcode`\%=\active\def%{\%}0}}%
\end{pgfscope}%
\begin{pgfscope}%
\pgfsetbuttcap%
\pgfsetroundjoin%
\definecolor{currentfill}{rgb}{0.000000,0.000000,0.000000}%
\pgfsetfillcolor{currentfill}%
\pgfsetlinewidth{0.803000pt}%
\definecolor{currentstroke}{rgb}{0.000000,0.000000,0.000000}%
\pgfsetstrokecolor{currentstroke}%
\pgfsetdash{}{0pt}%
\pgfsys@defobject{currentmarker}{\pgfqpoint{0.000000in}{-0.048611in}}{\pgfqpoint{0.000000in}{0.000000in}}{%
\pgfpathmoveto{\pgfqpoint{0.000000in}{0.000000in}}%
\pgfpathlineto{\pgfqpoint{0.000000in}{-0.048611in}}%
\pgfusepath{stroke,fill}%
}%
\begin{pgfscope}%
\pgfsys@transformshift{2.978338in}{0.550000in}%
\pgfsys@useobject{currentmarker}{}%
\end{pgfscope}%
\end{pgfscope}%
\begin{pgfscope}%
\definecolor{textcolor}{rgb}{0.000000,0.000000,0.000000}%
\pgfsetstrokecolor{textcolor}%
\pgfsetfillcolor{textcolor}%
\pgftext[x=2.978338in,y=0.452778in,,top]{\color{textcolor}{\sffamily\fontsize{10.000000}{12.000000}\selectfont\catcode`\^=\active\def^{\ifmmode\sp\else\^{}\fi}\catcode`\%=\active\def%{\%}50}}%
\end{pgfscope}%
\begin{pgfscope}%
\pgfsetbuttcap%
\pgfsetroundjoin%
\definecolor{currentfill}{rgb}{0.000000,0.000000,0.000000}%
\pgfsetfillcolor{currentfill}%
\pgfsetlinewidth{0.803000pt}%
\definecolor{currentstroke}{rgb}{0.000000,0.000000,0.000000}%
\pgfsetstrokecolor{currentstroke}%
\pgfsetdash{}{0pt}%
\pgfsys@defobject{currentmarker}{\pgfqpoint{0.000000in}{-0.048611in}}{\pgfqpoint{0.000000in}{0.000000in}}{%
\pgfpathmoveto{\pgfqpoint{0.000000in}{0.000000in}}%
\pgfpathlineto{\pgfqpoint{0.000000in}{-0.048611in}}%
\pgfusepath{stroke,fill}%
}%
\begin{pgfscope}%
\pgfsys@transformshift{4.354403in}{0.550000in}%
\pgfsys@useobject{currentmarker}{}%
\end{pgfscope}%
\end{pgfscope}%
\begin{pgfscope}%
\definecolor{textcolor}{rgb}{0.000000,0.000000,0.000000}%
\pgfsetstrokecolor{textcolor}%
\pgfsetfillcolor{textcolor}%
\pgftext[x=4.354403in,y=0.452778in,,top]{\color{textcolor}{\sffamily\fontsize{10.000000}{12.000000}\selectfont\catcode`\^=\active\def^{\ifmmode\sp\else\^{}\fi}\catcode`\%=\active\def%{\%}100}}%
\end{pgfscope}%
\begin{pgfscope}%
\pgfsetbuttcap%
\pgfsetroundjoin%
\definecolor{currentfill}{rgb}{0.000000,0.000000,0.000000}%
\pgfsetfillcolor{currentfill}%
\pgfsetlinewidth{0.803000pt}%
\definecolor{currentstroke}{rgb}{0.000000,0.000000,0.000000}%
\pgfsetstrokecolor{currentstroke}%
\pgfsetdash{}{0pt}%
\pgfsys@defobject{currentmarker}{\pgfqpoint{0.000000in}{-0.048611in}}{\pgfqpoint{0.000000in}{0.000000in}}{%
\pgfpathmoveto{\pgfqpoint{0.000000in}{0.000000in}}%
\pgfpathlineto{\pgfqpoint{0.000000in}{-0.048611in}}%
\pgfusepath{stroke,fill}%
}%
\begin{pgfscope}%
\pgfsys@transformshift{5.730469in}{0.550000in}%
\pgfsys@useobject{currentmarker}{}%
\end{pgfscope}%
\end{pgfscope}%
\begin{pgfscope}%
\definecolor{textcolor}{rgb}{0.000000,0.000000,0.000000}%
\pgfsetstrokecolor{textcolor}%
\pgfsetfillcolor{textcolor}%
\pgftext[x=5.730469in,y=0.452778in,,top]{\color{textcolor}{\sffamily\fontsize{10.000000}{12.000000}\selectfont\catcode`\^=\active\def^{\ifmmode\sp\else\^{}\fi}\catcode`\%=\active\def%{\%}150}}%
\end{pgfscope}%
\begin{pgfscope}%
\pgfsetbuttcap%
\pgfsetroundjoin%
\definecolor{currentfill}{rgb}{0.000000,0.000000,0.000000}%
\pgfsetfillcolor{currentfill}%
\pgfsetlinewidth{0.803000pt}%
\definecolor{currentstroke}{rgb}{0.000000,0.000000,0.000000}%
\pgfsetstrokecolor{currentstroke}%
\pgfsetdash{}{0pt}%
\pgfsys@defobject{currentmarker}{\pgfqpoint{0.000000in}{-0.048611in}}{\pgfqpoint{0.000000in}{0.000000in}}{%
\pgfpathmoveto{\pgfqpoint{0.000000in}{0.000000in}}%
\pgfpathlineto{\pgfqpoint{0.000000in}{-0.048611in}}%
\pgfusepath{stroke,fill}%
}%
\begin{pgfscope}%
\pgfsys@transformshift{7.106534in}{0.550000in}%
\pgfsys@useobject{currentmarker}{}%
\end{pgfscope}%
\end{pgfscope}%
\begin{pgfscope}%
\definecolor{textcolor}{rgb}{0.000000,0.000000,0.000000}%
\pgfsetstrokecolor{textcolor}%
\pgfsetfillcolor{textcolor}%
\pgftext[x=7.106534in,y=0.452778in,,top]{\color{textcolor}{\sffamily\fontsize{10.000000}{12.000000}\selectfont\catcode`\^=\active\def^{\ifmmode\sp\else\^{}\fi}\catcode`\%=\active\def%{\%}200}}%
\end{pgfscope}%
\begin{pgfscope}%
\pgfsetbuttcap%
\pgfsetroundjoin%
\definecolor{currentfill}{rgb}{0.000000,0.000000,0.000000}%
\pgfsetfillcolor{currentfill}%
\pgfsetlinewidth{0.803000pt}%
\definecolor{currentstroke}{rgb}{0.000000,0.000000,0.000000}%
\pgfsetstrokecolor{currentstroke}%
\pgfsetdash{}{0pt}%
\pgfsys@defobject{currentmarker}{\pgfqpoint{0.000000in}{-0.048611in}}{\pgfqpoint{0.000000in}{0.000000in}}{%
\pgfpathmoveto{\pgfqpoint{0.000000in}{0.000000in}}%
\pgfpathlineto{\pgfqpoint{0.000000in}{-0.048611in}}%
\pgfusepath{stroke,fill}%
}%
\begin{pgfscope}%
\pgfsys@transformshift{8.482599in}{0.550000in}%
\pgfsys@useobject{currentmarker}{}%
\end{pgfscope}%
\end{pgfscope}%
\begin{pgfscope}%
\definecolor{textcolor}{rgb}{0.000000,0.000000,0.000000}%
\pgfsetstrokecolor{textcolor}%
\pgfsetfillcolor{textcolor}%
\pgftext[x=8.482599in,y=0.452778in,,top]{\color{textcolor}{\sffamily\fontsize{10.000000}{12.000000}\selectfont\catcode`\^=\active\def^{\ifmmode\sp\else\^{}\fi}\catcode`\%=\active\def%{\%}250}}%
\end{pgfscope}%
\begin{pgfscope}%
\definecolor{textcolor}{rgb}{0.000000,0.000000,0.000000}%
\pgfsetstrokecolor{textcolor}%
\pgfsetfillcolor{textcolor}%
\pgftext[x=5.125000in,y=0.262809in,,top]{\color{textcolor}{\sffamily\fontsize{10.000000}{12.000000}\selectfont\catcode`\^=\active\def^{\ifmmode\sp\else\^{}\fi}\catcode`\%=\active\def%{\%}Intensidad}}%
\end{pgfscope}%
\begin{pgfscope}%
\pgfsetbuttcap%
\pgfsetroundjoin%
\definecolor{currentfill}{rgb}{0.000000,0.000000,0.000000}%
\pgfsetfillcolor{currentfill}%
\pgfsetlinewidth{0.803000pt}%
\definecolor{currentstroke}{rgb}{0.000000,0.000000,0.000000}%
\pgfsetstrokecolor{currentstroke}%
\pgfsetdash{}{0pt}%
\pgfsys@defobject{currentmarker}{\pgfqpoint{-0.048611in}{0.000000in}}{\pgfqpoint{-0.000000in}{0.000000in}}{%
\pgfpathmoveto{\pgfqpoint{-0.000000in}{0.000000in}}%
\pgfpathlineto{\pgfqpoint{-0.048611in}{0.000000in}}%
\pgfusepath{stroke,fill}%
}%
\begin{pgfscope}%
\pgfsys@transformshift{1.250000in}{0.550000in}%
\pgfsys@useobject{currentmarker}{}%
\end{pgfscope}%
\end{pgfscope}%
\begin{pgfscope}%
\definecolor{textcolor}{rgb}{0.000000,0.000000,0.000000}%
\pgfsetstrokecolor{textcolor}%
\pgfsetfillcolor{textcolor}%
\pgftext[x=1.064412in, y=0.497238in, left, base]{\color{textcolor}{\sffamily\fontsize{10.000000}{12.000000}\selectfont\catcode`\^=\active\def^{\ifmmode\sp\else\^{}\fi}\catcode`\%=\active\def%{\%}0}}%
\end{pgfscope}%
\begin{pgfscope}%
\pgfsetbuttcap%
\pgfsetroundjoin%
\definecolor{currentfill}{rgb}{0.000000,0.000000,0.000000}%
\pgfsetfillcolor{currentfill}%
\pgfsetlinewidth{0.803000pt}%
\definecolor{currentstroke}{rgb}{0.000000,0.000000,0.000000}%
\pgfsetstrokecolor{currentstroke}%
\pgfsetdash{}{0pt}%
\pgfsys@defobject{currentmarker}{\pgfqpoint{-0.048611in}{0.000000in}}{\pgfqpoint{-0.000000in}{0.000000in}}{%
\pgfpathmoveto{\pgfqpoint{-0.000000in}{0.000000in}}%
\pgfpathlineto{\pgfqpoint{-0.048611in}{0.000000in}}%
\pgfusepath{stroke,fill}%
}%
\begin{pgfscope}%
\pgfsys@transformshift{1.250000in}{1.341253in}%
\pgfsys@useobject{currentmarker}{}%
\end{pgfscope}%
\end{pgfscope}%
\begin{pgfscope}%
\definecolor{textcolor}{rgb}{0.000000,0.000000,0.000000}%
\pgfsetstrokecolor{textcolor}%
\pgfsetfillcolor{textcolor}%
\pgftext[x=0.799316in, y=1.288492in, left, base]{\color{textcolor}{\sffamily\fontsize{10.000000}{12.000000}\selectfont\catcode`\^=\active\def^{\ifmmode\sp\else\^{}\fi}\catcode`\%=\active\def%{\%}2000}}%
\end{pgfscope}%
\begin{pgfscope}%
\pgfsetbuttcap%
\pgfsetroundjoin%
\definecolor{currentfill}{rgb}{0.000000,0.000000,0.000000}%
\pgfsetfillcolor{currentfill}%
\pgfsetlinewidth{0.803000pt}%
\definecolor{currentstroke}{rgb}{0.000000,0.000000,0.000000}%
\pgfsetstrokecolor{currentstroke}%
\pgfsetdash{}{0pt}%
\pgfsys@defobject{currentmarker}{\pgfqpoint{-0.048611in}{0.000000in}}{\pgfqpoint{-0.000000in}{0.000000in}}{%
\pgfpathmoveto{\pgfqpoint{-0.000000in}{0.000000in}}%
\pgfpathlineto{\pgfqpoint{-0.048611in}{0.000000in}}%
\pgfusepath{stroke,fill}%
}%
\begin{pgfscope}%
\pgfsys@transformshift{1.250000in}{2.132506in}%
\pgfsys@useobject{currentmarker}{}%
\end{pgfscope}%
\end{pgfscope}%
\begin{pgfscope}%
\definecolor{textcolor}{rgb}{0.000000,0.000000,0.000000}%
\pgfsetstrokecolor{textcolor}%
\pgfsetfillcolor{textcolor}%
\pgftext[x=0.799316in, y=2.079745in, left, base]{\color{textcolor}{\sffamily\fontsize{10.000000}{12.000000}\selectfont\catcode`\^=\active\def^{\ifmmode\sp\else\^{}\fi}\catcode`\%=\active\def%{\%}4000}}%
\end{pgfscope}%
\begin{pgfscope}%
\pgfsetbuttcap%
\pgfsetroundjoin%
\definecolor{currentfill}{rgb}{0.000000,0.000000,0.000000}%
\pgfsetfillcolor{currentfill}%
\pgfsetlinewidth{0.803000pt}%
\definecolor{currentstroke}{rgb}{0.000000,0.000000,0.000000}%
\pgfsetstrokecolor{currentstroke}%
\pgfsetdash{}{0pt}%
\pgfsys@defobject{currentmarker}{\pgfqpoint{-0.048611in}{0.000000in}}{\pgfqpoint{-0.000000in}{0.000000in}}{%
\pgfpathmoveto{\pgfqpoint{-0.000000in}{0.000000in}}%
\pgfpathlineto{\pgfqpoint{-0.048611in}{0.000000in}}%
\pgfusepath{stroke,fill}%
}%
\begin{pgfscope}%
\pgfsys@transformshift{1.250000in}{2.923759in}%
\pgfsys@useobject{currentmarker}{}%
\end{pgfscope}%
\end{pgfscope}%
\begin{pgfscope}%
\definecolor{textcolor}{rgb}{0.000000,0.000000,0.000000}%
\pgfsetstrokecolor{textcolor}%
\pgfsetfillcolor{textcolor}%
\pgftext[x=0.799316in, y=2.870998in, left, base]{\color{textcolor}{\sffamily\fontsize{10.000000}{12.000000}\selectfont\catcode`\^=\active\def^{\ifmmode\sp\else\^{}\fi}\catcode`\%=\active\def%{\%}6000}}%
\end{pgfscope}%
\begin{pgfscope}%
\pgfsetbuttcap%
\pgfsetroundjoin%
\definecolor{currentfill}{rgb}{0.000000,0.000000,0.000000}%
\pgfsetfillcolor{currentfill}%
\pgfsetlinewidth{0.803000pt}%
\definecolor{currentstroke}{rgb}{0.000000,0.000000,0.000000}%
\pgfsetstrokecolor{currentstroke}%
\pgfsetdash{}{0pt}%
\pgfsys@defobject{currentmarker}{\pgfqpoint{-0.048611in}{0.000000in}}{\pgfqpoint{-0.000000in}{0.000000in}}{%
\pgfpathmoveto{\pgfqpoint{-0.000000in}{0.000000in}}%
\pgfpathlineto{\pgfqpoint{-0.048611in}{0.000000in}}%
\pgfusepath{stroke,fill}%
}%
\begin{pgfscope}%
\pgfsys@transformshift{1.250000in}{3.715012in}%
\pgfsys@useobject{currentmarker}{}%
\end{pgfscope}%
\end{pgfscope}%
\begin{pgfscope}%
\definecolor{textcolor}{rgb}{0.000000,0.000000,0.000000}%
\pgfsetstrokecolor{textcolor}%
\pgfsetfillcolor{textcolor}%
\pgftext[x=0.799316in, y=3.662251in, left, base]{\color{textcolor}{\sffamily\fontsize{10.000000}{12.000000}\selectfont\catcode`\^=\active\def^{\ifmmode\sp\else\^{}\fi}\catcode`\%=\active\def%{\%}8000}}%
\end{pgfscope}%
\begin{pgfscope}%
\definecolor{textcolor}{rgb}{0.000000,0.000000,0.000000}%
\pgfsetstrokecolor{textcolor}%
\pgfsetfillcolor{textcolor}%
\pgftext[x=0.743761in,y=2.475000in,,bottom,rotate=90.000000]{\color{textcolor}{\sffamily\fontsize{10.000000}{12.000000}\selectfont\catcode`\^=\active\def^{\ifmmode\sp\else\^{}\fi}\catcode`\%=\active\def%{\%}Frecuencia}}%
\end{pgfscope}%
\begin{pgfscope}%
\pgfsetrectcap%
\pgfsetmiterjoin%
\pgfsetlinewidth{0.803000pt}%
\definecolor{currentstroke}{rgb}{0.000000,0.000000,0.000000}%
\pgfsetstrokecolor{currentstroke}%
\pgfsetdash{}{0pt}%
\pgfpathmoveto{\pgfqpoint{1.250000in}{0.550000in}}%
\pgfpathlineto{\pgfqpoint{1.250000in}{4.400000in}}%
\pgfusepath{stroke}%
\end{pgfscope}%
\begin{pgfscope}%
\pgfsetrectcap%
\pgfsetmiterjoin%
\pgfsetlinewidth{0.803000pt}%
\definecolor{currentstroke}{rgb}{0.000000,0.000000,0.000000}%
\pgfsetstrokecolor{currentstroke}%
\pgfsetdash{}{0pt}%
\pgfpathmoveto{\pgfqpoint{9.000000in}{0.550000in}}%
\pgfpathlineto{\pgfqpoint{9.000000in}{4.400000in}}%
\pgfusepath{stroke}%
\end{pgfscope}%
\begin{pgfscope}%
\pgfsetrectcap%
\pgfsetmiterjoin%
\pgfsetlinewidth{0.803000pt}%
\definecolor{currentstroke}{rgb}{0.000000,0.000000,0.000000}%
\pgfsetstrokecolor{currentstroke}%
\pgfsetdash{}{0pt}%
\pgfpathmoveto{\pgfqpoint{1.250000in}{0.550000in}}%
\pgfpathlineto{\pgfqpoint{9.000000in}{0.550000in}}%
\pgfusepath{stroke}%
\end{pgfscope}%
\begin{pgfscope}%
\pgfsetrectcap%
\pgfsetmiterjoin%
\pgfsetlinewidth{0.803000pt}%
\definecolor{currentstroke}{rgb}{0.000000,0.000000,0.000000}%
\pgfsetstrokecolor{currentstroke}%
\pgfsetdash{}{0pt}%
\pgfpathmoveto{\pgfqpoint{1.250000in}{4.400000in}}%
\pgfpathlineto{\pgfqpoint{9.000000in}{4.400000in}}%
\pgfusepath{stroke}%
\end{pgfscope}%
\begin{pgfscope}%
\definecolor{textcolor}{rgb}{0.000000,0.000000,0.000000}%
\pgfsetstrokecolor{textcolor}%
\pgfsetfillcolor{textcolor}%
\pgftext[x=5.125000in,y=4.483333in,,base]{\color{textcolor}{\sffamily\fontsize{12.000000}{14.400000}\selectfont\catcode`\^=\active\def^{\ifmmode\sp\else\^{}\fi}\catcode`\%=\active\def%{\%}Histograma de BHEPL-D}}%
\end{pgfscope}%
\end{pgfpicture}%
\makeatother%
\endgroup%
}
		\caption{BHEPL-D}
		\label{fig:bhepl_2}
	\end{subfigure}

	\caption{Comparación de histogramas para la imagen 2015\_00026}
	\label{fig:histogramas_2}
\end{figure}

\subsubsection{Análisis de imagen 2015\_00027}

Texto de análisis para la tercera imagen...

\begin{figure}[H]
	\centering
	\begin{subfigure}[b]{0.45\textwidth}
		\resizebox{\linewidth}{!}{%% Creator: Matplotlib, PGF backend
%%
%% To include the figure in your LaTeX document, write
%%   \input{<filename>.pgf}
%%
%% Make sure the required packages are loaded in your preamble
%%   \usepackage{pgf}
%%
%% Also ensure that all the required font packages are loaded; for instance,
%% the lmodern package is sometimes necessary when using math font.
%%   \usepackage{lmodern}
%%
%% Figures using additional raster images can only be included by \input if
%% they are in the same directory as the main LaTeX file. For loading figures
%% from other directories you can use the `import` package
%%   \usepackage{import}
%%
%% and then include the figures with
%%   \import{<path to file>}{<filename>.pgf}
%%
%% Matplotlib used the following preamble
%%   \def\mathdefault#1{#1}
%%   \everymath=\expandafter{\the\everymath\displaystyle}
%%   \IfFileExists{scrextend.sty}{
%%     \usepackage[fontsize=10.000000pt]{scrextend}
%%   }{
%%     \renewcommand{\normalsize}{\fontsize{10.000000}{12.000000}\selectfont}
%%     \normalsize
%%   }
%%   
%%   \ifdefined\pdftexversion\else  % non-pdftex case.
%%     \usepackage{fontspec}
%%     \setmainfont{DejaVuSerif.ttf}[Path=\detokenize{/home/elias/.virtualenvs/src-pway/lib/python3.13/site-packages/matplotlib/mpl-data/fonts/ttf/}]
%%     \setsansfont{DejaVuSans.ttf}[Path=\detokenize{/home/elias/.virtualenvs/src-pway/lib/python3.13/site-packages/matplotlib/mpl-data/fonts/ttf/}]
%%     \setmonofont{DejaVuSansMono.ttf}[Path=\detokenize{/home/elias/.virtualenvs/src-pway/lib/python3.13/site-packages/matplotlib/mpl-data/fonts/ttf/}]
%%   \fi
%%   \makeatletter\@ifpackageloaded{underscore}{}{\usepackage[strings]{underscore}}\makeatother
%%
\begingroup%
\makeatletter%
\begin{pgfpicture}%
\pgfpathrectangle{\pgfpointorigin}{\pgfqpoint{10.000000in}{5.000000in}}%
\pgfusepath{use as bounding box, clip}%
\begin{pgfscope}%
\pgfsetbuttcap%
\pgfsetmiterjoin%
\definecolor{currentfill}{rgb}{1.000000,1.000000,1.000000}%
\pgfsetfillcolor{currentfill}%
\pgfsetlinewidth{0.000000pt}%
\definecolor{currentstroke}{rgb}{1.000000,1.000000,1.000000}%
\pgfsetstrokecolor{currentstroke}%
\pgfsetdash{}{0pt}%
\pgfpathmoveto{\pgfqpoint{0.000000in}{0.000000in}}%
\pgfpathlineto{\pgfqpoint{10.000000in}{0.000000in}}%
\pgfpathlineto{\pgfqpoint{10.000000in}{5.000000in}}%
\pgfpathlineto{\pgfqpoint{0.000000in}{5.000000in}}%
\pgfpathlineto{\pgfqpoint{0.000000in}{0.000000in}}%
\pgfpathclose%
\pgfusepath{fill}%
\end{pgfscope}%
\begin{pgfscope}%
\pgfsetbuttcap%
\pgfsetmiterjoin%
\definecolor{currentfill}{rgb}{1.000000,1.000000,1.000000}%
\pgfsetfillcolor{currentfill}%
\pgfsetlinewidth{0.000000pt}%
\definecolor{currentstroke}{rgb}{0.000000,0.000000,0.000000}%
\pgfsetstrokecolor{currentstroke}%
\pgfsetstrokeopacity{0.000000}%
\pgfsetdash{}{0pt}%
\pgfpathmoveto{\pgfqpoint{1.250000in}{0.550000in}}%
\pgfpathlineto{\pgfqpoint{9.000000in}{0.550000in}}%
\pgfpathlineto{\pgfqpoint{9.000000in}{4.400000in}}%
\pgfpathlineto{\pgfqpoint{1.250000in}{4.400000in}}%
\pgfpathlineto{\pgfqpoint{1.250000in}{0.550000in}}%
\pgfpathclose%
\pgfusepath{fill}%
\end{pgfscope}%
\begin{pgfscope}%
\pgfpathrectangle{\pgfqpoint{1.250000in}{0.550000in}}{\pgfqpoint{7.750000in}{3.850000in}}%
\pgfusepath{clip}%
\pgfsetbuttcap%
\pgfsetmiterjoin%
\definecolor{currentfill}{rgb}{0.501961,0.501961,0.501961}%
\pgfsetfillcolor{currentfill}%
\pgfsetfillopacity{0.700000}%
\pgfsetlinewidth{0.000000pt}%
\definecolor{currentstroke}{rgb}{0.000000,0.000000,0.000000}%
\pgfsetstrokecolor{currentstroke}%
\pgfsetstrokeopacity{0.700000}%
\pgfsetdash{}{0pt}%
\pgfpathmoveto{\pgfqpoint{1.602273in}{0.550000in}}%
\pgfpathlineto{\pgfqpoint{1.629794in}{0.550000in}}%
\pgfpathlineto{\pgfqpoint{1.629794in}{2.398200in}}%
\pgfpathlineto{\pgfqpoint{1.602273in}{2.398200in}}%
\pgfpathlineto{\pgfqpoint{1.602273in}{0.550000in}}%
\pgfpathclose%
\pgfusepath{fill}%
\end{pgfscope}%
\begin{pgfscope}%
\pgfpathrectangle{\pgfqpoint{1.250000in}{0.550000in}}{\pgfqpoint{7.750000in}{3.850000in}}%
\pgfusepath{clip}%
\pgfsetbuttcap%
\pgfsetmiterjoin%
\definecolor{currentfill}{rgb}{0.501961,0.501961,0.501961}%
\pgfsetfillcolor{currentfill}%
\pgfsetfillopacity{0.700000}%
\pgfsetlinewidth{0.000000pt}%
\definecolor{currentstroke}{rgb}{0.000000,0.000000,0.000000}%
\pgfsetstrokecolor{currentstroke}%
\pgfsetstrokeopacity{0.700000}%
\pgfsetdash{}{0pt}%
\pgfpathmoveto{\pgfqpoint{1.629794in}{0.550000in}}%
\pgfpathlineto{\pgfqpoint{1.657315in}{0.550000in}}%
\pgfpathlineto{\pgfqpoint{1.657315in}{4.216667in}}%
\pgfpathlineto{\pgfqpoint{1.629794in}{4.216667in}}%
\pgfpathlineto{\pgfqpoint{1.629794in}{0.550000in}}%
\pgfpathclose%
\pgfusepath{fill}%
\end{pgfscope}%
\begin{pgfscope}%
\pgfpathrectangle{\pgfqpoint{1.250000in}{0.550000in}}{\pgfqpoint{7.750000in}{3.850000in}}%
\pgfusepath{clip}%
\pgfsetbuttcap%
\pgfsetmiterjoin%
\definecolor{currentfill}{rgb}{0.501961,0.501961,0.501961}%
\pgfsetfillcolor{currentfill}%
\pgfsetfillopacity{0.700000}%
\pgfsetlinewidth{0.000000pt}%
\definecolor{currentstroke}{rgb}{0.000000,0.000000,0.000000}%
\pgfsetstrokecolor{currentstroke}%
\pgfsetstrokeopacity{0.700000}%
\pgfsetdash{}{0pt}%
\pgfpathmoveto{\pgfqpoint{1.657315in}{0.550000in}}%
\pgfpathlineto{\pgfqpoint{1.684837in}{0.550000in}}%
\pgfpathlineto{\pgfqpoint{1.684837in}{3.586192in}}%
\pgfpathlineto{\pgfqpoint{1.657315in}{3.586192in}}%
\pgfpathlineto{\pgfqpoint{1.657315in}{0.550000in}}%
\pgfpathclose%
\pgfusepath{fill}%
\end{pgfscope}%
\begin{pgfscope}%
\pgfpathrectangle{\pgfqpoint{1.250000in}{0.550000in}}{\pgfqpoint{7.750000in}{3.850000in}}%
\pgfusepath{clip}%
\pgfsetbuttcap%
\pgfsetmiterjoin%
\definecolor{currentfill}{rgb}{0.501961,0.501961,0.501961}%
\pgfsetfillcolor{currentfill}%
\pgfsetfillopacity{0.700000}%
\pgfsetlinewidth{0.000000pt}%
\definecolor{currentstroke}{rgb}{0.000000,0.000000,0.000000}%
\pgfsetstrokecolor{currentstroke}%
\pgfsetstrokeopacity{0.700000}%
\pgfsetdash{}{0pt}%
\pgfpathmoveto{\pgfqpoint{1.684837in}{0.550000in}}%
\pgfpathlineto{\pgfqpoint{1.712358in}{0.550000in}}%
\pgfpathlineto{\pgfqpoint{1.712358in}{2.872296in}}%
\pgfpathlineto{\pgfqpoint{1.684837in}{2.872296in}}%
\pgfpathlineto{\pgfqpoint{1.684837in}{0.550000in}}%
\pgfpathclose%
\pgfusepath{fill}%
\end{pgfscope}%
\begin{pgfscope}%
\pgfpathrectangle{\pgfqpoint{1.250000in}{0.550000in}}{\pgfqpoint{7.750000in}{3.850000in}}%
\pgfusepath{clip}%
\pgfsetbuttcap%
\pgfsetmiterjoin%
\definecolor{currentfill}{rgb}{0.501961,0.501961,0.501961}%
\pgfsetfillcolor{currentfill}%
\pgfsetfillopacity{0.700000}%
\pgfsetlinewidth{0.000000pt}%
\definecolor{currentstroke}{rgb}{0.000000,0.000000,0.000000}%
\pgfsetstrokecolor{currentstroke}%
\pgfsetstrokeopacity{0.700000}%
\pgfsetdash{}{0pt}%
\pgfpathmoveto{\pgfqpoint{1.712358in}{0.550000in}}%
\pgfpathlineto{\pgfqpoint{1.739879in}{0.550000in}}%
\pgfpathlineto{\pgfqpoint{1.739879in}{2.257238in}}%
\pgfpathlineto{\pgfqpoint{1.712358in}{2.257238in}}%
\pgfpathlineto{\pgfqpoint{1.712358in}{0.550000in}}%
\pgfpathclose%
\pgfusepath{fill}%
\end{pgfscope}%
\begin{pgfscope}%
\pgfpathrectangle{\pgfqpoint{1.250000in}{0.550000in}}{\pgfqpoint{7.750000in}{3.850000in}}%
\pgfusepath{clip}%
\pgfsetbuttcap%
\pgfsetmiterjoin%
\definecolor{currentfill}{rgb}{0.501961,0.501961,0.501961}%
\pgfsetfillcolor{currentfill}%
\pgfsetfillopacity{0.700000}%
\pgfsetlinewidth{0.000000pt}%
\definecolor{currentstroke}{rgb}{0.000000,0.000000,0.000000}%
\pgfsetstrokecolor{currentstroke}%
\pgfsetstrokeopacity{0.700000}%
\pgfsetdash{}{0pt}%
\pgfpathmoveto{\pgfqpoint{1.739879in}{0.550000in}}%
\pgfpathlineto{\pgfqpoint{1.767401in}{0.550000in}}%
\pgfpathlineto{\pgfqpoint{1.767401in}{1.879229in}}%
\pgfpathlineto{\pgfqpoint{1.739879in}{1.879229in}}%
\pgfpathlineto{\pgfqpoint{1.739879in}{0.550000in}}%
\pgfpathclose%
\pgfusepath{fill}%
\end{pgfscope}%
\begin{pgfscope}%
\pgfpathrectangle{\pgfqpoint{1.250000in}{0.550000in}}{\pgfqpoint{7.750000in}{3.850000in}}%
\pgfusepath{clip}%
\pgfsetbuttcap%
\pgfsetmiterjoin%
\definecolor{currentfill}{rgb}{0.501961,0.501961,0.501961}%
\pgfsetfillcolor{currentfill}%
\pgfsetfillopacity{0.700000}%
\pgfsetlinewidth{0.000000pt}%
\definecolor{currentstroke}{rgb}{0.000000,0.000000,0.000000}%
\pgfsetstrokecolor{currentstroke}%
\pgfsetstrokeopacity{0.700000}%
\pgfsetdash{}{0pt}%
\pgfpathmoveto{\pgfqpoint{1.767401in}{0.550000in}}%
\pgfpathlineto{\pgfqpoint{1.794922in}{0.550000in}}%
\pgfpathlineto{\pgfqpoint{1.794922in}{1.607766in}}%
\pgfpathlineto{\pgfqpoint{1.767401in}{1.607766in}}%
\pgfpathlineto{\pgfqpoint{1.767401in}{0.550000in}}%
\pgfpathclose%
\pgfusepath{fill}%
\end{pgfscope}%
\begin{pgfscope}%
\pgfpathrectangle{\pgfqpoint{1.250000in}{0.550000in}}{\pgfqpoint{7.750000in}{3.850000in}}%
\pgfusepath{clip}%
\pgfsetbuttcap%
\pgfsetmiterjoin%
\definecolor{currentfill}{rgb}{0.501961,0.501961,0.501961}%
\pgfsetfillcolor{currentfill}%
\pgfsetfillopacity{0.700000}%
\pgfsetlinewidth{0.000000pt}%
\definecolor{currentstroke}{rgb}{0.000000,0.000000,0.000000}%
\pgfsetstrokecolor{currentstroke}%
\pgfsetstrokeopacity{0.700000}%
\pgfsetdash{}{0pt}%
\pgfpathmoveto{\pgfqpoint{1.794922in}{0.550000in}}%
\pgfpathlineto{\pgfqpoint{1.822443in}{0.550000in}}%
\pgfpathlineto{\pgfqpoint{1.822443in}{1.512232in}}%
\pgfpathlineto{\pgfqpoint{1.794922in}{1.512232in}}%
\pgfpathlineto{\pgfqpoint{1.794922in}{0.550000in}}%
\pgfpathclose%
\pgfusepath{fill}%
\end{pgfscope}%
\begin{pgfscope}%
\pgfpathrectangle{\pgfqpoint{1.250000in}{0.550000in}}{\pgfqpoint{7.750000in}{3.850000in}}%
\pgfusepath{clip}%
\pgfsetbuttcap%
\pgfsetmiterjoin%
\definecolor{currentfill}{rgb}{0.501961,0.501961,0.501961}%
\pgfsetfillcolor{currentfill}%
\pgfsetfillopacity{0.700000}%
\pgfsetlinewidth{0.000000pt}%
\definecolor{currentstroke}{rgb}{0.000000,0.000000,0.000000}%
\pgfsetstrokecolor{currentstroke}%
\pgfsetstrokeopacity{0.700000}%
\pgfsetdash{}{0pt}%
\pgfpathmoveto{\pgfqpoint{1.822443in}{0.550000in}}%
\pgfpathlineto{\pgfqpoint{1.849964in}{0.550000in}}%
\pgfpathlineto{\pgfqpoint{1.849964in}{1.347317in}}%
\pgfpathlineto{\pgfqpoint{1.822443in}{1.347317in}}%
\pgfpathlineto{\pgfqpoint{1.822443in}{0.550000in}}%
\pgfpathclose%
\pgfusepath{fill}%
\end{pgfscope}%
\begin{pgfscope}%
\pgfpathrectangle{\pgfqpoint{1.250000in}{0.550000in}}{\pgfqpoint{7.750000in}{3.850000in}}%
\pgfusepath{clip}%
\pgfsetbuttcap%
\pgfsetmiterjoin%
\definecolor{currentfill}{rgb}{0.501961,0.501961,0.501961}%
\pgfsetfillcolor{currentfill}%
\pgfsetfillopacity{0.700000}%
\pgfsetlinewidth{0.000000pt}%
\definecolor{currentstroke}{rgb}{0.000000,0.000000,0.000000}%
\pgfsetstrokecolor{currentstroke}%
\pgfsetstrokeopacity{0.700000}%
\pgfsetdash{}{0pt}%
\pgfpathmoveto{\pgfqpoint{1.849964in}{0.550000in}}%
\pgfpathlineto{\pgfqpoint{1.877486in}{0.550000in}}%
\pgfpathlineto{\pgfqpoint{1.877486in}{1.249855in}}%
\pgfpathlineto{\pgfqpoint{1.849964in}{1.249855in}}%
\pgfpathlineto{\pgfqpoint{1.849964in}{0.550000in}}%
\pgfpathclose%
\pgfusepath{fill}%
\end{pgfscope}%
\begin{pgfscope}%
\pgfpathrectangle{\pgfqpoint{1.250000in}{0.550000in}}{\pgfqpoint{7.750000in}{3.850000in}}%
\pgfusepath{clip}%
\pgfsetbuttcap%
\pgfsetmiterjoin%
\definecolor{currentfill}{rgb}{0.501961,0.501961,0.501961}%
\pgfsetfillcolor{currentfill}%
\pgfsetfillopacity{0.700000}%
\pgfsetlinewidth{0.000000pt}%
\definecolor{currentstroke}{rgb}{0.000000,0.000000,0.000000}%
\pgfsetstrokecolor{currentstroke}%
\pgfsetstrokeopacity{0.700000}%
\pgfsetdash{}{0pt}%
\pgfpathmoveto{\pgfqpoint{1.877486in}{0.550000in}}%
\pgfpathlineto{\pgfqpoint{1.905007in}{0.550000in}}%
\pgfpathlineto{\pgfqpoint{1.905007in}{1.151291in}}%
\pgfpathlineto{\pgfqpoint{1.877486in}{1.151291in}}%
\pgfpathlineto{\pgfqpoint{1.877486in}{0.550000in}}%
\pgfpathclose%
\pgfusepath{fill}%
\end{pgfscope}%
\begin{pgfscope}%
\pgfpathrectangle{\pgfqpoint{1.250000in}{0.550000in}}{\pgfqpoint{7.750000in}{3.850000in}}%
\pgfusepath{clip}%
\pgfsetbuttcap%
\pgfsetmiterjoin%
\definecolor{currentfill}{rgb}{0.501961,0.501961,0.501961}%
\pgfsetfillcolor{currentfill}%
\pgfsetfillopacity{0.700000}%
\pgfsetlinewidth{0.000000pt}%
\definecolor{currentstroke}{rgb}{0.000000,0.000000,0.000000}%
\pgfsetstrokecolor{currentstroke}%
\pgfsetstrokeopacity{0.700000}%
\pgfsetdash{}{0pt}%
\pgfpathmoveto{\pgfqpoint{1.905007in}{0.550000in}}%
\pgfpathlineto{\pgfqpoint{1.932528in}{0.550000in}}%
\pgfpathlineto{\pgfqpoint{1.932528in}{1.082187in}}%
\pgfpathlineto{\pgfqpoint{1.905007in}{1.082187in}}%
\pgfpathlineto{\pgfqpoint{1.905007in}{0.550000in}}%
\pgfpathclose%
\pgfusepath{fill}%
\end{pgfscope}%
\begin{pgfscope}%
\pgfpathrectangle{\pgfqpoint{1.250000in}{0.550000in}}{\pgfqpoint{7.750000in}{3.850000in}}%
\pgfusepath{clip}%
\pgfsetbuttcap%
\pgfsetmiterjoin%
\definecolor{currentfill}{rgb}{0.501961,0.501961,0.501961}%
\pgfsetfillcolor{currentfill}%
\pgfsetfillopacity{0.700000}%
\pgfsetlinewidth{0.000000pt}%
\definecolor{currentstroke}{rgb}{0.000000,0.000000,0.000000}%
\pgfsetstrokecolor{currentstroke}%
\pgfsetstrokeopacity{0.700000}%
\pgfsetdash{}{0pt}%
\pgfpathmoveto{\pgfqpoint{1.932528in}{0.550000in}}%
\pgfpathlineto{\pgfqpoint{1.960050in}{0.550000in}}%
\pgfpathlineto{\pgfqpoint{1.960050in}{1.053554in}}%
\pgfpathlineto{\pgfqpoint{1.932528in}{1.053554in}}%
\pgfpathlineto{\pgfqpoint{1.932528in}{0.550000in}}%
\pgfpathclose%
\pgfusepath{fill}%
\end{pgfscope}%
\begin{pgfscope}%
\pgfpathrectangle{\pgfqpoint{1.250000in}{0.550000in}}{\pgfqpoint{7.750000in}{3.850000in}}%
\pgfusepath{clip}%
\pgfsetbuttcap%
\pgfsetmiterjoin%
\definecolor{currentfill}{rgb}{0.501961,0.501961,0.501961}%
\pgfsetfillcolor{currentfill}%
\pgfsetfillopacity{0.700000}%
\pgfsetlinewidth{0.000000pt}%
\definecolor{currentstroke}{rgb}{0.000000,0.000000,0.000000}%
\pgfsetstrokecolor{currentstroke}%
\pgfsetstrokeopacity{0.700000}%
\pgfsetdash{}{0pt}%
\pgfpathmoveto{\pgfqpoint{1.960050in}{0.550000in}}%
\pgfpathlineto{\pgfqpoint{1.987571in}{0.550000in}}%
\pgfpathlineto{\pgfqpoint{1.987571in}{1.011431in}}%
\pgfpathlineto{\pgfqpoint{1.960050in}{1.011431in}}%
\pgfpathlineto{\pgfqpoint{1.960050in}{0.550000in}}%
\pgfpathclose%
\pgfusepath{fill}%
\end{pgfscope}%
\begin{pgfscope}%
\pgfpathrectangle{\pgfqpoint{1.250000in}{0.550000in}}{\pgfqpoint{7.750000in}{3.850000in}}%
\pgfusepath{clip}%
\pgfsetbuttcap%
\pgfsetmiterjoin%
\definecolor{currentfill}{rgb}{0.501961,0.501961,0.501961}%
\pgfsetfillcolor{currentfill}%
\pgfsetfillopacity{0.700000}%
\pgfsetlinewidth{0.000000pt}%
\definecolor{currentstroke}{rgb}{0.000000,0.000000,0.000000}%
\pgfsetstrokecolor{currentstroke}%
\pgfsetstrokeopacity{0.700000}%
\pgfsetdash{}{0pt}%
\pgfpathmoveto{\pgfqpoint{1.987571in}{0.550000in}}%
\pgfpathlineto{\pgfqpoint{2.015092in}{0.550000in}}%
\pgfpathlineto{\pgfqpoint{2.015092in}{0.968757in}}%
\pgfpathlineto{\pgfqpoint{1.987571in}{0.968757in}}%
\pgfpathlineto{\pgfqpoint{1.987571in}{0.550000in}}%
\pgfpathclose%
\pgfusepath{fill}%
\end{pgfscope}%
\begin{pgfscope}%
\pgfpathrectangle{\pgfqpoint{1.250000in}{0.550000in}}{\pgfqpoint{7.750000in}{3.850000in}}%
\pgfusepath{clip}%
\pgfsetbuttcap%
\pgfsetmiterjoin%
\definecolor{currentfill}{rgb}{0.501961,0.501961,0.501961}%
\pgfsetfillcolor{currentfill}%
\pgfsetfillopacity{0.700000}%
\pgfsetlinewidth{0.000000pt}%
\definecolor{currentstroke}{rgb}{0.000000,0.000000,0.000000}%
\pgfsetstrokecolor{currentstroke}%
\pgfsetstrokeopacity{0.700000}%
\pgfsetdash{}{0pt}%
\pgfpathmoveto{\pgfqpoint{2.015092in}{0.550000in}}%
\pgfpathlineto{\pgfqpoint{2.042614in}{0.550000in}}%
\pgfpathlineto{\pgfqpoint{2.042614in}{0.912592in}}%
\pgfpathlineto{\pgfqpoint{2.015092in}{0.912592in}}%
\pgfpathlineto{\pgfqpoint{2.015092in}{0.550000in}}%
\pgfpathclose%
\pgfusepath{fill}%
\end{pgfscope}%
\begin{pgfscope}%
\pgfpathrectangle{\pgfqpoint{1.250000in}{0.550000in}}{\pgfqpoint{7.750000in}{3.850000in}}%
\pgfusepath{clip}%
\pgfsetbuttcap%
\pgfsetmiterjoin%
\definecolor{currentfill}{rgb}{0.501961,0.501961,0.501961}%
\pgfsetfillcolor{currentfill}%
\pgfsetfillopacity{0.700000}%
\pgfsetlinewidth{0.000000pt}%
\definecolor{currentstroke}{rgb}{0.000000,0.000000,0.000000}%
\pgfsetstrokecolor{currentstroke}%
\pgfsetstrokeopacity{0.700000}%
\pgfsetdash{}{0pt}%
\pgfpathmoveto{\pgfqpoint{2.042614in}{0.550000in}}%
\pgfpathlineto{\pgfqpoint{2.070135in}{0.550000in}}%
\pgfpathlineto{\pgfqpoint{2.070135in}{0.906260in}}%
\pgfpathlineto{\pgfqpoint{2.042614in}{0.906260in}}%
\pgfpathlineto{\pgfqpoint{2.042614in}{0.550000in}}%
\pgfpathclose%
\pgfusepath{fill}%
\end{pgfscope}%
\begin{pgfscope}%
\pgfpathrectangle{\pgfqpoint{1.250000in}{0.550000in}}{\pgfqpoint{7.750000in}{3.850000in}}%
\pgfusepath{clip}%
\pgfsetbuttcap%
\pgfsetmiterjoin%
\definecolor{currentfill}{rgb}{0.501961,0.501961,0.501961}%
\pgfsetfillcolor{currentfill}%
\pgfsetfillopacity{0.700000}%
\pgfsetlinewidth{0.000000pt}%
\definecolor{currentstroke}{rgb}{0.000000,0.000000,0.000000}%
\pgfsetstrokecolor{currentstroke}%
\pgfsetstrokeopacity{0.700000}%
\pgfsetdash{}{0pt}%
\pgfpathmoveto{\pgfqpoint{2.070135in}{0.550000in}}%
\pgfpathlineto{\pgfqpoint{2.097656in}{0.550000in}}%
\pgfpathlineto{\pgfqpoint{2.097656in}{0.871019in}}%
\pgfpathlineto{\pgfqpoint{2.070135in}{0.871019in}}%
\pgfpathlineto{\pgfqpoint{2.070135in}{0.550000in}}%
\pgfpathclose%
\pgfusepath{fill}%
\end{pgfscope}%
\begin{pgfscope}%
\pgfpathrectangle{\pgfqpoint{1.250000in}{0.550000in}}{\pgfqpoint{7.750000in}{3.850000in}}%
\pgfusepath{clip}%
\pgfsetbuttcap%
\pgfsetmiterjoin%
\definecolor{currentfill}{rgb}{0.501961,0.501961,0.501961}%
\pgfsetfillcolor{currentfill}%
\pgfsetfillopacity{0.700000}%
\pgfsetlinewidth{0.000000pt}%
\definecolor{currentstroke}{rgb}{0.000000,0.000000,0.000000}%
\pgfsetstrokecolor{currentstroke}%
\pgfsetstrokeopacity{0.700000}%
\pgfsetdash{}{0pt}%
\pgfpathmoveto{\pgfqpoint{2.097656in}{0.550000in}}%
\pgfpathlineto{\pgfqpoint{2.125178in}{0.550000in}}%
\pgfpathlineto{\pgfqpoint{2.125178in}{0.875975in}}%
\pgfpathlineto{\pgfqpoint{2.097656in}{0.875975in}}%
\pgfpathlineto{\pgfqpoint{2.097656in}{0.550000in}}%
\pgfpathclose%
\pgfusepath{fill}%
\end{pgfscope}%
\begin{pgfscope}%
\pgfpathrectangle{\pgfqpoint{1.250000in}{0.550000in}}{\pgfqpoint{7.750000in}{3.850000in}}%
\pgfusepath{clip}%
\pgfsetbuttcap%
\pgfsetmiterjoin%
\definecolor{currentfill}{rgb}{0.501961,0.501961,0.501961}%
\pgfsetfillcolor{currentfill}%
\pgfsetfillopacity{0.700000}%
\pgfsetlinewidth{0.000000pt}%
\definecolor{currentstroke}{rgb}{0.000000,0.000000,0.000000}%
\pgfsetstrokecolor{currentstroke}%
\pgfsetstrokeopacity{0.700000}%
\pgfsetdash{}{0pt}%
\pgfpathmoveto{\pgfqpoint{2.125178in}{0.550000in}}%
\pgfpathlineto{\pgfqpoint{2.152699in}{0.550000in}}%
\pgfpathlineto{\pgfqpoint{2.152699in}{0.870193in}}%
\pgfpathlineto{\pgfqpoint{2.125178in}{0.870193in}}%
\pgfpathlineto{\pgfqpoint{2.125178in}{0.550000in}}%
\pgfpathclose%
\pgfusepath{fill}%
\end{pgfscope}%
\begin{pgfscope}%
\pgfpathrectangle{\pgfqpoint{1.250000in}{0.550000in}}{\pgfqpoint{7.750000in}{3.850000in}}%
\pgfusepath{clip}%
\pgfsetbuttcap%
\pgfsetmiterjoin%
\definecolor{currentfill}{rgb}{0.501961,0.501961,0.501961}%
\pgfsetfillcolor{currentfill}%
\pgfsetfillopacity{0.700000}%
\pgfsetlinewidth{0.000000pt}%
\definecolor{currentstroke}{rgb}{0.000000,0.000000,0.000000}%
\pgfsetstrokecolor{currentstroke}%
\pgfsetstrokeopacity{0.700000}%
\pgfsetdash{}{0pt}%
\pgfpathmoveto{\pgfqpoint{2.152699in}{0.550000in}}%
\pgfpathlineto{\pgfqpoint{2.180220in}{0.550000in}}%
\pgfpathlineto{\pgfqpoint{2.180220in}{0.874598in}}%
\pgfpathlineto{\pgfqpoint{2.152699in}{0.874598in}}%
\pgfpathlineto{\pgfqpoint{2.152699in}{0.550000in}}%
\pgfpathclose%
\pgfusepath{fill}%
\end{pgfscope}%
\begin{pgfscope}%
\pgfpathrectangle{\pgfqpoint{1.250000in}{0.550000in}}{\pgfqpoint{7.750000in}{3.850000in}}%
\pgfusepath{clip}%
\pgfsetbuttcap%
\pgfsetmiterjoin%
\definecolor{currentfill}{rgb}{0.501961,0.501961,0.501961}%
\pgfsetfillcolor{currentfill}%
\pgfsetfillopacity{0.700000}%
\pgfsetlinewidth{0.000000pt}%
\definecolor{currentstroke}{rgb}{0.000000,0.000000,0.000000}%
\pgfsetstrokecolor{currentstroke}%
\pgfsetstrokeopacity{0.700000}%
\pgfsetdash{}{0pt}%
\pgfpathmoveto{\pgfqpoint{2.180220in}{0.550000in}}%
\pgfpathlineto{\pgfqpoint{2.207741in}{0.550000in}}%
\pgfpathlineto{\pgfqpoint{2.207741in}{0.885060in}}%
\pgfpathlineto{\pgfqpoint{2.180220in}{0.885060in}}%
\pgfpathlineto{\pgfqpoint{2.180220in}{0.550000in}}%
\pgfpathclose%
\pgfusepath{fill}%
\end{pgfscope}%
\begin{pgfscope}%
\pgfpathrectangle{\pgfqpoint{1.250000in}{0.550000in}}{\pgfqpoint{7.750000in}{3.850000in}}%
\pgfusepath{clip}%
\pgfsetbuttcap%
\pgfsetmiterjoin%
\definecolor{currentfill}{rgb}{0.501961,0.501961,0.501961}%
\pgfsetfillcolor{currentfill}%
\pgfsetfillopacity{0.700000}%
\pgfsetlinewidth{0.000000pt}%
\definecolor{currentstroke}{rgb}{0.000000,0.000000,0.000000}%
\pgfsetstrokecolor{currentstroke}%
\pgfsetstrokeopacity{0.700000}%
\pgfsetdash{}{0pt}%
\pgfpathmoveto{\pgfqpoint{2.207741in}{0.550000in}}%
\pgfpathlineto{\pgfqpoint{2.235263in}{0.550000in}}%
\pgfpathlineto{\pgfqpoint{2.235263in}{0.871845in}}%
\pgfpathlineto{\pgfqpoint{2.207741in}{0.871845in}}%
\pgfpathlineto{\pgfqpoint{2.207741in}{0.550000in}}%
\pgfpathclose%
\pgfusepath{fill}%
\end{pgfscope}%
\begin{pgfscope}%
\pgfpathrectangle{\pgfqpoint{1.250000in}{0.550000in}}{\pgfqpoint{7.750000in}{3.850000in}}%
\pgfusepath{clip}%
\pgfsetbuttcap%
\pgfsetmiterjoin%
\definecolor{currentfill}{rgb}{0.501961,0.501961,0.501961}%
\pgfsetfillcolor{currentfill}%
\pgfsetfillopacity{0.700000}%
\pgfsetlinewidth{0.000000pt}%
\definecolor{currentstroke}{rgb}{0.000000,0.000000,0.000000}%
\pgfsetstrokecolor{currentstroke}%
\pgfsetstrokeopacity{0.700000}%
\pgfsetdash{}{0pt}%
\pgfpathmoveto{\pgfqpoint{2.235263in}{0.550000in}}%
\pgfpathlineto{\pgfqpoint{2.262784in}{0.550000in}}%
\pgfpathlineto{\pgfqpoint{2.262784in}{0.822839in}}%
\pgfpathlineto{\pgfqpoint{2.235263in}{0.822839in}}%
\pgfpathlineto{\pgfqpoint{2.235263in}{0.550000in}}%
\pgfpathclose%
\pgfusepath{fill}%
\end{pgfscope}%
\begin{pgfscope}%
\pgfpathrectangle{\pgfqpoint{1.250000in}{0.550000in}}{\pgfqpoint{7.750000in}{3.850000in}}%
\pgfusepath{clip}%
\pgfsetbuttcap%
\pgfsetmiterjoin%
\definecolor{currentfill}{rgb}{0.501961,0.501961,0.501961}%
\pgfsetfillcolor{currentfill}%
\pgfsetfillopacity{0.700000}%
\pgfsetlinewidth{0.000000pt}%
\definecolor{currentstroke}{rgb}{0.000000,0.000000,0.000000}%
\pgfsetstrokecolor{currentstroke}%
\pgfsetstrokeopacity{0.700000}%
\pgfsetdash{}{0pt}%
\pgfpathmoveto{\pgfqpoint{2.262784in}{0.550000in}}%
\pgfpathlineto{\pgfqpoint{2.290305in}{0.550000in}}%
\pgfpathlineto{\pgfqpoint{2.290305in}{0.820912in}}%
\pgfpathlineto{\pgfqpoint{2.262784in}{0.820912in}}%
\pgfpathlineto{\pgfqpoint{2.262784in}{0.550000in}}%
\pgfpathclose%
\pgfusepath{fill}%
\end{pgfscope}%
\begin{pgfscope}%
\pgfpathrectangle{\pgfqpoint{1.250000in}{0.550000in}}{\pgfqpoint{7.750000in}{3.850000in}}%
\pgfusepath{clip}%
\pgfsetbuttcap%
\pgfsetmiterjoin%
\definecolor{currentfill}{rgb}{0.501961,0.501961,0.501961}%
\pgfsetfillcolor{currentfill}%
\pgfsetfillopacity{0.700000}%
\pgfsetlinewidth{0.000000pt}%
\definecolor{currentstroke}{rgb}{0.000000,0.000000,0.000000}%
\pgfsetstrokecolor{currentstroke}%
\pgfsetstrokeopacity{0.700000}%
\pgfsetdash{}{0pt}%
\pgfpathmoveto{\pgfqpoint{2.290305in}{0.550000in}}%
\pgfpathlineto{\pgfqpoint{2.317827in}{0.550000in}}%
\pgfpathlineto{\pgfqpoint{2.317827in}{0.812652in}}%
\pgfpathlineto{\pgfqpoint{2.290305in}{0.812652in}}%
\pgfpathlineto{\pgfqpoint{2.290305in}{0.550000in}}%
\pgfpathclose%
\pgfusepath{fill}%
\end{pgfscope}%
\begin{pgfscope}%
\pgfpathrectangle{\pgfqpoint{1.250000in}{0.550000in}}{\pgfqpoint{7.750000in}{3.850000in}}%
\pgfusepath{clip}%
\pgfsetbuttcap%
\pgfsetmiterjoin%
\definecolor{currentfill}{rgb}{0.501961,0.501961,0.501961}%
\pgfsetfillcolor{currentfill}%
\pgfsetfillopacity{0.700000}%
\pgfsetlinewidth{0.000000pt}%
\definecolor{currentstroke}{rgb}{0.000000,0.000000,0.000000}%
\pgfsetstrokecolor{currentstroke}%
\pgfsetstrokeopacity{0.700000}%
\pgfsetdash{}{0pt}%
\pgfpathmoveto{\pgfqpoint{2.317827in}{0.550000in}}%
\pgfpathlineto{\pgfqpoint{2.345348in}{0.550000in}}%
\pgfpathlineto{\pgfqpoint{2.345348in}{0.811000in}}%
\pgfpathlineto{\pgfqpoint{2.317827in}{0.811000in}}%
\pgfpathlineto{\pgfqpoint{2.317827in}{0.550000in}}%
\pgfpathclose%
\pgfusepath{fill}%
\end{pgfscope}%
\begin{pgfscope}%
\pgfpathrectangle{\pgfqpoint{1.250000in}{0.550000in}}{\pgfqpoint{7.750000in}{3.850000in}}%
\pgfusepath{clip}%
\pgfsetbuttcap%
\pgfsetmiterjoin%
\definecolor{currentfill}{rgb}{0.501961,0.501961,0.501961}%
\pgfsetfillcolor{currentfill}%
\pgfsetfillopacity{0.700000}%
\pgfsetlinewidth{0.000000pt}%
\definecolor{currentstroke}{rgb}{0.000000,0.000000,0.000000}%
\pgfsetstrokecolor{currentstroke}%
\pgfsetstrokeopacity{0.700000}%
\pgfsetdash{}{0pt}%
\pgfpathmoveto{\pgfqpoint{2.345348in}{0.550000in}}%
\pgfpathlineto{\pgfqpoint{2.372869in}{0.550000in}}%
\pgfpathlineto{\pgfqpoint{2.372869in}{0.829722in}}%
\pgfpathlineto{\pgfqpoint{2.345348in}{0.829722in}}%
\pgfpathlineto{\pgfqpoint{2.345348in}{0.550000in}}%
\pgfpathclose%
\pgfusepath{fill}%
\end{pgfscope}%
\begin{pgfscope}%
\pgfpathrectangle{\pgfqpoint{1.250000in}{0.550000in}}{\pgfqpoint{7.750000in}{3.850000in}}%
\pgfusepath{clip}%
\pgfsetbuttcap%
\pgfsetmiterjoin%
\definecolor{currentfill}{rgb}{0.501961,0.501961,0.501961}%
\pgfsetfillcolor{currentfill}%
\pgfsetfillopacity{0.700000}%
\pgfsetlinewidth{0.000000pt}%
\definecolor{currentstroke}{rgb}{0.000000,0.000000,0.000000}%
\pgfsetstrokecolor{currentstroke}%
\pgfsetstrokeopacity{0.700000}%
\pgfsetdash{}{0pt}%
\pgfpathmoveto{\pgfqpoint{2.372869in}{0.550000in}}%
\pgfpathlineto{\pgfqpoint{2.400391in}{0.550000in}}%
\pgfpathlineto{\pgfqpoint{2.400391in}{0.808247in}}%
\pgfpathlineto{\pgfqpoint{2.372869in}{0.808247in}}%
\pgfpathlineto{\pgfqpoint{2.372869in}{0.550000in}}%
\pgfpathclose%
\pgfusepath{fill}%
\end{pgfscope}%
\begin{pgfscope}%
\pgfpathrectangle{\pgfqpoint{1.250000in}{0.550000in}}{\pgfqpoint{7.750000in}{3.850000in}}%
\pgfusepath{clip}%
\pgfsetbuttcap%
\pgfsetmiterjoin%
\definecolor{currentfill}{rgb}{0.501961,0.501961,0.501961}%
\pgfsetfillcolor{currentfill}%
\pgfsetfillopacity{0.700000}%
\pgfsetlinewidth{0.000000pt}%
\definecolor{currentstroke}{rgb}{0.000000,0.000000,0.000000}%
\pgfsetstrokecolor{currentstroke}%
\pgfsetstrokeopacity{0.700000}%
\pgfsetdash{}{0pt}%
\pgfpathmoveto{\pgfqpoint{2.400391in}{0.550000in}}%
\pgfpathlineto{\pgfqpoint{2.427912in}{0.550000in}}%
\pgfpathlineto{\pgfqpoint{2.427912in}{0.814304in}}%
\pgfpathlineto{\pgfqpoint{2.400391in}{0.814304in}}%
\pgfpathlineto{\pgfqpoint{2.400391in}{0.550000in}}%
\pgfpathclose%
\pgfusepath{fill}%
\end{pgfscope}%
\begin{pgfscope}%
\pgfpathrectangle{\pgfqpoint{1.250000in}{0.550000in}}{\pgfqpoint{7.750000in}{3.850000in}}%
\pgfusepath{clip}%
\pgfsetbuttcap%
\pgfsetmiterjoin%
\definecolor{currentfill}{rgb}{0.501961,0.501961,0.501961}%
\pgfsetfillcolor{currentfill}%
\pgfsetfillopacity{0.700000}%
\pgfsetlinewidth{0.000000pt}%
\definecolor{currentstroke}{rgb}{0.000000,0.000000,0.000000}%
\pgfsetstrokecolor{currentstroke}%
\pgfsetstrokeopacity{0.700000}%
\pgfsetdash{}{0pt}%
\pgfpathmoveto{\pgfqpoint{2.427912in}{0.550000in}}%
\pgfpathlineto{\pgfqpoint{2.455433in}{0.550000in}}%
\pgfpathlineto{\pgfqpoint{2.455433in}{0.811000in}}%
\pgfpathlineto{\pgfqpoint{2.427912in}{0.811000in}}%
\pgfpathlineto{\pgfqpoint{2.427912in}{0.550000in}}%
\pgfpathclose%
\pgfusepath{fill}%
\end{pgfscope}%
\begin{pgfscope}%
\pgfpathrectangle{\pgfqpoint{1.250000in}{0.550000in}}{\pgfqpoint{7.750000in}{3.850000in}}%
\pgfusepath{clip}%
\pgfsetbuttcap%
\pgfsetmiterjoin%
\definecolor{currentfill}{rgb}{0.501961,0.501961,0.501961}%
\pgfsetfillcolor{currentfill}%
\pgfsetfillopacity{0.700000}%
\pgfsetlinewidth{0.000000pt}%
\definecolor{currentstroke}{rgb}{0.000000,0.000000,0.000000}%
\pgfsetstrokecolor{currentstroke}%
\pgfsetstrokeopacity{0.700000}%
\pgfsetdash{}{0pt}%
\pgfpathmoveto{\pgfqpoint{2.455433in}{0.550000in}}%
\pgfpathlineto{\pgfqpoint{2.482955in}{0.550000in}}%
\pgfpathlineto{\pgfqpoint{2.482955in}{0.792829in}}%
\pgfpathlineto{\pgfqpoint{2.455433in}{0.792829in}}%
\pgfpathlineto{\pgfqpoint{2.455433in}{0.550000in}}%
\pgfpathclose%
\pgfusepath{fill}%
\end{pgfscope}%
\begin{pgfscope}%
\pgfpathrectangle{\pgfqpoint{1.250000in}{0.550000in}}{\pgfqpoint{7.750000in}{3.850000in}}%
\pgfusepath{clip}%
\pgfsetbuttcap%
\pgfsetmiterjoin%
\definecolor{currentfill}{rgb}{0.501961,0.501961,0.501961}%
\pgfsetfillcolor{currentfill}%
\pgfsetfillopacity{0.700000}%
\pgfsetlinewidth{0.000000pt}%
\definecolor{currentstroke}{rgb}{0.000000,0.000000,0.000000}%
\pgfsetstrokecolor{currentstroke}%
\pgfsetstrokeopacity{0.700000}%
\pgfsetdash{}{0pt}%
\pgfpathmoveto{\pgfqpoint{2.482955in}{0.550000in}}%
\pgfpathlineto{\pgfqpoint{2.510476in}{0.550000in}}%
\pgfpathlineto{\pgfqpoint{2.510476in}{0.764747in}}%
\pgfpathlineto{\pgfqpoint{2.482955in}{0.764747in}}%
\pgfpathlineto{\pgfqpoint{2.482955in}{0.550000in}}%
\pgfpathclose%
\pgfusepath{fill}%
\end{pgfscope}%
\begin{pgfscope}%
\pgfpathrectangle{\pgfqpoint{1.250000in}{0.550000in}}{\pgfqpoint{7.750000in}{3.850000in}}%
\pgfusepath{clip}%
\pgfsetbuttcap%
\pgfsetmiterjoin%
\definecolor{currentfill}{rgb}{0.501961,0.501961,0.501961}%
\pgfsetfillcolor{currentfill}%
\pgfsetfillopacity{0.700000}%
\pgfsetlinewidth{0.000000pt}%
\definecolor{currentstroke}{rgb}{0.000000,0.000000,0.000000}%
\pgfsetstrokecolor{currentstroke}%
\pgfsetstrokeopacity{0.700000}%
\pgfsetdash{}{0pt}%
\pgfpathmoveto{\pgfqpoint{2.510476in}{0.550000in}}%
\pgfpathlineto{\pgfqpoint{2.537997in}{0.550000in}}%
\pgfpathlineto{\pgfqpoint{2.537997in}{0.770529in}}%
\pgfpathlineto{\pgfqpoint{2.510476in}{0.770529in}}%
\pgfpathlineto{\pgfqpoint{2.510476in}{0.550000in}}%
\pgfpathclose%
\pgfusepath{fill}%
\end{pgfscope}%
\begin{pgfscope}%
\pgfpathrectangle{\pgfqpoint{1.250000in}{0.550000in}}{\pgfqpoint{7.750000in}{3.850000in}}%
\pgfusepath{clip}%
\pgfsetbuttcap%
\pgfsetmiterjoin%
\definecolor{currentfill}{rgb}{0.501961,0.501961,0.501961}%
\pgfsetfillcolor{currentfill}%
\pgfsetfillopacity{0.700000}%
\pgfsetlinewidth{0.000000pt}%
\definecolor{currentstroke}{rgb}{0.000000,0.000000,0.000000}%
\pgfsetstrokecolor{currentstroke}%
\pgfsetstrokeopacity{0.700000}%
\pgfsetdash{}{0pt}%
\pgfpathmoveto{\pgfqpoint{2.537997in}{0.550000in}}%
\pgfpathlineto{\pgfqpoint{2.565518in}{0.550000in}}%
\pgfpathlineto{\pgfqpoint{2.565518in}{0.755386in}}%
\pgfpathlineto{\pgfqpoint{2.537997in}{0.755386in}}%
\pgfpathlineto{\pgfqpoint{2.537997in}{0.550000in}}%
\pgfpathclose%
\pgfusepath{fill}%
\end{pgfscope}%
\begin{pgfscope}%
\pgfpathrectangle{\pgfqpoint{1.250000in}{0.550000in}}{\pgfqpoint{7.750000in}{3.850000in}}%
\pgfusepath{clip}%
\pgfsetbuttcap%
\pgfsetmiterjoin%
\definecolor{currentfill}{rgb}{0.501961,0.501961,0.501961}%
\pgfsetfillcolor{currentfill}%
\pgfsetfillopacity{0.700000}%
\pgfsetlinewidth{0.000000pt}%
\definecolor{currentstroke}{rgb}{0.000000,0.000000,0.000000}%
\pgfsetstrokecolor{currentstroke}%
\pgfsetstrokeopacity{0.700000}%
\pgfsetdash{}{0pt}%
\pgfpathmoveto{\pgfqpoint{2.565518in}{0.550000in}}%
\pgfpathlineto{\pgfqpoint{2.593040in}{0.550000in}}%
\pgfpathlineto{\pgfqpoint{2.593040in}{0.779339in}}%
\pgfpathlineto{\pgfqpoint{2.565518in}{0.779339in}}%
\pgfpathlineto{\pgfqpoint{2.565518in}{0.550000in}}%
\pgfpathclose%
\pgfusepath{fill}%
\end{pgfscope}%
\begin{pgfscope}%
\pgfpathrectangle{\pgfqpoint{1.250000in}{0.550000in}}{\pgfqpoint{7.750000in}{3.850000in}}%
\pgfusepath{clip}%
\pgfsetbuttcap%
\pgfsetmiterjoin%
\definecolor{currentfill}{rgb}{0.501961,0.501961,0.501961}%
\pgfsetfillcolor{currentfill}%
\pgfsetfillopacity{0.700000}%
\pgfsetlinewidth{0.000000pt}%
\definecolor{currentstroke}{rgb}{0.000000,0.000000,0.000000}%
\pgfsetstrokecolor{currentstroke}%
\pgfsetstrokeopacity{0.700000}%
\pgfsetdash{}{0pt}%
\pgfpathmoveto{\pgfqpoint{2.593040in}{0.550000in}}%
\pgfpathlineto{\pgfqpoint{2.620561in}{0.550000in}}%
\pgfpathlineto{\pgfqpoint{2.620561in}{0.763646in}}%
\pgfpathlineto{\pgfqpoint{2.593040in}{0.763646in}}%
\pgfpathlineto{\pgfqpoint{2.593040in}{0.550000in}}%
\pgfpathclose%
\pgfusepath{fill}%
\end{pgfscope}%
\begin{pgfscope}%
\pgfpathrectangle{\pgfqpoint{1.250000in}{0.550000in}}{\pgfqpoint{7.750000in}{3.850000in}}%
\pgfusepath{clip}%
\pgfsetbuttcap%
\pgfsetmiterjoin%
\definecolor{currentfill}{rgb}{0.501961,0.501961,0.501961}%
\pgfsetfillcolor{currentfill}%
\pgfsetfillopacity{0.700000}%
\pgfsetlinewidth{0.000000pt}%
\definecolor{currentstroke}{rgb}{0.000000,0.000000,0.000000}%
\pgfsetstrokecolor{currentstroke}%
\pgfsetstrokeopacity{0.700000}%
\pgfsetdash{}{0pt}%
\pgfpathmoveto{\pgfqpoint{2.620561in}{0.550000in}}%
\pgfpathlineto{\pgfqpoint{2.648082in}{0.550000in}}%
\pgfpathlineto{\pgfqpoint{2.648082in}{0.786222in}}%
\pgfpathlineto{\pgfqpoint{2.620561in}{0.786222in}}%
\pgfpathlineto{\pgfqpoint{2.620561in}{0.550000in}}%
\pgfpathclose%
\pgfusepath{fill}%
\end{pgfscope}%
\begin{pgfscope}%
\pgfpathrectangle{\pgfqpoint{1.250000in}{0.550000in}}{\pgfqpoint{7.750000in}{3.850000in}}%
\pgfusepath{clip}%
\pgfsetbuttcap%
\pgfsetmiterjoin%
\definecolor{currentfill}{rgb}{0.501961,0.501961,0.501961}%
\pgfsetfillcolor{currentfill}%
\pgfsetfillopacity{0.700000}%
\pgfsetlinewidth{0.000000pt}%
\definecolor{currentstroke}{rgb}{0.000000,0.000000,0.000000}%
\pgfsetstrokecolor{currentstroke}%
\pgfsetstrokeopacity{0.700000}%
\pgfsetdash{}{0pt}%
\pgfpathmoveto{\pgfqpoint{2.648082in}{0.550000in}}%
\pgfpathlineto{\pgfqpoint{2.675604in}{0.550000in}}%
\pgfpathlineto{\pgfqpoint{2.675604in}{0.762269in}}%
\pgfpathlineto{\pgfqpoint{2.648082in}{0.762269in}}%
\pgfpathlineto{\pgfqpoint{2.648082in}{0.550000in}}%
\pgfpathclose%
\pgfusepath{fill}%
\end{pgfscope}%
\begin{pgfscope}%
\pgfpathrectangle{\pgfqpoint{1.250000in}{0.550000in}}{\pgfqpoint{7.750000in}{3.850000in}}%
\pgfusepath{clip}%
\pgfsetbuttcap%
\pgfsetmiterjoin%
\definecolor{currentfill}{rgb}{0.501961,0.501961,0.501961}%
\pgfsetfillcolor{currentfill}%
\pgfsetfillopacity{0.700000}%
\pgfsetlinewidth{0.000000pt}%
\definecolor{currentstroke}{rgb}{0.000000,0.000000,0.000000}%
\pgfsetstrokecolor{currentstroke}%
\pgfsetstrokeopacity{0.700000}%
\pgfsetdash{}{0pt}%
\pgfpathmoveto{\pgfqpoint{2.675604in}{0.550000in}}%
\pgfpathlineto{\pgfqpoint{2.703125in}{0.550000in}}%
\pgfpathlineto{\pgfqpoint{2.703125in}{0.749329in}}%
\pgfpathlineto{\pgfqpoint{2.675604in}{0.749329in}}%
\pgfpathlineto{\pgfqpoint{2.675604in}{0.550000in}}%
\pgfpathclose%
\pgfusepath{fill}%
\end{pgfscope}%
\begin{pgfscope}%
\pgfpathrectangle{\pgfqpoint{1.250000in}{0.550000in}}{\pgfqpoint{7.750000in}{3.850000in}}%
\pgfusepath{clip}%
\pgfsetbuttcap%
\pgfsetmiterjoin%
\definecolor{currentfill}{rgb}{0.501961,0.501961,0.501961}%
\pgfsetfillcolor{currentfill}%
\pgfsetfillopacity{0.700000}%
\pgfsetlinewidth{0.000000pt}%
\definecolor{currentstroke}{rgb}{0.000000,0.000000,0.000000}%
\pgfsetstrokecolor{currentstroke}%
\pgfsetstrokeopacity{0.700000}%
\pgfsetdash{}{0pt}%
\pgfpathmoveto{\pgfqpoint{2.703125in}{0.550000in}}%
\pgfpathlineto{\pgfqpoint{2.730646in}{0.550000in}}%
\pgfpathlineto{\pgfqpoint{2.730646in}{0.746851in}}%
\pgfpathlineto{\pgfqpoint{2.703125in}{0.746851in}}%
\pgfpathlineto{\pgfqpoint{2.703125in}{0.550000in}}%
\pgfpathclose%
\pgfusepath{fill}%
\end{pgfscope}%
\begin{pgfscope}%
\pgfpathrectangle{\pgfqpoint{1.250000in}{0.550000in}}{\pgfqpoint{7.750000in}{3.850000in}}%
\pgfusepath{clip}%
\pgfsetbuttcap%
\pgfsetmiterjoin%
\definecolor{currentfill}{rgb}{0.501961,0.501961,0.501961}%
\pgfsetfillcolor{currentfill}%
\pgfsetfillopacity{0.700000}%
\pgfsetlinewidth{0.000000pt}%
\definecolor{currentstroke}{rgb}{0.000000,0.000000,0.000000}%
\pgfsetstrokecolor{currentstroke}%
\pgfsetstrokeopacity{0.700000}%
\pgfsetdash{}{0pt}%
\pgfpathmoveto{\pgfqpoint{2.730646in}{0.550000in}}%
\pgfpathlineto{\pgfqpoint{2.758168in}{0.550000in}}%
\pgfpathlineto{\pgfqpoint{2.758168in}{0.733911in}}%
\pgfpathlineto{\pgfqpoint{2.730646in}{0.733911in}}%
\pgfpathlineto{\pgfqpoint{2.730646in}{0.550000in}}%
\pgfpathclose%
\pgfusepath{fill}%
\end{pgfscope}%
\begin{pgfscope}%
\pgfpathrectangle{\pgfqpoint{1.250000in}{0.550000in}}{\pgfqpoint{7.750000in}{3.850000in}}%
\pgfusepath{clip}%
\pgfsetbuttcap%
\pgfsetmiterjoin%
\definecolor{currentfill}{rgb}{0.501961,0.501961,0.501961}%
\pgfsetfillcolor{currentfill}%
\pgfsetfillopacity{0.700000}%
\pgfsetlinewidth{0.000000pt}%
\definecolor{currentstroke}{rgb}{0.000000,0.000000,0.000000}%
\pgfsetstrokecolor{currentstroke}%
\pgfsetstrokeopacity{0.700000}%
\pgfsetdash{}{0pt}%
\pgfpathmoveto{\pgfqpoint{2.758168in}{0.550000in}}%
\pgfpathlineto{\pgfqpoint{2.785689in}{0.550000in}}%
\pgfpathlineto{\pgfqpoint{2.785689in}{0.735839in}}%
\pgfpathlineto{\pgfqpoint{2.758168in}{0.735839in}}%
\pgfpathlineto{\pgfqpoint{2.758168in}{0.550000in}}%
\pgfpathclose%
\pgfusepath{fill}%
\end{pgfscope}%
\begin{pgfscope}%
\pgfpathrectangle{\pgfqpoint{1.250000in}{0.550000in}}{\pgfqpoint{7.750000in}{3.850000in}}%
\pgfusepath{clip}%
\pgfsetbuttcap%
\pgfsetmiterjoin%
\definecolor{currentfill}{rgb}{0.501961,0.501961,0.501961}%
\pgfsetfillcolor{currentfill}%
\pgfsetfillopacity{0.700000}%
\pgfsetlinewidth{0.000000pt}%
\definecolor{currentstroke}{rgb}{0.000000,0.000000,0.000000}%
\pgfsetstrokecolor{currentstroke}%
\pgfsetstrokeopacity{0.700000}%
\pgfsetdash{}{0pt}%
\pgfpathmoveto{\pgfqpoint{2.785689in}{0.550000in}}%
\pgfpathlineto{\pgfqpoint{2.813210in}{0.550000in}}%
\pgfpathlineto{\pgfqpoint{2.813210in}{0.730883in}}%
\pgfpathlineto{\pgfqpoint{2.785689in}{0.730883in}}%
\pgfpathlineto{\pgfqpoint{2.785689in}{0.550000in}}%
\pgfpathclose%
\pgfusepath{fill}%
\end{pgfscope}%
\begin{pgfscope}%
\pgfpathrectangle{\pgfqpoint{1.250000in}{0.550000in}}{\pgfqpoint{7.750000in}{3.850000in}}%
\pgfusepath{clip}%
\pgfsetbuttcap%
\pgfsetmiterjoin%
\definecolor{currentfill}{rgb}{0.501961,0.501961,0.501961}%
\pgfsetfillcolor{currentfill}%
\pgfsetfillopacity{0.700000}%
\pgfsetlinewidth{0.000000pt}%
\definecolor{currentstroke}{rgb}{0.000000,0.000000,0.000000}%
\pgfsetstrokecolor{currentstroke}%
\pgfsetstrokeopacity{0.700000}%
\pgfsetdash{}{0pt}%
\pgfpathmoveto{\pgfqpoint{2.813210in}{0.550000in}}%
\pgfpathlineto{\pgfqpoint{2.840732in}{0.550000in}}%
\pgfpathlineto{\pgfqpoint{2.840732in}{0.725377in}}%
\pgfpathlineto{\pgfqpoint{2.813210in}{0.725377in}}%
\pgfpathlineto{\pgfqpoint{2.813210in}{0.550000in}}%
\pgfpathclose%
\pgfusepath{fill}%
\end{pgfscope}%
\begin{pgfscope}%
\pgfpathrectangle{\pgfqpoint{1.250000in}{0.550000in}}{\pgfqpoint{7.750000in}{3.850000in}}%
\pgfusepath{clip}%
\pgfsetbuttcap%
\pgfsetmiterjoin%
\definecolor{currentfill}{rgb}{0.501961,0.501961,0.501961}%
\pgfsetfillcolor{currentfill}%
\pgfsetfillopacity{0.700000}%
\pgfsetlinewidth{0.000000pt}%
\definecolor{currentstroke}{rgb}{0.000000,0.000000,0.000000}%
\pgfsetstrokecolor{currentstroke}%
\pgfsetstrokeopacity{0.700000}%
\pgfsetdash{}{0pt}%
\pgfpathmoveto{\pgfqpoint{2.840732in}{0.550000in}}%
\pgfpathlineto{\pgfqpoint{2.868253in}{0.550000in}}%
\pgfpathlineto{\pgfqpoint{2.868253in}{0.726753in}}%
\pgfpathlineto{\pgfqpoint{2.840732in}{0.726753in}}%
\pgfpathlineto{\pgfqpoint{2.840732in}{0.550000in}}%
\pgfpathclose%
\pgfusepath{fill}%
\end{pgfscope}%
\begin{pgfscope}%
\pgfpathrectangle{\pgfqpoint{1.250000in}{0.550000in}}{\pgfqpoint{7.750000in}{3.850000in}}%
\pgfusepath{clip}%
\pgfsetbuttcap%
\pgfsetmiterjoin%
\definecolor{currentfill}{rgb}{0.501961,0.501961,0.501961}%
\pgfsetfillcolor{currentfill}%
\pgfsetfillopacity{0.700000}%
\pgfsetlinewidth{0.000000pt}%
\definecolor{currentstroke}{rgb}{0.000000,0.000000,0.000000}%
\pgfsetstrokecolor{currentstroke}%
\pgfsetstrokeopacity{0.700000}%
\pgfsetdash{}{0pt}%
\pgfpathmoveto{\pgfqpoint{2.868253in}{0.550000in}}%
\pgfpathlineto{\pgfqpoint{2.895774in}{0.550000in}}%
\pgfpathlineto{\pgfqpoint{2.895774in}{0.734187in}}%
\pgfpathlineto{\pgfqpoint{2.868253in}{0.734187in}}%
\pgfpathlineto{\pgfqpoint{2.868253in}{0.550000in}}%
\pgfpathclose%
\pgfusepath{fill}%
\end{pgfscope}%
\begin{pgfscope}%
\pgfpathrectangle{\pgfqpoint{1.250000in}{0.550000in}}{\pgfqpoint{7.750000in}{3.850000in}}%
\pgfusepath{clip}%
\pgfsetbuttcap%
\pgfsetmiterjoin%
\definecolor{currentfill}{rgb}{0.501961,0.501961,0.501961}%
\pgfsetfillcolor{currentfill}%
\pgfsetfillopacity{0.700000}%
\pgfsetlinewidth{0.000000pt}%
\definecolor{currentstroke}{rgb}{0.000000,0.000000,0.000000}%
\pgfsetstrokecolor{currentstroke}%
\pgfsetstrokeopacity{0.700000}%
\pgfsetdash{}{0pt}%
\pgfpathmoveto{\pgfqpoint{2.895774in}{0.550000in}}%
\pgfpathlineto{\pgfqpoint{2.923295in}{0.550000in}}%
\pgfpathlineto{\pgfqpoint{2.923295in}{0.750430in}}%
\pgfpathlineto{\pgfqpoint{2.895774in}{0.750430in}}%
\pgfpathlineto{\pgfqpoint{2.895774in}{0.550000in}}%
\pgfpathclose%
\pgfusepath{fill}%
\end{pgfscope}%
\begin{pgfscope}%
\pgfpathrectangle{\pgfqpoint{1.250000in}{0.550000in}}{\pgfqpoint{7.750000in}{3.850000in}}%
\pgfusepath{clip}%
\pgfsetbuttcap%
\pgfsetmiterjoin%
\definecolor{currentfill}{rgb}{0.501961,0.501961,0.501961}%
\pgfsetfillcolor{currentfill}%
\pgfsetfillopacity{0.700000}%
\pgfsetlinewidth{0.000000pt}%
\definecolor{currentstroke}{rgb}{0.000000,0.000000,0.000000}%
\pgfsetstrokecolor{currentstroke}%
\pgfsetstrokeopacity{0.700000}%
\pgfsetdash{}{0pt}%
\pgfpathmoveto{\pgfqpoint{2.923295in}{0.550000in}}%
\pgfpathlineto{\pgfqpoint{2.950817in}{0.550000in}}%
\pgfpathlineto{\pgfqpoint{2.950817in}{0.741896in}}%
\pgfpathlineto{\pgfqpoint{2.923295in}{0.741896in}}%
\pgfpathlineto{\pgfqpoint{2.923295in}{0.550000in}}%
\pgfpathclose%
\pgfusepath{fill}%
\end{pgfscope}%
\begin{pgfscope}%
\pgfpathrectangle{\pgfqpoint{1.250000in}{0.550000in}}{\pgfqpoint{7.750000in}{3.850000in}}%
\pgfusepath{clip}%
\pgfsetbuttcap%
\pgfsetmiterjoin%
\definecolor{currentfill}{rgb}{0.501961,0.501961,0.501961}%
\pgfsetfillcolor{currentfill}%
\pgfsetfillopacity{0.700000}%
\pgfsetlinewidth{0.000000pt}%
\definecolor{currentstroke}{rgb}{0.000000,0.000000,0.000000}%
\pgfsetstrokecolor{currentstroke}%
\pgfsetstrokeopacity{0.700000}%
\pgfsetdash{}{0pt}%
\pgfpathmoveto{\pgfqpoint{2.950817in}{0.550000in}}%
\pgfpathlineto{\pgfqpoint{2.978338in}{0.550000in}}%
\pgfpathlineto{\pgfqpoint{2.978338in}{0.738592in}}%
\pgfpathlineto{\pgfqpoint{2.950817in}{0.738592in}}%
\pgfpathlineto{\pgfqpoint{2.950817in}{0.550000in}}%
\pgfpathclose%
\pgfusepath{fill}%
\end{pgfscope}%
\begin{pgfscope}%
\pgfpathrectangle{\pgfqpoint{1.250000in}{0.550000in}}{\pgfqpoint{7.750000in}{3.850000in}}%
\pgfusepath{clip}%
\pgfsetbuttcap%
\pgfsetmiterjoin%
\definecolor{currentfill}{rgb}{0.501961,0.501961,0.501961}%
\pgfsetfillcolor{currentfill}%
\pgfsetfillopacity{0.700000}%
\pgfsetlinewidth{0.000000pt}%
\definecolor{currentstroke}{rgb}{0.000000,0.000000,0.000000}%
\pgfsetstrokecolor{currentstroke}%
\pgfsetstrokeopacity{0.700000}%
\pgfsetdash{}{0pt}%
\pgfpathmoveto{\pgfqpoint{2.978338in}{0.550000in}}%
\pgfpathlineto{\pgfqpoint{3.005859in}{0.550000in}}%
\pgfpathlineto{\pgfqpoint{3.005859in}{0.724551in}}%
\pgfpathlineto{\pgfqpoint{2.978338in}{0.724551in}}%
\pgfpathlineto{\pgfqpoint{2.978338in}{0.550000in}}%
\pgfpathclose%
\pgfusepath{fill}%
\end{pgfscope}%
\begin{pgfscope}%
\pgfpathrectangle{\pgfqpoint{1.250000in}{0.550000in}}{\pgfqpoint{7.750000in}{3.850000in}}%
\pgfusepath{clip}%
\pgfsetbuttcap%
\pgfsetmiterjoin%
\definecolor{currentfill}{rgb}{0.501961,0.501961,0.501961}%
\pgfsetfillcolor{currentfill}%
\pgfsetfillopacity{0.700000}%
\pgfsetlinewidth{0.000000pt}%
\definecolor{currentstroke}{rgb}{0.000000,0.000000,0.000000}%
\pgfsetstrokecolor{currentstroke}%
\pgfsetstrokeopacity{0.700000}%
\pgfsetdash{}{0pt}%
\pgfpathmoveto{\pgfqpoint{3.005859in}{0.550000in}}%
\pgfpathlineto{\pgfqpoint{3.033381in}{0.550000in}}%
\pgfpathlineto{\pgfqpoint{3.033381in}{0.736940in}}%
\pgfpathlineto{\pgfqpoint{3.005859in}{0.736940in}}%
\pgfpathlineto{\pgfqpoint{3.005859in}{0.550000in}}%
\pgfpathclose%
\pgfusepath{fill}%
\end{pgfscope}%
\begin{pgfscope}%
\pgfpathrectangle{\pgfqpoint{1.250000in}{0.550000in}}{\pgfqpoint{7.750000in}{3.850000in}}%
\pgfusepath{clip}%
\pgfsetbuttcap%
\pgfsetmiterjoin%
\definecolor{currentfill}{rgb}{0.501961,0.501961,0.501961}%
\pgfsetfillcolor{currentfill}%
\pgfsetfillopacity{0.700000}%
\pgfsetlinewidth{0.000000pt}%
\definecolor{currentstroke}{rgb}{0.000000,0.000000,0.000000}%
\pgfsetstrokecolor{currentstroke}%
\pgfsetstrokeopacity{0.700000}%
\pgfsetdash{}{0pt}%
\pgfpathmoveto{\pgfqpoint{3.033381in}{0.550000in}}%
\pgfpathlineto{\pgfqpoint{3.060902in}{0.550000in}}%
\pgfpathlineto{\pgfqpoint{3.060902in}{0.730883in}}%
\pgfpathlineto{\pgfqpoint{3.033381in}{0.730883in}}%
\pgfpathlineto{\pgfqpoint{3.033381in}{0.550000in}}%
\pgfpathclose%
\pgfusepath{fill}%
\end{pgfscope}%
\begin{pgfscope}%
\pgfpathrectangle{\pgfqpoint{1.250000in}{0.550000in}}{\pgfqpoint{7.750000in}{3.850000in}}%
\pgfusepath{clip}%
\pgfsetbuttcap%
\pgfsetmiterjoin%
\definecolor{currentfill}{rgb}{0.501961,0.501961,0.501961}%
\pgfsetfillcolor{currentfill}%
\pgfsetfillopacity{0.700000}%
\pgfsetlinewidth{0.000000pt}%
\definecolor{currentstroke}{rgb}{0.000000,0.000000,0.000000}%
\pgfsetstrokecolor{currentstroke}%
\pgfsetstrokeopacity{0.700000}%
\pgfsetdash{}{0pt}%
\pgfpathmoveto{\pgfqpoint{3.060902in}{0.550000in}}%
\pgfpathlineto{\pgfqpoint{3.088423in}{0.550000in}}%
\pgfpathlineto{\pgfqpoint{3.088423in}{0.710785in}}%
\pgfpathlineto{\pgfqpoint{3.060902in}{0.710785in}}%
\pgfpathlineto{\pgfqpoint{3.060902in}{0.550000in}}%
\pgfpathclose%
\pgfusepath{fill}%
\end{pgfscope}%
\begin{pgfscope}%
\pgfpathrectangle{\pgfqpoint{1.250000in}{0.550000in}}{\pgfqpoint{7.750000in}{3.850000in}}%
\pgfusepath{clip}%
\pgfsetbuttcap%
\pgfsetmiterjoin%
\definecolor{currentfill}{rgb}{0.501961,0.501961,0.501961}%
\pgfsetfillcolor{currentfill}%
\pgfsetfillopacity{0.700000}%
\pgfsetlinewidth{0.000000pt}%
\definecolor{currentstroke}{rgb}{0.000000,0.000000,0.000000}%
\pgfsetstrokecolor{currentstroke}%
\pgfsetstrokeopacity{0.700000}%
\pgfsetdash{}{0pt}%
\pgfpathmoveto{\pgfqpoint{3.088423in}{0.550000in}}%
\pgfpathlineto{\pgfqpoint{3.115945in}{0.550000in}}%
\pgfpathlineto{\pgfqpoint{3.115945in}{0.706655in}}%
\pgfpathlineto{\pgfqpoint{3.088423in}{0.706655in}}%
\pgfpathlineto{\pgfqpoint{3.088423in}{0.550000in}}%
\pgfpathclose%
\pgfusepath{fill}%
\end{pgfscope}%
\begin{pgfscope}%
\pgfpathrectangle{\pgfqpoint{1.250000in}{0.550000in}}{\pgfqpoint{7.750000in}{3.850000in}}%
\pgfusepath{clip}%
\pgfsetbuttcap%
\pgfsetmiterjoin%
\definecolor{currentfill}{rgb}{0.501961,0.501961,0.501961}%
\pgfsetfillcolor{currentfill}%
\pgfsetfillopacity{0.700000}%
\pgfsetlinewidth{0.000000pt}%
\definecolor{currentstroke}{rgb}{0.000000,0.000000,0.000000}%
\pgfsetstrokecolor{currentstroke}%
\pgfsetstrokeopacity{0.700000}%
\pgfsetdash{}{0pt}%
\pgfpathmoveto{\pgfqpoint{3.115945in}{0.550000in}}%
\pgfpathlineto{\pgfqpoint{3.143466in}{0.550000in}}%
\pgfpathlineto{\pgfqpoint{3.143466in}{0.704728in}}%
\pgfpathlineto{\pgfqpoint{3.115945in}{0.704728in}}%
\pgfpathlineto{\pgfqpoint{3.115945in}{0.550000in}}%
\pgfpathclose%
\pgfusepath{fill}%
\end{pgfscope}%
\begin{pgfscope}%
\pgfpathrectangle{\pgfqpoint{1.250000in}{0.550000in}}{\pgfqpoint{7.750000in}{3.850000in}}%
\pgfusepath{clip}%
\pgfsetbuttcap%
\pgfsetmiterjoin%
\definecolor{currentfill}{rgb}{0.501961,0.501961,0.501961}%
\pgfsetfillcolor{currentfill}%
\pgfsetfillopacity{0.700000}%
\pgfsetlinewidth{0.000000pt}%
\definecolor{currentstroke}{rgb}{0.000000,0.000000,0.000000}%
\pgfsetstrokecolor{currentstroke}%
\pgfsetstrokeopacity{0.700000}%
\pgfsetdash{}{0pt}%
\pgfpathmoveto{\pgfqpoint{3.143466in}{0.550000in}}%
\pgfpathlineto{\pgfqpoint{3.170987in}{0.550000in}}%
\pgfpathlineto{\pgfqpoint{3.170987in}{0.722073in}}%
\pgfpathlineto{\pgfqpoint{3.143466in}{0.722073in}}%
\pgfpathlineto{\pgfqpoint{3.143466in}{0.550000in}}%
\pgfpathclose%
\pgfusepath{fill}%
\end{pgfscope}%
\begin{pgfscope}%
\pgfpathrectangle{\pgfqpoint{1.250000in}{0.550000in}}{\pgfqpoint{7.750000in}{3.850000in}}%
\pgfusepath{clip}%
\pgfsetbuttcap%
\pgfsetmiterjoin%
\definecolor{currentfill}{rgb}{0.501961,0.501961,0.501961}%
\pgfsetfillcolor{currentfill}%
\pgfsetfillopacity{0.700000}%
\pgfsetlinewidth{0.000000pt}%
\definecolor{currentstroke}{rgb}{0.000000,0.000000,0.000000}%
\pgfsetstrokecolor{currentstroke}%
\pgfsetstrokeopacity{0.700000}%
\pgfsetdash{}{0pt}%
\pgfpathmoveto{\pgfqpoint{3.170987in}{0.550000in}}%
\pgfpathlineto{\pgfqpoint{3.198509in}{0.550000in}}%
\pgfpathlineto{\pgfqpoint{3.198509in}{0.723725in}}%
\pgfpathlineto{\pgfqpoint{3.170987in}{0.723725in}}%
\pgfpathlineto{\pgfqpoint{3.170987in}{0.550000in}}%
\pgfpathclose%
\pgfusepath{fill}%
\end{pgfscope}%
\begin{pgfscope}%
\pgfpathrectangle{\pgfqpoint{1.250000in}{0.550000in}}{\pgfqpoint{7.750000in}{3.850000in}}%
\pgfusepath{clip}%
\pgfsetbuttcap%
\pgfsetmiterjoin%
\definecolor{currentfill}{rgb}{0.501961,0.501961,0.501961}%
\pgfsetfillcolor{currentfill}%
\pgfsetfillopacity{0.700000}%
\pgfsetlinewidth{0.000000pt}%
\definecolor{currentstroke}{rgb}{0.000000,0.000000,0.000000}%
\pgfsetstrokecolor{currentstroke}%
\pgfsetstrokeopacity{0.700000}%
\pgfsetdash{}{0pt}%
\pgfpathmoveto{\pgfqpoint{3.198509in}{0.550000in}}%
\pgfpathlineto{\pgfqpoint{3.226030in}{0.550000in}}%
\pgfpathlineto{\pgfqpoint{3.226030in}{0.724826in}}%
\pgfpathlineto{\pgfqpoint{3.198509in}{0.724826in}}%
\pgfpathlineto{\pgfqpoint{3.198509in}{0.550000in}}%
\pgfpathclose%
\pgfusepath{fill}%
\end{pgfscope}%
\begin{pgfscope}%
\pgfpathrectangle{\pgfqpoint{1.250000in}{0.550000in}}{\pgfqpoint{7.750000in}{3.850000in}}%
\pgfusepath{clip}%
\pgfsetbuttcap%
\pgfsetmiterjoin%
\definecolor{currentfill}{rgb}{0.501961,0.501961,0.501961}%
\pgfsetfillcolor{currentfill}%
\pgfsetfillopacity{0.700000}%
\pgfsetlinewidth{0.000000pt}%
\definecolor{currentstroke}{rgb}{0.000000,0.000000,0.000000}%
\pgfsetstrokecolor{currentstroke}%
\pgfsetstrokeopacity{0.700000}%
\pgfsetdash{}{0pt}%
\pgfpathmoveto{\pgfqpoint{3.226030in}{0.550000in}}%
\pgfpathlineto{\pgfqpoint{3.253551in}{0.550000in}}%
\pgfpathlineto{\pgfqpoint{3.253551in}{0.718218in}}%
\pgfpathlineto{\pgfqpoint{3.226030in}{0.718218in}}%
\pgfpathlineto{\pgfqpoint{3.226030in}{0.550000in}}%
\pgfpathclose%
\pgfusepath{fill}%
\end{pgfscope}%
\begin{pgfscope}%
\pgfpathrectangle{\pgfqpoint{1.250000in}{0.550000in}}{\pgfqpoint{7.750000in}{3.850000in}}%
\pgfusepath{clip}%
\pgfsetbuttcap%
\pgfsetmiterjoin%
\definecolor{currentfill}{rgb}{0.501961,0.501961,0.501961}%
\pgfsetfillcolor{currentfill}%
\pgfsetfillopacity{0.700000}%
\pgfsetlinewidth{0.000000pt}%
\definecolor{currentstroke}{rgb}{0.000000,0.000000,0.000000}%
\pgfsetstrokecolor{currentstroke}%
\pgfsetstrokeopacity{0.700000}%
\pgfsetdash{}{0pt}%
\pgfpathmoveto{\pgfqpoint{3.253551in}{0.550000in}}%
\pgfpathlineto{\pgfqpoint{3.281072in}{0.550000in}}%
\pgfpathlineto{\pgfqpoint{3.281072in}{0.714639in}}%
\pgfpathlineto{\pgfqpoint{3.253551in}{0.714639in}}%
\pgfpathlineto{\pgfqpoint{3.253551in}{0.550000in}}%
\pgfpathclose%
\pgfusepath{fill}%
\end{pgfscope}%
\begin{pgfscope}%
\pgfpathrectangle{\pgfqpoint{1.250000in}{0.550000in}}{\pgfqpoint{7.750000in}{3.850000in}}%
\pgfusepath{clip}%
\pgfsetbuttcap%
\pgfsetmiterjoin%
\definecolor{currentfill}{rgb}{0.501961,0.501961,0.501961}%
\pgfsetfillcolor{currentfill}%
\pgfsetfillopacity{0.700000}%
\pgfsetlinewidth{0.000000pt}%
\definecolor{currentstroke}{rgb}{0.000000,0.000000,0.000000}%
\pgfsetstrokecolor{currentstroke}%
\pgfsetstrokeopacity{0.700000}%
\pgfsetdash{}{0pt}%
\pgfpathmoveto{\pgfqpoint{3.281072in}{0.550000in}}%
\pgfpathlineto{\pgfqpoint{3.308594in}{0.550000in}}%
\pgfpathlineto{\pgfqpoint{3.308594in}{0.710234in}}%
\pgfpathlineto{\pgfqpoint{3.281072in}{0.710234in}}%
\pgfpathlineto{\pgfqpoint{3.281072in}{0.550000in}}%
\pgfpathclose%
\pgfusepath{fill}%
\end{pgfscope}%
\begin{pgfscope}%
\pgfpathrectangle{\pgfqpoint{1.250000in}{0.550000in}}{\pgfqpoint{7.750000in}{3.850000in}}%
\pgfusepath{clip}%
\pgfsetbuttcap%
\pgfsetmiterjoin%
\definecolor{currentfill}{rgb}{0.501961,0.501961,0.501961}%
\pgfsetfillcolor{currentfill}%
\pgfsetfillopacity{0.700000}%
\pgfsetlinewidth{0.000000pt}%
\definecolor{currentstroke}{rgb}{0.000000,0.000000,0.000000}%
\pgfsetstrokecolor{currentstroke}%
\pgfsetstrokeopacity{0.700000}%
\pgfsetdash{}{0pt}%
\pgfpathmoveto{\pgfqpoint{3.308594in}{0.550000in}}%
\pgfpathlineto{\pgfqpoint{3.336115in}{0.550000in}}%
\pgfpathlineto{\pgfqpoint{3.336115in}{0.709408in}}%
\pgfpathlineto{\pgfqpoint{3.308594in}{0.709408in}}%
\pgfpathlineto{\pgfqpoint{3.308594in}{0.550000in}}%
\pgfpathclose%
\pgfusepath{fill}%
\end{pgfscope}%
\begin{pgfscope}%
\pgfpathrectangle{\pgfqpoint{1.250000in}{0.550000in}}{\pgfqpoint{7.750000in}{3.850000in}}%
\pgfusepath{clip}%
\pgfsetbuttcap%
\pgfsetmiterjoin%
\definecolor{currentfill}{rgb}{0.501961,0.501961,0.501961}%
\pgfsetfillcolor{currentfill}%
\pgfsetfillopacity{0.700000}%
\pgfsetlinewidth{0.000000pt}%
\definecolor{currentstroke}{rgb}{0.000000,0.000000,0.000000}%
\pgfsetstrokecolor{currentstroke}%
\pgfsetstrokeopacity{0.700000}%
\pgfsetdash{}{0pt}%
\pgfpathmoveto{\pgfqpoint{3.336115in}{0.550000in}}%
\pgfpathlineto{\pgfqpoint{3.363636in}{0.550000in}}%
\pgfpathlineto{\pgfqpoint{3.363636in}{0.709408in}}%
\pgfpathlineto{\pgfqpoint{3.336115in}{0.709408in}}%
\pgfpathlineto{\pgfqpoint{3.336115in}{0.550000in}}%
\pgfpathclose%
\pgfusepath{fill}%
\end{pgfscope}%
\begin{pgfscope}%
\pgfpathrectangle{\pgfqpoint{1.250000in}{0.550000in}}{\pgfqpoint{7.750000in}{3.850000in}}%
\pgfusepath{clip}%
\pgfsetbuttcap%
\pgfsetmiterjoin%
\definecolor{currentfill}{rgb}{0.501961,0.501961,0.501961}%
\pgfsetfillcolor{currentfill}%
\pgfsetfillopacity{0.700000}%
\pgfsetlinewidth{0.000000pt}%
\definecolor{currentstroke}{rgb}{0.000000,0.000000,0.000000}%
\pgfsetstrokecolor{currentstroke}%
\pgfsetstrokeopacity{0.700000}%
\pgfsetdash{}{0pt}%
\pgfpathmoveto{\pgfqpoint{3.363636in}{0.550000in}}%
\pgfpathlineto{\pgfqpoint{3.391158in}{0.550000in}}%
\pgfpathlineto{\pgfqpoint{3.391158in}{0.692063in}}%
\pgfpathlineto{\pgfqpoint{3.363636in}{0.692063in}}%
\pgfpathlineto{\pgfqpoint{3.363636in}{0.550000in}}%
\pgfpathclose%
\pgfusepath{fill}%
\end{pgfscope}%
\begin{pgfscope}%
\pgfpathrectangle{\pgfqpoint{1.250000in}{0.550000in}}{\pgfqpoint{7.750000in}{3.850000in}}%
\pgfusepath{clip}%
\pgfsetbuttcap%
\pgfsetmiterjoin%
\definecolor{currentfill}{rgb}{0.501961,0.501961,0.501961}%
\pgfsetfillcolor{currentfill}%
\pgfsetfillopacity{0.700000}%
\pgfsetlinewidth{0.000000pt}%
\definecolor{currentstroke}{rgb}{0.000000,0.000000,0.000000}%
\pgfsetstrokecolor{currentstroke}%
\pgfsetstrokeopacity{0.700000}%
\pgfsetdash{}{0pt}%
\pgfpathmoveto{\pgfqpoint{3.391158in}{0.550000in}}%
\pgfpathlineto{\pgfqpoint{3.418679in}{0.550000in}}%
\pgfpathlineto{\pgfqpoint{3.418679in}{0.692063in}}%
\pgfpathlineto{\pgfqpoint{3.391158in}{0.692063in}}%
\pgfpathlineto{\pgfqpoint{3.391158in}{0.550000in}}%
\pgfpathclose%
\pgfusepath{fill}%
\end{pgfscope}%
\begin{pgfscope}%
\pgfpathrectangle{\pgfqpoint{1.250000in}{0.550000in}}{\pgfqpoint{7.750000in}{3.850000in}}%
\pgfusepath{clip}%
\pgfsetbuttcap%
\pgfsetmiterjoin%
\definecolor{currentfill}{rgb}{0.501961,0.501961,0.501961}%
\pgfsetfillcolor{currentfill}%
\pgfsetfillopacity{0.700000}%
\pgfsetlinewidth{0.000000pt}%
\definecolor{currentstroke}{rgb}{0.000000,0.000000,0.000000}%
\pgfsetstrokecolor{currentstroke}%
\pgfsetstrokeopacity{0.700000}%
\pgfsetdash{}{0pt}%
\pgfpathmoveto{\pgfqpoint{3.418679in}{0.550000in}}%
\pgfpathlineto{\pgfqpoint{3.446200in}{0.550000in}}%
\pgfpathlineto{\pgfqpoint{3.446200in}{0.684905in}}%
\pgfpathlineto{\pgfqpoint{3.418679in}{0.684905in}}%
\pgfpathlineto{\pgfqpoint{3.418679in}{0.550000in}}%
\pgfpathclose%
\pgfusepath{fill}%
\end{pgfscope}%
\begin{pgfscope}%
\pgfpathrectangle{\pgfqpoint{1.250000in}{0.550000in}}{\pgfqpoint{7.750000in}{3.850000in}}%
\pgfusepath{clip}%
\pgfsetbuttcap%
\pgfsetmiterjoin%
\definecolor{currentfill}{rgb}{0.501961,0.501961,0.501961}%
\pgfsetfillcolor{currentfill}%
\pgfsetfillopacity{0.700000}%
\pgfsetlinewidth{0.000000pt}%
\definecolor{currentstroke}{rgb}{0.000000,0.000000,0.000000}%
\pgfsetstrokecolor{currentstroke}%
\pgfsetstrokeopacity{0.700000}%
\pgfsetdash{}{0pt}%
\pgfpathmoveto{\pgfqpoint{3.446200in}{0.550000in}}%
\pgfpathlineto{\pgfqpoint{3.473722in}{0.550000in}}%
\pgfpathlineto{\pgfqpoint{3.473722in}{0.688209in}}%
\pgfpathlineto{\pgfqpoint{3.446200in}{0.688209in}}%
\pgfpathlineto{\pgfqpoint{3.446200in}{0.550000in}}%
\pgfpathclose%
\pgfusepath{fill}%
\end{pgfscope}%
\begin{pgfscope}%
\pgfpathrectangle{\pgfqpoint{1.250000in}{0.550000in}}{\pgfqpoint{7.750000in}{3.850000in}}%
\pgfusepath{clip}%
\pgfsetbuttcap%
\pgfsetmiterjoin%
\definecolor{currentfill}{rgb}{0.501961,0.501961,0.501961}%
\pgfsetfillcolor{currentfill}%
\pgfsetfillopacity{0.700000}%
\pgfsetlinewidth{0.000000pt}%
\definecolor{currentstroke}{rgb}{0.000000,0.000000,0.000000}%
\pgfsetstrokecolor{currentstroke}%
\pgfsetstrokeopacity{0.700000}%
\pgfsetdash{}{0pt}%
\pgfpathmoveto{\pgfqpoint{3.473722in}{0.550000in}}%
\pgfpathlineto{\pgfqpoint{3.501243in}{0.550000in}}%
\pgfpathlineto{\pgfqpoint{3.501243in}{0.700048in}}%
\pgfpathlineto{\pgfqpoint{3.473722in}{0.700048in}}%
\pgfpathlineto{\pgfqpoint{3.473722in}{0.550000in}}%
\pgfpathclose%
\pgfusepath{fill}%
\end{pgfscope}%
\begin{pgfscope}%
\pgfpathrectangle{\pgfqpoint{1.250000in}{0.550000in}}{\pgfqpoint{7.750000in}{3.850000in}}%
\pgfusepath{clip}%
\pgfsetbuttcap%
\pgfsetmiterjoin%
\definecolor{currentfill}{rgb}{0.501961,0.501961,0.501961}%
\pgfsetfillcolor{currentfill}%
\pgfsetfillopacity{0.700000}%
\pgfsetlinewidth{0.000000pt}%
\definecolor{currentstroke}{rgb}{0.000000,0.000000,0.000000}%
\pgfsetstrokecolor{currentstroke}%
\pgfsetstrokeopacity{0.700000}%
\pgfsetdash{}{0pt}%
\pgfpathmoveto{\pgfqpoint{3.501243in}{0.550000in}}%
\pgfpathlineto{\pgfqpoint{3.528764in}{0.550000in}}%
\pgfpathlineto{\pgfqpoint{3.528764in}{0.694817in}}%
\pgfpathlineto{\pgfqpoint{3.501243in}{0.694817in}}%
\pgfpathlineto{\pgfqpoint{3.501243in}{0.550000in}}%
\pgfpathclose%
\pgfusepath{fill}%
\end{pgfscope}%
\begin{pgfscope}%
\pgfpathrectangle{\pgfqpoint{1.250000in}{0.550000in}}{\pgfqpoint{7.750000in}{3.850000in}}%
\pgfusepath{clip}%
\pgfsetbuttcap%
\pgfsetmiterjoin%
\definecolor{currentfill}{rgb}{0.501961,0.501961,0.501961}%
\pgfsetfillcolor{currentfill}%
\pgfsetfillopacity{0.700000}%
\pgfsetlinewidth{0.000000pt}%
\definecolor{currentstroke}{rgb}{0.000000,0.000000,0.000000}%
\pgfsetstrokecolor{currentstroke}%
\pgfsetstrokeopacity{0.700000}%
\pgfsetdash{}{0pt}%
\pgfpathmoveto{\pgfqpoint{3.528764in}{0.550000in}}%
\pgfpathlineto{\pgfqpoint{3.556286in}{0.550000in}}%
\pgfpathlineto{\pgfqpoint{3.556286in}{0.700048in}}%
\pgfpathlineto{\pgfqpoint{3.528764in}{0.700048in}}%
\pgfpathlineto{\pgfqpoint{3.528764in}{0.550000in}}%
\pgfpathclose%
\pgfusepath{fill}%
\end{pgfscope}%
\begin{pgfscope}%
\pgfpathrectangle{\pgfqpoint{1.250000in}{0.550000in}}{\pgfqpoint{7.750000in}{3.850000in}}%
\pgfusepath{clip}%
\pgfsetbuttcap%
\pgfsetmiterjoin%
\definecolor{currentfill}{rgb}{0.501961,0.501961,0.501961}%
\pgfsetfillcolor{currentfill}%
\pgfsetfillopacity{0.700000}%
\pgfsetlinewidth{0.000000pt}%
\definecolor{currentstroke}{rgb}{0.000000,0.000000,0.000000}%
\pgfsetstrokecolor{currentstroke}%
\pgfsetstrokeopacity{0.700000}%
\pgfsetdash{}{0pt}%
\pgfpathmoveto{\pgfqpoint{3.556286in}{0.550000in}}%
\pgfpathlineto{\pgfqpoint{3.583807in}{0.550000in}}%
\pgfpathlineto{\pgfqpoint{3.583807in}{0.697294in}}%
\pgfpathlineto{\pgfqpoint{3.556286in}{0.697294in}}%
\pgfpathlineto{\pgfqpoint{3.556286in}{0.550000in}}%
\pgfpathclose%
\pgfusepath{fill}%
\end{pgfscope}%
\begin{pgfscope}%
\pgfpathrectangle{\pgfqpoint{1.250000in}{0.550000in}}{\pgfqpoint{7.750000in}{3.850000in}}%
\pgfusepath{clip}%
\pgfsetbuttcap%
\pgfsetmiterjoin%
\definecolor{currentfill}{rgb}{0.501961,0.501961,0.501961}%
\pgfsetfillcolor{currentfill}%
\pgfsetfillopacity{0.700000}%
\pgfsetlinewidth{0.000000pt}%
\definecolor{currentstroke}{rgb}{0.000000,0.000000,0.000000}%
\pgfsetstrokecolor{currentstroke}%
\pgfsetstrokeopacity{0.700000}%
\pgfsetdash{}{0pt}%
\pgfpathmoveto{\pgfqpoint{3.583807in}{0.550000in}}%
\pgfpathlineto{\pgfqpoint{3.611328in}{0.550000in}}%
\pgfpathlineto{\pgfqpoint{3.611328in}{0.687108in}}%
\pgfpathlineto{\pgfqpoint{3.583807in}{0.687108in}}%
\pgfpathlineto{\pgfqpoint{3.583807in}{0.550000in}}%
\pgfpathclose%
\pgfusepath{fill}%
\end{pgfscope}%
\begin{pgfscope}%
\pgfpathrectangle{\pgfqpoint{1.250000in}{0.550000in}}{\pgfqpoint{7.750000in}{3.850000in}}%
\pgfusepath{clip}%
\pgfsetbuttcap%
\pgfsetmiterjoin%
\definecolor{currentfill}{rgb}{0.501961,0.501961,0.501961}%
\pgfsetfillcolor{currentfill}%
\pgfsetfillopacity{0.700000}%
\pgfsetlinewidth{0.000000pt}%
\definecolor{currentstroke}{rgb}{0.000000,0.000000,0.000000}%
\pgfsetstrokecolor{currentstroke}%
\pgfsetstrokeopacity{0.700000}%
\pgfsetdash{}{0pt}%
\pgfpathmoveto{\pgfqpoint{3.611328in}{0.550000in}}%
\pgfpathlineto{\pgfqpoint{3.638849in}{0.550000in}}%
\pgfpathlineto{\pgfqpoint{3.638849in}{0.684355in}}%
\pgfpathlineto{\pgfqpoint{3.611328in}{0.684355in}}%
\pgfpathlineto{\pgfqpoint{3.611328in}{0.550000in}}%
\pgfpathclose%
\pgfusepath{fill}%
\end{pgfscope}%
\begin{pgfscope}%
\pgfpathrectangle{\pgfqpoint{1.250000in}{0.550000in}}{\pgfqpoint{7.750000in}{3.850000in}}%
\pgfusepath{clip}%
\pgfsetbuttcap%
\pgfsetmiterjoin%
\definecolor{currentfill}{rgb}{0.501961,0.501961,0.501961}%
\pgfsetfillcolor{currentfill}%
\pgfsetfillopacity{0.700000}%
\pgfsetlinewidth{0.000000pt}%
\definecolor{currentstroke}{rgb}{0.000000,0.000000,0.000000}%
\pgfsetstrokecolor{currentstroke}%
\pgfsetstrokeopacity{0.700000}%
\pgfsetdash{}{0pt}%
\pgfpathmoveto{\pgfqpoint{3.638849in}{0.550000in}}%
\pgfpathlineto{\pgfqpoint{3.666371in}{0.550000in}}%
\pgfpathlineto{\pgfqpoint{3.666371in}{0.683804in}}%
\pgfpathlineto{\pgfqpoint{3.638849in}{0.683804in}}%
\pgfpathlineto{\pgfqpoint{3.638849in}{0.550000in}}%
\pgfpathclose%
\pgfusepath{fill}%
\end{pgfscope}%
\begin{pgfscope}%
\pgfpathrectangle{\pgfqpoint{1.250000in}{0.550000in}}{\pgfqpoint{7.750000in}{3.850000in}}%
\pgfusepath{clip}%
\pgfsetbuttcap%
\pgfsetmiterjoin%
\definecolor{currentfill}{rgb}{0.501961,0.501961,0.501961}%
\pgfsetfillcolor{currentfill}%
\pgfsetfillopacity{0.700000}%
\pgfsetlinewidth{0.000000pt}%
\definecolor{currentstroke}{rgb}{0.000000,0.000000,0.000000}%
\pgfsetstrokecolor{currentstroke}%
\pgfsetstrokeopacity{0.700000}%
\pgfsetdash{}{0pt}%
\pgfpathmoveto{\pgfqpoint{3.666371in}{0.550000in}}%
\pgfpathlineto{\pgfqpoint{3.693892in}{0.550000in}}%
\pgfpathlineto{\pgfqpoint{3.693892in}{0.672241in}}%
\pgfpathlineto{\pgfqpoint{3.666371in}{0.672241in}}%
\pgfpathlineto{\pgfqpoint{3.666371in}{0.550000in}}%
\pgfpathclose%
\pgfusepath{fill}%
\end{pgfscope}%
\begin{pgfscope}%
\pgfpathrectangle{\pgfqpoint{1.250000in}{0.550000in}}{\pgfqpoint{7.750000in}{3.850000in}}%
\pgfusepath{clip}%
\pgfsetbuttcap%
\pgfsetmiterjoin%
\definecolor{currentfill}{rgb}{0.501961,0.501961,0.501961}%
\pgfsetfillcolor{currentfill}%
\pgfsetfillopacity{0.700000}%
\pgfsetlinewidth{0.000000pt}%
\definecolor{currentstroke}{rgb}{0.000000,0.000000,0.000000}%
\pgfsetstrokecolor{currentstroke}%
\pgfsetstrokeopacity{0.700000}%
\pgfsetdash{}{0pt}%
\pgfpathmoveto{\pgfqpoint{3.693892in}{0.550000in}}%
\pgfpathlineto{\pgfqpoint{3.721413in}{0.550000in}}%
\pgfpathlineto{\pgfqpoint{3.721413in}{0.679399in}}%
\pgfpathlineto{\pgfqpoint{3.693892in}{0.679399in}}%
\pgfpathlineto{\pgfqpoint{3.693892in}{0.550000in}}%
\pgfpathclose%
\pgfusepath{fill}%
\end{pgfscope}%
\begin{pgfscope}%
\pgfpathrectangle{\pgfqpoint{1.250000in}{0.550000in}}{\pgfqpoint{7.750000in}{3.850000in}}%
\pgfusepath{clip}%
\pgfsetbuttcap%
\pgfsetmiterjoin%
\definecolor{currentfill}{rgb}{0.501961,0.501961,0.501961}%
\pgfsetfillcolor{currentfill}%
\pgfsetfillopacity{0.700000}%
\pgfsetlinewidth{0.000000pt}%
\definecolor{currentstroke}{rgb}{0.000000,0.000000,0.000000}%
\pgfsetstrokecolor{currentstroke}%
\pgfsetstrokeopacity{0.700000}%
\pgfsetdash{}{0pt}%
\pgfpathmoveto{\pgfqpoint{3.721413in}{0.550000in}}%
\pgfpathlineto{\pgfqpoint{3.748935in}{0.550000in}}%
\pgfpathlineto{\pgfqpoint{3.748935in}{0.665082in}}%
\pgfpathlineto{\pgfqpoint{3.721413in}{0.665082in}}%
\pgfpathlineto{\pgfqpoint{3.721413in}{0.550000in}}%
\pgfpathclose%
\pgfusepath{fill}%
\end{pgfscope}%
\begin{pgfscope}%
\pgfpathrectangle{\pgfqpoint{1.250000in}{0.550000in}}{\pgfqpoint{7.750000in}{3.850000in}}%
\pgfusepath{clip}%
\pgfsetbuttcap%
\pgfsetmiterjoin%
\definecolor{currentfill}{rgb}{0.501961,0.501961,0.501961}%
\pgfsetfillcolor{currentfill}%
\pgfsetfillopacity{0.700000}%
\pgfsetlinewidth{0.000000pt}%
\definecolor{currentstroke}{rgb}{0.000000,0.000000,0.000000}%
\pgfsetstrokecolor{currentstroke}%
\pgfsetstrokeopacity{0.700000}%
\pgfsetdash{}{0pt}%
\pgfpathmoveto{\pgfqpoint{3.748935in}{0.550000in}}%
\pgfpathlineto{\pgfqpoint{3.776456in}{0.550000in}}%
\pgfpathlineto{\pgfqpoint{3.776456in}{0.667010in}}%
\pgfpathlineto{\pgfqpoint{3.748935in}{0.667010in}}%
\pgfpathlineto{\pgfqpoint{3.748935in}{0.550000in}}%
\pgfpathclose%
\pgfusepath{fill}%
\end{pgfscope}%
\begin{pgfscope}%
\pgfpathrectangle{\pgfqpoint{1.250000in}{0.550000in}}{\pgfqpoint{7.750000in}{3.850000in}}%
\pgfusepath{clip}%
\pgfsetbuttcap%
\pgfsetmiterjoin%
\definecolor{currentfill}{rgb}{0.501961,0.501961,0.501961}%
\pgfsetfillcolor{currentfill}%
\pgfsetfillopacity{0.700000}%
\pgfsetlinewidth{0.000000pt}%
\definecolor{currentstroke}{rgb}{0.000000,0.000000,0.000000}%
\pgfsetstrokecolor{currentstroke}%
\pgfsetstrokeopacity{0.700000}%
\pgfsetdash{}{0pt}%
\pgfpathmoveto{\pgfqpoint{3.776456in}{0.550000in}}%
\pgfpathlineto{\pgfqpoint{3.803977in}{0.550000in}}%
\pgfpathlineto{\pgfqpoint{3.803977in}{0.660402in}}%
\pgfpathlineto{\pgfqpoint{3.776456in}{0.660402in}}%
\pgfpathlineto{\pgfqpoint{3.776456in}{0.550000in}}%
\pgfpathclose%
\pgfusepath{fill}%
\end{pgfscope}%
\begin{pgfscope}%
\pgfpathrectangle{\pgfqpoint{1.250000in}{0.550000in}}{\pgfqpoint{7.750000in}{3.850000in}}%
\pgfusepath{clip}%
\pgfsetbuttcap%
\pgfsetmiterjoin%
\definecolor{currentfill}{rgb}{0.501961,0.501961,0.501961}%
\pgfsetfillcolor{currentfill}%
\pgfsetfillopacity{0.700000}%
\pgfsetlinewidth{0.000000pt}%
\definecolor{currentstroke}{rgb}{0.000000,0.000000,0.000000}%
\pgfsetstrokecolor{currentstroke}%
\pgfsetstrokeopacity{0.700000}%
\pgfsetdash{}{0pt}%
\pgfpathmoveto{\pgfqpoint{3.803977in}{0.550000in}}%
\pgfpathlineto{\pgfqpoint{3.831499in}{0.550000in}}%
\pgfpathlineto{\pgfqpoint{3.831499in}{0.651867in}}%
\pgfpathlineto{\pgfqpoint{3.803977in}{0.651867in}}%
\pgfpathlineto{\pgfqpoint{3.803977in}{0.550000in}}%
\pgfpathclose%
\pgfusepath{fill}%
\end{pgfscope}%
\begin{pgfscope}%
\pgfpathrectangle{\pgfqpoint{1.250000in}{0.550000in}}{\pgfqpoint{7.750000in}{3.850000in}}%
\pgfusepath{clip}%
\pgfsetbuttcap%
\pgfsetmiterjoin%
\definecolor{currentfill}{rgb}{0.501961,0.501961,0.501961}%
\pgfsetfillcolor{currentfill}%
\pgfsetfillopacity{0.700000}%
\pgfsetlinewidth{0.000000pt}%
\definecolor{currentstroke}{rgb}{0.000000,0.000000,0.000000}%
\pgfsetstrokecolor{currentstroke}%
\pgfsetstrokeopacity{0.700000}%
\pgfsetdash{}{0pt}%
\pgfpathmoveto{\pgfqpoint{3.831499in}{0.550000in}}%
\pgfpathlineto{\pgfqpoint{3.859020in}{0.550000in}}%
\pgfpathlineto{\pgfqpoint{3.859020in}{0.648288in}}%
\pgfpathlineto{\pgfqpoint{3.831499in}{0.648288in}}%
\pgfpathlineto{\pgfqpoint{3.831499in}{0.550000in}}%
\pgfpathclose%
\pgfusepath{fill}%
\end{pgfscope}%
\begin{pgfscope}%
\pgfpathrectangle{\pgfqpoint{1.250000in}{0.550000in}}{\pgfqpoint{7.750000in}{3.850000in}}%
\pgfusepath{clip}%
\pgfsetbuttcap%
\pgfsetmiterjoin%
\definecolor{currentfill}{rgb}{0.501961,0.501961,0.501961}%
\pgfsetfillcolor{currentfill}%
\pgfsetfillopacity{0.700000}%
\pgfsetlinewidth{0.000000pt}%
\definecolor{currentstroke}{rgb}{0.000000,0.000000,0.000000}%
\pgfsetstrokecolor{currentstroke}%
\pgfsetstrokeopacity{0.700000}%
\pgfsetdash{}{0pt}%
\pgfpathmoveto{\pgfqpoint{3.859020in}{0.550000in}}%
\pgfpathlineto{\pgfqpoint{3.886541in}{0.550000in}}%
\pgfpathlineto{\pgfqpoint{3.886541in}{0.647462in}}%
\pgfpathlineto{\pgfqpoint{3.859020in}{0.647462in}}%
\pgfpathlineto{\pgfqpoint{3.859020in}{0.550000in}}%
\pgfpathclose%
\pgfusepath{fill}%
\end{pgfscope}%
\begin{pgfscope}%
\pgfpathrectangle{\pgfqpoint{1.250000in}{0.550000in}}{\pgfqpoint{7.750000in}{3.850000in}}%
\pgfusepath{clip}%
\pgfsetbuttcap%
\pgfsetmiterjoin%
\definecolor{currentfill}{rgb}{0.501961,0.501961,0.501961}%
\pgfsetfillcolor{currentfill}%
\pgfsetfillopacity{0.700000}%
\pgfsetlinewidth{0.000000pt}%
\definecolor{currentstroke}{rgb}{0.000000,0.000000,0.000000}%
\pgfsetstrokecolor{currentstroke}%
\pgfsetstrokeopacity{0.700000}%
\pgfsetdash{}{0pt}%
\pgfpathmoveto{\pgfqpoint{3.886541in}{0.550000in}}%
\pgfpathlineto{\pgfqpoint{3.914062in}{0.550000in}}%
\pgfpathlineto{\pgfqpoint{3.914062in}{0.654620in}}%
\pgfpathlineto{\pgfqpoint{3.886541in}{0.654620in}}%
\pgfpathlineto{\pgfqpoint{3.886541in}{0.550000in}}%
\pgfpathclose%
\pgfusepath{fill}%
\end{pgfscope}%
\begin{pgfscope}%
\pgfpathrectangle{\pgfqpoint{1.250000in}{0.550000in}}{\pgfqpoint{7.750000in}{3.850000in}}%
\pgfusepath{clip}%
\pgfsetbuttcap%
\pgfsetmiterjoin%
\definecolor{currentfill}{rgb}{0.501961,0.501961,0.501961}%
\pgfsetfillcolor{currentfill}%
\pgfsetfillopacity{0.700000}%
\pgfsetlinewidth{0.000000pt}%
\definecolor{currentstroke}{rgb}{0.000000,0.000000,0.000000}%
\pgfsetstrokecolor{currentstroke}%
\pgfsetstrokeopacity{0.700000}%
\pgfsetdash{}{0pt}%
\pgfpathmoveto{\pgfqpoint{3.914062in}{0.550000in}}%
\pgfpathlineto{\pgfqpoint{3.941584in}{0.550000in}}%
\pgfpathlineto{\pgfqpoint{3.941584in}{0.651041in}}%
\pgfpathlineto{\pgfqpoint{3.914062in}{0.651041in}}%
\pgfpathlineto{\pgfqpoint{3.914062in}{0.550000in}}%
\pgfpathclose%
\pgfusepath{fill}%
\end{pgfscope}%
\begin{pgfscope}%
\pgfpathrectangle{\pgfqpoint{1.250000in}{0.550000in}}{\pgfqpoint{7.750000in}{3.850000in}}%
\pgfusepath{clip}%
\pgfsetbuttcap%
\pgfsetmiterjoin%
\definecolor{currentfill}{rgb}{0.501961,0.501961,0.501961}%
\pgfsetfillcolor{currentfill}%
\pgfsetfillopacity{0.700000}%
\pgfsetlinewidth{0.000000pt}%
\definecolor{currentstroke}{rgb}{0.000000,0.000000,0.000000}%
\pgfsetstrokecolor{currentstroke}%
\pgfsetstrokeopacity{0.700000}%
\pgfsetdash{}{0pt}%
\pgfpathmoveto{\pgfqpoint{3.941584in}{0.550000in}}%
\pgfpathlineto{\pgfqpoint{3.969105in}{0.550000in}}%
\pgfpathlineto{\pgfqpoint{3.969105in}{0.655722in}}%
\pgfpathlineto{\pgfqpoint{3.941584in}{0.655722in}}%
\pgfpathlineto{\pgfqpoint{3.941584in}{0.550000in}}%
\pgfpathclose%
\pgfusepath{fill}%
\end{pgfscope}%
\begin{pgfscope}%
\pgfpathrectangle{\pgfqpoint{1.250000in}{0.550000in}}{\pgfqpoint{7.750000in}{3.850000in}}%
\pgfusepath{clip}%
\pgfsetbuttcap%
\pgfsetmiterjoin%
\definecolor{currentfill}{rgb}{0.501961,0.501961,0.501961}%
\pgfsetfillcolor{currentfill}%
\pgfsetfillopacity{0.700000}%
\pgfsetlinewidth{0.000000pt}%
\definecolor{currentstroke}{rgb}{0.000000,0.000000,0.000000}%
\pgfsetstrokecolor{currentstroke}%
\pgfsetstrokeopacity{0.700000}%
\pgfsetdash{}{0pt}%
\pgfpathmoveto{\pgfqpoint{3.969105in}{0.550000in}}%
\pgfpathlineto{\pgfqpoint{3.996626in}{0.550000in}}%
\pgfpathlineto{\pgfqpoint{3.996626in}{0.654070in}}%
\pgfpathlineto{\pgfqpoint{3.969105in}{0.654070in}}%
\pgfpathlineto{\pgfqpoint{3.969105in}{0.550000in}}%
\pgfpathclose%
\pgfusepath{fill}%
\end{pgfscope}%
\begin{pgfscope}%
\pgfpathrectangle{\pgfqpoint{1.250000in}{0.550000in}}{\pgfqpoint{7.750000in}{3.850000in}}%
\pgfusepath{clip}%
\pgfsetbuttcap%
\pgfsetmiterjoin%
\definecolor{currentfill}{rgb}{0.501961,0.501961,0.501961}%
\pgfsetfillcolor{currentfill}%
\pgfsetfillopacity{0.700000}%
\pgfsetlinewidth{0.000000pt}%
\definecolor{currentstroke}{rgb}{0.000000,0.000000,0.000000}%
\pgfsetstrokecolor{currentstroke}%
\pgfsetstrokeopacity{0.700000}%
\pgfsetdash{}{0pt}%
\pgfpathmoveto{\pgfqpoint{3.996626in}{0.550000in}}%
\pgfpathlineto{\pgfqpoint{4.024148in}{0.550000in}}%
\pgfpathlineto{\pgfqpoint{4.024148in}{0.642782in}}%
\pgfpathlineto{\pgfqpoint{3.996626in}{0.642782in}}%
\pgfpathlineto{\pgfqpoint{3.996626in}{0.550000in}}%
\pgfpathclose%
\pgfusepath{fill}%
\end{pgfscope}%
\begin{pgfscope}%
\pgfpathrectangle{\pgfqpoint{1.250000in}{0.550000in}}{\pgfqpoint{7.750000in}{3.850000in}}%
\pgfusepath{clip}%
\pgfsetbuttcap%
\pgfsetmiterjoin%
\definecolor{currentfill}{rgb}{0.501961,0.501961,0.501961}%
\pgfsetfillcolor{currentfill}%
\pgfsetfillopacity{0.700000}%
\pgfsetlinewidth{0.000000pt}%
\definecolor{currentstroke}{rgb}{0.000000,0.000000,0.000000}%
\pgfsetstrokecolor{currentstroke}%
\pgfsetstrokeopacity{0.700000}%
\pgfsetdash{}{0pt}%
\pgfpathmoveto{\pgfqpoint{4.024148in}{0.550000in}}%
\pgfpathlineto{\pgfqpoint{4.051669in}{0.550000in}}%
\pgfpathlineto{\pgfqpoint{4.051669in}{0.647462in}}%
\pgfpathlineto{\pgfqpoint{4.024148in}{0.647462in}}%
\pgfpathlineto{\pgfqpoint{4.024148in}{0.550000in}}%
\pgfpathclose%
\pgfusepath{fill}%
\end{pgfscope}%
\begin{pgfscope}%
\pgfpathrectangle{\pgfqpoint{1.250000in}{0.550000in}}{\pgfqpoint{7.750000in}{3.850000in}}%
\pgfusepath{clip}%
\pgfsetbuttcap%
\pgfsetmiterjoin%
\definecolor{currentfill}{rgb}{0.501961,0.501961,0.501961}%
\pgfsetfillcolor{currentfill}%
\pgfsetfillopacity{0.700000}%
\pgfsetlinewidth{0.000000pt}%
\definecolor{currentstroke}{rgb}{0.000000,0.000000,0.000000}%
\pgfsetstrokecolor{currentstroke}%
\pgfsetstrokeopacity{0.700000}%
\pgfsetdash{}{0pt}%
\pgfpathmoveto{\pgfqpoint{4.051669in}{0.550000in}}%
\pgfpathlineto{\pgfqpoint{4.079190in}{0.550000in}}%
\pgfpathlineto{\pgfqpoint{4.079190in}{0.633696in}}%
\pgfpathlineto{\pgfqpoint{4.051669in}{0.633696in}}%
\pgfpathlineto{\pgfqpoint{4.051669in}{0.550000in}}%
\pgfpathclose%
\pgfusepath{fill}%
\end{pgfscope}%
\begin{pgfscope}%
\pgfpathrectangle{\pgfqpoint{1.250000in}{0.550000in}}{\pgfqpoint{7.750000in}{3.850000in}}%
\pgfusepath{clip}%
\pgfsetbuttcap%
\pgfsetmiterjoin%
\definecolor{currentfill}{rgb}{0.501961,0.501961,0.501961}%
\pgfsetfillcolor{currentfill}%
\pgfsetfillopacity{0.700000}%
\pgfsetlinewidth{0.000000pt}%
\definecolor{currentstroke}{rgb}{0.000000,0.000000,0.000000}%
\pgfsetstrokecolor{currentstroke}%
\pgfsetstrokeopacity{0.700000}%
\pgfsetdash{}{0pt}%
\pgfpathmoveto{\pgfqpoint{4.079190in}{0.550000in}}%
\pgfpathlineto{\pgfqpoint{4.106712in}{0.550000in}}%
\pgfpathlineto{\pgfqpoint{4.106712in}{0.654620in}}%
\pgfpathlineto{\pgfqpoint{4.079190in}{0.654620in}}%
\pgfpathlineto{\pgfqpoint{4.079190in}{0.550000in}}%
\pgfpathclose%
\pgfusepath{fill}%
\end{pgfscope}%
\begin{pgfscope}%
\pgfpathrectangle{\pgfqpoint{1.250000in}{0.550000in}}{\pgfqpoint{7.750000in}{3.850000in}}%
\pgfusepath{clip}%
\pgfsetbuttcap%
\pgfsetmiterjoin%
\definecolor{currentfill}{rgb}{0.501961,0.501961,0.501961}%
\pgfsetfillcolor{currentfill}%
\pgfsetfillopacity{0.700000}%
\pgfsetlinewidth{0.000000pt}%
\definecolor{currentstroke}{rgb}{0.000000,0.000000,0.000000}%
\pgfsetstrokecolor{currentstroke}%
\pgfsetstrokeopacity{0.700000}%
\pgfsetdash{}{0pt}%
\pgfpathmoveto{\pgfqpoint{4.106712in}{0.550000in}}%
\pgfpathlineto{\pgfqpoint{4.134233in}{0.550000in}}%
\pgfpathlineto{\pgfqpoint{4.134233in}{0.652142in}}%
\pgfpathlineto{\pgfqpoint{4.106712in}{0.652142in}}%
\pgfpathlineto{\pgfqpoint{4.106712in}{0.550000in}}%
\pgfpathclose%
\pgfusepath{fill}%
\end{pgfscope}%
\begin{pgfscope}%
\pgfpathrectangle{\pgfqpoint{1.250000in}{0.550000in}}{\pgfqpoint{7.750000in}{3.850000in}}%
\pgfusepath{clip}%
\pgfsetbuttcap%
\pgfsetmiterjoin%
\definecolor{currentfill}{rgb}{0.501961,0.501961,0.501961}%
\pgfsetfillcolor{currentfill}%
\pgfsetfillopacity{0.700000}%
\pgfsetlinewidth{0.000000pt}%
\definecolor{currentstroke}{rgb}{0.000000,0.000000,0.000000}%
\pgfsetstrokecolor{currentstroke}%
\pgfsetstrokeopacity{0.700000}%
\pgfsetdash{}{0pt}%
\pgfpathmoveto{\pgfqpoint{4.134233in}{0.550000in}}%
\pgfpathlineto{\pgfqpoint{4.161754in}{0.550000in}}%
\pgfpathlineto{\pgfqpoint{4.161754in}{0.646361in}}%
\pgfpathlineto{\pgfqpoint{4.134233in}{0.646361in}}%
\pgfpathlineto{\pgfqpoint{4.134233in}{0.550000in}}%
\pgfpathclose%
\pgfusepath{fill}%
\end{pgfscope}%
\begin{pgfscope}%
\pgfpathrectangle{\pgfqpoint{1.250000in}{0.550000in}}{\pgfqpoint{7.750000in}{3.850000in}}%
\pgfusepath{clip}%
\pgfsetbuttcap%
\pgfsetmiterjoin%
\definecolor{currentfill}{rgb}{0.501961,0.501961,0.501961}%
\pgfsetfillcolor{currentfill}%
\pgfsetfillopacity{0.700000}%
\pgfsetlinewidth{0.000000pt}%
\definecolor{currentstroke}{rgb}{0.000000,0.000000,0.000000}%
\pgfsetstrokecolor{currentstroke}%
\pgfsetstrokeopacity{0.700000}%
\pgfsetdash{}{0pt}%
\pgfpathmoveto{\pgfqpoint{4.161754in}{0.550000in}}%
\pgfpathlineto{\pgfqpoint{4.189276in}{0.550000in}}%
\pgfpathlineto{\pgfqpoint{4.189276in}{0.651592in}}%
\pgfpathlineto{\pgfqpoint{4.161754in}{0.651592in}}%
\pgfpathlineto{\pgfqpoint{4.161754in}{0.550000in}}%
\pgfpathclose%
\pgfusepath{fill}%
\end{pgfscope}%
\begin{pgfscope}%
\pgfpathrectangle{\pgfqpoint{1.250000in}{0.550000in}}{\pgfqpoint{7.750000in}{3.850000in}}%
\pgfusepath{clip}%
\pgfsetbuttcap%
\pgfsetmiterjoin%
\definecolor{currentfill}{rgb}{0.501961,0.501961,0.501961}%
\pgfsetfillcolor{currentfill}%
\pgfsetfillopacity{0.700000}%
\pgfsetlinewidth{0.000000pt}%
\definecolor{currentstroke}{rgb}{0.000000,0.000000,0.000000}%
\pgfsetstrokecolor{currentstroke}%
\pgfsetstrokeopacity{0.700000}%
\pgfsetdash{}{0pt}%
\pgfpathmoveto{\pgfqpoint{4.189276in}{0.550000in}}%
\pgfpathlineto{\pgfqpoint{4.216797in}{0.550000in}}%
\pgfpathlineto{\pgfqpoint{4.216797in}{0.653794in}}%
\pgfpathlineto{\pgfqpoint{4.189276in}{0.653794in}}%
\pgfpathlineto{\pgfqpoint{4.189276in}{0.550000in}}%
\pgfpathclose%
\pgfusepath{fill}%
\end{pgfscope}%
\begin{pgfscope}%
\pgfpathrectangle{\pgfqpoint{1.250000in}{0.550000in}}{\pgfqpoint{7.750000in}{3.850000in}}%
\pgfusepath{clip}%
\pgfsetbuttcap%
\pgfsetmiterjoin%
\definecolor{currentfill}{rgb}{0.501961,0.501961,0.501961}%
\pgfsetfillcolor{currentfill}%
\pgfsetfillopacity{0.700000}%
\pgfsetlinewidth{0.000000pt}%
\definecolor{currentstroke}{rgb}{0.000000,0.000000,0.000000}%
\pgfsetstrokecolor{currentstroke}%
\pgfsetstrokeopacity{0.700000}%
\pgfsetdash{}{0pt}%
\pgfpathmoveto{\pgfqpoint{4.216797in}{0.550000in}}%
\pgfpathlineto{\pgfqpoint{4.244318in}{0.550000in}}%
\pgfpathlineto{\pgfqpoint{4.244318in}{0.641130in}}%
\pgfpathlineto{\pgfqpoint{4.216797in}{0.641130in}}%
\pgfpathlineto{\pgfqpoint{4.216797in}{0.550000in}}%
\pgfpathclose%
\pgfusepath{fill}%
\end{pgfscope}%
\begin{pgfscope}%
\pgfpathrectangle{\pgfqpoint{1.250000in}{0.550000in}}{\pgfqpoint{7.750000in}{3.850000in}}%
\pgfusepath{clip}%
\pgfsetbuttcap%
\pgfsetmiterjoin%
\definecolor{currentfill}{rgb}{0.501961,0.501961,0.501961}%
\pgfsetfillcolor{currentfill}%
\pgfsetfillopacity{0.700000}%
\pgfsetlinewidth{0.000000pt}%
\definecolor{currentstroke}{rgb}{0.000000,0.000000,0.000000}%
\pgfsetstrokecolor{currentstroke}%
\pgfsetstrokeopacity{0.700000}%
\pgfsetdash{}{0pt}%
\pgfpathmoveto{\pgfqpoint{4.244318in}{0.550000in}}%
\pgfpathlineto{\pgfqpoint{4.271839in}{0.550000in}}%
\pgfpathlineto{\pgfqpoint{4.271839in}{0.650491in}}%
\pgfpathlineto{\pgfqpoint{4.244318in}{0.650491in}}%
\pgfpathlineto{\pgfqpoint{4.244318in}{0.550000in}}%
\pgfpathclose%
\pgfusepath{fill}%
\end{pgfscope}%
\begin{pgfscope}%
\pgfpathrectangle{\pgfqpoint{1.250000in}{0.550000in}}{\pgfqpoint{7.750000in}{3.850000in}}%
\pgfusepath{clip}%
\pgfsetbuttcap%
\pgfsetmiterjoin%
\definecolor{currentfill}{rgb}{0.501961,0.501961,0.501961}%
\pgfsetfillcolor{currentfill}%
\pgfsetfillopacity{0.700000}%
\pgfsetlinewidth{0.000000pt}%
\definecolor{currentstroke}{rgb}{0.000000,0.000000,0.000000}%
\pgfsetstrokecolor{currentstroke}%
\pgfsetstrokeopacity{0.700000}%
\pgfsetdash{}{0pt}%
\pgfpathmoveto{\pgfqpoint{4.271839in}{0.550000in}}%
\pgfpathlineto{\pgfqpoint{4.299361in}{0.550000in}}%
\pgfpathlineto{\pgfqpoint{4.299361in}{0.636174in}}%
\pgfpathlineto{\pgfqpoint{4.271839in}{0.636174in}}%
\pgfpathlineto{\pgfqpoint{4.271839in}{0.550000in}}%
\pgfpathclose%
\pgfusepath{fill}%
\end{pgfscope}%
\begin{pgfscope}%
\pgfpathrectangle{\pgfqpoint{1.250000in}{0.550000in}}{\pgfqpoint{7.750000in}{3.850000in}}%
\pgfusepath{clip}%
\pgfsetbuttcap%
\pgfsetmiterjoin%
\definecolor{currentfill}{rgb}{0.501961,0.501961,0.501961}%
\pgfsetfillcolor{currentfill}%
\pgfsetfillopacity{0.700000}%
\pgfsetlinewidth{0.000000pt}%
\definecolor{currentstroke}{rgb}{0.000000,0.000000,0.000000}%
\pgfsetstrokecolor{currentstroke}%
\pgfsetstrokeopacity{0.700000}%
\pgfsetdash{}{0pt}%
\pgfpathmoveto{\pgfqpoint{4.299361in}{0.550000in}}%
\pgfpathlineto{\pgfqpoint{4.326882in}{0.550000in}}%
\pgfpathlineto{\pgfqpoint{4.326882in}{0.643332in}}%
\pgfpathlineto{\pgfqpoint{4.299361in}{0.643332in}}%
\pgfpathlineto{\pgfqpoint{4.299361in}{0.550000in}}%
\pgfpathclose%
\pgfusepath{fill}%
\end{pgfscope}%
\begin{pgfscope}%
\pgfpathrectangle{\pgfqpoint{1.250000in}{0.550000in}}{\pgfqpoint{7.750000in}{3.850000in}}%
\pgfusepath{clip}%
\pgfsetbuttcap%
\pgfsetmiterjoin%
\definecolor{currentfill}{rgb}{0.501961,0.501961,0.501961}%
\pgfsetfillcolor{currentfill}%
\pgfsetfillopacity{0.700000}%
\pgfsetlinewidth{0.000000pt}%
\definecolor{currentstroke}{rgb}{0.000000,0.000000,0.000000}%
\pgfsetstrokecolor{currentstroke}%
\pgfsetstrokeopacity{0.700000}%
\pgfsetdash{}{0pt}%
\pgfpathmoveto{\pgfqpoint{4.326882in}{0.550000in}}%
\pgfpathlineto{\pgfqpoint{4.354403in}{0.550000in}}%
\pgfpathlineto{\pgfqpoint{4.354403in}{0.651317in}}%
\pgfpathlineto{\pgfqpoint{4.326882in}{0.651317in}}%
\pgfpathlineto{\pgfqpoint{4.326882in}{0.550000in}}%
\pgfpathclose%
\pgfusepath{fill}%
\end{pgfscope}%
\begin{pgfscope}%
\pgfpathrectangle{\pgfqpoint{1.250000in}{0.550000in}}{\pgfqpoint{7.750000in}{3.850000in}}%
\pgfusepath{clip}%
\pgfsetbuttcap%
\pgfsetmiterjoin%
\definecolor{currentfill}{rgb}{0.501961,0.501961,0.501961}%
\pgfsetfillcolor{currentfill}%
\pgfsetfillopacity{0.700000}%
\pgfsetlinewidth{0.000000pt}%
\definecolor{currentstroke}{rgb}{0.000000,0.000000,0.000000}%
\pgfsetstrokecolor{currentstroke}%
\pgfsetstrokeopacity{0.700000}%
\pgfsetdash{}{0pt}%
\pgfpathmoveto{\pgfqpoint{4.354403in}{0.550000in}}%
\pgfpathlineto{\pgfqpoint{4.381925in}{0.550000in}}%
\pgfpathlineto{\pgfqpoint{4.381925in}{0.645810in}}%
\pgfpathlineto{\pgfqpoint{4.354403in}{0.645810in}}%
\pgfpathlineto{\pgfqpoint{4.354403in}{0.550000in}}%
\pgfpathclose%
\pgfusepath{fill}%
\end{pgfscope}%
\begin{pgfscope}%
\pgfpathrectangle{\pgfqpoint{1.250000in}{0.550000in}}{\pgfqpoint{7.750000in}{3.850000in}}%
\pgfusepath{clip}%
\pgfsetbuttcap%
\pgfsetmiterjoin%
\definecolor{currentfill}{rgb}{0.501961,0.501961,0.501961}%
\pgfsetfillcolor{currentfill}%
\pgfsetfillopacity{0.700000}%
\pgfsetlinewidth{0.000000pt}%
\definecolor{currentstroke}{rgb}{0.000000,0.000000,0.000000}%
\pgfsetstrokecolor{currentstroke}%
\pgfsetstrokeopacity{0.700000}%
\pgfsetdash{}{0pt}%
\pgfpathmoveto{\pgfqpoint{4.381925in}{0.550000in}}%
\pgfpathlineto{\pgfqpoint{4.409446in}{0.550000in}}%
\pgfpathlineto{\pgfqpoint{4.409446in}{0.646636in}}%
\pgfpathlineto{\pgfqpoint{4.381925in}{0.646636in}}%
\pgfpathlineto{\pgfqpoint{4.381925in}{0.550000in}}%
\pgfpathclose%
\pgfusepath{fill}%
\end{pgfscope}%
\begin{pgfscope}%
\pgfpathrectangle{\pgfqpoint{1.250000in}{0.550000in}}{\pgfqpoint{7.750000in}{3.850000in}}%
\pgfusepath{clip}%
\pgfsetbuttcap%
\pgfsetmiterjoin%
\definecolor{currentfill}{rgb}{0.501961,0.501961,0.501961}%
\pgfsetfillcolor{currentfill}%
\pgfsetfillopacity{0.700000}%
\pgfsetlinewidth{0.000000pt}%
\definecolor{currentstroke}{rgb}{0.000000,0.000000,0.000000}%
\pgfsetstrokecolor{currentstroke}%
\pgfsetstrokeopacity{0.700000}%
\pgfsetdash{}{0pt}%
\pgfpathmoveto{\pgfqpoint{4.409446in}{0.550000in}}%
\pgfpathlineto{\pgfqpoint{4.436967in}{0.550000in}}%
\pgfpathlineto{\pgfqpoint{4.436967in}{0.654620in}}%
\pgfpathlineto{\pgfqpoint{4.409446in}{0.654620in}}%
\pgfpathlineto{\pgfqpoint{4.409446in}{0.550000in}}%
\pgfpathclose%
\pgfusepath{fill}%
\end{pgfscope}%
\begin{pgfscope}%
\pgfpathrectangle{\pgfqpoint{1.250000in}{0.550000in}}{\pgfqpoint{7.750000in}{3.850000in}}%
\pgfusepath{clip}%
\pgfsetbuttcap%
\pgfsetmiterjoin%
\definecolor{currentfill}{rgb}{0.501961,0.501961,0.501961}%
\pgfsetfillcolor{currentfill}%
\pgfsetfillopacity{0.700000}%
\pgfsetlinewidth{0.000000pt}%
\definecolor{currentstroke}{rgb}{0.000000,0.000000,0.000000}%
\pgfsetstrokecolor{currentstroke}%
\pgfsetstrokeopacity{0.700000}%
\pgfsetdash{}{0pt}%
\pgfpathmoveto{\pgfqpoint{4.436967in}{0.550000in}}%
\pgfpathlineto{\pgfqpoint{4.464489in}{0.550000in}}%
\pgfpathlineto{\pgfqpoint{4.464489in}{0.646085in}}%
\pgfpathlineto{\pgfqpoint{4.436967in}{0.646085in}}%
\pgfpathlineto{\pgfqpoint{4.436967in}{0.550000in}}%
\pgfpathclose%
\pgfusepath{fill}%
\end{pgfscope}%
\begin{pgfscope}%
\pgfpathrectangle{\pgfqpoint{1.250000in}{0.550000in}}{\pgfqpoint{7.750000in}{3.850000in}}%
\pgfusepath{clip}%
\pgfsetbuttcap%
\pgfsetmiterjoin%
\definecolor{currentfill}{rgb}{0.501961,0.501961,0.501961}%
\pgfsetfillcolor{currentfill}%
\pgfsetfillopacity{0.700000}%
\pgfsetlinewidth{0.000000pt}%
\definecolor{currentstroke}{rgb}{0.000000,0.000000,0.000000}%
\pgfsetstrokecolor{currentstroke}%
\pgfsetstrokeopacity{0.700000}%
\pgfsetdash{}{0pt}%
\pgfpathmoveto{\pgfqpoint{4.464489in}{0.550000in}}%
\pgfpathlineto{\pgfqpoint{4.492010in}{0.550000in}}%
\pgfpathlineto{\pgfqpoint{4.492010in}{0.642231in}}%
\pgfpathlineto{\pgfqpoint{4.464489in}{0.642231in}}%
\pgfpathlineto{\pgfqpoint{4.464489in}{0.550000in}}%
\pgfpathclose%
\pgfusepath{fill}%
\end{pgfscope}%
\begin{pgfscope}%
\pgfpathrectangle{\pgfqpoint{1.250000in}{0.550000in}}{\pgfqpoint{7.750000in}{3.850000in}}%
\pgfusepath{clip}%
\pgfsetbuttcap%
\pgfsetmiterjoin%
\definecolor{currentfill}{rgb}{0.501961,0.501961,0.501961}%
\pgfsetfillcolor{currentfill}%
\pgfsetfillopacity{0.700000}%
\pgfsetlinewidth{0.000000pt}%
\definecolor{currentstroke}{rgb}{0.000000,0.000000,0.000000}%
\pgfsetstrokecolor{currentstroke}%
\pgfsetstrokeopacity{0.700000}%
\pgfsetdash{}{0pt}%
\pgfpathmoveto{\pgfqpoint{4.492010in}{0.550000in}}%
\pgfpathlineto{\pgfqpoint{4.519531in}{0.550000in}}%
\pgfpathlineto{\pgfqpoint{4.519531in}{0.648563in}}%
\pgfpathlineto{\pgfqpoint{4.492010in}{0.648563in}}%
\pgfpathlineto{\pgfqpoint{4.492010in}{0.550000in}}%
\pgfpathclose%
\pgfusepath{fill}%
\end{pgfscope}%
\begin{pgfscope}%
\pgfpathrectangle{\pgfqpoint{1.250000in}{0.550000in}}{\pgfqpoint{7.750000in}{3.850000in}}%
\pgfusepath{clip}%
\pgfsetbuttcap%
\pgfsetmiterjoin%
\definecolor{currentfill}{rgb}{0.501961,0.501961,0.501961}%
\pgfsetfillcolor{currentfill}%
\pgfsetfillopacity{0.700000}%
\pgfsetlinewidth{0.000000pt}%
\definecolor{currentstroke}{rgb}{0.000000,0.000000,0.000000}%
\pgfsetstrokecolor{currentstroke}%
\pgfsetstrokeopacity{0.700000}%
\pgfsetdash{}{0pt}%
\pgfpathmoveto{\pgfqpoint{4.519531in}{0.550000in}}%
\pgfpathlineto{\pgfqpoint{4.547053in}{0.550000in}}%
\pgfpathlineto{\pgfqpoint{4.547053in}{0.659576in}}%
\pgfpathlineto{\pgfqpoint{4.519531in}{0.659576in}}%
\pgfpathlineto{\pgfqpoint{4.519531in}{0.550000in}}%
\pgfpathclose%
\pgfusepath{fill}%
\end{pgfscope}%
\begin{pgfscope}%
\pgfpathrectangle{\pgfqpoint{1.250000in}{0.550000in}}{\pgfqpoint{7.750000in}{3.850000in}}%
\pgfusepath{clip}%
\pgfsetbuttcap%
\pgfsetmiterjoin%
\definecolor{currentfill}{rgb}{0.501961,0.501961,0.501961}%
\pgfsetfillcolor{currentfill}%
\pgfsetfillopacity{0.700000}%
\pgfsetlinewidth{0.000000pt}%
\definecolor{currentstroke}{rgb}{0.000000,0.000000,0.000000}%
\pgfsetstrokecolor{currentstroke}%
\pgfsetstrokeopacity{0.700000}%
\pgfsetdash{}{0pt}%
\pgfpathmoveto{\pgfqpoint{4.547053in}{0.550000in}}%
\pgfpathlineto{\pgfqpoint{4.574574in}{0.550000in}}%
\pgfpathlineto{\pgfqpoint{4.574574in}{0.652968in}}%
\pgfpathlineto{\pgfqpoint{4.547053in}{0.652968in}}%
\pgfpathlineto{\pgfqpoint{4.547053in}{0.550000in}}%
\pgfpathclose%
\pgfusepath{fill}%
\end{pgfscope}%
\begin{pgfscope}%
\pgfpathrectangle{\pgfqpoint{1.250000in}{0.550000in}}{\pgfqpoint{7.750000in}{3.850000in}}%
\pgfusepath{clip}%
\pgfsetbuttcap%
\pgfsetmiterjoin%
\definecolor{currentfill}{rgb}{0.501961,0.501961,0.501961}%
\pgfsetfillcolor{currentfill}%
\pgfsetfillopacity{0.700000}%
\pgfsetlinewidth{0.000000pt}%
\definecolor{currentstroke}{rgb}{0.000000,0.000000,0.000000}%
\pgfsetstrokecolor{currentstroke}%
\pgfsetstrokeopacity{0.700000}%
\pgfsetdash{}{0pt}%
\pgfpathmoveto{\pgfqpoint{4.574574in}{0.550000in}}%
\pgfpathlineto{\pgfqpoint{4.602095in}{0.550000in}}%
\pgfpathlineto{\pgfqpoint{4.602095in}{0.652968in}}%
\pgfpathlineto{\pgfqpoint{4.574574in}{0.652968in}}%
\pgfpathlineto{\pgfqpoint{4.574574in}{0.550000in}}%
\pgfpathclose%
\pgfusepath{fill}%
\end{pgfscope}%
\begin{pgfscope}%
\pgfpathrectangle{\pgfqpoint{1.250000in}{0.550000in}}{\pgfqpoint{7.750000in}{3.850000in}}%
\pgfusepath{clip}%
\pgfsetbuttcap%
\pgfsetmiterjoin%
\definecolor{currentfill}{rgb}{0.501961,0.501961,0.501961}%
\pgfsetfillcolor{currentfill}%
\pgfsetfillopacity{0.700000}%
\pgfsetlinewidth{0.000000pt}%
\definecolor{currentstroke}{rgb}{0.000000,0.000000,0.000000}%
\pgfsetstrokecolor{currentstroke}%
\pgfsetstrokeopacity{0.700000}%
\pgfsetdash{}{0pt}%
\pgfpathmoveto{\pgfqpoint{4.602095in}{0.550000in}}%
\pgfpathlineto{\pgfqpoint{4.629616in}{0.550000in}}%
\pgfpathlineto{\pgfqpoint{4.629616in}{0.652693in}}%
\pgfpathlineto{\pgfqpoint{4.602095in}{0.652693in}}%
\pgfpathlineto{\pgfqpoint{4.602095in}{0.550000in}}%
\pgfpathclose%
\pgfusepath{fill}%
\end{pgfscope}%
\begin{pgfscope}%
\pgfpathrectangle{\pgfqpoint{1.250000in}{0.550000in}}{\pgfqpoint{7.750000in}{3.850000in}}%
\pgfusepath{clip}%
\pgfsetbuttcap%
\pgfsetmiterjoin%
\definecolor{currentfill}{rgb}{0.501961,0.501961,0.501961}%
\pgfsetfillcolor{currentfill}%
\pgfsetfillopacity{0.700000}%
\pgfsetlinewidth{0.000000pt}%
\definecolor{currentstroke}{rgb}{0.000000,0.000000,0.000000}%
\pgfsetstrokecolor{currentstroke}%
\pgfsetstrokeopacity{0.700000}%
\pgfsetdash{}{0pt}%
\pgfpathmoveto{\pgfqpoint{4.629616in}{0.550000in}}%
\pgfpathlineto{\pgfqpoint{4.657138in}{0.550000in}}%
\pgfpathlineto{\pgfqpoint{4.657138in}{0.651317in}}%
\pgfpathlineto{\pgfqpoint{4.629616in}{0.651317in}}%
\pgfpathlineto{\pgfqpoint{4.629616in}{0.550000in}}%
\pgfpathclose%
\pgfusepath{fill}%
\end{pgfscope}%
\begin{pgfscope}%
\pgfpathrectangle{\pgfqpoint{1.250000in}{0.550000in}}{\pgfqpoint{7.750000in}{3.850000in}}%
\pgfusepath{clip}%
\pgfsetbuttcap%
\pgfsetmiterjoin%
\definecolor{currentfill}{rgb}{0.501961,0.501961,0.501961}%
\pgfsetfillcolor{currentfill}%
\pgfsetfillopacity{0.700000}%
\pgfsetlinewidth{0.000000pt}%
\definecolor{currentstroke}{rgb}{0.000000,0.000000,0.000000}%
\pgfsetstrokecolor{currentstroke}%
\pgfsetstrokeopacity{0.700000}%
\pgfsetdash{}{0pt}%
\pgfpathmoveto{\pgfqpoint{4.657138in}{0.550000in}}%
\pgfpathlineto{\pgfqpoint{4.684659in}{0.550000in}}%
\pgfpathlineto{\pgfqpoint{4.684659in}{0.655997in}}%
\pgfpathlineto{\pgfqpoint{4.657138in}{0.655997in}}%
\pgfpathlineto{\pgfqpoint{4.657138in}{0.550000in}}%
\pgfpathclose%
\pgfusepath{fill}%
\end{pgfscope}%
\begin{pgfscope}%
\pgfpathrectangle{\pgfqpoint{1.250000in}{0.550000in}}{\pgfqpoint{7.750000in}{3.850000in}}%
\pgfusepath{clip}%
\pgfsetbuttcap%
\pgfsetmiterjoin%
\definecolor{currentfill}{rgb}{0.501961,0.501961,0.501961}%
\pgfsetfillcolor{currentfill}%
\pgfsetfillopacity{0.700000}%
\pgfsetlinewidth{0.000000pt}%
\definecolor{currentstroke}{rgb}{0.000000,0.000000,0.000000}%
\pgfsetstrokecolor{currentstroke}%
\pgfsetstrokeopacity{0.700000}%
\pgfsetdash{}{0pt}%
\pgfpathmoveto{\pgfqpoint{4.684659in}{0.550000in}}%
\pgfpathlineto{\pgfqpoint{4.712180in}{0.550000in}}%
\pgfpathlineto{\pgfqpoint{4.712180in}{0.648839in}}%
\pgfpathlineto{\pgfqpoint{4.684659in}{0.648839in}}%
\pgfpathlineto{\pgfqpoint{4.684659in}{0.550000in}}%
\pgfpathclose%
\pgfusepath{fill}%
\end{pgfscope}%
\begin{pgfscope}%
\pgfpathrectangle{\pgfqpoint{1.250000in}{0.550000in}}{\pgfqpoint{7.750000in}{3.850000in}}%
\pgfusepath{clip}%
\pgfsetbuttcap%
\pgfsetmiterjoin%
\definecolor{currentfill}{rgb}{0.501961,0.501961,0.501961}%
\pgfsetfillcolor{currentfill}%
\pgfsetfillopacity{0.700000}%
\pgfsetlinewidth{0.000000pt}%
\definecolor{currentstroke}{rgb}{0.000000,0.000000,0.000000}%
\pgfsetstrokecolor{currentstroke}%
\pgfsetstrokeopacity{0.700000}%
\pgfsetdash{}{0pt}%
\pgfpathmoveto{\pgfqpoint{4.712180in}{0.550000in}}%
\pgfpathlineto{\pgfqpoint{4.739702in}{0.550000in}}%
\pgfpathlineto{\pgfqpoint{4.739702in}{0.654620in}}%
\pgfpathlineto{\pgfqpoint{4.712180in}{0.654620in}}%
\pgfpathlineto{\pgfqpoint{4.712180in}{0.550000in}}%
\pgfpathclose%
\pgfusepath{fill}%
\end{pgfscope}%
\begin{pgfscope}%
\pgfpathrectangle{\pgfqpoint{1.250000in}{0.550000in}}{\pgfqpoint{7.750000in}{3.850000in}}%
\pgfusepath{clip}%
\pgfsetbuttcap%
\pgfsetmiterjoin%
\definecolor{currentfill}{rgb}{0.501961,0.501961,0.501961}%
\pgfsetfillcolor{currentfill}%
\pgfsetfillopacity{0.700000}%
\pgfsetlinewidth{0.000000pt}%
\definecolor{currentstroke}{rgb}{0.000000,0.000000,0.000000}%
\pgfsetstrokecolor{currentstroke}%
\pgfsetstrokeopacity{0.700000}%
\pgfsetdash{}{0pt}%
\pgfpathmoveto{\pgfqpoint{4.739702in}{0.550000in}}%
\pgfpathlineto{\pgfqpoint{4.767223in}{0.550000in}}%
\pgfpathlineto{\pgfqpoint{4.767223in}{0.647187in}}%
\pgfpathlineto{\pgfqpoint{4.739702in}{0.647187in}}%
\pgfpathlineto{\pgfqpoint{4.739702in}{0.550000in}}%
\pgfpathclose%
\pgfusepath{fill}%
\end{pgfscope}%
\begin{pgfscope}%
\pgfpathrectangle{\pgfqpoint{1.250000in}{0.550000in}}{\pgfqpoint{7.750000in}{3.850000in}}%
\pgfusepath{clip}%
\pgfsetbuttcap%
\pgfsetmiterjoin%
\definecolor{currentfill}{rgb}{0.501961,0.501961,0.501961}%
\pgfsetfillcolor{currentfill}%
\pgfsetfillopacity{0.700000}%
\pgfsetlinewidth{0.000000pt}%
\definecolor{currentstroke}{rgb}{0.000000,0.000000,0.000000}%
\pgfsetstrokecolor{currentstroke}%
\pgfsetstrokeopacity{0.700000}%
\pgfsetdash{}{0pt}%
\pgfpathmoveto{\pgfqpoint{4.767223in}{0.550000in}}%
\pgfpathlineto{\pgfqpoint{4.794744in}{0.550000in}}%
\pgfpathlineto{\pgfqpoint{4.794744in}{0.657373in}}%
\pgfpathlineto{\pgfqpoint{4.767223in}{0.657373in}}%
\pgfpathlineto{\pgfqpoint{4.767223in}{0.550000in}}%
\pgfpathclose%
\pgfusepath{fill}%
\end{pgfscope}%
\begin{pgfscope}%
\pgfpathrectangle{\pgfqpoint{1.250000in}{0.550000in}}{\pgfqpoint{7.750000in}{3.850000in}}%
\pgfusepath{clip}%
\pgfsetbuttcap%
\pgfsetmiterjoin%
\definecolor{currentfill}{rgb}{0.501961,0.501961,0.501961}%
\pgfsetfillcolor{currentfill}%
\pgfsetfillopacity{0.700000}%
\pgfsetlinewidth{0.000000pt}%
\definecolor{currentstroke}{rgb}{0.000000,0.000000,0.000000}%
\pgfsetstrokecolor{currentstroke}%
\pgfsetstrokeopacity{0.700000}%
\pgfsetdash{}{0pt}%
\pgfpathmoveto{\pgfqpoint{4.794744in}{0.550000in}}%
\pgfpathlineto{\pgfqpoint{4.822266in}{0.550000in}}%
\pgfpathlineto{\pgfqpoint{4.822266in}{0.655997in}}%
\pgfpathlineto{\pgfqpoint{4.794744in}{0.655997in}}%
\pgfpathlineto{\pgfqpoint{4.794744in}{0.550000in}}%
\pgfpathclose%
\pgfusepath{fill}%
\end{pgfscope}%
\begin{pgfscope}%
\pgfpathrectangle{\pgfqpoint{1.250000in}{0.550000in}}{\pgfqpoint{7.750000in}{3.850000in}}%
\pgfusepath{clip}%
\pgfsetbuttcap%
\pgfsetmiterjoin%
\definecolor{currentfill}{rgb}{0.501961,0.501961,0.501961}%
\pgfsetfillcolor{currentfill}%
\pgfsetfillopacity{0.700000}%
\pgfsetlinewidth{0.000000pt}%
\definecolor{currentstroke}{rgb}{0.000000,0.000000,0.000000}%
\pgfsetstrokecolor{currentstroke}%
\pgfsetstrokeopacity{0.700000}%
\pgfsetdash{}{0pt}%
\pgfpathmoveto{\pgfqpoint{4.822266in}{0.550000in}}%
\pgfpathlineto{\pgfqpoint{4.849787in}{0.550000in}}%
\pgfpathlineto{\pgfqpoint{4.849787in}{0.659301in}}%
\pgfpathlineto{\pgfqpoint{4.822266in}{0.659301in}}%
\pgfpathlineto{\pgfqpoint{4.822266in}{0.550000in}}%
\pgfpathclose%
\pgfusepath{fill}%
\end{pgfscope}%
\begin{pgfscope}%
\pgfpathrectangle{\pgfqpoint{1.250000in}{0.550000in}}{\pgfqpoint{7.750000in}{3.850000in}}%
\pgfusepath{clip}%
\pgfsetbuttcap%
\pgfsetmiterjoin%
\definecolor{currentfill}{rgb}{0.501961,0.501961,0.501961}%
\pgfsetfillcolor{currentfill}%
\pgfsetfillopacity{0.700000}%
\pgfsetlinewidth{0.000000pt}%
\definecolor{currentstroke}{rgb}{0.000000,0.000000,0.000000}%
\pgfsetstrokecolor{currentstroke}%
\pgfsetstrokeopacity{0.700000}%
\pgfsetdash{}{0pt}%
\pgfpathmoveto{\pgfqpoint{4.849787in}{0.550000in}}%
\pgfpathlineto{\pgfqpoint{4.877308in}{0.550000in}}%
\pgfpathlineto{\pgfqpoint{4.877308in}{0.658750in}}%
\pgfpathlineto{\pgfqpoint{4.849787in}{0.658750in}}%
\pgfpathlineto{\pgfqpoint{4.849787in}{0.550000in}}%
\pgfpathclose%
\pgfusepath{fill}%
\end{pgfscope}%
\begin{pgfscope}%
\pgfpathrectangle{\pgfqpoint{1.250000in}{0.550000in}}{\pgfqpoint{7.750000in}{3.850000in}}%
\pgfusepath{clip}%
\pgfsetbuttcap%
\pgfsetmiterjoin%
\definecolor{currentfill}{rgb}{0.501961,0.501961,0.501961}%
\pgfsetfillcolor{currentfill}%
\pgfsetfillopacity{0.700000}%
\pgfsetlinewidth{0.000000pt}%
\definecolor{currentstroke}{rgb}{0.000000,0.000000,0.000000}%
\pgfsetstrokecolor{currentstroke}%
\pgfsetstrokeopacity{0.700000}%
\pgfsetdash{}{0pt}%
\pgfpathmoveto{\pgfqpoint{4.877308in}{0.550000in}}%
\pgfpathlineto{\pgfqpoint{4.904830in}{0.550000in}}%
\pgfpathlineto{\pgfqpoint{4.904830in}{0.658750in}}%
\pgfpathlineto{\pgfqpoint{4.877308in}{0.658750in}}%
\pgfpathlineto{\pgfqpoint{4.877308in}{0.550000in}}%
\pgfpathclose%
\pgfusepath{fill}%
\end{pgfscope}%
\begin{pgfscope}%
\pgfpathrectangle{\pgfqpoint{1.250000in}{0.550000in}}{\pgfqpoint{7.750000in}{3.850000in}}%
\pgfusepath{clip}%
\pgfsetbuttcap%
\pgfsetmiterjoin%
\definecolor{currentfill}{rgb}{0.501961,0.501961,0.501961}%
\pgfsetfillcolor{currentfill}%
\pgfsetfillopacity{0.700000}%
\pgfsetlinewidth{0.000000pt}%
\definecolor{currentstroke}{rgb}{0.000000,0.000000,0.000000}%
\pgfsetstrokecolor{currentstroke}%
\pgfsetstrokeopacity{0.700000}%
\pgfsetdash{}{0pt}%
\pgfpathmoveto{\pgfqpoint{4.904830in}{0.550000in}}%
\pgfpathlineto{\pgfqpoint{4.932351in}{0.550000in}}%
\pgfpathlineto{\pgfqpoint{4.932351in}{0.651041in}}%
\pgfpathlineto{\pgfqpoint{4.904830in}{0.651041in}}%
\pgfpathlineto{\pgfqpoint{4.904830in}{0.550000in}}%
\pgfpathclose%
\pgfusepath{fill}%
\end{pgfscope}%
\begin{pgfscope}%
\pgfpathrectangle{\pgfqpoint{1.250000in}{0.550000in}}{\pgfqpoint{7.750000in}{3.850000in}}%
\pgfusepath{clip}%
\pgfsetbuttcap%
\pgfsetmiterjoin%
\definecolor{currentfill}{rgb}{0.501961,0.501961,0.501961}%
\pgfsetfillcolor{currentfill}%
\pgfsetfillopacity{0.700000}%
\pgfsetlinewidth{0.000000pt}%
\definecolor{currentstroke}{rgb}{0.000000,0.000000,0.000000}%
\pgfsetstrokecolor{currentstroke}%
\pgfsetstrokeopacity{0.700000}%
\pgfsetdash{}{0pt}%
\pgfpathmoveto{\pgfqpoint{4.932351in}{0.550000in}}%
\pgfpathlineto{\pgfqpoint{4.959872in}{0.550000in}}%
\pgfpathlineto{\pgfqpoint{4.959872in}{0.659576in}}%
\pgfpathlineto{\pgfqpoint{4.932351in}{0.659576in}}%
\pgfpathlineto{\pgfqpoint{4.932351in}{0.550000in}}%
\pgfpathclose%
\pgfusepath{fill}%
\end{pgfscope}%
\begin{pgfscope}%
\pgfpathrectangle{\pgfqpoint{1.250000in}{0.550000in}}{\pgfqpoint{7.750000in}{3.850000in}}%
\pgfusepath{clip}%
\pgfsetbuttcap%
\pgfsetmiterjoin%
\definecolor{currentfill}{rgb}{0.501961,0.501961,0.501961}%
\pgfsetfillcolor{currentfill}%
\pgfsetfillopacity{0.700000}%
\pgfsetlinewidth{0.000000pt}%
\definecolor{currentstroke}{rgb}{0.000000,0.000000,0.000000}%
\pgfsetstrokecolor{currentstroke}%
\pgfsetstrokeopacity{0.700000}%
\pgfsetdash{}{0pt}%
\pgfpathmoveto{\pgfqpoint{4.959872in}{0.550000in}}%
\pgfpathlineto{\pgfqpoint{4.987393in}{0.550000in}}%
\pgfpathlineto{\pgfqpoint{4.987393in}{0.652142in}}%
\pgfpathlineto{\pgfqpoint{4.959872in}{0.652142in}}%
\pgfpathlineto{\pgfqpoint{4.959872in}{0.550000in}}%
\pgfpathclose%
\pgfusepath{fill}%
\end{pgfscope}%
\begin{pgfscope}%
\pgfpathrectangle{\pgfqpoint{1.250000in}{0.550000in}}{\pgfqpoint{7.750000in}{3.850000in}}%
\pgfusepath{clip}%
\pgfsetbuttcap%
\pgfsetmiterjoin%
\definecolor{currentfill}{rgb}{0.501961,0.501961,0.501961}%
\pgfsetfillcolor{currentfill}%
\pgfsetfillopacity{0.700000}%
\pgfsetlinewidth{0.000000pt}%
\definecolor{currentstroke}{rgb}{0.000000,0.000000,0.000000}%
\pgfsetstrokecolor{currentstroke}%
\pgfsetstrokeopacity{0.700000}%
\pgfsetdash{}{0pt}%
\pgfpathmoveto{\pgfqpoint{4.987393in}{0.550000in}}%
\pgfpathlineto{\pgfqpoint{5.014915in}{0.550000in}}%
\pgfpathlineto{\pgfqpoint{5.014915in}{0.657098in}}%
\pgfpathlineto{\pgfqpoint{4.987393in}{0.657098in}}%
\pgfpathlineto{\pgfqpoint{4.987393in}{0.550000in}}%
\pgfpathclose%
\pgfusepath{fill}%
\end{pgfscope}%
\begin{pgfscope}%
\pgfpathrectangle{\pgfqpoint{1.250000in}{0.550000in}}{\pgfqpoint{7.750000in}{3.850000in}}%
\pgfusepath{clip}%
\pgfsetbuttcap%
\pgfsetmiterjoin%
\definecolor{currentfill}{rgb}{0.501961,0.501961,0.501961}%
\pgfsetfillcolor{currentfill}%
\pgfsetfillopacity{0.700000}%
\pgfsetlinewidth{0.000000pt}%
\definecolor{currentstroke}{rgb}{0.000000,0.000000,0.000000}%
\pgfsetstrokecolor{currentstroke}%
\pgfsetstrokeopacity{0.700000}%
\pgfsetdash{}{0pt}%
\pgfpathmoveto{\pgfqpoint{5.014915in}{0.550000in}}%
\pgfpathlineto{\pgfqpoint{5.042436in}{0.550000in}}%
\pgfpathlineto{\pgfqpoint{5.042436in}{0.646911in}}%
\pgfpathlineto{\pgfqpoint{5.014915in}{0.646911in}}%
\pgfpathlineto{\pgfqpoint{5.014915in}{0.550000in}}%
\pgfpathclose%
\pgfusepath{fill}%
\end{pgfscope}%
\begin{pgfscope}%
\pgfpathrectangle{\pgfqpoint{1.250000in}{0.550000in}}{\pgfqpoint{7.750000in}{3.850000in}}%
\pgfusepath{clip}%
\pgfsetbuttcap%
\pgfsetmiterjoin%
\definecolor{currentfill}{rgb}{0.501961,0.501961,0.501961}%
\pgfsetfillcolor{currentfill}%
\pgfsetfillopacity{0.700000}%
\pgfsetlinewidth{0.000000pt}%
\definecolor{currentstroke}{rgb}{0.000000,0.000000,0.000000}%
\pgfsetstrokecolor{currentstroke}%
\pgfsetstrokeopacity{0.700000}%
\pgfsetdash{}{0pt}%
\pgfpathmoveto{\pgfqpoint{5.042436in}{0.550000in}}%
\pgfpathlineto{\pgfqpoint{5.069957in}{0.550000in}}%
\pgfpathlineto{\pgfqpoint{5.069957in}{0.664256in}}%
\pgfpathlineto{\pgfqpoint{5.042436in}{0.664256in}}%
\pgfpathlineto{\pgfqpoint{5.042436in}{0.550000in}}%
\pgfpathclose%
\pgfusepath{fill}%
\end{pgfscope}%
\begin{pgfscope}%
\pgfpathrectangle{\pgfqpoint{1.250000in}{0.550000in}}{\pgfqpoint{7.750000in}{3.850000in}}%
\pgfusepath{clip}%
\pgfsetbuttcap%
\pgfsetmiterjoin%
\definecolor{currentfill}{rgb}{0.501961,0.501961,0.501961}%
\pgfsetfillcolor{currentfill}%
\pgfsetfillopacity{0.700000}%
\pgfsetlinewidth{0.000000pt}%
\definecolor{currentstroke}{rgb}{0.000000,0.000000,0.000000}%
\pgfsetstrokecolor{currentstroke}%
\pgfsetstrokeopacity{0.700000}%
\pgfsetdash{}{0pt}%
\pgfpathmoveto{\pgfqpoint{5.069957in}{0.550000in}}%
\pgfpathlineto{\pgfqpoint{5.097479in}{0.550000in}}%
\pgfpathlineto{\pgfqpoint{5.097479in}{0.657924in}}%
\pgfpathlineto{\pgfqpoint{5.069957in}{0.657924in}}%
\pgfpathlineto{\pgfqpoint{5.069957in}{0.550000in}}%
\pgfpathclose%
\pgfusepath{fill}%
\end{pgfscope}%
\begin{pgfscope}%
\pgfpathrectangle{\pgfqpoint{1.250000in}{0.550000in}}{\pgfqpoint{7.750000in}{3.850000in}}%
\pgfusepath{clip}%
\pgfsetbuttcap%
\pgfsetmiterjoin%
\definecolor{currentfill}{rgb}{0.501961,0.501961,0.501961}%
\pgfsetfillcolor{currentfill}%
\pgfsetfillopacity{0.700000}%
\pgfsetlinewidth{0.000000pt}%
\definecolor{currentstroke}{rgb}{0.000000,0.000000,0.000000}%
\pgfsetstrokecolor{currentstroke}%
\pgfsetstrokeopacity{0.700000}%
\pgfsetdash{}{0pt}%
\pgfpathmoveto{\pgfqpoint{5.097479in}{0.550000in}}%
\pgfpathlineto{\pgfqpoint{5.125000in}{0.550000in}}%
\pgfpathlineto{\pgfqpoint{5.125000in}{0.660127in}}%
\pgfpathlineto{\pgfqpoint{5.097479in}{0.660127in}}%
\pgfpathlineto{\pgfqpoint{5.097479in}{0.550000in}}%
\pgfpathclose%
\pgfusepath{fill}%
\end{pgfscope}%
\begin{pgfscope}%
\pgfpathrectangle{\pgfqpoint{1.250000in}{0.550000in}}{\pgfqpoint{7.750000in}{3.850000in}}%
\pgfusepath{clip}%
\pgfsetbuttcap%
\pgfsetmiterjoin%
\definecolor{currentfill}{rgb}{0.501961,0.501961,0.501961}%
\pgfsetfillcolor{currentfill}%
\pgfsetfillopacity{0.700000}%
\pgfsetlinewidth{0.000000pt}%
\definecolor{currentstroke}{rgb}{0.000000,0.000000,0.000000}%
\pgfsetstrokecolor{currentstroke}%
\pgfsetstrokeopacity{0.700000}%
\pgfsetdash{}{0pt}%
\pgfpathmoveto{\pgfqpoint{5.125000in}{0.550000in}}%
\pgfpathlineto{\pgfqpoint{5.152521in}{0.550000in}}%
\pgfpathlineto{\pgfqpoint{5.152521in}{0.662880in}}%
\pgfpathlineto{\pgfqpoint{5.125000in}{0.662880in}}%
\pgfpathlineto{\pgfqpoint{5.125000in}{0.550000in}}%
\pgfpathclose%
\pgfusepath{fill}%
\end{pgfscope}%
\begin{pgfscope}%
\pgfpathrectangle{\pgfqpoint{1.250000in}{0.550000in}}{\pgfqpoint{7.750000in}{3.850000in}}%
\pgfusepath{clip}%
\pgfsetbuttcap%
\pgfsetmiterjoin%
\definecolor{currentfill}{rgb}{0.501961,0.501961,0.501961}%
\pgfsetfillcolor{currentfill}%
\pgfsetfillopacity{0.700000}%
\pgfsetlinewidth{0.000000pt}%
\definecolor{currentstroke}{rgb}{0.000000,0.000000,0.000000}%
\pgfsetstrokecolor{currentstroke}%
\pgfsetstrokeopacity{0.700000}%
\pgfsetdash{}{0pt}%
\pgfpathmoveto{\pgfqpoint{5.152521in}{0.550000in}}%
\pgfpathlineto{\pgfqpoint{5.180043in}{0.550000in}}%
\pgfpathlineto{\pgfqpoint{5.180043in}{0.668937in}}%
\pgfpathlineto{\pgfqpoint{5.152521in}{0.668937in}}%
\pgfpathlineto{\pgfqpoint{5.152521in}{0.550000in}}%
\pgfpathclose%
\pgfusepath{fill}%
\end{pgfscope}%
\begin{pgfscope}%
\pgfpathrectangle{\pgfqpoint{1.250000in}{0.550000in}}{\pgfqpoint{7.750000in}{3.850000in}}%
\pgfusepath{clip}%
\pgfsetbuttcap%
\pgfsetmiterjoin%
\definecolor{currentfill}{rgb}{0.501961,0.501961,0.501961}%
\pgfsetfillcolor{currentfill}%
\pgfsetfillopacity{0.700000}%
\pgfsetlinewidth{0.000000pt}%
\definecolor{currentstroke}{rgb}{0.000000,0.000000,0.000000}%
\pgfsetstrokecolor{currentstroke}%
\pgfsetstrokeopacity{0.700000}%
\pgfsetdash{}{0pt}%
\pgfpathmoveto{\pgfqpoint{5.180043in}{0.550000in}}%
\pgfpathlineto{\pgfqpoint{5.207564in}{0.550000in}}%
\pgfpathlineto{\pgfqpoint{5.207564in}{0.678298in}}%
\pgfpathlineto{\pgfqpoint{5.180043in}{0.678298in}}%
\pgfpathlineto{\pgfqpoint{5.180043in}{0.550000in}}%
\pgfpathclose%
\pgfusepath{fill}%
\end{pgfscope}%
\begin{pgfscope}%
\pgfpathrectangle{\pgfqpoint{1.250000in}{0.550000in}}{\pgfqpoint{7.750000in}{3.850000in}}%
\pgfusepath{clip}%
\pgfsetbuttcap%
\pgfsetmiterjoin%
\definecolor{currentfill}{rgb}{0.501961,0.501961,0.501961}%
\pgfsetfillcolor{currentfill}%
\pgfsetfillopacity{0.700000}%
\pgfsetlinewidth{0.000000pt}%
\definecolor{currentstroke}{rgb}{0.000000,0.000000,0.000000}%
\pgfsetstrokecolor{currentstroke}%
\pgfsetstrokeopacity{0.700000}%
\pgfsetdash{}{0pt}%
\pgfpathmoveto{\pgfqpoint{5.207564in}{0.550000in}}%
\pgfpathlineto{\pgfqpoint{5.235085in}{0.550000in}}%
\pgfpathlineto{\pgfqpoint{5.235085in}{0.680500in}}%
\pgfpathlineto{\pgfqpoint{5.207564in}{0.680500in}}%
\pgfpathlineto{\pgfqpoint{5.207564in}{0.550000in}}%
\pgfpathclose%
\pgfusepath{fill}%
\end{pgfscope}%
\begin{pgfscope}%
\pgfpathrectangle{\pgfqpoint{1.250000in}{0.550000in}}{\pgfqpoint{7.750000in}{3.850000in}}%
\pgfusepath{clip}%
\pgfsetbuttcap%
\pgfsetmiterjoin%
\definecolor{currentfill}{rgb}{0.501961,0.501961,0.501961}%
\pgfsetfillcolor{currentfill}%
\pgfsetfillopacity{0.700000}%
\pgfsetlinewidth{0.000000pt}%
\definecolor{currentstroke}{rgb}{0.000000,0.000000,0.000000}%
\pgfsetstrokecolor{currentstroke}%
\pgfsetstrokeopacity{0.700000}%
\pgfsetdash{}{0pt}%
\pgfpathmoveto{\pgfqpoint{5.235085in}{0.550000in}}%
\pgfpathlineto{\pgfqpoint{5.262607in}{0.550000in}}%
\pgfpathlineto{\pgfqpoint{5.262607in}{0.677472in}}%
\pgfpathlineto{\pgfqpoint{5.235085in}{0.677472in}}%
\pgfpathlineto{\pgfqpoint{5.235085in}{0.550000in}}%
\pgfpathclose%
\pgfusepath{fill}%
\end{pgfscope}%
\begin{pgfscope}%
\pgfpathrectangle{\pgfqpoint{1.250000in}{0.550000in}}{\pgfqpoint{7.750000in}{3.850000in}}%
\pgfusepath{clip}%
\pgfsetbuttcap%
\pgfsetmiterjoin%
\definecolor{currentfill}{rgb}{0.501961,0.501961,0.501961}%
\pgfsetfillcolor{currentfill}%
\pgfsetfillopacity{0.700000}%
\pgfsetlinewidth{0.000000pt}%
\definecolor{currentstroke}{rgb}{0.000000,0.000000,0.000000}%
\pgfsetstrokecolor{currentstroke}%
\pgfsetstrokeopacity{0.700000}%
\pgfsetdash{}{0pt}%
\pgfpathmoveto{\pgfqpoint{5.262607in}{0.550000in}}%
\pgfpathlineto{\pgfqpoint{5.290128in}{0.550000in}}%
\pgfpathlineto{\pgfqpoint{5.290128in}{0.677196in}}%
\pgfpathlineto{\pgfqpoint{5.262607in}{0.677196in}}%
\pgfpathlineto{\pgfqpoint{5.262607in}{0.550000in}}%
\pgfpathclose%
\pgfusepath{fill}%
\end{pgfscope}%
\begin{pgfscope}%
\pgfpathrectangle{\pgfqpoint{1.250000in}{0.550000in}}{\pgfqpoint{7.750000in}{3.850000in}}%
\pgfusepath{clip}%
\pgfsetbuttcap%
\pgfsetmiterjoin%
\definecolor{currentfill}{rgb}{0.501961,0.501961,0.501961}%
\pgfsetfillcolor{currentfill}%
\pgfsetfillopacity{0.700000}%
\pgfsetlinewidth{0.000000pt}%
\definecolor{currentstroke}{rgb}{0.000000,0.000000,0.000000}%
\pgfsetstrokecolor{currentstroke}%
\pgfsetstrokeopacity{0.700000}%
\pgfsetdash{}{0pt}%
\pgfpathmoveto{\pgfqpoint{5.290128in}{0.550000in}}%
\pgfpathlineto{\pgfqpoint{5.317649in}{0.550000in}}%
\pgfpathlineto{\pgfqpoint{5.317649in}{0.683253in}}%
\pgfpathlineto{\pgfqpoint{5.290128in}{0.683253in}}%
\pgfpathlineto{\pgfqpoint{5.290128in}{0.550000in}}%
\pgfpathclose%
\pgfusepath{fill}%
\end{pgfscope}%
\begin{pgfscope}%
\pgfpathrectangle{\pgfqpoint{1.250000in}{0.550000in}}{\pgfqpoint{7.750000in}{3.850000in}}%
\pgfusepath{clip}%
\pgfsetbuttcap%
\pgfsetmiterjoin%
\definecolor{currentfill}{rgb}{0.501961,0.501961,0.501961}%
\pgfsetfillcolor{currentfill}%
\pgfsetfillopacity{0.700000}%
\pgfsetlinewidth{0.000000pt}%
\definecolor{currentstroke}{rgb}{0.000000,0.000000,0.000000}%
\pgfsetstrokecolor{currentstroke}%
\pgfsetstrokeopacity{0.700000}%
\pgfsetdash{}{0pt}%
\pgfpathmoveto{\pgfqpoint{5.317649in}{0.550000in}}%
\pgfpathlineto{\pgfqpoint{5.345170in}{0.550000in}}%
\pgfpathlineto{\pgfqpoint{5.345170in}{0.675269in}}%
\pgfpathlineto{\pgfqpoint{5.317649in}{0.675269in}}%
\pgfpathlineto{\pgfqpoint{5.317649in}{0.550000in}}%
\pgfpathclose%
\pgfusepath{fill}%
\end{pgfscope}%
\begin{pgfscope}%
\pgfpathrectangle{\pgfqpoint{1.250000in}{0.550000in}}{\pgfqpoint{7.750000in}{3.850000in}}%
\pgfusepath{clip}%
\pgfsetbuttcap%
\pgfsetmiterjoin%
\definecolor{currentfill}{rgb}{0.501961,0.501961,0.501961}%
\pgfsetfillcolor{currentfill}%
\pgfsetfillopacity{0.700000}%
\pgfsetlinewidth{0.000000pt}%
\definecolor{currentstroke}{rgb}{0.000000,0.000000,0.000000}%
\pgfsetstrokecolor{currentstroke}%
\pgfsetstrokeopacity{0.700000}%
\pgfsetdash{}{0pt}%
\pgfpathmoveto{\pgfqpoint{5.345170in}{0.550000in}}%
\pgfpathlineto{\pgfqpoint{5.372692in}{0.550000in}}%
\pgfpathlineto{\pgfqpoint{5.372692in}{0.679949in}}%
\pgfpathlineto{\pgfqpoint{5.345170in}{0.679949in}}%
\pgfpathlineto{\pgfqpoint{5.345170in}{0.550000in}}%
\pgfpathclose%
\pgfusepath{fill}%
\end{pgfscope}%
\begin{pgfscope}%
\pgfpathrectangle{\pgfqpoint{1.250000in}{0.550000in}}{\pgfqpoint{7.750000in}{3.850000in}}%
\pgfusepath{clip}%
\pgfsetbuttcap%
\pgfsetmiterjoin%
\definecolor{currentfill}{rgb}{0.501961,0.501961,0.501961}%
\pgfsetfillcolor{currentfill}%
\pgfsetfillopacity{0.700000}%
\pgfsetlinewidth{0.000000pt}%
\definecolor{currentstroke}{rgb}{0.000000,0.000000,0.000000}%
\pgfsetstrokecolor{currentstroke}%
\pgfsetstrokeopacity{0.700000}%
\pgfsetdash{}{0pt}%
\pgfpathmoveto{\pgfqpoint{5.372692in}{0.550000in}}%
\pgfpathlineto{\pgfqpoint{5.400213in}{0.550000in}}%
\pgfpathlineto{\pgfqpoint{5.400213in}{0.679949in}}%
\pgfpathlineto{\pgfqpoint{5.372692in}{0.679949in}}%
\pgfpathlineto{\pgfqpoint{5.372692in}{0.550000in}}%
\pgfpathclose%
\pgfusepath{fill}%
\end{pgfscope}%
\begin{pgfscope}%
\pgfpathrectangle{\pgfqpoint{1.250000in}{0.550000in}}{\pgfqpoint{7.750000in}{3.850000in}}%
\pgfusepath{clip}%
\pgfsetbuttcap%
\pgfsetmiterjoin%
\definecolor{currentfill}{rgb}{0.501961,0.501961,0.501961}%
\pgfsetfillcolor{currentfill}%
\pgfsetfillopacity{0.700000}%
\pgfsetlinewidth{0.000000pt}%
\definecolor{currentstroke}{rgb}{0.000000,0.000000,0.000000}%
\pgfsetstrokecolor{currentstroke}%
\pgfsetstrokeopacity{0.700000}%
\pgfsetdash{}{0pt}%
\pgfpathmoveto{\pgfqpoint{5.400213in}{0.550000in}}%
\pgfpathlineto{\pgfqpoint{5.427734in}{0.550000in}}%
\pgfpathlineto{\pgfqpoint{5.427734in}{0.684079in}}%
\pgfpathlineto{\pgfqpoint{5.400213in}{0.684079in}}%
\pgfpathlineto{\pgfqpoint{5.400213in}{0.550000in}}%
\pgfpathclose%
\pgfusepath{fill}%
\end{pgfscope}%
\begin{pgfscope}%
\pgfpathrectangle{\pgfqpoint{1.250000in}{0.550000in}}{\pgfqpoint{7.750000in}{3.850000in}}%
\pgfusepath{clip}%
\pgfsetbuttcap%
\pgfsetmiterjoin%
\definecolor{currentfill}{rgb}{0.501961,0.501961,0.501961}%
\pgfsetfillcolor{currentfill}%
\pgfsetfillopacity{0.700000}%
\pgfsetlinewidth{0.000000pt}%
\definecolor{currentstroke}{rgb}{0.000000,0.000000,0.000000}%
\pgfsetstrokecolor{currentstroke}%
\pgfsetstrokeopacity{0.700000}%
\pgfsetdash{}{0pt}%
\pgfpathmoveto{\pgfqpoint{5.427734in}{0.550000in}}%
\pgfpathlineto{\pgfqpoint{5.455256in}{0.550000in}}%
\pgfpathlineto{\pgfqpoint{5.455256in}{0.697570in}}%
\pgfpathlineto{\pgfqpoint{5.427734in}{0.697570in}}%
\pgfpathlineto{\pgfqpoint{5.427734in}{0.550000in}}%
\pgfpathclose%
\pgfusepath{fill}%
\end{pgfscope}%
\begin{pgfscope}%
\pgfpathrectangle{\pgfqpoint{1.250000in}{0.550000in}}{\pgfqpoint{7.750000in}{3.850000in}}%
\pgfusepath{clip}%
\pgfsetbuttcap%
\pgfsetmiterjoin%
\definecolor{currentfill}{rgb}{0.501961,0.501961,0.501961}%
\pgfsetfillcolor{currentfill}%
\pgfsetfillopacity{0.700000}%
\pgfsetlinewidth{0.000000pt}%
\definecolor{currentstroke}{rgb}{0.000000,0.000000,0.000000}%
\pgfsetstrokecolor{currentstroke}%
\pgfsetstrokeopacity{0.700000}%
\pgfsetdash{}{0pt}%
\pgfpathmoveto{\pgfqpoint{5.455256in}{0.550000in}}%
\pgfpathlineto{\pgfqpoint{5.482777in}{0.550000in}}%
\pgfpathlineto{\pgfqpoint{5.482777in}{0.686282in}}%
\pgfpathlineto{\pgfqpoint{5.455256in}{0.686282in}}%
\pgfpathlineto{\pgfqpoint{5.455256in}{0.550000in}}%
\pgfpathclose%
\pgfusepath{fill}%
\end{pgfscope}%
\begin{pgfscope}%
\pgfpathrectangle{\pgfqpoint{1.250000in}{0.550000in}}{\pgfqpoint{7.750000in}{3.850000in}}%
\pgfusepath{clip}%
\pgfsetbuttcap%
\pgfsetmiterjoin%
\definecolor{currentfill}{rgb}{0.501961,0.501961,0.501961}%
\pgfsetfillcolor{currentfill}%
\pgfsetfillopacity{0.700000}%
\pgfsetlinewidth{0.000000pt}%
\definecolor{currentstroke}{rgb}{0.000000,0.000000,0.000000}%
\pgfsetstrokecolor{currentstroke}%
\pgfsetstrokeopacity{0.700000}%
\pgfsetdash{}{0pt}%
\pgfpathmoveto{\pgfqpoint{5.482777in}{0.550000in}}%
\pgfpathlineto{\pgfqpoint{5.510298in}{0.550000in}}%
\pgfpathlineto{\pgfqpoint{5.510298in}{0.682427in}}%
\pgfpathlineto{\pgfqpoint{5.482777in}{0.682427in}}%
\pgfpathlineto{\pgfqpoint{5.482777in}{0.550000in}}%
\pgfpathclose%
\pgfusepath{fill}%
\end{pgfscope}%
\begin{pgfscope}%
\pgfpathrectangle{\pgfqpoint{1.250000in}{0.550000in}}{\pgfqpoint{7.750000in}{3.850000in}}%
\pgfusepath{clip}%
\pgfsetbuttcap%
\pgfsetmiterjoin%
\definecolor{currentfill}{rgb}{0.501961,0.501961,0.501961}%
\pgfsetfillcolor{currentfill}%
\pgfsetfillopacity{0.700000}%
\pgfsetlinewidth{0.000000pt}%
\definecolor{currentstroke}{rgb}{0.000000,0.000000,0.000000}%
\pgfsetstrokecolor{currentstroke}%
\pgfsetstrokeopacity{0.700000}%
\pgfsetdash{}{0pt}%
\pgfpathmoveto{\pgfqpoint{5.510298in}{0.550000in}}%
\pgfpathlineto{\pgfqpoint{5.537820in}{0.550000in}}%
\pgfpathlineto{\pgfqpoint{5.537820in}{0.683529in}}%
\pgfpathlineto{\pgfqpoint{5.510298in}{0.683529in}}%
\pgfpathlineto{\pgfqpoint{5.510298in}{0.550000in}}%
\pgfpathclose%
\pgfusepath{fill}%
\end{pgfscope}%
\begin{pgfscope}%
\pgfpathrectangle{\pgfqpoint{1.250000in}{0.550000in}}{\pgfqpoint{7.750000in}{3.850000in}}%
\pgfusepath{clip}%
\pgfsetbuttcap%
\pgfsetmiterjoin%
\definecolor{currentfill}{rgb}{0.501961,0.501961,0.501961}%
\pgfsetfillcolor{currentfill}%
\pgfsetfillopacity{0.700000}%
\pgfsetlinewidth{0.000000pt}%
\definecolor{currentstroke}{rgb}{0.000000,0.000000,0.000000}%
\pgfsetstrokecolor{currentstroke}%
\pgfsetstrokeopacity{0.700000}%
\pgfsetdash{}{0pt}%
\pgfpathmoveto{\pgfqpoint{5.537820in}{0.550000in}}%
\pgfpathlineto{\pgfqpoint{5.565341in}{0.550000in}}%
\pgfpathlineto{\pgfqpoint{5.565341in}{0.684079in}}%
\pgfpathlineto{\pgfqpoint{5.537820in}{0.684079in}}%
\pgfpathlineto{\pgfqpoint{5.537820in}{0.550000in}}%
\pgfpathclose%
\pgfusepath{fill}%
\end{pgfscope}%
\begin{pgfscope}%
\pgfpathrectangle{\pgfqpoint{1.250000in}{0.550000in}}{\pgfqpoint{7.750000in}{3.850000in}}%
\pgfusepath{clip}%
\pgfsetbuttcap%
\pgfsetmiterjoin%
\definecolor{currentfill}{rgb}{0.501961,0.501961,0.501961}%
\pgfsetfillcolor{currentfill}%
\pgfsetfillopacity{0.700000}%
\pgfsetlinewidth{0.000000pt}%
\definecolor{currentstroke}{rgb}{0.000000,0.000000,0.000000}%
\pgfsetstrokecolor{currentstroke}%
\pgfsetstrokeopacity{0.700000}%
\pgfsetdash{}{0pt}%
\pgfpathmoveto{\pgfqpoint{5.565341in}{0.550000in}}%
\pgfpathlineto{\pgfqpoint{5.592862in}{0.550000in}}%
\pgfpathlineto{\pgfqpoint{5.592862in}{0.678022in}}%
\pgfpathlineto{\pgfqpoint{5.565341in}{0.678022in}}%
\pgfpathlineto{\pgfqpoint{5.565341in}{0.550000in}}%
\pgfpathclose%
\pgfusepath{fill}%
\end{pgfscope}%
\begin{pgfscope}%
\pgfpathrectangle{\pgfqpoint{1.250000in}{0.550000in}}{\pgfqpoint{7.750000in}{3.850000in}}%
\pgfusepath{clip}%
\pgfsetbuttcap%
\pgfsetmiterjoin%
\definecolor{currentfill}{rgb}{0.501961,0.501961,0.501961}%
\pgfsetfillcolor{currentfill}%
\pgfsetfillopacity{0.700000}%
\pgfsetlinewidth{0.000000pt}%
\definecolor{currentstroke}{rgb}{0.000000,0.000000,0.000000}%
\pgfsetstrokecolor{currentstroke}%
\pgfsetstrokeopacity{0.700000}%
\pgfsetdash{}{0pt}%
\pgfpathmoveto{\pgfqpoint{5.592862in}{0.550000in}}%
\pgfpathlineto{\pgfqpoint{5.620384in}{0.550000in}}%
\pgfpathlineto{\pgfqpoint{5.620384in}{0.692063in}}%
\pgfpathlineto{\pgfqpoint{5.592862in}{0.692063in}}%
\pgfpathlineto{\pgfqpoint{5.592862in}{0.550000in}}%
\pgfpathclose%
\pgfusepath{fill}%
\end{pgfscope}%
\begin{pgfscope}%
\pgfpathrectangle{\pgfqpoint{1.250000in}{0.550000in}}{\pgfqpoint{7.750000in}{3.850000in}}%
\pgfusepath{clip}%
\pgfsetbuttcap%
\pgfsetmiterjoin%
\definecolor{currentfill}{rgb}{0.501961,0.501961,0.501961}%
\pgfsetfillcolor{currentfill}%
\pgfsetfillopacity{0.700000}%
\pgfsetlinewidth{0.000000pt}%
\definecolor{currentstroke}{rgb}{0.000000,0.000000,0.000000}%
\pgfsetstrokecolor{currentstroke}%
\pgfsetstrokeopacity{0.700000}%
\pgfsetdash{}{0pt}%
\pgfpathmoveto{\pgfqpoint{5.620384in}{0.550000in}}%
\pgfpathlineto{\pgfqpoint{5.647905in}{0.550000in}}%
\pgfpathlineto{\pgfqpoint{5.647905in}{0.681051in}}%
\pgfpathlineto{\pgfqpoint{5.620384in}{0.681051in}}%
\pgfpathlineto{\pgfqpoint{5.620384in}{0.550000in}}%
\pgfpathclose%
\pgfusepath{fill}%
\end{pgfscope}%
\begin{pgfscope}%
\pgfpathrectangle{\pgfqpoint{1.250000in}{0.550000in}}{\pgfqpoint{7.750000in}{3.850000in}}%
\pgfusepath{clip}%
\pgfsetbuttcap%
\pgfsetmiterjoin%
\definecolor{currentfill}{rgb}{0.501961,0.501961,0.501961}%
\pgfsetfillcolor{currentfill}%
\pgfsetfillopacity{0.700000}%
\pgfsetlinewidth{0.000000pt}%
\definecolor{currentstroke}{rgb}{0.000000,0.000000,0.000000}%
\pgfsetstrokecolor{currentstroke}%
\pgfsetstrokeopacity{0.700000}%
\pgfsetdash{}{0pt}%
\pgfpathmoveto{\pgfqpoint{5.647905in}{0.550000in}}%
\pgfpathlineto{\pgfqpoint{5.675426in}{0.550000in}}%
\pgfpathlineto{\pgfqpoint{5.675426in}{0.697570in}}%
\pgfpathlineto{\pgfqpoint{5.647905in}{0.697570in}}%
\pgfpathlineto{\pgfqpoint{5.647905in}{0.550000in}}%
\pgfpathclose%
\pgfusepath{fill}%
\end{pgfscope}%
\begin{pgfscope}%
\pgfpathrectangle{\pgfqpoint{1.250000in}{0.550000in}}{\pgfqpoint{7.750000in}{3.850000in}}%
\pgfusepath{clip}%
\pgfsetbuttcap%
\pgfsetmiterjoin%
\definecolor{currentfill}{rgb}{0.501961,0.501961,0.501961}%
\pgfsetfillcolor{currentfill}%
\pgfsetfillopacity{0.700000}%
\pgfsetlinewidth{0.000000pt}%
\definecolor{currentstroke}{rgb}{0.000000,0.000000,0.000000}%
\pgfsetstrokecolor{currentstroke}%
\pgfsetstrokeopacity{0.700000}%
\pgfsetdash{}{0pt}%
\pgfpathmoveto{\pgfqpoint{5.675426in}{0.550000in}}%
\pgfpathlineto{\pgfqpoint{5.702947in}{0.550000in}}%
\pgfpathlineto{\pgfqpoint{5.702947in}{0.702250in}}%
\pgfpathlineto{\pgfqpoint{5.675426in}{0.702250in}}%
\pgfpathlineto{\pgfqpoint{5.675426in}{0.550000in}}%
\pgfpathclose%
\pgfusepath{fill}%
\end{pgfscope}%
\begin{pgfscope}%
\pgfpathrectangle{\pgfqpoint{1.250000in}{0.550000in}}{\pgfqpoint{7.750000in}{3.850000in}}%
\pgfusepath{clip}%
\pgfsetbuttcap%
\pgfsetmiterjoin%
\definecolor{currentfill}{rgb}{0.501961,0.501961,0.501961}%
\pgfsetfillcolor{currentfill}%
\pgfsetfillopacity{0.700000}%
\pgfsetlinewidth{0.000000pt}%
\definecolor{currentstroke}{rgb}{0.000000,0.000000,0.000000}%
\pgfsetstrokecolor{currentstroke}%
\pgfsetstrokeopacity{0.700000}%
\pgfsetdash{}{0pt}%
\pgfpathmoveto{\pgfqpoint{5.702947in}{0.550000in}}%
\pgfpathlineto{\pgfqpoint{5.730469in}{0.550000in}}%
\pgfpathlineto{\pgfqpoint{5.730469in}{0.697845in}}%
\pgfpathlineto{\pgfqpoint{5.702947in}{0.697845in}}%
\pgfpathlineto{\pgfqpoint{5.702947in}{0.550000in}}%
\pgfpathclose%
\pgfusepath{fill}%
\end{pgfscope}%
\begin{pgfscope}%
\pgfpathrectangle{\pgfqpoint{1.250000in}{0.550000in}}{\pgfqpoint{7.750000in}{3.850000in}}%
\pgfusepath{clip}%
\pgfsetbuttcap%
\pgfsetmiterjoin%
\definecolor{currentfill}{rgb}{0.501961,0.501961,0.501961}%
\pgfsetfillcolor{currentfill}%
\pgfsetfillopacity{0.700000}%
\pgfsetlinewidth{0.000000pt}%
\definecolor{currentstroke}{rgb}{0.000000,0.000000,0.000000}%
\pgfsetstrokecolor{currentstroke}%
\pgfsetstrokeopacity{0.700000}%
\pgfsetdash{}{0pt}%
\pgfpathmoveto{\pgfqpoint{5.730469in}{0.550000in}}%
\pgfpathlineto{\pgfqpoint{5.757990in}{0.550000in}}%
\pgfpathlineto{\pgfqpoint{5.757990in}{0.703627in}}%
\pgfpathlineto{\pgfqpoint{5.730469in}{0.703627in}}%
\pgfpathlineto{\pgfqpoint{5.730469in}{0.550000in}}%
\pgfpathclose%
\pgfusepath{fill}%
\end{pgfscope}%
\begin{pgfscope}%
\pgfpathrectangle{\pgfqpoint{1.250000in}{0.550000in}}{\pgfqpoint{7.750000in}{3.850000in}}%
\pgfusepath{clip}%
\pgfsetbuttcap%
\pgfsetmiterjoin%
\definecolor{currentfill}{rgb}{0.501961,0.501961,0.501961}%
\pgfsetfillcolor{currentfill}%
\pgfsetfillopacity{0.700000}%
\pgfsetlinewidth{0.000000pt}%
\definecolor{currentstroke}{rgb}{0.000000,0.000000,0.000000}%
\pgfsetstrokecolor{currentstroke}%
\pgfsetstrokeopacity{0.700000}%
\pgfsetdash{}{0pt}%
\pgfpathmoveto{\pgfqpoint{5.757990in}{0.550000in}}%
\pgfpathlineto{\pgfqpoint{5.785511in}{0.550000in}}%
\pgfpathlineto{\pgfqpoint{5.785511in}{0.709133in}}%
\pgfpathlineto{\pgfqpoint{5.757990in}{0.709133in}}%
\pgfpathlineto{\pgfqpoint{5.757990in}{0.550000in}}%
\pgfpathclose%
\pgfusepath{fill}%
\end{pgfscope}%
\begin{pgfscope}%
\pgfpathrectangle{\pgfqpoint{1.250000in}{0.550000in}}{\pgfqpoint{7.750000in}{3.850000in}}%
\pgfusepath{clip}%
\pgfsetbuttcap%
\pgfsetmiterjoin%
\definecolor{currentfill}{rgb}{0.501961,0.501961,0.501961}%
\pgfsetfillcolor{currentfill}%
\pgfsetfillopacity{0.700000}%
\pgfsetlinewidth{0.000000pt}%
\definecolor{currentstroke}{rgb}{0.000000,0.000000,0.000000}%
\pgfsetstrokecolor{currentstroke}%
\pgfsetstrokeopacity{0.700000}%
\pgfsetdash{}{0pt}%
\pgfpathmoveto{\pgfqpoint{5.785511in}{0.550000in}}%
\pgfpathlineto{\pgfqpoint{5.813033in}{0.550000in}}%
\pgfpathlineto{\pgfqpoint{5.813033in}{0.703351in}}%
\pgfpathlineto{\pgfqpoint{5.785511in}{0.703351in}}%
\pgfpathlineto{\pgfqpoint{5.785511in}{0.550000in}}%
\pgfpathclose%
\pgfusepath{fill}%
\end{pgfscope}%
\begin{pgfscope}%
\pgfpathrectangle{\pgfqpoint{1.250000in}{0.550000in}}{\pgfqpoint{7.750000in}{3.850000in}}%
\pgfusepath{clip}%
\pgfsetbuttcap%
\pgfsetmiterjoin%
\definecolor{currentfill}{rgb}{0.501961,0.501961,0.501961}%
\pgfsetfillcolor{currentfill}%
\pgfsetfillopacity{0.700000}%
\pgfsetlinewidth{0.000000pt}%
\definecolor{currentstroke}{rgb}{0.000000,0.000000,0.000000}%
\pgfsetstrokecolor{currentstroke}%
\pgfsetstrokeopacity{0.700000}%
\pgfsetdash{}{0pt}%
\pgfpathmoveto{\pgfqpoint{5.813033in}{0.550000in}}%
\pgfpathlineto{\pgfqpoint{5.840554in}{0.550000in}}%
\pgfpathlineto{\pgfqpoint{5.840554in}{0.717117in}}%
\pgfpathlineto{\pgfqpoint{5.813033in}{0.717117in}}%
\pgfpathlineto{\pgfqpoint{5.813033in}{0.550000in}}%
\pgfpathclose%
\pgfusepath{fill}%
\end{pgfscope}%
\begin{pgfscope}%
\pgfpathrectangle{\pgfqpoint{1.250000in}{0.550000in}}{\pgfqpoint{7.750000in}{3.850000in}}%
\pgfusepath{clip}%
\pgfsetbuttcap%
\pgfsetmiterjoin%
\definecolor{currentfill}{rgb}{0.501961,0.501961,0.501961}%
\pgfsetfillcolor{currentfill}%
\pgfsetfillopacity{0.700000}%
\pgfsetlinewidth{0.000000pt}%
\definecolor{currentstroke}{rgb}{0.000000,0.000000,0.000000}%
\pgfsetstrokecolor{currentstroke}%
\pgfsetstrokeopacity{0.700000}%
\pgfsetdash{}{0pt}%
\pgfpathmoveto{\pgfqpoint{5.840554in}{0.550000in}}%
\pgfpathlineto{\pgfqpoint{5.868075in}{0.550000in}}%
\pgfpathlineto{\pgfqpoint{5.868075in}{0.711060in}}%
\pgfpathlineto{\pgfqpoint{5.840554in}{0.711060in}}%
\pgfpathlineto{\pgfqpoint{5.840554in}{0.550000in}}%
\pgfpathclose%
\pgfusepath{fill}%
\end{pgfscope}%
\begin{pgfscope}%
\pgfpathrectangle{\pgfqpoint{1.250000in}{0.550000in}}{\pgfqpoint{7.750000in}{3.850000in}}%
\pgfusepath{clip}%
\pgfsetbuttcap%
\pgfsetmiterjoin%
\definecolor{currentfill}{rgb}{0.501961,0.501961,0.501961}%
\pgfsetfillcolor{currentfill}%
\pgfsetfillopacity{0.700000}%
\pgfsetlinewidth{0.000000pt}%
\definecolor{currentstroke}{rgb}{0.000000,0.000000,0.000000}%
\pgfsetstrokecolor{currentstroke}%
\pgfsetstrokeopacity{0.700000}%
\pgfsetdash{}{0pt}%
\pgfpathmoveto{\pgfqpoint{5.868075in}{0.550000in}}%
\pgfpathlineto{\pgfqpoint{5.895597in}{0.550000in}}%
\pgfpathlineto{\pgfqpoint{5.895597in}{0.701975in}}%
\pgfpathlineto{\pgfqpoint{5.868075in}{0.701975in}}%
\pgfpathlineto{\pgfqpoint{5.868075in}{0.550000in}}%
\pgfpathclose%
\pgfusepath{fill}%
\end{pgfscope}%
\begin{pgfscope}%
\pgfpathrectangle{\pgfqpoint{1.250000in}{0.550000in}}{\pgfqpoint{7.750000in}{3.850000in}}%
\pgfusepath{clip}%
\pgfsetbuttcap%
\pgfsetmiterjoin%
\definecolor{currentfill}{rgb}{0.501961,0.501961,0.501961}%
\pgfsetfillcolor{currentfill}%
\pgfsetfillopacity{0.700000}%
\pgfsetlinewidth{0.000000pt}%
\definecolor{currentstroke}{rgb}{0.000000,0.000000,0.000000}%
\pgfsetstrokecolor{currentstroke}%
\pgfsetstrokeopacity{0.700000}%
\pgfsetdash{}{0pt}%
\pgfpathmoveto{\pgfqpoint{5.895597in}{0.550000in}}%
\pgfpathlineto{\pgfqpoint{5.923118in}{0.550000in}}%
\pgfpathlineto{\pgfqpoint{5.923118in}{0.704728in}}%
\pgfpathlineto{\pgfqpoint{5.895597in}{0.704728in}}%
\pgfpathlineto{\pgfqpoint{5.895597in}{0.550000in}}%
\pgfpathclose%
\pgfusepath{fill}%
\end{pgfscope}%
\begin{pgfscope}%
\pgfpathrectangle{\pgfqpoint{1.250000in}{0.550000in}}{\pgfqpoint{7.750000in}{3.850000in}}%
\pgfusepath{clip}%
\pgfsetbuttcap%
\pgfsetmiterjoin%
\definecolor{currentfill}{rgb}{0.501961,0.501961,0.501961}%
\pgfsetfillcolor{currentfill}%
\pgfsetfillopacity{0.700000}%
\pgfsetlinewidth{0.000000pt}%
\definecolor{currentstroke}{rgb}{0.000000,0.000000,0.000000}%
\pgfsetstrokecolor{currentstroke}%
\pgfsetstrokeopacity{0.700000}%
\pgfsetdash{}{0pt}%
\pgfpathmoveto{\pgfqpoint{5.923118in}{0.550000in}}%
\pgfpathlineto{\pgfqpoint{5.950639in}{0.550000in}}%
\pgfpathlineto{\pgfqpoint{5.950639in}{0.705829in}}%
\pgfpathlineto{\pgfqpoint{5.923118in}{0.705829in}}%
\pgfpathlineto{\pgfqpoint{5.923118in}{0.550000in}}%
\pgfpathclose%
\pgfusepath{fill}%
\end{pgfscope}%
\begin{pgfscope}%
\pgfpathrectangle{\pgfqpoint{1.250000in}{0.550000in}}{\pgfqpoint{7.750000in}{3.850000in}}%
\pgfusepath{clip}%
\pgfsetbuttcap%
\pgfsetmiterjoin%
\definecolor{currentfill}{rgb}{0.501961,0.501961,0.501961}%
\pgfsetfillcolor{currentfill}%
\pgfsetfillopacity{0.700000}%
\pgfsetlinewidth{0.000000pt}%
\definecolor{currentstroke}{rgb}{0.000000,0.000000,0.000000}%
\pgfsetstrokecolor{currentstroke}%
\pgfsetstrokeopacity{0.700000}%
\pgfsetdash{}{0pt}%
\pgfpathmoveto{\pgfqpoint{5.950639in}{0.550000in}}%
\pgfpathlineto{\pgfqpoint{5.978161in}{0.550000in}}%
\pgfpathlineto{\pgfqpoint{5.978161in}{0.720972in}}%
\pgfpathlineto{\pgfqpoint{5.950639in}{0.720972in}}%
\pgfpathlineto{\pgfqpoint{5.950639in}{0.550000in}}%
\pgfpathclose%
\pgfusepath{fill}%
\end{pgfscope}%
\begin{pgfscope}%
\pgfpathrectangle{\pgfqpoint{1.250000in}{0.550000in}}{\pgfqpoint{7.750000in}{3.850000in}}%
\pgfusepath{clip}%
\pgfsetbuttcap%
\pgfsetmiterjoin%
\definecolor{currentfill}{rgb}{0.501961,0.501961,0.501961}%
\pgfsetfillcolor{currentfill}%
\pgfsetfillopacity{0.700000}%
\pgfsetlinewidth{0.000000pt}%
\definecolor{currentstroke}{rgb}{0.000000,0.000000,0.000000}%
\pgfsetstrokecolor{currentstroke}%
\pgfsetstrokeopacity{0.700000}%
\pgfsetdash{}{0pt}%
\pgfpathmoveto{\pgfqpoint{5.978161in}{0.550000in}}%
\pgfpathlineto{\pgfqpoint{6.005682in}{0.550000in}}%
\pgfpathlineto{\pgfqpoint{6.005682in}{0.697294in}}%
\pgfpathlineto{\pgfqpoint{5.978161in}{0.697294in}}%
\pgfpathlineto{\pgfqpoint{5.978161in}{0.550000in}}%
\pgfpathclose%
\pgfusepath{fill}%
\end{pgfscope}%
\begin{pgfscope}%
\pgfpathrectangle{\pgfqpoint{1.250000in}{0.550000in}}{\pgfqpoint{7.750000in}{3.850000in}}%
\pgfusepath{clip}%
\pgfsetbuttcap%
\pgfsetmiterjoin%
\definecolor{currentfill}{rgb}{0.501961,0.501961,0.501961}%
\pgfsetfillcolor{currentfill}%
\pgfsetfillopacity{0.700000}%
\pgfsetlinewidth{0.000000pt}%
\definecolor{currentstroke}{rgb}{0.000000,0.000000,0.000000}%
\pgfsetstrokecolor{currentstroke}%
\pgfsetstrokeopacity{0.700000}%
\pgfsetdash{}{0pt}%
\pgfpathmoveto{\pgfqpoint{6.005682in}{0.550000in}}%
\pgfpathlineto{\pgfqpoint{6.033203in}{0.550000in}}%
\pgfpathlineto{\pgfqpoint{6.033203in}{0.688484in}}%
\pgfpathlineto{\pgfqpoint{6.005682in}{0.688484in}}%
\pgfpathlineto{\pgfqpoint{6.005682in}{0.550000in}}%
\pgfpathclose%
\pgfusepath{fill}%
\end{pgfscope}%
\begin{pgfscope}%
\pgfpathrectangle{\pgfqpoint{1.250000in}{0.550000in}}{\pgfqpoint{7.750000in}{3.850000in}}%
\pgfusepath{clip}%
\pgfsetbuttcap%
\pgfsetmiterjoin%
\definecolor{currentfill}{rgb}{0.501961,0.501961,0.501961}%
\pgfsetfillcolor{currentfill}%
\pgfsetfillopacity{0.700000}%
\pgfsetlinewidth{0.000000pt}%
\definecolor{currentstroke}{rgb}{0.000000,0.000000,0.000000}%
\pgfsetstrokecolor{currentstroke}%
\pgfsetstrokeopacity{0.700000}%
\pgfsetdash{}{0pt}%
\pgfpathmoveto{\pgfqpoint{6.033203in}{0.550000in}}%
\pgfpathlineto{\pgfqpoint{6.060724in}{0.550000in}}%
\pgfpathlineto{\pgfqpoint{6.060724in}{0.681877in}}%
\pgfpathlineto{\pgfqpoint{6.033203in}{0.681877in}}%
\pgfpathlineto{\pgfqpoint{6.033203in}{0.550000in}}%
\pgfpathclose%
\pgfusepath{fill}%
\end{pgfscope}%
\begin{pgfscope}%
\pgfpathrectangle{\pgfqpoint{1.250000in}{0.550000in}}{\pgfqpoint{7.750000in}{3.850000in}}%
\pgfusepath{clip}%
\pgfsetbuttcap%
\pgfsetmiterjoin%
\definecolor{currentfill}{rgb}{0.501961,0.501961,0.501961}%
\pgfsetfillcolor{currentfill}%
\pgfsetfillopacity{0.700000}%
\pgfsetlinewidth{0.000000pt}%
\definecolor{currentstroke}{rgb}{0.000000,0.000000,0.000000}%
\pgfsetstrokecolor{currentstroke}%
\pgfsetstrokeopacity{0.700000}%
\pgfsetdash{}{0pt}%
\pgfpathmoveto{\pgfqpoint{6.060724in}{0.550000in}}%
\pgfpathlineto{\pgfqpoint{6.088246in}{0.550000in}}%
\pgfpathlineto{\pgfqpoint{6.088246in}{0.675269in}}%
\pgfpathlineto{\pgfqpoint{6.060724in}{0.675269in}}%
\pgfpathlineto{\pgfqpoint{6.060724in}{0.550000in}}%
\pgfpathclose%
\pgfusepath{fill}%
\end{pgfscope}%
\begin{pgfscope}%
\pgfpathrectangle{\pgfqpoint{1.250000in}{0.550000in}}{\pgfqpoint{7.750000in}{3.850000in}}%
\pgfusepath{clip}%
\pgfsetbuttcap%
\pgfsetmiterjoin%
\definecolor{currentfill}{rgb}{0.501961,0.501961,0.501961}%
\pgfsetfillcolor{currentfill}%
\pgfsetfillopacity{0.700000}%
\pgfsetlinewidth{0.000000pt}%
\definecolor{currentstroke}{rgb}{0.000000,0.000000,0.000000}%
\pgfsetstrokecolor{currentstroke}%
\pgfsetstrokeopacity{0.700000}%
\pgfsetdash{}{0pt}%
\pgfpathmoveto{\pgfqpoint{6.088246in}{0.550000in}}%
\pgfpathlineto{\pgfqpoint{6.115767in}{0.550000in}}%
\pgfpathlineto{\pgfqpoint{6.115767in}{0.667285in}}%
\pgfpathlineto{\pgfqpoint{6.088246in}{0.667285in}}%
\pgfpathlineto{\pgfqpoint{6.088246in}{0.550000in}}%
\pgfpathclose%
\pgfusepath{fill}%
\end{pgfscope}%
\begin{pgfscope}%
\pgfpathrectangle{\pgfqpoint{1.250000in}{0.550000in}}{\pgfqpoint{7.750000in}{3.850000in}}%
\pgfusepath{clip}%
\pgfsetbuttcap%
\pgfsetmiterjoin%
\definecolor{currentfill}{rgb}{0.501961,0.501961,0.501961}%
\pgfsetfillcolor{currentfill}%
\pgfsetfillopacity{0.700000}%
\pgfsetlinewidth{0.000000pt}%
\definecolor{currentstroke}{rgb}{0.000000,0.000000,0.000000}%
\pgfsetstrokecolor{currentstroke}%
\pgfsetstrokeopacity{0.700000}%
\pgfsetdash{}{0pt}%
\pgfpathmoveto{\pgfqpoint{6.115767in}{0.550000in}}%
\pgfpathlineto{\pgfqpoint{6.143288in}{0.550000in}}%
\pgfpathlineto{\pgfqpoint{6.143288in}{0.662604in}}%
\pgfpathlineto{\pgfqpoint{6.115767in}{0.662604in}}%
\pgfpathlineto{\pgfqpoint{6.115767in}{0.550000in}}%
\pgfpathclose%
\pgfusepath{fill}%
\end{pgfscope}%
\begin{pgfscope}%
\pgfpathrectangle{\pgfqpoint{1.250000in}{0.550000in}}{\pgfqpoint{7.750000in}{3.850000in}}%
\pgfusepath{clip}%
\pgfsetbuttcap%
\pgfsetmiterjoin%
\definecolor{currentfill}{rgb}{0.501961,0.501961,0.501961}%
\pgfsetfillcolor{currentfill}%
\pgfsetfillopacity{0.700000}%
\pgfsetlinewidth{0.000000pt}%
\definecolor{currentstroke}{rgb}{0.000000,0.000000,0.000000}%
\pgfsetstrokecolor{currentstroke}%
\pgfsetstrokeopacity{0.700000}%
\pgfsetdash{}{0pt}%
\pgfpathmoveto{\pgfqpoint{6.143288in}{0.550000in}}%
\pgfpathlineto{\pgfqpoint{6.170810in}{0.550000in}}%
\pgfpathlineto{\pgfqpoint{6.170810in}{0.665082in}}%
\pgfpathlineto{\pgfqpoint{6.143288in}{0.665082in}}%
\pgfpathlineto{\pgfqpoint{6.143288in}{0.550000in}}%
\pgfpathclose%
\pgfusepath{fill}%
\end{pgfscope}%
\begin{pgfscope}%
\pgfpathrectangle{\pgfqpoint{1.250000in}{0.550000in}}{\pgfqpoint{7.750000in}{3.850000in}}%
\pgfusepath{clip}%
\pgfsetbuttcap%
\pgfsetmiterjoin%
\definecolor{currentfill}{rgb}{0.501961,0.501961,0.501961}%
\pgfsetfillcolor{currentfill}%
\pgfsetfillopacity{0.700000}%
\pgfsetlinewidth{0.000000pt}%
\definecolor{currentstroke}{rgb}{0.000000,0.000000,0.000000}%
\pgfsetstrokecolor{currentstroke}%
\pgfsetstrokeopacity{0.700000}%
\pgfsetdash{}{0pt}%
\pgfpathmoveto{\pgfqpoint{6.170810in}{0.550000in}}%
\pgfpathlineto{\pgfqpoint{6.198331in}{0.550000in}}%
\pgfpathlineto{\pgfqpoint{6.198331in}{0.672241in}}%
\pgfpathlineto{\pgfqpoint{6.170810in}{0.672241in}}%
\pgfpathlineto{\pgfqpoint{6.170810in}{0.550000in}}%
\pgfpathclose%
\pgfusepath{fill}%
\end{pgfscope}%
\begin{pgfscope}%
\pgfpathrectangle{\pgfqpoint{1.250000in}{0.550000in}}{\pgfqpoint{7.750000in}{3.850000in}}%
\pgfusepath{clip}%
\pgfsetbuttcap%
\pgfsetmiterjoin%
\definecolor{currentfill}{rgb}{0.501961,0.501961,0.501961}%
\pgfsetfillcolor{currentfill}%
\pgfsetfillopacity{0.700000}%
\pgfsetlinewidth{0.000000pt}%
\definecolor{currentstroke}{rgb}{0.000000,0.000000,0.000000}%
\pgfsetstrokecolor{currentstroke}%
\pgfsetstrokeopacity{0.700000}%
\pgfsetdash{}{0pt}%
\pgfpathmoveto{\pgfqpoint{6.198331in}{0.550000in}}%
\pgfpathlineto{\pgfqpoint{6.225852in}{0.550000in}}%
\pgfpathlineto{\pgfqpoint{6.225852in}{0.654345in}}%
\pgfpathlineto{\pgfqpoint{6.198331in}{0.654345in}}%
\pgfpathlineto{\pgfqpoint{6.198331in}{0.550000in}}%
\pgfpathclose%
\pgfusepath{fill}%
\end{pgfscope}%
\begin{pgfscope}%
\pgfpathrectangle{\pgfqpoint{1.250000in}{0.550000in}}{\pgfqpoint{7.750000in}{3.850000in}}%
\pgfusepath{clip}%
\pgfsetbuttcap%
\pgfsetmiterjoin%
\definecolor{currentfill}{rgb}{0.501961,0.501961,0.501961}%
\pgfsetfillcolor{currentfill}%
\pgfsetfillopacity{0.700000}%
\pgfsetlinewidth{0.000000pt}%
\definecolor{currentstroke}{rgb}{0.000000,0.000000,0.000000}%
\pgfsetstrokecolor{currentstroke}%
\pgfsetstrokeopacity{0.700000}%
\pgfsetdash{}{0pt}%
\pgfpathmoveto{\pgfqpoint{6.225852in}{0.550000in}}%
\pgfpathlineto{\pgfqpoint{6.253374in}{0.550000in}}%
\pgfpathlineto{\pgfqpoint{6.253374in}{0.648563in}}%
\pgfpathlineto{\pgfqpoint{6.225852in}{0.648563in}}%
\pgfpathlineto{\pgfqpoint{6.225852in}{0.550000in}}%
\pgfpathclose%
\pgfusepath{fill}%
\end{pgfscope}%
\begin{pgfscope}%
\pgfpathrectangle{\pgfqpoint{1.250000in}{0.550000in}}{\pgfqpoint{7.750000in}{3.850000in}}%
\pgfusepath{clip}%
\pgfsetbuttcap%
\pgfsetmiterjoin%
\definecolor{currentfill}{rgb}{0.501961,0.501961,0.501961}%
\pgfsetfillcolor{currentfill}%
\pgfsetfillopacity{0.700000}%
\pgfsetlinewidth{0.000000pt}%
\definecolor{currentstroke}{rgb}{0.000000,0.000000,0.000000}%
\pgfsetstrokecolor{currentstroke}%
\pgfsetstrokeopacity{0.700000}%
\pgfsetdash{}{0pt}%
\pgfpathmoveto{\pgfqpoint{6.253374in}{0.550000in}}%
\pgfpathlineto{\pgfqpoint{6.280895in}{0.550000in}}%
\pgfpathlineto{\pgfqpoint{6.280895in}{0.646085in}}%
\pgfpathlineto{\pgfqpoint{6.253374in}{0.646085in}}%
\pgfpathlineto{\pgfqpoint{6.253374in}{0.550000in}}%
\pgfpathclose%
\pgfusepath{fill}%
\end{pgfscope}%
\begin{pgfscope}%
\pgfpathrectangle{\pgfqpoint{1.250000in}{0.550000in}}{\pgfqpoint{7.750000in}{3.850000in}}%
\pgfusepath{clip}%
\pgfsetbuttcap%
\pgfsetmiterjoin%
\definecolor{currentfill}{rgb}{0.501961,0.501961,0.501961}%
\pgfsetfillcolor{currentfill}%
\pgfsetfillopacity{0.700000}%
\pgfsetlinewidth{0.000000pt}%
\definecolor{currentstroke}{rgb}{0.000000,0.000000,0.000000}%
\pgfsetstrokecolor{currentstroke}%
\pgfsetstrokeopacity{0.700000}%
\pgfsetdash{}{0pt}%
\pgfpathmoveto{\pgfqpoint{6.280895in}{0.550000in}}%
\pgfpathlineto{\pgfqpoint{6.308416in}{0.550000in}}%
\pgfpathlineto{\pgfqpoint{6.308416in}{0.643608in}}%
\pgfpathlineto{\pgfqpoint{6.280895in}{0.643608in}}%
\pgfpathlineto{\pgfqpoint{6.280895in}{0.550000in}}%
\pgfpathclose%
\pgfusepath{fill}%
\end{pgfscope}%
\begin{pgfscope}%
\pgfpathrectangle{\pgfqpoint{1.250000in}{0.550000in}}{\pgfqpoint{7.750000in}{3.850000in}}%
\pgfusepath{clip}%
\pgfsetbuttcap%
\pgfsetmiterjoin%
\definecolor{currentfill}{rgb}{0.501961,0.501961,0.501961}%
\pgfsetfillcolor{currentfill}%
\pgfsetfillopacity{0.700000}%
\pgfsetlinewidth{0.000000pt}%
\definecolor{currentstroke}{rgb}{0.000000,0.000000,0.000000}%
\pgfsetstrokecolor{currentstroke}%
\pgfsetstrokeopacity{0.700000}%
\pgfsetdash{}{0pt}%
\pgfpathmoveto{\pgfqpoint{6.308416in}{0.550000in}}%
\pgfpathlineto{\pgfqpoint{6.335937in}{0.550000in}}%
\pgfpathlineto{\pgfqpoint{6.335937in}{0.642506in}}%
\pgfpathlineto{\pgfqpoint{6.308416in}{0.642506in}}%
\pgfpathlineto{\pgfqpoint{6.308416in}{0.550000in}}%
\pgfpathclose%
\pgfusepath{fill}%
\end{pgfscope}%
\begin{pgfscope}%
\pgfpathrectangle{\pgfqpoint{1.250000in}{0.550000in}}{\pgfqpoint{7.750000in}{3.850000in}}%
\pgfusepath{clip}%
\pgfsetbuttcap%
\pgfsetmiterjoin%
\definecolor{currentfill}{rgb}{0.501961,0.501961,0.501961}%
\pgfsetfillcolor{currentfill}%
\pgfsetfillopacity{0.700000}%
\pgfsetlinewidth{0.000000pt}%
\definecolor{currentstroke}{rgb}{0.000000,0.000000,0.000000}%
\pgfsetstrokecolor{currentstroke}%
\pgfsetstrokeopacity{0.700000}%
\pgfsetdash{}{0pt}%
\pgfpathmoveto{\pgfqpoint{6.335938in}{0.550000in}}%
\pgfpathlineto{\pgfqpoint{6.363459in}{0.550000in}}%
\pgfpathlineto{\pgfqpoint{6.363459in}{0.635348in}}%
\pgfpathlineto{\pgfqpoint{6.335938in}{0.635348in}}%
\pgfpathlineto{\pgfqpoint{6.335938in}{0.550000in}}%
\pgfpathclose%
\pgfusepath{fill}%
\end{pgfscope}%
\begin{pgfscope}%
\pgfpathrectangle{\pgfqpoint{1.250000in}{0.550000in}}{\pgfqpoint{7.750000in}{3.850000in}}%
\pgfusepath{clip}%
\pgfsetbuttcap%
\pgfsetmiterjoin%
\definecolor{currentfill}{rgb}{0.501961,0.501961,0.501961}%
\pgfsetfillcolor{currentfill}%
\pgfsetfillopacity{0.700000}%
\pgfsetlinewidth{0.000000pt}%
\definecolor{currentstroke}{rgb}{0.000000,0.000000,0.000000}%
\pgfsetstrokecolor{currentstroke}%
\pgfsetstrokeopacity{0.700000}%
\pgfsetdash{}{0pt}%
\pgfpathmoveto{\pgfqpoint{6.363459in}{0.550000in}}%
\pgfpathlineto{\pgfqpoint{6.390980in}{0.550000in}}%
\pgfpathlineto{\pgfqpoint{6.390980in}{0.636449in}}%
\pgfpathlineto{\pgfqpoint{6.363459in}{0.636449in}}%
\pgfpathlineto{\pgfqpoint{6.363459in}{0.550000in}}%
\pgfpathclose%
\pgfusepath{fill}%
\end{pgfscope}%
\begin{pgfscope}%
\pgfpathrectangle{\pgfqpoint{1.250000in}{0.550000in}}{\pgfqpoint{7.750000in}{3.850000in}}%
\pgfusepath{clip}%
\pgfsetbuttcap%
\pgfsetmiterjoin%
\definecolor{currentfill}{rgb}{0.501961,0.501961,0.501961}%
\pgfsetfillcolor{currentfill}%
\pgfsetfillopacity{0.700000}%
\pgfsetlinewidth{0.000000pt}%
\definecolor{currentstroke}{rgb}{0.000000,0.000000,0.000000}%
\pgfsetstrokecolor{currentstroke}%
\pgfsetstrokeopacity{0.700000}%
\pgfsetdash{}{0pt}%
\pgfpathmoveto{\pgfqpoint{6.390980in}{0.550000in}}%
\pgfpathlineto{\pgfqpoint{6.418501in}{0.550000in}}%
\pgfpathlineto{\pgfqpoint{6.418501in}{0.619655in}}%
\pgfpathlineto{\pgfqpoint{6.390980in}{0.619655in}}%
\pgfpathlineto{\pgfqpoint{6.390980in}{0.550000in}}%
\pgfpathclose%
\pgfusepath{fill}%
\end{pgfscope}%
\begin{pgfscope}%
\pgfpathrectangle{\pgfqpoint{1.250000in}{0.550000in}}{\pgfqpoint{7.750000in}{3.850000in}}%
\pgfusepath{clip}%
\pgfsetbuttcap%
\pgfsetmiterjoin%
\definecolor{currentfill}{rgb}{0.501961,0.501961,0.501961}%
\pgfsetfillcolor{currentfill}%
\pgfsetfillopacity{0.700000}%
\pgfsetlinewidth{0.000000pt}%
\definecolor{currentstroke}{rgb}{0.000000,0.000000,0.000000}%
\pgfsetstrokecolor{currentstroke}%
\pgfsetstrokeopacity{0.700000}%
\pgfsetdash{}{0pt}%
\pgfpathmoveto{\pgfqpoint{6.418501in}{0.550000in}}%
\pgfpathlineto{\pgfqpoint{6.446023in}{0.550000in}}%
\pgfpathlineto{\pgfqpoint{6.446023in}{0.624886in}}%
\pgfpathlineto{\pgfqpoint{6.418501in}{0.624886in}}%
\pgfpathlineto{\pgfqpoint{6.418501in}{0.550000in}}%
\pgfpathclose%
\pgfusepath{fill}%
\end{pgfscope}%
\begin{pgfscope}%
\pgfpathrectangle{\pgfqpoint{1.250000in}{0.550000in}}{\pgfqpoint{7.750000in}{3.850000in}}%
\pgfusepath{clip}%
\pgfsetbuttcap%
\pgfsetmiterjoin%
\definecolor{currentfill}{rgb}{0.501961,0.501961,0.501961}%
\pgfsetfillcolor{currentfill}%
\pgfsetfillopacity{0.700000}%
\pgfsetlinewidth{0.000000pt}%
\definecolor{currentstroke}{rgb}{0.000000,0.000000,0.000000}%
\pgfsetstrokecolor{currentstroke}%
\pgfsetstrokeopacity{0.700000}%
\pgfsetdash{}{0pt}%
\pgfpathmoveto{\pgfqpoint{6.446023in}{0.550000in}}%
\pgfpathlineto{\pgfqpoint{6.473544in}{0.550000in}}%
\pgfpathlineto{\pgfqpoint{6.473544in}{0.616627in}}%
\pgfpathlineto{\pgfqpoint{6.446023in}{0.616627in}}%
\pgfpathlineto{\pgfqpoint{6.446023in}{0.550000in}}%
\pgfpathclose%
\pgfusepath{fill}%
\end{pgfscope}%
\begin{pgfscope}%
\pgfpathrectangle{\pgfqpoint{1.250000in}{0.550000in}}{\pgfqpoint{7.750000in}{3.850000in}}%
\pgfusepath{clip}%
\pgfsetbuttcap%
\pgfsetmiterjoin%
\definecolor{currentfill}{rgb}{0.501961,0.501961,0.501961}%
\pgfsetfillcolor{currentfill}%
\pgfsetfillopacity{0.700000}%
\pgfsetlinewidth{0.000000pt}%
\definecolor{currentstroke}{rgb}{0.000000,0.000000,0.000000}%
\pgfsetstrokecolor{currentstroke}%
\pgfsetstrokeopacity{0.700000}%
\pgfsetdash{}{0pt}%
\pgfpathmoveto{\pgfqpoint{6.473544in}{0.550000in}}%
\pgfpathlineto{\pgfqpoint{6.501065in}{0.550000in}}%
\pgfpathlineto{\pgfqpoint{6.501065in}{0.621582in}}%
\pgfpathlineto{\pgfqpoint{6.473544in}{0.621582in}}%
\pgfpathlineto{\pgfqpoint{6.473544in}{0.550000in}}%
\pgfpathclose%
\pgfusepath{fill}%
\end{pgfscope}%
\begin{pgfscope}%
\pgfpathrectangle{\pgfqpoint{1.250000in}{0.550000in}}{\pgfqpoint{7.750000in}{3.850000in}}%
\pgfusepath{clip}%
\pgfsetbuttcap%
\pgfsetmiterjoin%
\definecolor{currentfill}{rgb}{0.501961,0.501961,0.501961}%
\pgfsetfillcolor{currentfill}%
\pgfsetfillopacity{0.700000}%
\pgfsetlinewidth{0.000000pt}%
\definecolor{currentstroke}{rgb}{0.000000,0.000000,0.000000}%
\pgfsetstrokecolor{currentstroke}%
\pgfsetstrokeopacity{0.700000}%
\pgfsetdash{}{0pt}%
\pgfpathmoveto{\pgfqpoint{6.501065in}{0.550000in}}%
\pgfpathlineto{\pgfqpoint{6.528587in}{0.550000in}}%
\pgfpathlineto{\pgfqpoint{6.528587in}{0.608367in}}%
\pgfpathlineto{\pgfqpoint{6.501065in}{0.608367in}}%
\pgfpathlineto{\pgfqpoint{6.501065in}{0.550000in}}%
\pgfpathclose%
\pgfusepath{fill}%
\end{pgfscope}%
\begin{pgfscope}%
\pgfpathrectangle{\pgfqpoint{1.250000in}{0.550000in}}{\pgfqpoint{7.750000in}{3.850000in}}%
\pgfusepath{clip}%
\pgfsetbuttcap%
\pgfsetmiterjoin%
\definecolor{currentfill}{rgb}{0.501961,0.501961,0.501961}%
\pgfsetfillcolor{currentfill}%
\pgfsetfillopacity{0.700000}%
\pgfsetlinewidth{0.000000pt}%
\definecolor{currentstroke}{rgb}{0.000000,0.000000,0.000000}%
\pgfsetstrokecolor{currentstroke}%
\pgfsetstrokeopacity{0.700000}%
\pgfsetdash{}{0pt}%
\pgfpathmoveto{\pgfqpoint{6.528587in}{0.550000in}}%
\pgfpathlineto{\pgfqpoint{6.556108in}{0.550000in}}%
\pgfpathlineto{\pgfqpoint{6.556108in}{0.604788in}}%
\pgfpathlineto{\pgfqpoint{6.528587in}{0.604788in}}%
\pgfpathlineto{\pgfqpoint{6.528587in}{0.550000in}}%
\pgfpathclose%
\pgfusepath{fill}%
\end{pgfscope}%
\begin{pgfscope}%
\pgfpathrectangle{\pgfqpoint{1.250000in}{0.550000in}}{\pgfqpoint{7.750000in}{3.850000in}}%
\pgfusepath{clip}%
\pgfsetbuttcap%
\pgfsetmiterjoin%
\definecolor{currentfill}{rgb}{0.501961,0.501961,0.501961}%
\pgfsetfillcolor{currentfill}%
\pgfsetfillopacity{0.700000}%
\pgfsetlinewidth{0.000000pt}%
\definecolor{currentstroke}{rgb}{0.000000,0.000000,0.000000}%
\pgfsetstrokecolor{currentstroke}%
\pgfsetstrokeopacity{0.700000}%
\pgfsetdash{}{0pt}%
\pgfpathmoveto{\pgfqpoint{6.556108in}{0.550000in}}%
\pgfpathlineto{\pgfqpoint{6.583629in}{0.550000in}}%
\pgfpathlineto{\pgfqpoint{6.583629in}{0.611396in}}%
\pgfpathlineto{\pgfqpoint{6.556108in}{0.611396in}}%
\pgfpathlineto{\pgfqpoint{6.556108in}{0.550000in}}%
\pgfpathclose%
\pgfusepath{fill}%
\end{pgfscope}%
\begin{pgfscope}%
\pgfpathrectangle{\pgfqpoint{1.250000in}{0.550000in}}{\pgfqpoint{7.750000in}{3.850000in}}%
\pgfusepath{clip}%
\pgfsetbuttcap%
\pgfsetmiterjoin%
\definecolor{currentfill}{rgb}{0.501961,0.501961,0.501961}%
\pgfsetfillcolor{currentfill}%
\pgfsetfillopacity{0.700000}%
\pgfsetlinewidth{0.000000pt}%
\definecolor{currentstroke}{rgb}{0.000000,0.000000,0.000000}%
\pgfsetstrokecolor{currentstroke}%
\pgfsetstrokeopacity{0.700000}%
\pgfsetdash{}{0pt}%
\pgfpathmoveto{\pgfqpoint{6.583629in}{0.550000in}}%
\pgfpathlineto{\pgfqpoint{6.611151in}{0.550000in}}%
\pgfpathlineto{\pgfqpoint{6.611151in}{0.598180in}}%
\pgfpathlineto{\pgfqpoint{6.583629in}{0.598180in}}%
\pgfpathlineto{\pgfqpoint{6.583629in}{0.550000in}}%
\pgfpathclose%
\pgfusepath{fill}%
\end{pgfscope}%
\begin{pgfscope}%
\pgfpathrectangle{\pgfqpoint{1.250000in}{0.550000in}}{\pgfqpoint{7.750000in}{3.850000in}}%
\pgfusepath{clip}%
\pgfsetbuttcap%
\pgfsetmiterjoin%
\definecolor{currentfill}{rgb}{0.501961,0.501961,0.501961}%
\pgfsetfillcolor{currentfill}%
\pgfsetfillopacity{0.700000}%
\pgfsetlinewidth{0.000000pt}%
\definecolor{currentstroke}{rgb}{0.000000,0.000000,0.000000}%
\pgfsetstrokecolor{currentstroke}%
\pgfsetstrokeopacity{0.700000}%
\pgfsetdash{}{0pt}%
\pgfpathmoveto{\pgfqpoint{6.611151in}{0.550000in}}%
\pgfpathlineto{\pgfqpoint{6.638672in}{0.550000in}}%
\pgfpathlineto{\pgfqpoint{6.638672in}{0.611396in}}%
\pgfpathlineto{\pgfqpoint{6.611151in}{0.611396in}}%
\pgfpathlineto{\pgfqpoint{6.611151in}{0.550000in}}%
\pgfpathclose%
\pgfusepath{fill}%
\end{pgfscope}%
\begin{pgfscope}%
\pgfpathrectangle{\pgfqpoint{1.250000in}{0.550000in}}{\pgfqpoint{7.750000in}{3.850000in}}%
\pgfusepath{clip}%
\pgfsetbuttcap%
\pgfsetmiterjoin%
\definecolor{currentfill}{rgb}{0.501961,0.501961,0.501961}%
\pgfsetfillcolor{currentfill}%
\pgfsetfillopacity{0.700000}%
\pgfsetlinewidth{0.000000pt}%
\definecolor{currentstroke}{rgb}{0.000000,0.000000,0.000000}%
\pgfsetstrokecolor{currentstroke}%
\pgfsetstrokeopacity{0.700000}%
\pgfsetdash{}{0pt}%
\pgfpathmoveto{\pgfqpoint{6.638672in}{0.550000in}}%
\pgfpathlineto{\pgfqpoint{6.666193in}{0.550000in}}%
\pgfpathlineto{\pgfqpoint{6.666193in}{0.606715in}}%
\pgfpathlineto{\pgfqpoint{6.638672in}{0.606715in}}%
\pgfpathlineto{\pgfqpoint{6.638672in}{0.550000in}}%
\pgfpathclose%
\pgfusepath{fill}%
\end{pgfscope}%
\begin{pgfscope}%
\pgfpathrectangle{\pgfqpoint{1.250000in}{0.550000in}}{\pgfqpoint{7.750000in}{3.850000in}}%
\pgfusepath{clip}%
\pgfsetbuttcap%
\pgfsetmiterjoin%
\definecolor{currentfill}{rgb}{0.501961,0.501961,0.501961}%
\pgfsetfillcolor{currentfill}%
\pgfsetfillopacity{0.700000}%
\pgfsetlinewidth{0.000000pt}%
\definecolor{currentstroke}{rgb}{0.000000,0.000000,0.000000}%
\pgfsetstrokecolor{currentstroke}%
\pgfsetstrokeopacity{0.700000}%
\pgfsetdash{}{0pt}%
\pgfpathmoveto{\pgfqpoint{6.666193in}{0.550000in}}%
\pgfpathlineto{\pgfqpoint{6.693714in}{0.550000in}}%
\pgfpathlineto{\pgfqpoint{6.693714in}{0.609468in}}%
\pgfpathlineto{\pgfqpoint{6.666193in}{0.609468in}}%
\pgfpathlineto{\pgfqpoint{6.666193in}{0.550000in}}%
\pgfpathclose%
\pgfusepath{fill}%
\end{pgfscope}%
\begin{pgfscope}%
\pgfpathrectangle{\pgfqpoint{1.250000in}{0.550000in}}{\pgfqpoint{7.750000in}{3.850000in}}%
\pgfusepath{clip}%
\pgfsetbuttcap%
\pgfsetmiterjoin%
\definecolor{currentfill}{rgb}{0.501961,0.501961,0.501961}%
\pgfsetfillcolor{currentfill}%
\pgfsetfillopacity{0.700000}%
\pgfsetlinewidth{0.000000pt}%
\definecolor{currentstroke}{rgb}{0.000000,0.000000,0.000000}%
\pgfsetstrokecolor{currentstroke}%
\pgfsetstrokeopacity{0.700000}%
\pgfsetdash{}{0pt}%
\pgfpathmoveto{\pgfqpoint{6.693714in}{0.550000in}}%
\pgfpathlineto{\pgfqpoint{6.721236in}{0.550000in}}%
\pgfpathlineto{\pgfqpoint{6.721236in}{0.601760in}}%
\pgfpathlineto{\pgfqpoint{6.693714in}{0.601760in}}%
\pgfpathlineto{\pgfqpoint{6.693714in}{0.550000in}}%
\pgfpathclose%
\pgfusepath{fill}%
\end{pgfscope}%
\begin{pgfscope}%
\pgfpathrectangle{\pgfqpoint{1.250000in}{0.550000in}}{\pgfqpoint{7.750000in}{3.850000in}}%
\pgfusepath{clip}%
\pgfsetbuttcap%
\pgfsetmiterjoin%
\definecolor{currentfill}{rgb}{0.501961,0.501961,0.501961}%
\pgfsetfillcolor{currentfill}%
\pgfsetfillopacity{0.700000}%
\pgfsetlinewidth{0.000000pt}%
\definecolor{currentstroke}{rgb}{0.000000,0.000000,0.000000}%
\pgfsetstrokecolor{currentstroke}%
\pgfsetstrokeopacity{0.700000}%
\pgfsetdash{}{0pt}%
\pgfpathmoveto{\pgfqpoint{6.721236in}{0.550000in}}%
\pgfpathlineto{\pgfqpoint{6.748757in}{0.550000in}}%
\pgfpathlineto{\pgfqpoint{6.748757in}{0.609193in}}%
\pgfpathlineto{\pgfqpoint{6.721236in}{0.609193in}}%
\pgfpathlineto{\pgfqpoint{6.721236in}{0.550000in}}%
\pgfpathclose%
\pgfusepath{fill}%
\end{pgfscope}%
\begin{pgfscope}%
\pgfpathrectangle{\pgfqpoint{1.250000in}{0.550000in}}{\pgfqpoint{7.750000in}{3.850000in}}%
\pgfusepath{clip}%
\pgfsetbuttcap%
\pgfsetmiterjoin%
\definecolor{currentfill}{rgb}{0.501961,0.501961,0.501961}%
\pgfsetfillcolor{currentfill}%
\pgfsetfillopacity{0.700000}%
\pgfsetlinewidth{0.000000pt}%
\definecolor{currentstroke}{rgb}{0.000000,0.000000,0.000000}%
\pgfsetstrokecolor{currentstroke}%
\pgfsetstrokeopacity{0.700000}%
\pgfsetdash{}{0pt}%
\pgfpathmoveto{\pgfqpoint{6.748757in}{0.550000in}}%
\pgfpathlineto{\pgfqpoint{6.776278in}{0.550000in}}%
\pgfpathlineto{\pgfqpoint{6.776278in}{0.603962in}}%
\pgfpathlineto{\pgfqpoint{6.748757in}{0.603962in}}%
\pgfpathlineto{\pgfqpoint{6.748757in}{0.550000in}}%
\pgfpathclose%
\pgfusepath{fill}%
\end{pgfscope}%
\begin{pgfscope}%
\pgfpathrectangle{\pgfqpoint{1.250000in}{0.550000in}}{\pgfqpoint{7.750000in}{3.850000in}}%
\pgfusepath{clip}%
\pgfsetbuttcap%
\pgfsetmiterjoin%
\definecolor{currentfill}{rgb}{0.501961,0.501961,0.501961}%
\pgfsetfillcolor{currentfill}%
\pgfsetfillopacity{0.700000}%
\pgfsetlinewidth{0.000000pt}%
\definecolor{currentstroke}{rgb}{0.000000,0.000000,0.000000}%
\pgfsetstrokecolor{currentstroke}%
\pgfsetstrokeopacity{0.700000}%
\pgfsetdash{}{0pt}%
\pgfpathmoveto{\pgfqpoint{6.776278in}{0.550000in}}%
\pgfpathlineto{\pgfqpoint{6.803800in}{0.550000in}}%
\pgfpathlineto{\pgfqpoint{6.803800in}{0.599832in}}%
\pgfpathlineto{\pgfqpoint{6.776278in}{0.599832in}}%
\pgfpathlineto{\pgfqpoint{6.776278in}{0.550000in}}%
\pgfpathclose%
\pgfusepath{fill}%
\end{pgfscope}%
\begin{pgfscope}%
\pgfpathrectangle{\pgfqpoint{1.250000in}{0.550000in}}{\pgfqpoint{7.750000in}{3.850000in}}%
\pgfusepath{clip}%
\pgfsetbuttcap%
\pgfsetmiterjoin%
\definecolor{currentfill}{rgb}{0.501961,0.501961,0.501961}%
\pgfsetfillcolor{currentfill}%
\pgfsetfillopacity{0.700000}%
\pgfsetlinewidth{0.000000pt}%
\definecolor{currentstroke}{rgb}{0.000000,0.000000,0.000000}%
\pgfsetstrokecolor{currentstroke}%
\pgfsetstrokeopacity{0.700000}%
\pgfsetdash{}{0pt}%
\pgfpathmoveto{\pgfqpoint{6.803800in}{0.550000in}}%
\pgfpathlineto{\pgfqpoint{6.831321in}{0.550000in}}%
\pgfpathlineto{\pgfqpoint{6.831321in}{0.600108in}}%
\pgfpathlineto{\pgfqpoint{6.803800in}{0.600108in}}%
\pgfpathlineto{\pgfqpoint{6.803800in}{0.550000in}}%
\pgfpathclose%
\pgfusepath{fill}%
\end{pgfscope}%
\begin{pgfscope}%
\pgfpathrectangle{\pgfqpoint{1.250000in}{0.550000in}}{\pgfqpoint{7.750000in}{3.850000in}}%
\pgfusepath{clip}%
\pgfsetbuttcap%
\pgfsetmiterjoin%
\definecolor{currentfill}{rgb}{0.501961,0.501961,0.501961}%
\pgfsetfillcolor{currentfill}%
\pgfsetfillopacity{0.700000}%
\pgfsetlinewidth{0.000000pt}%
\definecolor{currentstroke}{rgb}{0.000000,0.000000,0.000000}%
\pgfsetstrokecolor{currentstroke}%
\pgfsetstrokeopacity{0.700000}%
\pgfsetdash{}{0pt}%
\pgfpathmoveto{\pgfqpoint{6.831321in}{0.550000in}}%
\pgfpathlineto{\pgfqpoint{6.858842in}{0.550000in}}%
\pgfpathlineto{\pgfqpoint{6.858842in}{0.599832in}}%
\pgfpathlineto{\pgfqpoint{6.831321in}{0.599832in}}%
\pgfpathlineto{\pgfqpoint{6.831321in}{0.550000in}}%
\pgfpathclose%
\pgfusepath{fill}%
\end{pgfscope}%
\begin{pgfscope}%
\pgfpathrectangle{\pgfqpoint{1.250000in}{0.550000in}}{\pgfqpoint{7.750000in}{3.850000in}}%
\pgfusepath{clip}%
\pgfsetbuttcap%
\pgfsetmiterjoin%
\definecolor{currentfill}{rgb}{0.501961,0.501961,0.501961}%
\pgfsetfillcolor{currentfill}%
\pgfsetfillopacity{0.700000}%
\pgfsetlinewidth{0.000000pt}%
\definecolor{currentstroke}{rgb}{0.000000,0.000000,0.000000}%
\pgfsetstrokecolor{currentstroke}%
\pgfsetstrokeopacity{0.700000}%
\pgfsetdash{}{0pt}%
\pgfpathmoveto{\pgfqpoint{6.858842in}{0.550000in}}%
\pgfpathlineto{\pgfqpoint{6.886364in}{0.550000in}}%
\pgfpathlineto{\pgfqpoint{6.886364in}{0.593225in}}%
\pgfpathlineto{\pgfqpoint{6.858842in}{0.593225in}}%
\pgfpathlineto{\pgfqpoint{6.858842in}{0.550000in}}%
\pgfpathclose%
\pgfusepath{fill}%
\end{pgfscope}%
\begin{pgfscope}%
\pgfpathrectangle{\pgfqpoint{1.250000in}{0.550000in}}{\pgfqpoint{7.750000in}{3.850000in}}%
\pgfusepath{clip}%
\pgfsetbuttcap%
\pgfsetmiterjoin%
\definecolor{currentfill}{rgb}{0.501961,0.501961,0.501961}%
\pgfsetfillcolor{currentfill}%
\pgfsetfillopacity{0.700000}%
\pgfsetlinewidth{0.000000pt}%
\definecolor{currentstroke}{rgb}{0.000000,0.000000,0.000000}%
\pgfsetstrokecolor{currentstroke}%
\pgfsetstrokeopacity{0.700000}%
\pgfsetdash{}{0pt}%
\pgfpathmoveto{\pgfqpoint{6.886364in}{0.550000in}}%
\pgfpathlineto{\pgfqpoint{6.913885in}{0.550000in}}%
\pgfpathlineto{\pgfqpoint{6.913885in}{0.600108in}}%
\pgfpathlineto{\pgfqpoint{6.886364in}{0.600108in}}%
\pgfpathlineto{\pgfqpoint{6.886364in}{0.550000in}}%
\pgfpathclose%
\pgfusepath{fill}%
\end{pgfscope}%
\begin{pgfscope}%
\pgfpathrectangle{\pgfqpoint{1.250000in}{0.550000in}}{\pgfqpoint{7.750000in}{3.850000in}}%
\pgfusepath{clip}%
\pgfsetbuttcap%
\pgfsetmiterjoin%
\definecolor{currentfill}{rgb}{0.501961,0.501961,0.501961}%
\pgfsetfillcolor{currentfill}%
\pgfsetfillopacity{0.700000}%
\pgfsetlinewidth{0.000000pt}%
\definecolor{currentstroke}{rgb}{0.000000,0.000000,0.000000}%
\pgfsetstrokecolor{currentstroke}%
\pgfsetstrokeopacity{0.700000}%
\pgfsetdash{}{0pt}%
\pgfpathmoveto{\pgfqpoint{6.913885in}{0.550000in}}%
\pgfpathlineto{\pgfqpoint{6.941406in}{0.550000in}}%
\pgfpathlineto{\pgfqpoint{6.941406in}{0.595978in}}%
\pgfpathlineto{\pgfqpoint{6.913885in}{0.595978in}}%
\pgfpathlineto{\pgfqpoint{6.913885in}{0.550000in}}%
\pgfpathclose%
\pgfusepath{fill}%
\end{pgfscope}%
\begin{pgfscope}%
\pgfpathrectangle{\pgfqpoint{1.250000in}{0.550000in}}{\pgfqpoint{7.750000in}{3.850000in}}%
\pgfusepath{clip}%
\pgfsetbuttcap%
\pgfsetmiterjoin%
\definecolor{currentfill}{rgb}{0.501961,0.501961,0.501961}%
\pgfsetfillcolor{currentfill}%
\pgfsetfillopacity{0.700000}%
\pgfsetlinewidth{0.000000pt}%
\definecolor{currentstroke}{rgb}{0.000000,0.000000,0.000000}%
\pgfsetstrokecolor{currentstroke}%
\pgfsetstrokeopacity{0.700000}%
\pgfsetdash{}{0pt}%
\pgfpathmoveto{\pgfqpoint{6.941406in}{0.550000in}}%
\pgfpathlineto{\pgfqpoint{6.968928in}{0.550000in}}%
\pgfpathlineto{\pgfqpoint{6.968928in}{0.591848in}}%
\pgfpathlineto{\pgfqpoint{6.941406in}{0.591848in}}%
\pgfpathlineto{\pgfqpoint{6.941406in}{0.550000in}}%
\pgfpathclose%
\pgfusepath{fill}%
\end{pgfscope}%
\begin{pgfscope}%
\pgfpathrectangle{\pgfqpoint{1.250000in}{0.550000in}}{\pgfqpoint{7.750000in}{3.850000in}}%
\pgfusepath{clip}%
\pgfsetbuttcap%
\pgfsetmiterjoin%
\definecolor{currentfill}{rgb}{0.501961,0.501961,0.501961}%
\pgfsetfillcolor{currentfill}%
\pgfsetfillopacity{0.700000}%
\pgfsetlinewidth{0.000000pt}%
\definecolor{currentstroke}{rgb}{0.000000,0.000000,0.000000}%
\pgfsetstrokecolor{currentstroke}%
\pgfsetstrokeopacity{0.700000}%
\pgfsetdash{}{0pt}%
\pgfpathmoveto{\pgfqpoint{6.968928in}{0.550000in}}%
\pgfpathlineto{\pgfqpoint{6.996449in}{0.550000in}}%
\pgfpathlineto{\pgfqpoint{6.996449in}{0.588269in}}%
\pgfpathlineto{\pgfqpoint{6.968928in}{0.588269in}}%
\pgfpathlineto{\pgfqpoint{6.968928in}{0.550000in}}%
\pgfpathclose%
\pgfusepath{fill}%
\end{pgfscope}%
\begin{pgfscope}%
\pgfpathrectangle{\pgfqpoint{1.250000in}{0.550000in}}{\pgfqpoint{7.750000in}{3.850000in}}%
\pgfusepath{clip}%
\pgfsetbuttcap%
\pgfsetmiterjoin%
\definecolor{currentfill}{rgb}{0.501961,0.501961,0.501961}%
\pgfsetfillcolor{currentfill}%
\pgfsetfillopacity{0.700000}%
\pgfsetlinewidth{0.000000pt}%
\definecolor{currentstroke}{rgb}{0.000000,0.000000,0.000000}%
\pgfsetstrokecolor{currentstroke}%
\pgfsetstrokeopacity{0.700000}%
\pgfsetdash{}{0pt}%
\pgfpathmoveto{\pgfqpoint{6.996449in}{0.550000in}}%
\pgfpathlineto{\pgfqpoint{7.023970in}{0.550000in}}%
\pgfpathlineto{\pgfqpoint{7.023970in}{0.600108in}}%
\pgfpathlineto{\pgfqpoint{6.996449in}{0.600108in}}%
\pgfpathlineto{\pgfqpoint{6.996449in}{0.550000in}}%
\pgfpathclose%
\pgfusepath{fill}%
\end{pgfscope}%
\begin{pgfscope}%
\pgfpathrectangle{\pgfqpoint{1.250000in}{0.550000in}}{\pgfqpoint{7.750000in}{3.850000in}}%
\pgfusepath{clip}%
\pgfsetbuttcap%
\pgfsetmiterjoin%
\definecolor{currentfill}{rgb}{0.501961,0.501961,0.501961}%
\pgfsetfillcolor{currentfill}%
\pgfsetfillopacity{0.700000}%
\pgfsetlinewidth{0.000000pt}%
\definecolor{currentstroke}{rgb}{0.000000,0.000000,0.000000}%
\pgfsetstrokecolor{currentstroke}%
\pgfsetstrokeopacity{0.700000}%
\pgfsetdash{}{0pt}%
\pgfpathmoveto{\pgfqpoint{7.023970in}{0.550000in}}%
\pgfpathlineto{\pgfqpoint{7.051491in}{0.550000in}}%
\pgfpathlineto{\pgfqpoint{7.051491in}{0.589646in}}%
\pgfpathlineto{\pgfqpoint{7.023970in}{0.589646in}}%
\pgfpathlineto{\pgfqpoint{7.023970in}{0.550000in}}%
\pgfpathclose%
\pgfusepath{fill}%
\end{pgfscope}%
\begin{pgfscope}%
\pgfpathrectangle{\pgfqpoint{1.250000in}{0.550000in}}{\pgfqpoint{7.750000in}{3.850000in}}%
\pgfusepath{clip}%
\pgfsetbuttcap%
\pgfsetmiterjoin%
\definecolor{currentfill}{rgb}{0.501961,0.501961,0.501961}%
\pgfsetfillcolor{currentfill}%
\pgfsetfillopacity{0.700000}%
\pgfsetlinewidth{0.000000pt}%
\definecolor{currentstroke}{rgb}{0.000000,0.000000,0.000000}%
\pgfsetstrokecolor{currentstroke}%
\pgfsetstrokeopacity{0.700000}%
\pgfsetdash{}{0pt}%
\pgfpathmoveto{\pgfqpoint{7.051491in}{0.550000in}}%
\pgfpathlineto{\pgfqpoint{7.079013in}{0.550000in}}%
\pgfpathlineto{\pgfqpoint{7.079013in}{0.597630in}}%
\pgfpathlineto{\pgfqpoint{7.051491in}{0.597630in}}%
\pgfpathlineto{\pgfqpoint{7.051491in}{0.550000in}}%
\pgfpathclose%
\pgfusepath{fill}%
\end{pgfscope}%
\begin{pgfscope}%
\pgfpathrectangle{\pgfqpoint{1.250000in}{0.550000in}}{\pgfqpoint{7.750000in}{3.850000in}}%
\pgfusepath{clip}%
\pgfsetbuttcap%
\pgfsetmiterjoin%
\definecolor{currentfill}{rgb}{0.501961,0.501961,0.501961}%
\pgfsetfillcolor{currentfill}%
\pgfsetfillopacity{0.700000}%
\pgfsetlinewidth{0.000000pt}%
\definecolor{currentstroke}{rgb}{0.000000,0.000000,0.000000}%
\pgfsetstrokecolor{currentstroke}%
\pgfsetstrokeopacity{0.700000}%
\pgfsetdash{}{0pt}%
\pgfpathmoveto{\pgfqpoint{7.079013in}{0.550000in}}%
\pgfpathlineto{\pgfqpoint{7.106534in}{0.550000in}}%
\pgfpathlineto{\pgfqpoint{7.106534in}{0.591573in}}%
\pgfpathlineto{\pgfqpoint{7.079013in}{0.591573in}}%
\pgfpathlineto{\pgfqpoint{7.079013in}{0.550000in}}%
\pgfpathclose%
\pgfusepath{fill}%
\end{pgfscope}%
\begin{pgfscope}%
\pgfpathrectangle{\pgfqpoint{1.250000in}{0.550000in}}{\pgfqpoint{7.750000in}{3.850000in}}%
\pgfusepath{clip}%
\pgfsetbuttcap%
\pgfsetmiterjoin%
\definecolor{currentfill}{rgb}{0.501961,0.501961,0.501961}%
\pgfsetfillcolor{currentfill}%
\pgfsetfillopacity{0.700000}%
\pgfsetlinewidth{0.000000pt}%
\definecolor{currentstroke}{rgb}{0.000000,0.000000,0.000000}%
\pgfsetstrokecolor{currentstroke}%
\pgfsetstrokeopacity{0.700000}%
\pgfsetdash{}{0pt}%
\pgfpathmoveto{\pgfqpoint{7.106534in}{0.550000in}}%
\pgfpathlineto{\pgfqpoint{7.134055in}{0.550000in}}%
\pgfpathlineto{\pgfqpoint{7.134055in}{0.592399in}}%
\pgfpathlineto{\pgfqpoint{7.106534in}{0.592399in}}%
\pgfpathlineto{\pgfqpoint{7.106534in}{0.550000in}}%
\pgfpathclose%
\pgfusepath{fill}%
\end{pgfscope}%
\begin{pgfscope}%
\pgfpathrectangle{\pgfqpoint{1.250000in}{0.550000in}}{\pgfqpoint{7.750000in}{3.850000in}}%
\pgfusepath{clip}%
\pgfsetbuttcap%
\pgfsetmiterjoin%
\definecolor{currentfill}{rgb}{0.501961,0.501961,0.501961}%
\pgfsetfillcolor{currentfill}%
\pgfsetfillopacity{0.700000}%
\pgfsetlinewidth{0.000000pt}%
\definecolor{currentstroke}{rgb}{0.000000,0.000000,0.000000}%
\pgfsetstrokecolor{currentstroke}%
\pgfsetstrokeopacity{0.700000}%
\pgfsetdash{}{0pt}%
\pgfpathmoveto{\pgfqpoint{7.134055in}{0.550000in}}%
\pgfpathlineto{\pgfqpoint{7.161577in}{0.550000in}}%
\pgfpathlineto{\pgfqpoint{7.161577in}{0.598731in}}%
\pgfpathlineto{\pgfqpoint{7.134055in}{0.598731in}}%
\pgfpathlineto{\pgfqpoint{7.134055in}{0.550000in}}%
\pgfpathclose%
\pgfusepath{fill}%
\end{pgfscope}%
\begin{pgfscope}%
\pgfpathrectangle{\pgfqpoint{1.250000in}{0.550000in}}{\pgfqpoint{7.750000in}{3.850000in}}%
\pgfusepath{clip}%
\pgfsetbuttcap%
\pgfsetmiterjoin%
\definecolor{currentfill}{rgb}{0.501961,0.501961,0.501961}%
\pgfsetfillcolor{currentfill}%
\pgfsetfillopacity{0.700000}%
\pgfsetlinewidth{0.000000pt}%
\definecolor{currentstroke}{rgb}{0.000000,0.000000,0.000000}%
\pgfsetstrokecolor{currentstroke}%
\pgfsetstrokeopacity{0.700000}%
\pgfsetdash{}{0pt}%
\pgfpathmoveto{\pgfqpoint{7.161577in}{0.550000in}}%
\pgfpathlineto{\pgfqpoint{7.189098in}{0.550000in}}%
\pgfpathlineto{\pgfqpoint{7.189098in}{0.584965in}}%
\pgfpathlineto{\pgfqpoint{7.161577in}{0.584965in}}%
\pgfpathlineto{\pgfqpoint{7.161577in}{0.550000in}}%
\pgfpathclose%
\pgfusepath{fill}%
\end{pgfscope}%
\begin{pgfscope}%
\pgfpathrectangle{\pgfqpoint{1.250000in}{0.550000in}}{\pgfqpoint{7.750000in}{3.850000in}}%
\pgfusepath{clip}%
\pgfsetbuttcap%
\pgfsetmiterjoin%
\definecolor{currentfill}{rgb}{0.501961,0.501961,0.501961}%
\pgfsetfillcolor{currentfill}%
\pgfsetfillopacity{0.700000}%
\pgfsetlinewidth{0.000000pt}%
\definecolor{currentstroke}{rgb}{0.000000,0.000000,0.000000}%
\pgfsetstrokecolor{currentstroke}%
\pgfsetstrokeopacity{0.700000}%
\pgfsetdash{}{0pt}%
\pgfpathmoveto{\pgfqpoint{7.189098in}{0.550000in}}%
\pgfpathlineto{\pgfqpoint{7.216619in}{0.550000in}}%
\pgfpathlineto{\pgfqpoint{7.216619in}{0.594326in}}%
\pgfpathlineto{\pgfqpoint{7.189098in}{0.594326in}}%
\pgfpathlineto{\pgfqpoint{7.189098in}{0.550000in}}%
\pgfpathclose%
\pgfusepath{fill}%
\end{pgfscope}%
\begin{pgfscope}%
\pgfpathrectangle{\pgfqpoint{1.250000in}{0.550000in}}{\pgfqpoint{7.750000in}{3.850000in}}%
\pgfusepath{clip}%
\pgfsetbuttcap%
\pgfsetmiterjoin%
\definecolor{currentfill}{rgb}{0.501961,0.501961,0.501961}%
\pgfsetfillcolor{currentfill}%
\pgfsetfillopacity{0.700000}%
\pgfsetlinewidth{0.000000pt}%
\definecolor{currentstroke}{rgb}{0.000000,0.000000,0.000000}%
\pgfsetstrokecolor{currentstroke}%
\pgfsetstrokeopacity{0.700000}%
\pgfsetdash{}{0pt}%
\pgfpathmoveto{\pgfqpoint{7.216619in}{0.550000in}}%
\pgfpathlineto{\pgfqpoint{7.244141in}{0.550000in}}%
\pgfpathlineto{\pgfqpoint{7.244141in}{0.590747in}}%
\pgfpathlineto{\pgfqpoint{7.216619in}{0.590747in}}%
\pgfpathlineto{\pgfqpoint{7.216619in}{0.550000in}}%
\pgfpathclose%
\pgfusepath{fill}%
\end{pgfscope}%
\begin{pgfscope}%
\pgfpathrectangle{\pgfqpoint{1.250000in}{0.550000in}}{\pgfqpoint{7.750000in}{3.850000in}}%
\pgfusepath{clip}%
\pgfsetbuttcap%
\pgfsetmiterjoin%
\definecolor{currentfill}{rgb}{0.501961,0.501961,0.501961}%
\pgfsetfillcolor{currentfill}%
\pgfsetfillopacity{0.700000}%
\pgfsetlinewidth{0.000000pt}%
\definecolor{currentstroke}{rgb}{0.000000,0.000000,0.000000}%
\pgfsetstrokecolor{currentstroke}%
\pgfsetstrokeopacity{0.700000}%
\pgfsetdash{}{0pt}%
\pgfpathmoveto{\pgfqpoint{7.244141in}{0.550000in}}%
\pgfpathlineto{\pgfqpoint{7.271662in}{0.550000in}}%
\pgfpathlineto{\pgfqpoint{7.271662in}{0.591573in}}%
\pgfpathlineto{\pgfqpoint{7.244141in}{0.591573in}}%
\pgfpathlineto{\pgfqpoint{7.244141in}{0.550000in}}%
\pgfpathclose%
\pgfusepath{fill}%
\end{pgfscope}%
\begin{pgfscope}%
\pgfpathrectangle{\pgfqpoint{1.250000in}{0.550000in}}{\pgfqpoint{7.750000in}{3.850000in}}%
\pgfusepath{clip}%
\pgfsetbuttcap%
\pgfsetmiterjoin%
\definecolor{currentfill}{rgb}{0.501961,0.501961,0.501961}%
\pgfsetfillcolor{currentfill}%
\pgfsetfillopacity{0.700000}%
\pgfsetlinewidth{0.000000pt}%
\definecolor{currentstroke}{rgb}{0.000000,0.000000,0.000000}%
\pgfsetstrokecolor{currentstroke}%
\pgfsetstrokeopacity{0.700000}%
\pgfsetdash{}{0pt}%
\pgfpathmoveto{\pgfqpoint{7.271662in}{0.550000in}}%
\pgfpathlineto{\pgfqpoint{7.299183in}{0.550000in}}%
\pgfpathlineto{\pgfqpoint{7.299183in}{0.593775in}}%
\pgfpathlineto{\pgfqpoint{7.271662in}{0.593775in}}%
\pgfpathlineto{\pgfqpoint{7.271662in}{0.550000in}}%
\pgfpathclose%
\pgfusepath{fill}%
\end{pgfscope}%
\begin{pgfscope}%
\pgfpathrectangle{\pgfqpoint{1.250000in}{0.550000in}}{\pgfqpoint{7.750000in}{3.850000in}}%
\pgfusepath{clip}%
\pgfsetbuttcap%
\pgfsetmiterjoin%
\definecolor{currentfill}{rgb}{0.501961,0.501961,0.501961}%
\pgfsetfillcolor{currentfill}%
\pgfsetfillopacity{0.700000}%
\pgfsetlinewidth{0.000000pt}%
\definecolor{currentstroke}{rgb}{0.000000,0.000000,0.000000}%
\pgfsetstrokecolor{currentstroke}%
\pgfsetstrokeopacity{0.700000}%
\pgfsetdash{}{0pt}%
\pgfpathmoveto{\pgfqpoint{7.299183in}{0.550000in}}%
\pgfpathlineto{\pgfqpoint{7.326705in}{0.550000in}}%
\pgfpathlineto{\pgfqpoint{7.326705in}{0.588820in}}%
\pgfpathlineto{\pgfqpoint{7.299183in}{0.588820in}}%
\pgfpathlineto{\pgfqpoint{7.299183in}{0.550000in}}%
\pgfpathclose%
\pgfusepath{fill}%
\end{pgfscope}%
\begin{pgfscope}%
\pgfpathrectangle{\pgfqpoint{1.250000in}{0.550000in}}{\pgfqpoint{7.750000in}{3.850000in}}%
\pgfusepath{clip}%
\pgfsetbuttcap%
\pgfsetmiterjoin%
\definecolor{currentfill}{rgb}{0.501961,0.501961,0.501961}%
\pgfsetfillcolor{currentfill}%
\pgfsetfillopacity{0.700000}%
\pgfsetlinewidth{0.000000pt}%
\definecolor{currentstroke}{rgb}{0.000000,0.000000,0.000000}%
\pgfsetstrokecolor{currentstroke}%
\pgfsetstrokeopacity{0.700000}%
\pgfsetdash{}{0pt}%
\pgfpathmoveto{\pgfqpoint{7.326705in}{0.550000in}}%
\pgfpathlineto{\pgfqpoint{7.354226in}{0.550000in}}%
\pgfpathlineto{\pgfqpoint{7.354226in}{0.587443in}}%
\pgfpathlineto{\pgfqpoint{7.326705in}{0.587443in}}%
\pgfpathlineto{\pgfqpoint{7.326705in}{0.550000in}}%
\pgfpathclose%
\pgfusepath{fill}%
\end{pgfscope}%
\begin{pgfscope}%
\pgfpathrectangle{\pgfqpoint{1.250000in}{0.550000in}}{\pgfqpoint{7.750000in}{3.850000in}}%
\pgfusepath{clip}%
\pgfsetbuttcap%
\pgfsetmiterjoin%
\definecolor{currentfill}{rgb}{0.501961,0.501961,0.501961}%
\pgfsetfillcolor{currentfill}%
\pgfsetfillopacity{0.700000}%
\pgfsetlinewidth{0.000000pt}%
\definecolor{currentstroke}{rgb}{0.000000,0.000000,0.000000}%
\pgfsetstrokecolor{currentstroke}%
\pgfsetstrokeopacity{0.700000}%
\pgfsetdash{}{0pt}%
\pgfpathmoveto{\pgfqpoint{7.354226in}{0.550000in}}%
\pgfpathlineto{\pgfqpoint{7.381747in}{0.550000in}}%
\pgfpathlineto{\pgfqpoint{7.381747in}{0.583589in}}%
\pgfpathlineto{\pgfqpoint{7.354226in}{0.583589in}}%
\pgfpathlineto{\pgfqpoint{7.354226in}{0.550000in}}%
\pgfpathclose%
\pgfusepath{fill}%
\end{pgfscope}%
\begin{pgfscope}%
\pgfpathrectangle{\pgfqpoint{1.250000in}{0.550000in}}{\pgfqpoint{7.750000in}{3.850000in}}%
\pgfusepath{clip}%
\pgfsetbuttcap%
\pgfsetmiterjoin%
\definecolor{currentfill}{rgb}{0.501961,0.501961,0.501961}%
\pgfsetfillcolor{currentfill}%
\pgfsetfillopacity{0.700000}%
\pgfsetlinewidth{0.000000pt}%
\definecolor{currentstroke}{rgb}{0.000000,0.000000,0.000000}%
\pgfsetstrokecolor{currentstroke}%
\pgfsetstrokeopacity{0.700000}%
\pgfsetdash{}{0pt}%
\pgfpathmoveto{\pgfqpoint{7.381747in}{0.550000in}}%
\pgfpathlineto{\pgfqpoint{7.409268in}{0.550000in}}%
\pgfpathlineto{\pgfqpoint{7.409268in}{0.588269in}}%
\pgfpathlineto{\pgfqpoint{7.381747in}{0.588269in}}%
\pgfpathlineto{\pgfqpoint{7.381747in}{0.550000in}}%
\pgfpathclose%
\pgfusepath{fill}%
\end{pgfscope}%
\begin{pgfscope}%
\pgfpathrectangle{\pgfqpoint{1.250000in}{0.550000in}}{\pgfqpoint{7.750000in}{3.850000in}}%
\pgfusepath{clip}%
\pgfsetbuttcap%
\pgfsetmiterjoin%
\definecolor{currentfill}{rgb}{0.501961,0.501961,0.501961}%
\pgfsetfillcolor{currentfill}%
\pgfsetfillopacity{0.700000}%
\pgfsetlinewidth{0.000000pt}%
\definecolor{currentstroke}{rgb}{0.000000,0.000000,0.000000}%
\pgfsetstrokecolor{currentstroke}%
\pgfsetstrokeopacity{0.700000}%
\pgfsetdash{}{0pt}%
\pgfpathmoveto{\pgfqpoint{7.409268in}{0.550000in}}%
\pgfpathlineto{\pgfqpoint{7.436790in}{0.550000in}}%
\pgfpathlineto{\pgfqpoint{7.436790in}{0.584965in}}%
\pgfpathlineto{\pgfqpoint{7.409268in}{0.584965in}}%
\pgfpathlineto{\pgfqpoint{7.409268in}{0.550000in}}%
\pgfpathclose%
\pgfusepath{fill}%
\end{pgfscope}%
\begin{pgfscope}%
\pgfpathrectangle{\pgfqpoint{1.250000in}{0.550000in}}{\pgfqpoint{7.750000in}{3.850000in}}%
\pgfusepath{clip}%
\pgfsetbuttcap%
\pgfsetmiterjoin%
\definecolor{currentfill}{rgb}{0.501961,0.501961,0.501961}%
\pgfsetfillcolor{currentfill}%
\pgfsetfillopacity{0.700000}%
\pgfsetlinewidth{0.000000pt}%
\definecolor{currentstroke}{rgb}{0.000000,0.000000,0.000000}%
\pgfsetstrokecolor{currentstroke}%
\pgfsetstrokeopacity{0.700000}%
\pgfsetdash{}{0pt}%
\pgfpathmoveto{\pgfqpoint{7.436790in}{0.550000in}}%
\pgfpathlineto{\pgfqpoint{7.464311in}{0.550000in}}%
\pgfpathlineto{\pgfqpoint{7.464311in}{0.587443in}}%
\pgfpathlineto{\pgfqpoint{7.436790in}{0.587443in}}%
\pgfpathlineto{\pgfqpoint{7.436790in}{0.550000in}}%
\pgfpathclose%
\pgfusepath{fill}%
\end{pgfscope}%
\begin{pgfscope}%
\pgfpathrectangle{\pgfqpoint{1.250000in}{0.550000in}}{\pgfqpoint{7.750000in}{3.850000in}}%
\pgfusepath{clip}%
\pgfsetbuttcap%
\pgfsetmiterjoin%
\definecolor{currentfill}{rgb}{0.501961,0.501961,0.501961}%
\pgfsetfillcolor{currentfill}%
\pgfsetfillopacity{0.700000}%
\pgfsetlinewidth{0.000000pt}%
\definecolor{currentstroke}{rgb}{0.000000,0.000000,0.000000}%
\pgfsetstrokecolor{currentstroke}%
\pgfsetstrokeopacity{0.700000}%
\pgfsetdash{}{0pt}%
\pgfpathmoveto{\pgfqpoint{7.464311in}{0.550000in}}%
\pgfpathlineto{\pgfqpoint{7.491832in}{0.550000in}}%
\pgfpathlineto{\pgfqpoint{7.491832in}{0.586892in}}%
\pgfpathlineto{\pgfqpoint{7.464311in}{0.586892in}}%
\pgfpathlineto{\pgfqpoint{7.464311in}{0.550000in}}%
\pgfpathclose%
\pgfusepath{fill}%
\end{pgfscope}%
\begin{pgfscope}%
\pgfpathrectangle{\pgfqpoint{1.250000in}{0.550000in}}{\pgfqpoint{7.750000in}{3.850000in}}%
\pgfusepath{clip}%
\pgfsetbuttcap%
\pgfsetmiterjoin%
\definecolor{currentfill}{rgb}{0.501961,0.501961,0.501961}%
\pgfsetfillcolor{currentfill}%
\pgfsetfillopacity{0.700000}%
\pgfsetlinewidth{0.000000pt}%
\definecolor{currentstroke}{rgb}{0.000000,0.000000,0.000000}%
\pgfsetstrokecolor{currentstroke}%
\pgfsetstrokeopacity{0.700000}%
\pgfsetdash{}{0pt}%
\pgfpathmoveto{\pgfqpoint{7.491832in}{0.550000in}}%
\pgfpathlineto{\pgfqpoint{7.519354in}{0.550000in}}%
\pgfpathlineto{\pgfqpoint{7.519354in}{0.584415in}}%
\pgfpathlineto{\pgfqpoint{7.491832in}{0.584415in}}%
\pgfpathlineto{\pgfqpoint{7.491832in}{0.550000in}}%
\pgfpathclose%
\pgfusepath{fill}%
\end{pgfscope}%
\begin{pgfscope}%
\pgfpathrectangle{\pgfqpoint{1.250000in}{0.550000in}}{\pgfqpoint{7.750000in}{3.850000in}}%
\pgfusepath{clip}%
\pgfsetbuttcap%
\pgfsetmiterjoin%
\definecolor{currentfill}{rgb}{0.501961,0.501961,0.501961}%
\pgfsetfillcolor{currentfill}%
\pgfsetfillopacity{0.700000}%
\pgfsetlinewidth{0.000000pt}%
\definecolor{currentstroke}{rgb}{0.000000,0.000000,0.000000}%
\pgfsetstrokecolor{currentstroke}%
\pgfsetstrokeopacity{0.700000}%
\pgfsetdash{}{0pt}%
\pgfpathmoveto{\pgfqpoint{7.519354in}{0.550000in}}%
\pgfpathlineto{\pgfqpoint{7.546875in}{0.550000in}}%
\pgfpathlineto{\pgfqpoint{7.546875in}{0.578082in}}%
\pgfpathlineto{\pgfqpoint{7.519354in}{0.578082in}}%
\pgfpathlineto{\pgfqpoint{7.519354in}{0.550000in}}%
\pgfpathclose%
\pgfusepath{fill}%
\end{pgfscope}%
\begin{pgfscope}%
\pgfpathrectangle{\pgfqpoint{1.250000in}{0.550000in}}{\pgfqpoint{7.750000in}{3.850000in}}%
\pgfusepath{clip}%
\pgfsetbuttcap%
\pgfsetmiterjoin%
\definecolor{currentfill}{rgb}{0.501961,0.501961,0.501961}%
\pgfsetfillcolor{currentfill}%
\pgfsetfillopacity{0.700000}%
\pgfsetlinewidth{0.000000pt}%
\definecolor{currentstroke}{rgb}{0.000000,0.000000,0.000000}%
\pgfsetstrokecolor{currentstroke}%
\pgfsetstrokeopacity{0.700000}%
\pgfsetdash{}{0pt}%
\pgfpathmoveto{\pgfqpoint{7.546875in}{0.550000in}}%
\pgfpathlineto{\pgfqpoint{7.574396in}{0.550000in}}%
\pgfpathlineto{\pgfqpoint{7.574396in}{0.581111in}}%
\pgfpathlineto{\pgfqpoint{7.546875in}{0.581111in}}%
\pgfpathlineto{\pgfqpoint{7.546875in}{0.550000in}}%
\pgfpathclose%
\pgfusepath{fill}%
\end{pgfscope}%
\begin{pgfscope}%
\pgfpathrectangle{\pgfqpoint{1.250000in}{0.550000in}}{\pgfqpoint{7.750000in}{3.850000in}}%
\pgfusepath{clip}%
\pgfsetbuttcap%
\pgfsetmiterjoin%
\definecolor{currentfill}{rgb}{0.501961,0.501961,0.501961}%
\pgfsetfillcolor{currentfill}%
\pgfsetfillopacity{0.700000}%
\pgfsetlinewidth{0.000000pt}%
\definecolor{currentstroke}{rgb}{0.000000,0.000000,0.000000}%
\pgfsetstrokecolor{currentstroke}%
\pgfsetstrokeopacity{0.700000}%
\pgfsetdash{}{0pt}%
\pgfpathmoveto{\pgfqpoint{7.574396in}{0.550000in}}%
\pgfpathlineto{\pgfqpoint{7.601918in}{0.550000in}}%
\pgfpathlineto{\pgfqpoint{7.601918in}{0.583313in}}%
\pgfpathlineto{\pgfqpoint{7.574396in}{0.583313in}}%
\pgfpathlineto{\pgfqpoint{7.574396in}{0.550000in}}%
\pgfpathclose%
\pgfusepath{fill}%
\end{pgfscope}%
\begin{pgfscope}%
\pgfpathrectangle{\pgfqpoint{1.250000in}{0.550000in}}{\pgfqpoint{7.750000in}{3.850000in}}%
\pgfusepath{clip}%
\pgfsetbuttcap%
\pgfsetmiterjoin%
\definecolor{currentfill}{rgb}{0.501961,0.501961,0.501961}%
\pgfsetfillcolor{currentfill}%
\pgfsetfillopacity{0.700000}%
\pgfsetlinewidth{0.000000pt}%
\definecolor{currentstroke}{rgb}{0.000000,0.000000,0.000000}%
\pgfsetstrokecolor{currentstroke}%
\pgfsetstrokeopacity{0.700000}%
\pgfsetdash{}{0pt}%
\pgfpathmoveto{\pgfqpoint{7.601918in}{0.550000in}}%
\pgfpathlineto{\pgfqpoint{7.629439in}{0.550000in}}%
\pgfpathlineto{\pgfqpoint{7.629439in}{0.583038in}}%
\pgfpathlineto{\pgfqpoint{7.601918in}{0.583038in}}%
\pgfpathlineto{\pgfqpoint{7.601918in}{0.550000in}}%
\pgfpathclose%
\pgfusepath{fill}%
\end{pgfscope}%
\begin{pgfscope}%
\pgfpathrectangle{\pgfqpoint{1.250000in}{0.550000in}}{\pgfqpoint{7.750000in}{3.850000in}}%
\pgfusepath{clip}%
\pgfsetbuttcap%
\pgfsetmiterjoin%
\definecolor{currentfill}{rgb}{0.501961,0.501961,0.501961}%
\pgfsetfillcolor{currentfill}%
\pgfsetfillopacity{0.700000}%
\pgfsetlinewidth{0.000000pt}%
\definecolor{currentstroke}{rgb}{0.000000,0.000000,0.000000}%
\pgfsetstrokecolor{currentstroke}%
\pgfsetstrokeopacity{0.700000}%
\pgfsetdash{}{0pt}%
\pgfpathmoveto{\pgfqpoint{7.629439in}{0.550000in}}%
\pgfpathlineto{\pgfqpoint{7.656960in}{0.550000in}}%
\pgfpathlineto{\pgfqpoint{7.656960in}{0.580560in}}%
\pgfpathlineto{\pgfqpoint{7.629439in}{0.580560in}}%
\pgfpathlineto{\pgfqpoint{7.629439in}{0.550000in}}%
\pgfpathclose%
\pgfusepath{fill}%
\end{pgfscope}%
\begin{pgfscope}%
\pgfpathrectangle{\pgfqpoint{1.250000in}{0.550000in}}{\pgfqpoint{7.750000in}{3.850000in}}%
\pgfusepath{clip}%
\pgfsetbuttcap%
\pgfsetmiterjoin%
\definecolor{currentfill}{rgb}{0.501961,0.501961,0.501961}%
\pgfsetfillcolor{currentfill}%
\pgfsetfillopacity{0.700000}%
\pgfsetlinewidth{0.000000pt}%
\definecolor{currentstroke}{rgb}{0.000000,0.000000,0.000000}%
\pgfsetstrokecolor{currentstroke}%
\pgfsetstrokeopacity{0.700000}%
\pgfsetdash{}{0pt}%
\pgfpathmoveto{\pgfqpoint{7.656960in}{0.550000in}}%
\pgfpathlineto{\pgfqpoint{7.684482in}{0.550000in}}%
\pgfpathlineto{\pgfqpoint{7.684482in}{0.582763in}}%
\pgfpathlineto{\pgfqpoint{7.656960in}{0.582763in}}%
\pgfpathlineto{\pgfqpoint{7.656960in}{0.550000in}}%
\pgfpathclose%
\pgfusepath{fill}%
\end{pgfscope}%
\begin{pgfscope}%
\pgfpathrectangle{\pgfqpoint{1.250000in}{0.550000in}}{\pgfqpoint{7.750000in}{3.850000in}}%
\pgfusepath{clip}%
\pgfsetbuttcap%
\pgfsetmiterjoin%
\definecolor{currentfill}{rgb}{0.501961,0.501961,0.501961}%
\pgfsetfillcolor{currentfill}%
\pgfsetfillopacity{0.700000}%
\pgfsetlinewidth{0.000000pt}%
\definecolor{currentstroke}{rgb}{0.000000,0.000000,0.000000}%
\pgfsetstrokecolor{currentstroke}%
\pgfsetstrokeopacity{0.700000}%
\pgfsetdash{}{0pt}%
\pgfpathmoveto{\pgfqpoint{7.684482in}{0.550000in}}%
\pgfpathlineto{\pgfqpoint{7.712003in}{0.550000in}}%
\pgfpathlineto{\pgfqpoint{7.712003in}{0.581661in}}%
\pgfpathlineto{\pgfqpoint{7.684482in}{0.581661in}}%
\pgfpathlineto{\pgfqpoint{7.684482in}{0.550000in}}%
\pgfpathclose%
\pgfusepath{fill}%
\end{pgfscope}%
\begin{pgfscope}%
\pgfpathrectangle{\pgfqpoint{1.250000in}{0.550000in}}{\pgfqpoint{7.750000in}{3.850000in}}%
\pgfusepath{clip}%
\pgfsetbuttcap%
\pgfsetmiterjoin%
\definecolor{currentfill}{rgb}{0.501961,0.501961,0.501961}%
\pgfsetfillcolor{currentfill}%
\pgfsetfillopacity{0.700000}%
\pgfsetlinewidth{0.000000pt}%
\definecolor{currentstroke}{rgb}{0.000000,0.000000,0.000000}%
\pgfsetstrokecolor{currentstroke}%
\pgfsetstrokeopacity{0.700000}%
\pgfsetdash{}{0pt}%
\pgfpathmoveto{\pgfqpoint{7.712003in}{0.550000in}}%
\pgfpathlineto{\pgfqpoint{7.739524in}{0.550000in}}%
\pgfpathlineto{\pgfqpoint{7.739524in}{0.586066in}}%
\pgfpathlineto{\pgfqpoint{7.712003in}{0.586066in}}%
\pgfpathlineto{\pgfqpoint{7.712003in}{0.550000in}}%
\pgfpathclose%
\pgfusepath{fill}%
\end{pgfscope}%
\begin{pgfscope}%
\pgfpathrectangle{\pgfqpoint{1.250000in}{0.550000in}}{\pgfqpoint{7.750000in}{3.850000in}}%
\pgfusepath{clip}%
\pgfsetbuttcap%
\pgfsetmiterjoin%
\definecolor{currentfill}{rgb}{0.501961,0.501961,0.501961}%
\pgfsetfillcolor{currentfill}%
\pgfsetfillopacity{0.700000}%
\pgfsetlinewidth{0.000000pt}%
\definecolor{currentstroke}{rgb}{0.000000,0.000000,0.000000}%
\pgfsetstrokecolor{currentstroke}%
\pgfsetstrokeopacity{0.700000}%
\pgfsetdash{}{0pt}%
\pgfpathmoveto{\pgfqpoint{7.739524in}{0.550000in}}%
\pgfpathlineto{\pgfqpoint{7.767045in}{0.550000in}}%
\pgfpathlineto{\pgfqpoint{7.767045in}{0.587443in}}%
\pgfpathlineto{\pgfqpoint{7.739524in}{0.587443in}}%
\pgfpathlineto{\pgfqpoint{7.739524in}{0.550000in}}%
\pgfpathclose%
\pgfusepath{fill}%
\end{pgfscope}%
\begin{pgfscope}%
\pgfpathrectangle{\pgfqpoint{1.250000in}{0.550000in}}{\pgfqpoint{7.750000in}{3.850000in}}%
\pgfusepath{clip}%
\pgfsetbuttcap%
\pgfsetmiterjoin%
\definecolor{currentfill}{rgb}{0.501961,0.501961,0.501961}%
\pgfsetfillcolor{currentfill}%
\pgfsetfillopacity{0.700000}%
\pgfsetlinewidth{0.000000pt}%
\definecolor{currentstroke}{rgb}{0.000000,0.000000,0.000000}%
\pgfsetstrokecolor{currentstroke}%
\pgfsetstrokeopacity{0.700000}%
\pgfsetdash{}{0pt}%
\pgfpathmoveto{\pgfqpoint{7.767045in}{0.550000in}}%
\pgfpathlineto{\pgfqpoint{7.794567in}{0.550000in}}%
\pgfpathlineto{\pgfqpoint{7.794567in}{0.582487in}}%
\pgfpathlineto{\pgfqpoint{7.767045in}{0.582487in}}%
\pgfpathlineto{\pgfqpoint{7.767045in}{0.550000in}}%
\pgfpathclose%
\pgfusepath{fill}%
\end{pgfscope}%
\begin{pgfscope}%
\pgfpathrectangle{\pgfqpoint{1.250000in}{0.550000in}}{\pgfqpoint{7.750000in}{3.850000in}}%
\pgfusepath{clip}%
\pgfsetbuttcap%
\pgfsetmiterjoin%
\definecolor{currentfill}{rgb}{0.501961,0.501961,0.501961}%
\pgfsetfillcolor{currentfill}%
\pgfsetfillopacity{0.700000}%
\pgfsetlinewidth{0.000000pt}%
\definecolor{currentstroke}{rgb}{0.000000,0.000000,0.000000}%
\pgfsetstrokecolor{currentstroke}%
\pgfsetstrokeopacity{0.700000}%
\pgfsetdash{}{0pt}%
\pgfpathmoveto{\pgfqpoint{7.794567in}{0.550000in}}%
\pgfpathlineto{\pgfqpoint{7.822088in}{0.550000in}}%
\pgfpathlineto{\pgfqpoint{7.822088in}{0.589095in}}%
\pgfpathlineto{\pgfqpoint{7.794567in}{0.589095in}}%
\pgfpathlineto{\pgfqpoint{7.794567in}{0.550000in}}%
\pgfpathclose%
\pgfusepath{fill}%
\end{pgfscope}%
\begin{pgfscope}%
\pgfpathrectangle{\pgfqpoint{1.250000in}{0.550000in}}{\pgfqpoint{7.750000in}{3.850000in}}%
\pgfusepath{clip}%
\pgfsetbuttcap%
\pgfsetmiterjoin%
\definecolor{currentfill}{rgb}{0.501961,0.501961,0.501961}%
\pgfsetfillcolor{currentfill}%
\pgfsetfillopacity{0.700000}%
\pgfsetlinewidth{0.000000pt}%
\definecolor{currentstroke}{rgb}{0.000000,0.000000,0.000000}%
\pgfsetstrokecolor{currentstroke}%
\pgfsetstrokeopacity{0.700000}%
\pgfsetdash{}{0pt}%
\pgfpathmoveto{\pgfqpoint{7.822088in}{0.550000in}}%
\pgfpathlineto{\pgfqpoint{7.849609in}{0.550000in}}%
\pgfpathlineto{\pgfqpoint{7.849609in}{0.580835in}}%
\pgfpathlineto{\pgfqpoint{7.822088in}{0.580835in}}%
\pgfpathlineto{\pgfqpoint{7.822088in}{0.550000in}}%
\pgfpathclose%
\pgfusepath{fill}%
\end{pgfscope}%
\begin{pgfscope}%
\pgfpathrectangle{\pgfqpoint{1.250000in}{0.550000in}}{\pgfqpoint{7.750000in}{3.850000in}}%
\pgfusepath{clip}%
\pgfsetbuttcap%
\pgfsetmiterjoin%
\definecolor{currentfill}{rgb}{0.501961,0.501961,0.501961}%
\pgfsetfillcolor{currentfill}%
\pgfsetfillopacity{0.700000}%
\pgfsetlinewidth{0.000000pt}%
\definecolor{currentstroke}{rgb}{0.000000,0.000000,0.000000}%
\pgfsetstrokecolor{currentstroke}%
\pgfsetstrokeopacity{0.700000}%
\pgfsetdash{}{0pt}%
\pgfpathmoveto{\pgfqpoint{7.849609in}{0.550000in}}%
\pgfpathlineto{\pgfqpoint{7.877131in}{0.550000in}}%
\pgfpathlineto{\pgfqpoint{7.877131in}{0.578082in}}%
\pgfpathlineto{\pgfqpoint{7.849609in}{0.578082in}}%
\pgfpathlineto{\pgfqpoint{7.849609in}{0.550000in}}%
\pgfpathclose%
\pgfusepath{fill}%
\end{pgfscope}%
\begin{pgfscope}%
\pgfpathrectangle{\pgfqpoint{1.250000in}{0.550000in}}{\pgfqpoint{7.750000in}{3.850000in}}%
\pgfusepath{clip}%
\pgfsetbuttcap%
\pgfsetmiterjoin%
\definecolor{currentfill}{rgb}{0.501961,0.501961,0.501961}%
\pgfsetfillcolor{currentfill}%
\pgfsetfillopacity{0.700000}%
\pgfsetlinewidth{0.000000pt}%
\definecolor{currentstroke}{rgb}{0.000000,0.000000,0.000000}%
\pgfsetstrokecolor{currentstroke}%
\pgfsetstrokeopacity{0.700000}%
\pgfsetdash{}{0pt}%
\pgfpathmoveto{\pgfqpoint{7.877131in}{0.550000in}}%
\pgfpathlineto{\pgfqpoint{7.904652in}{0.550000in}}%
\pgfpathlineto{\pgfqpoint{7.904652in}{0.588544in}}%
\pgfpathlineto{\pgfqpoint{7.877131in}{0.588544in}}%
\pgfpathlineto{\pgfqpoint{7.877131in}{0.550000in}}%
\pgfpathclose%
\pgfusepath{fill}%
\end{pgfscope}%
\begin{pgfscope}%
\pgfpathrectangle{\pgfqpoint{1.250000in}{0.550000in}}{\pgfqpoint{7.750000in}{3.850000in}}%
\pgfusepath{clip}%
\pgfsetbuttcap%
\pgfsetmiterjoin%
\definecolor{currentfill}{rgb}{0.501961,0.501961,0.501961}%
\pgfsetfillcolor{currentfill}%
\pgfsetfillopacity{0.700000}%
\pgfsetlinewidth{0.000000pt}%
\definecolor{currentstroke}{rgb}{0.000000,0.000000,0.000000}%
\pgfsetstrokecolor{currentstroke}%
\pgfsetstrokeopacity{0.700000}%
\pgfsetdash{}{0pt}%
\pgfpathmoveto{\pgfqpoint{7.904652in}{0.550000in}}%
\pgfpathlineto{\pgfqpoint{7.932173in}{0.550000in}}%
\pgfpathlineto{\pgfqpoint{7.932173in}{0.582212in}}%
\pgfpathlineto{\pgfqpoint{7.904652in}{0.582212in}}%
\pgfpathlineto{\pgfqpoint{7.904652in}{0.550000in}}%
\pgfpathclose%
\pgfusepath{fill}%
\end{pgfscope}%
\begin{pgfscope}%
\pgfpathrectangle{\pgfqpoint{1.250000in}{0.550000in}}{\pgfqpoint{7.750000in}{3.850000in}}%
\pgfusepath{clip}%
\pgfsetbuttcap%
\pgfsetmiterjoin%
\definecolor{currentfill}{rgb}{0.501961,0.501961,0.501961}%
\pgfsetfillcolor{currentfill}%
\pgfsetfillopacity{0.700000}%
\pgfsetlinewidth{0.000000pt}%
\definecolor{currentstroke}{rgb}{0.000000,0.000000,0.000000}%
\pgfsetstrokecolor{currentstroke}%
\pgfsetstrokeopacity{0.700000}%
\pgfsetdash{}{0pt}%
\pgfpathmoveto{\pgfqpoint{7.932173in}{0.550000in}}%
\pgfpathlineto{\pgfqpoint{7.959695in}{0.550000in}}%
\pgfpathlineto{\pgfqpoint{7.959695in}{0.583589in}}%
\pgfpathlineto{\pgfqpoint{7.932173in}{0.583589in}}%
\pgfpathlineto{\pgfqpoint{7.932173in}{0.550000in}}%
\pgfpathclose%
\pgfusepath{fill}%
\end{pgfscope}%
\begin{pgfscope}%
\pgfpathrectangle{\pgfqpoint{1.250000in}{0.550000in}}{\pgfqpoint{7.750000in}{3.850000in}}%
\pgfusepath{clip}%
\pgfsetbuttcap%
\pgfsetmiterjoin%
\definecolor{currentfill}{rgb}{0.501961,0.501961,0.501961}%
\pgfsetfillcolor{currentfill}%
\pgfsetfillopacity{0.700000}%
\pgfsetlinewidth{0.000000pt}%
\definecolor{currentstroke}{rgb}{0.000000,0.000000,0.000000}%
\pgfsetstrokecolor{currentstroke}%
\pgfsetstrokeopacity{0.700000}%
\pgfsetdash{}{0pt}%
\pgfpathmoveto{\pgfqpoint{7.959695in}{0.550000in}}%
\pgfpathlineto{\pgfqpoint{7.987216in}{0.550000in}}%
\pgfpathlineto{\pgfqpoint{7.987216in}{0.582487in}}%
\pgfpathlineto{\pgfqpoint{7.959695in}{0.582487in}}%
\pgfpathlineto{\pgfqpoint{7.959695in}{0.550000in}}%
\pgfpathclose%
\pgfusepath{fill}%
\end{pgfscope}%
\begin{pgfscope}%
\pgfpathrectangle{\pgfqpoint{1.250000in}{0.550000in}}{\pgfqpoint{7.750000in}{3.850000in}}%
\pgfusepath{clip}%
\pgfsetbuttcap%
\pgfsetmiterjoin%
\definecolor{currentfill}{rgb}{0.501961,0.501961,0.501961}%
\pgfsetfillcolor{currentfill}%
\pgfsetfillopacity{0.700000}%
\pgfsetlinewidth{0.000000pt}%
\definecolor{currentstroke}{rgb}{0.000000,0.000000,0.000000}%
\pgfsetstrokecolor{currentstroke}%
\pgfsetstrokeopacity{0.700000}%
\pgfsetdash{}{0pt}%
\pgfpathmoveto{\pgfqpoint{7.987216in}{0.550000in}}%
\pgfpathlineto{\pgfqpoint{8.014737in}{0.550000in}}%
\pgfpathlineto{\pgfqpoint{8.014737in}{0.589646in}}%
\pgfpathlineto{\pgfqpoint{7.987216in}{0.589646in}}%
\pgfpathlineto{\pgfqpoint{7.987216in}{0.550000in}}%
\pgfpathclose%
\pgfusepath{fill}%
\end{pgfscope}%
\begin{pgfscope}%
\pgfpathrectangle{\pgfqpoint{1.250000in}{0.550000in}}{\pgfqpoint{7.750000in}{3.850000in}}%
\pgfusepath{clip}%
\pgfsetbuttcap%
\pgfsetmiterjoin%
\definecolor{currentfill}{rgb}{0.501961,0.501961,0.501961}%
\pgfsetfillcolor{currentfill}%
\pgfsetfillopacity{0.700000}%
\pgfsetlinewidth{0.000000pt}%
\definecolor{currentstroke}{rgb}{0.000000,0.000000,0.000000}%
\pgfsetstrokecolor{currentstroke}%
\pgfsetstrokeopacity{0.700000}%
\pgfsetdash{}{0pt}%
\pgfpathmoveto{\pgfqpoint{8.014737in}{0.550000in}}%
\pgfpathlineto{\pgfqpoint{8.042259in}{0.550000in}}%
\pgfpathlineto{\pgfqpoint{8.042259in}{0.584139in}}%
\pgfpathlineto{\pgfqpoint{8.014737in}{0.584139in}}%
\pgfpathlineto{\pgfqpoint{8.014737in}{0.550000in}}%
\pgfpathclose%
\pgfusepath{fill}%
\end{pgfscope}%
\begin{pgfscope}%
\pgfpathrectangle{\pgfqpoint{1.250000in}{0.550000in}}{\pgfqpoint{7.750000in}{3.850000in}}%
\pgfusepath{clip}%
\pgfsetbuttcap%
\pgfsetmiterjoin%
\definecolor{currentfill}{rgb}{0.501961,0.501961,0.501961}%
\pgfsetfillcolor{currentfill}%
\pgfsetfillopacity{0.700000}%
\pgfsetlinewidth{0.000000pt}%
\definecolor{currentstroke}{rgb}{0.000000,0.000000,0.000000}%
\pgfsetstrokecolor{currentstroke}%
\pgfsetstrokeopacity{0.700000}%
\pgfsetdash{}{0pt}%
\pgfpathmoveto{\pgfqpoint{8.042259in}{0.550000in}}%
\pgfpathlineto{\pgfqpoint{8.069780in}{0.550000in}}%
\pgfpathlineto{\pgfqpoint{8.069780in}{0.581937in}}%
\pgfpathlineto{\pgfqpoint{8.042259in}{0.581937in}}%
\pgfpathlineto{\pgfqpoint{8.042259in}{0.550000in}}%
\pgfpathclose%
\pgfusepath{fill}%
\end{pgfscope}%
\begin{pgfscope}%
\pgfpathrectangle{\pgfqpoint{1.250000in}{0.550000in}}{\pgfqpoint{7.750000in}{3.850000in}}%
\pgfusepath{clip}%
\pgfsetbuttcap%
\pgfsetmiterjoin%
\definecolor{currentfill}{rgb}{0.501961,0.501961,0.501961}%
\pgfsetfillcolor{currentfill}%
\pgfsetfillopacity{0.700000}%
\pgfsetlinewidth{0.000000pt}%
\definecolor{currentstroke}{rgb}{0.000000,0.000000,0.000000}%
\pgfsetstrokecolor{currentstroke}%
\pgfsetstrokeopacity{0.700000}%
\pgfsetdash{}{0pt}%
\pgfpathmoveto{\pgfqpoint{8.069780in}{0.550000in}}%
\pgfpathlineto{\pgfqpoint{8.097301in}{0.550000in}}%
\pgfpathlineto{\pgfqpoint{8.097301in}{0.585241in}}%
\pgfpathlineto{\pgfqpoint{8.069780in}{0.585241in}}%
\pgfpathlineto{\pgfqpoint{8.069780in}{0.550000in}}%
\pgfpathclose%
\pgfusepath{fill}%
\end{pgfscope}%
\begin{pgfscope}%
\pgfpathrectangle{\pgfqpoint{1.250000in}{0.550000in}}{\pgfqpoint{7.750000in}{3.850000in}}%
\pgfusepath{clip}%
\pgfsetbuttcap%
\pgfsetmiterjoin%
\definecolor{currentfill}{rgb}{0.501961,0.501961,0.501961}%
\pgfsetfillcolor{currentfill}%
\pgfsetfillopacity{0.700000}%
\pgfsetlinewidth{0.000000pt}%
\definecolor{currentstroke}{rgb}{0.000000,0.000000,0.000000}%
\pgfsetstrokecolor{currentstroke}%
\pgfsetstrokeopacity{0.700000}%
\pgfsetdash{}{0pt}%
\pgfpathmoveto{\pgfqpoint{8.097301in}{0.550000in}}%
\pgfpathlineto{\pgfqpoint{8.124822in}{0.550000in}}%
\pgfpathlineto{\pgfqpoint{8.124822in}{0.580835in}}%
\pgfpathlineto{\pgfqpoint{8.097301in}{0.580835in}}%
\pgfpathlineto{\pgfqpoint{8.097301in}{0.550000in}}%
\pgfpathclose%
\pgfusepath{fill}%
\end{pgfscope}%
\begin{pgfscope}%
\pgfpathrectangle{\pgfqpoint{1.250000in}{0.550000in}}{\pgfqpoint{7.750000in}{3.850000in}}%
\pgfusepath{clip}%
\pgfsetbuttcap%
\pgfsetmiterjoin%
\definecolor{currentfill}{rgb}{0.501961,0.501961,0.501961}%
\pgfsetfillcolor{currentfill}%
\pgfsetfillopacity{0.700000}%
\pgfsetlinewidth{0.000000pt}%
\definecolor{currentstroke}{rgb}{0.000000,0.000000,0.000000}%
\pgfsetstrokecolor{currentstroke}%
\pgfsetstrokeopacity{0.700000}%
\pgfsetdash{}{0pt}%
\pgfpathmoveto{\pgfqpoint{8.124822in}{0.550000in}}%
\pgfpathlineto{\pgfqpoint{8.152344in}{0.550000in}}%
\pgfpathlineto{\pgfqpoint{8.152344in}{0.577532in}}%
\pgfpathlineto{\pgfqpoint{8.124822in}{0.577532in}}%
\pgfpathlineto{\pgfqpoint{8.124822in}{0.550000in}}%
\pgfpathclose%
\pgfusepath{fill}%
\end{pgfscope}%
\begin{pgfscope}%
\pgfpathrectangle{\pgfqpoint{1.250000in}{0.550000in}}{\pgfqpoint{7.750000in}{3.850000in}}%
\pgfusepath{clip}%
\pgfsetbuttcap%
\pgfsetmiterjoin%
\definecolor{currentfill}{rgb}{0.501961,0.501961,0.501961}%
\pgfsetfillcolor{currentfill}%
\pgfsetfillopacity{0.700000}%
\pgfsetlinewidth{0.000000pt}%
\definecolor{currentstroke}{rgb}{0.000000,0.000000,0.000000}%
\pgfsetstrokecolor{currentstroke}%
\pgfsetstrokeopacity{0.700000}%
\pgfsetdash{}{0pt}%
\pgfpathmoveto{\pgfqpoint{8.152344in}{0.550000in}}%
\pgfpathlineto{\pgfqpoint{8.179865in}{0.550000in}}%
\pgfpathlineto{\pgfqpoint{8.179865in}{0.583589in}}%
\pgfpathlineto{\pgfqpoint{8.152344in}{0.583589in}}%
\pgfpathlineto{\pgfqpoint{8.152344in}{0.550000in}}%
\pgfpathclose%
\pgfusepath{fill}%
\end{pgfscope}%
\begin{pgfscope}%
\pgfpathrectangle{\pgfqpoint{1.250000in}{0.550000in}}{\pgfqpoint{7.750000in}{3.850000in}}%
\pgfusepath{clip}%
\pgfsetbuttcap%
\pgfsetmiterjoin%
\definecolor{currentfill}{rgb}{0.501961,0.501961,0.501961}%
\pgfsetfillcolor{currentfill}%
\pgfsetfillopacity{0.700000}%
\pgfsetlinewidth{0.000000pt}%
\definecolor{currentstroke}{rgb}{0.000000,0.000000,0.000000}%
\pgfsetstrokecolor{currentstroke}%
\pgfsetstrokeopacity{0.700000}%
\pgfsetdash{}{0pt}%
\pgfpathmoveto{\pgfqpoint{8.179865in}{0.550000in}}%
\pgfpathlineto{\pgfqpoint{8.207386in}{0.550000in}}%
\pgfpathlineto{\pgfqpoint{8.207386in}{0.579734in}}%
\pgfpathlineto{\pgfqpoint{8.179865in}{0.579734in}}%
\pgfpathlineto{\pgfqpoint{8.179865in}{0.550000in}}%
\pgfpathclose%
\pgfusepath{fill}%
\end{pgfscope}%
\begin{pgfscope}%
\pgfpathrectangle{\pgfqpoint{1.250000in}{0.550000in}}{\pgfqpoint{7.750000in}{3.850000in}}%
\pgfusepath{clip}%
\pgfsetbuttcap%
\pgfsetmiterjoin%
\definecolor{currentfill}{rgb}{0.501961,0.501961,0.501961}%
\pgfsetfillcolor{currentfill}%
\pgfsetfillopacity{0.700000}%
\pgfsetlinewidth{0.000000pt}%
\definecolor{currentstroke}{rgb}{0.000000,0.000000,0.000000}%
\pgfsetstrokecolor{currentstroke}%
\pgfsetstrokeopacity{0.700000}%
\pgfsetdash{}{0pt}%
\pgfpathmoveto{\pgfqpoint{8.207386in}{0.550000in}}%
\pgfpathlineto{\pgfqpoint{8.234908in}{0.550000in}}%
\pgfpathlineto{\pgfqpoint{8.234908in}{0.583313in}}%
\pgfpathlineto{\pgfqpoint{8.207386in}{0.583313in}}%
\pgfpathlineto{\pgfqpoint{8.207386in}{0.550000in}}%
\pgfpathclose%
\pgfusepath{fill}%
\end{pgfscope}%
\begin{pgfscope}%
\pgfpathrectangle{\pgfqpoint{1.250000in}{0.550000in}}{\pgfqpoint{7.750000in}{3.850000in}}%
\pgfusepath{clip}%
\pgfsetbuttcap%
\pgfsetmiterjoin%
\definecolor{currentfill}{rgb}{0.501961,0.501961,0.501961}%
\pgfsetfillcolor{currentfill}%
\pgfsetfillopacity{0.700000}%
\pgfsetlinewidth{0.000000pt}%
\definecolor{currentstroke}{rgb}{0.000000,0.000000,0.000000}%
\pgfsetstrokecolor{currentstroke}%
\pgfsetstrokeopacity{0.700000}%
\pgfsetdash{}{0pt}%
\pgfpathmoveto{\pgfqpoint{8.234908in}{0.550000in}}%
\pgfpathlineto{\pgfqpoint{8.262429in}{0.550000in}}%
\pgfpathlineto{\pgfqpoint{8.262429in}{0.580285in}}%
\pgfpathlineto{\pgfqpoint{8.234908in}{0.580285in}}%
\pgfpathlineto{\pgfqpoint{8.234908in}{0.550000in}}%
\pgfpathclose%
\pgfusepath{fill}%
\end{pgfscope}%
\begin{pgfscope}%
\pgfpathrectangle{\pgfqpoint{1.250000in}{0.550000in}}{\pgfqpoint{7.750000in}{3.850000in}}%
\pgfusepath{clip}%
\pgfsetbuttcap%
\pgfsetmiterjoin%
\definecolor{currentfill}{rgb}{0.501961,0.501961,0.501961}%
\pgfsetfillcolor{currentfill}%
\pgfsetfillopacity{0.700000}%
\pgfsetlinewidth{0.000000pt}%
\definecolor{currentstroke}{rgb}{0.000000,0.000000,0.000000}%
\pgfsetstrokecolor{currentstroke}%
\pgfsetstrokeopacity{0.700000}%
\pgfsetdash{}{0pt}%
\pgfpathmoveto{\pgfqpoint{8.262429in}{0.550000in}}%
\pgfpathlineto{\pgfqpoint{8.289950in}{0.550000in}}%
\pgfpathlineto{\pgfqpoint{8.289950in}{0.578082in}}%
\pgfpathlineto{\pgfqpoint{8.262429in}{0.578082in}}%
\pgfpathlineto{\pgfqpoint{8.262429in}{0.550000in}}%
\pgfpathclose%
\pgfusepath{fill}%
\end{pgfscope}%
\begin{pgfscope}%
\pgfpathrectangle{\pgfqpoint{1.250000in}{0.550000in}}{\pgfqpoint{7.750000in}{3.850000in}}%
\pgfusepath{clip}%
\pgfsetbuttcap%
\pgfsetmiterjoin%
\definecolor{currentfill}{rgb}{0.501961,0.501961,0.501961}%
\pgfsetfillcolor{currentfill}%
\pgfsetfillopacity{0.700000}%
\pgfsetlinewidth{0.000000pt}%
\definecolor{currentstroke}{rgb}{0.000000,0.000000,0.000000}%
\pgfsetstrokecolor{currentstroke}%
\pgfsetstrokeopacity{0.700000}%
\pgfsetdash{}{0pt}%
\pgfpathmoveto{\pgfqpoint{8.289950in}{0.550000in}}%
\pgfpathlineto{\pgfqpoint{8.317472in}{0.550000in}}%
\pgfpathlineto{\pgfqpoint{8.317472in}{0.584965in}}%
\pgfpathlineto{\pgfqpoint{8.289950in}{0.584965in}}%
\pgfpathlineto{\pgfqpoint{8.289950in}{0.550000in}}%
\pgfpathclose%
\pgfusepath{fill}%
\end{pgfscope}%
\begin{pgfscope}%
\pgfpathrectangle{\pgfqpoint{1.250000in}{0.550000in}}{\pgfqpoint{7.750000in}{3.850000in}}%
\pgfusepath{clip}%
\pgfsetbuttcap%
\pgfsetmiterjoin%
\definecolor{currentfill}{rgb}{0.501961,0.501961,0.501961}%
\pgfsetfillcolor{currentfill}%
\pgfsetfillopacity{0.700000}%
\pgfsetlinewidth{0.000000pt}%
\definecolor{currentstroke}{rgb}{0.000000,0.000000,0.000000}%
\pgfsetstrokecolor{currentstroke}%
\pgfsetstrokeopacity{0.700000}%
\pgfsetdash{}{0pt}%
\pgfpathmoveto{\pgfqpoint{8.317472in}{0.550000in}}%
\pgfpathlineto{\pgfqpoint{8.344993in}{0.550000in}}%
\pgfpathlineto{\pgfqpoint{8.344993in}{0.577532in}}%
\pgfpathlineto{\pgfqpoint{8.317472in}{0.577532in}}%
\pgfpathlineto{\pgfqpoint{8.317472in}{0.550000in}}%
\pgfpathclose%
\pgfusepath{fill}%
\end{pgfscope}%
\begin{pgfscope}%
\pgfpathrectangle{\pgfqpoint{1.250000in}{0.550000in}}{\pgfqpoint{7.750000in}{3.850000in}}%
\pgfusepath{clip}%
\pgfsetbuttcap%
\pgfsetmiterjoin%
\definecolor{currentfill}{rgb}{0.501961,0.501961,0.501961}%
\pgfsetfillcolor{currentfill}%
\pgfsetfillopacity{0.700000}%
\pgfsetlinewidth{0.000000pt}%
\definecolor{currentstroke}{rgb}{0.000000,0.000000,0.000000}%
\pgfsetstrokecolor{currentstroke}%
\pgfsetstrokeopacity{0.700000}%
\pgfsetdash{}{0pt}%
\pgfpathmoveto{\pgfqpoint{8.344993in}{0.550000in}}%
\pgfpathlineto{\pgfqpoint{8.372514in}{0.550000in}}%
\pgfpathlineto{\pgfqpoint{8.372514in}{0.578082in}}%
\pgfpathlineto{\pgfqpoint{8.344993in}{0.578082in}}%
\pgfpathlineto{\pgfqpoint{8.344993in}{0.550000in}}%
\pgfpathclose%
\pgfusepath{fill}%
\end{pgfscope}%
\begin{pgfscope}%
\pgfpathrectangle{\pgfqpoint{1.250000in}{0.550000in}}{\pgfqpoint{7.750000in}{3.850000in}}%
\pgfusepath{clip}%
\pgfsetbuttcap%
\pgfsetmiterjoin%
\definecolor{currentfill}{rgb}{0.501961,0.501961,0.501961}%
\pgfsetfillcolor{currentfill}%
\pgfsetfillopacity{0.700000}%
\pgfsetlinewidth{0.000000pt}%
\definecolor{currentstroke}{rgb}{0.000000,0.000000,0.000000}%
\pgfsetstrokecolor{currentstroke}%
\pgfsetstrokeopacity{0.700000}%
\pgfsetdash{}{0pt}%
\pgfpathmoveto{\pgfqpoint{8.372514in}{0.550000in}}%
\pgfpathlineto{\pgfqpoint{8.400036in}{0.550000in}}%
\pgfpathlineto{\pgfqpoint{8.400036in}{0.581937in}}%
\pgfpathlineto{\pgfqpoint{8.372514in}{0.581937in}}%
\pgfpathlineto{\pgfqpoint{8.372514in}{0.550000in}}%
\pgfpathclose%
\pgfusepath{fill}%
\end{pgfscope}%
\begin{pgfscope}%
\pgfpathrectangle{\pgfqpoint{1.250000in}{0.550000in}}{\pgfqpoint{7.750000in}{3.850000in}}%
\pgfusepath{clip}%
\pgfsetbuttcap%
\pgfsetmiterjoin%
\definecolor{currentfill}{rgb}{0.501961,0.501961,0.501961}%
\pgfsetfillcolor{currentfill}%
\pgfsetfillopacity{0.700000}%
\pgfsetlinewidth{0.000000pt}%
\definecolor{currentstroke}{rgb}{0.000000,0.000000,0.000000}%
\pgfsetstrokecolor{currentstroke}%
\pgfsetstrokeopacity{0.700000}%
\pgfsetdash{}{0pt}%
\pgfpathmoveto{\pgfqpoint{8.400036in}{0.550000in}}%
\pgfpathlineto{\pgfqpoint{8.427557in}{0.550000in}}%
\pgfpathlineto{\pgfqpoint{8.427557in}{0.585791in}}%
\pgfpathlineto{\pgfqpoint{8.400036in}{0.585791in}}%
\pgfpathlineto{\pgfqpoint{8.400036in}{0.550000in}}%
\pgfpathclose%
\pgfusepath{fill}%
\end{pgfscope}%
\begin{pgfscope}%
\pgfpathrectangle{\pgfqpoint{1.250000in}{0.550000in}}{\pgfqpoint{7.750000in}{3.850000in}}%
\pgfusepath{clip}%
\pgfsetbuttcap%
\pgfsetmiterjoin%
\definecolor{currentfill}{rgb}{0.501961,0.501961,0.501961}%
\pgfsetfillcolor{currentfill}%
\pgfsetfillopacity{0.700000}%
\pgfsetlinewidth{0.000000pt}%
\definecolor{currentstroke}{rgb}{0.000000,0.000000,0.000000}%
\pgfsetstrokecolor{currentstroke}%
\pgfsetstrokeopacity{0.700000}%
\pgfsetdash{}{0pt}%
\pgfpathmoveto{\pgfqpoint{8.427557in}{0.550000in}}%
\pgfpathlineto{\pgfqpoint{8.455078in}{0.550000in}}%
\pgfpathlineto{\pgfqpoint{8.455078in}{0.577807in}}%
\pgfpathlineto{\pgfqpoint{8.427557in}{0.577807in}}%
\pgfpathlineto{\pgfqpoint{8.427557in}{0.550000in}}%
\pgfpathclose%
\pgfusepath{fill}%
\end{pgfscope}%
\begin{pgfscope}%
\pgfpathrectangle{\pgfqpoint{1.250000in}{0.550000in}}{\pgfqpoint{7.750000in}{3.850000in}}%
\pgfusepath{clip}%
\pgfsetbuttcap%
\pgfsetmiterjoin%
\definecolor{currentfill}{rgb}{0.501961,0.501961,0.501961}%
\pgfsetfillcolor{currentfill}%
\pgfsetfillopacity{0.700000}%
\pgfsetlinewidth{0.000000pt}%
\definecolor{currentstroke}{rgb}{0.000000,0.000000,0.000000}%
\pgfsetstrokecolor{currentstroke}%
\pgfsetstrokeopacity{0.700000}%
\pgfsetdash{}{0pt}%
\pgfpathmoveto{\pgfqpoint{8.455078in}{0.550000in}}%
\pgfpathlineto{\pgfqpoint{8.482599in}{0.550000in}}%
\pgfpathlineto{\pgfqpoint{8.482599in}{0.585791in}}%
\pgfpathlineto{\pgfqpoint{8.455078in}{0.585791in}}%
\pgfpathlineto{\pgfqpoint{8.455078in}{0.550000in}}%
\pgfpathclose%
\pgfusepath{fill}%
\end{pgfscope}%
\begin{pgfscope}%
\pgfpathrectangle{\pgfqpoint{1.250000in}{0.550000in}}{\pgfqpoint{7.750000in}{3.850000in}}%
\pgfusepath{clip}%
\pgfsetbuttcap%
\pgfsetmiterjoin%
\definecolor{currentfill}{rgb}{0.501961,0.501961,0.501961}%
\pgfsetfillcolor{currentfill}%
\pgfsetfillopacity{0.700000}%
\pgfsetlinewidth{0.000000pt}%
\definecolor{currentstroke}{rgb}{0.000000,0.000000,0.000000}%
\pgfsetstrokecolor{currentstroke}%
\pgfsetstrokeopacity{0.700000}%
\pgfsetdash{}{0pt}%
\pgfpathmoveto{\pgfqpoint{8.482599in}{0.550000in}}%
\pgfpathlineto{\pgfqpoint{8.510121in}{0.550000in}}%
\pgfpathlineto{\pgfqpoint{8.510121in}{0.582487in}}%
\pgfpathlineto{\pgfqpoint{8.482599in}{0.582487in}}%
\pgfpathlineto{\pgfqpoint{8.482599in}{0.550000in}}%
\pgfpathclose%
\pgfusepath{fill}%
\end{pgfscope}%
\begin{pgfscope}%
\pgfpathrectangle{\pgfqpoint{1.250000in}{0.550000in}}{\pgfqpoint{7.750000in}{3.850000in}}%
\pgfusepath{clip}%
\pgfsetbuttcap%
\pgfsetmiterjoin%
\definecolor{currentfill}{rgb}{0.501961,0.501961,0.501961}%
\pgfsetfillcolor{currentfill}%
\pgfsetfillopacity{0.700000}%
\pgfsetlinewidth{0.000000pt}%
\definecolor{currentstroke}{rgb}{0.000000,0.000000,0.000000}%
\pgfsetstrokecolor{currentstroke}%
\pgfsetstrokeopacity{0.700000}%
\pgfsetdash{}{0pt}%
\pgfpathmoveto{\pgfqpoint{8.510121in}{0.550000in}}%
\pgfpathlineto{\pgfqpoint{8.537642in}{0.550000in}}%
\pgfpathlineto{\pgfqpoint{8.537642in}{0.587994in}}%
\pgfpathlineto{\pgfqpoint{8.510121in}{0.587994in}}%
\pgfpathlineto{\pgfqpoint{8.510121in}{0.550000in}}%
\pgfpathclose%
\pgfusepath{fill}%
\end{pgfscope}%
\begin{pgfscope}%
\pgfpathrectangle{\pgfqpoint{1.250000in}{0.550000in}}{\pgfqpoint{7.750000in}{3.850000in}}%
\pgfusepath{clip}%
\pgfsetbuttcap%
\pgfsetmiterjoin%
\definecolor{currentfill}{rgb}{0.501961,0.501961,0.501961}%
\pgfsetfillcolor{currentfill}%
\pgfsetfillopacity{0.700000}%
\pgfsetlinewidth{0.000000pt}%
\definecolor{currentstroke}{rgb}{0.000000,0.000000,0.000000}%
\pgfsetstrokecolor{currentstroke}%
\pgfsetstrokeopacity{0.700000}%
\pgfsetdash{}{0pt}%
\pgfpathmoveto{\pgfqpoint{8.537642in}{0.550000in}}%
\pgfpathlineto{\pgfqpoint{8.565163in}{0.550000in}}%
\pgfpathlineto{\pgfqpoint{8.565163in}{0.584139in}}%
\pgfpathlineto{\pgfqpoint{8.537642in}{0.584139in}}%
\pgfpathlineto{\pgfqpoint{8.537642in}{0.550000in}}%
\pgfpathclose%
\pgfusepath{fill}%
\end{pgfscope}%
\begin{pgfscope}%
\pgfpathrectangle{\pgfqpoint{1.250000in}{0.550000in}}{\pgfqpoint{7.750000in}{3.850000in}}%
\pgfusepath{clip}%
\pgfsetbuttcap%
\pgfsetmiterjoin%
\definecolor{currentfill}{rgb}{0.501961,0.501961,0.501961}%
\pgfsetfillcolor{currentfill}%
\pgfsetfillopacity{0.700000}%
\pgfsetlinewidth{0.000000pt}%
\definecolor{currentstroke}{rgb}{0.000000,0.000000,0.000000}%
\pgfsetstrokecolor{currentstroke}%
\pgfsetstrokeopacity{0.700000}%
\pgfsetdash{}{0pt}%
\pgfpathmoveto{\pgfqpoint{8.565163in}{0.550000in}}%
\pgfpathlineto{\pgfqpoint{8.592685in}{0.550000in}}%
\pgfpathlineto{\pgfqpoint{8.592685in}{0.575604in}}%
\pgfpathlineto{\pgfqpoint{8.565163in}{0.575604in}}%
\pgfpathlineto{\pgfqpoint{8.565163in}{0.550000in}}%
\pgfpathclose%
\pgfusepath{fill}%
\end{pgfscope}%
\begin{pgfscope}%
\pgfpathrectangle{\pgfqpoint{1.250000in}{0.550000in}}{\pgfqpoint{7.750000in}{3.850000in}}%
\pgfusepath{clip}%
\pgfsetbuttcap%
\pgfsetmiterjoin%
\definecolor{currentfill}{rgb}{0.501961,0.501961,0.501961}%
\pgfsetfillcolor{currentfill}%
\pgfsetfillopacity{0.700000}%
\pgfsetlinewidth{0.000000pt}%
\definecolor{currentstroke}{rgb}{0.000000,0.000000,0.000000}%
\pgfsetstrokecolor{currentstroke}%
\pgfsetstrokeopacity{0.700000}%
\pgfsetdash{}{0pt}%
\pgfpathmoveto{\pgfqpoint{8.592685in}{0.550000in}}%
\pgfpathlineto{\pgfqpoint{8.620206in}{0.550000in}}%
\pgfpathlineto{\pgfqpoint{8.620206in}{0.563215in}}%
\pgfpathlineto{\pgfqpoint{8.592685in}{0.563215in}}%
\pgfpathlineto{\pgfqpoint{8.592685in}{0.550000in}}%
\pgfpathclose%
\pgfusepath{fill}%
\end{pgfscope}%
\begin{pgfscope}%
\pgfpathrectangle{\pgfqpoint{1.250000in}{0.550000in}}{\pgfqpoint{7.750000in}{3.850000in}}%
\pgfusepath{clip}%
\pgfsetbuttcap%
\pgfsetmiterjoin%
\definecolor{currentfill}{rgb}{0.501961,0.501961,0.501961}%
\pgfsetfillcolor{currentfill}%
\pgfsetfillopacity{0.700000}%
\pgfsetlinewidth{0.000000pt}%
\definecolor{currentstroke}{rgb}{0.000000,0.000000,0.000000}%
\pgfsetstrokecolor{currentstroke}%
\pgfsetstrokeopacity{0.700000}%
\pgfsetdash{}{0pt}%
\pgfpathmoveto{\pgfqpoint{8.620206in}{0.550000in}}%
\pgfpathlineto{\pgfqpoint{8.647727in}{0.550000in}}%
\pgfpathlineto{\pgfqpoint{8.647727in}{0.559361in}}%
\pgfpathlineto{\pgfqpoint{8.620206in}{0.559361in}}%
\pgfpathlineto{\pgfqpoint{8.620206in}{0.550000in}}%
\pgfpathclose%
\pgfusepath{fill}%
\end{pgfscope}%
\begin{pgfscope}%
\pgfsetbuttcap%
\pgfsetroundjoin%
\definecolor{currentfill}{rgb}{0.000000,0.000000,0.000000}%
\pgfsetfillcolor{currentfill}%
\pgfsetlinewidth{0.803000pt}%
\definecolor{currentstroke}{rgb}{0.000000,0.000000,0.000000}%
\pgfsetstrokecolor{currentstroke}%
\pgfsetdash{}{0pt}%
\pgfsys@defobject{currentmarker}{\pgfqpoint{0.000000in}{-0.048611in}}{\pgfqpoint{0.000000in}{0.000000in}}{%
\pgfpathmoveto{\pgfqpoint{0.000000in}{0.000000in}}%
\pgfpathlineto{\pgfqpoint{0.000000in}{-0.048611in}}%
\pgfusepath{stroke,fill}%
}%
\begin{pgfscope}%
\pgfsys@transformshift{1.602273in}{0.550000in}%
\pgfsys@useobject{currentmarker}{}%
\end{pgfscope}%
\end{pgfscope}%
\begin{pgfscope}%
\definecolor{textcolor}{rgb}{0.000000,0.000000,0.000000}%
\pgfsetstrokecolor{textcolor}%
\pgfsetfillcolor{textcolor}%
\pgftext[x=1.602273in,y=0.452778in,,top]{\color{textcolor}{\sffamily\fontsize{10.000000}{12.000000}\selectfont\catcode`\^=\active\def^{\ifmmode\sp\else\^{}\fi}\catcode`\%=\active\def%{\%}0}}%
\end{pgfscope}%
\begin{pgfscope}%
\pgfsetbuttcap%
\pgfsetroundjoin%
\definecolor{currentfill}{rgb}{0.000000,0.000000,0.000000}%
\pgfsetfillcolor{currentfill}%
\pgfsetlinewidth{0.803000pt}%
\definecolor{currentstroke}{rgb}{0.000000,0.000000,0.000000}%
\pgfsetstrokecolor{currentstroke}%
\pgfsetdash{}{0pt}%
\pgfsys@defobject{currentmarker}{\pgfqpoint{0.000000in}{-0.048611in}}{\pgfqpoint{0.000000in}{0.000000in}}{%
\pgfpathmoveto{\pgfqpoint{0.000000in}{0.000000in}}%
\pgfpathlineto{\pgfqpoint{0.000000in}{-0.048611in}}%
\pgfusepath{stroke,fill}%
}%
\begin{pgfscope}%
\pgfsys@transformshift{2.978338in}{0.550000in}%
\pgfsys@useobject{currentmarker}{}%
\end{pgfscope}%
\end{pgfscope}%
\begin{pgfscope}%
\definecolor{textcolor}{rgb}{0.000000,0.000000,0.000000}%
\pgfsetstrokecolor{textcolor}%
\pgfsetfillcolor{textcolor}%
\pgftext[x=2.978338in,y=0.452778in,,top]{\color{textcolor}{\sffamily\fontsize{10.000000}{12.000000}\selectfont\catcode`\^=\active\def^{\ifmmode\sp\else\^{}\fi}\catcode`\%=\active\def%{\%}50}}%
\end{pgfscope}%
\begin{pgfscope}%
\pgfsetbuttcap%
\pgfsetroundjoin%
\definecolor{currentfill}{rgb}{0.000000,0.000000,0.000000}%
\pgfsetfillcolor{currentfill}%
\pgfsetlinewidth{0.803000pt}%
\definecolor{currentstroke}{rgb}{0.000000,0.000000,0.000000}%
\pgfsetstrokecolor{currentstroke}%
\pgfsetdash{}{0pt}%
\pgfsys@defobject{currentmarker}{\pgfqpoint{0.000000in}{-0.048611in}}{\pgfqpoint{0.000000in}{0.000000in}}{%
\pgfpathmoveto{\pgfqpoint{0.000000in}{0.000000in}}%
\pgfpathlineto{\pgfqpoint{0.000000in}{-0.048611in}}%
\pgfusepath{stroke,fill}%
}%
\begin{pgfscope}%
\pgfsys@transformshift{4.354403in}{0.550000in}%
\pgfsys@useobject{currentmarker}{}%
\end{pgfscope}%
\end{pgfscope}%
\begin{pgfscope}%
\definecolor{textcolor}{rgb}{0.000000,0.000000,0.000000}%
\pgfsetstrokecolor{textcolor}%
\pgfsetfillcolor{textcolor}%
\pgftext[x=4.354403in,y=0.452778in,,top]{\color{textcolor}{\sffamily\fontsize{10.000000}{12.000000}\selectfont\catcode`\^=\active\def^{\ifmmode\sp\else\^{}\fi}\catcode`\%=\active\def%{\%}100}}%
\end{pgfscope}%
\begin{pgfscope}%
\pgfsetbuttcap%
\pgfsetroundjoin%
\definecolor{currentfill}{rgb}{0.000000,0.000000,0.000000}%
\pgfsetfillcolor{currentfill}%
\pgfsetlinewidth{0.803000pt}%
\definecolor{currentstroke}{rgb}{0.000000,0.000000,0.000000}%
\pgfsetstrokecolor{currentstroke}%
\pgfsetdash{}{0pt}%
\pgfsys@defobject{currentmarker}{\pgfqpoint{0.000000in}{-0.048611in}}{\pgfqpoint{0.000000in}{0.000000in}}{%
\pgfpathmoveto{\pgfqpoint{0.000000in}{0.000000in}}%
\pgfpathlineto{\pgfqpoint{0.000000in}{-0.048611in}}%
\pgfusepath{stroke,fill}%
}%
\begin{pgfscope}%
\pgfsys@transformshift{5.730469in}{0.550000in}%
\pgfsys@useobject{currentmarker}{}%
\end{pgfscope}%
\end{pgfscope}%
\begin{pgfscope}%
\definecolor{textcolor}{rgb}{0.000000,0.000000,0.000000}%
\pgfsetstrokecolor{textcolor}%
\pgfsetfillcolor{textcolor}%
\pgftext[x=5.730469in,y=0.452778in,,top]{\color{textcolor}{\sffamily\fontsize{10.000000}{12.000000}\selectfont\catcode`\^=\active\def^{\ifmmode\sp\else\^{}\fi}\catcode`\%=\active\def%{\%}150}}%
\end{pgfscope}%
\begin{pgfscope}%
\pgfsetbuttcap%
\pgfsetroundjoin%
\definecolor{currentfill}{rgb}{0.000000,0.000000,0.000000}%
\pgfsetfillcolor{currentfill}%
\pgfsetlinewidth{0.803000pt}%
\definecolor{currentstroke}{rgb}{0.000000,0.000000,0.000000}%
\pgfsetstrokecolor{currentstroke}%
\pgfsetdash{}{0pt}%
\pgfsys@defobject{currentmarker}{\pgfqpoint{0.000000in}{-0.048611in}}{\pgfqpoint{0.000000in}{0.000000in}}{%
\pgfpathmoveto{\pgfqpoint{0.000000in}{0.000000in}}%
\pgfpathlineto{\pgfqpoint{0.000000in}{-0.048611in}}%
\pgfusepath{stroke,fill}%
}%
\begin{pgfscope}%
\pgfsys@transformshift{7.106534in}{0.550000in}%
\pgfsys@useobject{currentmarker}{}%
\end{pgfscope}%
\end{pgfscope}%
\begin{pgfscope}%
\definecolor{textcolor}{rgb}{0.000000,0.000000,0.000000}%
\pgfsetstrokecolor{textcolor}%
\pgfsetfillcolor{textcolor}%
\pgftext[x=7.106534in,y=0.452778in,,top]{\color{textcolor}{\sffamily\fontsize{10.000000}{12.000000}\selectfont\catcode`\^=\active\def^{\ifmmode\sp\else\^{}\fi}\catcode`\%=\active\def%{\%}200}}%
\end{pgfscope}%
\begin{pgfscope}%
\pgfsetbuttcap%
\pgfsetroundjoin%
\definecolor{currentfill}{rgb}{0.000000,0.000000,0.000000}%
\pgfsetfillcolor{currentfill}%
\pgfsetlinewidth{0.803000pt}%
\definecolor{currentstroke}{rgb}{0.000000,0.000000,0.000000}%
\pgfsetstrokecolor{currentstroke}%
\pgfsetdash{}{0pt}%
\pgfsys@defobject{currentmarker}{\pgfqpoint{0.000000in}{-0.048611in}}{\pgfqpoint{0.000000in}{0.000000in}}{%
\pgfpathmoveto{\pgfqpoint{0.000000in}{0.000000in}}%
\pgfpathlineto{\pgfqpoint{0.000000in}{-0.048611in}}%
\pgfusepath{stroke,fill}%
}%
\begin{pgfscope}%
\pgfsys@transformshift{8.482599in}{0.550000in}%
\pgfsys@useobject{currentmarker}{}%
\end{pgfscope}%
\end{pgfscope}%
\begin{pgfscope}%
\definecolor{textcolor}{rgb}{0.000000,0.000000,0.000000}%
\pgfsetstrokecolor{textcolor}%
\pgfsetfillcolor{textcolor}%
\pgftext[x=8.482599in,y=0.452778in,,top]{\color{textcolor}{\sffamily\fontsize{10.000000}{12.000000}\selectfont\catcode`\^=\active\def^{\ifmmode\sp\else\^{}\fi}\catcode`\%=\active\def%{\%}250}}%
\end{pgfscope}%
\begin{pgfscope}%
\definecolor{textcolor}{rgb}{0.000000,0.000000,0.000000}%
\pgfsetstrokecolor{textcolor}%
\pgfsetfillcolor{textcolor}%
\pgftext[x=5.125000in,y=0.262809in,,top]{\color{textcolor}{\sffamily\fontsize{10.000000}{12.000000}\selectfont\catcode`\^=\active\def^{\ifmmode\sp\else\^{}\fi}\catcode`\%=\active\def%{\%}Intensidad}}%
\end{pgfscope}%
\begin{pgfscope}%
\pgfsetbuttcap%
\pgfsetroundjoin%
\definecolor{currentfill}{rgb}{0.000000,0.000000,0.000000}%
\pgfsetfillcolor{currentfill}%
\pgfsetlinewidth{0.803000pt}%
\definecolor{currentstroke}{rgb}{0.000000,0.000000,0.000000}%
\pgfsetstrokecolor{currentstroke}%
\pgfsetdash{}{0pt}%
\pgfsys@defobject{currentmarker}{\pgfqpoint{-0.048611in}{0.000000in}}{\pgfqpoint{-0.000000in}{0.000000in}}{%
\pgfpathmoveto{\pgfqpoint{-0.000000in}{0.000000in}}%
\pgfpathlineto{\pgfqpoint{-0.048611in}{0.000000in}}%
\pgfusepath{stroke,fill}%
}%
\begin{pgfscope}%
\pgfsys@transformshift{1.250000in}{0.550000in}%
\pgfsys@useobject{currentmarker}{}%
\end{pgfscope}%
\end{pgfscope}%
\begin{pgfscope}%
\definecolor{textcolor}{rgb}{0.000000,0.000000,0.000000}%
\pgfsetstrokecolor{textcolor}%
\pgfsetfillcolor{textcolor}%
\pgftext[x=1.064412in, y=0.497238in, left, base]{\color{textcolor}{\sffamily\fontsize{10.000000}{12.000000}\selectfont\catcode`\^=\active\def^{\ifmmode\sp\else\^{}\fi}\catcode`\%=\active\def%{\%}0}}%
\end{pgfscope}%
\begin{pgfscope}%
\pgfsetbuttcap%
\pgfsetroundjoin%
\definecolor{currentfill}{rgb}{0.000000,0.000000,0.000000}%
\pgfsetfillcolor{currentfill}%
\pgfsetlinewidth{0.803000pt}%
\definecolor{currentstroke}{rgb}{0.000000,0.000000,0.000000}%
\pgfsetstrokecolor{currentstroke}%
\pgfsetdash{}{0pt}%
\pgfsys@defobject{currentmarker}{\pgfqpoint{-0.048611in}{0.000000in}}{\pgfqpoint{-0.000000in}{0.000000in}}{%
\pgfpathmoveto{\pgfqpoint{-0.000000in}{0.000000in}}%
\pgfpathlineto{\pgfqpoint{-0.048611in}{0.000000in}}%
\pgfusepath{stroke,fill}%
}%
\begin{pgfscope}%
\pgfsys@transformshift{1.250000in}{1.100633in}%
\pgfsys@useobject{currentmarker}{}%
\end{pgfscope}%
\end{pgfscope}%
\begin{pgfscope}%
\definecolor{textcolor}{rgb}{0.000000,0.000000,0.000000}%
\pgfsetstrokecolor{textcolor}%
\pgfsetfillcolor{textcolor}%
\pgftext[x=0.799316in, y=1.047872in, left, base]{\color{textcolor}{\sffamily\fontsize{10.000000}{12.000000}\selectfont\catcode`\^=\active\def^{\ifmmode\sp\else\^{}\fi}\catcode`\%=\active\def%{\%}2000}}%
\end{pgfscope}%
\begin{pgfscope}%
\pgfsetbuttcap%
\pgfsetroundjoin%
\definecolor{currentfill}{rgb}{0.000000,0.000000,0.000000}%
\pgfsetfillcolor{currentfill}%
\pgfsetlinewidth{0.803000pt}%
\definecolor{currentstroke}{rgb}{0.000000,0.000000,0.000000}%
\pgfsetstrokecolor{currentstroke}%
\pgfsetdash{}{0pt}%
\pgfsys@defobject{currentmarker}{\pgfqpoint{-0.048611in}{0.000000in}}{\pgfqpoint{-0.000000in}{0.000000in}}{%
\pgfpathmoveto{\pgfqpoint{-0.000000in}{0.000000in}}%
\pgfpathlineto{\pgfqpoint{-0.048611in}{0.000000in}}%
\pgfusepath{stroke,fill}%
}%
\begin{pgfscope}%
\pgfsys@transformshift{1.250000in}{1.651266in}%
\pgfsys@useobject{currentmarker}{}%
\end{pgfscope}%
\end{pgfscope}%
\begin{pgfscope}%
\definecolor{textcolor}{rgb}{0.000000,0.000000,0.000000}%
\pgfsetstrokecolor{textcolor}%
\pgfsetfillcolor{textcolor}%
\pgftext[x=0.799316in, y=1.598505in, left, base]{\color{textcolor}{\sffamily\fontsize{10.000000}{12.000000}\selectfont\catcode`\^=\active\def^{\ifmmode\sp\else\^{}\fi}\catcode`\%=\active\def%{\%}4000}}%
\end{pgfscope}%
\begin{pgfscope}%
\pgfsetbuttcap%
\pgfsetroundjoin%
\definecolor{currentfill}{rgb}{0.000000,0.000000,0.000000}%
\pgfsetfillcolor{currentfill}%
\pgfsetlinewidth{0.803000pt}%
\definecolor{currentstroke}{rgb}{0.000000,0.000000,0.000000}%
\pgfsetstrokecolor{currentstroke}%
\pgfsetdash{}{0pt}%
\pgfsys@defobject{currentmarker}{\pgfqpoint{-0.048611in}{0.000000in}}{\pgfqpoint{-0.000000in}{0.000000in}}{%
\pgfpathmoveto{\pgfqpoint{-0.000000in}{0.000000in}}%
\pgfpathlineto{\pgfqpoint{-0.048611in}{0.000000in}}%
\pgfusepath{stroke,fill}%
}%
\begin{pgfscope}%
\pgfsys@transformshift{1.250000in}{2.201900in}%
\pgfsys@useobject{currentmarker}{}%
\end{pgfscope}%
\end{pgfscope}%
\begin{pgfscope}%
\definecolor{textcolor}{rgb}{0.000000,0.000000,0.000000}%
\pgfsetstrokecolor{textcolor}%
\pgfsetfillcolor{textcolor}%
\pgftext[x=0.799316in, y=2.149138in, left, base]{\color{textcolor}{\sffamily\fontsize{10.000000}{12.000000}\selectfont\catcode`\^=\active\def^{\ifmmode\sp\else\^{}\fi}\catcode`\%=\active\def%{\%}6000}}%
\end{pgfscope}%
\begin{pgfscope}%
\pgfsetbuttcap%
\pgfsetroundjoin%
\definecolor{currentfill}{rgb}{0.000000,0.000000,0.000000}%
\pgfsetfillcolor{currentfill}%
\pgfsetlinewidth{0.803000pt}%
\definecolor{currentstroke}{rgb}{0.000000,0.000000,0.000000}%
\pgfsetstrokecolor{currentstroke}%
\pgfsetdash{}{0pt}%
\pgfsys@defobject{currentmarker}{\pgfqpoint{-0.048611in}{0.000000in}}{\pgfqpoint{-0.000000in}{0.000000in}}{%
\pgfpathmoveto{\pgfqpoint{-0.000000in}{0.000000in}}%
\pgfpathlineto{\pgfqpoint{-0.048611in}{0.000000in}}%
\pgfusepath{stroke,fill}%
}%
\begin{pgfscope}%
\pgfsys@transformshift{1.250000in}{2.752533in}%
\pgfsys@useobject{currentmarker}{}%
\end{pgfscope}%
\end{pgfscope}%
\begin{pgfscope}%
\definecolor{textcolor}{rgb}{0.000000,0.000000,0.000000}%
\pgfsetstrokecolor{textcolor}%
\pgfsetfillcolor{textcolor}%
\pgftext[x=0.799316in, y=2.699771in, left, base]{\color{textcolor}{\sffamily\fontsize{10.000000}{12.000000}\selectfont\catcode`\^=\active\def^{\ifmmode\sp\else\^{}\fi}\catcode`\%=\active\def%{\%}8000}}%
\end{pgfscope}%
\begin{pgfscope}%
\pgfsetbuttcap%
\pgfsetroundjoin%
\definecolor{currentfill}{rgb}{0.000000,0.000000,0.000000}%
\pgfsetfillcolor{currentfill}%
\pgfsetlinewidth{0.803000pt}%
\definecolor{currentstroke}{rgb}{0.000000,0.000000,0.000000}%
\pgfsetstrokecolor{currentstroke}%
\pgfsetdash{}{0pt}%
\pgfsys@defobject{currentmarker}{\pgfqpoint{-0.048611in}{0.000000in}}{\pgfqpoint{-0.000000in}{0.000000in}}{%
\pgfpathmoveto{\pgfqpoint{-0.000000in}{0.000000in}}%
\pgfpathlineto{\pgfqpoint{-0.048611in}{0.000000in}}%
\pgfusepath{stroke,fill}%
}%
\begin{pgfscope}%
\pgfsys@transformshift{1.250000in}{3.303166in}%
\pgfsys@useobject{currentmarker}{}%
\end{pgfscope}%
\end{pgfscope}%
\begin{pgfscope}%
\definecolor{textcolor}{rgb}{0.000000,0.000000,0.000000}%
\pgfsetstrokecolor{textcolor}%
\pgfsetfillcolor{textcolor}%
\pgftext[x=0.710951in, y=3.250405in, left, base]{\color{textcolor}{\sffamily\fontsize{10.000000}{12.000000}\selectfont\catcode`\^=\active\def^{\ifmmode\sp\else\^{}\fi}\catcode`\%=\active\def%{\%}10000}}%
\end{pgfscope}%
\begin{pgfscope}%
\pgfsetbuttcap%
\pgfsetroundjoin%
\definecolor{currentfill}{rgb}{0.000000,0.000000,0.000000}%
\pgfsetfillcolor{currentfill}%
\pgfsetlinewidth{0.803000pt}%
\definecolor{currentstroke}{rgb}{0.000000,0.000000,0.000000}%
\pgfsetstrokecolor{currentstroke}%
\pgfsetdash{}{0pt}%
\pgfsys@defobject{currentmarker}{\pgfqpoint{-0.048611in}{0.000000in}}{\pgfqpoint{-0.000000in}{0.000000in}}{%
\pgfpathmoveto{\pgfqpoint{-0.000000in}{0.000000in}}%
\pgfpathlineto{\pgfqpoint{-0.048611in}{0.000000in}}%
\pgfusepath{stroke,fill}%
}%
\begin{pgfscope}%
\pgfsys@transformshift{1.250000in}{3.853799in}%
\pgfsys@useobject{currentmarker}{}%
\end{pgfscope}%
\end{pgfscope}%
\begin{pgfscope}%
\definecolor{textcolor}{rgb}{0.000000,0.000000,0.000000}%
\pgfsetstrokecolor{textcolor}%
\pgfsetfillcolor{textcolor}%
\pgftext[x=0.710951in, y=3.801038in, left, base]{\color{textcolor}{\sffamily\fontsize{10.000000}{12.000000}\selectfont\catcode`\^=\active\def^{\ifmmode\sp\else\^{}\fi}\catcode`\%=\active\def%{\%}12000}}%
\end{pgfscope}%
\begin{pgfscope}%
\definecolor{textcolor}{rgb}{0.000000,0.000000,0.000000}%
\pgfsetstrokecolor{textcolor}%
\pgfsetfillcolor{textcolor}%
\pgftext[x=0.655396in,y=2.475000in,,bottom,rotate=90.000000]{\color{textcolor}{\sffamily\fontsize{10.000000}{12.000000}\selectfont\catcode`\^=\active\def^{\ifmmode\sp\else\^{}\fi}\catcode`\%=\active\def%{\%}Frecuencia}}%
\end{pgfscope}%
\begin{pgfscope}%
\pgfsetrectcap%
\pgfsetmiterjoin%
\pgfsetlinewidth{0.803000pt}%
\definecolor{currentstroke}{rgb}{0.000000,0.000000,0.000000}%
\pgfsetstrokecolor{currentstroke}%
\pgfsetdash{}{0pt}%
\pgfpathmoveto{\pgfqpoint{1.250000in}{0.550000in}}%
\pgfpathlineto{\pgfqpoint{1.250000in}{4.400000in}}%
\pgfusepath{stroke}%
\end{pgfscope}%
\begin{pgfscope}%
\pgfsetrectcap%
\pgfsetmiterjoin%
\pgfsetlinewidth{0.803000pt}%
\definecolor{currentstroke}{rgb}{0.000000,0.000000,0.000000}%
\pgfsetstrokecolor{currentstroke}%
\pgfsetdash{}{0pt}%
\pgfpathmoveto{\pgfqpoint{9.000000in}{0.550000in}}%
\pgfpathlineto{\pgfqpoint{9.000000in}{4.400000in}}%
\pgfusepath{stroke}%
\end{pgfscope}%
\begin{pgfscope}%
\pgfsetrectcap%
\pgfsetmiterjoin%
\pgfsetlinewidth{0.803000pt}%
\definecolor{currentstroke}{rgb}{0.000000,0.000000,0.000000}%
\pgfsetstrokecolor{currentstroke}%
\pgfsetdash{}{0pt}%
\pgfpathmoveto{\pgfqpoint{1.250000in}{0.550000in}}%
\pgfpathlineto{\pgfqpoint{9.000000in}{0.550000in}}%
\pgfusepath{stroke}%
\end{pgfscope}%
\begin{pgfscope}%
\pgfsetrectcap%
\pgfsetmiterjoin%
\pgfsetlinewidth{0.803000pt}%
\definecolor{currentstroke}{rgb}{0.000000,0.000000,0.000000}%
\pgfsetstrokecolor{currentstroke}%
\pgfsetdash{}{0pt}%
\pgfpathmoveto{\pgfqpoint{1.250000in}{4.400000in}}%
\pgfpathlineto{\pgfqpoint{9.000000in}{4.400000in}}%
\pgfusepath{stroke}%
\end{pgfscope}%
\begin{pgfscope}%
\definecolor{textcolor}{rgb}{0.000000,0.000000,0.000000}%
\pgfsetstrokecolor{textcolor}%
\pgfsetfillcolor{textcolor}%
\pgftext[x=5.125000in,y=4.483333in,,base]{\color{textcolor}{\sffamily\fontsize{12.000000}{14.400000}\selectfont\catcode`\^=\active\def^{\ifmmode\sp\else\^{}\fi}\catcode`\%=\active\def%{\%}Histograma de Original}}%
\end{pgfscope}%
\end{pgfpicture}%
\makeatother%
\endgroup%
}
		\caption{Original}
		\label{fig:original_3}
	\end{subfigure}
	\hfill
	\begin{subfigure}[b]{0.45\textwidth}
		\resizebox{\linewidth}{!}{%% Creator: Matplotlib, PGF backend
%%
%% To include the figure in your LaTeX document, write
%%   \input{<filename>.pgf}
%%
%% Make sure the required packages are loaded in your preamble
%%   \usepackage{pgf}
%%
%% Also ensure that all the required font packages are loaded; for instance,
%% the lmodern package is sometimes necessary when using math font.
%%   \usepackage{lmodern}
%%
%% Figures using additional raster images can only be included by \input if
%% they are in the same directory as the main LaTeX file. For loading figures
%% from other directories you can use the `import` package
%%   \usepackage{import}
%%
%% and then include the figures with
%%   \import{<path to file>}{<filename>.pgf}
%%
%% Matplotlib used the following preamble
%%   \def\mathdefault#1{#1}
%%   \everymath=\expandafter{\the\everymath\displaystyle}
%%   \IfFileExists{scrextend.sty}{
%%     \usepackage[fontsize=10.000000pt]{scrextend}
%%   }{
%%     \renewcommand{\normalsize}{\fontsize{10.000000}{12.000000}\selectfont}
%%     \normalsize
%%   }
%%   
%%   \ifdefined\pdftexversion\else  % non-pdftex case.
%%     \usepackage{fontspec}
%%     \setmainfont{DejaVuSerif.ttf}[Path=\detokenize{/home/elias/.virtualenvs/src-pway/lib/python3.13/site-packages/matplotlib/mpl-data/fonts/ttf/}]
%%     \setsansfont{DejaVuSans.ttf}[Path=\detokenize{/home/elias/.virtualenvs/src-pway/lib/python3.13/site-packages/matplotlib/mpl-data/fonts/ttf/}]
%%     \setmonofont{DejaVuSansMono.ttf}[Path=\detokenize{/home/elias/.virtualenvs/src-pway/lib/python3.13/site-packages/matplotlib/mpl-data/fonts/ttf/}]
%%   \fi
%%   \makeatletter\@ifpackageloaded{underscore}{}{\usepackage[strings]{underscore}}\makeatother
%%
\begingroup%
\makeatletter%
\begin{pgfpicture}%
\pgfpathrectangle{\pgfpointorigin}{\pgfqpoint{10.000000in}{5.000000in}}%
\pgfusepath{use as bounding box, clip}%
\begin{pgfscope}%
\pgfsetbuttcap%
\pgfsetmiterjoin%
\definecolor{currentfill}{rgb}{1.000000,1.000000,1.000000}%
\pgfsetfillcolor{currentfill}%
\pgfsetlinewidth{0.000000pt}%
\definecolor{currentstroke}{rgb}{1.000000,1.000000,1.000000}%
\pgfsetstrokecolor{currentstroke}%
\pgfsetdash{}{0pt}%
\pgfpathmoveto{\pgfqpoint{0.000000in}{0.000000in}}%
\pgfpathlineto{\pgfqpoint{10.000000in}{0.000000in}}%
\pgfpathlineto{\pgfqpoint{10.000000in}{5.000000in}}%
\pgfpathlineto{\pgfqpoint{0.000000in}{5.000000in}}%
\pgfpathlineto{\pgfqpoint{0.000000in}{0.000000in}}%
\pgfpathclose%
\pgfusepath{fill}%
\end{pgfscope}%
\begin{pgfscope}%
\pgfsetbuttcap%
\pgfsetmiterjoin%
\definecolor{currentfill}{rgb}{1.000000,1.000000,1.000000}%
\pgfsetfillcolor{currentfill}%
\pgfsetlinewidth{0.000000pt}%
\definecolor{currentstroke}{rgb}{0.000000,0.000000,0.000000}%
\pgfsetstrokecolor{currentstroke}%
\pgfsetstrokeopacity{0.000000}%
\pgfsetdash{}{0pt}%
\pgfpathmoveto{\pgfqpoint{1.250000in}{0.550000in}}%
\pgfpathlineto{\pgfqpoint{9.000000in}{0.550000in}}%
\pgfpathlineto{\pgfqpoint{9.000000in}{4.400000in}}%
\pgfpathlineto{\pgfqpoint{1.250000in}{4.400000in}}%
\pgfpathlineto{\pgfqpoint{1.250000in}{0.550000in}}%
\pgfpathclose%
\pgfusepath{fill}%
\end{pgfscope}%
\begin{pgfscope}%
\pgfpathrectangle{\pgfqpoint{1.250000in}{0.550000in}}{\pgfqpoint{7.750000in}{3.850000in}}%
\pgfusepath{clip}%
\pgfsetbuttcap%
\pgfsetmiterjoin%
\definecolor{currentfill}{rgb}{0.501961,0.501961,0.501961}%
\pgfsetfillcolor{currentfill}%
\pgfsetfillopacity{0.700000}%
\pgfsetlinewidth{0.000000pt}%
\definecolor{currentstroke}{rgb}{0.000000,0.000000,0.000000}%
\pgfsetstrokecolor{currentstroke}%
\pgfsetstrokeopacity{0.700000}%
\pgfsetdash{}{0pt}%
\pgfpathmoveto{\pgfqpoint{1.602273in}{0.550000in}}%
\pgfpathlineto{\pgfqpoint{1.629794in}{0.550000in}}%
\pgfpathlineto{\pgfqpoint{1.629794in}{1.041835in}}%
\pgfpathlineto{\pgfqpoint{1.602273in}{1.041835in}}%
\pgfpathlineto{\pgfqpoint{1.602273in}{0.550000in}}%
\pgfpathclose%
\pgfusepath{fill}%
\end{pgfscope}%
\begin{pgfscope}%
\pgfpathrectangle{\pgfqpoint{1.250000in}{0.550000in}}{\pgfqpoint{7.750000in}{3.850000in}}%
\pgfusepath{clip}%
\pgfsetbuttcap%
\pgfsetmiterjoin%
\definecolor{currentfill}{rgb}{0.501961,0.501961,0.501961}%
\pgfsetfillcolor{currentfill}%
\pgfsetfillopacity{0.700000}%
\pgfsetlinewidth{0.000000pt}%
\definecolor{currentstroke}{rgb}{0.000000,0.000000,0.000000}%
\pgfsetstrokecolor{currentstroke}%
\pgfsetstrokeopacity{0.700000}%
\pgfsetdash{}{0pt}%
\pgfpathmoveto{\pgfqpoint{1.629794in}{0.550000in}}%
\pgfpathlineto{\pgfqpoint{1.657315in}{0.550000in}}%
\pgfpathlineto{\pgfqpoint{1.657315in}{2.069699in}}%
\pgfpathlineto{\pgfqpoint{1.629794in}{2.069699in}}%
\pgfpathlineto{\pgfqpoint{1.629794in}{0.550000in}}%
\pgfpathclose%
\pgfusepath{fill}%
\end{pgfscope}%
\begin{pgfscope}%
\pgfpathrectangle{\pgfqpoint{1.250000in}{0.550000in}}{\pgfqpoint{7.750000in}{3.850000in}}%
\pgfusepath{clip}%
\pgfsetbuttcap%
\pgfsetmiterjoin%
\definecolor{currentfill}{rgb}{0.501961,0.501961,0.501961}%
\pgfsetfillcolor{currentfill}%
\pgfsetfillopacity{0.700000}%
\pgfsetlinewidth{0.000000pt}%
\definecolor{currentstroke}{rgb}{0.000000,0.000000,0.000000}%
\pgfsetstrokecolor{currentstroke}%
\pgfsetstrokeopacity{0.700000}%
\pgfsetdash{}{0pt}%
\pgfpathmoveto{\pgfqpoint{1.657315in}{0.550000in}}%
\pgfpathlineto{\pgfqpoint{1.684837in}{0.550000in}}%
\pgfpathlineto{\pgfqpoint{1.684837in}{1.567885in}}%
\pgfpathlineto{\pgfqpoint{1.657315in}{1.567885in}}%
\pgfpathlineto{\pgfqpoint{1.657315in}{0.550000in}}%
\pgfpathclose%
\pgfusepath{fill}%
\end{pgfscope}%
\begin{pgfscope}%
\pgfpathrectangle{\pgfqpoint{1.250000in}{0.550000in}}{\pgfqpoint{7.750000in}{3.850000in}}%
\pgfusepath{clip}%
\pgfsetbuttcap%
\pgfsetmiterjoin%
\definecolor{currentfill}{rgb}{0.501961,0.501961,0.501961}%
\pgfsetfillcolor{currentfill}%
\pgfsetfillopacity{0.700000}%
\pgfsetlinewidth{0.000000pt}%
\definecolor{currentstroke}{rgb}{0.000000,0.000000,0.000000}%
\pgfsetstrokecolor{currentstroke}%
\pgfsetstrokeopacity{0.700000}%
\pgfsetdash{}{0pt}%
\pgfpathmoveto{\pgfqpoint{1.684837in}{0.550000in}}%
\pgfpathlineto{\pgfqpoint{1.712358in}{0.550000in}}%
\pgfpathlineto{\pgfqpoint{1.712358in}{1.999844in}}%
\pgfpathlineto{\pgfqpoint{1.684837in}{1.999844in}}%
\pgfpathlineto{\pgfqpoint{1.684837in}{0.550000in}}%
\pgfpathclose%
\pgfusepath{fill}%
\end{pgfscope}%
\begin{pgfscope}%
\pgfpathrectangle{\pgfqpoint{1.250000in}{0.550000in}}{\pgfqpoint{7.750000in}{3.850000in}}%
\pgfusepath{clip}%
\pgfsetbuttcap%
\pgfsetmiterjoin%
\definecolor{currentfill}{rgb}{0.501961,0.501961,0.501961}%
\pgfsetfillcolor{currentfill}%
\pgfsetfillopacity{0.700000}%
\pgfsetlinewidth{0.000000pt}%
\definecolor{currentstroke}{rgb}{0.000000,0.000000,0.000000}%
\pgfsetstrokecolor{currentstroke}%
\pgfsetstrokeopacity{0.700000}%
\pgfsetdash{}{0pt}%
\pgfpathmoveto{\pgfqpoint{1.712358in}{0.550000in}}%
\pgfpathlineto{\pgfqpoint{1.739879in}{0.550000in}}%
\pgfpathlineto{\pgfqpoint{1.739879in}{2.326309in}}%
\pgfpathlineto{\pgfqpoint{1.712358in}{2.326309in}}%
\pgfpathlineto{\pgfqpoint{1.712358in}{0.550000in}}%
\pgfpathclose%
\pgfusepath{fill}%
\end{pgfscope}%
\begin{pgfscope}%
\pgfpathrectangle{\pgfqpoint{1.250000in}{0.550000in}}{\pgfqpoint{7.750000in}{3.850000in}}%
\pgfusepath{clip}%
\pgfsetbuttcap%
\pgfsetmiterjoin%
\definecolor{currentfill}{rgb}{0.501961,0.501961,0.501961}%
\pgfsetfillcolor{currentfill}%
\pgfsetfillopacity{0.700000}%
\pgfsetlinewidth{0.000000pt}%
\definecolor{currentstroke}{rgb}{0.000000,0.000000,0.000000}%
\pgfsetstrokecolor{currentstroke}%
\pgfsetstrokeopacity{0.700000}%
\pgfsetdash{}{0pt}%
\pgfpathmoveto{\pgfqpoint{1.739879in}{0.550000in}}%
\pgfpathlineto{\pgfqpoint{1.767401in}{0.550000in}}%
\pgfpathlineto{\pgfqpoint{1.767401in}{2.290669in}}%
\pgfpathlineto{\pgfqpoint{1.739879in}{2.290669in}}%
\pgfpathlineto{\pgfqpoint{1.739879in}{0.550000in}}%
\pgfpathclose%
\pgfusepath{fill}%
\end{pgfscope}%
\begin{pgfscope}%
\pgfpathrectangle{\pgfqpoint{1.250000in}{0.550000in}}{\pgfqpoint{7.750000in}{3.850000in}}%
\pgfusepath{clip}%
\pgfsetbuttcap%
\pgfsetmiterjoin%
\definecolor{currentfill}{rgb}{0.501961,0.501961,0.501961}%
\pgfsetfillcolor{currentfill}%
\pgfsetfillopacity{0.700000}%
\pgfsetlinewidth{0.000000pt}%
\definecolor{currentstroke}{rgb}{0.000000,0.000000,0.000000}%
\pgfsetstrokecolor{currentstroke}%
\pgfsetstrokeopacity{0.700000}%
\pgfsetdash{}{0pt}%
\pgfpathmoveto{\pgfqpoint{1.767401in}{0.550000in}}%
\pgfpathlineto{\pgfqpoint{1.794922in}{0.550000in}}%
\pgfpathlineto{\pgfqpoint{1.794922in}{2.237921in}}%
\pgfpathlineto{\pgfqpoint{1.767401in}{2.237921in}}%
\pgfpathlineto{\pgfqpoint{1.767401in}{0.550000in}}%
\pgfpathclose%
\pgfusepath{fill}%
\end{pgfscope}%
\begin{pgfscope}%
\pgfpathrectangle{\pgfqpoint{1.250000in}{0.550000in}}{\pgfqpoint{7.750000in}{3.850000in}}%
\pgfusepath{clip}%
\pgfsetbuttcap%
\pgfsetmiterjoin%
\definecolor{currentfill}{rgb}{0.501961,0.501961,0.501961}%
\pgfsetfillcolor{currentfill}%
\pgfsetfillopacity{0.700000}%
\pgfsetlinewidth{0.000000pt}%
\definecolor{currentstroke}{rgb}{0.000000,0.000000,0.000000}%
\pgfsetstrokecolor{currentstroke}%
\pgfsetstrokeopacity{0.700000}%
\pgfsetdash{}{0pt}%
\pgfpathmoveto{\pgfqpoint{1.794922in}{0.550000in}}%
\pgfpathlineto{\pgfqpoint{1.822443in}{0.550000in}}%
\pgfpathlineto{\pgfqpoint{1.822443in}{2.269285in}}%
\pgfpathlineto{\pgfqpoint{1.794922in}{2.269285in}}%
\pgfpathlineto{\pgfqpoint{1.794922in}{0.550000in}}%
\pgfpathclose%
\pgfusepath{fill}%
\end{pgfscope}%
\begin{pgfscope}%
\pgfpathrectangle{\pgfqpoint{1.250000in}{0.550000in}}{\pgfqpoint{7.750000in}{3.850000in}}%
\pgfusepath{clip}%
\pgfsetbuttcap%
\pgfsetmiterjoin%
\definecolor{currentfill}{rgb}{0.501961,0.501961,0.501961}%
\pgfsetfillcolor{currentfill}%
\pgfsetfillopacity{0.700000}%
\pgfsetlinewidth{0.000000pt}%
\definecolor{currentstroke}{rgb}{0.000000,0.000000,0.000000}%
\pgfsetstrokecolor{currentstroke}%
\pgfsetstrokeopacity{0.700000}%
\pgfsetdash{}{0pt}%
\pgfpathmoveto{\pgfqpoint{1.822443in}{0.550000in}}%
\pgfpathlineto{\pgfqpoint{1.849964in}{0.550000in}}%
\pgfpathlineto{\pgfqpoint{1.849964in}{2.058294in}}%
\pgfpathlineto{\pgfqpoint{1.822443in}{2.058294in}}%
\pgfpathlineto{\pgfqpoint{1.822443in}{0.550000in}}%
\pgfpathclose%
\pgfusepath{fill}%
\end{pgfscope}%
\begin{pgfscope}%
\pgfpathrectangle{\pgfqpoint{1.250000in}{0.550000in}}{\pgfqpoint{7.750000in}{3.850000in}}%
\pgfusepath{clip}%
\pgfsetbuttcap%
\pgfsetmiterjoin%
\definecolor{currentfill}{rgb}{0.501961,0.501961,0.501961}%
\pgfsetfillcolor{currentfill}%
\pgfsetfillopacity{0.700000}%
\pgfsetlinewidth{0.000000pt}%
\definecolor{currentstroke}{rgb}{0.000000,0.000000,0.000000}%
\pgfsetstrokecolor{currentstroke}%
\pgfsetstrokeopacity{0.700000}%
\pgfsetdash{}{0pt}%
\pgfpathmoveto{\pgfqpoint{1.849964in}{0.550000in}}%
\pgfpathlineto{\pgfqpoint{1.877486in}{0.550000in}}%
\pgfpathlineto{\pgfqpoint{1.877486in}{2.049741in}}%
\pgfpathlineto{\pgfqpoint{1.849964in}{2.049741in}}%
\pgfpathlineto{\pgfqpoint{1.849964in}{0.550000in}}%
\pgfpathclose%
\pgfusepath{fill}%
\end{pgfscope}%
\begin{pgfscope}%
\pgfpathrectangle{\pgfqpoint{1.250000in}{0.550000in}}{\pgfqpoint{7.750000in}{3.850000in}}%
\pgfusepath{clip}%
\pgfsetbuttcap%
\pgfsetmiterjoin%
\definecolor{currentfill}{rgb}{0.501961,0.501961,0.501961}%
\pgfsetfillcolor{currentfill}%
\pgfsetfillopacity{0.700000}%
\pgfsetlinewidth{0.000000pt}%
\definecolor{currentstroke}{rgb}{0.000000,0.000000,0.000000}%
\pgfsetstrokecolor{currentstroke}%
\pgfsetstrokeopacity{0.700000}%
\pgfsetdash{}{0pt}%
\pgfpathmoveto{\pgfqpoint{1.877486in}{0.550000in}}%
\pgfpathlineto{\pgfqpoint{1.905007in}{0.550000in}}%
\pgfpathlineto{\pgfqpoint{1.905007in}{2.001270in}}%
\pgfpathlineto{\pgfqpoint{1.877486in}{2.001270in}}%
\pgfpathlineto{\pgfqpoint{1.877486in}{0.550000in}}%
\pgfpathclose%
\pgfusepath{fill}%
\end{pgfscope}%
\begin{pgfscope}%
\pgfpathrectangle{\pgfqpoint{1.250000in}{0.550000in}}{\pgfqpoint{7.750000in}{3.850000in}}%
\pgfusepath{clip}%
\pgfsetbuttcap%
\pgfsetmiterjoin%
\definecolor{currentfill}{rgb}{0.501961,0.501961,0.501961}%
\pgfsetfillcolor{currentfill}%
\pgfsetfillopacity{0.700000}%
\pgfsetlinewidth{0.000000pt}%
\definecolor{currentstroke}{rgb}{0.000000,0.000000,0.000000}%
\pgfsetstrokecolor{currentstroke}%
\pgfsetstrokeopacity{0.700000}%
\pgfsetdash{}{0pt}%
\pgfpathmoveto{\pgfqpoint{1.905007in}{0.550000in}}%
\pgfpathlineto{\pgfqpoint{1.932528in}{0.550000in}}%
\pgfpathlineto{\pgfqpoint{1.932528in}{2.109616in}}%
\pgfpathlineto{\pgfqpoint{1.905007in}{2.109616in}}%
\pgfpathlineto{\pgfqpoint{1.905007in}{0.550000in}}%
\pgfpathclose%
\pgfusepath{fill}%
\end{pgfscope}%
\begin{pgfscope}%
\pgfpathrectangle{\pgfqpoint{1.250000in}{0.550000in}}{\pgfqpoint{7.750000in}{3.850000in}}%
\pgfusepath{clip}%
\pgfsetbuttcap%
\pgfsetmiterjoin%
\definecolor{currentfill}{rgb}{0.501961,0.501961,0.501961}%
\pgfsetfillcolor{currentfill}%
\pgfsetfillopacity{0.700000}%
\pgfsetlinewidth{0.000000pt}%
\definecolor{currentstroke}{rgb}{0.000000,0.000000,0.000000}%
\pgfsetstrokecolor{currentstroke}%
\pgfsetstrokeopacity{0.700000}%
\pgfsetdash{}{0pt}%
\pgfpathmoveto{\pgfqpoint{1.932528in}{0.550000in}}%
\pgfpathlineto{\pgfqpoint{1.960050in}{0.550000in}}%
\pgfpathlineto{\pgfqpoint{1.960050in}{2.207983in}}%
\pgfpathlineto{\pgfqpoint{1.932528in}{2.207983in}}%
\pgfpathlineto{\pgfqpoint{1.932528in}{0.550000in}}%
\pgfpathclose%
\pgfusepath{fill}%
\end{pgfscope}%
\begin{pgfscope}%
\pgfpathrectangle{\pgfqpoint{1.250000in}{0.550000in}}{\pgfqpoint{7.750000in}{3.850000in}}%
\pgfusepath{clip}%
\pgfsetbuttcap%
\pgfsetmiterjoin%
\definecolor{currentfill}{rgb}{0.501961,0.501961,0.501961}%
\pgfsetfillcolor{currentfill}%
\pgfsetfillopacity{0.700000}%
\pgfsetlinewidth{0.000000pt}%
\definecolor{currentstroke}{rgb}{0.000000,0.000000,0.000000}%
\pgfsetstrokecolor{currentstroke}%
\pgfsetstrokeopacity{0.700000}%
\pgfsetdash{}{0pt}%
\pgfpathmoveto{\pgfqpoint{1.960050in}{0.550000in}}%
\pgfpathlineto{\pgfqpoint{1.987571in}{0.550000in}}%
\pgfpathlineto{\pgfqpoint{1.987571in}{2.302074in}}%
\pgfpathlineto{\pgfqpoint{1.960050in}{2.302074in}}%
\pgfpathlineto{\pgfqpoint{1.960050in}{0.550000in}}%
\pgfpathclose%
\pgfusepath{fill}%
\end{pgfscope}%
\begin{pgfscope}%
\pgfpathrectangle{\pgfqpoint{1.250000in}{0.550000in}}{\pgfqpoint{7.750000in}{3.850000in}}%
\pgfusepath{clip}%
\pgfsetbuttcap%
\pgfsetmiterjoin%
\definecolor{currentfill}{rgb}{0.501961,0.501961,0.501961}%
\pgfsetfillcolor{currentfill}%
\pgfsetfillopacity{0.700000}%
\pgfsetlinewidth{0.000000pt}%
\definecolor{currentstroke}{rgb}{0.000000,0.000000,0.000000}%
\pgfsetstrokecolor{currentstroke}%
\pgfsetstrokeopacity{0.700000}%
\pgfsetdash{}{0pt}%
\pgfpathmoveto{\pgfqpoint{1.987571in}{0.550000in}}%
\pgfpathlineto{\pgfqpoint{2.015092in}{0.550000in}}%
\pgfpathlineto{\pgfqpoint{2.015092in}{2.145257in}}%
\pgfpathlineto{\pgfqpoint{1.987571in}{2.145257in}}%
\pgfpathlineto{\pgfqpoint{1.987571in}{0.550000in}}%
\pgfpathclose%
\pgfusepath{fill}%
\end{pgfscope}%
\begin{pgfscope}%
\pgfpathrectangle{\pgfqpoint{1.250000in}{0.550000in}}{\pgfqpoint{7.750000in}{3.850000in}}%
\pgfusepath{clip}%
\pgfsetbuttcap%
\pgfsetmiterjoin%
\definecolor{currentfill}{rgb}{0.501961,0.501961,0.501961}%
\pgfsetfillcolor{currentfill}%
\pgfsetfillopacity{0.700000}%
\pgfsetlinewidth{0.000000pt}%
\definecolor{currentstroke}{rgb}{0.000000,0.000000,0.000000}%
\pgfsetstrokecolor{currentstroke}%
\pgfsetstrokeopacity{0.700000}%
\pgfsetdash{}{0pt}%
\pgfpathmoveto{\pgfqpoint{2.015092in}{0.550000in}}%
\pgfpathlineto{\pgfqpoint{2.042614in}{0.550000in}}%
\pgfpathlineto{\pgfqpoint{2.042614in}{2.188025in}}%
\pgfpathlineto{\pgfqpoint{2.015092in}{2.188025in}}%
\pgfpathlineto{\pgfqpoint{2.015092in}{0.550000in}}%
\pgfpathclose%
\pgfusepath{fill}%
\end{pgfscope}%
\begin{pgfscope}%
\pgfpathrectangle{\pgfqpoint{1.250000in}{0.550000in}}{\pgfqpoint{7.750000in}{3.850000in}}%
\pgfusepath{clip}%
\pgfsetbuttcap%
\pgfsetmiterjoin%
\definecolor{currentfill}{rgb}{0.501961,0.501961,0.501961}%
\pgfsetfillcolor{currentfill}%
\pgfsetfillopacity{0.700000}%
\pgfsetlinewidth{0.000000pt}%
\definecolor{currentstroke}{rgb}{0.000000,0.000000,0.000000}%
\pgfsetstrokecolor{currentstroke}%
\pgfsetstrokeopacity{0.700000}%
\pgfsetdash{}{0pt}%
\pgfpathmoveto{\pgfqpoint{2.042614in}{0.550000in}}%
\pgfpathlineto{\pgfqpoint{2.070135in}{0.550000in}}%
\pgfpathlineto{\pgfqpoint{2.070135in}{2.222240in}}%
\pgfpathlineto{\pgfqpoint{2.042614in}{2.222240in}}%
\pgfpathlineto{\pgfqpoint{2.042614in}{0.550000in}}%
\pgfpathclose%
\pgfusepath{fill}%
\end{pgfscope}%
\begin{pgfscope}%
\pgfpathrectangle{\pgfqpoint{1.250000in}{0.550000in}}{\pgfqpoint{7.750000in}{3.850000in}}%
\pgfusepath{clip}%
\pgfsetbuttcap%
\pgfsetmiterjoin%
\definecolor{currentfill}{rgb}{0.501961,0.501961,0.501961}%
\pgfsetfillcolor{currentfill}%
\pgfsetfillopacity{0.700000}%
\pgfsetlinewidth{0.000000pt}%
\definecolor{currentstroke}{rgb}{0.000000,0.000000,0.000000}%
\pgfsetstrokecolor{currentstroke}%
\pgfsetstrokeopacity{0.700000}%
\pgfsetdash{}{0pt}%
\pgfpathmoveto{\pgfqpoint{2.070135in}{0.550000in}}%
\pgfpathlineto{\pgfqpoint{2.097656in}{0.550000in}}%
\pgfpathlineto{\pgfqpoint{2.097656in}{2.168066in}}%
\pgfpathlineto{\pgfqpoint{2.070135in}{2.168066in}}%
\pgfpathlineto{\pgfqpoint{2.070135in}{0.550000in}}%
\pgfpathclose%
\pgfusepath{fill}%
\end{pgfscope}%
\begin{pgfscope}%
\pgfpathrectangle{\pgfqpoint{1.250000in}{0.550000in}}{\pgfqpoint{7.750000in}{3.850000in}}%
\pgfusepath{clip}%
\pgfsetbuttcap%
\pgfsetmiterjoin%
\definecolor{currentfill}{rgb}{0.501961,0.501961,0.501961}%
\pgfsetfillcolor{currentfill}%
\pgfsetfillopacity{0.700000}%
\pgfsetlinewidth{0.000000pt}%
\definecolor{currentstroke}{rgb}{0.000000,0.000000,0.000000}%
\pgfsetstrokecolor{currentstroke}%
\pgfsetstrokeopacity{0.700000}%
\pgfsetdash{}{0pt}%
\pgfpathmoveto{\pgfqpoint{2.097656in}{0.550000in}}%
\pgfpathlineto{\pgfqpoint{2.125178in}{0.550000in}}%
\pgfpathlineto{\pgfqpoint{2.125178in}{2.132426in}}%
\pgfpathlineto{\pgfqpoint{2.097656in}{2.132426in}}%
\pgfpathlineto{\pgfqpoint{2.097656in}{0.550000in}}%
\pgfpathclose%
\pgfusepath{fill}%
\end{pgfscope}%
\begin{pgfscope}%
\pgfpathrectangle{\pgfqpoint{1.250000in}{0.550000in}}{\pgfqpoint{7.750000in}{3.850000in}}%
\pgfusepath{clip}%
\pgfsetbuttcap%
\pgfsetmiterjoin%
\definecolor{currentfill}{rgb}{0.501961,0.501961,0.501961}%
\pgfsetfillcolor{currentfill}%
\pgfsetfillopacity{0.700000}%
\pgfsetlinewidth{0.000000pt}%
\definecolor{currentstroke}{rgb}{0.000000,0.000000,0.000000}%
\pgfsetstrokecolor{currentstroke}%
\pgfsetstrokeopacity{0.700000}%
\pgfsetdash{}{0pt}%
\pgfpathmoveto{\pgfqpoint{2.125178in}{0.550000in}}%
\pgfpathlineto{\pgfqpoint{2.152699in}{0.550000in}}%
\pgfpathlineto{\pgfqpoint{2.152699in}{2.276413in}}%
\pgfpathlineto{\pgfqpoint{2.125178in}{2.276413in}}%
\pgfpathlineto{\pgfqpoint{2.125178in}{0.550000in}}%
\pgfpathclose%
\pgfusepath{fill}%
\end{pgfscope}%
\begin{pgfscope}%
\pgfpathrectangle{\pgfqpoint{1.250000in}{0.550000in}}{\pgfqpoint{7.750000in}{3.850000in}}%
\pgfusepath{clip}%
\pgfsetbuttcap%
\pgfsetmiterjoin%
\definecolor{currentfill}{rgb}{0.501961,0.501961,0.501961}%
\pgfsetfillcolor{currentfill}%
\pgfsetfillopacity{0.700000}%
\pgfsetlinewidth{0.000000pt}%
\definecolor{currentstroke}{rgb}{0.000000,0.000000,0.000000}%
\pgfsetstrokecolor{currentstroke}%
\pgfsetstrokeopacity{0.700000}%
\pgfsetdash{}{0pt}%
\pgfpathmoveto{\pgfqpoint{2.152699in}{0.550000in}}%
\pgfpathlineto{\pgfqpoint{2.180220in}{0.550000in}}%
\pgfpathlineto{\pgfqpoint{2.180220in}{2.206558in}}%
\pgfpathlineto{\pgfqpoint{2.152699in}{2.206558in}}%
\pgfpathlineto{\pgfqpoint{2.152699in}{0.550000in}}%
\pgfpathclose%
\pgfusepath{fill}%
\end{pgfscope}%
\begin{pgfscope}%
\pgfpathrectangle{\pgfqpoint{1.250000in}{0.550000in}}{\pgfqpoint{7.750000in}{3.850000in}}%
\pgfusepath{clip}%
\pgfsetbuttcap%
\pgfsetmiterjoin%
\definecolor{currentfill}{rgb}{0.501961,0.501961,0.501961}%
\pgfsetfillcolor{currentfill}%
\pgfsetfillopacity{0.700000}%
\pgfsetlinewidth{0.000000pt}%
\definecolor{currentstroke}{rgb}{0.000000,0.000000,0.000000}%
\pgfsetstrokecolor{currentstroke}%
\pgfsetstrokeopacity{0.700000}%
\pgfsetdash{}{0pt}%
\pgfpathmoveto{\pgfqpoint{2.180220in}{0.550000in}}%
\pgfpathlineto{\pgfqpoint{2.207741in}{0.550000in}}%
\pgfpathlineto{\pgfqpoint{2.207741in}{2.303499in}}%
\pgfpathlineto{\pgfqpoint{2.180220in}{2.303499in}}%
\pgfpathlineto{\pgfqpoint{2.180220in}{0.550000in}}%
\pgfpathclose%
\pgfusepath{fill}%
\end{pgfscope}%
\begin{pgfscope}%
\pgfpathrectangle{\pgfqpoint{1.250000in}{0.550000in}}{\pgfqpoint{7.750000in}{3.850000in}}%
\pgfusepath{clip}%
\pgfsetbuttcap%
\pgfsetmiterjoin%
\definecolor{currentfill}{rgb}{0.501961,0.501961,0.501961}%
\pgfsetfillcolor{currentfill}%
\pgfsetfillopacity{0.700000}%
\pgfsetlinewidth{0.000000pt}%
\definecolor{currentstroke}{rgb}{0.000000,0.000000,0.000000}%
\pgfsetstrokecolor{currentstroke}%
\pgfsetstrokeopacity{0.700000}%
\pgfsetdash{}{0pt}%
\pgfpathmoveto{\pgfqpoint{2.207741in}{0.550000in}}%
\pgfpathlineto{\pgfqpoint{2.235263in}{0.550000in}}%
\pgfpathlineto{\pgfqpoint{2.235263in}{2.363375in}}%
\pgfpathlineto{\pgfqpoint{2.207741in}{2.363375in}}%
\pgfpathlineto{\pgfqpoint{2.207741in}{0.550000in}}%
\pgfpathclose%
\pgfusepath{fill}%
\end{pgfscope}%
\begin{pgfscope}%
\pgfpathrectangle{\pgfqpoint{1.250000in}{0.550000in}}{\pgfqpoint{7.750000in}{3.850000in}}%
\pgfusepath{clip}%
\pgfsetbuttcap%
\pgfsetmiterjoin%
\definecolor{currentfill}{rgb}{0.501961,0.501961,0.501961}%
\pgfsetfillcolor{currentfill}%
\pgfsetfillopacity{0.700000}%
\pgfsetlinewidth{0.000000pt}%
\definecolor{currentstroke}{rgb}{0.000000,0.000000,0.000000}%
\pgfsetstrokecolor{currentstroke}%
\pgfsetstrokeopacity{0.700000}%
\pgfsetdash{}{0pt}%
\pgfpathmoveto{\pgfqpoint{2.235263in}{0.550000in}}%
\pgfpathlineto{\pgfqpoint{2.262784in}{0.550000in}}%
\pgfpathlineto{\pgfqpoint{2.262784in}{2.363375in}}%
\pgfpathlineto{\pgfqpoint{2.235263in}{2.363375in}}%
\pgfpathlineto{\pgfqpoint{2.235263in}{0.550000in}}%
\pgfpathclose%
\pgfusepath{fill}%
\end{pgfscope}%
\begin{pgfscope}%
\pgfpathrectangle{\pgfqpoint{1.250000in}{0.550000in}}{\pgfqpoint{7.750000in}{3.850000in}}%
\pgfusepath{clip}%
\pgfsetbuttcap%
\pgfsetmiterjoin%
\definecolor{currentfill}{rgb}{0.501961,0.501961,0.501961}%
\pgfsetfillcolor{currentfill}%
\pgfsetfillopacity{0.700000}%
\pgfsetlinewidth{0.000000pt}%
\definecolor{currentstroke}{rgb}{0.000000,0.000000,0.000000}%
\pgfsetstrokecolor{currentstroke}%
\pgfsetstrokeopacity{0.700000}%
\pgfsetdash{}{0pt}%
\pgfpathmoveto{\pgfqpoint{2.262784in}{0.550000in}}%
\pgfpathlineto{\pgfqpoint{2.290305in}{0.550000in}}%
\pgfpathlineto{\pgfqpoint{2.290305in}{2.451763in}}%
\pgfpathlineto{\pgfqpoint{2.262784in}{2.451763in}}%
\pgfpathlineto{\pgfqpoint{2.262784in}{0.550000in}}%
\pgfpathclose%
\pgfusepath{fill}%
\end{pgfscope}%
\begin{pgfscope}%
\pgfpathrectangle{\pgfqpoint{1.250000in}{0.550000in}}{\pgfqpoint{7.750000in}{3.850000in}}%
\pgfusepath{clip}%
\pgfsetbuttcap%
\pgfsetmiterjoin%
\definecolor{currentfill}{rgb}{0.501961,0.501961,0.501961}%
\pgfsetfillcolor{currentfill}%
\pgfsetfillopacity{0.700000}%
\pgfsetlinewidth{0.000000pt}%
\definecolor{currentstroke}{rgb}{0.000000,0.000000,0.000000}%
\pgfsetstrokecolor{currentstroke}%
\pgfsetstrokeopacity{0.700000}%
\pgfsetdash{}{0pt}%
\pgfpathmoveto{\pgfqpoint{2.290305in}{0.550000in}}%
\pgfpathlineto{\pgfqpoint{2.317827in}{0.550000in}}%
\pgfpathlineto{\pgfqpoint{2.317827in}{2.527320in}}%
\pgfpathlineto{\pgfqpoint{2.290305in}{2.527320in}}%
\pgfpathlineto{\pgfqpoint{2.290305in}{0.550000in}}%
\pgfpathclose%
\pgfusepath{fill}%
\end{pgfscope}%
\begin{pgfscope}%
\pgfpathrectangle{\pgfqpoint{1.250000in}{0.550000in}}{\pgfqpoint{7.750000in}{3.850000in}}%
\pgfusepath{clip}%
\pgfsetbuttcap%
\pgfsetmiterjoin%
\definecolor{currentfill}{rgb}{0.501961,0.501961,0.501961}%
\pgfsetfillcolor{currentfill}%
\pgfsetfillopacity{0.700000}%
\pgfsetlinewidth{0.000000pt}%
\definecolor{currentstroke}{rgb}{0.000000,0.000000,0.000000}%
\pgfsetstrokecolor{currentstroke}%
\pgfsetstrokeopacity{0.700000}%
\pgfsetdash{}{0pt}%
\pgfpathmoveto{\pgfqpoint{2.317827in}{0.550000in}}%
\pgfpathlineto{\pgfqpoint{2.345348in}{0.550000in}}%
\pgfpathlineto{\pgfqpoint{2.345348in}{2.610005in}}%
\pgfpathlineto{\pgfqpoint{2.317827in}{2.610005in}}%
\pgfpathlineto{\pgfqpoint{2.317827in}{0.550000in}}%
\pgfpathclose%
\pgfusepath{fill}%
\end{pgfscope}%
\begin{pgfscope}%
\pgfpathrectangle{\pgfqpoint{1.250000in}{0.550000in}}{\pgfqpoint{7.750000in}{3.850000in}}%
\pgfusepath{clip}%
\pgfsetbuttcap%
\pgfsetmiterjoin%
\definecolor{currentfill}{rgb}{0.501961,0.501961,0.501961}%
\pgfsetfillcolor{currentfill}%
\pgfsetfillopacity{0.700000}%
\pgfsetlinewidth{0.000000pt}%
\definecolor{currentstroke}{rgb}{0.000000,0.000000,0.000000}%
\pgfsetstrokecolor{currentstroke}%
\pgfsetstrokeopacity{0.700000}%
\pgfsetdash{}{0pt}%
\pgfpathmoveto{\pgfqpoint{2.345348in}{0.550000in}}%
\pgfpathlineto{\pgfqpoint{2.372869in}{0.550000in}}%
\pgfpathlineto{\pgfqpoint{2.372869in}{2.493105in}}%
\pgfpathlineto{\pgfqpoint{2.345348in}{2.493105in}}%
\pgfpathlineto{\pgfqpoint{2.345348in}{0.550000in}}%
\pgfpathclose%
\pgfusepath{fill}%
\end{pgfscope}%
\begin{pgfscope}%
\pgfpathrectangle{\pgfqpoint{1.250000in}{0.550000in}}{\pgfqpoint{7.750000in}{3.850000in}}%
\pgfusepath{clip}%
\pgfsetbuttcap%
\pgfsetmiterjoin%
\definecolor{currentfill}{rgb}{0.501961,0.501961,0.501961}%
\pgfsetfillcolor{currentfill}%
\pgfsetfillopacity{0.700000}%
\pgfsetlinewidth{0.000000pt}%
\definecolor{currentstroke}{rgb}{0.000000,0.000000,0.000000}%
\pgfsetstrokecolor{currentstroke}%
\pgfsetstrokeopacity{0.700000}%
\pgfsetdash{}{0pt}%
\pgfpathmoveto{\pgfqpoint{2.372869in}{0.550000in}}%
\pgfpathlineto{\pgfqpoint{2.400391in}{0.550000in}}%
\pgfpathlineto{\pgfqpoint{2.400391in}{0.550000in}}%
\pgfpathlineto{\pgfqpoint{2.372869in}{0.550000in}}%
\pgfpathlineto{\pgfqpoint{2.372869in}{0.550000in}}%
\pgfpathclose%
\pgfusepath{fill}%
\end{pgfscope}%
\begin{pgfscope}%
\pgfpathrectangle{\pgfqpoint{1.250000in}{0.550000in}}{\pgfqpoint{7.750000in}{3.850000in}}%
\pgfusepath{clip}%
\pgfsetbuttcap%
\pgfsetmiterjoin%
\definecolor{currentfill}{rgb}{0.501961,0.501961,0.501961}%
\pgfsetfillcolor{currentfill}%
\pgfsetfillopacity{0.700000}%
\pgfsetlinewidth{0.000000pt}%
\definecolor{currentstroke}{rgb}{0.000000,0.000000,0.000000}%
\pgfsetstrokecolor{currentstroke}%
\pgfsetstrokeopacity{0.700000}%
\pgfsetdash{}{0pt}%
\pgfpathmoveto{\pgfqpoint{2.400391in}{0.550000in}}%
\pgfpathlineto{\pgfqpoint{2.427912in}{0.550000in}}%
\pgfpathlineto{\pgfqpoint{2.427912in}{2.655625in}}%
\pgfpathlineto{\pgfqpoint{2.400391in}{2.655625in}}%
\pgfpathlineto{\pgfqpoint{2.400391in}{0.550000in}}%
\pgfpathclose%
\pgfusepath{fill}%
\end{pgfscope}%
\begin{pgfscope}%
\pgfpathrectangle{\pgfqpoint{1.250000in}{0.550000in}}{\pgfqpoint{7.750000in}{3.850000in}}%
\pgfusepath{clip}%
\pgfsetbuttcap%
\pgfsetmiterjoin%
\definecolor{currentfill}{rgb}{0.501961,0.501961,0.501961}%
\pgfsetfillcolor{currentfill}%
\pgfsetfillopacity{0.700000}%
\pgfsetlinewidth{0.000000pt}%
\definecolor{currentstroke}{rgb}{0.000000,0.000000,0.000000}%
\pgfsetstrokecolor{currentstroke}%
\pgfsetstrokeopacity{0.700000}%
\pgfsetdash{}{0pt}%
\pgfpathmoveto{\pgfqpoint{2.427912in}{0.550000in}}%
\pgfpathlineto{\pgfqpoint{2.455433in}{0.550000in}}%
\pgfpathlineto{\pgfqpoint{2.455433in}{2.493105in}}%
\pgfpathlineto{\pgfqpoint{2.427912in}{2.493105in}}%
\pgfpathlineto{\pgfqpoint{2.427912in}{0.550000in}}%
\pgfpathclose%
\pgfusepath{fill}%
\end{pgfscope}%
\begin{pgfscope}%
\pgfpathrectangle{\pgfqpoint{1.250000in}{0.550000in}}{\pgfqpoint{7.750000in}{3.850000in}}%
\pgfusepath{clip}%
\pgfsetbuttcap%
\pgfsetmiterjoin%
\definecolor{currentfill}{rgb}{0.501961,0.501961,0.501961}%
\pgfsetfillcolor{currentfill}%
\pgfsetfillopacity{0.700000}%
\pgfsetlinewidth{0.000000pt}%
\definecolor{currentstroke}{rgb}{0.000000,0.000000,0.000000}%
\pgfsetstrokecolor{currentstroke}%
\pgfsetstrokeopacity{0.700000}%
\pgfsetdash{}{0pt}%
\pgfpathmoveto{\pgfqpoint{2.455433in}{0.550000in}}%
\pgfpathlineto{\pgfqpoint{2.482955in}{0.550000in}}%
\pgfpathlineto{\pgfqpoint{2.482955in}{2.574365in}}%
\pgfpathlineto{\pgfqpoint{2.455433in}{2.574365in}}%
\pgfpathlineto{\pgfqpoint{2.455433in}{0.550000in}}%
\pgfpathclose%
\pgfusepath{fill}%
\end{pgfscope}%
\begin{pgfscope}%
\pgfpathrectangle{\pgfqpoint{1.250000in}{0.550000in}}{\pgfqpoint{7.750000in}{3.850000in}}%
\pgfusepath{clip}%
\pgfsetbuttcap%
\pgfsetmiterjoin%
\definecolor{currentfill}{rgb}{0.501961,0.501961,0.501961}%
\pgfsetfillcolor{currentfill}%
\pgfsetfillopacity{0.700000}%
\pgfsetlinewidth{0.000000pt}%
\definecolor{currentstroke}{rgb}{0.000000,0.000000,0.000000}%
\pgfsetstrokecolor{currentstroke}%
\pgfsetstrokeopacity{0.700000}%
\pgfsetdash{}{0pt}%
\pgfpathmoveto{\pgfqpoint{2.482955in}{0.550000in}}%
\pgfpathlineto{\pgfqpoint{2.510476in}{0.550000in}}%
\pgfpathlineto{\pgfqpoint{2.510476in}{2.531597in}}%
\pgfpathlineto{\pgfqpoint{2.482955in}{2.531597in}}%
\pgfpathlineto{\pgfqpoint{2.482955in}{0.550000in}}%
\pgfpathclose%
\pgfusepath{fill}%
\end{pgfscope}%
\begin{pgfscope}%
\pgfpathrectangle{\pgfqpoint{1.250000in}{0.550000in}}{\pgfqpoint{7.750000in}{3.850000in}}%
\pgfusepath{clip}%
\pgfsetbuttcap%
\pgfsetmiterjoin%
\definecolor{currentfill}{rgb}{0.501961,0.501961,0.501961}%
\pgfsetfillcolor{currentfill}%
\pgfsetfillopacity{0.700000}%
\pgfsetlinewidth{0.000000pt}%
\definecolor{currentstroke}{rgb}{0.000000,0.000000,0.000000}%
\pgfsetstrokecolor{currentstroke}%
\pgfsetstrokeopacity{0.700000}%
\pgfsetdash{}{0pt}%
\pgfpathmoveto{\pgfqpoint{2.510476in}{0.550000in}}%
\pgfpathlineto{\pgfqpoint{2.537997in}{0.550000in}}%
\pgfpathlineto{\pgfqpoint{2.537997in}{2.591472in}}%
\pgfpathlineto{\pgfqpoint{2.510476in}{2.591472in}}%
\pgfpathlineto{\pgfqpoint{2.510476in}{0.550000in}}%
\pgfpathclose%
\pgfusepath{fill}%
\end{pgfscope}%
\begin{pgfscope}%
\pgfpathrectangle{\pgfqpoint{1.250000in}{0.550000in}}{\pgfqpoint{7.750000in}{3.850000in}}%
\pgfusepath{clip}%
\pgfsetbuttcap%
\pgfsetmiterjoin%
\definecolor{currentfill}{rgb}{0.501961,0.501961,0.501961}%
\pgfsetfillcolor{currentfill}%
\pgfsetfillopacity{0.700000}%
\pgfsetlinewidth{0.000000pt}%
\definecolor{currentstroke}{rgb}{0.000000,0.000000,0.000000}%
\pgfsetstrokecolor{currentstroke}%
\pgfsetstrokeopacity{0.700000}%
\pgfsetdash{}{0pt}%
\pgfpathmoveto{\pgfqpoint{2.537997in}{0.550000in}}%
\pgfpathlineto{\pgfqpoint{2.565518in}{0.550000in}}%
\pgfpathlineto{\pgfqpoint{2.565518in}{2.456039in}}%
\pgfpathlineto{\pgfqpoint{2.537997in}{2.456039in}}%
\pgfpathlineto{\pgfqpoint{2.537997in}{0.550000in}}%
\pgfpathclose%
\pgfusepath{fill}%
\end{pgfscope}%
\begin{pgfscope}%
\pgfpathrectangle{\pgfqpoint{1.250000in}{0.550000in}}{\pgfqpoint{7.750000in}{3.850000in}}%
\pgfusepath{clip}%
\pgfsetbuttcap%
\pgfsetmiterjoin%
\definecolor{currentfill}{rgb}{0.501961,0.501961,0.501961}%
\pgfsetfillcolor{currentfill}%
\pgfsetfillopacity{0.700000}%
\pgfsetlinewidth{0.000000pt}%
\definecolor{currentstroke}{rgb}{0.000000,0.000000,0.000000}%
\pgfsetstrokecolor{currentstroke}%
\pgfsetstrokeopacity{0.700000}%
\pgfsetdash{}{0pt}%
\pgfpathmoveto{\pgfqpoint{2.565518in}{0.550000in}}%
\pgfpathlineto{\pgfqpoint{2.593040in}{0.550000in}}%
\pgfpathlineto{\pgfqpoint{2.593040in}{2.538725in}}%
\pgfpathlineto{\pgfqpoint{2.565518in}{2.538725in}}%
\pgfpathlineto{\pgfqpoint{2.565518in}{0.550000in}}%
\pgfpathclose%
\pgfusepath{fill}%
\end{pgfscope}%
\begin{pgfscope}%
\pgfpathrectangle{\pgfqpoint{1.250000in}{0.550000in}}{\pgfqpoint{7.750000in}{3.850000in}}%
\pgfusepath{clip}%
\pgfsetbuttcap%
\pgfsetmiterjoin%
\definecolor{currentfill}{rgb}{0.501961,0.501961,0.501961}%
\pgfsetfillcolor{currentfill}%
\pgfsetfillopacity{0.700000}%
\pgfsetlinewidth{0.000000pt}%
\definecolor{currentstroke}{rgb}{0.000000,0.000000,0.000000}%
\pgfsetstrokecolor{currentstroke}%
\pgfsetstrokeopacity{0.700000}%
\pgfsetdash{}{0pt}%
\pgfpathmoveto{\pgfqpoint{2.593040in}{0.550000in}}%
\pgfpathlineto{\pgfqpoint{2.620561in}{0.550000in}}%
\pgfpathlineto{\pgfqpoint{2.620561in}{2.456039in}}%
\pgfpathlineto{\pgfqpoint{2.593040in}{2.456039in}}%
\pgfpathlineto{\pgfqpoint{2.593040in}{0.550000in}}%
\pgfpathclose%
\pgfusepath{fill}%
\end{pgfscope}%
\begin{pgfscope}%
\pgfpathrectangle{\pgfqpoint{1.250000in}{0.550000in}}{\pgfqpoint{7.750000in}{3.850000in}}%
\pgfusepath{clip}%
\pgfsetbuttcap%
\pgfsetmiterjoin%
\definecolor{currentfill}{rgb}{0.501961,0.501961,0.501961}%
\pgfsetfillcolor{currentfill}%
\pgfsetfillopacity{0.700000}%
\pgfsetlinewidth{0.000000pt}%
\definecolor{currentstroke}{rgb}{0.000000,0.000000,0.000000}%
\pgfsetstrokecolor{currentstroke}%
\pgfsetstrokeopacity{0.700000}%
\pgfsetdash{}{0pt}%
\pgfpathmoveto{\pgfqpoint{2.620561in}{0.550000in}}%
\pgfpathlineto{\pgfqpoint{2.648082in}{0.550000in}}%
\pgfpathlineto{\pgfqpoint{2.648082in}{2.438932in}}%
\pgfpathlineto{\pgfqpoint{2.620561in}{2.438932in}}%
\pgfpathlineto{\pgfqpoint{2.620561in}{0.550000in}}%
\pgfpathclose%
\pgfusepath{fill}%
\end{pgfscope}%
\begin{pgfscope}%
\pgfpathrectangle{\pgfqpoint{1.250000in}{0.550000in}}{\pgfqpoint{7.750000in}{3.850000in}}%
\pgfusepath{clip}%
\pgfsetbuttcap%
\pgfsetmiterjoin%
\definecolor{currentfill}{rgb}{0.501961,0.501961,0.501961}%
\pgfsetfillcolor{currentfill}%
\pgfsetfillopacity{0.700000}%
\pgfsetlinewidth{0.000000pt}%
\definecolor{currentstroke}{rgb}{0.000000,0.000000,0.000000}%
\pgfsetstrokecolor{currentstroke}%
\pgfsetstrokeopacity{0.700000}%
\pgfsetdash{}{0pt}%
\pgfpathmoveto{\pgfqpoint{2.648082in}{0.550000in}}%
\pgfpathlineto{\pgfqpoint{2.675604in}{0.550000in}}%
\pgfpathlineto{\pgfqpoint{2.675604in}{2.427527in}}%
\pgfpathlineto{\pgfqpoint{2.648082in}{2.427527in}}%
\pgfpathlineto{\pgfqpoint{2.648082in}{0.550000in}}%
\pgfpathclose%
\pgfusepath{fill}%
\end{pgfscope}%
\begin{pgfscope}%
\pgfpathrectangle{\pgfqpoint{1.250000in}{0.550000in}}{\pgfqpoint{7.750000in}{3.850000in}}%
\pgfusepath{clip}%
\pgfsetbuttcap%
\pgfsetmiterjoin%
\definecolor{currentfill}{rgb}{0.501961,0.501961,0.501961}%
\pgfsetfillcolor{currentfill}%
\pgfsetfillopacity{0.700000}%
\pgfsetlinewidth{0.000000pt}%
\definecolor{currentstroke}{rgb}{0.000000,0.000000,0.000000}%
\pgfsetstrokecolor{currentstroke}%
\pgfsetstrokeopacity{0.700000}%
\pgfsetdash{}{0pt}%
\pgfpathmoveto{\pgfqpoint{2.675604in}{0.550000in}}%
\pgfpathlineto{\pgfqpoint{2.703125in}{0.550000in}}%
\pgfpathlineto{\pgfqpoint{2.703125in}{2.340565in}}%
\pgfpathlineto{\pgfqpoint{2.675604in}{2.340565in}}%
\pgfpathlineto{\pgfqpoint{2.675604in}{0.550000in}}%
\pgfpathclose%
\pgfusepath{fill}%
\end{pgfscope}%
\begin{pgfscope}%
\pgfpathrectangle{\pgfqpoint{1.250000in}{0.550000in}}{\pgfqpoint{7.750000in}{3.850000in}}%
\pgfusepath{clip}%
\pgfsetbuttcap%
\pgfsetmiterjoin%
\definecolor{currentfill}{rgb}{0.501961,0.501961,0.501961}%
\pgfsetfillcolor{currentfill}%
\pgfsetfillopacity{0.700000}%
\pgfsetlinewidth{0.000000pt}%
\definecolor{currentstroke}{rgb}{0.000000,0.000000,0.000000}%
\pgfsetstrokecolor{currentstroke}%
\pgfsetstrokeopacity{0.700000}%
\pgfsetdash{}{0pt}%
\pgfpathmoveto{\pgfqpoint{2.703125in}{0.550000in}}%
\pgfpathlineto{\pgfqpoint{2.730646in}{0.550000in}}%
\pgfpathlineto{\pgfqpoint{2.730646in}{2.427527in}}%
\pgfpathlineto{\pgfqpoint{2.703125in}{2.427527in}}%
\pgfpathlineto{\pgfqpoint{2.703125in}{0.550000in}}%
\pgfpathclose%
\pgfusepath{fill}%
\end{pgfscope}%
\begin{pgfscope}%
\pgfpathrectangle{\pgfqpoint{1.250000in}{0.550000in}}{\pgfqpoint{7.750000in}{3.850000in}}%
\pgfusepath{clip}%
\pgfsetbuttcap%
\pgfsetmiterjoin%
\definecolor{currentfill}{rgb}{0.501961,0.501961,0.501961}%
\pgfsetfillcolor{currentfill}%
\pgfsetfillopacity{0.700000}%
\pgfsetlinewidth{0.000000pt}%
\definecolor{currentstroke}{rgb}{0.000000,0.000000,0.000000}%
\pgfsetstrokecolor{currentstroke}%
\pgfsetstrokeopacity{0.700000}%
\pgfsetdash{}{0pt}%
\pgfpathmoveto{\pgfqpoint{2.730646in}{0.550000in}}%
\pgfpathlineto{\pgfqpoint{2.758168in}{0.550000in}}%
\pgfpathlineto{\pgfqpoint{2.758168in}{2.427527in}}%
\pgfpathlineto{\pgfqpoint{2.730646in}{2.427527in}}%
\pgfpathlineto{\pgfqpoint{2.730646in}{0.550000in}}%
\pgfpathclose%
\pgfusepath{fill}%
\end{pgfscope}%
\begin{pgfscope}%
\pgfpathrectangle{\pgfqpoint{1.250000in}{0.550000in}}{\pgfqpoint{7.750000in}{3.850000in}}%
\pgfusepath{clip}%
\pgfsetbuttcap%
\pgfsetmiterjoin%
\definecolor{currentfill}{rgb}{0.501961,0.501961,0.501961}%
\pgfsetfillcolor{currentfill}%
\pgfsetfillopacity{0.700000}%
\pgfsetlinewidth{0.000000pt}%
\definecolor{currentstroke}{rgb}{0.000000,0.000000,0.000000}%
\pgfsetstrokecolor{currentstroke}%
\pgfsetstrokeopacity{0.700000}%
\pgfsetdash{}{0pt}%
\pgfpathmoveto{\pgfqpoint{2.758168in}{0.550000in}}%
\pgfpathlineto{\pgfqpoint{2.785689in}{0.550000in}}%
\pgfpathlineto{\pgfqpoint{2.785689in}{2.363375in}}%
\pgfpathlineto{\pgfqpoint{2.758168in}{2.363375in}}%
\pgfpathlineto{\pgfqpoint{2.758168in}{0.550000in}}%
\pgfpathclose%
\pgfusepath{fill}%
\end{pgfscope}%
\begin{pgfscope}%
\pgfpathrectangle{\pgfqpoint{1.250000in}{0.550000in}}{\pgfqpoint{7.750000in}{3.850000in}}%
\pgfusepath{clip}%
\pgfsetbuttcap%
\pgfsetmiterjoin%
\definecolor{currentfill}{rgb}{0.501961,0.501961,0.501961}%
\pgfsetfillcolor{currentfill}%
\pgfsetfillopacity{0.700000}%
\pgfsetlinewidth{0.000000pt}%
\definecolor{currentstroke}{rgb}{0.000000,0.000000,0.000000}%
\pgfsetstrokecolor{currentstroke}%
\pgfsetstrokeopacity{0.700000}%
\pgfsetdash{}{0pt}%
\pgfpathmoveto{\pgfqpoint{2.785689in}{0.550000in}}%
\pgfpathlineto{\pgfqpoint{2.813210in}{0.550000in}}%
\pgfpathlineto{\pgfqpoint{2.813210in}{2.387610in}}%
\pgfpathlineto{\pgfqpoint{2.785689in}{2.387610in}}%
\pgfpathlineto{\pgfqpoint{2.785689in}{0.550000in}}%
\pgfpathclose%
\pgfusepath{fill}%
\end{pgfscope}%
\begin{pgfscope}%
\pgfpathrectangle{\pgfqpoint{1.250000in}{0.550000in}}{\pgfqpoint{7.750000in}{3.850000in}}%
\pgfusepath{clip}%
\pgfsetbuttcap%
\pgfsetmiterjoin%
\definecolor{currentfill}{rgb}{0.501961,0.501961,0.501961}%
\pgfsetfillcolor{currentfill}%
\pgfsetfillopacity{0.700000}%
\pgfsetlinewidth{0.000000pt}%
\definecolor{currentstroke}{rgb}{0.000000,0.000000,0.000000}%
\pgfsetstrokecolor{currentstroke}%
\pgfsetstrokeopacity{0.700000}%
\pgfsetdash{}{0pt}%
\pgfpathmoveto{\pgfqpoint{2.813210in}{0.550000in}}%
\pgfpathlineto{\pgfqpoint{2.840732in}{0.550000in}}%
\pgfpathlineto{\pgfqpoint{2.840732in}{4.151089in}}%
\pgfpathlineto{\pgfqpoint{2.813210in}{4.151089in}}%
\pgfpathlineto{\pgfqpoint{2.813210in}{0.550000in}}%
\pgfpathclose%
\pgfusepath{fill}%
\end{pgfscope}%
\begin{pgfscope}%
\pgfpathrectangle{\pgfqpoint{1.250000in}{0.550000in}}{\pgfqpoint{7.750000in}{3.850000in}}%
\pgfusepath{clip}%
\pgfsetbuttcap%
\pgfsetmiterjoin%
\definecolor{currentfill}{rgb}{0.501961,0.501961,0.501961}%
\pgfsetfillcolor{currentfill}%
\pgfsetfillopacity{0.700000}%
\pgfsetlinewidth{0.000000pt}%
\definecolor{currentstroke}{rgb}{0.000000,0.000000,0.000000}%
\pgfsetstrokecolor{currentstroke}%
\pgfsetstrokeopacity{0.700000}%
\pgfsetdash{}{0pt}%
\pgfpathmoveto{\pgfqpoint{2.840732in}{0.550000in}}%
\pgfpathlineto{\pgfqpoint{2.868253in}{0.550000in}}%
\pgfpathlineto{\pgfqpoint{2.868253in}{2.406143in}}%
\pgfpathlineto{\pgfqpoint{2.840732in}{2.406143in}}%
\pgfpathlineto{\pgfqpoint{2.840732in}{0.550000in}}%
\pgfpathclose%
\pgfusepath{fill}%
\end{pgfscope}%
\begin{pgfscope}%
\pgfpathrectangle{\pgfqpoint{1.250000in}{0.550000in}}{\pgfqpoint{7.750000in}{3.850000in}}%
\pgfusepath{clip}%
\pgfsetbuttcap%
\pgfsetmiterjoin%
\definecolor{currentfill}{rgb}{0.501961,0.501961,0.501961}%
\pgfsetfillcolor{currentfill}%
\pgfsetfillopacity{0.700000}%
\pgfsetlinewidth{0.000000pt}%
\definecolor{currentstroke}{rgb}{0.000000,0.000000,0.000000}%
\pgfsetstrokecolor{currentstroke}%
\pgfsetstrokeopacity{0.700000}%
\pgfsetdash{}{0pt}%
\pgfpathmoveto{\pgfqpoint{2.868253in}{0.550000in}}%
\pgfpathlineto{\pgfqpoint{2.895774in}{0.550000in}}%
\pgfpathlineto{\pgfqpoint{2.895774in}{2.424676in}}%
\pgfpathlineto{\pgfqpoint{2.868253in}{2.424676in}}%
\pgfpathlineto{\pgfqpoint{2.868253in}{0.550000in}}%
\pgfpathclose%
\pgfusepath{fill}%
\end{pgfscope}%
\begin{pgfscope}%
\pgfpathrectangle{\pgfqpoint{1.250000in}{0.550000in}}{\pgfqpoint{7.750000in}{3.850000in}}%
\pgfusepath{clip}%
\pgfsetbuttcap%
\pgfsetmiterjoin%
\definecolor{currentfill}{rgb}{0.501961,0.501961,0.501961}%
\pgfsetfillcolor{currentfill}%
\pgfsetfillopacity{0.700000}%
\pgfsetlinewidth{0.000000pt}%
\definecolor{currentstroke}{rgb}{0.000000,0.000000,0.000000}%
\pgfsetstrokecolor{currentstroke}%
\pgfsetstrokeopacity{0.700000}%
\pgfsetdash{}{0pt}%
\pgfpathmoveto{\pgfqpoint{2.895774in}{0.550000in}}%
\pgfpathlineto{\pgfqpoint{2.923295in}{0.550000in}}%
\pgfpathlineto{\pgfqpoint{2.923295in}{2.393313in}}%
\pgfpathlineto{\pgfqpoint{2.895774in}{2.393313in}}%
\pgfpathlineto{\pgfqpoint{2.895774in}{0.550000in}}%
\pgfpathclose%
\pgfusepath{fill}%
\end{pgfscope}%
\begin{pgfscope}%
\pgfpathrectangle{\pgfqpoint{1.250000in}{0.550000in}}{\pgfqpoint{7.750000in}{3.850000in}}%
\pgfusepath{clip}%
\pgfsetbuttcap%
\pgfsetmiterjoin%
\definecolor{currentfill}{rgb}{0.501961,0.501961,0.501961}%
\pgfsetfillcolor{currentfill}%
\pgfsetfillopacity{0.700000}%
\pgfsetlinewidth{0.000000pt}%
\definecolor{currentstroke}{rgb}{0.000000,0.000000,0.000000}%
\pgfsetstrokecolor{currentstroke}%
\pgfsetstrokeopacity{0.700000}%
\pgfsetdash{}{0pt}%
\pgfpathmoveto{\pgfqpoint{2.923295in}{0.550000in}}%
\pgfpathlineto{\pgfqpoint{2.950817in}{0.550000in}}%
\pgfpathlineto{\pgfqpoint{2.950817in}{2.420399in}}%
\pgfpathlineto{\pgfqpoint{2.923295in}{2.420399in}}%
\pgfpathlineto{\pgfqpoint{2.923295in}{0.550000in}}%
\pgfpathclose%
\pgfusepath{fill}%
\end{pgfscope}%
\begin{pgfscope}%
\pgfpathrectangle{\pgfqpoint{1.250000in}{0.550000in}}{\pgfqpoint{7.750000in}{3.850000in}}%
\pgfusepath{clip}%
\pgfsetbuttcap%
\pgfsetmiterjoin%
\definecolor{currentfill}{rgb}{0.501961,0.501961,0.501961}%
\pgfsetfillcolor{currentfill}%
\pgfsetfillopacity{0.700000}%
\pgfsetlinewidth{0.000000pt}%
\definecolor{currentstroke}{rgb}{0.000000,0.000000,0.000000}%
\pgfsetstrokecolor{currentstroke}%
\pgfsetstrokeopacity{0.700000}%
\pgfsetdash{}{0pt}%
\pgfpathmoveto{\pgfqpoint{2.950817in}{0.550000in}}%
\pgfpathlineto{\pgfqpoint{2.978338in}{0.550000in}}%
\pgfpathlineto{\pgfqpoint{2.978338in}{2.467444in}}%
\pgfpathlineto{\pgfqpoint{2.950817in}{2.467444in}}%
\pgfpathlineto{\pgfqpoint{2.950817in}{0.550000in}}%
\pgfpathclose%
\pgfusepath{fill}%
\end{pgfscope}%
\begin{pgfscope}%
\pgfpathrectangle{\pgfqpoint{1.250000in}{0.550000in}}{\pgfqpoint{7.750000in}{3.850000in}}%
\pgfusepath{clip}%
\pgfsetbuttcap%
\pgfsetmiterjoin%
\definecolor{currentfill}{rgb}{0.501961,0.501961,0.501961}%
\pgfsetfillcolor{currentfill}%
\pgfsetfillopacity{0.700000}%
\pgfsetlinewidth{0.000000pt}%
\definecolor{currentstroke}{rgb}{0.000000,0.000000,0.000000}%
\pgfsetstrokecolor{currentstroke}%
\pgfsetstrokeopacity{0.700000}%
\pgfsetdash{}{0pt}%
\pgfpathmoveto{\pgfqpoint{2.978338in}{0.550000in}}%
\pgfpathlineto{\pgfqpoint{3.005859in}{0.550000in}}%
\pgfpathlineto{\pgfqpoint{3.005859in}{2.426102in}}%
\pgfpathlineto{\pgfqpoint{2.978338in}{2.426102in}}%
\pgfpathlineto{\pgfqpoint{2.978338in}{0.550000in}}%
\pgfpathclose%
\pgfusepath{fill}%
\end{pgfscope}%
\begin{pgfscope}%
\pgfpathrectangle{\pgfqpoint{1.250000in}{0.550000in}}{\pgfqpoint{7.750000in}{3.850000in}}%
\pgfusepath{clip}%
\pgfsetbuttcap%
\pgfsetmiterjoin%
\definecolor{currentfill}{rgb}{0.501961,0.501961,0.501961}%
\pgfsetfillcolor{currentfill}%
\pgfsetfillopacity{0.700000}%
\pgfsetlinewidth{0.000000pt}%
\definecolor{currentstroke}{rgb}{0.000000,0.000000,0.000000}%
\pgfsetstrokecolor{currentstroke}%
\pgfsetstrokeopacity{0.700000}%
\pgfsetdash{}{0pt}%
\pgfpathmoveto{\pgfqpoint{3.005859in}{0.550000in}}%
\pgfpathlineto{\pgfqpoint{3.033381in}{0.550000in}}%
\pgfpathlineto{\pgfqpoint{3.033381in}{2.615708in}}%
\pgfpathlineto{\pgfqpoint{3.005859in}{2.615708in}}%
\pgfpathlineto{\pgfqpoint{3.005859in}{0.550000in}}%
\pgfpathclose%
\pgfusepath{fill}%
\end{pgfscope}%
\begin{pgfscope}%
\pgfpathrectangle{\pgfqpoint{1.250000in}{0.550000in}}{\pgfqpoint{7.750000in}{3.850000in}}%
\pgfusepath{clip}%
\pgfsetbuttcap%
\pgfsetmiterjoin%
\definecolor{currentfill}{rgb}{0.501961,0.501961,0.501961}%
\pgfsetfillcolor{currentfill}%
\pgfsetfillopacity{0.700000}%
\pgfsetlinewidth{0.000000pt}%
\definecolor{currentstroke}{rgb}{0.000000,0.000000,0.000000}%
\pgfsetstrokecolor{currentstroke}%
\pgfsetstrokeopacity{0.700000}%
\pgfsetdash{}{0pt}%
\pgfpathmoveto{\pgfqpoint{3.033381in}{0.550000in}}%
\pgfpathlineto{\pgfqpoint{3.060902in}{0.550000in}}%
\pgfpathlineto{\pgfqpoint{3.060902in}{2.421825in}}%
\pgfpathlineto{\pgfqpoint{3.033381in}{2.421825in}}%
\pgfpathlineto{\pgfqpoint{3.033381in}{0.550000in}}%
\pgfpathclose%
\pgfusepath{fill}%
\end{pgfscope}%
\begin{pgfscope}%
\pgfpathrectangle{\pgfqpoint{1.250000in}{0.550000in}}{\pgfqpoint{7.750000in}{3.850000in}}%
\pgfusepath{clip}%
\pgfsetbuttcap%
\pgfsetmiterjoin%
\definecolor{currentfill}{rgb}{0.501961,0.501961,0.501961}%
\pgfsetfillcolor{currentfill}%
\pgfsetfillopacity{0.700000}%
\pgfsetlinewidth{0.000000pt}%
\definecolor{currentstroke}{rgb}{0.000000,0.000000,0.000000}%
\pgfsetstrokecolor{currentstroke}%
\pgfsetstrokeopacity{0.700000}%
\pgfsetdash{}{0pt}%
\pgfpathmoveto{\pgfqpoint{3.060902in}{0.550000in}}%
\pgfpathlineto{\pgfqpoint{3.088423in}{0.550000in}}%
\pgfpathlineto{\pgfqpoint{3.088423in}{2.584344in}}%
\pgfpathlineto{\pgfqpoint{3.060902in}{2.584344in}}%
\pgfpathlineto{\pgfqpoint{3.060902in}{0.550000in}}%
\pgfpathclose%
\pgfusepath{fill}%
\end{pgfscope}%
\begin{pgfscope}%
\pgfpathrectangle{\pgfqpoint{1.250000in}{0.550000in}}{\pgfqpoint{7.750000in}{3.850000in}}%
\pgfusepath{clip}%
\pgfsetbuttcap%
\pgfsetmiterjoin%
\definecolor{currentfill}{rgb}{0.501961,0.501961,0.501961}%
\pgfsetfillcolor{currentfill}%
\pgfsetfillopacity{0.700000}%
\pgfsetlinewidth{0.000000pt}%
\definecolor{currentstroke}{rgb}{0.000000,0.000000,0.000000}%
\pgfsetstrokecolor{currentstroke}%
\pgfsetstrokeopacity{0.700000}%
\pgfsetdash{}{0pt}%
\pgfpathmoveto{\pgfqpoint{3.088423in}{0.550000in}}%
\pgfpathlineto{\pgfqpoint{3.115945in}{0.550000in}}%
\pgfpathlineto{\pgfqpoint{3.115945in}{2.537299in}}%
\pgfpathlineto{\pgfqpoint{3.088423in}{2.537299in}}%
\pgfpathlineto{\pgfqpoint{3.088423in}{0.550000in}}%
\pgfpathclose%
\pgfusepath{fill}%
\end{pgfscope}%
\begin{pgfscope}%
\pgfpathrectangle{\pgfqpoint{1.250000in}{0.550000in}}{\pgfqpoint{7.750000in}{3.850000in}}%
\pgfusepath{clip}%
\pgfsetbuttcap%
\pgfsetmiterjoin%
\definecolor{currentfill}{rgb}{0.501961,0.501961,0.501961}%
\pgfsetfillcolor{currentfill}%
\pgfsetfillopacity{0.700000}%
\pgfsetlinewidth{0.000000pt}%
\definecolor{currentstroke}{rgb}{0.000000,0.000000,0.000000}%
\pgfsetstrokecolor{currentstroke}%
\pgfsetstrokeopacity{0.700000}%
\pgfsetdash{}{0pt}%
\pgfpathmoveto{\pgfqpoint{3.115945in}{0.550000in}}%
\pgfpathlineto{\pgfqpoint{3.143466in}{0.550000in}}%
\pgfpathlineto{\pgfqpoint{3.143466in}{2.658476in}}%
\pgfpathlineto{\pgfqpoint{3.115945in}{2.658476in}}%
\pgfpathlineto{\pgfqpoint{3.115945in}{0.550000in}}%
\pgfpathclose%
\pgfusepath{fill}%
\end{pgfscope}%
\begin{pgfscope}%
\pgfpathrectangle{\pgfqpoint{1.250000in}{0.550000in}}{\pgfqpoint{7.750000in}{3.850000in}}%
\pgfusepath{clip}%
\pgfsetbuttcap%
\pgfsetmiterjoin%
\definecolor{currentfill}{rgb}{0.501961,0.501961,0.501961}%
\pgfsetfillcolor{currentfill}%
\pgfsetfillopacity{0.700000}%
\pgfsetlinewidth{0.000000pt}%
\definecolor{currentstroke}{rgb}{0.000000,0.000000,0.000000}%
\pgfsetstrokecolor{currentstroke}%
\pgfsetstrokeopacity{0.700000}%
\pgfsetdash{}{0pt}%
\pgfpathmoveto{\pgfqpoint{3.143466in}{0.550000in}}%
\pgfpathlineto{\pgfqpoint{3.170987in}{0.550000in}}%
\pgfpathlineto{\pgfqpoint{3.170987in}{2.588621in}}%
\pgfpathlineto{\pgfqpoint{3.143466in}{2.588621in}}%
\pgfpathlineto{\pgfqpoint{3.143466in}{0.550000in}}%
\pgfpathclose%
\pgfusepath{fill}%
\end{pgfscope}%
\begin{pgfscope}%
\pgfpathrectangle{\pgfqpoint{1.250000in}{0.550000in}}{\pgfqpoint{7.750000in}{3.850000in}}%
\pgfusepath{clip}%
\pgfsetbuttcap%
\pgfsetmiterjoin%
\definecolor{currentfill}{rgb}{0.501961,0.501961,0.501961}%
\pgfsetfillcolor{currentfill}%
\pgfsetfillopacity{0.700000}%
\pgfsetlinewidth{0.000000pt}%
\definecolor{currentstroke}{rgb}{0.000000,0.000000,0.000000}%
\pgfsetstrokecolor{currentstroke}%
\pgfsetstrokeopacity{0.700000}%
\pgfsetdash{}{0pt}%
\pgfpathmoveto{\pgfqpoint{3.170987in}{0.550000in}}%
\pgfpathlineto{\pgfqpoint{3.198509in}{0.550000in}}%
\pgfpathlineto{\pgfqpoint{3.198509in}{0.550000in}}%
\pgfpathlineto{\pgfqpoint{3.170987in}{0.550000in}}%
\pgfpathlineto{\pgfqpoint{3.170987in}{0.550000in}}%
\pgfpathclose%
\pgfusepath{fill}%
\end{pgfscope}%
\begin{pgfscope}%
\pgfpathrectangle{\pgfqpoint{1.250000in}{0.550000in}}{\pgfqpoint{7.750000in}{3.850000in}}%
\pgfusepath{clip}%
\pgfsetbuttcap%
\pgfsetmiterjoin%
\definecolor{currentfill}{rgb}{0.501961,0.501961,0.501961}%
\pgfsetfillcolor{currentfill}%
\pgfsetfillopacity{0.700000}%
\pgfsetlinewidth{0.000000pt}%
\definecolor{currentstroke}{rgb}{0.000000,0.000000,0.000000}%
\pgfsetstrokecolor{currentstroke}%
\pgfsetstrokeopacity{0.700000}%
\pgfsetdash{}{0pt}%
\pgfpathmoveto{\pgfqpoint{3.198509in}{0.550000in}}%
\pgfpathlineto{\pgfqpoint{3.226030in}{0.550000in}}%
\pgfpathlineto{\pgfqpoint{3.226030in}{2.631389in}}%
\pgfpathlineto{\pgfqpoint{3.198509in}{2.631389in}}%
\pgfpathlineto{\pgfqpoint{3.198509in}{0.550000in}}%
\pgfpathclose%
\pgfusepath{fill}%
\end{pgfscope}%
\begin{pgfscope}%
\pgfpathrectangle{\pgfqpoint{1.250000in}{0.550000in}}{\pgfqpoint{7.750000in}{3.850000in}}%
\pgfusepath{clip}%
\pgfsetbuttcap%
\pgfsetmiterjoin%
\definecolor{currentfill}{rgb}{0.501961,0.501961,0.501961}%
\pgfsetfillcolor{currentfill}%
\pgfsetfillopacity{0.700000}%
\pgfsetlinewidth{0.000000pt}%
\definecolor{currentstroke}{rgb}{0.000000,0.000000,0.000000}%
\pgfsetstrokecolor{currentstroke}%
\pgfsetstrokeopacity{0.700000}%
\pgfsetdash{}{0pt}%
\pgfpathmoveto{\pgfqpoint{3.226030in}{0.550000in}}%
\pgfpathlineto{\pgfqpoint{3.253551in}{0.550000in}}%
\pgfpathlineto{\pgfqpoint{3.253551in}{2.712649in}}%
\pgfpathlineto{\pgfqpoint{3.226030in}{2.712649in}}%
\pgfpathlineto{\pgfqpoint{3.226030in}{0.550000in}}%
\pgfpathclose%
\pgfusepath{fill}%
\end{pgfscope}%
\begin{pgfscope}%
\pgfpathrectangle{\pgfqpoint{1.250000in}{0.550000in}}{\pgfqpoint{7.750000in}{3.850000in}}%
\pgfusepath{clip}%
\pgfsetbuttcap%
\pgfsetmiterjoin%
\definecolor{currentfill}{rgb}{0.501961,0.501961,0.501961}%
\pgfsetfillcolor{currentfill}%
\pgfsetfillopacity{0.700000}%
\pgfsetlinewidth{0.000000pt}%
\definecolor{currentstroke}{rgb}{0.000000,0.000000,0.000000}%
\pgfsetstrokecolor{currentstroke}%
\pgfsetstrokeopacity{0.700000}%
\pgfsetdash{}{0pt}%
\pgfpathmoveto{\pgfqpoint{3.253551in}{0.550000in}}%
\pgfpathlineto{\pgfqpoint{3.281072in}{0.550000in}}%
\pgfpathlineto{\pgfqpoint{3.281072in}{2.662753in}}%
\pgfpathlineto{\pgfqpoint{3.253551in}{2.662753in}}%
\pgfpathlineto{\pgfqpoint{3.253551in}{0.550000in}}%
\pgfpathclose%
\pgfusepath{fill}%
\end{pgfscope}%
\begin{pgfscope}%
\pgfpathrectangle{\pgfqpoint{1.250000in}{0.550000in}}{\pgfqpoint{7.750000in}{3.850000in}}%
\pgfusepath{clip}%
\pgfsetbuttcap%
\pgfsetmiterjoin%
\definecolor{currentfill}{rgb}{0.501961,0.501961,0.501961}%
\pgfsetfillcolor{currentfill}%
\pgfsetfillopacity{0.700000}%
\pgfsetlinewidth{0.000000pt}%
\definecolor{currentstroke}{rgb}{0.000000,0.000000,0.000000}%
\pgfsetstrokecolor{currentstroke}%
\pgfsetstrokeopacity{0.700000}%
\pgfsetdash{}{0pt}%
\pgfpathmoveto{\pgfqpoint{3.281072in}{0.550000in}}%
\pgfpathlineto{\pgfqpoint{3.308594in}{0.550000in}}%
\pgfpathlineto{\pgfqpoint{3.308594in}{2.627112in}}%
\pgfpathlineto{\pgfqpoint{3.281072in}{2.627112in}}%
\pgfpathlineto{\pgfqpoint{3.281072in}{0.550000in}}%
\pgfpathclose%
\pgfusepath{fill}%
\end{pgfscope}%
\begin{pgfscope}%
\pgfpathrectangle{\pgfqpoint{1.250000in}{0.550000in}}{\pgfqpoint{7.750000in}{3.850000in}}%
\pgfusepath{clip}%
\pgfsetbuttcap%
\pgfsetmiterjoin%
\definecolor{currentfill}{rgb}{0.501961,0.501961,0.501961}%
\pgfsetfillcolor{currentfill}%
\pgfsetfillopacity{0.700000}%
\pgfsetlinewidth{0.000000pt}%
\definecolor{currentstroke}{rgb}{0.000000,0.000000,0.000000}%
\pgfsetstrokecolor{currentstroke}%
\pgfsetstrokeopacity{0.700000}%
\pgfsetdash{}{0pt}%
\pgfpathmoveto{\pgfqpoint{3.308594in}{0.550000in}}%
\pgfpathlineto{\pgfqpoint{3.336115in}{0.550000in}}%
\pgfpathlineto{\pgfqpoint{3.336115in}{2.829549in}}%
\pgfpathlineto{\pgfqpoint{3.308594in}{2.829549in}}%
\pgfpathlineto{\pgfqpoint{3.308594in}{0.550000in}}%
\pgfpathclose%
\pgfusepath{fill}%
\end{pgfscope}%
\begin{pgfscope}%
\pgfpathrectangle{\pgfqpoint{1.250000in}{0.550000in}}{\pgfqpoint{7.750000in}{3.850000in}}%
\pgfusepath{clip}%
\pgfsetbuttcap%
\pgfsetmiterjoin%
\definecolor{currentfill}{rgb}{0.501961,0.501961,0.501961}%
\pgfsetfillcolor{currentfill}%
\pgfsetfillopacity{0.700000}%
\pgfsetlinewidth{0.000000pt}%
\definecolor{currentstroke}{rgb}{0.000000,0.000000,0.000000}%
\pgfsetstrokecolor{currentstroke}%
\pgfsetstrokeopacity{0.700000}%
\pgfsetdash{}{0pt}%
\pgfpathmoveto{\pgfqpoint{3.336115in}{0.550000in}}%
\pgfpathlineto{\pgfqpoint{3.363636in}{0.550000in}}%
\pgfpathlineto{\pgfqpoint{3.363636in}{2.724054in}}%
\pgfpathlineto{\pgfqpoint{3.336115in}{2.724054in}}%
\pgfpathlineto{\pgfqpoint{3.336115in}{0.550000in}}%
\pgfpathclose%
\pgfusepath{fill}%
\end{pgfscope}%
\begin{pgfscope}%
\pgfpathrectangle{\pgfqpoint{1.250000in}{0.550000in}}{\pgfqpoint{7.750000in}{3.850000in}}%
\pgfusepath{clip}%
\pgfsetbuttcap%
\pgfsetmiterjoin%
\definecolor{currentfill}{rgb}{0.501961,0.501961,0.501961}%
\pgfsetfillcolor{currentfill}%
\pgfsetfillopacity{0.700000}%
\pgfsetlinewidth{0.000000pt}%
\definecolor{currentstroke}{rgb}{0.000000,0.000000,0.000000}%
\pgfsetstrokecolor{currentstroke}%
\pgfsetstrokeopacity{0.700000}%
\pgfsetdash{}{0pt}%
\pgfpathmoveto{\pgfqpoint{3.363636in}{0.550000in}}%
\pgfpathlineto{\pgfqpoint{3.391158in}{0.550000in}}%
\pgfpathlineto{\pgfqpoint{3.391158in}{2.806739in}}%
\pgfpathlineto{\pgfqpoint{3.363636in}{2.806739in}}%
\pgfpathlineto{\pgfqpoint{3.363636in}{0.550000in}}%
\pgfpathclose%
\pgfusepath{fill}%
\end{pgfscope}%
\begin{pgfscope}%
\pgfpathrectangle{\pgfqpoint{1.250000in}{0.550000in}}{\pgfqpoint{7.750000in}{3.850000in}}%
\pgfusepath{clip}%
\pgfsetbuttcap%
\pgfsetmiterjoin%
\definecolor{currentfill}{rgb}{0.501961,0.501961,0.501961}%
\pgfsetfillcolor{currentfill}%
\pgfsetfillopacity{0.700000}%
\pgfsetlinewidth{0.000000pt}%
\definecolor{currentstroke}{rgb}{0.000000,0.000000,0.000000}%
\pgfsetstrokecolor{currentstroke}%
\pgfsetstrokeopacity{0.700000}%
\pgfsetdash{}{0pt}%
\pgfpathmoveto{\pgfqpoint{3.391158in}{0.550000in}}%
\pgfpathlineto{\pgfqpoint{3.418679in}{0.550000in}}%
\pgfpathlineto{\pgfqpoint{3.418679in}{2.734033in}}%
\pgfpathlineto{\pgfqpoint{3.391158in}{2.734033in}}%
\pgfpathlineto{\pgfqpoint{3.391158in}{0.550000in}}%
\pgfpathclose%
\pgfusepath{fill}%
\end{pgfscope}%
\begin{pgfscope}%
\pgfpathrectangle{\pgfqpoint{1.250000in}{0.550000in}}{\pgfqpoint{7.750000in}{3.850000in}}%
\pgfusepath{clip}%
\pgfsetbuttcap%
\pgfsetmiterjoin%
\definecolor{currentfill}{rgb}{0.501961,0.501961,0.501961}%
\pgfsetfillcolor{currentfill}%
\pgfsetfillopacity{0.700000}%
\pgfsetlinewidth{0.000000pt}%
\definecolor{currentstroke}{rgb}{0.000000,0.000000,0.000000}%
\pgfsetstrokecolor{currentstroke}%
\pgfsetstrokeopacity{0.700000}%
\pgfsetdash{}{0pt}%
\pgfpathmoveto{\pgfqpoint{3.418679in}{0.550000in}}%
\pgfpathlineto{\pgfqpoint{3.446200in}{0.550000in}}%
\pgfpathlineto{\pgfqpoint{3.446200in}{2.811016in}}%
\pgfpathlineto{\pgfqpoint{3.418679in}{2.811016in}}%
\pgfpathlineto{\pgfqpoint{3.418679in}{0.550000in}}%
\pgfpathclose%
\pgfusepath{fill}%
\end{pgfscope}%
\begin{pgfscope}%
\pgfpathrectangle{\pgfqpoint{1.250000in}{0.550000in}}{\pgfqpoint{7.750000in}{3.850000in}}%
\pgfusepath{clip}%
\pgfsetbuttcap%
\pgfsetmiterjoin%
\definecolor{currentfill}{rgb}{0.501961,0.501961,0.501961}%
\pgfsetfillcolor{currentfill}%
\pgfsetfillopacity{0.700000}%
\pgfsetlinewidth{0.000000pt}%
\definecolor{currentstroke}{rgb}{0.000000,0.000000,0.000000}%
\pgfsetstrokecolor{currentstroke}%
\pgfsetstrokeopacity{0.700000}%
\pgfsetdash{}{0pt}%
\pgfpathmoveto{\pgfqpoint{3.446200in}{0.550000in}}%
\pgfpathlineto{\pgfqpoint{3.473722in}{0.550000in}}%
\pgfpathlineto{\pgfqpoint{3.473722in}{2.752566in}}%
\pgfpathlineto{\pgfqpoint{3.446200in}{2.752566in}}%
\pgfpathlineto{\pgfqpoint{3.446200in}{0.550000in}}%
\pgfpathclose%
\pgfusepath{fill}%
\end{pgfscope}%
\begin{pgfscope}%
\pgfpathrectangle{\pgfqpoint{1.250000in}{0.550000in}}{\pgfqpoint{7.750000in}{3.850000in}}%
\pgfusepath{clip}%
\pgfsetbuttcap%
\pgfsetmiterjoin%
\definecolor{currentfill}{rgb}{0.501961,0.501961,0.501961}%
\pgfsetfillcolor{currentfill}%
\pgfsetfillopacity{0.700000}%
\pgfsetlinewidth{0.000000pt}%
\definecolor{currentstroke}{rgb}{0.000000,0.000000,0.000000}%
\pgfsetstrokecolor{currentstroke}%
\pgfsetstrokeopacity{0.700000}%
\pgfsetdash{}{0pt}%
\pgfpathmoveto{\pgfqpoint{3.473722in}{0.550000in}}%
\pgfpathlineto{\pgfqpoint{3.501243in}{0.550000in}}%
\pgfpathlineto{\pgfqpoint{3.501243in}{2.714075in}}%
\pgfpathlineto{\pgfqpoint{3.473722in}{2.714075in}}%
\pgfpathlineto{\pgfqpoint{3.473722in}{0.550000in}}%
\pgfpathclose%
\pgfusepath{fill}%
\end{pgfscope}%
\begin{pgfscope}%
\pgfpathrectangle{\pgfqpoint{1.250000in}{0.550000in}}{\pgfqpoint{7.750000in}{3.850000in}}%
\pgfusepath{clip}%
\pgfsetbuttcap%
\pgfsetmiterjoin%
\definecolor{currentfill}{rgb}{0.501961,0.501961,0.501961}%
\pgfsetfillcolor{currentfill}%
\pgfsetfillopacity{0.700000}%
\pgfsetlinewidth{0.000000pt}%
\definecolor{currentstroke}{rgb}{0.000000,0.000000,0.000000}%
\pgfsetstrokecolor{currentstroke}%
\pgfsetstrokeopacity{0.700000}%
\pgfsetdash{}{0pt}%
\pgfpathmoveto{\pgfqpoint{3.501243in}{0.550000in}}%
\pgfpathlineto{\pgfqpoint{3.528764in}{0.550000in}}%
\pgfpathlineto{\pgfqpoint{3.528764in}{2.724054in}}%
\pgfpathlineto{\pgfqpoint{3.501243in}{2.724054in}}%
\pgfpathlineto{\pgfqpoint{3.501243in}{0.550000in}}%
\pgfpathclose%
\pgfusepath{fill}%
\end{pgfscope}%
\begin{pgfscope}%
\pgfpathrectangle{\pgfqpoint{1.250000in}{0.550000in}}{\pgfqpoint{7.750000in}{3.850000in}}%
\pgfusepath{clip}%
\pgfsetbuttcap%
\pgfsetmiterjoin%
\definecolor{currentfill}{rgb}{0.501961,0.501961,0.501961}%
\pgfsetfillcolor{currentfill}%
\pgfsetfillopacity{0.700000}%
\pgfsetlinewidth{0.000000pt}%
\definecolor{currentstroke}{rgb}{0.000000,0.000000,0.000000}%
\pgfsetstrokecolor{currentstroke}%
\pgfsetstrokeopacity{0.700000}%
\pgfsetdash{}{0pt}%
\pgfpathmoveto{\pgfqpoint{3.528764in}{0.550000in}}%
\pgfpathlineto{\pgfqpoint{3.556286in}{0.550000in}}%
\pgfpathlineto{\pgfqpoint{3.556286in}{2.719777in}}%
\pgfpathlineto{\pgfqpoint{3.528764in}{2.719777in}}%
\pgfpathlineto{\pgfqpoint{3.528764in}{0.550000in}}%
\pgfpathclose%
\pgfusepath{fill}%
\end{pgfscope}%
\begin{pgfscope}%
\pgfpathrectangle{\pgfqpoint{1.250000in}{0.550000in}}{\pgfqpoint{7.750000in}{3.850000in}}%
\pgfusepath{clip}%
\pgfsetbuttcap%
\pgfsetmiterjoin%
\definecolor{currentfill}{rgb}{0.501961,0.501961,0.501961}%
\pgfsetfillcolor{currentfill}%
\pgfsetfillopacity{0.700000}%
\pgfsetlinewidth{0.000000pt}%
\definecolor{currentstroke}{rgb}{0.000000,0.000000,0.000000}%
\pgfsetstrokecolor{currentstroke}%
\pgfsetstrokeopacity{0.700000}%
\pgfsetdash{}{0pt}%
\pgfpathmoveto{\pgfqpoint{3.556286in}{0.550000in}}%
\pgfpathlineto{\pgfqpoint{3.583807in}{0.550000in}}%
\pgfpathlineto{\pgfqpoint{3.583807in}{2.849508in}}%
\pgfpathlineto{\pgfqpoint{3.556286in}{2.849508in}}%
\pgfpathlineto{\pgfqpoint{3.556286in}{0.550000in}}%
\pgfpathclose%
\pgfusepath{fill}%
\end{pgfscope}%
\begin{pgfscope}%
\pgfpathrectangle{\pgfqpoint{1.250000in}{0.550000in}}{\pgfqpoint{7.750000in}{3.850000in}}%
\pgfusepath{clip}%
\pgfsetbuttcap%
\pgfsetmiterjoin%
\definecolor{currentfill}{rgb}{0.501961,0.501961,0.501961}%
\pgfsetfillcolor{currentfill}%
\pgfsetfillopacity{0.700000}%
\pgfsetlinewidth{0.000000pt}%
\definecolor{currentstroke}{rgb}{0.000000,0.000000,0.000000}%
\pgfsetstrokecolor{currentstroke}%
\pgfsetstrokeopacity{0.700000}%
\pgfsetdash{}{0pt}%
\pgfpathmoveto{\pgfqpoint{3.583807in}{0.550000in}}%
\pgfpathlineto{\pgfqpoint{3.611328in}{0.550000in}}%
\pgfpathlineto{\pgfqpoint{3.611328in}{3.017729in}}%
\pgfpathlineto{\pgfqpoint{3.583807in}{3.017729in}}%
\pgfpathlineto{\pgfqpoint{3.583807in}{0.550000in}}%
\pgfpathclose%
\pgfusepath{fill}%
\end{pgfscope}%
\begin{pgfscope}%
\pgfpathrectangle{\pgfqpoint{1.250000in}{0.550000in}}{\pgfqpoint{7.750000in}{3.850000in}}%
\pgfusepath{clip}%
\pgfsetbuttcap%
\pgfsetmiterjoin%
\definecolor{currentfill}{rgb}{0.501961,0.501961,0.501961}%
\pgfsetfillcolor{currentfill}%
\pgfsetfillopacity{0.700000}%
\pgfsetlinewidth{0.000000pt}%
\definecolor{currentstroke}{rgb}{0.000000,0.000000,0.000000}%
\pgfsetstrokecolor{currentstroke}%
\pgfsetstrokeopacity{0.700000}%
\pgfsetdash{}{0pt}%
\pgfpathmoveto{\pgfqpoint{3.611328in}{0.550000in}}%
\pgfpathlineto{\pgfqpoint{3.638849in}{0.550000in}}%
\pgfpathlineto{\pgfqpoint{3.638849in}{2.993494in}}%
\pgfpathlineto{\pgfqpoint{3.611328in}{2.993494in}}%
\pgfpathlineto{\pgfqpoint{3.611328in}{0.550000in}}%
\pgfpathclose%
\pgfusepath{fill}%
\end{pgfscope}%
\begin{pgfscope}%
\pgfpathrectangle{\pgfqpoint{1.250000in}{0.550000in}}{\pgfqpoint{7.750000in}{3.850000in}}%
\pgfusepath{clip}%
\pgfsetbuttcap%
\pgfsetmiterjoin%
\definecolor{currentfill}{rgb}{0.501961,0.501961,0.501961}%
\pgfsetfillcolor{currentfill}%
\pgfsetfillopacity{0.700000}%
\pgfsetlinewidth{0.000000pt}%
\definecolor{currentstroke}{rgb}{0.000000,0.000000,0.000000}%
\pgfsetstrokecolor{currentstroke}%
\pgfsetstrokeopacity{0.700000}%
\pgfsetdash{}{0pt}%
\pgfpathmoveto{\pgfqpoint{3.638849in}{0.550000in}}%
\pgfpathlineto{\pgfqpoint{3.666371in}{0.550000in}}%
\pgfpathlineto{\pgfqpoint{3.666371in}{2.900829in}}%
\pgfpathlineto{\pgfqpoint{3.638849in}{2.900829in}}%
\pgfpathlineto{\pgfqpoint{3.638849in}{0.550000in}}%
\pgfpathclose%
\pgfusepath{fill}%
\end{pgfscope}%
\begin{pgfscope}%
\pgfpathrectangle{\pgfqpoint{1.250000in}{0.550000in}}{\pgfqpoint{7.750000in}{3.850000in}}%
\pgfusepath{clip}%
\pgfsetbuttcap%
\pgfsetmiterjoin%
\definecolor{currentfill}{rgb}{0.501961,0.501961,0.501961}%
\pgfsetfillcolor{currentfill}%
\pgfsetfillopacity{0.700000}%
\pgfsetlinewidth{0.000000pt}%
\definecolor{currentstroke}{rgb}{0.000000,0.000000,0.000000}%
\pgfsetstrokecolor{currentstroke}%
\pgfsetstrokeopacity{0.700000}%
\pgfsetdash{}{0pt}%
\pgfpathmoveto{\pgfqpoint{3.666371in}{0.550000in}}%
\pgfpathlineto{\pgfqpoint{3.693892in}{0.550000in}}%
\pgfpathlineto{\pgfqpoint{3.693892in}{2.853784in}}%
\pgfpathlineto{\pgfqpoint{3.666371in}{2.853784in}}%
\pgfpathlineto{\pgfqpoint{3.666371in}{0.550000in}}%
\pgfpathclose%
\pgfusepath{fill}%
\end{pgfscope}%
\begin{pgfscope}%
\pgfpathrectangle{\pgfqpoint{1.250000in}{0.550000in}}{\pgfqpoint{7.750000in}{3.850000in}}%
\pgfusepath{clip}%
\pgfsetbuttcap%
\pgfsetmiterjoin%
\definecolor{currentfill}{rgb}{0.501961,0.501961,0.501961}%
\pgfsetfillcolor{currentfill}%
\pgfsetfillopacity{0.700000}%
\pgfsetlinewidth{0.000000pt}%
\definecolor{currentstroke}{rgb}{0.000000,0.000000,0.000000}%
\pgfsetstrokecolor{currentstroke}%
\pgfsetstrokeopacity{0.700000}%
\pgfsetdash{}{0pt}%
\pgfpathmoveto{\pgfqpoint{3.693892in}{0.550000in}}%
\pgfpathlineto{\pgfqpoint{3.721413in}{0.550000in}}%
\pgfpathlineto{\pgfqpoint{3.721413in}{2.791058in}}%
\pgfpathlineto{\pgfqpoint{3.693892in}{2.791058in}}%
\pgfpathlineto{\pgfqpoint{3.693892in}{0.550000in}}%
\pgfpathclose%
\pgfusepath{fill}%
\end{pgfscope}%
\begin{pgfscope}%
\pgfpathrectangle{\pgfqpoint{1.250000in}{0.550000in}}{\pgfqpoint{7.750000in}{3.850000in}}%
\pgfusepath{clip}%
\pgfsetbuttcap%
\pgfsetmiterjoin%
\definecolor{currentfill}{rgb}{0.501961,0.501961,0.501961}%
\pgfsetfillcolor{currentfill}%
\pgfsetfillopacity{0.700000}%
\pgfsetlinewidth{0.000000pt}%
\definecolor{currentstroke}{rgb}{0.000000,0.000000,0.000000}%
\pgfsetstrokecolor{currentstroke}%
\pgfsetstrokeopacity{0.700000}%
\pgfsetdash{}{0pt}%
\pgfpathmoveto{\pgfqpoint{3.721413in}{0.550000in}}%
\pgfpathlineto{\pgfqpoint{3.748935in}{0.550000in}}%
\pgfpathlineto{\pgfqpoint{3.748935in}{2.816719in}}%
\pgfpathlineto{\pgfqpoint{3.721413in}{2.816719in}}%
\pgfpathlineto{\pgfqpoint{3.721413in}{0.550000in}}%
\pgfpathclose%
\pgfusepath{fill}%
\end{pgfscope}%
\begin{pgfscope}%
\pgfpathrectangle{\pgfqpoint{1.250000in}{0.550000in}}{\pgfqpoint{7.750000in}{3.850000in}}%
\pgfusepath{clip}%
\pgfsetbuttcap%
\pgfsetmiterjoin%
\definecolor{currentfill}{rgb}{0.501961,0.501961,0.501961}%
\pgfsetfillcolor{currentfill}%
\pgfsetfillopacity{0.700000}%
\pgfsetlinewidth{0.000000pt}%
\definecolor{currentstroke}{rgb}{0.000000,0.000000,0.000000}%
\pgfsetstrokecolor{currentstroke}%
\pgfsetstrokeopacity{0.700000}%
\pgfsetdash{}{0pt}%
\pgfpathmoveto{\pgfqpoint{3.748935in}{0.550000in}}%
\pgfpathlineto{\pgfqpoint{3.776456in}{0.550000in}}%
\pgfpathlineto{\pgfqpoint{3.776456in}{0.550000in}}%
\pgfpathlineto{\pgfqpoint{3.748935in}{0.550000in}}%
\pgfpathlineto{\pgfqpoint{3.748935in}{0.550000in}}%
\pgfpathclose%
\pgfusepath{fill}%
\end{pgfscope}%
\begin{pgfscope}%
\pgfpathrectangle{\pgfqpoint{1.250000in}{0.550000in}}{\pgfqpoint{7.750000in}{3.850000in}}%
\pgfusepath{clip}%
\pgfsetbuttcap%
\pgfsetmiterjoin%
\definecolor{currentfill}{rgb}{0.501961,0.501961,0.501961}%
\pgfsetfillcolor{currentfill}%
\pgfsetfillopacity{0.700000}%
\pgfsetlinewidth{0.000000pt}%
\definecolor{currentstroke}{rgb}{0.000000,0.000000,0.000000}%
\pgfsetstrokecolor{currentstroke}%
\pgfsetstrokeopacity{0.700000}%
\pgfsetdash{}{0pt}%
\pgfpathmoveto{\pgfqpoint{3.776456in}{0.550000in}}%
\pgfpathlineto{\pgfqpoint{3.803977in}{0.550000in}}%
\pgfpathlineto{\pgfqpoint{3.803977in}{2.843805in}}%
\pgfpathlineto{\pgfqpoint{3.776456in}{2.843805in}}%
\pgfpathlineto{\pgfqpoint{3.776456in}{0.550000in}}%
\pgfpathclose%
\pgfusepath{fill}%
\end{pgfscope}%
\begin{pgfscope}%
\pgfpathrectangle{\pgfqpoint{1.250000in}{0.550000in}}{\pgfqpoint{7.750000in}{3.850000in}}%
\pgfusepath{clip}%
\pgfsetbuttcap%
\pgfsetmiterjoin%
\definecolor{currentfill}{rgb}{0.501961,0.501961,0.501961}%
\pgfsetfillcolor{currentfill}%
\pgfsetfillopacity{0.700000}%
\pgfsetlinewidth{0.000000pt}%
\definecolor{currentstroke}{rgb}{0.000000,0.000000,0.000000}%
\pgfsetstrokecolor{currentstroke}%
\pgfsetstrokeopacity{0.700000}%
\pgfsetdash{}{0pt}%
\pgfpathmoveto{\pgfqpoint{3.803977in}{0.550000in}}%
\pgfpathlineto{\pgfqpoint{3.831499in}{0.550000in}}%
\pgfpathlineto{\pgfqpoint{3.831499in}{2.785355in}}%
\pgfpathlineto{\pgfqpoint{3.803977in}{2.785355in}}%
\pgfpathlineto{\pgfqpoint{3.803977in}{0.550000in}}%
\pgfpathclose%
\pgfusepath{fill}%
\end{pgfscope}%
\begin{pgfscope}%
\pgfpathrectangle{\pgfqpoint{1.250000in}{0.550000in}}{\pgfqpoint{7.750000in}{3.850000in}}%
\pgfusepath{clip}%
\pgfsetbuttcap%
\pgfsetmiterjoin%
\definecolor{currentfill}{rgb}{0.501961,0.501961,0.501961}%
\pgfsetfillcolor{currentfill}%
\pgfsetfillopacity{0.700000}%
\pgfsetlinewidth{0.000000pt}%
\definecolor{currentstroke}{rgb}{0.000000,0.000000,0.000000}%
\pgfsetstrokecolor{currentstroke}%
\pgfsetstrokeopacity{0.700000}%
\pgfsetdash{}{0pt}%
\pgfpathmoveto{\pgfqpoint{3.831499in}{0.550000in}}%
\pgfpathlineto{\pgfqpoint{3.859020in}{0.550000in}}%
\pgfpathlineto{\pgfqpoint{3.859020in}{2.869466in}}%
\pgfpathlineto{\pgfqpoint{3.831499in}{2.869466in}}%
\pgfpathlineto{\pgfqpoint{3.831499in}{0.550000in}}%
\pgfpathclose%
\pgfusepath{fill}%
\end{pgfscope}%
\begin{pgfscope}%
\pgfpathrectangle{\pgfqpoint{1.250000in}{0.550000in}}{\pgfqpoint{7.750000in}{3.850000in}}%
\pgfusepath{clip}%
\pgfsetbuttcap%
\pgfsetmiterjoin%
\definecolor{currentfill}{rgb}{0.501961,0.501961,0.501961}%
\pgfsetfillcolor{currentfill}%
\pgfsetfillopacity{0.700000}%
\pgfsetlinewidth{0.000000pt}%
\definecolor{currentstroke}{rgb}{0.000000,0.000000,0.000000}%
\pgfsetstrokecolor{currentstroke}%
\pgfsetstrokeopacity{0.700000}%
\pgfsetdash{}{0pt}%
\pgfpathmoveto{\pgfqpoint{3.859020in}{0.550000in}}%
\pgfpathlineto{\pgfqpoint{3.886541in}{0.550000in}}%
\pgfpathlineto{\pgfqpoint{3.886541in}{2.759694in}}%
\pgfpathlineto{\pgfqpoint{3.859020in}{2.759694in}}%
\pgfpathlineto{\pgfqpoint{3.859020in}{0.550000in}}%
\pgfpathclose%
\pgfusepath{fill}%
\end{pgfscope}%
\begin{pgfscope}%
\pgfpathrectangle{\pgfqpoint{1.250000in}{0.550000in}}{\pgfqpoint{7.750000in}{3.850000in}}%
\pgfusepath{clip}%
\pgfsetbuttcap%
\pgfsetmiterjoin%
\definecolor{currentfill}{rgb}{0.501961,0.501961,0.501961}%
\pgfsetfillcolor{currentfill}%
\pgfsetfillopacity{0.700000}%
\pgfsetlinewidth{0.000000pt}%
\definecolor{currentstroke}{rgb}{0.000000,0.000000,0.000000}%
\pgfsetstrokecolor{currentstroke}%
\pgfsetstrokeopacity{0.700000}%
\pgfsetdash{}{0pt}%
\pgfpathmoveto{\pgfqpoint{3.886541in}{0.550000in}}%
\pgfpathlineto{\pgfqpoint{3.914062in}{0.550000in}}%
\pgfpathlineto{\pgfqpoint{3.914062in}{2.836677in}}%
\pgfpathlineto{\pgfqpoint{3.886541in}{2.836677in}}%
\pgfpathlineto{\pgfqpoint{3.886541in}{0.550000in}}%
\pgfpathclose%
\pgfusepath{fill}%
\end{pgfscope}%
\begin{pgfscope}%
\pgfpathrectangle{\pgfqpoint{1.250000in}{0.550000in}}{\pgfqpoint{7.750000in}{3.850000in}}%
\pgfusepath{clip}%
\pgfsetbuttcap%
\pgfsetmiterjoin%
\definecolor{currentfill}{rgb}{0.501961,0.501961,0.501961}%
\pgfsetfillcolor{currentfill}%
\pgfsetfillopacity{0.700000}%
\pgfsetlinewidth{0.000000pt}%
\definecolor{currentstroke}{rgb}{0.000000,0.000000,0.000000}%
\pgfsetstrokecolor{currentstroke}%
\pgfsetstrokeopacity{0.700000}%
\pgfsetdash{}{0pt}%
\pgfpathmoveto{\pgfqpoint{3.914062in}{0.550000in}}%
\pgfpathlineto{\pgfqpoint{3.941584in}{0.550000in}}%
\pgfpathlineto{\pgfqpoint{3.941584in}{2.749715in}}%
\pgfpathlineto{\pgfqpoint{3.914062in}{2.749715in}}%
\pgfpathlineto{\pgfqpoint{3.914062in}{0.550000in}}%
\pgfpathclose%
\pgfusepath{fill}%
\end{pgfscope}%
\begin{pgfscope}%
\pgfpathrectangle{\pgfqpoint{1.250000in}{0.550000in}}{\pgfqpoint{7.750000in}{3.850000in}}%
\pgfusepath{clip}%
\pgfsetbuttcap%
\pgfsetmiterjoin%
\definecolor{currentfill}{rgb}{0.501961,0.501961,0.501961}%
\pgfsetfillcolor{currentfill}%
\pgfsetfillopacity{0.700000}%
\pgfsetlinewidth{0.000000pt}%
\definecolor{currentstroke}{rgb}{0.000000,0.000000,0.000000}%
\pgfsetstrokecolor{currentstroke}%
\pgfsetstrokeopacity{0.700000}%
\pgfsetdash{}{0pt}%
\pgfpathmoveto{\pgfqpoint{3.941584in}{0.550000in}}%
\pgfpathlineto{\pgfqpoint{3.969105in}{0.550000in}}%
\pgfpathlineto{\pgfqpoint{3.969105in}{2.762545in}}%
\pgfpathlineto{\pgfqpoint{3.941584in}{2.762545in}}%
\pgfpathlineto{\pgfqpoint{3.941584in}{0.550000in}}%
\pgfpathclose%
\pgfusepath{fill}%
\end{pgfscope}%
\begin{pgfscope}%
\pgfpathrectangle{\pgfqpoint{1.250000in}{0.550000in}}{\pgfqpoint{7.750000in}{3.850000in}}%
\pgfusepath{clip}%
\pgfsetbuttcap%
\pgfsetmiterjoin%
\definecolor{currentfill}{rgb}{0.501961,0.501961,0.501961}%
\pgfsetfillcolor{currentfill}%
\pgfsetfillopacity{0.700000}%
\pgfsetlinewidth{0.000000pt}%
\definecolor{currentstroke}{rgb}{0.000000,0.000000,0.000000}%
\pgfsetstrokecolor{currentstroke}%
\pgfsetstrokeopacity{0.700000}%
\pgfsetdash{}{0pt}%
\pgfpathmoveto{\pgfqpoint{3.969105in}{0.550000in}}%
\pgfpathlineto{\pgfqpoint{3.996626in}{0.550000in}}%
\pgfpathlineto{\pgfqpoint{3.996626in}{2.719777in}}%
\pgfpathlineto{\pgfqpoint{3.969105in}{2.719777in}}%
\pgfpathlineto{\pgfqpoint{3.969105in}{0.550000in}}%
\pgfpathclose%
\pgfusepath{fill}%
\end{pgfscope}%
\begin{pgfscope}%
\pgfpathrectangle{\pgfqpoint{1.250000in}{0.550000in}}{\pgfqpoint{7.750000in}{3.850000in}}%
\pgfusepath{clip}%
\pgfsetbuttcap%
\pgfsetmiterjoin%
\definecolor{currentfill}{rgb}{0.501961,0.501961,0.501961}%
\pgfsetfillcolor{currentfill}%
\pgfsetfillopacity{0.700000}%
\pgfsetlinewidth{0.000000pt}%
\definecolor{currentstroke}{rgb}{0.000000,0.000000,0.000000}%
\pgfsetstrokecolor{currentstroke}%
\pgfsetstrokeopacity{0.700000}%
\pgfsetdash{}{0pt}%
\pgfpathmoveto{\pgfqpoint{3.996626in}{0.550000in}}%
\pgfpathlineto{\pgfqpoint{4.024148in}{0.550000in}}%
\pgfpathlineto{\pgfqpoint{4.024148in}{2.637092in}}%
\pgfpathlineto{\pgfqpoint{3.996626in}{2.637092in}}%
\pgfpathlineto{\pgfqpoint{3.996626in}{0.550000in}}%
\pgfpathclose%
\pgfusepath{fill}%
\end{pgfscope}%
\begin{pgfscope}%
\pgfpathrectangle{\pgfqpoint{1.250000in}{0.550000in}}{\pgfqpoint{7.750000in}{3.850000in}}%
\pgfusepath{clip}%
\pgfsetbuttcap%
\pgfsetmiterjoin%
\definecolor{currentfill}{rgb}{0.501961,0.501961,0.501961}%
\pgfsetfillcolor{currentfill}%
\pgfsetfillopacity{0.700000}%
\pgfsetlinewidth{0.000000pt}%
\definecolor{currentstroke}{rgb}{0.000000,0.000000,0.000000}%
\pgfsetstrokecolor{currentstroke}%
\pgfsetstrokeopacity{0.700000}%
\pgfsetdash{}{0pt}%
\pgfpathmoveto{\pgfqpoint{4.024148in}{0.550000in}}%
\pgfpathlineto{\pgfqpoint{4.051669in}{0.550000in}}%
\pgfpathlineto{\pgfqpoint{4.051669in}{2.608580in}}%
\pgfpathlineto{\pgfqpoint{4.024148in}{2.608580in}}%
\pgfpathlineto{\pgfqpoint{4.024148in}{0.550000in}}%
\pgfpathclose%
\pgfusepath{fill}%
\end{pgfscope}%
\begin{pgfscope}%
\pgfpathrectangle{\pgfqpoint{1.250000in}{0.550000in}}{\pgfqpoint{7.750000in}{3.850000in}}%
\pgfusepath{clip}%
\pgfsetbuttcap%
\pgfsetmiterjoin%
\definecolor{currentfill}{rgb}{0.501961,0.501961,0.501961}%
\pgfsetfillcolor{currentfill}%
\pgfsetfillopacity{0.700000}%
\pgfsetlinewidth{0.000000pt}%
\definecolor{currentstroke}{rgb}{0.000000,0.000000,0.000000}%
\pgfsetstrokecolor{currentstroke}%
\pgfsetstrokeopacity{0.700000}%
\pgfsetdash{}{0pt}%
\pgfpathmoveto{\pgfqpoint{4.051669in}{0.550000in}}%
\pgfpathlineto{\pgfqpoint{4.079190in}{0.550000in}}%
\pgfpathlineto{\pgfqpoint{4.079190in}{2.685562in}}%
\pgfpathlineto{\pgfqpoint{4.051669in}{2.685562in}}%
\pgfpathlineto{\pgfqpoint{4.051669in}{0.550000in}}%
\pgfpathclose%
\pgfusepath{fill}%
\end{pgfscope}%
\begin{pgfscope}%
\pgfpathrectangle{\pgfqpoint{1.250000in}{0.550000in}}{\pgfqpoint{7.750000in}{3.850000in}}%
\pgfusepath{clip}%
\pgfsetbuttcap%
\pgfsetmiterjoin%
\definecolor{currentfill}{rgb}{0.501961,0.501961,0.501961}%
\pgfsetfillcolor{currentfill}%
\pgfsetfillopacity{0.700000}%
\pgfsetlinewidth{0.000000pt}%
\definecolor{currentstroke}{rgb}{0.000000,0.000000,0.000000}%
\pgfsetstrokecolor{currentstroke}%
\pgfsetstrokeopacity{0.700000}%
\pgfsetdash{}{0pt}%
\pgfpathmoveto{\pgfqpoint{4.079190in}{0.550000in}}%
\pgfpathlineto{\pgfqpoint{4.106712in}{0.550000in}}%
\pgfpathlineto{\pgfqpoint{4.106712in}{2.578642in}}%
\pgfpathlineto{\pgfqpoint{4.079190in}{2.578642in}}%
\pgfpathlineto{\pgfqpoint{4.079190in}{0.550000in}}%
\pgfpathclose%
\pgfusepath{fill}%
\end{pgfscope}%
\begin{pgfscope}%
\pgfpathrectangle{\pgfqpoint{1.250000in}{0.550000in}}{\pgfqpoint{7.750000in}{3.850000in}}%
\pgfusepath{clip}%
\pgfsetbuttcap%
\pgfsetmiterjoin%
\definecolor{currentfill}{rgb}{0.501961,0.501961,0.501961}%
\pgfsetfillcolor{currentfill}%
\pgfsetfillopacity{0.700000}%
\pgfsetlinewidth{0.000000pt}%
\definecolor{currentstroke}{rgb}{0.000000,0.000000,0.000000}%
\pgfsetstrokecolor{currentstroke}%
\pgfsetstrokeopacity{0.700000}%
\pgfsetdash{}{0pt}%
\pgfpathmoveto{\pgfqpoint{4.106712in}{0.550000in}}%
\pgfpathlineto{\pgfqpoint{4.134233in}{0.550000in}}%
\pgfpathlineto{\pgfqpoint{4.134233in}{2.448911in}}%
\pgfpathlineto{\pgfqpoint{4.106712in}{2.448911in}}%
\pgfpathlineto{\pgfqpoint{4.106712in}{0.550000in}}%
\pgfpathclose%
\pgfusepath{fill}%
\end{pgfscope}%
\begin{pgfscope}%
\pgfpathrectangle{\pgfqpoint{1.250000in}{0.550000in}}{\pgfqpoint{7.750000in}{3.850000in}}%
\pgfusepath{clip}%
\pgfsetbuttcap%
\pgfsetmiterjoin%
\definecolor{currentfill}{rgb}{0.501961,0.501961,0.501961}%
\pgfsetfillcolor{currentfill}%
\pgfsetfillopacity{0.700000}%
\pgfsetlinewidth{0.000000pt}%
\definecolor{currentstroke}{rgb}{0.000000,0.000000,0.000000}%
\pgfsetstrokecolor{currentstroke}%
\pgfsetstrokeopacity{0.700000}%
\pgfsetdash{}{0pt}%
\pgfpathmoveto{\pgfqpoint{4.134233in}{0.550000in}}%
\pgfpathlineto{\pgfqpoint{4.161754in}{0.550000in}}%
\pgfpathlineto{\pgfqpoint{4.161754in}{4.216667in}}%
\pgfpathlineto{\pgfqpoint{4.134233in}{4.216667in}}%
\pgfpathlineto{\pgfqpoint{4.134233in}{0.550000in}}%
\pgfpathclose%
\pgfusepath{fill}%
\end{pgfscope}%
\begin{pgfscope}%
\pgfpathrectangle{\pgfqpoint{1.250000in}{0.550000in}}{\pgfqpoint{7.750000in}{3.850000in}}%
\pgfusepath{clip}%
\pgfsetbuttcap%
\pgfsetmiterjoin%
\definecolor{currentfill}{rgb}{0.501961,0.501961,0.501961}%
\pgfsetfillcolor{currentfill}%
\pgfsetfillopacity{0.700000}%
\pgfsetlinewidth{0.000000pt}%
\definecolor{currentstroke}{rgb}{0.000000,0.000000,0.000000}%
\pgfsetstrokecolor{currentstroke}%
\pgfsetstrokeopacity{0.700000}%
\pgfsetdash{}{0pt}%
\pgfpathmoveto{\pgfqpoint{4.161754in}{0.550000in}}%
\pgfpathlineto{\pgfqpoint{4.189276in}{0.550000in}}%
\pgfpathlineto{\pgfqpoint{4.189276in}{2.441783in}}%
\pgfpathlineto{\pgfqpoint{4.161754in}{2.441783in}}%
\pgfpathlineto{\pgfqpoint{4.161754in}{0.550000in}}%
\pgfpathclose%
\pgfusepath{fill}%
\end{pgfscope}%
\begin{pgfscope}%
\pgfpathrectangle{\pgfqpoint{1.250000in}{0.550000in}}{\pgfqpoint{7.750000in}{3.850000in}}%
\pgfusepath{clip}%
\pgfsetbuttcap%
\pgfsetmiterjoin%
\definecolor{currentfill}{rgb}{0.501961,0.501961,0.501961}%
\pgfsetfillcolor{currentfill}%
\pgfsetfillopacity{0.700000}%
\pgfsetlinewidth{0.000000pt}%
\definecolor{currentstroke}{rgb}{0.000000,0.000000,0.000000}%
\pgfsetstrokecolor{currentstroke}%
\pgfsetstrokeopacity{0.700000}%
\pgfsetdash{}{0pt}%
\pgfpathmoveto{\pgfqpoint{4.189276in}{0.550000in}}%
\pgfpathlineto{\pgfqpoint{4.216797in}{0.550000in}}%
\pgfpathlineto{\pgfqpoint{4.216797in}{2.304925in}}%
\pgfpathlineto{\pgfqpoint{4.189276in}{2.304925in}}%
\pgfpathlineto{\pgfqpoint{4.189276in}{0.550000in}}%
\pgfpathclose%
\pgfusepath{fill}%
\end{pgfscope}%
\begin{pgfscope}%
\pgfpathrectangle{\pgfqpoint{1.250000in}{0.550000in}}{\pgfqpoint{7.750000in}{3.850000in}}%
\pgfusepath{clip}%
\pgfsetbuttcap%
\pgfsetmiterjoin%
\definecolor{currentfill}{rgb}{0.501961,0.501961,0.501961}%
\pgfsetfillcolor{currentfill}%
\pgfsetfillopacity{0.700000}%
\pgfsetlinewidth{0.000000pt}%
\definecolor{currentstroke}{rgb}{0.000000,0.000000,0.000000}%
\pgfsetstrokecolor{currentstroke}%
\pgfsetstrokeopacity{0.700000}%
\pgfsetdash{}{0pt}%
\pgfpathmoveto{\pgfqpoint{4.216797in}{0.550000in}}%
\pgfpathlineto{\pgfqpoint{4.244318in}{0.550000in}}%
\pgfpathlineto{\pgfqpoint{4.244318in}{2.282115in}}%
\pgfpathlineto{\pgfqpoint{4.216797in}{2.282115in}}%
\pgfpathlineto{\pgfqpoint{4.216797in}{0.550000in}}%
\pgfpathclose%
\pgfusepath{fill}%
\end{pgfscope}%
\begin{pgfscope}%
\pgfpathrectangle{\pgfqpoint{1.250000in}{0.550000in}}{\pgfqpoint{7.750000in}{3.850000in}}%
\pgfusepath{clip}%
\pgfsetbuttcap%
\pgfsetmiterjoin%
\definecolor{currentfill}{rgb}{0.501961,0.501961,0.501961}%
\pgfsetfillcolor{currentfill}%
\pgfsetfillopacity{0.700000}%
\pgfsetlinewidth{0.000000pt}%
\definecolor{currentstroke}{rgb}{0.000000,0.000000,0.000000}%
\pgfsetstrokecolor{currentstroke}%
\pgfsetstrokeopacity{0.700000}%
\pgfsetdash{}{0pt}%
\pgfpathmoveto{\pgfqpoint{4.244318in}{0.550000in}}%
\pgfpathlineto{\pgfqpoint{4.271839in}{0.550000in}}%
\pgfpathlineto{\pgfqpoint{4.271839in}{0.550000in}}%
\pgfpathlineto{\pgfqpoint{4.244318in}{0.550000in}}%
\pgfpathlineto{\pgfqpoint{4.244318in}{0.550000in}}%
\pgfpathclose%
\pgfusepath{fill}%
\end{pgfscope}%
\begin{pgfscope}%
\pgfpathrectangle{\pgfqpoint{1.250000in}{0.550000in}}{\pgfqpoint{7.750000in}{3.850000in}}%
\pgfusepath{clip}%
\pgfsetbuttcap%
\pgfsetmiterjoin%
\definecolor{currentfill}{rgb}{0.501961,0.501961,0.501961}%
\pgfsetfillcolor{currentfill}%
\pgfsetfillopacity{0.700000}%
\pgfsetlinewidth{0.000000pt}%
\definecolor{currentstroke}{rgb}{0.000000,0.000000,0.000000}%
\pgfsetstrokecolor{currentstroke}%
\pgfsetstrokeopacity{0.700000}%
\pgfsetdash{}{0pt}%
\pgfpathmoveto{\pgfqpoint{4.271839in}{0.550000in}}%
\pgfpathlineto{\pgfqpoint{4.299361in}{0.550000in}}%
\pgfpathlineto{\pgfqpoint{4.299361in}{2.289243in}}%
\pgfpathlineto{\pgfqpoint{4.271839in}{2.289243in}}%
\pgfpathlineto{\pgfqpoint{4.271839in}{0.550000in}}%
\pgfpathclose%
\pgfusepath{fill}%
\end{pgfscope}%
\begin{pgfscope}%
\pgfpathrectangle{\pgfqpoint{1.250000in}{0.550000in}}{\pgfqpoint{7.750000in}{3.850000in}}%
\pgfusepath{clip}%
\pgfsetbuttcap%
\pgfsetmiterjoin%
\definecolor{currentfill}{rgb}{0.501961,0.501961,0.501961}%
\pgfsetfillcolor{currentfill}%
\pgfsetfillopacity{0.700000}%
\pgfsetlinewidth{0.000000pt}%
\definecolor{currentstroke}{rgb}{0.000000,0.000000,0.000000}%
\pgfsetstrokecolor{currentstroke}%
\pgfsetstrokeopacity{0.700000}%
\pgfsetdash{}{0pt}%
\pgfpathmoveto{\pgfqpoint{4.299361in}{0.550000in}}%
\pgfpathlineto{\pgfqpoint{4.326882in}{0.550000in}}%
\pgfpathlineto{\pgfqpoint{4.326882in}{0.550000in}}%
\pgfpathlineto{\pgfqpoint{4.299361in}{0.550000in}}%
\pgfpathlineto{\pgfqpoint{4.299361in}{0.550000in}}%
\pgfpathclose%
\pgfusepath{fill}%
\end{pgfscope}%
\begin{pgfscope}%
\pgfpathrectangle{\pgfqpoint{1.250000in}{0.550000in}}{\pgfqpoint{7.750000in}{3.850000in}}%
\pgfusepath{clip}%
\pgfsetbuttcap%
\pgfsetmiterjoin%
\definecolor{currentfill}{rgb}{0.501961,0.501961,0.501961}%
\pgfsetfillcolor{currentfill}%
\pgfsetfillopacity{0.700000}%
\pgfsetlinewidth{0.000000pt}%
\definecolor{currentstroke}{rgb}{0.000000,0.000000,0.000000}%
\pgfsetstrokecolor{currentstroke}%
\pgfsetstrokeopacity{0.700000}%
\pgfsetdash{}{0pt}%
\pgfpathmoveto{\pgfqpoint{4.326882in}{0.550000in}}%
\pgfpathlineto{\pgfqpoint{4.354403in}{0.550000in}}%
\pgfpathlineto{\pgfqpoint{4.354403in}{2.129575in}}%
\pgfpathlineto{\pgfqpoint{4.326882in}{2.129575in}}%
\pgfpathlineto{\pgfqpoint{4.326882in}{0.550000in}}%
\pgfpathclose%
\pgfusepath{fill}%
\end{pgfscope}%
\begin{pgfscope}%
\pgfpathrectangle{\pgfqpoint{1.250000in}{0.550000in}}{\pgfqpoint{7.750000in}{3.850000in}}%
\pgfusepath{clip}%
\pgfsetbuttcap%
\pgfsetmiterjoin%
\definecolor{currentfill}{rgb}{0.501961,0.501961,0.501961}%
\pgfsetfillcolor{currentfill}%
\pgfsetfillopacity{0.700000}%
\pgfsetlinewidth{0.000000pt}%
\definecolor{currentstroke}{rgb}{0.000000,0.000000,0.000000}%
\pgfsetstrokecolor{currentstroke}%
\pgfsetstrokeopacity{0.700000}%
\pgfsetdash{}{0pt}%
\pgfpathmoveto{\pgfqpoint{4.354403in}{0.550000in}}%
\pgfpathlineto{\pgfqpoint{4.381925in}{0.550000in}}%
\pgfpathlineto{\pgfqpoint{4.381925in}{0.550000in}}%
\pgfpathlineto{\pgfqpoint{4.354403in}{0.550000in}}%
\pgfpathlineto{\pgfqpoint{4.354403in}{0.550000in}}%
\pgfpathclose%
\pgfusepath{fill}%
\end{pgfscope}%
\begin{pgfscope}%
\pgfpathrectangle{\pgfqpoint{1.250000in}{0.550000in}}{\pgfqpoint{7.750000in}{3.850000in}}%
\pgfusepath{clip}%
\pgfsetbuttcap%
\pgfsetmiterjoin%
\definecolor{currentfill}{rgb}{0.501961,0.501961,0.501961}%
\pgfsetfillcolor{currentfill}%
\pgfsetfillopacity{0.700000}%
\pgfsetlinewidth{0.000000pt}%
\definecolor{currentstroke}{rgb}{0.000000,0.000000,0.000000}%
\pgfsetstrokecolor{currentstroke}%
\pgfsetstrokeopacity{0.700000}%
\pgfsetdash{}{0pt}%
\pgfpathmoveto{\pgfqpoint{4.381925in}{0.550000in}}%
\pgfpathlineto{\pgfqpoint{4.409446in}{0.550000in}}%
\pgfpathlineto{\pgfqpoint{4.409446in}{2.280689in}}%
\pgfpathlineto{\pgfqpoint{4.381925in}{2.280689in}}%
\pgfpathlineto{\pgfqpoint{4.381925in}{0.550000in}}%
\pgfpathclose%
\pgfusepath{fill}%
\end{pgfscope}%
\begin{pgfscope}%
\pgfpathrectangle{\pgfqpoint{1.250000in}{0.550000in}}{\pgfqpoint{7.750000in}{3.850000in}}%
\pgfusepath{clip}%
\pgfsetbuttcap%
\pgfsetmiterjoin%
\definecolor{currentfill}{rgb}{0.501961,0.501961,0.501961}%
\pgfsetfillcolor{currentfill}%
\pgfsetfillopacity{0.700000}%
\pgfsetlinewidth{0.000000pt}%
\definecolor{currentstroke}{rgb}{0.000000,0.000000,0.000000}%
\pgfsetstrokecolor{currentstroke}%
\pgfsetstrokeopacity{0.700000}%
\pgfsetdash{}{0pt}%
\pgfpathmoveto{\pgfqpoint{4.409446in}{0.550000in}}%
\pgfpathlineto{\pgfqpoint{4.436967in}{0.550000in}}%
\pgfpathlineto{\pgfqpoint{4.436967in}{0.550000in}}%
\pgfpathlineto{\pgfqpoint{4.409446in}{0.550000in}}%
\pgfpathlineto{\pgfqpoint{4.409446in}{0.550000in}}%
\pgfpathclose%
\pgfusepath{fill}%
\end{pgfscope}%
\begin{pgfscope}%
\pgfpathrectangle{\pgfqpoint{1.250000in}{0.550000in}}{\pgfqpoint{7.750000in}{3.850000in}}%
\pgfusepath{clip}%
\pgfsetbuttcap%
\pgfsetmiterjoin%
\definecolor{currentfill}{rgb}{0.501961,0.501961,0.501961}%
\pgfsetfillcolor{currentfill}%
\pgfsetfillopacity{0.700000}%
\pgfsetlinewidth{0.000000pt}%
\definecolor{currentstroke}{rgb}{0.000000,0.000000,0.000000}%
\pgfsetstrokecolor{currentstroke}%
\pgfsetstrokeopacity{0.700000}%
\pgfsetdash{}{0pt}%
\pgfpathmoveto{\pgfqpoint{4.436967in}{0.550000in}}%
\pgfpathlineto{\pgfqpoint{4.464489in}{0.550000in}}%
\pgfpathlineto{\pgfqpoint{4.464489in}{2.073976in}}%
\pgfpathlineto{\pgfqpoint{4.436967in}{2.073976in}}%
\pgfpathlineto{\pgfqpoint{4.436967in}{0.550000in}}%
\pgfpathclose%
\pgfusepath{fill}%
\end{pgfscope}%
\begin{pgfscope}%
\pgfpathrectangle{\pgfqpoint{1.250000in}{0.550000in}}{\pgfqpoint{7.750000in}{3.850000in}}%
\pgfusepath{clip}%
\pgfsetbuttcap%
\pgfsetmiterjoin%
\definecolor{currentfill}{rgb}{0.501961,0.501961,0.501961}%
\pgfsetfillcolor{currentfill}%
\pgfsetfillopacity{0.700000}%
\pgfsetlinewidth{0.000000pt}%
\definecolor{currentstroke}{rgb}{0.000000,0.000000,0.000000}%
\pgfsetstrokecolor{currentstroke}%
\pgfsetstrokeopacity{0.700000}%
\pgfsetdash{}{0pt}%
\pgfpathmoveto{\pgfqpoint{4.464489in}{0.550000in}}%
\pgfpathlineto{\pgfqpoint{4.492010in}{0.550000in}}%
\pgfpathlineto{\pgfqpoint{4.492010in}{0.550000in}}%
\pgfpathlineto{\pgfqpoint{4.464489in}{0.550000in}}%
\pgfpathlineto{\pgfqpoint{4.464489in}{0.550000in}}%
\pgfpathclose%
\pgfusepath{fill}%
\end{pgfscope}%
\begin{pgfscope}%
\pgfpathrectangle{\pgfqpoint{1.250000in}{0.550000in}}{\pgfqpoint{7.750000in}{3.850000in}}%
\pgfusepath{clip}%
\pgfsetbuttcap%
\pgfsetmiterjoin%
\definecolor{currentfill}{rgb}{0.501961,0.501961,0.501961}%
\pgfsetfillcolor{currentfill}%
\pgfsetfillopacity{0.700000}%
\pgfsetlinewidth{0.000000pt}%
\definecolor{currentstroke}{rgb}{0.000000,0.000000,0.000000}%
\pgfsetstrokecolor{currentstroke}%
\pgfsetstrokeopacity{0.700000}%
\pgfsetdash{}{0pt}%
\pgfpathmoveto{\pgfqpoint{4.492010in}{0.550000in}}%
\pgfpathlineto{\pgfqpoint{4.519531in}{0.550000in}}%
\pgfpathlineto{\pgfqpoint{4.519531in}{2.142405in}}%
\pgfpathlineto{\pgfqpoint{4.492010in}{2.142405in}}%
\pgfpathlineto{\pgfqpoint{4.492010in}{0.550000in}}%
\pgfpathclose%
\pgfusepath{fill}%
\end{pgfscope}%
\begin{pgfscope}%
\pgfpathrectangle{\pgfqpoint{1.250000in}{0.550000in}}{\pgfqpoint{7.750000in}{3.850000in}}%
\pgfusepath{clip}%
\pgfsetbuttcap%
\pgfsetmiterjoin%
\definecolor{currentfill}{rgb}{0.501961,0.501961,0.501961}%
\pgfsetfillcolor{currentfill}%
\pgfsetfillopacity{0.700000}%
\pgfsetlinewidth{0.000000pt}%
\definecolor{currentstroke}{rgb}{0.000000,0.000000,0.000000}%
\pgfsetstrokecolor{currentstroke}%
\pgfsetstrokeopacity{0.700000}%
\pgfsetdash{}{0pt}%
\pgfpathmoveto{\pgfqpoint{4.519531in}{0.550000in}}%
\pgfpathlineto{\pgfqpoint{4.547053in}{0.550000in}}%
\pgfpathlineto{\pgfqpoint{4.547053in}{0.550000in}}%
\pgfpathlineto{\pgfqpoint{4.519531in}{0.550000in}}%
\pgfpathlineto{\pgfqpoint{4.519531in}{0.550000in}}%
\pgfpathclose%
\pgfusepath{fill}%
\end{pgfscope}%
\begin{pgfscope}%
\pgfpathrectangle{\pgfqpoint{1.250000in}{0.550000in}}{\pgfqpoint{7.750000in}{3.850000in}}%
\pgfusepath{clip}%
\pgfsetbuttcap%
\pgfsetmiterjoin%
\definecolor{currentfill}{rgb}{0.501961,0.501961,0.501961}%
\pgfsetfillcolor{currentfill}%
\pgfsetfillopacity{0.700000}%
\pgfsetlinewidth{0.000000pt}%
\definecolor{currentstroke}{rgb}{0.000000,0.000000,0.000000}%
\pgfsetstrokecolor{currentstroke}%
\pgfsetstrokeopacity{0.700000}%
\pgfsetdash{}{0pt}%
\pgfpathmoveto{\pgfqpoint{4.547053in}{0.550000in}}%
\pgfpathlineto{\pgfqpoint{4.574574in}{0.550000in}}%
\pgfpathlineto{\pgfqpoint{4.574574in}{2.086807in}}%
\pgfpathlineto{\pgfqpoint{4.547053in}{2.086807in}}%
\pgfpathlineto{\pgfqpoint{4.547053in}{0.550000in}}%
\pgfpathclose%
\pgfusepath{fill}%
\end{pgfscope}%
\begin{pgfscope}%
\pgfpathrectangle{\pgfqpoint{1.250000in}{0.550000in}}{\pgfqpoint{7.750000in}{3.850000in}}%
\pgfusepath{clip}%
\pgfsetbuttcap%
\pgfsetmiterjoin%
\definecolor{currentfill}{rgb}{0.501961,0.501961,0.501961}%
\pgfsetfillcolor{currentfill}%
\pgfsetfillopacity{0.700000}%
\pgfsetlinewidth{0.000000pt}%
\definecolor{currentstroke}{rgb}{0.000000,0.000000,0.000000}%
\pgfsetstrokecolor{currentstroke}%
\pgfsetstrokeopacity{0.700000}%
\pgfsetdash{}{0pt}%
\pgfpathmoveto{\pgfqpoint{4.574574in}{0.550000in}}%
\pgfpathlineto{\pgfqpoint{4.602095in}{0.550000in}}%
\pgfpathlineto{\pgfqpoint{4.602095in}{0.550000in}}%
\pgfpathlineto{\pgfqpoint{4.574574in}{0.550000in}}%
\pgfpathlineto{\pgfqpoint{4.574574in}{0.550000in}}%
\pgfpathclose%
\pgfusepath{fill}%
\end{pgfscope}%
\begin{pgfscope}%
\pgfpathrectangle{\pgfqpoint{1.250000in}{0.550000in}}{\pgfqpoint{7.750000in}{3.850000in}}%
\pgfusepath{clip}%
\pgfsetbuttcap%
\pgfsetmiterjoin%
\definecolor{currentfill}{rgb}{0.501961,0.501961,0.501961}%
\pgfsetfillcolor{currentfill}%
\pgfsetfillopacity{0.700000}%
\pgfsetlinewidth{0.000000pt}%
\definecolor{currentstroke}{rgb}{0.000000,0.000000,0.000000}%
\pgfsetstrokecolor{currentstroke}%
\pgfsetstrokeopacity{0.700000}%
\pgfsetdash{}{0pt}%
\pgfpathmoveto{\pgfqpoint{4.602095in}{0.550000in}}%
\pgfpathlineto{\pgfqpoint{4.629616in}{0.550000in}}%
\pgfpathlineto{\pgfqpoint{4.629616in}{2.039762in}}%
\pgfpathlineto{\pgfqpoint{4.602095in}{2.039762in}}%
\pgfpathlineto{\pgfqpoint{4.602095in}{0.550000in}}%
\pgfpathclose%
\pgfusepath{fill}%
\end{pgfscope}%
\begin{pgfscope}%
\pgfpathrectangle{\pgfqpoint{1.250000in}{0.550000in}}{\pgfqpoint{7.750000in}{3.850000in}}%
\pgfusepath{clip}%
\pgfsetbuttcap%
\pgfsetmiterjoin%
\definecolor{currentfill}{rgb}{0.501961,0.501961,0.501961}%
\pgfsetfillcolor{currentfill}%
\pgfsetfillopacity{0.700000}%
\pgfsetlinewidth{0.000000pt}%
\definecolor{currentstroke}{rgb}{0.000000,0.000000,0.000000}%
\pgfsetstrokecolor{currentstroke}%
\pgfsetstrokeopacity{0.700000}%
\pgfsetdash{}{0pt}%
\pgfpathmoveto{\pgfqpoint{4.629616in}{0.550000in}}%
\pgfpathlineto{\pgfqpoint{4.657138in}{0.550000in}}%
\pgfpathlineto{\pgfqpoint{4.657138in}{0.550000in}}%
\pgfpathlineto{\pgfqpoint{4.629616in}{0.550000in}}%
\pgfpathlineto{\pgfqpoint{4.629616in}{0.550000in}}%
\pgfpathclose%
\pgfusepath{fill}%
\end{pgfscope}%
\begin{pgfscope}%
\pgfpathrectangle{\pgfqpoint{1.250000in}{0.550000in}}{\pgfqpoint{7.750000in}{3.850000in}}%
\pgfusepath{clip}%
\pgfsetbuttcap%
\pgfsetmiterjoin%
\definecolor{currentfill}{rgb}{0.501961,0.501961,0.501961}%
\pgfsetfillcolor{currentfill}%
\pgfsetfillopacity{0.700000}%
\pgfsetlinewidth{0.000000pt}%
\definecolor{currentstroke}{rgb}{0.000000,0.000000,0.000000}%
\pgfsetstrokecolor{currentstroke}%
\pgfsetstrokeopacity{0.700000}%
\pgfsetdash{}{0pt}%
\pgfpathmoveto{\pgfqpoint{4.657138in}{0.550000in}}%
\pgfpathlineto{\pgfqpoint{4.684659in}{0.550000in}}%
\pgfpathlineto{\pgfqpoint{4.684659in}{2.078253in}}%
\pgfpathlineto{\pgfqpoint{4.657138in}{2.078253in}}%
\pgfpathlineto{\pgfqpoint{4.657138in}{0.550000in}}%
\pgfpathclose%
\pgfusepath{fill}%
\end{pgfscope}%
\begin{pgfscope}%
\pgfpathrectangle{\pgfqpoint{1.250000in}{0.550000in}}{\pgfqpoint{7.750000in}{3.850000in}}%
\pgfusepath{clip}%
\pgfsetbuttcap%
\pgfsetmiterjoin%
\definecolor{currentfill}{rgb}{0.501961,0.501961,0.501961}%
\pgfsetfillcolor{currentfill}%
\pgfsetfillopacity{0.700000}%
\pgfsetlinewidth{0.000000pt}%
\definecolor{currentstroke}{rgb}{0.000000,0.000000,0.000000}%
\pgfsetstrokecolor{currentstroke}%
\pgfsetstrokeopacity{0.700000}%
\pgfsetdash{}{0pt}%
\pgfpathmoveto{\pgfqpoint{4.684659in}{0.550000in}}%
\pgfpathlineto{\pgfqpoint{4.712180in}{0.550000in}}%
\pgfpathlineto{\pgfqpoint{4.712180in}{0.550000in}}%
\pgfpathlineto{\pgfqpoint{4.684659in}{0.550000in}}%
\pgfpathlineto{\pgfqpoint{4.684659in}{0.550000in}}%
\pgfpathclose%
\pgfusepath{fill}%
\end{pgfscope}%
\begin{pgfscope}%
\pgfpathrectangle{\pgfqpoint{1.250000in}{0.550000in}}{\pgfqpoint{7.750000in}{3.850000in}}%
\pgfusepath{clip}%
\pgfsetbuttcap%
\pgfsetmiterjoin%
\definecolor{currentfill}{rgb}{0.501961,0.501961,0.501961}%
\pgfsetfillcolor{currentfill}%
\pgfsetfillopacity{0.700000}%
\pgfsetlinewidth{0.000000pt}%
\definecolor{currentstroke}{rgb}{0.000000,0.000000,0.000000}%
\pgfsetstrokecolor{currentstroke}%
\pgfsetstrokeopacity{0.700000}%
\pgfsetdash{}{0pt}%
\pgfpathmoveto{\pgfqpoint{4.712180in}{0.550000in}}%
\pgfpathlineto{\pgfqpoint{4.739702in}{0.550000in}}%
\pgfpathlineto{\pgfqpoint{4.739702in}{2.098212in}}%
\pgfpathlineto{\pgfqpoint{4.712180in}{2.098212in}}%
\pgfpathlineto{\pgfqpoint{4.712180in}{0.550000in}}%
\pgfpathclose%
\pgfusepath{fill}%
\end{pgfscope}%
\begin{pgfscope}%
\pgfpathrectangle{\pgfqpoint{1.250000in}{0.550000in}}{\pgfqpoint{7.750000in}{3.850000in}}%
\pgfusepath{clip}%
\pgfsetbuttcap%
\pgfsetmiterjoin%
\definecolor{currentfill}{rgb}{0.501961,0.501961,0.501961}%
\pgfsetfillcolor{currentfill}%
\pgfsetfillopacity{0.700000}%
\pgfsetlinewidth{0.000000pt}%
\definecolor{currentstroke}{rgb}{0.000000,0.000000,0.000000}%
\pgfsetstrokecolor{currentstroke}%
\pgfsetstrokeopacity{0.700000}%
\pgfsetdash{}{0pt}%
\pgfpathmoveto{\pgfqpoint{4.739702in}{0.550000in}}%
\pgfpathlineto{\pgfqpoint{4.767223in}{0.550000in}}%
\pgfpathlineto{\pgfqpoint{4.767223in}{0.550000in}}%
\pgfpathlineto{\pgfqpoint{4.739702in}{0.550000in}}%
\pgfpathlineto{\pgfqpoint{4.739702in}{0.550000in}}%
\pgfpathclose%
\pgfusepath{fill}%
\end{pgfscope}%
\begin{pgfscope}%
\pgfpathrectangle{\pgfqpoint{1.250000in}{0.550000in}}{\pgfqpoint{7.750000in}{3.850000in}}%
\pgfusepath{clip}%
\pgfsetbuttcap%
\pgfsetmiterjoin%
\definecolor{currentfill}{rgb}{0.501961,0.501961,0.501961}%
\pgfsetfillcolor{currentfill}%
\pgfsetfillopacity{0.700000}%
\pgfsetlinewidth{0.000000pt}%
\definecolor{currentstroke}{rgb}{0.000000,0.000000,0.000000}%
\pgfsetstrokecolor{currentstroke}%
\pgfsetstrokeopacity{0.700000}%
\pgfsetdash{}{0pt}%
\pgfpathmoveto{\pgfqpoint{4.767223in}{0.550000in}}%
\pgfpathlineto{\pgfqpoint{4.794744in}{0.550000in}}%
\pgfpathlineto{\pgfqpoint{4.794744in}{1.937118in}}%
\pgfpathlineto{\pgfqpoint{4.767223in}{1.937118in}}%
\pgfpathlineto{\pgfqpoint{4.767223in}{0.550000in}}%
\pgfpathclose%
\pgfusepath{fill}%
\end{pgfscope}%
\begin{pgfscope}%
\pgfpathrectangle{\pgfqpoint{1.250000in}{0.550000in}}{\pgfqpoint{7.750000in}{3.850000in}}%
\pgfusepath{clip}%
\pgfsetbuttcap%
\pgfsetmiterjoin%
\definecolor{currentfill}{rgb}{0.501961,0.501961,0.501961}%
\pgfsetfillcolor{currentfill}%
\pgfsetfillopacity{0.700000}%
\pgfsetlinewidth{0.000000pt}%
\definecolor{currentstroke}{rgb}{0.000000,0.000000,0.000000}%
\pgfsetstrokecolor{currentstroke}%
\pgfsetstrokeopacity{0.700000}%
\pgfsetdash{}{0pt}%
\pgfpathmoveto{\pgfqpoint{4.794744in}{0.550000in}}%
\pgfpathlineto{\pgfqpoint{4.822266in}{0.550000in}}%
\pgfpathlineto{\pgfqpoint{4.822266in}{0.550000in}}%
\pgfpathlineto{\pgfqpoint{4.794744in}{0.550000in}}%
\pgfpathlineto{\pgfqpoint{4.794744in}{0.550000in}}%
\pgfpathclose%
\pgfusepath{fill}%
\end{pgfscope}%
\begin{pgfscope}%
\pgfpathrectangle{\pgfqpoint{1.250000in}{0.550000in}}{\pgfqpoint{7.750000in}{3.850000in}}%
\pgfusepath{clip}%
\pgfsetbuttcap%
\pgfsetmiterjoin%
\definecolor{currentfill}{rgb}{0.501961,0.501961,0.501961}%
\pgfsetfillcolor{currentfill}%
\pgfsetfillopacity{0.700000}%
\pgfsetlinewidth{0.000000pt}%
\definecolor{currentstroke}{rgb}{0.000000,0.000000,0.000000}%
\pgfsetstrokecolor{currentstroke}%
\pgfsetstrokeopacity{0.700000}%
\pgfsetdash{}{0pt}%
\pgfpathmoveto{\pgfqpoint{4.822266in}{0.550000in}}%
\pgfpathlineto{\pgfqpoint{4.849787in}{0.550000in}}%
\pgfpathlineto{\pgfqpoint{4.849787in}{1.928564in}}%
\pgfpathlineto{\pgfqpoint{4.822266in}{1.928564in}}%
\pgfpathlineto{\pgfqpoint{4.822266in}{0.550000in}}%
\pgfpathclose%
\pgfusepath{fill}%
\end{pgfscope}%
\begin{pgfscope}%
\pgfpathrectangle{\pgfqpoint{1.250000in}{0.550000in}}{\pgfqpoint{7.750000in}{3.850000in}}%
\pgfusepath{clip}%
\pgfsetbuttcap%
\pgfsetmiterjoin%
\definecolor{currentfill}{rgb}{0.501961,0.501961,0.501961}%
\pgfsetfillcolor{currentfill}%
\pgfsetfillopacity{0.700000}%
\pgfsetlinewidth{0.000000pt}%
\definecolor{currentstroke}{rgb}{0.000000,0.000000,0.000000}%
\pgfsetstrokecolor{currentstroke}%
\pgfsetstrokeopacity{0.700000}%
\pgfsetdash{}{0pt}%
\pgfpathmoveto{\pgfqpoint{4.849787in}{0.550000in}}%
\pgfpathlineto{\pgfqpoint{4.877308in}{0.550000in}}%
\pgfpathlineto{\pgfqpoint{4.877308in}{0.550000in}}%
\pgfpathlineto{\pgfqpoint{4.849787in}{0.550000in}}%
\pgfpathlineto{\pgfqpoint{4.849787in}{0.550000in}}%
\pgfpathclose%
\pgfusepath{fill}%
\end{pgfscope}%
\begin{pgfscope}%
\pgfpathrectangle{\pgfqpoint{1.250000in}{0.550000in}}{\pgfqpoint{7.750000in}{3.850000in}}%
\pgfusepath{clip}%
\pgfsetbuttcap%
\pgfsetmiterjoin%
\definecolor{currentfill}{rgb}{0.501961,0.501961,0.501961}%
\pgfsetfillcolor{currentfill}%
\pgfsetfillopacity{0.700000}%
\pgfsetlinewidth{0.000000pt}%
\definecolor{currentstroke}{rgb}{0.000000,0.000000,0.000000}%
\pgfsetstrokecolor{currentstroke}%
\pgfsetstrokeopacity{0.700000}%
\pgfsetdash{}{0pt}%
\pgfpathmoveto{\pgfqpoint{4.877308in}{0.550000in}}%
\pgfpathlineto{\pgfqpoint{4.904830in}{0.550000in}}%
\pgfpathlineto{\pgfqpoint{4.904830in}{1.942820in}}%
\pgfpathlineto{\pgfqpoint{4.877308in}{1.942820in}}%
\pgfpathlineto{\pgfqpoint{4.877308in}{0.550000in}}%
\pgfpathclose%
\pgfusepath{fill}%
\end{pgfscope}%
\begin{pgfscope}%
\pgfpathrectangle{\pgfqpoint{1.250000in}{0.550000in}}{\pgfqpoint{7.750000in}{3.850000in}}%
\pgfusepath{clip}%
\pgfsetbuttcap%
\pgfsetmiterjoin%
\definecolor{currentfill}{rgb}{0.501961,0.501961,0.501961}%
\pgfsetfillcolor{currentfill}%
\pgfsetfillopacity{0.700000}%
\pgfsetlinewidth{0.000000pt}%
\definecolor{currentstroke}{rgb}{0.000000,0.000000,0.000000}%
\pgfsetstrokecolor{currentstroke}%
\pgfsetstrokeopacity{0.700000}%
\pgfsetdash{}{0pt}%
\pgfpathmoveto{\pgfqpoint{4.904830in}{0.550000in}}%
\pgfpathlineto{\pgfqpoint{4.932351in}{0.550000in}}%
\pgfpathlineto{\pgfqpoint{4.932351in}{0.550000in}}%
\pgfpathlineto{\pgfqpoint{4.904830in}{0.550000in}}%
\pgfpathlineto{\pgfqpoint{4.904830in}{0.550000in}}%
\pgfpathclose%
\pgfusepath{fill}%
\end{pgfscope}%
\begin{pgfscope}%
\pgfpathrectangle{\pgfqpoint{1.250000in}{0.550000in}}{\pgfqpoint{7.750000in}{3.850000in}}%
\pgfusepath{clip}%
\pgfsetbuttcap%
\pgfsetmiterjoin%
\definecolor{currentfill}{rgb}{0.501961,0.501961,0.501961}%
\pgfsetfillcolor{currentfill}%
\pgfsetfillopacity{0.700000}%
\pgfsetlinewidth{0.000000pt}%
\definecolor{currentstroke}{rgb}{0.000000,0.000000,0.000000}%
\pgfsetstrokecolor{currentstroke}%
\pgfsetstrokeopacity{0.700000}%
\pgfsetdash{}{0pt}%
\pgfpathmoveto{\pgfqpoint{4.932351in}{0.550000in}}%
\pgfpathlineto{\pgfqpoint{4.959872in}{0.550000in}}%
\pgfpathlineto{\pgfqpoint{4.959872in}{1.854432in}}%
\pgfpathlineto{\pgfqpoint{4.932351in}{1.854432in}}%
\pgfpathlineto{\pgfqpoint{4.932351in}{0.550000in}}%
\pgfpathclose%
\pgfusepath{fill}%
\end{pgfscope}%
\begin{pgfscope}%
\pgfpathrectangle{\pgfqpoint{1.250000in}{0.550000in}}{\pgfqpoint{7.750000in}{3.850000in}}%
\pgfusepath{clip}%
\pgfsetbuttcap%
\pgfsetmiterjoin%
\definecolor{currentfill}{rgb}{0.501961,0.501961,0.501961}%
\pgfsetfillcolor{currentfill}%
\pgfsetfillopacity{0.700000}%
\pgfsetlinewidth{0.000000pt}%
\definecolor{currentstroke}{rgb}{0.000000,0.000000,0.000000}%
\pgfsetstrokecolor{currentstroke}%
\pgfsetstrokeopacity{0.700000}%
\pgfsetdash{}{0pt}%
\pgfpathmoveto{\pgfqpoint{4.959872in}{0.550000in}}%
\pgfpathlineto{\pgfqpoint{4.987393in}{0.550000in}}%
\pgfpathlineto{\pgfqpoint{4.987393in}{0.550000in}}%
\pgfpathlineto{\pgfqpoint{4.959872in}{0.550000in}}%
\pgfpathlineto{\pgfqpoint{4.959872in}{0.550000in}}%
\pgfpathclose%
\pgfusepath{fill}%
\end{pgfscope}%
\begin{pgfscope}%
\pgfpathrectangle{\pgfqpoint{1.250000in}{0.550000in}}{\pgfqpoint{7.750000in}{3.850000in}}%
\pgfusepath{clip}%
\pgfsetbuttcap%
\pgfsetmiterjoin%
\definecolor{currentfill}{rgb}{0.501961,0.501961,0.501961}%
\pgfsetfillcolor{currentfill}%
\pgfsetfillopacity{0.700000}%
\pgfsetlinewidth{0.000000pt}%
\definecolor{currentstroke}{rgb}{0.000000,0.000000,0.000000}%
\pgfsetstrokecolor{currentstroke}%
\pgfsetstrokeopacity{0.700000}%
\pgfsetdash{}{0pt}%
\pgfpathmoveto{\pgfqpoint{4.987393in}{0.550000in}}%
\pgfpathlineto{\pgfqpoint{5.014915in}{0.550000in}}%
\pgfpathlineto{\pgfqpoint{5.014915in}{1.827346in}}%
\pgfpathlineto{\pgfqpoint{4.987393in}{1.827346in}}%
\pgfpathlineto{\pgfqpoint{4.987393in}{0.550000in}}%
\pgfpathclose%
\pgfusepath{fill}%
\end{pgfscope}%
\begin{pgfscope}%
\pgfpathrectangle{\pgfqpoint{1.250000in}{0.550000in}}{\pgfqpoint{7.750000in}{3.850000in}}%
\pgfusepath{clip}%
\pgfsetbuttcap%
\pgfsetmiterjoin%
\definecolor{currentfill}{rgb}{0.501961,0.501961,0.501961}%
\pgfsetfillcolor{currentfill}%
\pgfsetfillopacity{0.700000}%
\pgfsetlinewidth{0.000000pt}%
\definecolor{currentstroke}{rgb}{0.000000,0.000000,0.000000}%
\pgfsetstrokecolor{currentstroke}%
\pgfsetstrokeopacity{0.700000}%
\pgfsetdash{}{0pt}%
\pgfpathmoveto{\pgfqpoint{5.014915in}{0.550000in}}%
\pgfpathlineto{\pgfqpoint{5.042436in}{0.550000in}}%
\pgfpathlineto{\pgfqpoint{5.042436in}{0.550000in}}%
\pgfpathlineto{\pgfqpoint{5.014915in}{0.550000in}}%
\pgfpathlineto{\pgfqpoint{5.014915in}{0.550000in}}%
\pgfpathclose%
\pgfusepath{fill}%
\end{pgfscope}%
\begin{pgfscope}%
\pgfpathrectangle{\pgfqpoint{1.250000in}{0.550000in}}{\pgfqpoint{7.750000in}{3.850000in}}%
\pgfusepath{clip}%
\pgfsetbuttcap%
\pgfsetmiterjoin%
\definecolor{currentfill}{rgb}{0.501961,0.501961,0.501961}%
\pgfsetfillcolor{currentfill}%
\pgfsetfillopacity{0.700000}%
\pgfsetlinewidth{0.000000pt}%
\definecolor{currentstroke}{rgb}{0.000000,0.000000,0.000000}%
\pgfsetstrokecolor{currentstroke}%
\pgfsetstrokeopacity{0.700000}%
\pgfsetdash{}{0pt}%
\pgfpathmoveto{\pgfqpoint{5.042436in}{0.550000in}}%
\pgfpathlineto{\pgfqpoint{5.069957in}{0.550000in}}%
\pgfpathlineto{\pgfqpoint{5.069957in}{1.766045in}}%
\pgfpathlineto{\pgfqpoint{5.042436in}{1.766045in}}%
\pgfpathlineto{\pgfqpoint{5.042436in}{0.550000in}}%
\pgfpathclose%
\pgfusepath{fill}%
\end{pgfscope}%
\begin{pgfscope}%
\pgfpathrectangle{\pgfqpoint{1.250000in}{0.550000in}}{\pgfqpoint{7.750000in}{3.850000in}}%
\pgfusepath{clip}%
\pgfsetbuttcap%
\pgfsetmiterjoin%
\definecolor{currentfill}{rgb}{0.501961,0.501961,0.501961}%
\pgfsetfillcolor{currentfill}%
\pgfsetfillopacity{0.700000}%
\pgfsetlinewidth{0.000000pt}%
\definecolor{currentstroke}{rgb}{0.000000,0.000000,0.000000}%
\pgfsetstrokecolor{currentstroke}%
\pgfsetstrokeopacity{0.700000}%
\pgfsetdash{}{0pt}%
\pgfpathmoveto{\pgfqpoint{5.069957in}{0.550000in}}%
\pgfpathlineto{\pgfqpoint{5.097479in}{0.550000in}}%
\pgfpathlineto{\pgfqpoint{5.097479in}{0.550000in}}%
\pgfpathlineto{\pgfqpoint{5.069957in}{0.550000in}}%
\pgfpathlineto{\pgfqpoint{5.069957in}{0.550000in}}%
\pgfpathclose%
\pgfusepath{fill}%
\end{pgfscope}%
\begin{pgfscope}%
\pgfpathrectangle{\pgfqpoint{1.250000in}{0.550000in}}{\pgfqpoint{7.750000in}{3.850000in}}%
\pgfusepath{clip}%
\pgfsetbuttcap%
\pgfsetmiterjoin%
\definecolor{currentfill}{rgb}{0.501961,0.501961,0.501961}%
\pgfsetfillcolor{currentfill}%
\pgfsetfillopacity{0.700000}%
\pgfsetlinewidth{0.000000pt}%
\definecolor{currentstroke}{rgb}{0.000000,0.000000,0.000000}%
\pgfsetstrokecolor{currentstroke}%
\pgfsetstrokeopacity{0.700000}%
\pgfsetdash{}{0pt}%
\pgfpathmoveto{\pgfqpoint{5.097479in}{0.550000in}}%
\pgfpathlineto{\pgfqpoint{5.125000in}{0.550000in}}%
\pgfpathlineto{\pgfqpoint{5.125000in}{1.784578in}}%
\pgfpathlineto{\pgfqpoint{5.097479in}{1.784578in}}%
\pgfpathlineto{\pgfqpoint{5.097479in}{0.550000in}}%
\pgfpathclose%
\pgfusepath{fill}%
\end{pgfscope}%
\begin{pgfscope}%
\pgfpathrectangle{\pgfqpoint{1.250000in}{0.550000in}}{\pgfqpoint{7.750000in}{3.850000in}}%
\pgfusepath{clip}%
\pgfsetbuttcap%
\pgfsetmiterjoin%
\definecolor{currentfill}{rgb}{0.501961,0.501961,0.501961}%
\pgfsetfillcolor{currentfill}%
\pgfsetfillopacity{0.700000}%
\pgfsetlinewidth{0.000000pt}%
\definecolor{currentstroke}{rgb}{0.000000,0.000000,0.000000}%
\pgfsetstrokecolor{currentstroke}%
\pgfsetstrokeopacity{0.700000}%
\pgfsetdash{}{0pt}%
\pgfpathmoveto{\pgfqpoint{5.125000in}{0.550000in}}%
\pgfpathlineto{\pgfqpoint{5.152521in}{0.550000in}}%
\pgfpathlineto{\pgfqpoint{5.152521in}{0.550000in}}%
\pgfpathlineto{\pgfqpoint{5.125000in}{0.550000in}}%
\pgfpathlineto{\pgfqpoint{5.125000in}{0.550000in}}%
\pgfpathclose%
\pgfusepath{fill}%
\end{pgfscope}%
\begin{pgfscope}%
\pgfpathrectangle{\pgfqpoint{1.250000in}{0.550000in}}{\pgfqpoint{7.750000in}{3.850000in}}%
\pgfusepath{clip}%
\pgfsetbuttcap%
\pgfsetmiterjoin%
\definecolor{currentfill}{rgb}{0.501961,0.501961,0.501961}%
\pgfsetfillcolor{currentfill}%
\pgfsetfillopacity{0.700000}%
\pgfsetlinewidth{0.000000pt}%
\definecolor{currentstroke}{rgb}{0.000000,0.000000,0.000000}%
\pgfsetstrokecolor{currentstroke}%
\pgfsetstrokeopacity{0.700000}%
\pgfsetdash{}{0pt}%
\pgfpathmoveto{\pgfqpoint{5.152521in}{0.550000in}}%
\pgfpathlineto{\pgfqpoint{5.180043in}{0.550000in}}%
\pgfpathlineto{\pgfqpoint{5.180043in}{1.734681in}}%
\pgfpathlineto{\pgfqpoint{5.152521in}{1.734681in}}%
\pgfpathlineto{\pgfqpoint{5.152521in}{0.550000in}}%
\pgfpathclose%
\pgfusepath{fill}%
\end{pgfscope}%
\begin{pgfscope}%
\pgfpathrectangle{\pgfqpoint{1.250000in}{0.550000in}}{\pgfqpoint{7.750000in}{3.850000in}}%
\pgfusepath{clip}%
\pgfsetbuttcap%
\pgfsetmiterjoin%
\definecolor{currentfill}{rgb}{0.501961,0.501961,0.501961}%
\pgfsetfillcolor{currentfill}%
\pgfsetfillopacity{0.700000}%
\pgfsetlinewidth{0.000000pt}%
\definecolor{currentstroke}{rgb}{0.000000,0.000000,0.000000}%
\pgfsetstrokecolor{currentstroke}%
\pgfsetstrokeopacity{0.700000}%
\pgfsetdash{}{0pt}%
\pgfpathmoveto{\pgfqpoint{5.180043in}{0.550000in}}%
\pgfpathlineto{\pgfqpoint{5.207564in}{0.550000in}}%
\pgfpathlineto{\pgfqpoint{5.207564in}{0.550000in}}%
\pgfpathlineto{\pgfqpoint{5.180043in}{0.550000in}}%
\pgfpathlineto{\pgfqpoint{5.180043in}{0.550000in}}%
\pgfpathclose%
\pgfusepath{fill}%
\end{pgfscope}%
\begin{pgfscope}%
\pgfpathrectangle{\pgfqpoint{1.250000in}{0.550000in}}{\pgfqpoint{7.750000in}{3.850000in}}%
\pgfusepath{clip}%
\pgfsetbuttcap%
\pgfsetmiterjoin%
\definecolor{currentfill}{rgb}{0.501961,0.501961,0.501961}%
\pgfsetfillcolor{currentfill}%
\pgfsetfillopacity{0.700000}%
\pgfsetlinewidth{0.000000pt}%
\definecolor{currentstroke}{rgb}{0.000000,0.000000,0.000000}%
\pgfsetstrokecolor{currentstroke}%
\pgfsetstrokeopacity{0.700000}%
\pgfsetdash{}{0pt}%
\pgfpathmoveto{\pgfqpoint{5.207564in}{0.550000in}}%
\pgfpathlineto{\pgfqpoint{5.235085in}{0.550000in}}%
\pgfpathlineto{\pgfqpoint{5.235085in}{1.690487in}}%
\pgfpathlineto{\pgfqpoint{5.207564in}{1.690487in}}%
\pgfpathlineto{\pgfqpoint{5.207564in}{0.550000in}}%
\pgfpathclose%
\pgfusepath{fill}%
\end{pgfscope}%
\begin{pgfscope}%
\pgfpathrectangle{\pgfqpoint{1.250000in}{0.550000in}}{\pgfqpoint{7.750000in}{3.850000in}}%
\pgfusepath{clip}%
\pgfsetbuttcap%
\pgfsetmiterjoin%
\definecolor{currentfill}{rgb}{0.501961,0.501961,0.501961}%
\pgfsetfillcolor{currentfill}%
\pgfsetfillopacity{0.700000}%
\pgfsetlinewidth{0.000000pt}%
\definecolor{currentstroke}{rgb}{0.000000,0.000000,0.000000}%
\pgfsetstrokecolor{currentstroke}%
\pgfsetstrokeopacity{0.700000}%
\pgfsetdash{}{0pt}%
\pgfpathmoveto{\pgfqpoint{5.235085in}{0.550000in}}%
\pgfpathlineto{\pgfqpoint{5.262607in}{0.550000in}}%
\pgfpathlineto{\pgfqpoint{5.262607in}{0.550000in}}%
\pgfpathlineto{\pgfqpoint{5.235085in}{0.550000in}}%
\pgfpathlineto{\pgfqpoint{5.235085in}{0.550000in}}%
\pgfpathclose%
\pgfusepath{fill}%
\end{pgfscope}%
\begin{pgfscope}%
\pgfpathrectangle{\pgfqpoint{1.250000in}{0.550000in}}{\pgfqpoint{7.750000in}{3.850000in}}%
\pgfusepath{clip}%
\pgfsetbuttcap%
\pgfsetmiterjoin%
\definecolor{currentfill}{rgb}{0.501961,0.501961,0.501961}%
\pgfsetfillcolor{currentfill}%
\pgfsetfillopacity{0.700000}%
\pgfsetlinewidth{0.000000pt}%
\definecolor{currentstroke}{rgb}{0.000000,0.000000,0.000000}%
\pgfsetstrokecolor{currentstroke}%
\pgfsetstrokeopacity{0.700000}%
\pgfsetdash{}{0pt}%
\pgfpathmoveto{\pgfqpoint{5.262607in}{0.550000in}}%
\pgfpathlineto{\pgfqpoint{5.290128in}{0.550000in}}%
\pgfpathlineto{\pgfqpoint{5.290128in}{1.644868in}}%
\pgfpathlineto{\pgfqpoint{5.262607in}{1.644868in}}%
\pgfpathlineto{\pgfqpoint{5.262607in}{0.550000in}}%
\pgfpathclose%
\pgfusepath{fill}%
\end{pgfscope}%
\begin{pgfscope}%
\pgfpathrectangle{\pgfqpoint{1.250000in}{0.550000in}}{\pgfqpoint{7.750000in}{3.850000in}}%
\pgfusepath{clip}%
\pgfsetbuttcap%
\pgfsetmiterjoin%
\definecolor{currentfill}{rgb}{0.501961,0.501961,0.501961}%
\pgfsetfillcolor{currentfill}%
\pgfsetfillopacity{0.700000}%
\pgfsetlinewidth{0.000000pt}%
\definecolor{currentstroke}{rgb}{0.000000,0.000000,0.000000}%
\pgfsetstrokecolor{currentstroke}%
\pgfsetstrokeopacity{0.700000}%
\pgfsetdash{}{0pt}%
\pgfpathmoveto{\pgfqpoint{5.290128in}{0.550000in}}%
\pgfpathlineto{\pgfqpoint{5.317649in}{0.550000in}}%
\pgfpathlineto{\pgfqpoint{5.317649in}{0.550000in}}%
\pgfpathlineto{\pgfqpoint{5.290128in}{0.550000in}}%
\pgfpathlineto{\pgfqpoint{5.290128in}{0.550000in}}%
\pgfpathclose%
\pgfusepath{fill}%
\end{pgfscope}%
\begin{pgfscope}%
\pgfpathrectangle{\pgfqpoint{1.250000in}{0.550000in}}{\pgfqpoint{7.750000in}{3.850000in}}%
\pgfusepath{clip}%
\pgfsetbuttcap%
\pgfsetmiterjoin%
\definecolor{currentfill}{rgb}{0.501961,0.501961,0.501961}%
\pgfsetfillcolor{currentfill}%
\pgfsetfillopacity{0.700000}%
\pgfsetlinewidth{0.000000pt}%
\definecolor{currentstroke}{rgb}{0.000000,0.000000,0.000000}%
\pgfsetstrokecolor{currentstroke}%
\pgfsetstrokeopacity{0.700000}%
\pgfsetdash{}{0pt}%
\pgfpathmoveto{\pgfqpoint{5.317649in}{0.550000in}}%
\pgfpathlineto{\pgfqpoint{5.345170in}{0.550000in}}%
\pgfpathlineto{\pgfqpoint{5.345170in}{1.602100in}}%
\pgfpathlineto{\pgfqpoint{5.317649in}{1.602100in}}%
\pgfpathlineto{\pgfqpoint{5.317649in}{0.550000in}}%
\pgfpathclose%
\pgfusepath{fill}%
\end{pgfscope}%
\begin{pgfscope}%
\pgfpathrectangle{\pgfqpoint{1.250000in}{0.550000in}}{\pgfqpoint{7.750000in}{3.850000in}}%
\pgfusepath{clip}%
\pgfsetbuttcap%
\pgfsetmiterjoin%
\definecolor{currentfill}{rgb}{0.501961,0.501961,0.501961}%
\pgfsetfillcolor{currentfill}%
\pgfsetfillopacity{0.700000}%
\pgfsetlinewidth{0.000000pt}%
\definecolor{currentstroke}{rgb}{0.000000,0.000000,0.000000}%
\pgfsetstrokecolor{currentstroke}%
\pgfsetstrokeopacity{0.700000}%
\pgfsetdash{}{0pt}%
\pgfpathmoveto{\pgfqpoint{5.345170in}{0.550000in}}%
\pgfpathlineto{\pgfqpoint{5.372692in}{0.550000in}}%
\pgfpathlineto{\pgfqpoint{5.372692in}{0.550000in}}%
\pgfpathlineto{\pgfqpoint{5.345170in}{0.550000in}}%
\pgfpathlineto{\pgfqpoint{5.345170in}{0.550000in}}%
\pgfpathclose%
\pgfusepath{fill}%
\end{pgfscope}%
\begin{pgfscope}%
\pgfpathrectangle{\pgfqpoint{1.250000in}{0.550000in}}{\pgfqpoint{7.750000in}{3.850000in}}%
\pgfusepath{clip}%
\pgfsetbuttcap%
\pgfsetmiterjoin%
\definecolor{currentfill}{rgb}{0.501961,0.501961,0.501961}%
\pgfsetfillcolor{currentfill}%
\pgfsetfillopacity{0.700000}%
\pgfsetlinewidth{0.000000pt}%
\definecolor{currentstroke}{rgb}{0.000000,0.000000,0.000000}%
\pgfsetstrokecolor{currentstroke}%
\pgfsetstrokeopacity{0.700000}%
\pgfsetdash{}{0pt}%
\pgfpathmoveto{\pgfqpoint{5.372692in}{0.550000in}}%
\pgfpathlineto{\pgfqpoint{5.400213in}{0.550000in}}%
\pgfpathlineto{\pgfqpoint{5.400213in}{1.592120in}}%
\pgfpathlineto{\pgfqpoint{5.372692in}{1.592120in}}%
\pgfpathlineto{\pgfqpoint{5.372692in}{0.550000in}}%
\pgfpathclose%
\pgfusepath{fill}%
\end{pgfscope}%
\begin{pgfscope}%
\pgfpathrectangle{\pgfqpoint{1.250000in}{0.550000in}}{\pgfqpoint{7.750000in}{3.850000in}}%
\pgfusepath{clip}%
\pgfsetbuttcap%
\pgfsetmiterjoin%
\definecolor{currentfill}{rgb}{0.501961,0.501961,0.501961}%
\pgfsetfillcolor{currentfill}%
\pgfsetfillopacity{0.700000}%
\pgfsetlinewidth{0.000000pt}%
\definecolor{currentstroke}{rgb}{0.000000,0.000000,0.000000}%
\pgfsetstrokecolor{currentstroke}%
\pgfsetstrokeopacity{0.700000}%
\pgfsetdash{}{0pt}%
\pgfpathmoveto{\pgfqpoint{5.400213in}{0.550000in}}%
\pgfpathlineto{\pgfqpoint{5.427734in}{0.550000in}}%
\pgfpathlineto{\pgfqpoint{5.427734in}{0.550000in}}%
\pgfpathlineto{\pgfqpoint{5.400213in}{0.550000in}}%
\pgfpathlineto{\pgfqpoint{5.400213in}{0.550000in}}%
\pgfpathclose%
\pgfusepath{fill}%
\end{pgfscope}%
\begin{pgfscope}%
\pgfpathrectangle{\pgfqpoint{1.250000in}{0.550000in}}{\pgfqpoint{7.750000in}{3.850000in}}%
\pgfusepath{clip}%
\pgfsetbuttcap%
\pgfsetmiterjoin%
\definecolor{currentfill}{rgb}{0.501961,0.501961,0.501961}%
\pgfsetfillcolor{currentfill}%
\pgfsetfillopacity{0.700000}%
\pgfsetlinewidth{0.000000pt}%
\definecolor{currentstroke}{rgb}{0.000000,0.000000,0.000000}%
\pgfsetstrokecolor{currentstroke}%
\pgfsetstrokeopacity{0.700000}%
\pgfsetdash{}{0pt}%
\pgfpathmoveto{\pgfqpoint{5.427734in}{0.550000in}}%
\pgfpathlineto{\pgfqpoint{5.455256in}{0.550000in}}%
\pgfpathlineto{\pgfqpoint{5.455256in}{1.510861in}}%
\pgfpathlineto{\pgfqpoint{5.427734in}{1.510861in}}%
\pgfpathlineto{\pgfqpoint{5.427734in}{0.550000in}}%
\pgfpathclose%
\pgfusepath{fill}%
\end{pgfscope}%
\begin{pgfscope}%
\pgfpathrectangle{\pgfqpoint{1.250000in}{0.550000in}}{\pgfqpoint{7.750000in}{3.850000in}}%
\pgfusepath{clip}%
\pgfsetbuttcap%
\pgfsetmiterjoin%
\definecolor{currentfill}{rgb}{0.501961,0.501961,0.501961}%
\pgfsetfillcolor{currentfill}%
\pgfsetfillopacity{0.700000}%
\pgfsetlinewidth{0.000000pt}%
\definecolor{currentstroke}{rgb}{0.000000,0.000000,0.000000}%
\pgfsetstrokecolor{currentstroke}%
\pgfsetstrokeopacity{0.700000}%
\pgfsetdash{}{0pt}%
\pgfpathmoveto{\pgfqpoint{5.455256in}{0.550000in}}%
\pgfpathlineto{\pgfqpoint{5.482777in}{0.550000in}}%
\pgfpathlineto{\pgfqpoint{5.482777in}{0.550000in}}%
\pgfpathlineto{\pgfqpoint{5.455256in}{0.550000in}}%
\pgfpathlineto{\pgfqpoint{5.455256in}{0.550000in}}%
\pgfpathclose%
\pgfusepath{fill}%
\end{pgfscope}%
\begin{pgfscope}%
\pgfpathrectangle{\pgfqpoint{1.250000in}{0.550000in}}{\pgfqpoint{7.750000in}{3.850000in}}%
\pgfusepath{clip}%
\pgfsetbuttcap%
\pgfsetmiterjoin%
\definecolor{currentfill}{rgb}{0.501961,0.501961,0.501961}%
\pgfsetfillcolor{currentfill}%
\pgfsetfillopacity{0.700000}%
\pgfsetlinewidth{0.000000pt}%
\definecolor{currentstroke}{rgb}{0.000000,0.000000,0.000000}%
\pgfsetstrokecolor{currentstroke}%
\pgfsetstrokeopacity{0.700000}%
\pgfsetdash{}{0pt}%
\pgfpathmoveto{\pgfqpoint{5.482777in}{0.550000in}}%
\pgfpathlineto{\pgfqpoint{5.510298in}{0.550000in}}%
\pgfpathlineto{\pgfqpoint{5.510298in}{1.508009in}}%
\pgfpathlineto{\pgfqpoint{5.482777in}{1.508009in}}%
\pgfpathlineto{\pgfqpoint{5.482777in}{0.550000in}}%
\pgfpathclose%
\pgfusepath{fill}%
\end{pgfscope}%
\begin{pgfscope}%
\pgfpathrectangle{\pgfqpoint{1.250000in}{0.550000in}}{\pgfqpoint{7.750000in}{3.850000in}}%
\pgfusepath{clip}%
\pgfsetbuttcap%
\pgfsetmiterjoin%
\definecolor{currentfill}{rgb}{0.501961,0.501961,0.501961}%
\pgfsetfillcolor{currentfill}%
\pgfsetfillopacity{0.700000}%
\pgfsetlinewidth{0.000000pt}%
\definecolor{currentstroke}{rgb}{0.000000,0.000000,0.000000}%
\pgfsetstrokecolor{currentstroke}%
\pgfsetstrokeopacity{0.700000}%
\pgfsetdash{}{0pt}%
\pgfpathmoveto{\pgfqpoint{5.510298in}{0.550000in}}%
\pgfpathlineto{\pgfqpoint{5.537820in}{0.550000in}}%
\pgfpathlineto{\pgfqpoint{5.537820in}{0.550000in}}%
\pgfpathlineto{\pgfqpoint{5.510298in}{0.550000in}}%
\pgfpathlineto{\pgfqpoint{5.510298in}{0.550000in}}%
\pgfpathclose%
\pgfusepath{fill}%
\end{pgfscope}%
\begin{pgfscope}%
\pgfpathrectangle{\pgfqpoint{1.250000in}{0.550000in}}{\pgfqpoint{7.750000in}{3.850000in}}%
\pgfusepath{clip}%
\pgfsetbuttcap%
\pgfsetmiterjoin%
\definecolor{currentfill}{rgb}{0.501961,0.501961,0.501961}%
\pgfsetfillcolor{currentfill}%
\pgfsetfillopacity{0.700000}%
\pgfsetlinewidth{0.000000pt}%
\definecolor{currentstroke}{rgb}{0.000000,0.000000,0.000000}%
\pgfsetstrokecolor{currentstroke}%
\pgfsetstrokeopacity{0.700000}%
\pgfsetdash{}{0pt}%
\pgfpathmoveto{\pgfqpoint{5.537820in}{0.550000in}}%
\pgfpathlineto{\pgfqpoint{5.565341in}{0.550000in}}%
\pgfpathlineto{\pgfqpoint{5.565341in}{1.382556in}}%
\pgfpathlineto{\pgfqpoint{5.537820in}{1.382556in}}%
\pgfpathlineto{\pgfqpoint{5.537820in}{0.550000in}}%
\pgfpathclose%
\pgfusepath{fill}%
\end{pgfscope}%
\begin{pgfscope}%
\pgfpathrectangle{\pgfqpoint{1.250000in}{0.550000in}}{\pgfqpoint{7.750000in}{3.850000in}}%
\pgfusepath{clip}%
\pgfsetbuttcap%
\pgfsetmiterjoin%
\definecolor{currentfill}{rgb}{0.501961,0.501961,0.501961}%
\pgfsetfillcolor{currentfill}%
\pgfsetfillopacity{0.700000}%
\pgfsetlinewidth{0.000000pt}%
\definecolor{currentstroke}{rgb}{0.000000,0.000000,0.000000}%
\pgfsetstrokecolor{currentstroke}%
\pgfsetstrokeopacity{0.700000}%
\pgfsetdash{}{0pt}%
\pgfpathmoveto{\pgfqpoint{5.565341in}{0.550000in}}%
\pgfpathlineto{\pgfqpoint{5.592862in}{0.550000in}}%
\pgfpathlineto{\pgfqpoint{5.592862in}{0.550000in}}%
\pgfpathlineto{\pgfqpoint{5.565341in}{0.550000in}}%
\pgfpathlineto{\pgfqpoint{5.565341in}{0.550000in}}%
\pgfpathclose%
\pgfusepath{fill}%
\end{pgfscope}%
\begin{pgfscope}%
\pgfpathrectangle{\pgfqpoint{1.250000in}{0.550000in}}{\pgfqpoint{7.750000in}{3.850000in}}%
\pgfusepath{clip}%
\pgfsetbuttcap%
\pgfsetmiterjoin%
\definecolor{currentfill}{rgb}{0.501961,0.501961,0.501961}%
\pgfsetfillcolor{currentfill}%
\pgfsetfillopacity{0.700000}%
\pgfsetlinewidth{0.000000pt}%
\definecolor{currentstroke}{rgb}{0.000000,0.000000,0.000000}%
\pgfsetstrokecolor{currentstroke}%
\pgfsetstrokeopacity{0.700000}%
\pgfsetdash{}{0pt}%
\pgfpathmoveto{\pgfqpoint{5.592862in}{0.550000in}}%
\pgfpathlineto{\pgfqpoint{5.620384in}{0.550000in}}%
\pgfpathlineto{\pgfqpoint{5.620384in}{1.449559in}}%
\pgfpathlineto{\pgfqpoint{5.592862in}{1.449559in}}%
\pgfpathlineto{\pgfqpoint{5.592862in}{0.550000in}}%
\pgfpathclose%
\pgfusepath{fill}%
\end{pgfscope}%
\begin{pgfscope}%
\pgfpathrectangle{\pgfqpoint{1.250000in}{0.550000in}}{\pgfqpoint{7.750000in}{3.850000in}}%
\pgfusepath{clip}%
\pgfsetbuttcap%
\pgfsetmiterjoin%
\definecolor{currentfill}{rgb}{0.501961,0.501961,0.501961}%
\pgfsetfillcolor{currentfill}%
\pgfsetfillopacity{0.700000}%
\pgfsetlinewidth{0.000000pt}%
\definecolor{currentstroke}{rgb}{0.000000,0.000000,0.000000}%
\pgfsetstrokecolor{currentstroke}%
\pgfsetstrokeopacity{0.700000}%
\pgfsetdash{}{0pt}%
\pgfpathmoveto{\pgfqpoint{5.620384in}{0.550000in}}%
\pgfpathlineto{\pgfqpoint{5.647905in}{0.550000in}}%
\pgfpathlineto{\pgfqpoint{5.647905in}{0.550000in}}%
\pgfpathlineto{\pgfqpoint{5.620384in}{0.550000in}}%
\pgfpathlineto{\pgfqpoint{5.620384in}{0.550000in}}%
\pgfpathclose%
\pgfusepath{fill}%
\end{pgfscope}%
\begin{pgfscope}%
\pgfpathrectangle{\pgfqpoint{1.250000in}{0.550000in}}{\pgfqpoint{7.750000in}{3.850000in}}%
\pgfusepath{clip}%
\pgfsetbuttcap%
\pgfsetmiterjoin%
\definecolor{currentfill}{rgb}{0.501961,0.501961,0.501961}%
\pgfsetfillcolor{currentfill}%
\pgfsetfillopacity{0.700000}%
\pgfsetlinewidth{0.000000pt}%
\definecolor{currentstroke}{rgb}{0.000000,0.000000,0.000000}%
\pgfsetstrokecolor{currentstroke}%
\pgfsetstrokeopacity{0.700000}%
\pgfsetdash{}{0pt}%
\pgfpathmoveto{\pgfqpoint{5.647905in}{0.550000in}}%
\pgfpathlineto{\pgfqpoint{5.675426in}{0.550000in}}%
\pgfpathlineto{\pgfqpoint{5.675426in}{1.425324in}}%
\pgfpathlineto{\pgfqpoint{5.647905in}{1.425324in}}%
\pgfpathlineto{\pgfqpoint{5.647905in}{0.550000in}}%
\pgfpathclose%
\pgfusepath{fill}%
\end{pgfscope}%
\begin{pgfscope}%
\pgfpathrectangle{\pgfqpoint{1.250000in}{0.550000in}}{\pgfqpoint{7.750000in}{3.850000in}}%
\pgfusepath{clip}%
\pgfsetbuttcap%
\pgfsetmiterjoin%
\definecolor{currentfill}{rgb}{0.501961,0.501961,0.501961}%
\pgfsetfillcolor{currentfill}%
\pgfsetfillopacity{0.700000}%
\pgfsetlinewidth{0.000000pt}%
\definecolor{currentstroke}{rgb}{0.000000,0.000000,0.000000}%
\pgfsetstrokecolor{currentstroke}%
\pgfsetstrokeopacity{0.700000}%
\pgfsetdash{}{0pt}%
\pgfpathmoveto{\pgfqpoint{5.675426in}{0.550000in}}%
\pgfpathlineto{\pgfqpoint{5.702947in}{0.550000in}}%
\pgfpathlineto{\pgfqpoint{5.702947in}{0.550000in}}%
\pgfpathlineto{\pgfqpoint{5.675426in}{0.550000in}}%
\pgfpathlineto{\pgfqpoint{5.675426in}{0.550000in}}%
\pgfpathclose%
\pgfusepath{fill}%
\end{pgfscope}%
\begin{pgfscope}%
\pgfpathrectangle{\pgfqpoint{1.250000in}{0.550000in}}{\pgfqpoint{7.750000in}{3.850000in}}%
\pgfusepath{clip}%
\pgfsetbuttcap%
\pgfsetmiterjoin%
\definecolor{currentfill}{rgb}{0.501961,0.501961,0.501961}%
\pgfsetfillcolor{currentfill}%
\pgfsetfillopacity{0.700000}%
\pgfsetlinewidth{0.000000pt}%
\definecolor{currentstroke}{rgb}{0.000000,0.000000,0.000000}%
\pgfsetstrokecolor{currentstroke}%
\pgfsetstrokeopacity{0.700000}%
\pgfsetdash{}{0pt}%
\pgfpathmoveto{\pgfqpoint{5.702947in}{0.550000in}}%
\pgfpathlineto{\pgfqpoint{5.730469in}{0.550000in}}%
\pgfpathlineto{\pgfqpoint{5.730469in}{1.445283in}}%
\pgfpathlineto{\pgfqpoint{5.702947in}{1.445283in}}%
\pgfpathlineto{\pgfqpoint{5.702947in}{0.550000in}}%
\pgfpathclose%
\pgfusepath{fill}%
\end{pgfscope}%
\begin{pgfscope}%
\pgfpathrectangle{\pgfqpoint{1.250000in}{0.550000in}}{\pgfqpoint{7.750000in}{3.850000in}}%
\pgfusepath{clip}%
\pgfsetbuttcap%
\pgfsetmiterjoin%
\definecolor{currentfill}{rgb}{0.501961,0.501961,0.501961}%
\pgfsetfillcolor{currentfill}%
\pgfsetfillopacity{0.700000}%
\pgfsetlinewidth{0.000000pt}%
\definecolor{currentstroke}{rgb}{0.000000,0.000000,0.000000}%
\pgfsetstrokecolor{currentstroke}%
\pgfsetstrokeopacity{0.700000}%
\pgfsetdash{}{0pt}%
\pgfpathmoveto{\pgfqpoint{5.730469in}{0.550000in}}%
\pgfpathlineto{\pgfqpoint{5.757990in}{0.550000in}}%
\pgfpathlineto{\pgfqpoint{5.757990in}{0.550000in}}%
\pgfpathlineto{\pgfqpoint{5.730469in}{0.550000in}}%
\pgfpathlineto{\pgfqpoint{5.730469in}{0.550000in}}%
\pgfpathclose%
\pgfusepath{fill}%
\end{pgfscope}%
\begin{pgfscope}%
\pgfpathrectangle{\pgfqpoint{1.250000in}{0.550000in}}{\pgfqpoint{7.750000in}{3.850000in}}%
\pgfusepath{clip}%
\pgfsetbuttcap%
\pgfsetmiterjoin%
\definecolor{currentfill}{rgb}{0.501961,0.501961,0.501961}%
\pgfsetfillcolor{currentfill}%
\pgfsetfillopacity{0.700000}%
\pgfsetlinewidth{0.000000pt}%
\definecolor{currentstroke}{rgb}{0.000000,0.000000,0.000000}%
\pgfsetstrokecolor{currentstroke}%
\pgfsetstrokeopacity{0.700000}%
\pgfsetdash{}{0pt}%
\pgfpathmoveto{\pgfqpoint{5.757990in}{0.550000in}}%
\pgfpathlineto{\pgfqpoint{5.785511in}{0.550000in}}%
\pgfpathlineto{\pgfqpoint{5.785511in}{1.374002in}}%
\pgfpathlineto{\pgfqpoint{5.757990in}{1.374002in}}%
\pgfpathlineto{\pgfqpoint{5.757990in}{0.550000in}}%
\pgfpathclose%
\pgfusepath{fill}%
\end{pgfscope}%
\begin{pgfscope}%
\pgfpathrectangle{\pgfqpoint{1.250000in}{0.550000in}}{\pgfqpoint{7.750000in}{3.850000in}}%
\pgfusepath{clip}%
\pgfsetbuttcap%
\pgfsetmiterjoin%
\definecolor{currentfill}{rgb}{0.501961,0.501961,0.501961}%
\pgfsetfillcolor{currentfill}%
\pgfsetfillopacity{0.700000}%
\pgfsetlinewidth{0.000000pt}%
\definecolor{currentstroke}{rgb}{0.000000,0.000000,0.000000}%
\pgfsetstrokecolor{currentstroke}%
\pgfsetstrokeopacity{0.700000}%
\pgfsetdash{}{0pt}%
\pgfpathmoveto{\pgfqpoint{5.785511in}{0.550000in}}%
\pgfpathlineto{\pgfqpoint{5.813033in}{0.550000in}}%
\pgfpathlineto{\pgfqpoint{5.813033in}{0.550000in}}%
\pgfpathlineto{\pgfqpoint{5.785511in}{0.550000in}}%
\pgfpathlineto{\pgfqpoint{5.785511in}{0.550000in}}%
\pgfpathclose%
\pgfusepath{fill}%
\end{pgfscope}%
\begin{pgfscope}%
\pgfpathrectangle{\pgfqpoint{1.250000in}{0.550000in}}{\pgfqpoint{7.750000in}{3.850000in}}%
\pgfusepath{clip}%
\pgfsetbuttcap%
\pgfsetmiterjoin%
\definecolor{currentfill}{rgb}{0.501961,0.501961,0.501961}%
\pgfsetfillcolor{currentfill}%
\pgfsetfillopacity{0.700000}%
\pgfsetlinewidth{0.000000pt}%
\definecolor{currentstroke}{rgb}{0.000000,0.000000,0.000000}%
\pgfsetstrokecolor{currentstroke}%
\pgfsetstrokeopacity{0.700000}%
\pgfsetdash{}{0pt}%
\pgfpathmoveto{\pgfqpoint{5.813033in}{0.550000in}}%
\pgfpathlineto{\pgfqpoint{5.840554in}{0.550000in}}%
\pgfpathlineto{\pgfqpoint{5.840554in}{1.379705in}}%
\pgfpathlineto{\pgfqpoint{5.813033in}{1.379705in}}%
\pgfpathlineto{\pgfqpoint{5.813033in}{0.550000in}}%
\pgfpathclose%
\pgfusepath{fill}%
\end{pgfscope}%
\begin{pgfscope}%
\pgfpathrectangle{\pgfqpoint{1.250000in}{0.550000in}}{\pgfqpoint{7.750000in}{3.850000in}}%
\pgfusepath{clip}%
\pgfsetbuttcap%
\pgfsetmiterjoin%
\definecolor{currentfill}{rgb}{0.501961,0.501961,0.501961}%
\pgfsetfillcolor{currentfill}%
\pgfsetfillopacity{0.700000}%
\pgfsetlinewidth{0.000000pt}%
\definecolor{currentstroke}{rgb}{0.000000,0.000000,0.000000}%
\pgfsetstrokecolor{currentstroke}%
\pgfsetstrokeopacity{0.700000}%
\pgfsetdash{}{0pt}%
\pgfpathmoveto{\pgfqpoint{5.840554in}{0.550000in}}%
\pgfpathlineto{\pgfqpoint{5.868075in}{0.550000in}}%
\pgfpathlineto{\pgfqpoint{5.868075in}{0.550000in}}%
\pgfpathlineto{\pgfqpoint{5.840554in}{0.550000in}}%
\pgfpathlineto{\pgfqpoint{5.840554in}{0.550000in}}%
\pgfpathclose%
\pgfusepath{fill}%
\end{pgfscope}%
\begin{pgfscope}%
\pgfpathrectangle{\pgfqpoint{1.250000in}{0.550000in}}{\pgfqpoint{7.750000in}{3.850000in}}%
\pgfusepath{clip}%
\pgfsetbuttcap%
\pgfsetmiterjoin%
\definecolor{currentfill}{rgb}{0.501961,0.501961,0.501961}%
\pgfsetfillcolor{currentfill}%
\pgfsetfillopacity{0.700000}%
\pgfsetlinewidth{0.000000pt}%
\definecolor{currentstroke}{rgb}{0.000000,0.000000,0.000000}%
\pgfsetstrokecolor{currentstroke}%
\pgfsetstrokeopacity{0.700000}%
\pgfsetdash{}{0pt}%
\pgfpathmoveto{\pgfqpoint{5.868075in}{0.550000in}}%
\pgfpathlineto{\pgfqpoint{5.895597in}{0.550000in}}%
\pgfpathlineto{\pgfqpoint{5.895597in}{1.278486in}}%
\pgfpathlineto{\pgfqpoint{5.868075in}{1.278486in}}%
\pgfpathlineto{\pgfqpoint{5.868075in}{0.550000in}}%
\pgfpathclose%
\pgfusepath{fill}%
\end{pgfscope}%
\begin{pgfscope}%
\pgfpathrectangle{\pgfqpoint{1.250000in}{0.550000in}}{\pgfqpoint{7.750000in}{3.850000in}}%
\pgfusepath{clip}%
\pgfsetbuttcap%
\pgfsetmiterjoin%
\definecolor{currentfill}{rgb}{0.501961,0.501961,0.501961}%
\pgfsetfillcolor{currentfill}%
\pgfsetfillopacity{0.700000}%
\pgfsetlinewidth{0.000000pt}%
\definecolor{currentstroke}{rgb}{0.000000,0.000000,0.000000}%
\pgfsetstrokecolor{currentstroke}%
\pgfsetstrokeopacity{0.700000}%
\pgfsetdash{}{0pt}%
\pgfpathmoveto{\pgfqpoint{5.895597in}{0.550000in}}%
\pgfpathlineto{\pgfqpoint{5.923118in}{0.550000in}}%
\pgfpathlineto{\pgfqpoint{5.923118in}{0.550000in}}%
\pgfpathlineto{\pgfqpoint{5.895597in}{0.550000in}}%
\pgfpathlineto{\pgfqpoint{5.895597in}{0.550000in}}%
\pgfpathclose%
\pgfusepath{fill}%
\end{pgfscope}%
\begin{pgfscope}%
\pgfpathrectangle{\pgfqpoint{1.250000in}{0.550000in}}{\pgfqpoint{7.750000in}{3.850000in}}%
\pgfusepath{clip}%
\pgfsetbuttcap%
\pgfsetmiterjoin%
\definecolor{currentfill}{rgb}{0.501961,0.501961,0.501961}%
\pgfsetfillcolor{currentfill}%
\pgfsetfillopacity{0.700000}%
\pgfsetlinewidth{0.000000pt}%
\definecolor{currentstroke}{rgb}{0.000000,0.000000,0.000000}%
\pgfsetstrokecolor{currentstroke}%
\pgfsetstrokeopacity{0.700000}%
\pgfsetdash{}{0pt}%
\pgfpathmoveto{\pgfqpoint{5.923118in}{0.550000in}}%
\pgfpathlineto{\pgfqpoint{5.950639in}{0.550000in}}%
\pgfpathlineto{\pgfqpoint{5.950639in}{1.272784in}}%
\pgfpathlineto{\pgfqpoint{5.923118in}{1.272784in}}%
\pgfpathlineto{\pgfqpoint{5.923118in}{0.550000in}}%
\pgfpathclose%
\pgfusepath{fill}%
\end{pgfscope}%
\begin{pgfscope}%
\pgfpathrectangle{\pgfqpoint{1.250000in}{0.550000in}}{\pgfqpoint{7.750000in}{3.850000in}}%
\pgfusepath{clip}%
\pgfsetbuttcap%
\pgfsetmiterjoin%
\definecolor{currentfill}{rgb}{0.501961,0.501961,0.501961}%
\pgfsetfillcolor{currentfill}%
\pgfsetfillopacity{0.700000}%
\pgfsetlinewidth{0.000000pt}%
\definecolor{currentstroke}{rgb}{0.000000,0.000000,0.000000}%
\pgfsetstrokecolor{currentstroke}%
\pgfsetstrokeopacity{0.700000}%
\pgfsetdash{}{0pt}%
\pgfpathmoveto{\pgfqpoint{5.950639in}{0.550000in}}%
\pgfpathlineto{\pgfqpoint{5.978161in}{0.550000in}}%
\pgfpathlineto{\pgfqpoint{5.978161in}{1.314126in}}%
\pgfpathlineto{\pgfqpoint{5.950639in}{1.314126in}}%
\pgfpathlineto{\pgfqpoint{5.950639in}{0.550000in}}%
\pgfpathclose%
\pgfusepath{fill}%
\end{pgfscope}%
\begin{pgfscope}%
\pgfpathrectangle{\pgfqpoint{1.250000in}{0.550000in}}{\pgfqpoint{7.750000in}{3.850000in}}%
\pgfusepath{clip}%
\pgfsetbuttcap%
\pgfsetmiterjoin%
\definecolor{currentfill}{rgb}{0.501961,0.501961,0.501961}%
\pgfsetfillcolor{currentfill}%
\pgfsetfillopacity{0.700000}%
\pgfsetlinewidth{0.000000pt}%
\definecolor{currentstroke}{rgb}{0.000000,0.000000,0.000000}%
\pgfsetstrokecolor{currentstroke}%
\pgfsetstrokeopacity{0.700000}%
\pgfsetdash{}{0pt}%
\pgfpathmoveto{\pgfqpoint{5.978161in}{0.550000in}}%
\pgfpathlineto{\pgfqpoint{6.005682in}{0.550000in}}%
\pgfpathlineto{\pgfqpoint{6.005682in}{0.550000in}}%
\pgfpathlineto{\pgfqpoint{5.978161in}{0.550000in}}%
\pgfpathlineto{\pgfqpoint{5.978161in}{0.550000in}}%
\pgfpathclose%
\pgfusepath{fill}%
\end{pgfscope}%
\begin{pgfscope}%
\pgfpathrectangle{\pgfqpoint{1.250000in}{0.550000in}}{\pgfqpoint{7.750000in}{3.850000in}}%
\pgfusepath{clip}%
\pgfsetbuttcap%
\pgfsetmiterjoin%
\definecolor{currentfill}{rgb}{0.501961,0.501961,0.501961}%
\pgfsetfillcolor{currentfill}%
\pgfsetfillopacity{0.700000}%
\pgfsetlinewidth{0.000000pt}%
\definecolor{currentstroke}{rgb}{0.000000,0.000000,0.000000}%
\pgfsetstrokecolor{currentstroke}%
\pgfsetstrokeopacity{0.700000}%
\pgfsetdash{}{0pt}%
\pgfpathmoveto{\pgfqpoint{6.005682in}{0.550000in}}%
\pgfpathlineto{\pgfqpoint{6.033203in}{0.550000in}}%
\pgfpathlineto{\pgfqpoint{6.033203in}{1.261379in}}%
\pgfpathlineto{\pgfqpoint{6.005682in}{1.261379in}}%
\pgfpathlineto{\pgfqpoint{6.005682in}{0.550000in}}%
\pgfpathclose%
\pgfusepath{fill}%
\end{pgfscope}%
\begin{pgfscope}%
\pgfpathrectangle{\pgfqpoint{1.250000in}{0.550000in}}{\pgfqpoint{7.750000in}{3.850000in}}%
\pgfusepath{clip}%
\pgfsetbuttcap%
\pgfsetmiterjoin%
\definecolor{currentfill}{rgb}{0.501961,0.501961,0.501961}%
\pgfsetfillcolor{currentfill}%
\pgfsetfillopacity{0.700000}%
\pgfsetlinewidth{0.000000pt}%
\definecolor{currentstroke}{rgb}{0.000000,0.000000,0.000000}%
\pgfsetstrokecolor{currentstroke}%
\pgfsetstrokeopacity{0.700000}%
\pgfsetdash{}{0pt}%
\pgfpathmoveto{\pgfqpoint{6.033203in}{0.550000in}}%
\pgfpathlineto{\pgfqpoint{6.060724in}{0.550000in}}%
\pgfpathlineto{\pgfqpoint{6.060724in}{0.550000in}}%
\pgfpathlineto{\pgfqpoint{6.033203in}{0.550000in}}%
\pgfpathlineto{\pgfqpoint{6.033203in}{0.550000in}}%
\pgfpathclose%
\pgfusepath{fill}%
\end{pgfscope}%
\begin{pgfscope}%
\pgfpathrectangle{\pgfqpoint{1.250000in}{0.550000in}}{\pgfqpoint{7.750000in}{3.850000in}}%
\pgfusepath{clip}%
\pgfsetbuttcap%
\pgfsetmiterjoin%
\definecolor{currentfill}{rgb}{0.501961,0.501961,0.501961}%
\pgfsetfillcolor{currentfill}%
\pgfsetfillopacity{0.700000}%
\pgfsetlinewidth{0.000000pt}%
\definecolor{currentstroke}{rgb}{0.000000,0.000000,0.000000}%
\pgfsetstrokecolor{currentstroke}%
\pgfsetstrokeopacity{0.700000}%
\pgfsetdash{}{0pt}%
\pgfpathmoveto{\pgfqpoint{6.060724in}{0.550000in}}%
\pgfpathlineto{\pgfqpoint{6.088246in}{0.550000in}}%
\pgfpathlineto{\pgfqpoint{6.088246in}{1.207206in}}%
\pgfpathlineto{\pgfqpoint{6.060724in}{1.207206in}}%
\pgfpathlineto{\pgfqpoint{6.060724in}{0.550000in}}%
\pgfpathclose%
\pgfusepath{fill}%
\end{pgfscope}%
\begin{pgfscope}%
\pgfpathrectangle{\pgfqpoint{1.250000in}{0.550000in}}{\pgfqpoint{7.750000in}{3.850000in}}%
\pgfusepath{clip}%
\pgfsetbuttcap%
\pgfsetmiterjoin%
\definecolor{currentfill}{rgb}{0.501961,0.501961,0.501961}%
\pgfsetfillcolor{currentfill}%
\pgfsetfillopacity{0.700000}%
\pgfsetlinewidth{0.000000pt}%
\definecolor{currentstroke}{rgb}{0.000000,0.000000,0.000000}%
\pgfsetstrokecolor{currentstroke}%
\pgfsetstrokeopacity{0.700000}%
\pgfsetdash{}{0pt}%
\pgfpathmoveto{\pgfqpoint{6.088246in}{0.550000in}}%
\pgfpathlineto{\pgfqpoint{6.115767in}{0.550000in}}%
\pgfpathlineto{\pgfqpoint{6.115767in}{0.550000in}}%
\pgfpathlineto{\pgfqpoint{6.088246in}{0.550000in}}%
\pgfpathlineto{\pgfqpoint{6.088246in}{0.550000in}}%
\pgfpathclose%
\pgfusepath{fill}%
\end{pgfscope}%
\begin{pgfscope}%
\pgfpathrectangle{\pgfqpoint{1.250000in}{0.550000in}}{\pgfqpoint{7.750000in}{3.850000in}}%
\pgfusepath{clip}%
\pgfsetbuttcap%
\pgfsetmiterjoin%
\definecolor{currentfill}{rgb}{0.501961,0.501961,0.501961}%
\pgfsetfillcolor{currentfill}%
\pgfsetfillopacity{0.700000}%
\pgfsetlinewidth{0.000000pt}%
\definecolor{currentstroke}{rgb}{0.000000,0.000000,0.000000}%
\pgfsetstrokecolor{currentstroke}%
\pgfsetstrokeopacity{0.700000}%
\pgfsetdash{}{0pt}%
\pgfpathmoveto{\pgfqpoint{6.115767in}{0.550000in}}%
\pgfpathlineto{\pgfqpoint{6.143288in}{0.550000in}}%
\pgfpathlineto{\pgfqpoint{6.143288in}{1.194375in}}%
\pgfpathlineto{\pgfqpoint{6.115767in}{1.194375in}}%
\pgfpathlineto{\pgfqpoint{6.115767in}{0.550000in}}%
\pgfpathclose%
\pgfusepath{fill}%
\end{pgfscope}%
\begin{pgfscope}%
\pgfpathrectangle{\pgfqpoint{1.250000in}{0.550000in}}{\pgfqpoint{7.750000in}{3.850000in}}%
\pgfusepath{clip}%
\pgfsetbuttcap%
\pgfsetmiterjoin%
\definecolor{currentfill}{rgb}{0.501961,0.501961,0.501961}%
\pgfsetfillcolor{currentfill}%
\pgfsetfillopacity{0.700000}%
\pgfsetlinewidth{0.000000pt}%
\definecolor{currentstroke}{rgb}{0.000000,0.000000,0.000000}%
\pgfsetstrokecolor{currentstroke}%
\pgfsetstrokeopacity{0.700000}%
\pgfsetdash{}{0pt}%
\pgfpathmoveto{\pgfqpoint{6.143288in}{0.550000in}}%
\pgfpathlineto{\pgfqpoint{6.170810in}{0.550000in}}%
\pgfpathlineto{\pgfqpoint{6.170810in}{0.550000in}}%
\pgfpathlineto{\pgfqpoint{6.143288in}{0.550000in}}%
\pgfpathlineto{\pgfqpoint{6.143288in}{0.550000in}}%
\pgfpathclose%
\pgfusepath{fill}%
\end{pgfscope}%
\begin{pgfscope}%
\pgfpathrectangle{\pgfqpoint{1.250000in}{0.550000in}}{\pgfqpoint{7.750000in}{3.850000in}}%
\pgfusepath{clip}%
\pgfsetbuttcap%
\pgfsetmiterjoin%
\definecolor{currentfill}{rgb}{0.501961,0.501961,0.501961}%
\pgfsetfillcolor{currentfill}%
\pgfsetfillopacity{0.700000}%
\pgfsetlinewidth{0.000000pt}%
\definecolor{currentstroke}{rgb}{0.000000,0.000000,0.000000}%
\pgfsetstrokecolor{currentstroke}%
\pgfsetstrokeopacity{0.700000}%
\pgfsetdash{}{0pt}%
\pgfpathmoveto{\pgfqpoint{6.170810in}{0.550000in}}%
\pgfpathlineto{\pgfqpoint{6.198331in}{0.550000in}}%
\pgfpathlineto{\pgfqpoint{6.198331in}{1.222888in}}%
\pgfpathlineto{\pgfqpoint{6.170810in}{1.222888in}}%
\pgfpathlineto{\pgfqpoint{6.170810in}{0.550000in}}%
\pgfpathclose%
\pgfusepath{fill}%
\end{pgfscope}%
\begin{pgfscope}%
\pgfpathrectangle{\pgfqpoint{1.250000in}{0.550000in}}{\pgfqpoint{7.750000in}{3.850000in}}%
\pgfusepath{clip}%
\pgfsetbuttcap%
\pgfsetmiterjoin%
\definecolor{currentfill}{rgb}{0.501961,0.501961,0.501961}%
\pgfsetfillcolor{currentfill}%
\pgfsetfillopacity{0.700000}%
\pgfsetlinewidth{0.000000pt}%
\definecolor{currentstroke}{rgb}{0.000000,0.000000,0.000000}%
\pgfsetstrokecolor{currentstroke}%
\pgfsetstrokeopacity{0.700000}%
\pgfsetdash{}{0pt}%
\pgfpathmoveto{\pgfqpoint{6.198331in}{0.550000in}}%
\pgfpathlineto{\pgfqpoint{6.225852in}{0.550000in}}%
\pgfpathlineto{\pgfqpoint{6.225852in}{0.550000in}}%
\pgfpathlineto{\pgfqpoint{6.198331in}{0.550000in}}%
\pgfpathlineto{\pgfqpoint{6.198331in}{0.550000in}}%
\pgfpathclose%
\pgfusepath{fill}%
\end{pgfscope}%
\begin{pgfscope}%
\pgfpathrectangle{\pgfqpoint{1.250000in}{0.550000in}}{\pgfqpoint{7.750000in}{3.850000in}}%
\pgfusepath{clip}%
\pgfsetbuttcap%
\pgfsetmiterjoin%
\definecolor{currentfill}{rgb}{0.501961,0.501961,0.501961}%
\pgfsetfillcolor{currentfill}%
\pgfsetfillopacity{0.700000}%
\pgfsetlinewidth{0.000000pt}%
\definecolor{currentstroke}{rgb}{0.000000,0.000000,0.000000}%
\pgfsetstrokecolor{currentstroke}%
\pgfsetstrokeopacity{0.700000}%
\pgfsetdash{}{0pt}%
\pgfpathmoveto{\pgfqpoint{6.225852in}{0.550000in}}%
\pgfpathlineto{\pgfqpoint{6.253374in}{0.550000in}}%
\pgfpathlineto{\pgfqpoint{6.253374in}{1.191524in}}%
\pgfpathlineto{\pgfqpoint{6.225852in}{1.191524in}}%
\pgfpathlineto{\pgfqpoint{6.225852in}{0.550000in}}%
\pgfpathclose%
\pgfusepath{fill}%
\end{pgfscope}%
\begin{pgfscope}%
\pgfpathrectangle{\pgfqpoint{1.250000in}{0.550000in}}{\pgfqpoint{7.750000in}{3.850000in}}%
\pgfusepath{clip}%
\pgfsetbuttcap%
\pgfsetmiterjoin%
\definecolor{currentfill}{rgb}{0.501961,0.501961,0.501961}%
\pgfsetfillcolor{currentfill}%
\pgfsetfillopacity{0.700000}%
\pgfsetlinewidth{0.000000pt}%
\definecolor{currentstroke}{rgb}{0.000000,0.000000,0.000000}%
\pgfsetstrokecolor{currentstroke}%
\pgfsetstrokeopacity{0.700000}%
\pgfsetdash{}{0pt}%
\pgfpathmoveto{\pgfqpoint{6.253374in}{0.550000in}}%
\pgfpathlineto{\pgfqpoint{6.280895in}{0.550000in}}%
\pgfpathlineto{\pgfqpoint{6.280895in}{0.550000in}}%
\pgfpathlineto{\pgfqpoint{6.253374in}{0.550000in}}%
\pgfpathlineto{\pgfqpoint{6.253374in}{0.550000in}}%
\pgfpathclose%
\pgfusepath{fill}%
\end{pgfscope}%
\begin{pgfscope}%
\pgfpathrectangle{\pgfqpoint{1.250000in}{0.550000in}}{\pgfqpoint{7.750000in}{3.850000in}}%
\pgfusepath{clip}%
\pgfsetbuttcap%
\pgfsetmiterjoin%
\definecolor{currentfill}{rgb}{0.501961,0.501961,0.501961}%
\pgfsetfillcolor{currentfill}%
\pgfsetfillopacity{0.700000}%
\pgfsetlinewidth{0.000000pt}%
\definecolor{currentstroke}{rgb}{0.000000,0.000000,0.000000}%
\pgfsetstrokecolor{currentstroke}%
\pgfsetstrokeopacity{0.700000}%
\pgfsetdash{}{0pt}%
\pgfpathmoveto{\pgfqpoint{6.280895in}{0.550000in}}%
\pgfpathlineto{\pgfqpoint{6.308416in}{0.550000in}}%
\pgfpathlineto{\pgfqpoint{6.308416in}{1.161586in}}%
\pgfpathlineto{\pgfqpoint{6.280895in}{1.161586in}}%
\pgfpathlineto{\pgfqpoint{6.280895in}{0.550000in}}%
\pgfpathclose%
\pgfusepath{fill}%
\end{pgfscope}%
\begin{pgfscope}%
\pgfpathrectangle{\pgfqpoint{1.250000in}{0.550000in}}{\pgfqpoint{7.750000in}{3.850000in}}%
\pgfusepath{clip}%
\pgfsetbuttcap%
\pgfsetmiterjoin%
\definecolor{currentfill}{rgb}{0.501961,0.501961,0.501961}%
\pgfsetfillcolor{currentfill}%
\pgfsetfillopacity{0.700000}%
\pgfsetlinewidth{0.000000pt}%
\definecolor{currentstroke}{rgb}{0.000000,0.000000,0.000000}%
\pgfsetstrokecolor{currentstroke}%
\pgfsetstrokeopacity{0.700000}%
\pgfsetdash{}{0pt}%
\pgfpathmoveto{\pgfqpoint{6.308416in}{0.550000in}}%
\pgfpathlineto{\pgfqpoint{6.335937in}{0.550000in}}%
\pgfpathlineto{\pgfqpoint{6.335937in}{0.550000in}}%
\pgfpathlineto{\pgfqpoint{6.308416in}{0.550000in}}%
\pgfpathlineto{\pgfqpoint{6.308416in}{0.550000in}}%
\pgfpathclose%
\pgfusepath{fill}%
\end{pgfscope}%
\begin{pgfscope}%
\pgfpathrectangle{\pgfqpoint{1.250000in}{0.550000in}}{\pgfqpoint{7.750000in}{3.850000in}}%
\pgfusepath{clip}%
\pgfsetbuttcap%
\pgfsetmiterjoin%
\definecolor{currentfill}{rgb}{0.501961,0.501961,0.501961}%
\pgfsetfillcolor{currentfill}%
\pgfsetfillopacity{0.700000}%
\pgfsetlinewidth{0.000000pt}%
\definecolor{currentstroke}{rgb}{0.000000,0.000000,0.000000}%
\pgfsetstrokecolor{currentstroke}%
\pgfsetstrokeopacity{0.700000}%
\pgfsetdash{}{0pt}%
\pgfpathmoveto{\pgfqpoint{6.335938in}{0.550000in}}%
\pgfpathlineto{\pgfqpoint{6.363459in}{0.550000in}}%
\pgfpathlineto{\pgfqpoint{6.363459in}{1.110264in}}%
\pgfpathlineto{\pgfqpoint{6.335938in}{1.110264in}}%
\pgfpathlineto{\pgfqpoint{6.335938in}{0.550000in}}%
\pgfpathclose%
\pgfusepath{fill}%
\end{pgfscope}%
\begin{pgfscope}%
\pgfpathrectangle{\pgfqpoint{1.250000in}{0.550000in}}{\pgfqpoint{7.750000in}{3.850000in}}%
\pgfusepath{clip}%
\pgfsetbuttcap%
\pgfsetmiterjoin%
\definecolor{currentfill}{rgb}{0.501961,0.501961,0.501961}%
\pgfsetfillcolor{currentfill}%
\pgfsetfillopacity{0.700000}%
\pgfsetlinewidth{0.000000pt}%
\definecolor{currentstroke}{rgb}{0.000000,0.000000,0.000000}%
\pgfsetstrokecolor{currentstroke}%
\pgfsetstrokeopacity{0.700000}%
\pgfsetdash{}{0pt}%
\pgfpathmoveto{\pgfqpoint{6.363459in}{0.550000in}}%
\pgfpathlineto{\pgfqpoint{6.390980in}{0.550000in}}%
\pgfpathlineto{\pgfqpoint{6.390980in}{0.550000in}}%
\pgfpathlineto{\pgfqpoint{6.363459in}{0.550000in}}%
\pgfpathlineto{\pgfqpoint{6.363459in}{0.550000in}}%
\pgfpathclose%
\pgfusepath{fill}%
\end{pgfscope}%
\begin{pgfscope}%
\pgfpathrectangle{\pgfqpoint{1.250000in}{0.550000in}}{\pgfqpoint{7.750000in}{3.850000in}}%
\pgfusepath{clip}%
\pgfsetbuttcap%
\pgfsetmiterjoin%
\definecolor{currentfill}{rgb}{0.501961,0.501961,0.501961}%
\pgfsetfillcolor{currentfill}%
\pgfsetfillopacity{0.700000}%
\pgfsetlinewidth{0.000000pt}%
\definecolor{currentstroke}{rgb}{0.000000,0.000000,0.000000}%
\pgfsetstrokecolor{currentstroke}%
\pgfsetstrokeopacity{0.700000}%
\pgfsetdash{}{0pt}%
\pgfpathmoveto{\pgfqpoint{6.390980in}{0.550000in}}%
\pgfpathlineto{\pgfqpoint{6.418501in}{0.550000in}}%
\pgfpathlineto{\pgfqpoint{6.418501in}{1.125946in}}%
\pgfpathlineto{\pgfqpoint{6.390980in}{1.125946in}}%
\pgfpathlineto{\pgfqpoint{6.390980in}{0.550000in}}%
\pgfpathclose%
\pgfusepath{fill}%
\end{pgfscope}%
\begin{pgfscope}%
\pgfpathrectangle{\pgfqpoint{1.250000in}{0.550000in}}{\pgfqpoint{7.750000in}{3.850000in}}%
\pgfusepath{clip}%
\pgfsetbuttcap%
\pgfsetmiterjoin%
\definecolor{currentfill}{rgb}{0.501961,0.501961,0.501961}%
\pgfsetfillcolor{currentfill}%
\pgfsetfillopacity{0.700000}%
\pgfsetlinewidth{0.000000pt}%
\definecolor{currentstroke}{rgb}{0.000000,0.000000,0.000000}%
\pgfsetstrokecolor{currentstroke}%
\pgfsetstrokeopacity{0.700000}%
\pgfsetdash{}{0pt}%
\pgfpathmoveto{\pgfqpoint{6.418501in}{0.550000in}}%
\pgfpathlineto{\pgfqpoint{6.446023in}{0.550000in}}%
\pgfpathlineto{\pgfqpoint{6.446023in}{0.550000in}}%
\pgfpathlineto{\pgfqpoint{6.418501in}{0.550000in}}%
\pgfpathlineto{\pgfqpoint{6.418501in}{0.550000in}}%
\pgfpathclose%
\pgfusepath{fill}%
\end{pgfscope}%
\begin{pgfscope}%
\pgfpathrectangle{\pgfqpoint{1.250000in}{0.550000in}}{\pgfqpoint{7.750000in}{3.850000in}}%
\pgfusepath{clip}%
\pgfsetbuttcap%
\pgfsetmiterjoin%
\definecolor{currentfill}{rgb}{0.501961,0.501961,0.501961}%
\pgfsetfillcolor{currentfill}%
\pgfsetfillopacity{0.700000}%
\pgfsetlinewidth{0.000000pt}%
\definecolor{currentstroke}{rgb}{0.000000,0.000000,0.000000}%
\pgfsetstrokecolor{currentstroke}%
\pgfsetstrokeopacity{0.700000}%
\pgfsetdash{}{0pt}%
\pgfpathmoveto{\pgfqpoint{6.446023in}{0.550000in}}%
\pgfpathlineto{\pgfqpoint{6.473544in}{0.550000in}}%
\pgfpathlineto{\pgfqpoint{6.473544in}{1.086029in}}%
\pgfpathlineto{\pgfqpoint{6.446023in}{1.086029in}}%
\pgfpathlineto{\pgfqpoint{6.446023in}{0.550000in}}%
\pgfpathclose%
\pgfusepath{fill}%
\end{pgfscope}%
\begin{pgfscope}%
\pgfpathrectangle{\pgfqpoint{1.250000in}{0.550000in}}{\pgfqpoint{7.750000in}{3.850000in}}%
\pgfusepath{clip}%
\pgfsetbuttcap%
\pgfsetmiterjoin%
\definecolor{currentfill}{rgb}{0.501961,0.501961,0.501961}%
\pgfsetfillcolor{currentfill}%
\pgfsetfillopacity{0.700000}%
\pgfsetlinewidth{0.000000pt}%
\definecolor{currentstroke}{rgb}{0.000000,0.000000,0.000000}%
\pgfsetstrokecolor{currentstroke}%
\pgfsetstrokeopacity{0.700000}%
\pgfsetdash{}{0pt}%
\pgfpathmoveto{\pgfqpoint{6.473544in}{0.550000in}}%
\pgfpathlineto{\pgfqpoint{6.501065in}{0.550000in}}%
\pgfpathlineto{\pgfqpoint{6.501065in}{0.550000in}}%
\pgfpathlineto{\pgfqpoint{6.473544in}{0.550000in}}%
\pgfpathlineto{\pgfqpoint{6.473544in}{0.550000in}}%
\pgfpathclose%
\pgfusepath{fill}%
\end{pgfscope}%
\begin{pgfscope}%
\pgfpathrectangle{\pgfqpoint{1.250000in}{0.550000in}}{\pgfqpoint{7.750000in}{3.850000in}}%
\pgfusepath{clip}%
\pgfsetbuttcap%
\pgfsetmiterjoin%
\definecolor{currentfill}{rgb}{0.501961,0.501961,0.501961}%
\pgfsetfillcolor{currentfill}%
\pgfsetfillopacity{0.700000}%
\pgfsetlinewidth{0.000000pt}%
\definecolor{currentstroke}{rgb}{0.000000,0.000000,0.000000}%
\pgfsetstrokecolor{currentstroke}%
\pgfsetstrokeopacity{0.700000}%
\pgfsetdash{}{0pt}%
\pgfpathmoveto{\pgfqpoint{6.501065in}{0.550000in}}%
\pgfpathlineto{\pgfqpoint{6.528587in}{0.550000in}}%
\pgfpathlineto{\pgfqpoint{6.528587in}{1.103136in}}%
\pgfpathlineto{\pgfqpoint{6.501065in}{1.103136in}}%
\pgfpathlineto{\pgfqpoint{6.501065in}{0.550000in}}%
\pgfpathclose%
\pgfusepath{fill}%
\end{pgfscope}%
\begin{pgfscope}%
\pgfpathrectangle{\pgfqpoint{1.250000in}{0.550000in}}{\pgfqpoint{7.750000in}{3.850000in}}%
\pgfusepath{clip}%
\pgfsetbuttcap%
\pgfsetmiterjoin%
\definecolor{currentfill}{rgb}{0.501961,0.501961,0.501961}%
\pgfsetfillcolor{currentfill}%
\pgfsetfillopacity{0.700000}%
\pgfsetlinewidth{0.000000pt}%
\definecolor{currentstroke}{rgb}{0.000000,0.000000,0.000000}%
\pgfsetstrokecolor{currentstroke}%
\pgfsetstrokeopacity{0.700000}%
\pgfsetdash{}{0pt}%
\pgfpathmoveto{\pgfqpoint{6.528587in}{0.550000in}}%
\pgfpathlineto{\pgfqpoint{6.556108in}{0.550000in}}%
\pgfpathlineto{\pgfqpoint{6.556108in}{0.550000in}}%
\pgfpathlineto{\pgfqpoint{6.528587in}{0.550000in}}%
\pgfpathlineto{\pgfqpoint{6.528587in}{0.550000in}}%
\pgfpathclose%
\pgfusepath{fill}%
\end{pgfscope}%
\begin{pgfscope}%
\pgfpathrectangle{\pgfqpoint{1.250000in}{0.550000in}}{\pgfqpoint{7.750000in}{3.850000in}}%
\pgfusepath{clip}%
\pgfsetbuttcap%
\pgfsetmiterjoin%
\definecolor{currentfill}{rgb}{0.501961,0.501961,0.501961}%
\pgfsetfillcolor{currentfill}%
\pgfsetfillopacity{0.700000}%
\pgfsetlinewidth{0.000000pt}%
\definecolor{currentstroke}{rgb}{0.000000,0.000000,0.000000}%
\pgfsetstrokecolor{currentstroke}%
\pgfsetstrokeopacity{0.700000}%
\pgfsetdash{}{0pt}%
\pgfpathmoveto{\pgfqpoint{6.556108in}{0.550000in}}%
\pgfpathlineto{\pgfqpoint{6.583629in}{0.550000in}}%
\pgfpathlineto{\pgfqpoint{6.583629in}{1.044686in}}%
\pgfpathlineto{\pgfqpoint{6.556108in}{1.044686in}}%
\pgfpathlineto{\pgfqpoint{6.556108in}{0.550000in}}%
\pgfpathclose%
\pgfusepath{fill}%
\end{pgfscope}%
\begin{pgfscope}%
\pgfpathrectangle{\pgfqpoint{1.250000in}{0.550000in}}{\pgfqpoint{7.750000in}{3.850000in}}%
\pgfusepath{clip}%
\pgfsetbuttcap%
\pgfsetmiterjoin%
\definecolor{currentfill}{rgb}{0.501961,0.501961,0.501961}%
\pgfsetfillcolor{currentfill}%
\pgfsetfillopacity{0.700000}%
\pgfsetlinewidth{0.000000pt}%
\definecolor{currentstroke}{rgb}{0.000000,0.000000,0.000000}%
\pgfsetstrokecolor{currentstroke}%
\pgfsetstrokeopacity{0.700000}%
\pgfsetdash{}{0pt}%
\pgfpathmoveto{\pgfqpoint{6.583629in}{0.550000in}}%
\pgfpathlineto{\pgfqpoint{6.611151in}{0.550000in}}%
\pgfpathlineto{\pgfqpoint{6.611151in}{0.550000in}}%
\pgfpathlineto{\pgfqpoint{6.583629in}{0.550000in}}%
\pgfpathlineto{\pgfqpoint{6.583629in}{0.550000in}}%
\pgfpathclose%
\pgfusepath{fill}%
\end{pgfscope}%
\begin{pgfscope}%
\pgfpathrectangle{\pgfqpoint{1.250000in}{0.550000in}}{\pgfqpoint{7.750000in}{3.850000in}}%
\pgfusepath{clip}%
\pgfsetbuttcap%
\pgfsetmiterjoin%
\definecolor{currentfill}{rgb}{0.501961,0.501961,0.501961}%
\pgfsetfillcolor{currentfill}%
\pgfsetfillopacity{0.700000}%
\pgfsetlinewidth{0.000000pt}%
\definecolor{currentstroke}{rgb}{0.000000,0.000000,0.000000}%
\pgfsetstrokecolor{currentstroke}%
\pgfsetstrokeopacity{0.700000}%
\pgfsetdash{}{0pt}%
\pgfpathmoveto{\pgfqpoint{6.611151in}{0.550000in}}%
\pgfpathlineto{\pgfqpoint{6.638672in}{0.550000in}}%
\pgfpathlineto{\pgfqpoint{6.638672in}{1.033281in}}%
\pgfpathlineto{\pgfqpoint{6.611151in}{1.033281in}}%
\pgfpathlineto{\pgfqpoint{6.611151in}{0.550000in}}%
\pgfpathclose%
\pgfusepath{fill}%
\end{pgfscope}%
\begin{pgfscope}%
\pgfpathrectangle{\pgfqpoint{1.250000in}{0.550000in}}{\pgfqpoint{7.750000in}{3.850000in}}%
\pgfusepath{clip}%
\pgfsetbuttcap%
\pgfsetmiterjoin%
\definecolor{currentfill}{rgb}{0.501961,0.501961,0.501961}%
\pgfsetfillcolor{currentfill}%
\pgfsetfillopacity{0.700000}%
\pgfsetlinewidth{0.000000pt}%
\definecolor{currentstroke}{rgb}{0.000000,0.000000,0.000000}%
\pgfsetstrokecolor{currentstroke}%
\pgfsetstrokeopacity{0.700000}%
\pgfsetdash{}{0pt}%
\pgfpathmoveto{\pgfqpoint{6.638672in}{0.550000in}}%
\pgfpathlineto{\pgfqpoint{6.666193in}{0.550000in}}%
\pgfpathlineto{\pgfqpoint{6.666193in}{0.550000in}}%
\pgfpathlineto{\pgfqpoint{6.638672in}{0.550000in}}%
\pgfpathlineto{\pgfqpoint{6.638672in}{0.550000in}}%
\pgfpathclose%
\pgfusepath{fill}%
\end{pgfscope}%
\begin{pgfscope}%
\pgfpathrectangle{\pgfqpoint{1.250000in}{0.550000in}}{\pgfqpoint{7.750000in}{3.850000in}}%
\pgfusepath{clip}%
\pgfsetbuttcap%
\pgfsetmiterjoin%
\definecolor{currentfill}{rgb}{0.501961,0.501961,0.501961}%
\pgfsetfillcolor{currentfill}%
\pgfsetfillopacity{0.700000}%
\pgfsetlinewidth{0.000000pt}%
\definecolor{currentstroke}{rgb}{0.000000,0.000000,0.000000}%
\pgfsetstrokecolor{currentstroke}%
\pgfsetstrokeopacity{0.700000}%
\pgfsetdash{}{0pt}%
\pgfpathmoveto{\pgfqpoint{6.666193in}{0.550000in}}%
\pgfpathlineto{\pgfqpoint{6.693714in}{0.550000in}}%
\pgfpathlineto{\pgfqpoint{6.693714in}{1.063219in}}%
\pgfpathlineto{\pgfqpoint{6.666193in}{1.063219in}}%
\pgfpathlineto{\pgfqpoint{6.666193in}{0.550000in}}%
\pgfpathclose%
\pgfusepath{fill}%
\end{pgfscope}%
\begin{pgfscope}%
\pgfpathrectangle{\pgfqpoint{1.250000in}{0.550000in}}{\pgfqpoint{7.750000in}{3.850000in}}%
\pgfusepath{clip}%
\pgfsetbuttcap%
\pgfsetmiterjoin%
\definecolor{currentfill}{rgb}{0.501961,0.501961,0.501961}%
\pgfsetfillcolor{currentfill}%
\pgfsetfillopacity{0.700000}%
\pgfsetlinewidth{0.000000pt}%
\definecolor{currentstroke}{rgb}{0.000000,0.000000,0.000000}%
\pgfsetstrokecolor{currentstroke}%
\pgfsetstrokeopacity{0.700000}%
\pgfsetdash{}{0pt}%
\pgfpathmoveto{\pgfqpoint{6.693714in}{0.550000in}}%
\pgfpathlineto{\pgfqpoint{6.721236in}{0.550000in}}%
\pgfpathlineto{\pgfqpoint{6.721236in}{0.550000in}}%
\pgfpathlineto{\pgfqpoint{6.693714in}{0.550000in}}%
\pgfpathlineto{\pgfqpoint{6.693714in}{0.550000in}}%
\pgfpathclose%
\pgfusepath{fill}%
\end{pgfscope}%
\begin{pgfscope}%
\pgfpathrectangle{\pgfqpoint{1.250000in}{0.550000in}}{\pgfqpoint{7.750000in}{3.850000in}}%
\pgfusepath{clip}%
\pgfsetbuttcap%
\pgfsetmiterjoin%
\definecolor{currentfill}{rgb}{0.501961,0.501961,0.501961}%
\pgfsetfillcolor{currentfill}%
\pgfsetfillopacity{0.700000}%
\pgfsetlinewidth{0.000000pt}%
\definecolor{currentstroke}{rgb}{0.000000,0.000000,0.000000}%
\pgfsetstrokecolor{currentstroke}%
\pgfsetstrokeopacity{0.700000}%
\pgfsetdash{}{0pt}%
\pgfpathmoveto{\pgfqpoint{6.721236in}{0.550000in}}%
\pgfpathlineto{\pgfqpoint{6.748757in}{0.550000in}}%
\pgfpathlineto{\pgfqpoint{6.748757in}{1.017600in}}%
\pgfpathlineto{\pgfqpoint{6.721236in}{1.017600in}}%
\pgfpathlineto{\pgfqpoint{6.721236in}{0.550000in}}%
\pgfpathclose%
\pgfusepath{fill}%
\end{pgfscope}%
\begin{pgfscope}%
\pgfpathrectangle{\pgfqpoint{1.250000in}{0.550000in}}{\pgfqpoint{7.750000in}{3.850000in}}%
\pgfusepath{clip}%
\pgfsetbuttcap%
\pgfsetmiterjoin%
\definecolor{currentfill}{rgb}{0.501961,0.501961,0.501961}%
\pgfsetfillcolor{currentfill}%
\pgfsetfillopacity{0.700000}%
\pgfsetlinewidth{0.000000pt}%
\definecolor{currentstroke}{rgb}{0.000000,0.000000,0.000000}%
\pgfsetstrokecolor{currentstroke}%
\pgfsetstrokeopacity{0.700000}%
\pgfsetdash{}{0pt}%
\pgfpathmoveto{\pgfqpoint{6.748757in}{0.550000in}}%
\pgfpathlineto{\pgfqpoint{6.776278in}{0.550000in}}%
\pgfpathlineto{\pgfqpoint{6.776278in}{0.550000in}}%
\pgfpathlineto{\pgfqpoint{6.748757in}{0.550000in}}%
\pgfpathlineto{\pgfqpoint{6.748757in}{0.550000in}}%
\pgfpathclose%
\pgfusepath{fill}%
\end{pgfscope}%
\begin{pgfscope}%
\pgfpathrectangle{\pgfqpoint{1.250000in}{0.550000in}}{\pgfqpoint{7.750000in}{3.850000in}}%
\pgfusepath{clip}%
\pgfsetbuttcap%
\pgfsetmiterjoin%
\definecolor{currentfill}{rgb}{0.501961,0.501961,0.501961}%
\pgfsetfillcolor{currentfill}%
\pgfsetfillopacity{0.700000}%
\pgfsetlinewidth{0.000000pt}%
\definecolor{currentstroke}{rgb}{0.000000,0.000000,0.000000}%
\pgfsetstrokecolor{currentstroke}%
\pgfsetstrokeopacity{0.700000}%
\pgfsetdash{}{0pt}%
\pgfpathmoveto{\pgfqpoint{6.776278in}{0.550000in}}%
\pgfpathlineto{\pgfqpoint{6.803800in}{0.550000in}}%
\pgfpathlineto{\pgfqpoint{6.803800in}{0.977683in}}%
\pgfpathlineto{\pgfqpoint{6.776278in}{0.977683in}}%
\pgfpathlineto{\pgfqpoint{6.776278in}{0.550000in}}%
\pgfpathclose%
\pgfusepath{fill}%
\end{pgfscope}%
\begin{pgfscope}%
\pgfpathrectangle{\pgfqpoint{1.250000in}{0.550000in}}{\pgfqpoint{7.750000in}{3.850000in}}%
\pgfusepath{clip}%
\pgfsetbuttcap%
\pgfsetmiterjoin%
\definecolor{currentfill}{rgb}{0.501961,0.501961,0.501961}%
\pgfsetfillcolor{currentfill}%
\pgfsetfillopacity{0.700000}%
\pgfsetlinewidth{0.000000pt}%
\definecolor{currentstroke}{rgb}{0.000000,0.000000,0.000000}%
\pgfsetstrokecolor{currentstroke}%
\pgfsetstrokeopacity{0.700000}%
\pgfsetdash{}{0pt}%
\pgfpathmoveto{\pgfqpoint{6.803800in}{0.550000in}}%
\pgfpathlineto{\pgfqpoint{6.831321in}{0.550000in}}%
\pgfpathlineto{\pgfqpoint{6.831321in}{0.550000in}}%
\pgfpathlineto{\pgfqpoint{6.803800in}{0.550000in}}%
\pgfpathlineto{\pgfqpoint{6.803800in}{0.550000in}}%
\pgfpathclose%
\pgfusepath{fill}%
\end{pgfscope}%
\begin{pgfscope}%
\pgfpathrectangle{\pgfqpoint{1.250000in}{0.550000in}}{\pgfqpoint{7.750000in}{3.850000in}}%
\pgfusepath{clip}%
\pgfsetbuttcap%
\pgfsetmiterjoin%
\definecolor{currentfill}{rgb}{0.501961,0.501961,0.501961}%
\pgfsetfillcolor{currentfill}%
\pgfsetfillopacity{0.700000}%
\pgfsetlinewidth{0.000000pt}%
\definecolor{currentstroke}{rgb}{0.000000,0.000000,0.000000}%
\pgfsetstrokecolor{currentstroke}%
\pgfsetstrokeopacity{0.700000}%
\pgfsetdash{}{0pt}%
\pgfpathmoveto{\pgfqpoint{6.831321in}{0.550000in}}%
\pgfpathlineto{\pgfqpoint{6.858842in}{0.550000in}}%
\pgfpathlineto{\pgfqpoint{6.858842in}{1.001918in}}%
\pgfpathlineto{\pgfqpoint{6.831321in}{1.001918in}}%
\pgfpathlineto{\pgfqpoint{6.831321in}{0.550000in}}%
\pgfpathclose%
\pgfusepath{fill}%
\end{pgfscope}%
\begin{pgfscope}%
\pgfpathrectangle{\pgfqpoint{1.250000in}{0.550000in}}{\pgfqpoint{7.750000in}{3.850000in}}%
\pgfusepath{clip}%
\pgfsetbuttcap%
\pgfsetmiterjoin%
\definecolor{currentfill}{rgb}{0.501961,0.501961,0.501961}%
\pgfsetfillcolor{currentfill}%
\pgfsetfillopacity{0.700000}%
\pgfsetlinewidth{0.000000pt}%
\definecolor{currentstroke}{rgb}{0.000000,0.000000,0.000000}%
\pgfsetstrokecolor{currentstroke}%
\pgfsetstrokeopacity{0.700000}%
\pgfsetdash{}{0pt}%
\pgfpathmoveto{\pgfqpoint{6.858842in}{0.550000in}}%
\pgfpathlineto{\pgfqpoint{6.886364in}{0.550000in}}%
\pgfpathlineto{\pgfqpoint{6.886364in}{0.550000in}}%
\pgfpathlineto{\pgfqpoint{6.858842in}{0.550000in}}%
\pgfpathlineto{\pgfqpoint{6.858842in}{0.550000in}}%
\pgfpathclose%
\pgfusepath{fill}%
\end{pgfscope}%
\begin{pgfscope}%
\pgfpathrectangle{\pgfqpoint{1.250000in}{0.550000in}}{\pgfqpoint{7.750000in}{3.850000in}}%
\pgfusepath{clip}%
\pgfsetbuttcap%
\pgfsetmiterjoin%
\definecolor{currentfill}{rgb}{0.501961,0.501961,0.501961}%
\pgfsetfillcolor{currentfill}%
\pgfsetfillopacity{0.700000}%
\pgfsetlinewidth{0.000000pt}%
\definecolor{currentstroke}{rgb}{0.000000,0.000000,0.000000}%
\pgfsetstrokecolor{currentstroke}%
\pgfsetstrokeopacity{0.700000}%
\pgfsetdash{}{0pt}%
\pgfpathmoveto{\pgfqpoint{6.886364in}{0.550000in}}%
\pgfpathlineto{\pgfqpoint{6.913885in}{0.550000in}}%
\pgfpathlineto{\pgfqpoint{6.913885in}{1.009046in}}%
\pgfpathlineto{\pgfqpoint{6.886364in}{1.009046in}}%
\pgfpathlineto{\pgfqpoint{6.886364in}{0.550000in}}%
\pgfpathclose%
\pgfusepath{fill}%
\end{pgfscope}%
\begin{pgfscope}%
\pgfpathrectangle{\pgfqpoint{1.250000in}{0.550000in}}{\pgfqpoint{7.750000in}{3.850000in}}%
\pgfusepath{clip}%
\pgfsetbuttcap%
\pgfsetmiterjoin%
\definecolor{currentfill}{rgb}{0.501961,0.501961,0.501961}%
\pgfsetfillcolor{currentfill}%
\pgfsetfillopacity{0.700000}%
\pgfsetlinewidth{0.000000pt}%
\definecolor{currentstroke}{rgb}{0.000000,0.000000,0.000000}%
\pgfsetstrokecolor{currentstroke}%
\pgfsetstrokeopacity{0.700000}%
\pgfsetdash{}{0pt}%
\pgfpathmoveto{\pgfqpoint{6.913885in}{0.550000in}}%
\pgfpathlineto{\pgfqpoint{6.941406in}{0.550000in}}%
\pgfpathlineto{\pgfqpoint{6.941406in}{0.550000in}}%
\pgfpathlineto{\pgfqpoint{6.913885in}{0.550000in}}%
\pgfpathlineto{\pgfqpoint{6.913885in}{0.550000in}}%
\pgfpathclose%
\pgfusepath{fill}%
\end{pgfscope}%
\begin{pgfscope}%
\pgfpathrectangle{\pgfqpoint{1.250000in}{0.550000in}}{\pgfqpoint{7.750000in}{3.850000in}}%
\pgfusepath{clip}%
\pgfsetbuttcap%
\pgfsetmiterjoin%
\definecolor{currentfill}{rgb}{0.501961,0.501961,0.501961}%
\pgfsetfillcolor{currentfill}%
\pgfsetfillopacity{0.700000}%
\pgfsetlinewidth{0.000000pt}%
\definecolor{currentstroke}{rgb}{0.000000,0.000000,0.000000}%
\pgfsetstrokecolor{currentstroke}%
\pgfsetstrokeopacity{0.700000}%
\pgfsetdash{}{0pt}%
\pgfpathmoveto{\pgfqpoint{6.941406in}{0.550000in}}%
\pgfpathlineto{\pgfqpoint{6.968928in}{0.550000in}}%
\pgfpathlineto{\pgfqpoint{6.968928in}{0.990513in}}%
\pgfpathlineto{\pgfqpoint{6.941406in}{0.990513in}}%
\pgfpathlineto{\pgfqpoint{6.941406in}{0.550000in}}%
\pgfpathclose%
\pgfusepath{fill}%
\end{pgfscope}%
\begin{pgfscope}%
\pgfpathrectangle{\pgfqpoint{1.250000in}{0.550000in}}{\pgfqpoint{7.750000in}{3.850000in}}%
\pgfusepath{clip}%
\pgfsetbuttcap%
\pgfsetmiterjoin%
\definecolor{currentfill}{rgb}{0.501961,0.501961,0.501961}%
\pgfsetfillcolor{currentfill}%
\pgfsetfillopacity{0.700000}%
\pgfsetlinewidth{0.000000pt}%
\definecolor{currentstroke}{rgb}{0.000000,0.000000,0.000000}%
\pgfsetstrokecolor{currentstroke}%
\pgfsetstrokeopacity{0.700000}%
\pgfsetdash{}{0pt}%
\pgfpathmoveto{\pgfqpoint{6.968928in}{0.550000in}}%
\pgfpathlineto{\pgfqpoint{6.996449in}{0.550000in}}%
\pgfpathlineto{\pgfqpoint{6.996449in}{0.550000in}}%
\pgfpathlineto{\pgfqpoint{6.968928in}{0.550000in}}%
\pgfpathlineto{\pgfqpoint{6.968928in}{0.550000in}}%
\pgfpathclose%
\pgfusepath{fill}%
\end{pgfscope}%
\begin{pgfscope}%
\pgfpathrectangle{\pgfqpoint{1.250000in}{0.550000in}}{\pgfqpoint{7.750000in}{3.850000in}}%
\pgfusepath{clip}%
\pgfsetbuttcap%
\pgfsetmiterjoin%
\definecolor{currentfill}{rgb}{0.501961,0.501961,0.501961}%
\pgfsetfillcolor{currentfill}%
\pgfsetfillopacity{0.700000}%
\pgfsetlinewidth{0.000000pt}%
\definecolor{currentstroke}{rgb}{0.000000,0.000000,0.000000}%
\pgfsetstrokecolor{currentstroke}%
\pgfsetstrokeopacity{0.700000}%
\pgfsetdash{}{0pt}%
\pgfpathmoveto{\pgfqpoint{6.996449in}{0.550000in}}%
\pgfpathlineto{\pgfqpoint{7.023970in}{0.550000in}}%
\pgfpathlineto{\pgfqpoint{7.023970in}{0.943468in}}%
\pgfpathlineto{\pgfqpoint{6.996449in}{0.943468in}}%
\pgfpathlineto{\pgfqpoint{6.996449in}{0.550000in}}%
\pgfpathclose%
\pgfusepath{fill}%
\end{pgfscope}%
\begin{pgfscope}%
\pgfpathrectangle{\pgfqpoint{1.250000in}{0.550000in}}{\pgfqpoint{7.750000in}{3.850000in}}%
\pgfusepath{clip}%
\pgfsetbuttcap%
\pgfsetmiterjoin%
\definecolor{currentfill}{rgb}{0.501961,0.501961,0.501961}%
\pgfsetfillcolor{currentfill}%
\pgfsetfillopacity{0.700000}%
\pgfsetlinewidth{0.000000pt}%
\definecolor{currentstroke}{rgb}{0.000000,0.000000,0.000000}%
\pgfsetstrokecolor{currentstroke}%
\pgfsetstrokeopacity{0.700000}%
\pgfsetdash{}{0pt}%
\pgfpathmoveto{\pgfqpoint{7.023970in}{0.550000in}}%
\pgfpathlineto{\pgfqpoint{7.051491in}{0.550000in}}%
\pgfpathlineto{\pgfqpoint{7.051491in}{0.550000in}}%
\pgfpathlineto{\pgfqpoint{7.023970in}{0.550000in}}%
\pgfpathlineto{\pgfqpoint{7.023970in}{0.550000in}}%
\pgfpathclose%
\pgfusepath{fill}%
\end{pgfscope}%
\begin{pgfscope}%
\pgfpathrectangle{\pgfqpoint{1.250000in}{0.550000in}}{\pgfqpoint{7.750000in}{3.850000in}}%
\pgfusepath{clip}%
\pgfsetbuttcap%
\pgfsetmiterjoin%
\definecolor{currentfill}{rgb}{0.501961,0.501961,0.501961}%
\pgfsetfillcolor{currentfill}%
\pgfsetfillopacity{0.700000}%
\pgfsetlinewidth{0.000000pt}%
\definecolor{currentstroke}{rgb}{0.000000,0.000000,0.000000}%
\pgfsetstrokecolor{currentstroke}%
\pgfsetstrokeopacity{0.700000}%
\pgfsetdash{}{0pt}%
\pgfpathmoveto{\pgfqpoint{7.051491in}{0.550000in}}%
\pgfpathlineto{\pgfqpoint{7.079013in}{0.550000in}}%
\pgfpathlineto{\pgfqpoint{7.079013in}{0.974832in}}%
\pgfpathlineto{\pgfqpoint{7.051491in}{0.974832in}}%
\pgfpathlineto{\pgfqpoint{7.051491in}{0.550000in}}%
\pgfpathclose%
\pgfusepath{fill}%
\end{pgfscope}%
\begin{pgfscope}%
\pgfpathrectangle{\pgfqpoint{1.250000in}{0.550000in}}{\pgfqpoint{7.750000in}{3.850000in}}%
\pgfusepath{clip}%
\pgfsetbuttcap%
\pgfsetmiterjoin%
\definecolor{currentfill}{rgb}{0.501961,0.501961,0.501961}%
\pgfsetfillcolor{currentfill}%
\pgfsetfillopacity{0.700000}%
\pgfsetlinewidth{0.000000pt}%
\definecolor{currentstroke}{rgb}{0.000000,0.000000,0.000000}%
\pgfsetstrokecolor{currentstroke}%
\pgfsetstrokeopacity{0.700000}%
\pgfsetdash{}{0pt}%
\pgfpathmoveto{\pgfqpoint{7.079013in}{0.550000in}}%
\pgfpathlineto{\pgfqpoint{7.106534in}{0.550000in}}%
\pgfpathlineto{\pgfqpoint{7.106534in}{0.550000in}}%
\pgfpathlineto{\pgfqpoint{7.079013in}{0.550000in}}%
\pgfpathlineto{\pgfqpoint{7.079013in}{0.550000in}}%
\pgfpathclose%
\pgfusepath{fill}%
\end{pgfscope}%
\begin{pgfscope}%
\pgfpathrectangle{\pgfqpoint{1.250000in}{0.550000in}}{\pgfqpoint{7.750000in}{3.850000in}}%
\pgfusepath{clip}%
\pgfsetbuttcap%
\pgfsetmiterjoin%
\definecolor{currentfill}{rgb}{0.501961,0.501961,0.501961}%
\pgfsetfillcolor{currentfill}%
\pgfsetfillopacity{0.700000}%
\pgfsetlinewidth{0.000000pt}%
\definecolor{currentstroke}{rgb}{0.000000,0.000000,0.000000}%
\pgfsetstrokecolor{currentstroke}%
\pgfsetstrokeopacity{0.700000}%
\pgfsetdash{}{0pt}%
\pgfpathmoveto{\pgfqpoint{7.106534in}{0.550000in}}%
\pgfpathlineto{\pgfqpoint{7.134055in}{0.550000in}}%
\pgfpathlineto{\pgfqpoint{7.134055in}{0.953447in}}%
\pgfpathlineto{\pgfqpoint{7.106534in}{0.953447in}}%
\pgfpathlineto{\pgfqpoint{7.106534in}{0.550000in}}%
\pgfpathclose%
\pgfusepath{fill}%
\end{pgfscope}%
\begin{pgfscope}%
\pgfpathrectangle{\pgfqpoint{1.250000in}{0.550000in}}{\pgfqpoint{7.750000in}{3.850000in}}%
\pgfusepath{clip}%
\pgfsetbuttcap%
\pgfsetmiterjoin%
\definecolor{currentfill}{rgb}{0.501961,0.501961,0.501961}%
\pgfsetfillcolor{currentfill}%
\pgfsetfillopacity{0.700000}%
\pgfsetlinewidth{0.000000pt}%
\definecolor{currentstroke}{rgb}{0.000000,0.000000,0.000000}%
\pgfsetstrokecolor{currentstroke}%
\pgfsetstrokeopacity{0.700000}%
\pgfsetdash{}{0pt}%
\pgfpathmoveto{\pgfqpoint{7.134055in}{0.550000in}}%
\pgfpathlineto{\pgfqpoint{7.161577in}{0.550000in}}%
\pgfpathlineto{\pgfqpoint{7.161577in}{0.550000in}}%
\pgfpathlineto{\pgfqpoint{7.134055in}{0.550000in}}%
\pgfpathlineto{\pgfqpoint{7.134055in}{0.550000in}}%
\pgfpathclose%
\pgfusepath{fill}%
\end{pgfscope}%
\begin{pgfscope}%
\pgfpathrectangle{\pgfqpoint{1.250000in}{0.550000in}}{\pgfqpoint{7.750000in}{3.850000in}}%
\pgfusepath{clip}%
\pgfsetbuttcap%
\pgfsetmiterjoin%
\definecolor{currentfill}{rgb}{0.501961,0.501961,0.501961}%
\pgfsetfillcolor{currentfill}%
\pgfsetfillopacity{0.700000}%
\pgfsetlinewidth{0.000000pt}%
\definecolor{currentstroke}{rgb}{0.000000,0.000000,0.000000}%
\pgfsetstrokecolor{currentstroke}%
\pgfsetstrokeopacity{0.700000}%
\pgfsetdash{}{0pt}%
\pgfpathmoveto{\pgfqpoint{7.161577in}{0.550000in}}%
\pgfpathlineto{\pgfqpoint{7.189098in}{0.550000in}}%
\pgfpathlineto{\pgfqpoint{7.189098in}{0.949171in}}%
\pgfpathlineto{\pgfqpoint{7.161577in}{0.949171in}}%
\pgfpathlineto{\pgfqpoint{7.161577in}{0.550000in}}%
\pgfpathclose%
\pgfusepath{fill}%
\end{pgfscope}%
\begin{pgfscope}%
\pgfpathrectangle{\pgfqpoint{1.250000in}{0.550000in}}{\pgfqpoint{7.750000in}{3.850000in}}%
\pgfusepath{clip}%
\pgfsetbuttcap%
\pgfsetmiterjoin%
\definecolor{currentfill}{rgb}{0.501961,0.501961,0.501961}%
\pgfsetfillcolor{currentfill}%
\pgfsetfillopacity{0.700000}%
\pgfsetlinewidth{0.000000pt}%
\definecolor{currentstroke}{rgb}{0.000000,0.000000,0.000000}%
\pgfsetstrokecolor{currentstroke}%
\pgfsetstrokeopacity{0.700000}%
\pgfsetdash{}{0pt}%
\pgfpathmoveto{\pgfqpoint{7.189098in}{0.550000in}}%
\pgfpathlineto{\pgfqpoint{7.216619in}{0.550000in}}%
\pgfpathlineto{\pgfqpoint{7.216619in}{0.550000in}}%
\pgfpathlineto{\pgfqpoint{7.189098in}{0.550000in}}%
\pgfpathlineto{\pgfqpoint{7.189098in}{0.550000in}}%
\pgfpathclose%
\pgfusepath{fill}%
\end{pgfscope}%
\begin{pgfscope}%
\pgfpathrectangle{\pgfqpoint{1.250000in}{0.550000in}}{\pgfqpoint{7.750000in}{3.850000in}}%
\pgfusepath{clip}%
\pgfsetbuttcap%
\pgfsetmiterjoin%
\definecolor{currentfill}{rgb}{0.501961,0.501961,0.501961}%
\pgfsetfillcolor{currentfill}%
\pgfsetfillopacity{0.700000}%
\pgfsetlinewidth{0.000000pt}%
\definecolor{currentstroke}{rgb}{0.000000,0.000000,0.000000}%
\pgfsetstrokecolor{currentstroke}%
\pgfsetstrokeopacity{0.700000}%
\pgfsetdash{}{0pt}%
\pgfpathmoveto{\pgfqpoint{7.216619in}{0.550000in}}%
\pgfpathlineto{\pgfqpoint{7.244141in}{0.550000in}}%
\pgfpathlineto{\pgfqpoint{7.244141in}{0.934914in}}%
\pgfpathlineto{\pgfqpoint{7.216619in}{0.934914in}}%
\pgfpathlineto{\pgfqpoint{7.216619in}{0.550000in}}%
\pgfpathclose%
\pgfusepath{fill}%
\end{pgfscope}%
\begin{pgfscope}%
\pgfpathrectangle{\pgfqpoint{1.250000in}{0.550000in}}{\pgfqpoint{7.750000in}{3.850000in}}%
\pgfusepath{clip}%
\pgfsetbuttcap%
\pgfsetmiterjoin%
\definecolor{currentfill}{rgb}{0.501961,0.501961,0.501961}%
\pgfsetfillcolor{currentfill}%
\pgfsetfillopacity{0.700000}%
\pgfsetlinewidth{0.000000pt}%
\definecolor{currentstroke}{rgb}{0.000000,0.000000,0.000000}%
\pgfsetstrokecolor{currentstroke}%
\pgfsetstrokeopacity{0.700000}%
\pgfsetdash{}{0pt}%
\pgfpathmoveto{\pgfqpoint{7.244141in}{0.550000in}}%
\pgfpathlineto{\pgfqpoint{7.271662in}{0.550000in}}%
\pgfpathlineto{\pgfqpoint{7.271662in}{0.550000in}}%
\pgfpathlineto{\pgfqpoint{7.244141in}{0.550000in}}%
\pgfpathlineto{\pgfqpoint{7.244141in}{0.550000in}}%
\pgfpathclose%
\pgfusepath{fill}%
\end{pgfscope}%
\begin{pgfscope}%
\pgfpathrectangle{\pgfqpoint{1.250000in}{0.550000in}}{\pgfqpoint{7.750000in}{3.850000in}}%
\pgfusepath{clip}%
\pgfsetbuttcap%
\pgfsetmiterjoin%
\definecolor{currentfill}{rgb}{0.501961,0.501961,0.501961}%
\pgfsetfillcolor{currentfill}%
\pgfsetfillopacity{0.700000}%
\pgfsetlinewidth{0.000000pt}%
\definecolor{currentstroke}{rgb}{0.000000,0.000000,0.000000}%
\pgfsetstrokecolor{currentstroke}%
\pgfsetstrokeopacity{0.700000}%
\pgfsetdash{}{0pt}%
\pgfpathmoveto{\pgfqpoint{7.271662in}{0.550000in}}%
\pgfpathlineto{\pgfqpoint{7.299183in}{0.550000in}}%
\pgfpathlineto{\pgfqpoint{7.299183in}{0.855080in}}%
\pgfpathlineto{\pgfqpoint{7.271662in}{0.855080in}}%
\pgfpathlineto{\pgfqpoint{7.271662in}{0.550000in}}%
\pgfpathclose%
\pgfusepath{fill}%
\end{pgfscope}%
\begin{pgfscope}%
\pgfpathrectangle{\pgfqpoint{1.250000in}{0.550000in}}{\pgfqpoint{7.750000in}{3.850000in}}%
\pgfusepath{clip}%
\pgfsetbuttcap%
\pgfsetmiterjoin%
\definecolor{currentfill}{rgb}{0.501961,0.501961,0.501961}%
\pgfsetfillcolor{currentfill}%
\pgfsetfillopacity{0.700000}%
\pgfsetlinewidth{0.000000pt}%
\definecolor{currentstroke}{rgb}{0.000000,0.000000,0.000000}%
\pgfsetstrokecolor{currentstroke}%
\pgfsetstrokeopacity{0.700000}%
\pgfsetdash{}{0pt}%
\pgfpathmoveto{\pgfqpoint{7.299183in}{0.550000in}}%
\pgfpathlineto{\pgfqpoint{7.326705in}{0.550000in}}%
\pgfpathlineto{\pgfqpoint{7.326705in}{0.550000in}}%
\pgfpathlineto{\pgfqpoint{7.299183in}{0.550000in}}%
\pgfpathlineto{\pgfqpoint{7.299183in}{0.550000in}}%
\pgfpathclose%
\pgfusepath{fill}%
\end{pgfscope}%
\begin{pgfscope}%
\pgfpathrectangle{\pgfqpoint{1.250000in}{0.550000in}}{\pgfqpoint{7.750000in}{3.850000in}}%
\pgfusepath{clip}%
\pgfsetbuttcap%
\pgfsetmiterjoin%
\definecolor{currentfill}{rgb}{0.501961,0.501961,0.501961}%
\pgfsetfillcolor{currentfill}%
\pgfsetfillopacity{0.700000}%
\pgfsetlinewidth{0.000000pt}%
\definecolor{currentstroke}{rgb}{0.000000,0.000000,0.000000}%
\pgfsetstrokecolor{currentstroke}%
\pgfsetstrokeopacity{0.700000}%
\pgfsetdash{}{0pt}%
\pgfpathmoveto{\pgfqpoint{7.326705in}{0.550000in}}%
\pgfpathlineto{\pgfqpoint{7.354226in}{0.550000in}}%
\pgfpathlineto{\pgfqpoint{7.354226in}{0.904977in}}%
\pgfpathlineto{\pgfqpoint{7.326705in}{0.904977in}}%
\pgfpathlineto{\pgfqpoint{7.326705in}{0.550000in}}%
\pgfpathclose%
\pgfusepath{fill}%
\end{pgfscope}%
\begin{pgfscope}%
\pgfpathrectangle{\pgfqpoint{1.250000in}{0.550000in}}{\pgfqpoint{7.750000in}{3.850000in}}%
\pgfusepath{clip}%
\pgfsetbuttcap%
\pgfsetmiterjoin%
\definecolor{currentfill}{rgb}{0.501961,0.501961,0.501961}%
\pgfsetfillcolor{currentfill}%
\pgfsetfillopacity{0.700000}%
\pgfsetlinewidth{0.000000pt}%
\definecolor{currentstroke}{rgb}{0.000000,0.000000,0.000000}%
\pgfsetstrokecolor{currentstroke}%
\pgfsetstrokeopacity{0.700000}%
\pgfsetdash{}{0pt}%
\pgfpathmoveto{\pgfqpoint{7.354226in}{0.550000in}}%
\pgfpathlineto{\pgfqpoint{7.381747in}{0.550000in}}%
\pgfpathlineto{\pgfqpoint{7.381747in}{0.880741in}}%
\pgfpathlineto{\pgfqpoint{7.354226in}{0.880741in}}%
\pgfpathlineto{\pgfqpoint{7.354226in}{0.550000in}}%
\pgfpathclose%
\pgfusepath{fill}%
\end{pgfscope}%
\begin{pgfscope}%
\pgfpathrectangle{\pgfqpoint{1.250000in}{0.550000in}}{\pgfqpoint{7.750000in}{3.850000in}}%
\pgfusepath{clip}%
\pgfsetbuttcap%
\pgfsetmiterjoin%
\definecolor{currentfill}{rgb}{0.501961,0.501961,0.501961}%
\pgfsetfillcolor{currentfill}%
\pgfsetfillopacity{0.700000}%
\pgfsetlinewidth{0.000000pt}%
\definecolor{currentstroke}{rgb}{0.000000,0.000000,0.000000}%
\pgfsetstrokecolor{currentstroke}%
\pgfsetstrokeopacity{0.700000}%
\pgfsetdash{}{0pt}%
\pgfpathmoveto{\pgfqpoint{7.381747in}{0.550000in}}%
\pgfpathlineto{\pgfqpoint{7.409268in}{0.550000in}}%
\pgfpathlineto{\pgfqpoint{7.409268in}{0.550000in}}%
\pgfpathlineto{\pgfqpoint{7.381747in}{0.550000in}}%
\pgfpathlineto{\pgfqpoint{7.381747in}{0.550000in}}%
\pgfpathclose%
\pgfusepath{fill}%
\end{pgfscope}%
\begin{pgfscope}%
\pgfpathrectangle{\pgfqpoint{1.250000in}{0.550000in}}{\pgfqpoint{7.750000in}{3.850000in}}%
\pgfusepath{clip}%
\pgfsetbuttcap%
\pgfsetmiterjoin%
\definecolor{currentfill}{rgb}{0.501961,0.501961,0.501961}%
\pgfsetfillcolor{currentfill}%
\pgfsetfillopacity{0.700000}%
\pgfsetlinewidth{0.000000pt}%
\definecolor{currentstroke}{rgb}{0.000000,0.000000,0.000000}%
\pgfsetstrokecolor{currentstroke}%
\pgfsetstrokeopacity{0.700000}%
\pgfsetdash{}{0pt}%
\pgfpathmoveto{\pgfqpoint{7.409268in}{0.550000in}}%
\pgfpathlineto{\pgfqpoint{7.436790in}{0.550000in}}%
\pgfpathlineto{\pgfqpoint{7.436790in}{0.875039in}}%
\pgfpathlineto{\pgfqpoint{7.409268in}{0.875039in}}%
\pgfpathlineto{\pgfqpoint{7.409268in}{0.550000in}}%
\pgfpathclose%
\pgfusepath{fill}%
\end{pgfscope}%
\begin{pgfscope}%
\pgfpathrectangle{\pgfqpoint{1.250000in}{0.550000in}}{\pgfqpoint{7.750000in}{3.850000in}}%
\pgfusepath{clip}%
\pgfsetbuttcap%
\pgfsetmiterjoin%
\definecolor{currentfill}{rgb}{0.501961,0.501961,0.501961}%
\pgfsetfillcolor{currentfill}%
\pgfsetfillopacity{0.700000}%
\pgfsetlinewidth{0.000000pt}%
\definecolor{currentstroke}{rgb}{0.000000,0.000000,0.000000}%
\pgfsetstrokecolor{currentstroke}%
\pgfsetstrokeopacity{0.700000}%
\pgfsetdash{}{0pt}%
\pgfpathmoveto{\pgfqpoint{7.436790in}{0.550000in}}%
\pgfpathlineto{\pgfqpoint{7.464311in}{0.550000in}}%
\pgfpathlineto{\pgfqpoint{7.464311in}{0.550000in}}%
\pgfpathlineto{\pgfqpoint{7.436790in}{0.550000in}}%
\pgfpathlineto{\pgfqpoint{7.436790in}{0.550000in}}%
\pgfpathclose%
\pgfusepath{fill}%
\end{pgfscope}%
\begin{pgfscope}%
\pgfpathrectangle{\pgfqpoint{1.250000in}{0.550000in}}{\pgfqpoint{7.750000in}{3.850000in}}%
\pgfusepath{clip}%
\pgfsetbuttcap%
\pgfsetmiterjoin%
\definecolor{currentfill}{rgb}{0.501961,0.501961,0.501961}%
\pgfsetfillcolor{currentfill}%
\pgfsetfillopacity{0.700000}%
\pgfsetlinewidth{0.000000pt}%
\definecolor{currentstroke}{rgb}{0.000000,0.000000,0.000000}%
\pgfsetstrokecolor{currentstroke}%
\pgfsetstrokeopacity{0.700000}%
\pgfsetdash{}{0pt}%
\pgfpathmoveto{\pgfqpoint{7.464311in}{0.550000in}}%
\pgfpathlineto{\pgfqpoint{7.491832in}{0.550000in}}%
\pgfpathlineto{\pgfqpoint{7.491832in}{0.929212in}}%
\pgfpathlineto{\pgfqpoint{7.464311in}{0.929212in}}%
\pgfpathlineto{\pgfqpoint{7.464311in}{0.550000in}}%
\pgfpathclose%
\pgfusepath{fill}%
\end{pgfscope}%
\begin{pgfscope}%
\pgfpathrectangle{\pgfqpoint{1.250000in}{0.550000in}}{\pgfqpoint{7.750000in}{3.850000in}}%
\pgfusepath{clip}%
\pgfsetbuttcap%
\pgfsetmiterjoin%
\definecolor{currentfill}{rgb}{0.501961,0.501961,0.501961}%
\pgfsetfillcolor{currentfill}%
\pgfsetfillopacity{0.700000}%
\pgfsetlinewidth{0.000000pt}%
\definecolor{currentstroke}{rgb}{0.000000,0.000000,0.000000}%
\pgfsetstrokecolor{currentstroke}%
\pgfsetstrokeopacity{0.700000}%
\pgfsetdash{}{0pt}%
\pgfpathmoveto{\pgfqpoint{7.491832in}{0.550000in}}%
\pgfpathlineto{\pgfqpoint{7.519354in}{0.550000in}}%
\pgfpathlineto{\pgfqpoint{7.519354in}{0.550000in}}%
\pgfpathlineto{\pgfqpoint{7.491832in}{0.550000in}}%
\pgfpathlineto{\pgfqpoint{7.491832in}{0.550000in}}%
\pgfpathclose%
\pgfusepath{fill}%
\end{pgfscope}%
\begin{pgfscope}%
\pgfpathrectangle{\pgfqpoint{1.250000in}{0.550000in}}{\pgfqpoint{7.750000in}{3.850000in}}%
\pgfusepath{clip}%
\pgfsetbuttcap%
\pgfsetmiterjoin%
\definecolor{currentfill}{rgb}{0.501961,0.501961,0.501961}%
\pgfsetfillcolor{currentfill}%
\pgfsetfillopacity{0.700000}%
\pgfsetlinewidth{0.000000pt}%
\definecolor{currentstroke}{rgb}{0.000000,0.000000,0.000000}%
\pgfsetstrokecolor{currentstroke}%
\pgfsetstrokeopacity{0.700000}%
\pgfsetdash{}{0pt}%
\pgfpathmoveto{\pgfqpoint{7.519354in}{0.550000in}}%
\pgfpathlineto{\pgfqpoint{7.546875in}{0.550000in}}%
\pgfpathlineto{\pgfqpoint{7.546875in}{0.856506in}}%
\pgfpathlineto{\pgfqpoint{7.519354in}{0.856506in}}%
\pgfpathlineto{\pgfqpoint{7.519354in}{0.550000in}}%
\pgfpathclose%
\pgfusepath{fill}%
\end{pgfscope}%
\begin{pgfscope}%
\pgfpathrectangle{\pgfqpoint{1.250000in}{0.550000in}}{\pgfqpoint{7.750000in}{3.850000in}}%
\pgfusepath{clip}%
\pgfsetbuttcap%
\pgfsetmiterjoin%
\definecolor{currentfill}{rgb}{0.501961,0.501961,0.501961}%
\pgfsetfillcolor{currentfill}%
\pgfsetfillopacity{0.700000}%
\pgfsetlinewidth{0.000000pt}%
\definecolor{currentstroke}{rgb}{0.000000,0.000000,0.000000}%
\pgfsetstrokecolor{currentstroke}%
\pgfsetstrokeopacity{0.700000}%
\pgfsetdash{}{0pt}%
\pgfpathmoveto{\pgfqpoint{7.546875in}{0.550000in}}%
\pgfpathlineto{\pgfqpoint{7.574396in}{0.550000in}}%
\pgfpathlineto{\pgfqpoint{7.574396in}{0.805184in}}%
\pgfpathlineto{\pgfqpoint{7.546875in}{0.805184in}}%
\pgfpathlineto{\pgfqpoint{7.546875in}{0.550000in}}%
\pgfpathclose%
\pgfusepath{fill}%
\end{pgfscope}%
\begin{pgfscope}%
\pgfpathrectangle{\pgfqpoint{1.250000in}{0.550000in}}{\pgfqpoint{7.750000in}{3.850000in}}%
\pgfusepath{clip}%
\pgfsetbuttcap%
\pgfsetmiterjoin%
\definecolor{currentfill}{rgb}{0.501961,0.501961,0.501961}%
\pgfsetfillcolor{currentfill}%
\pgfsetfillopacity{0.700000}%
\pgfsetlinewidth{0.000000pt}%
\definecolor{currentstroke}{rgb}{0.000000,0.000000,0.000000}%
\pgfsetstrokecolor{currentstroke}%
\pgfsetstrokeopacity{0.700000}%
\pgfsetdash{}{0pt}%
\pgfpathmoveto{\pgfqpoint{7.574396in}{0.550000in}}%
\pgfpathlineto{\pgfqpoint{7.601918in}{0.550000in}}%
\pgfpathlineto{\pgfqpoint{7.601918in}{0.799482in}}%
\pgfpathlineto{\pgfqpoint{7.574396in}{0.799482in}}%
\pgfpathlineto{\pgfqpoint{7.574396in}{0.550000in}}%
\pgfpathclose%
\pgfusepath{fill}%
\end{pgfscope}%
\begin{pgfscope}%
\pgfpathrectangle{\pgfqpoint{1.250000in}{0.550000in}}{\pgfqpoint{7.750000in}{3.850000in}}%
\pgfusepath{clip}%
\pgfsetbuttcap%
\pgfsetmiterjoin%
\definecolor{currentfill}{rgb}{0.501961,0.501961,0.501961}%
\pgfsetfillcolor{currentfill}%
\pgfsetfillopacity{0.700000}%
\pgfsetlinewidth{0.000000pt}%
\definecolor{currentstroke}{rgb}{0.000000,0.000000,0.000000}%
\pgfsetstrokecolor{currentstroke}%
\pgfsetstrokeopacity{0.700000}%
\pgfsetdash{}{0pt}%
\pgfpathmoveto{\pgfqpoint{7.601918in}{0.550000in}}%
\pgfpathlineto{\pgfqpoint{7.629439in}{0.550000in}}%
\pgfpathlineto{\pgfqpoint{7.629439in}{0.550000in}}%
\pgfpathlineto{\pgfqpoint{7.601918in}{0.550000in}}%
\pgfpathlineto{\pgfqpoint{7.601918in}{0.550000in}}%
\pgfpathclose%
\pgfusepath{fill}%
\end{pgfscope}%
\begin{pgfscope}%
\pgfpathrectangle{\pgfqpoint{1.250000in}{0.550000in}}{\pgfqpoint{7.750000in}{3.850000in}}%
\pgfusepath{clip}%
\pgfsetbuttcap%
\pgfsetmiterjoin%
\definecolor{currentfill}{rgb}{0.501961,0.501961,0.501961}%
\pgfsetfillcolor{currentfill}%
\pgfsetfillopacity{0.700000}%
\pgfsetlinewidth{0.000000pt}%
\definecolor{currentstroke}{rgb}{0.000000,0.000000,0.000000}%
\pgfsetstrokecolor{currentstroke}%
\pgfsetstrokeopacity{0.700000}%
\pgfsetdash{}{0pt}%
\pgfpathmoveto{\pgfqpoint{7.629439in}{0.550000in}}%
\pgfpathlineto{\pgfqpoint{7.656960in}{0.550000in}}%
\pgfpathlineto{\pgfqpoint{7.656960in}{0.809461in}}%
\pgfpathlineto{\pgfqpoint{7.629439in}{0.809461in}}%
\pgfpathlineto{\pgfqpoint{7.629439in}{0.550000in}}%
\pgfpathclose%
\pgfusepath{fill}%
\end{pgfscope}%
\begin{pgfscope}%
\pgfpathrectangle{\pgfqpoint{1.250000in}{0.550000in}}{\pgfqpoint{7.750000in}{3.850000in}}%
\pgfusepath{clip}%
\pgfsetbuttcap%
\pgfsetmiterjoin%
\definecolor{currentfill}{rgb}{0.501961,0.501961,0.501961}%
\pgfsetfillcolor{currentfill}%
\pgfsetfillopacity{0.700000}%
\pgfsetlinewidth{0.000000pt}%
\definecolor{currentstroke}{rgb}{0.000000,0.000000,0.000000}%
\pgfsetstrokecolor{currentstroke}%
\pgfsetstrokeopacity{0.700000}%
\pgfsetdash{}{0pt}%
\pgfpathmoveto{\pgfqpoint{7.656960in}{0.550000in}}%
\pgfpathlineto{\pgfqpoint{7.684482in}{0.550000in}}%
\pgfpathlineto{\pgfqpoint{7.684482in}{0.805184in}}%
\pgfpathlineto{\pgfqpoint{7.656960in}{0.805184in}}%
\pgfpathlineto{\pgfqpoint{7.656960in}{0.550000in}}%
\pgfpathclose%
\pgfusepath{fill}%
\end{pgfscope}%
\begin{pgfscope}%
\pgfpathrectangle{\pgfqpoint{1.250000in}{0.550000in}}{\pgfqpoint{7.750000in}{3.850000in}}%
\pgfusepath{clip}%
\pgfsetbuttcap%
\pgfsetmiterjoin%
\definecolor{currentfill}{rgb}{0.501961,0.501961,0.501961}%
\pgfsetfillcolor{currentfill}%
\pgfsetfillopacity{0.700000}%
\pgfsetlinewidth{0.000000pt}%
\definecolor{currentstroke}{rgb}{0.000000,0.000000,0.000000}%
\pgfsetstrokecolor{currentstroke}%
\pgfsetstrokeopacity{0.700000}%
\pgfsetdash{}{0pt}%
\pgfpathmoveto{\pgfqpoint{7.684482in}{0.550000in}}%
\pgfpathlineto{\pgfqpoint{7.712003in}{0.550000in}}%
\pgfpathlineto{\pgfqpoint{7.712003in}{0.813738in}}%
\pgfpathlineto{\pgfqpoint{7.684482in}{0.813738in}}%
\pgfpathlineto{\pgfqpoint{7.684482in}{0.550000in}}%
\pgfpathclose%
\pgfusepath{fill}%
\end{pgfscope}%
\begin{pgfscope}%
\pgfpathrectangle{\pgfqpoint{1.250000in}{0.550000in}}{\pgfqpoint{7.750000in}{3.850000in}}%
\pgfusepath{clip}%
\pgfsetbuttcap%
\pgfsetmiterjoin%
\definecolor{currentfill}{rgb}{0.501961,0.501961,0.501961}%
\pgfsetfillcolor{currentfill}%
\pgfsetfillopacity{0.700000}%
\pgfsetlinewidth{0.000000pt}%
\definecolor{currentstroke}{rgb}{0.000000,0.000000,0.000000}%
\pgfsetstrokecolor{currentstroke}%
\pgfsetstrokeopacity{0.700000}%
\pgfsetdash{}{0pt}%
\pgfpathmoveto{\pgfqpoint{7.712003in}{0.550000in}}%
\pgfpathlineto{\pgfqpoint{7.739524in}{0.550000in}}%
\pgfpathlineto{\pgfqpoint{7.739524in}{0.550000in}}%
\pgfpathlineto{\pgfqpoint{7.712003in}{0.550000in}}%
\pgfpathlineto{\pgfqpoint{7.712003in}{0.550000in}}%
\pgfpathclose%
\pgfusepath{fill}%
\end{pgfscope}%
\begin{pgfscope}%
\pgfpathrectangle{\pgfqpoint{1.250000in}{0.550000in}}{\pgfqpoint{7.750000in}{3.850000in}}%
\pgfusepath{clip}%
\pgfsetbuttcap%
\pgfsetmiterjoin%
\definecolor{currentfill}{rgb}{0.501961,0.501961,0.501961}%
\pgfsetfillcolor{currentfill}%
\pgfsetfillopacity{0.700000}%
\pgfsetlinewidth{0.000000pt}%
\definecolor{currentstroke}{rgb}{0.000000,0.000000,0.000000}%
\pgfsetstrokecolor{currentstroke}%
\pgfsetstrokeopacity{0.700000}%
\pgfsetdash{}{0pt}%
\pgfpathmoveto{\pgfqpoint{7.739524in}{0.550000in}}%
\pgfpathlineto{\pgfqpoint{7.767045in}{0.550000in}}%
\pgfpathlineto{\pgfqpoint{7.767045in}{0.782374in}}%
\pgfpathlineto{\pgfqpoint{7.739524in}{0.782374in}}%
\pgfpathlineto{\pgfqpoint{7.739524in}{0.550000in}}%
\pgfpathclose%
\pgfusepath{fill}%
\end{pgfscope}%
\begin{pgfscope}%
\pgfpathrectangle{\pgfqpoint{1.250000in}{0.550000in}}{\pgfqpoint{7.750000in}{3.850000in}}%
\pgfusepath{clip}%
\pgfsetbuttcap%
\pgfsetmiterjoin%
\definecolor{currentfill}{rgb}{0.501961,0.501961,0.501961}%
\pgfsetfillcolor{currentfill}%
\pgfsetfillopacity{0.700000}%
\pgfsetlinewidth{0.000000pt}%
\definecolor{currentstroke}{rgb}{0.000000,0.000000,0.000000}%
\pgfsetstrokecolor{currentstroke}%
\pgfsetstrokeopacity{0.700000}%
\pgfsetdash{}{0pt}%
\pgfpathmoveto{\pgfqpoint{7.767045in}{0.550000in}}%
\pgfpathlineto{\pgfqpoint{7.794567in}{0.550000in}}%
\pgfpathlineto{\pgfqpoint{7.794567in}{0.768118in}}%
\pgfpathlineto{\pgfqpoint{7.767045in}{0.768118in}}%
\pgfpathlineto{\pgfqpoint{7.767045in}{0.550000in}}%
\pgfpathclose%
\pgfusepath{fill}%
\end{pgfscope}%
\begin{pgfscope}%
\pgfpathrectangle{\pgfqpoint{1.250000in}{0.550000in}}{\pgfqpoint{7.750000in}{3.850000in}}%
\pgfusepath{clip}%
\pgfsetbuttcap%
\pgfsetmiterjoin%
\definecolor{currentfill}{rgb}{0.501961,0.501961,0.501961}%
\pgfsetfillcolor{currentfill}%
\pgfsetfillopacity{0.700000}%
\pgfsetlinewidth{0.000000pt}%
\definecolor{currentstroke}{rgb}{0.000000,0.000000,0.000000}%
\pgfsetstrokecolor{currentstroke}%
\pgfsetstrokeopacity{0.700000}%
\pgfsetdash{}{0pt}%
\pgfpathmoveto{\pgfqpoint{7.794567in}{0.550000in}}%
\pgfpathlineto{\pgfqpoint{7.822088in}{0.550000in}}%
\pgfpathlineto{\pgfqpoint{7.822088in}{0.775246in}}%
\pgfpathlineto{\pgfqpoint{7.794567in}{0.775246in}}%
\pgfpathlineto{\pgfqpoint{7.794567in}{0.550000in}}%
\pgfpathclose%
\pgfusepath{fill}%
\end{pgfscope}%
\begin{pgfscope}%
\pgfpathrectangle{\pgfqpoint{1.250000in}{0.550000in}}{\pgfqpoint{7.750000in}{3.850000in}}%
\pgfusepath{clip}%
\pgfsetbuttcap%
\pgfsetmiterjoin%
\definecolor{currentfill}{rgb}{0.501961,0.501961,0.501961}%
\pgfsetfillcolor{currentfill}%
\pgfsetfillopacity{0.700000}%
\pgfsetlinewidth{0.000000pt}%
\definecolor{currentstroke}{rgb}{0.000000,0.000000,0.000000}%
\pgfsetstrokecolor{currentstroke}%
\pgfsetstrokeopacity{0.700000}%
\pgfsetdash{}{0pt}%
\pgfpathmoveto{\pgfqpoint{7.822088in}{0.550000in}}%
\pgfpathlineto{\pgfqpoint{7.849609in}{0.550000in}}%
\pgfpathlineto{\pgfqpoint{7.849609in}{0.773821in}}%
\pgfpathlineto{\pgfqpoint{7.822088in}{0.773821in}}%
\pgfpathlineto{\pgfqpoint{7.822088in}{0.550000in}}%
\pgfpathclose%
\pgfusepath{fill}%
\end{pgfscope}%
\begin{pgfscope}%
\pgfpathrectangle{\pgfqpoint{1.250000in}{0.550000in}}{\pgfqpoint{7.750000in}{3.850000in}}%
\pgfusepath{clip}%
\pgfsetbuttcap%
\pgfsetmiterjoin%
\definecolor{currentfill}{rgb}{0.501961,0.501961,0.501961}%
\pgfsetfillcolor{currentfill}%
\pgfsetfillopacity{0.700000}%
\pgfsetlinewidth{0.000000pt}%
\definecolor{currentstroke}{rgb}{0.000000,0.000000,0.000000}%
\pgfsetstrokecolor{currentstroke}%
\pgfsetstrokeopacity{0.700000}%
\pgfsetdash{}{0pt}%
\pgfpathmoveto{\pgfqpoint{7.849609in}{0.550000in}}%
\pgfpathlineto{\pgfqpoint{7.877131in}{0.550000in}}%
\pgfpathlineto{\pgfqpoint{7.877131in}{0.745308in}}%
\pgfpathlineto{\pgfqpoint{7.849609in}{0.745308in}}%
\pgfpathlineto{\pgfqpoint{7.849609in}{0.550000in}}%
\pgfpathclose%
\pgfusepath{fill}%
\end{pgfscope}%
\begin{pgfscope}%
\pgfpathrectangle{\pgfqpoint{1.250000in}{0.550000in}}{\pgfqpoint{7.750000in}{3.850000in}}%
\pgfusepath{clip}%
\pgfsetbuttcap%
\pgfsetmiterjoin%
\definecolor{currentfill}{rgb}{0.501961,0.501961,0.501961}%
\pgfsetfillcolor{currentfill}%
\pgfsetfillopacity{0.700000}%
\pgfsetlinewidth{0.000000pt}%
\definecolor{currentstroke}{rgb}{0.000000,0.000000,0.000000}%
\pgfsetstrokecolor{currentstroke}%
\pgfsetstrokeopacity{0.700000}%
\pgfsetdash{}{0pt}%
\pgfpathmoveto{\pgfqpoint{7.877131in}{0.550000in}}%
\pgfpathlineto{\pgfqpoint{7.904652in}{0.550000in}}%
\pgfpathlineto{\pgfqpoint{7.904652in}{0.749585in}}%
\pgfpathlineto{\pgfqpoint{7.877131in}{0.749585in}}%
\pgfpathlineto{\pgfqpoint{7.877131in}{0.550000in}}%
\pgfpathclose%
\pgfusepath{fill}%
\end{pgfscope}%
\begin{pgfscope}%
\pgfpathrectangle{\pgfqpoint{1.250000in}{0.550000in}}{\pgfqpoint{7.750000in}{3.850000in}}%
\pgfusepath{clip}%
\pgfsetbuttcap%
\pgfsetmiterjoin%
\definecolor{currentfill}{rgb}{0.501961,0.501961,0.501961}%
\pgfsetfillcolor{currentfill}%
\pgfsetfillopacity{0.700000}%
\pgfsetlinewidth{0.000000pt}%
\definecolor{currentstroke}{rgb}{0.000000,0.000000,0.000000}%
\pgfsetstrokecolor{currentstroke}%
\pgfsetstrokeopacity{0.700000}%
\pgfsetdash{}{0pt}%
\pgfpathmoveto{\pgfqpoint{7.904652in}{0.550000in}}%
\pgfpathlineto{\pgfqpoint{7.932173in}{0.550000in}}%
\pgfpathlineto{\pgfqpoint{7.932173in}{0.709668in}}%
\pgfpathlineto{\pgfqpoint{7.904652in}{0.709668in}}%
\pgfpathlineto{\pgfqpoint{7.904652in}{0.550000in}}%
\pgfpathclose%
\pgfusepath{fill}%
\end{pgfscope}%
\begin{pgfscope}%
\pgfpathrectangle{\pgfqpoint{1.250000in}{0.550000in}}{\pgfqpoint{7.750000in}{3.850000in}}%
\pgfusepath{clip}%
\pgfsetbuttcap%
\pgfsetmiterjoin%
\definecolor{currentfill}{rgb}{0.501961,0.501961,0.501961}%
\pgfsetfillcolor{currentfill}%
\pgfsetfillopacity{0.700000}%
\pgfsetlinewidth{0.000000pt}%
\definecolor{currentstroke}{rgb}{0.000000,0.000000,0.000000}%
\pgfsetstrokecolor{currentstroke}%
\pgfsetstrokeopacity{0.700000}%
\pgfsetdash{}{0pt}%
\pgfpathmoveto{\pgfqpoint{7.932173in}{0.550000in}}%
\pgfpathlineto{\pgfqpoint{7.959695in}{0.550000in}}%
\pgfpathlineto{\pgfqpoint{7.959695in}{0.729627in}}%
\pgfpathlineto{\pgfqpoint{7.932173in}{0.729627in}}%
\pgfpathlineto{\pgfqpoint{7.932173in}{0.550000in}}%
\pgfpathclose%
\pgfusepath{fill}%
\end{pgfscope}%
\begin{pgfscope}%
\pgfpathrectangle{\pgfqpoint{1.250000in}{0.550000in}}{\pgfqpoint{7.750000in}{3.850000in}}%
\pgfusepath{clip}%
\pgfsetbuttcap%
\pgfsetmiterjoin%
\definecolor{currentfill}{rgb}{0.501961,0.501961,0.501961}%
\pgfsetfillcolor{currentfill}%
\pgfsetfillopacity{0.700000}%
\pgfsetlinewidth{0.000000pt}%
\definecolor{currentstroke}{rgb}{0.000000,0.000000,0.000000}%
\pgfsetstrokecolor{currentstroke}%
\pgfsetstrokeopacity{0.700000}%
\pgfsetdash{}{0pt}%
\pgfpathmoveto{\pgfqpoint{7.959695in}{0.550000in}}%
\pgfpathlineto{\pgfqpoint{7.987216in}{0.550000in}}%
\pgfpathlineto{\pgfqpoint{7.987216in}{0.726776in}}%
\pgfpathlineto{\pgfqpoint{7.959695in}{0.726776in}}%
\pgfpathlineto{\pgfqpoint{7.959695in}{0.550000in}}%
\pgfpathclose%
\pgfusepath{fill}%
\end{pgfscope}%
\begin{pgfscope}%
\pgfpathrectangle{\pgfqpoint{1.250000in}{0.550000in}}{\pgfqpoint{7.750000in}{3.850000in}}%
\pgfusepath{clip}%
\pgfsetbuttcap%
\pgfsetmiterjoin%
\definecolor{currentfill}{rgb}{0.501961,0.501961,0.501961}%
\pgfsetfillcolor{currentfill}%
\pgfsetfillopacity{0.700000}%
\pgfsetlinewidth{0.000000pt}%
\definecolor{currentstroke}{rgb}{0.000000,0.000000,0.000000}%
\pgfsetstrokecolor{currentstroke}%
\pgfsetstrokeopacity{0.700000}%
\pgfsetdash{}{0pt}%
\pgfpathmoveto{\pgfqpoint{7.987216in}{0.550000in}}%
\pgfpathlineto{\pgfqpoint{8.014737in}{0.550000in}}%
\pgfpathlineto{\pgfqpoint{8.014737in}{0.709668in}}%
\pgfpathlineto{\pgfqpoint{7.987216in}{0.709668in}}%
\pgfpathlineto{\pgfqpoint{7.987216in}{0.550000in}}%
\pgfpathclose%
\pgfusepath{fill}%
\end{pgfscope}%
\begin{pgfscope}%
\pgfpathrectangle{\pgfqpoint{1.250000in}{0.550000in}}{\pgfqpoint{7.750000in}{3.850000in}}%
\pgfusepath{clip}%
\pgfsetbuttcap%
\pgfsetmiterjoin%
\definecolor{currentfill}{rgb}{0.501961,0.501961,0.501961}%
\pgfsetfillcolor{currentfill}%
\pgfsetfillopacity{0.700000}%
\pgfsetlinewidth{0.000000pt}%
\definecolor{currentstroke}{rgb}{0.000000,0.000000,0.000000}%
\pgfsetstrokecolor{currentstroke}%
\pgfsetstrokeopacity{0.700000}%
\pgfsetdash{}{0pt}%
\pgfpathmoveto{\pgfqpoint{8.014737in}{0.550000in}}%
\pgfpathlineto{\pgfqpoint{8.042259in}{0.550000in}}%
\pgfpathlineto{\pgfqpoint{8.042259in}{0.693987in}}%
\pgfpathlineto{\pgfqpoint{8.014737in}{0.693987in}}%
\pgfpathlineto{\pgfqpoint{8.014737in}{0.550000in}}%
\pgfpathclose%
\pgfusepath{fill}%
\end{pgfscope}%
\begin{pgfscope}%
\pgfpathrectangle{\pgfqpoint{1.250000in}{0.550000in}}{\pgfqpoint{7.750000in}{3.850000in}}%
\pgfusepath{clip}%
\pgfsetbuttcap%
\pgfsetmiterjoin%
\definecolor{currentfill}{rgb}{0.501961,0.501961,0.501961}%
\pgfsetfillcolor{currentfill}%
\pgfsetfillopacity{0.700000}%
\pgfsetlinewidth{0.000000pt}%
\definecolor{currentstroke}{rgb}{0.000000,0.000000,0.000000}%
\pgfsetstrokecolor{currentstroke}%
\pgfsetstrokeopacity{0.700000}%
\pgfsetdash{}{0pt}%
\pgfpathmoveto{\pgfqpoint{8.042259in}{0.550000in}}%
\pgfpathlineto{\pgfqpoint{8.069780in}{0.550000in}}%
\pgfpathlineto{\pgfqpoint{8.069780in}{0.856506in}}%
\pgfpathlineto{\pgfqpoint{8.042259in}{0.856506in}}%
\pgfpathlineto{\pgfqpoint{8.042259in}{0.550000in}}%
\pgfpathclose%
\pgfusepath{fill}%
\end{pgfscope}%
\begin{pgfscope}%
\pgfpathrectangle{\pgfqpoint{1.250000in}{0.550000in}}{\pgfqpoint{7.750000in}{3.850000in}}%
\pgfusepath{clip}%
\pgfsetbuttcap%
\pgfsetmiterjoin%
\definecolor{currentfill}{rgb}{0.501961,0.501961,0.501961}%
\pgfsetfillcolor{currentfill}%
\pgfsetfillopacity{0.700000}%
\pgfsetlinewidth{0.000000pt}%
\definecolor{currentstroke}{rgb}{0.000000,0.000000,0.000000}%
\pgfsetstrokecolor{currentstroke}%
\pgfsetstrokeopacity{0.700000}%
\pgfsetdash{}{0pt}%
\pgfpathmoveto{\pgfqpoint{8.069780in}{0.550000in}}%
\pgfpathlineto{\pgfqpoint{8.097301in}{0.550000in}}%
\pgfpathlineto{\pgfqpoint{8.097301in}{0.684007in}}%
\pgfpathlineto{\pgfqpoint{8.069780in}{0.684007in}}%
\pgfpathlineto{\pgfqpoint{8.069780in}{0.550000in}}%
\pgfpathclose%
\pgfusepath{fill}%
\end{pgfscope}%
\begin{pgfscope}%
\pgfpathrectangle{\pgfqpoint{1.250000in}{0.550000in}}{\pgfqpoint{7.750000in}{3.850000in}}%
\pgfusepath{clip}%
\pgfsetbuttcap%
\pgfsetmiterjoin%
\definecolor{currentfill}{rgb}{0.501961,0.501961,0.501961}%
\pgfsetfillcolor{currentfill}%
\pgfsetfillopacity{0.700000}%
\pgfsetlinewidth{0.000000pt}%
\definecolor{currentstroke}{rgb}{0.000000,0.000000,0.000000}%
\pgfsetstrokecolor{currentstroke}%
\pgfsetstrokeopacity{0.700000}%
\pgfsetdash{}{0pt}%
\pgfpathmoveto{\pgfqpoint{8.097301in}{0.550000in}}%
\pgfpathlineto{\pgfqpoint{8.124822in}{0.550000in}}%
\pgfpathlineto{\pgfqpoint{8.124822in}{0.800907in}}%
\pgfpathlineto{\pgfqpoint{8.097301in}{0.800907in}}%
\pgfpathlineto{\pgfqpoint{8.097301in}{0.550000in}}%
\pgfpathclose%
\pgfusepath{fill}%
\end{pgfscope}%
\begin{pgfscope}%
\pgfpathrectangle{\pgfqpoint{1.250000in}{0.550000in}}{\pgfqpoint{7.750000in}{3.850000in}}%
\pgfusepath{clip}%
\pgfsetbuttcap%
\pgfsetmiterjoin%
\definecolor{currentfill}{rgb}{0.501961,0.501961,0.501961}%
\pgfsetfillcolor{currentfill}%
\pgfsetfillopacity{0.700000}%
\pgfsetlinewidth{0.000000pt}%
\definecolor{currentstroke}{rgb}{0.000000,0.000000,0.000000}%
\pgfsetstrokecolor{currentstroke}%
\pgfsetstrokeopacity{0.700000}%
\pgfsetdash{}{0pt}%
\pgfpathmoveto{\pgfqpoint{8.124822in}{0.550000in}}%
\pgfpathlineto{\pgfqpoint{8.152344in}{0.550000in}}%
\pgfpathlineto{\pgfqpoint{8.152344in}{0.691135in}}%
\pgfpathlineto{\pgfqpoint{8.124822in}{0.691135in}}%
\pgfpathlineto{\pgfqpoint{8.124822in}{0.550000in}}%
\pgfpathclose%
\pgfusepath{fill}%
\end{pgfscope}%
\begin{pgfscope}%
\pgfpathrectangle{\pgfqpoint{1.250000in}{0.550000in}}{\pgfqpoint{7.750000in}{3.850000in}}%
\pgfusepath{clip}%
\pgfsetbuttcap%
\pgfsetmiterjoin%
\definecolor{currentfill}{rgb}{0.501961,0.501961,0.501961}%
\pgfsetfillcolor{currentfill}%
\pgfsetfillopacity{0.700000}%
\pgfsetlinewidth{0.000000pt}%
\definecolor{currentstroke}{rgb}{0.000000,0.000000,0.000000}%
\pgfsetstrokecolor{currentstroke}%
\pgfsetstrokeopacity{0.700000}%
\pgfsetdash{}{0pt}%
\pgfpathmoveto{\pgfqpoint{8.152344in}{0.550000in}}%
\pgfpathlineto{\pgfqpoint{8.179865in}{0.550000in}}%
\pgfpathlineto{\pgfqpoint{8.179865in}{0.776672in}}%
\pgfpathlineto{\pgfqpoint{8.152344in}{0.776672in}}%
\pgfpathlineto{\pgfqpoint{8.152344in}{0.550000in}}%
\pgfpathclose%
\pgfusepath{fill}%
\end{pgfscope}%
\begin{pgfscope}%
\pgfpathrectangle{\pgfqpoint{1.250000in}{0.550000in}}{\pgfqpoint{7.750000in}{3.850000in}}%
\pgfusepath{clip}%
\pgfsetbuttcap%
\pgfsetmiterjoin%
\definecolor{currentfill}{rgb}{0.501961,0.501961,0.501961}%
\pgfsetfillcolor{currentfill}%
\pgfsetfillopacity{0.700000}%
\pgfsetlinewidth{0.000000pt}%
\definecolor{currentstroke}{rgb}{0.000000,0.000000,0.000000}%
\pgfsetstrokecolor{currentstroke}%
\pgfsetstrokeopacity{0.700000}%
\pgfsetdash{}{0pt}%
\pgfpathmoveto{\pgfqpoint{8.179865in}{0.550000in}}%
\pgfpathlineto{\pgfqpoint{8.207386in}{0.550000in}}%
\pgfpathlineto{\pgfqpoint{8.207386in}{0.782374in}}%
\pgfpathlineto{\pgfqpoint{8.179865in}{0.782374in}}%
\pgfpathlineto{\pgfqpoint{8.179865in}{0.550000in}}%
\pgfpathclose%
\pgfusepath{fill}%
\end{pgfscope}%
\begin{pgfscope}%
\pgfpathrectangle{\pgfqpoint{1.250000in}{0.550000in}}{\pgfqpoint{7.750000in}{3.850000in}}%
\pgfusepath{clip}%
\pgfsetbuttcap%
\pgfsetmiterjoin%
\definecolor{currentfill}{rgb}{0.501961,0.501961,0.501961}%
\pgfsetfillcolor{currentfill}%
\pgfsetfillopacity{0.700000}%
\pgfsetlinewidth{0.000000pt}%
\definecolor{currentstroke}{rgb}{0.000000,0.000000,0.000000}%
\pgfsetstrokecolor{currentstroke}%
\pgfsetstrokeopacity{0.700000}%
\pgfsetdash{}{0pt}%
\pgfpathmoveto{\pgfqpoint{8.207386in}{0.550000in}}%
\pgfpathlineto{\pgfqpoint{8.234908in}{0.550000in}}%
\pgfpathlineto{\pgfqpoint{8.234908in}{0.718222in}}%
\pgfpathlineto{\pgfqpoint{8.207386in}{0.718222in}}%
\pgfpathlineto{\pgfqpoint{8.207386in}{0.550000in}}%
\pgfpathclose%
\pgfusepath{fill}%
\end{pgfscope}%
\begin{pgfscope}%
\pgfpathrectangle{\pgfqpoint{1.250000in}{0.550000in}}{\pgfqpoint{7.750000in}{3.850000in}}%
\pgfusepath{clip}%
\pgfsetbuttcap%
\pgfsetmiterjoin%
\definecolor{currentfill}{rgb}{0.501961,0.501961,0.501961}%
\pgfsetfillcolor{currentfill}%
\pgfsetfillopacity{0.700000}%
\pgfsetlinewidth{0.000000pt}%
\definecolor{currentstroke}{rgb}{0.000000,0.000000,0.000000}%
\pgfsetstrokecolor{currentstroke}%
\pgfsetstrokeopacity{0.700000}%
\pgfsetdash{}{0pt}%
\pgfpathmoveto{\pgfqpoint{8.234908in}{0.550000in}}%
\pgfpathlineto{\pgfqpoint{8.262429in}{0.550000in}}%
\pgfpathlineto{\pgfqpoint{8.262429in}{0.679730in}}%
\pgfpathlineto{\pgfqpoint{8.234908in}{0.679730in}}%
\pgfpathlineto{\pgfqpoint{8.234908in}{0.550000in}}%
\pgfpathclose%
\pgfusepath{fill}%
\end{pgfscope}%
\begin{pgfscope}%
\pgfpathrectangle{\pgfqpoint{1.250000in}{0.550000in}}{\pgfqpoint{7.750000in}{3.850000in}}%
\pgfusepath{clip}%
\pgfsetbuttcap%
\pgfsetmiterjoin%
\definecolor{currentfill}{rgb}{0.501961,0.501961,0.501961}%
\pgfsetfillcolor{currentfill}%
\pgfsetfillopacity{0.700000}%
\pgfsetlinewidth{0.000000pt}%
\definecolor{currentstroke}{rgb}{0.000000,0.000000,0.000000}%
\pgfsetstrokecolor{currentstroke}%
\pgfsetstrokeopacity{0.700000}%
\pgfsetdash{}{0pt}%
\pgfpathmoveto{\pgfqpoint{8.262429in}{0.550000in}}%
\pgfpathlineto{\pgfqpoint{8.289950in}{0.550000in}}%
\pgfpathlineto{\pgfqpoint{8.289950in}{0.749585in}}%
\pgfpathlineto{\pgfqpoint{8.262429in}{0.749585in}}%
\pgfpathlineto{\pgfqpoint{8.262429in}{0.550000in}}%
\pgfpathclose%
\pgfusepath{fill}%
\end{pgfscope}%
\begin{pgfscope}%
\pgfpathrectangle{\pgfqpoint{1.250000in}{0.550000in}}{\pgfqpoint{7.750000in}{3.850000in}}%
\pgfusepath{clip}%
\pgfsetbuttcap%
\pgfsetmiterjoin%
\definecolor{currentfill}{rgb}{0.501961,0.501961,0.501961}%
\pgfsetfillcolor{currentfill}%
\pgfsetfillopacity{0.700000}%
\pgfsetlinewidth{0.000000pt}%
\definecolor{currentstroke}{rgb}{0.000000,0.000000,0.000000}%
\pgfsetstrokecolor{currentstroke}%
\pgfsetstrokeopacity{0.700000}%
\pgfsetdash{}{0pt}%
\pgfpathmoveto{\pgfqpoint{8.289950in}{0.550000in}}%
\pgfpathlineto{\pgfqpoint{8.317472in}{0.550000in}}%
\pgfpathlineto{\pgfqpoint{8.317472in}{0.792354in}}%
\pgfpathlineto{\pgfqpoint{8.289950in}{0.792354in}}%
\pgfpathlineto{\pgfqpoint{8.289950in}{0.550000in}}%
\pgfpathclose%
\pgfusepath{fill}%
\end{pgfscope}%
\begin{pgfscope}%
\pgfpathrectangle{\pgfqpoint{1.250000in}{0.550000in}}{\pgfqpoint{7.750000in}{3.850000in}}%
\pgfusepath{clip}%
\pgfsetbuttcap%
\pgfsetmiterjoin%
\definecolor{currentfill}{rgb}{0.501961,0.501961,0.501961}%
\pgfsetfillcolor{currentfill}%
\pgfsetfillopacity{0.700000}%
\pgfsetlinewidth{0.000000pt}%
\definecolor{currentstroke}{rgb}{0.000000,0.000000,0.000000}%
\pgfsetstrokecolor{currentstroke}%
\pgfsetstrokeopacity{0.700000}%
\pgfsetdash{}{0pt}%
\pgfpathmoveto{\pgfqpoint{8.317472in}{0.550000in}}%
\pgfpathlineto{\pgfqpoint{8.344993in}{0.550000in}}%
\pgfpathlineto{\pgfqpoint{8.344993in}{0.709668in}}%
\pgfpathlineto{\pgfqpoint{8.317472in}{0.709668in}}%
\pgfpathlineto{\pgfqpoint{8.317472in}{0.550000in}}%
\pgfpathclose%
\pgfusepath{fill}%
\end{pgfscope}%
\begin{pgfscope}%
\pgfpathrectangle{\pgfqpoint{1.250000in}{0.550000in}}{\pgfqpoint{7.750000in}{3.850000in}}%
\pgfusepath{clip}%
\pgfsetbuttcap%
\pgfsetmiterjoin%
\definecolor{currentfill}{rgb}{0.501961,0.501961,0.501961}%
\pgfsetfillcolor{currentfill}%
\pgfsetfillopacity{0.700000}%
\pgfsetlinewidth{0.000000pt}%
\definecolor{currentstroke}{rgb}{0.000000,0.000000,0.000000}%
\pgfsetstrokecolor{currentstroke}%
\pgfsetstrokeopacity{0.700000}%
\pgfsetdash{}{0pt}%
\pgfpathmoveto{\pgfqpoint{8.344993in}{0.550000in}}%
\pgfpathlineto{\pgfqpoint{8.372514in}{0.550000in}}%
\pgfpathlineto{\pgfqpoint{8.372514in}{0.785226in}}%
\pgfpathlineto{\pgfqpoint{8.344993in}{0.785226in}}%
\pgfpathlineto{\pgfqpoint{8.344993in}{0.550000in}}%
\pgfpathclose%
\pgfusepath{fill}%
\end{pgfscope}%
\begin{pgfscope}%
\pgfpathrectangle{\pgfqpoint{1.250000in}{0.550000in}}{\pgfqpoint{7.750000in}{3.850000in}}%
\pgfusepath{clip}%
\pgfsetbuttcap%
\pgfsetmiterjoin%
\definecolor{currentfill}{rgb}{0.501961,0.501961,0.501961}%
\pgfsetfillcolor{currentfill}%
\pgfsetfillopacity{0.700000}%
\pgfsetlinewidth{0.000000pt}%
\definecolor{currentstroke}{rgb}{0.000000,0.000000,0.000000}%
\pgfsetstrokecolor{currentstroke}%
\pgfsetstrokeopacity{0.700000}%
\pgfsetdash{}{0pt}%
\pgfpathmoveto{\pgfqpoint{8.372514in}{0.550000in}}%
\pgfpathlineto{\pgfqpoint{8.400036in}{0.550000in}}%
\pgfpathlineto{\pgfqpoint{8.400036in}{0.705391in}}%
\pgfpathlineto{\pgfqpoint{8.372514in}{0.705391in}}%
\pgfpathlineto{\pgfqpoint{8.372514in}{0.550000in}}%
\pgfpathclose%
\pgfusepath{fill}%
\end{pgfscope}%
\begin{pgfscope}%
\pgfpathrectangle{\pgfqpoint{1.250000in}{0.550000in}}{\pgfqpoint{7.750000in}{3.850000in}}%
\pgfusepath{clip}%
\pgfsetbuttcap%
\pgfsetmiterjoin%
\definecolor{currentfill}{rgb}{0.501961,0.501961,0.501961}%
\pgfsetfillcolor{currentfill}%
\pgfsetfillopacity{0.700000}%
\pgfsetlinewidth{0.000000pt}%
\definecolor{currentstroke}{rgb}{0.000000,0.000000,0.000000}%
\pgfsetstrokecolor{currentstroke}%
\pgfsetstrokeopacity{0.700000}%
\pgfsetdash{}{0pt}%
\pgfpathmoveto{\pgfqpoint{8.400036in}{0.550000in}}%
\pgfpathlineto{\pgfqpoint{8.427557in}{0.550000in}}%
\pgfpathlineto{\pgfqpoint{8.427557in}{0.772395in}}%
\pgfpathlineto{\pgfqpoint{8.400036in}{0.772395in}}%
\pgfpathlineto{\pgfqpoint{8.400036in}{0.550000in}}%
\pgfpathclose%
\pgfusepath{fill}%
\end{pgfscope}%
\begin{pgfscope}%
\pgfpathrectangle{\pgfqpoint{1.250000in}{0.550000in}}{\pgfqpoint{7.750000in}{3.850000in}}%
\pgfusepath{clip}%
\pgfsetbuttcap%
\pgfsetmiterjoin%
\definecolor{currentfill}{rgb}{0.501961,0.501961,0.501961}%
\pgfsetfillcolor{currentfill}%
\pgfsetfillopacity{0.700000}%
\pgfsetlinewidth{0.000000pt}%
\definecolor{currentstroke}{rgb}{0.000000,0.000000,0.000000}%
\pgfsetstrokecolor{currentstroke}%
\pgfsetstrokeopacity{0.700000}%
\pgfsetdash{}{0pt}%
\pgfpathmoveto{\pgfqpoint{8.427557in}{0.550000in}}%
\pgfpathlineto{\pgfqpoint{8.455078in}{0.550000in}}%
\pgfpathlineto{\pgfqpoint{8.455078in}{0.742457in}}%
\pgfpathlineto{\pgfqpoint{8.427557in}{0.742457in}}%
\pgfpathlineto{\pgfqpoint{8.427557in}{0.550000in}}%
\pgfpathclose%
\pgfusepath{fill}%
\end{pgfscope}%
\begin{pgfscope}%
\pgfpathrectangle{\pgfqpoint{1.250000in}{0.550000in}}{\pgfqpoint{7.750000in}{3.850000in}}%
\pgfusepath{clip}%
\pgfsetbuttcap%
\pgfsetmiterjoin%
\definecolor{currentfill}{rgb}{0.501961,0.501961,0.501961}%
\pgfsetfillcolor{currentfill}%
\pgfsetfillopacity{0.700000}%
\pgfsetlinewidth{0.000000pt}%
\definecolor{currentstroke}{rgb}{0.000000,0.000000,0.000000}%
\pgfsetstrokecolor{currentstroke}%
\pgfsetstrokeopacity{0.700000}%
\pgfsetdash{}{0pt}%
\pgfpathmoveto{\pgfqpoint{8.455078in}{0.550000in}}%
\pgfpathlineto{\pgfqpoint{8.482599in}{0.550000in}}%
\pgfpathlineto{\pgfqpoint{8.482599in}{0.738180in}}%
\pgfpathlineto{\pgfqpoint{8.455078in}{0.738180in}}%
\pgfpathlineto{\pgfqpoint{8.455078in}{0.550000in}}%
\pgfpathclose%
\pgfusepath{fill}%
\end{pgfscope}%
\begin{pgfscope}%
\pgfpathrectangle{\pgfqpoint{1.250000in}{0.550000in}}{\pgfqpoint{7.750000in}{3.850000in}}%
\pgfusepath{clip}%
\pgfsetbuttcap%
\pgfsetmiterjoin%
\definecolor{currentfill}{rgb}{0.501961,0.501961,0.501961}%
\pgfsetfillcolor{currentfill}%
\pgfsetfillopacity{0.700000}%
\pgfsetlinewidth{0.000000pt}%
\definecolor{currentstroke}{rgb}{0.000000,0.000000,0.000000}%
\pgfsetstrokecolor{currentstroke}%
\pgfsetstrokeopacity{0.700000}%
\pgfsetdash{}{0pt}%
\pgfpathmoveto{\pgfqpoint{8.482599in}{0.550000in}}%
\pgfpathlineto{\pgfqpoint{8.510121in}{0.550000in}}%
\pgfpathlineto{\pgfqpoint{8.510121in}{0.760990in}}%
\pgfpathlineto{\pgfqpoint{8.482599in}{0.760990in}}%
\pgfpathlineto{\pgfqpoint{8.482599in}{0.550000in}}%
\pgfpathclose%
\pgfusepath{fill}%
\end{pgfscope}%
\begin{pgfscope}%
\pgfpathrectangle{\pgfqpoint{1.250000in}{0.550000in}}{\pgfqpoint{7.750000in}{3.850000in}}%
\pgfusepath{clip}%
\pgfsetbuttcap%
\pgfsetmiterjoin%
\definecolor{currentfill}{rgb}{0.501961,0.501961,0.501961}%
\pgfsetfillcolor{currentfill}%
\pgfsetfillopacity{0.700000}%
\pgfsetlinewidth{0.000000pt}%
\definecolor{currentstroke}{rgb}{0.000000,0.000000,0.000000}%
\pgfsetstrokecolor{currentstroke}%
\pgfsetstrokeopacity{0.700000}%
\pgfsetdash{}{0pt}%
\pgfpathmoveto{\pgfqpoint{8.510121in}{0.550000in}}%
\pgfpathlineto{\pgfqpoint{8.537642in}{0.550000in}}%
\pgfpathlineto{\pgfqpoint{8.537642in}{0.733904in}}%
\pgfpathlineto{\pgfqpoint{8.510121in}{0.733904in}}%
\pgfpathlineto{\pgfqpoint{8.510121in}{0.550000in}}%
\pgfpathclose%
\pgfusepath{fill}%
\end{pgfscope}%
\begin{pgfscope}%
\pgfpathrectangle{\pgfqpoint{1.250000in}{0.550000in}}{\pgfqpoint{7.750000in}{3.850000in}}%
\pgfusepath{clip}%
\pgfsetbuttcap%
\pgfsetmiterjoin%
\definecolor{currentfill}{rgb}{0.501961,0.501961,0.501961}%
\pgfsetfillcolor{currentfill}%
\pgfsetfillopacity{0.700000}%
\pgfsetlinewidth{0.000000pt}%
\definecolor{currentstroke}{rgb}{0.000000,0.000000,0.000000}%
\pgfsetstrokecolor{currentstroke}%
\pgfsetstrokeopacity{0.700000}%
\pgfsetdash{}{0pt}%
\pgfpathmoveto{\pgfqpoint{8.537642in}{0.550000in}}%
\pgfpathlineto{\pgfqpoint{8.565163in}{0.550000in}}%
\pgfpathlineto{\pgfqpoint{8.565163in}{0.741032in}}%
\pgfpathlineto{\pgfqpoint{8.537642in}{0.741032in}}%
\pgfpathlineto{\pgfqpoint{8.537642in}{0.550000in}}%
\pgfpathclose%
\pgfusepath{fill}%
\end{pgfscope}%
\begin{pgfscope}%
\pgfpathrectangle{\pgfqpoint{1.250000in}{0.550000in}}{\pgfqpoint{7.750000in}{3.850000in}}%
\pgfusepath{clip}%
\pgfsetbuttcap%
\pgfsetmiterjoin%
\definecolor{currentfill}{rgb}{0.501961,0.501961,0.501961}%
\pgfsetfillcolor{currentfill}%
\pgfsetfillopacity{0.700000}%
\pgfsetlinewidth{0.000000pt}%
\definecolor{currentstroke}{rgb}{0.000000,0.000000,0.000000}%
\pgfsetstrokecolor{currentstroke}%
\pgfsetstrokeopacity{0.700000}%
\pgfsetdash{}{0pt}%
\pgfpathmoveto{\pgfqpoint{8.565163in}{0.550000in}}%
\pgfpathlineto{\pgfqpoint{8.592685in}{0.550000in}}%
\pgfpathlineto{\pgfqpoint{8.592685in}{0.728201in}}%
\pgfpathlineto{\pgfqpoint{8.565163in}{0.728201in}}%
\pgfpathlineto{\pgfqpoint{8.565163in}{0.550000in}}%
\pgfpathclose%
\pgfusepath{fill}%
\end{pgfscope}%
\begin{pgfscope}%
\pgfpathrectangle{\pgfqpoint{1.250000in}{0.550000in}}{\pgfqpoint{7.750000in}{3.850000in}}%
\pgfusepath{clip}%
\pgfsetbuttcap%
\pgfsetmiterjoin%
\definecolor{currentfill}{rgb}{0.501961,0.501961,0.501961}%
\pgfsetfillcolor{currentfill}%
\pgfsetfillopacity{0.700000}%
\pgfsetlinewidth{0.000000pt}%
\definecolor{currentstroke}{rgb}{0.000000,0.000000,0.000000}%
\pgfsetstrokecolor{currentstroke}%
\pgfsetstrokeopacity{0.700000}%
\pgfsetdash{}{0pt}%
\pgfpathmoveto{\pgfqpoint{8.592685in}{0.550000in}}%
\pgfpathlineto{\pgfqpoint{8.620206in}{0.550000in}}%
\pgfpathlineto{\pgfqpoint{8.620206in}{0.735329in}}%
\pgfpathlineto{\pgfqpoint{8.592685in}{0.735329in}}%
\pgfpathlineto{\pgfqpoint{8.592685in}{0.550000in}}%
\pgfpathclose%
\pgfusepath{fill}%
\end{pgfscope}%
\begin{pgfscope}%
\pgfpathrectangle{\pgfqpoint{1.250000in}{0.550000in}}{\pgfqpoint{7.750000in}{3.850000in}}%
\pgfusepath{clip}%
\pgfsetbuttcap%
\pgfsetmiterjoin%
\definecolor{currentfill}{rgb}{0.501961,0.501961,0.501961}%
\pgfsetfillcolor{currentfill}%
\pgfsetfillopacity{0.700000}%
\pgfsetlinewidth{0.000000pt}%
\definecolor{currentstroke}{rgb}{0.000000,0.000000,0.000000}%
\pgfsetstrokecolor{currentstroke}%
\pgfsetstrokeopacity{0.700000}%
\pgfsetdash{}{0pt}%
\pgfpathmoveto{\pgfqpoint{8.620206in}{0.550000in}}%
\pgfpathlineto{\pgfqpoint{8.647727in}{0.550000in}}%
\pgfpathlineto{\pgfqpoint{8.647727in}{0.666900in}}%
\pgfpathlineto{\pgfqpoint{8.620206in}{0.666900in}}%
\pgfpathlineto{\pgfqpoint{8.620206in}{0.550000in}}%
\pgfpathclose%
\pgfusepath{fill}%
\end{pgfscope}%
\begin{pgfscope}%
\pgfsetbuttcap%
\pgfsetroundjoin%
\definecolor{currentfill}{rgb}{0.000000,0.000000,0.000000}%
\pgfsetfillcolor{currentfill}%
\pgfsetlinewidth{0.803000pt}%
\definecolor{currentstroke}{rgb}{0.000000,0.000000,0.000000}%
\pgfsetstrokecolor{currentstroke}%
\pgfsetdash{}{0pt}%
\pgfsys@defobject{currentmarker}{\pgfqpoint{0.000000in}{-0.048611in}}{\pgfqpoint{0.000000in}{0.000000in}}{%
\pgfpathmoveto{\pgfqpoint{0.000000in}{0.000000in}}%
\pgfpathlineto{\pgfqpoint{0.000000in}{-0.048611in}}%
\pgfusepath{stroke,fill}%
}%
\begin{pgfscope}%
\pgfsys@transformshift{1.602273in}{0.550000in}%
\pgfsys@useobject{currentmarker}{}%
\end{pgfscope}%
\end{pgfscope}%
\begin{pgfscope}%
\definecolor{textcolor}{rgb}{0.000000,0.000000,0.000000}%
\pgfsetstrokecolor{textcolor}%
\pgfsetfillcolor{textcolor}%
\pgftext[x=1.602273in,y=0.452778in,,top]{\color{textcolor}{\sffamily\fontsize{10.000000}{12.000000}\selectfont\catcode`\^=\active\def^{\ifmmode\sp\else\^{}\fi}\catcode`\%=\active\def%{\%}0}}%
\end{pgfscope}%
\begin{pgfscope}%
\pgfsetbuttcap%
\pgfsetroundjoin%
\definecolor{currentfill}{rgb}{0.000000,0.000000,0.000000}%
\pgfsetfillcolor{currentfill}%
\pgfsetlinewidth{0.803000pt}%
\definecolor{currentstroke}{rgb}{0.000000,0.000000,0.000000}%
\pgfsetstrokecolor{currentstroke}%
\pgfsetdash{}{0pt}%
\pgfsys@defobject{currentmarker}{\pgfqpoint{0.000000in}{-0.048611in}}{\pgfqpoint{0.000000in}{0.000000in}}{%
\pgfpathmoveto{\pgfqpoint{0.000000in}{0.000000in}}%
\pgfpathlineto{\pgfqpoint{0.000000in}{-0.048611in}}%
\pgfusepath{stroke,fill}%
}%
\begin{pgfscope}%
\pgfsys@transformshift{2.978338in}{0.550000in}%
\pgfsys@useobject{currentmarker}{}%
\end{pgfscope}%
\end{pgfscope}%
\begin{pgfscope}%
\definecolor{textcolor}{rgb}{0.000000,0.000000,0.000000}%
\pgfsetstrokecolor{textcolor}%
\pgfsetfillcolor{textcolor}%
\pgftext[x=2.978338in,y=0.452778in,,top]{\color{textcolor}{\sffamily\fontsize{10.000000}{12.000000}\selectfont\catcode`\^=\active\def^{\ifmmode\sp\else\^{}\fi}\catcode`\%=\active\def%{\%}50}}%
\end{pgfscope}%
\begin{pgfscope}%
\pgfsetbuttcap%
\pgfsetroundjoin%
\definecolor{currentfill}{rgb}{0.000000,0.000000,0.000000}%
\pgfsetfillcolor{currentfill}%
\pgfsetlinewidth{0.803000pt}%
\definecolor{currentstroke}{rgb}{0.000000,0.000000,0.000000}%
\pgfsetstrokecolor{currentstroke}%
\pgfsetdash{}{0pt}%
\pgfsys@defobject{currentmarker}{\pgfqpoint{0.000000in}{-0.048611in}}{\pgfqpoint{0.000000in}{0.000000in}}{%
\pgfpathmoveto{\pgfqpoint{0.000000in}{0.000000in}}%
\pgfpathlineto{\pgfqpoint{0.000000in}{-0.048611in}}%
\pgfusepath{stroke,fill}%
}%
\begin{pgfscope}%
\pgfsys@transformshift{4.354403in}{0.550000in}%
\pgfsys@useobject{currentmarker}{}%
\end{pgfscope}%
\end{pgfscope}%
\begin{pgfscope}%
\definecolor{textcolor}{rgb}{0.000000,0.000000,0.000000}%
\pgfsetstrokecolor{textcolor}%
\pgfsetfillcolor{textcolor}%
\pgftext[x=4.354403in,y=0.452778in,,top]{\color{textcolor}{\sffamily\fontsize{10.000000}{12.000000}\selectfont\catcode`\^=\active\def^{\ifmmode\sp\else\^{}\fi}\catcode`\%=\active\def%{\%}100}}%
\end{pgfscope}%
\begin{pgfscope}%
\pgfsetbuttcap%
\pgfsetroundjoin%
\definecolor{currentfill}{rgb}{0.000000,0.000000,0.000000}%
\pgfsetfillcolor{currentfill}%
\pgfsetlinewidth{0.803000pt}%
\definecolor{currentstroke}{rgb}{0.000000,0.000000,0.000000}%
\pgfsetstrokecolor{currentstroke}%
\pgfsetdash{}{0pt}%
\pgfsys@defobject{currentmarker}{\pgfqpoint{0.000000in}{-0.048611in}}{\pgfqpoint{0.000000in}{0.000000in}}{%
\pgfpathmoveto{\pgfqpoint{0.000000in}{0.000000in}}%
\pgfpathlineto{\pgfqpoint{0.000000in}{-0.048611in}}%
\pgfusepath{stroke,fill}%
}%
\begin{pgfscope}%
\pgfsys@transformshift{5.730469in}{0.550000in}%
\pgfsys@useobject{currentmarker}{}%
\end{pgfscope}%
\end{pgfscope}%
\begin{pgfscope}%
\definecolor{textcolor}{rgb}{0.000000,0.000000,0.000000}%
\pgfsetstrokecolor{textcolor}%
\pgfsetfillcolor{textcolor}%
\pgftext[x=5.730469in,y=0.452778in,,top]{\color{textcolor}{\sffamily\fontsize{10.000000}{12.000000}\selectfont\catcode`\^=\active\def^{\ifmmode\sp\else\^{}\fi}\catcode`\%=\active\def%{\%}150}}%
\end{pgfscope}%
\begin{pgfscope}%
\pgfsetbuttcap%
\pgfsetroundjoin%
\definecolor{currentfill}{rgb}{0.000000,0.000000,0.000000}%
\pgfsetfillcolor{currentfill}%
\pgfsetlinewidth{0.803000pt}%
\definecolor{currentstroke}{rgb}{0.000000,0.000000,0.000000}%
\pgfsetstrokecolor{currentstroke}%
\pgfsetdash{}{0pt}%
\pgfsys@defobject{currentmarker}{\pgfqpoint{0.000000in}{-0.048611in}}{\pgfqpoint{0.000000in}{0.000000in}}{%
\pgfpathmoveto{\pgfqpoint{0.000000in}{0.000000in}}%
\pgfpathlineto{\pgfqpoint{0.000000in}{-0.048611in}}%
\pgfusepath{stroke,fill}%
}%
\begin{pgfscope}%
\pgfsys@transformshift{7.106534in}{0.550000in}%
\pgfsys@useobject{currentmarker}{}%
\end{pgfscope}%
\end{pgfscope}%
\begin{pgfscope}%
\definecolor{textcolor}{rgb}{0.000000,0.000000,0.000000}%
\pgfsetstrokecolor{textcolor}%
\pgfsetfillcolor{textcolor}%
\pgftext[x=7.106534in,y=0.452778in,,top]{\color{textcolor}{\sffamily\fontsize{10.000000}{12.000000}\selectfont\catcode`\^=\active\def^{\ifmmode\sp\else\^{}\fi}\catcode`\%=\active\def%{\%}200}}%
\end{pgfscope}%
\begin{pgfscope}%
\pgfsetbuttcap%
\pgfsetroundjoin%
\definecolor{currentfill}{rgb}{0.000000,0.000000,0.000000}%
\pgfsetfillcolor{currentfill}%
\pgfsetlinewidth{0.803000pt}%
\definecolor{currentstroke}{rgb}{0.000000,0.000000,0.000000}%
\pgfsetstrokecolor{currentstroke}%
\pgfsetdash{}{0pt}%
\pgfsys@defobject{currentmarker}{\pgfqpoint{0.000000in}{-0.048611in}}{\pgfqpoint{0.000000in}{0.000000in}}{%
\pgfpathmoveto{\pgfqpoint{0.000000in}{0.000000in}}%
\pgfpathlineto{\pgfqpoint{0.000000in}{-0.048611in}}%
\pgfusepath{stroke,fill}%
}%
\begin{pgfscope}%
\pgfsys@transformshift{8.482599in}{0.550000in}%
\pgfsys@useobject{currentmarker}{}%
\end{pgfscope}%
\end{pgfscope}%
\begin{pgfscope}%
\definecolor{textcolor}{rgb}{0.000000,0.000000,0.000000}%
\pgfsetstrokecolor{textcolor}%
\pgfsetfillcolor{textcolor}%
\pgftext[x=8.482599in,y=0.452778in,,top]{\color{textcolor}{\sffamily\fontsize{10.000000}{12.000000}\selectfont\catcode`\^=\active\def^{\ifmmode\sp\else\^{}\fi}\catcode`\%=\active\def%{\%}250}}%
\end{pgfscope}%
\begin{pgfscope}%
\definecolor{textcolor}{rgb}{0.000000,0.000000,0.000000}%
\pgfsetstrokecolor{textcolor}%
\pgfsetfillcolor{textcolor}%
\pgftext[x=5.125000in,y=0.262809in,,top]{\color{textcolor}{\sffamily\fontsize{10.000000}{12.000000}\selectfont\catcode`\^=\active\def^{\ifmmode\sp\else\^{}\fi}\catcode`\%=\active\def%{\%}Intensidad}}%
\end{pgfscope}%
\begin{pgfscope}%
\pgfsetbuttcap%
\pgfsetroundjoin%
\definecolor{currentfill}{rgb}{0.000000,0.000000,0.000000}%
\pgfsetfillcolor{currentfill}%
\pgfsetlinewidth{0.803000pt}%
\definecolor{currentstroke}{rgb}{0.000000,0.000000,0.000000}%
\pgfsetstrokecolor{currentstroke}%
\pgfsetdash{}{0pt}%
\pgfsys@defobject{currentmarker}{\pgfqpoint{-0.048611in}{0.000000in}}{\pgfqpoint{-0.000000in}{0.000000in}}{%
\pgfpathmoveto{\pgfqpoint{-0.000000in}{0.000000in}}%
\pgfpathlineto{\pgfqpoint{-0.048611in}{0.000000in}}%
\pgfusepath{stroke,fill}%
}%
\begin{pgfscope}%
\pgfsys@transformshift{1.250000in}{0.550000in}%
\pgfsys@useobject{currentmarker}{}%
\end{pgfscope}%
\end{pgfscope}%
\begin{pgfscope}%
\definecolor{textcolor}{rgb}{0.000000,0.000000,0.000000}%
\pgfsetstrokecolor{textcolor}%
\pgfsetfillcolor{textcolor}%
\pgftext[x=1.064412in, y=0.497238in, left, base]{\color{textcolor}{\sffamily\fontsize{10.000000}{12.000000}\selectfont\catcode`\^=\active\def^{\ifmmode\sp\else\^{}\fi}\catcode`\%=\active\def%{\%}0}}%
\end{pgfscope}%
\begin{pgfscope}%
\pgfsetbuttcap%
\pgfsetroundjoin%
\definecolor{currentfill}{rgb}{0.000000,0.000000,0.000000}%
\pgfsetfillcolor{currentfill}%
\pgfsetlinewidth{0.803000pt}%
\definecolor{currentstroke}{rgb}{0.000000,0.000000,0.000000}%
\pgfsetstrokecolor{currentstroke}%
\pgfsetdash{}{0pt}%
\pgfsys@defobject{currentmarker}{\pgfqpoint{-0.048611in}{0.000000in}}{\pgfqpoint{-0.000000in}{0.000000in}}{%
\pgfpathmoveto{\pgfqpoint{-0.000000in}{0.000000in}}%
\pgfpathlineto{\pgfqpoint{-0.048611in}{0.000000in}}%
\pgfusepath{stroke,fill}%
}%
\begin{pgfscope}%
\pgfsys@transformshift{1.250000in}{1.262805in}%
\pgfsys@useobject{currentmarker}{}%
\end{pgfscope}%
\end{pgfscope}%
\begin{pgfscope}%
\definecolor{textcolor}{rgb}{0.000000,0.000000,0.000000}%
\pgfsetstrokecolor{textcolor}%
\pgfsetfillcolor{textcolor}%
\pgftext[x=0.887682in, y=1.210043in, left, base]{\color{textcolor}{\sffamily\fontsize{10.000000}{12.000000}\selectfont\catcode`\^=\active\def^{\ifmmode\sp\else\^{}\fi}\catcode`\%=\active\def%{\%}500}}%
\end{pgfscope}%
\begin{pgfscope}%
\pgfsetbuttcap%
\pgfsetroundjoin%
\definecolor{currentfill}{rgb}{0.000000,0.000000,0.000000}%
\pgfsetfillcolor{currentfill}%
\pgfsetlinewidth{0.803000pt}%
\definecolor{currentstroke}{rgb}{0.000000,0.000000,0.000000}%
\pgfsetstrokecolor{currentstroke}%
\pgfsetdash{}{0pt}%
\pgfsys@defobject{currentmarker}{\pgfqpoint{-0.048611in}{0.000000in}}{\pgfqpoint{-0.000000in}{0.000000in}}{%
\pgfpathmoveto{\pgfqpoint{-0.000000in}{0.000000in}}%
\pgfpathlineto{\pgfqpoint{-0.048611in}{0.000000in}}%
\pgfusepath{stroke,fill}%
}%
\begin{pgfscope}%
\pgfsys@transformshift{1.250000in}{1.975609in}%
\pgfsys@useobject{currentmarker}{}%
\end{pgfscope}%
\end{pgfscope}%
\begin{pgfscope}%
\definecolor{textcolor}{rgb}{0.000000,0.000000,0.000000}%
\pgfsetstrokecolor{textcolor}%
\pgfsetfillcolor{textcolor}%
\pgftext[x=0.799316in, y=1.922848in, left, base]{\color{textcolor}{\sffamily\fontsize{10.000000}{12.000000}\selectfont\catcode`\^=\active\def^{\ifmmode\sp\else\^{}\fi}\catcode`\%=\active\def%{\%}1000}}%
\end{pgfscope}%
\begin{pgfscope}%
\pgfsetbuttcap%
\pgfsetroundjoin%
\definecolor{currentfill}{rgb}{0.000000,0.000000,0.000000}%
\pgfsetfillcolor{currentfill}%
\pgfsetlinewidth{0.803000pt}%
\definecolor{currentstroke}{rgb}{0.000000,0.000000,0.000000}%
\pgfsetstrokecolor{currentstroke}%
\pgfsetdash{}{0pt}%
\pgfsys@defobject{currentmarker}{\pgfqpoint{-0.048611in}{0.000000in}}{\pgfqpoint{-0.000000in}{0.000000in}}{%
\pgfpathmoveto{\pgfqpoint{-0.000000in}{0.000000in}}%
\pgfpathlineto{\pgfqpoint{-0.048611in}{0.000000in}}%
\pgfusepath{stroke,fill}%
}%
\begin{pgfscope}%
\pgfsys@transformshift{1.250000in}{2.688414in}%
\pgfsys@useobject{currentmarker}{}%
\end{pgfscope}%
\end{pgfscope}%
\begin{pgfscope}%
\definecolor{textcolor}{rgb}{0.000000,0.000000,0.000000}%
\pgfsetstrokecolor{textcolor}%
\pgfsetfillcolor{textcolor}%
\pgftext[x=0.799316in, y=2.635652in, left, base]{\color{textcolor}{\sffamily\fontsize{10.000000}{12.000000}\selectfont\catcode`\^=\active\def^{\ifmmode\sp\else\^{}\fi}\catcode`\%=\active\def%{\%}1500}}%
\end{pgfscope}%
\begin{pgfscope}%
\pgfsetbuttcap%
\pgfsetroundjoin%
\definecolor{currentfill}{rgb}{0.000000,0.000000,0.000000}%
\pgfsetfillcolor{currentfill}%
\pgfsetlinewidth{0.803000pt}%
\definecolor{currentstroke}{rgb}{0.000000,0.000000,0.000000}%
\pgfsetstrokecolor{currentstroke}%
\pgfsetdash{}{0pt}%
\pgfsys@defobject{currentmarker}{\pgfqpoint{-0.048611in}{0.000000in}}{\pgfqpoint{-0.000000in}{0.000000in}}{%
\pgfpathmoveto{\pgfqpoint{-0.000000in}{0.000000in}}%
\pgfpathlineto{\pgfqpoint{-0.048611in}{0.000000in}}%
\pgfusepath{stroke,fill}%
}%
\begin{pgfscope}%
\pgfsys@transformshift{1.250000in}{3.401218in}%
\pgfsys@useobject{currentmarker}{}%
\end{pgfscope}%
\end{pgfscope}%
\begin{pgfscope}%
\definecolor{textcolor}{rgb}{0.000000,0.000000,0.000000}%
\pgfsetstrokecolor{textcolor}%
\pgfsetfillcolor{textcolor}%
\pgftext[x=0.799316in, y=3.348457in, left, base]{\color{textcolor}{\sffamily\fontsize{10.000000}{12.000000}\selectfont\catcode`\^=\active\def^{\ifmmode\sp\else\^{}\fi}\catcode`\%=\active\def%{\%}2000}}%
\end{pgfscope}%
\begin{pgfscope}%
\pgfsetbuttcap%
\pgfsetroundjoin%
\definecolor{currentfill}{rgb}{0.000000,0.000000,0.000000}%
\pgfsetfillcolor{currentfill}%
\pgfsetlinewidth{0.803000pt}%
\definecolor{currentstroke}{rgb}{0.000000,0.000000,0.000000}%
\pgfsetstrokecolor{currentstroke}%
\pgfsetdash{}{0pt}%
\pgfsys@defobject{currentmarker}{\pgfqpoint{-0.048611in}{0.000000in}}{\pgfqpoint{-0.000000in}{0.000000in}}{%
\pgfpathmoveto{\pgfqpoint{-0.000000in}{0.000000in}}%
\pgfpathlineto{\pgfqpoint{-0.048611in}{0.000000in}}%
\pgfusepath{stroke,fill}%
}%
\begin{pgfscope}%
\pgfsys@transformshift{1.250000in}{4.114023in}%
\pgfsys@useobject{currentmarker}{}%
\end{pgfscope}%
\end{pgfscope}%
\begin{pgfscope}%
\definecolor{textcolor}{rgb}{0.000000,0.000000,0.000000}%
\pgfsetstrokecolor{textcolor}%
\pgfsetfillcolor{textcolor}%
\pgftext[x=0.799316in, y=4.061261in, left, base]{\color{textcolor}{\sffamily\fontsize{10.000000}{12.000000}\selectfont\catcode`\^=\active\def^{\ifmmode\sp\else\^{}\fi}\catcode`\%=\active\def%{\%}2500}}%
\end{pgfscope}%
\begin{pgfscope}%
\definecolor{textcolor}{rgb}{0.000000,0.000000,0.000000}%
\pgfsetstrokecolor{textcolor}%
\pgfsetfillcolor{textcolor}%
\pgftext[x=0.743761in,y=2.475000in,,bottom,rotate=90.000000]{\color{textcolor}{\sffamily\fontsize{10.000000}{12.000000}\selectfont\catcode`\^=\active\def^{\ifmmode\sp\else\^{}\fi}\catcode`\%=\active\def%{\%}Frecuencia}}%
\end{pgfscope}%
\begin{pgfscope}%
\pgfsetrectcap%
\pgfsetmiterjoin%
\pgfsetlinewidth{0.803000pt}%
\definecolor{currentstroke}{rgb}{0.000000,0.000000,0.000000}%
\pgfsetstrokecolor{currentstroke}%
\pgfsetdash{}{0pt}%
\pgfpathmoveto{\pgfqpoint{1.250000in}{0.550000in}}%
\pgfpathlineto{\pgfqpoint{1.250000in}{4.400000in}}%
\pgfusepath{stroke}%
\end{pgfscope}%
\begin{pgfscope}%
\pgfsetrectcap%
\pgfsetmiterjoin%
\pgfsetlinewidth{0.803000pt}%
\definecolor{currentstroke}{rgb}{0.000000,0.000000,0.000000}%
\pgfsetstrokecolor{currentstroke}%
\pgfsetdash{}{0pt}%
\pgfpathmoveto{\pgfqpoint{9.000000in}{0.550000in}}%
\pgfpathlineto{\pgfqpoint{9.000000in}{4.400000in}}%
\pgfusepath{stroke}%
\end{pgfscope}%
\begin{pgfscope}%
\pgfsetrectcap%
\pgfsetmiterjoin%
\pgfsetlinewidth{0.803000pt}%
\definecolor{currentstroke}{rgb}{0.000000,0.000000,0.000000}%
\pgfsetstrokecolor{currentstroke}%
\pgfsetdash{}{0pt}%
\pgfpathmoveto{\pgfqpoint{1.250000in}{0.550000in}}%
\pgfpathlineto{\pgfqpoint{9.000000in}{0.550000in}}%
\pgfusepath{stroke}%
\end{pgfscope}%
\begin{pgfscope}%
\pgfsetrectcap%
\pgfsetmiterjoin%
\pgfsetlinewidth{0.803000pt}%
\definecolor{currentstroke}{rgb}{0.000000,0.000000,0.000000}%
\pgfsetstrokecolor{currentstroke}%
\pgfsetdash{}{0pt}%
\pgfpathmoveto{\pgfqpoint{1.250000in}{4.400000in}}%
\pgfpathlineto{\pgfqpoint{9.000000in}{4.400000in}}%
\pgfusepath{stroke}%
\end{pgfscope}%
\begin{pgfscope}%
\definecolor{textcolor}{rgb}{0.000000,0.000000,0.000000}%
\pgfsetstrokecolor{textcolor}%
\pgfsetfillcolor{textcolor}%
\pgftext[x=5.125000in,y=4.483333in,,base]{\color{textcolor}{\sffamily\fontsize{12.000000}{14.400000}\selectfont\catcode`\^=\active\def^{\ifmmode\sp\else\^{}\fi}\catcode`\%=\active\def%{\%}Histograma de DQHEPL}}%
\end{pgfscope}%
\end{pgfpicture}%
\makeatother%
\endgroup%
}
		\caption{DQHEPL}
		\label{fig:dqhepl_3}
	\end{subfigure}

	\vspace{5mm}

	\begin{subfigure}[b]{0.45\textwidth}
		\resizebox{\linewidth}{!}{%% Creator: Matplotlib, PGF backend
%%
%% To include the figure in your LaTeX document, write
%%   \input{<filename>.pgf}
%%
%% Make sure the required packages are loaded in your preamble
%%   \usepackage{pgf}
%%
%% Also ensure that all the required font packages are loaded; for instance,
%% the lmodern package is sometimes necessary when using math font.
%%   \usepackage{lmodern}
%%
%% Figures using additional raster images can only be included by \input if
%% they are in the same directory as the main LaTeX file. For loading figures
%% from other directories you can use the `import` package
%%   \usepackage{import}
%%
%% and then include the figures with
%%   \import{<path to file>}{<filename>.pgf}
%%
%% Matplotlib used the following preamble
%%   \def\mathdefault#1{#1}
%%   \everymath=\expandafter{\the\everymath\displaystyle}
%%   \IfFileExists{scrextend.sty}{
%%     \usepackage[fontsize=10.000000pt]{scrextend}
%%   }{
%%     \renewcommand{\normalsize}{\fontsize{10.000000}{12.000000}\selectfont}
%%     \normalsize
%%   }
%%   
%%   \ifdefined\pdftexversion\else  % non-pdftex case.
%%     \usepackage{fontspec}
%%     \setmainfont{DejaVuSerif.ttf}[Path=\detokenize{/home/elias/.virtualenvs/src-pway/lib/python3.13/site-packages/matplotlib/mpl-data/fonts/ttf/}]
%%     \setsansfont{DejaVuSans.ttf}[Path=\detokenize{/home/elias/.virtualenvs/src-pway/lib/python3.13/site-packages/matplotlib/mpl-data/fonts/ttf/}]
%%     \setmonofont{DejaVuSansMono.ttf}[Path=\detokenize{/home/elias/.virtualenvs/src-pway/lib/python3.13/site-packages/matplotlib/mpl-data/fonts/ttf/}]
%%   \fi
%%   \makeatletter\@ifpackageloaded{underscore}{}{\usepackage[strings]{underscore}}\makeatother
%%
\begingroup%
\makeatletter%
\begin{pgfpicture}%
\pgfpathrectangle{\pgfpointorigin}{\pgfqpoint{10.000000in}{5.000000in}}%
\pgfusepath{use as bounding box, clip}%
\begin{pgfscope}%
\pgfsetbuttcap%
\pgfsetmiterjoin%
\definecolor{currentfill}{rgb}{1.000000,1.000000,1.000000}%
\pgfsetfillcolor{currentfill}%
\pgfsetlinewidth{0.000000pt}%
\definecolor{currentstroke}{rgb}{1.000000,1.000000,1.000000}%
\pgfsetstrokecolor{currentstroke}%
\pgfsetdash{}{0pt}%
\pgfpathmoveto{\pgfqpoint{0.000000in}{0.000000in}}%
\pgfpathlineto{\pgfqpoint{10.000000in}{0.000000in}}%
\pgfpathlineto{\pgfqpoint{10.000000in}{5.000000in}}%
\pgfpathlineto{\pgfqpoint{0.000000in}{5.000000in}}%
\pgfpathlineto{\pgfqpoint{0.000000in}{0.000000in}}%
\pgfpathclose%
\pgfusepath{fill}%
\end{pgfscope}%
\begin{pgfscope}%
\pgfsetbuttcap%
\pgfsetmiterjoin%
\definecolor{currentfill}{rgb}{1.000000,1.000000,1.000000}%
\pgfsetfillcolor{currentfill}%
\pgfsetlinewidth{0.000000pt}%
\definecolor{currentstroke}{rgb}{0.000000,0.000000,0.000000}%
\pgfsetstrokecolor{currentstroke}%
\pgfsetstrokeopacity{0.000000}%
\pgfsetdash{}{0pt}%
\pgfpathmoveto{\pgfqpoint{1.250000in}{0.550000in}}%
\pgfpathlineto{\pgfqpoint{9.000000in}{0.550000in}}%
\pgfpathlineto{\pgfqpoint{9.000000in}{4.400000in}}%
\pgfpathlineto{\pgfqpoint{1.250000in}{4.400000in}}%
\pgfpathlineto{\pgfqpoint{1.250000in}{0.550000in}}%
\pgfpathclose%
\pgfusepath{fill}%
\end{pgfscope}%
\begin{pgfscope}%
\pgfpathrectangle{\pgfqpoint{1.250000in}{0.550000in}}{\pgfqpoint{7.750000in}{3.850000in}}%
\pgfusepath{clip}%
\pgfsetbuttcap%
\pgfsetmiterjoin%
\definecolor{currentfill}{rgb}{0.501961,0.501961,0.501961}%
\pgfsetfillcolor{currentfill}%
\pgfsetfillopacity{0.700000}%
\pgfsetlinewidth{0.000000pt}%
\definecolor{currentstroke}{rgb}{0.000000,0.000000,0.000000}%
\pgfsetstrokecolor{currentstroke}%
\pgfsetstrokeopacity{0.700000}%
\pgfsetdash{}{0pt}%
\pgfpathmoveto{\pgfqpoint{1.602273in}{0.550000in}}%
\pgfpathlineto{\pgfqpoint{1.629794in}{0.550000in}}%
\pgfpathlineto{\pgfqpoint{1.629794in}{4.216667in}}%
\pgfpathlineto{\pgfqpoint{1.602273in}{4.216667in}}%
\pgfpathlineto{\pgfqpoint{1.602273in}{0.550000in}}%
\pgfpathclose%
\pgfusepath{fill}%
\end{pgfscope}%
\begin{pgfscope}%
\pgfpathrectangle{\pgfqpoint{1.250000in}{0.550000in}}{\pgfqpoint{7.750000in}{3.850000in}}%
\pgfusepath{clip}%
\pgfsetbuttcap%
\pgfsetmiterjoin%
\definecolor{currentfill}{rgb}{0.501961,0.501961,0.501961}%
\pgfsetfillcolor{currentfill}%
\pgfsetfillopacity{0.700000}%
\pgfsetlinewidth{0.000000pt}%
\definecolor{currentstroke}{rgb}{0.000000,0.000000,0.000000}%
\pgfsetstrokecolor{currentstroke}%
\pgfsetstrokeopacity{0.700000}%
\pgfsetdash{}{0pt}%
\pgfpathmoveto{\pgfqpoint{1.629794in}{0.550000in}}%
\pgfpathlineto{\pgfqpoint{1.657315in}{0.550000in}}%
\pgfpathlineto{\pgfqpoint{1.657315in}{0.550000in}}%
\pgfpathlineto{\pgfqpoint{1.629794in}{0.550000in}}%
\pgfpathlineto{\pgfqpoint{1.629794in}{0.550000in}}%
\pgfpathclose%
\pgfusepath{fill}%
\end{pgfscope}%
\begin{pgfscope}%
\pgfpathrectangle{\pgfqpoint{1.250000in}{0.550000in}}{\pgfqpoint{7.750000in}{3.850000in}}%
\pgfusepath{clip}%
\pgfsetbuttcap%
\pgfsetmiterjoin%
\definecolor{currentfill}{rgb}{0.501961,0.501961,0.501961}%
\pgfsetfillcolor{currentfill}%
\pgfsetfillopacity{0.700000}%
\pgfsetlinewidth{0.000000pt}%
\definecolor{currentstroke}{rgb}{0.000000,0.000000,0.000000}%
\pgfsetstrokecolor{currentstroke}%
\pgfsetstrokeopacity{0.700000}%
\pgfsetdash{}{0pt}%
\pgfpathmoveto{\pgfqpoint{1.657315in}{0.550000in}}%
\pgfpathlineto{\pgfqpoint{1.684837in}{0.550000in}}%
\pgfpathlineto{\pgfqpoint{1.684837in}{0.550000in}}%
\pgfpathlineto{\pgfqpoint{1.657315in}{0.550000in}}%
\pgfpathlineto{\pgfqpoint{1.657315in}{0.550000in}}%
\pgfpathclose%
\pgfusepath{fill}%
\end{pgfscope}%
\begin{pgfscope}%
\pgfpathrectangle{\pgfqpoint{1.250000in}{0.550000in}}{\pgfqpoint{7.750000in}{3.850000in}}%
\pgfusepath{clip}%
\pgfsetbuttcap%
\pgfsetmiterjoin%
\definecolor{currentfill}{rgb}{0.501961,0.501961,0.501961}%
\pgfsetfillcolor{currentfill}%
\pgfsetfillopacity{0.700000}%
\pgfsetlinewidth{0.000000pt}%
\definecolor{currentstroke}{rgb}{0.000000,0.000000,0.000000}%
\pgfsetstrokecolor{currentstroke}%
\pgfsetstrokeopacity{0.700000}%
\pgfsetdash{}{0pt}%
\pgfpathmoveto{\pgfqpoint{1.684837in}{0.550000in}}%
\pgfpathlineto{\pgfqpoint{1.712358in}{0.550000in}}%
\pgfpathlineto{\pgfqpoint{1.712358in}{0.550000in}}%
\pgfpathlineto{\pgfqpoint{1.684837in}{0.550000in}}%
\pgfpathlineto{\pgfqpoint{1.684837in}{0.550000in}}%
\pgfpathclose%
\pgfusepath{fill}%
\end{pgfscope}%
\begin{pgfscope}%
\pgfpathrectangle{\pgfqpoint{1.250000in}{0.550000in}}{\pgfqpoint{7.750000in}{3.850000in}}%
\pgfusepath{clip}%
\pgfsetbuttcap%
\pgfsetmiterjoin%
\definecolor{currentfill}{rgb}{0.501961,0.501961,0.501961}%
\pgfsetfillcolor{currentfill}%
\pgfsetfillopacity{0.700000}%
\pgfsetlinewidth{0.000000pt}%
\definecolor{currentstroke}{rgb}{0.000000,0.000000,0.000000}%
\pgfsetstrokecolor{currentstroke}%
\pgfsetstrokeopacity{0.700000}%
\pgfsetdash{}{0pt}%
\pgfpathmoveto{\pgfqpoint{1.712358in}{0.550000in}}%
\pgfpathlineto{\pgfqpoint{1.739879in}{0.550000in}}%
\pgfpathlineto{\pgfqpoint{1.739879in}{0.550000in}}%
\pgfpathlineto{\pgfqpoint{1.712358in}{0.550000in}}%
\pgfpathlineto{\pgfqpoint{1.712358in}{0.550000in}}%
\pgfpathclose%
\pgfusepath{fill}%
\end{pgfscope}%
\begin{pgfscope}%
\pgfpathrectangle{\pgfqpoint{1.250000in}{0.550000in}}{\pgfqpoint{7.750000in}{3.850000in}}%
\pgfusepath{clip}%
\pgfsetbuttcap%
\pgfsetmiterjoin%
\definecolor{currentfill}{rgb}{0.501961,0.501961,0.501961}%
\pgfsetfillcolor{currentfill}%
\pgfsetfillopacity{0.700000}%
\pgfsetlinewidth{0.000000pt}%
\definecolor{currentstroke}{rgb}{0.000000,0.000000,0.000000}%
\pgfsetstrokecolor{currentstroke}%
\pgfsetstrokeopacity{0.700000}%
\pgfsetdash{}{0pt}%
\pgfpathmoveto{\pgfqpoint{1.739879in}{0.550000in}}%
\pgfpathlineto{\pgfqpoint{1.767401in}{0.550000in}}%
\pgfpathlineto{\pgfqpoint{1.767401in}{0.550000in}}%
\pgfpathlineto{\pgfqpoint{1.739879in}{0.550000in}}%
\pgfpathlineto{\pgfqpoint{1.739879in}{0.550000in}}%
\pgfpathclose%
\pgfusepath{fill}%
\end{pgfscope}%
\begin{pgfscope}%
\pgfpathrectangle{\pgfqpoint{1.250000in}{0.550000in}}{\pgfqpoint{7.750000in}{3.850000in}}%
\pgfusepath{clip}%
\pgfsetbuttcap%
\pgfsetmiterjoin%
\definecolor{currentfill}{rgb}{0.501961,0.501961,0.501961}%
\pgfsetfillcolor{currentfill}%
\pgfsetfillopacity{0.700000}%
\pgfsetlinewidth{0.000000pt}%
\definecolor{currentstroke}{rgb}{0.000000,0.000000,0.000000}%
\pgfsetstrokecolor{currentstroke}%
\pgfsetstrokeopacity{0.700000}%
\pgfsetdash{}{0pt}%
\pgfpathmoveto{\pgfqpoint{1.767401in}{0.550000in}}%
\pgfpathlineto{\pgfqpoint{1.794922in}{0.550000in}}%
\pgfpathlineto{\pgfqpoint{1.794922in}{0.550000in}}%
\pgfpathlineto{\pgfqpoint{1.767401in}{0.550000in}}%
\pgfpathlineto{\pgfqpoint{1.767401in}{0.550000in}}%
\pgfpathclose%
\pgfusepath{fill}%
\end{pgfscope}%
\begin{pgfscope}%
\pgfpathrectangle{\pgfqpoint{1.250000in}{0.550000in}}{\pgfqpoint{7.750000in}{3.850000in}}%
\pgfusepath{clip}%
\pgfsetbuttcap%
\pgfsetmiterjoin%
\definecolor{currentfill}{rgb}{0.501961,0.501961,0.501961}%
\pgfsetfillcolor{currentfill}%
\pgfsetfillopacity{0.700000}%
\pgfsetlinewidth{0.000000pt}%
\definecolor{currentstroke}{rgb}{0.000000,0.000000,0.000000}%
\pgfsetstrokecolor{currentstroke}%
\pgfsetstrokeopacity{0.700000}%
\pgfsetdash{}{0pt}%
\pgfpathmoveto{\pgfqpoint{1.794922in}{0.550000in}}%
\pgfpathlineto{\pgfqpoint{1.822443in}{0.550000in}}%
\pgfpathlineto{\pgfqpoint{1.822443in}{0.550000in}}%
\pgfpathlineto{\pgfqpoint{1.794922in}{0.550000in}}%
\pgfpathlineto{\pgfqpoint{1.794922in}{0.550000in}}%
\pgfpathclose%
\pgfusepath{fill}%
\end{pgfscope}%
\begin{pgfscope}%
\pgfpathrectangle{\pgfqpoint{1.250000in}{0.550000in}}{\pgfqpoint{7.750000in}{3.850000in}}%
\pgfusepath{clip}%
\pgfsetbuttcap%
\pgfsetmiterjoin%
\definecolor{currentfill}{rgb}{0.501961,0.501961,0.501961}%
\pgfsetfillcolor{currentfill}%
\pgfsetfillopacity{0.700000}%
\pgfsetlinewidth{0.000000pt}%
\definecolor{currentstroke}{rgb}{0.000000,0.000000,0.000000}%
\pgfsetstrokecolor{currentstroke}%
\pgfsetstrokeopacity{0.700000}%
\pgfsetdash{}{0pt}%
\pgfpathmoveto{\pgfqpoint{1.822443in}{0.550000in}}%
\pgfpathlineto{\pgfqpoint{1.849964in}{0.550000in}}%
\pgfpathlineto{\pgfqpoint{1.849964in}{0.550000in}}%
\pgfpathlineto{\pgfqpoint{1.822443in}{0.550000in}}%
\pgfpathlineto{\pgfqpoint{1.822443in}{0.550000in}}%
\pgfpathclose%
\pgfusepath{fill}%
\end{pgfscope}%
\begin{pgfscope}%
\pgfpathrectangle{\pgfqpoint{1.250000in}{0.550000in}}{\pgfqpoint{7.750000in}{3.850000in}}%
\pgfusepath{clip}%
\pgfsetbuttcap%
\pgfsetmiterjoin%
\definecolor{currentfill}{rgb}{0.501961,0.501961,0.501961}%
\pgfsetfillcolor{currentfill}%
\pgfsetfillopacity{0.700000}%
\pgfsetlinewidth{0.000000pt}%
\definecolor{currentstroke}{rgb}{0.000000,0.000000,0.000000}%
\pgfsetstrokecolor{currentstroke}%
\pgfsetstrokeopacity{0.700000}%
\pgfsetdash{}{0pt}%
\pgfpathmoveto{\pgfqpoint{1.849964in}{0.550000in}}%
\pgfpathlineto{\pgfqpoint{1.877486in}{0.550000in}}%
\pgfpathlineto{\pgfqpoint{1.877486in}{0.550000in}}%
\pgfpathlineto{\pgfqpoint{1.849964in}{0.550000in}}%
\pgfpathlineto{\pgfqpoint{1.849964in}{0.550000in}}%
\pgfpathclose%
\pgfusepath{fill}%
\end{pgfscope}%
\begin{pgfscope}%
\pgfpathrectangle{\pgfqpoint{1.250000in}{0.550000in}}{\pgfqpoint{7.750000in}{3.850000in}}%
\pgfusepath{clip}%
\pgfsetbuttcap%
\pgfsetmiterjoin%
\definecolor{currentfill}{rgb}{0.501961,0.501961,0.501961}%
\pgfsetfillcolor{currentfill}%
\pgfsetfillopacity{0.700000}%
\pgfsetlinewidth{0.000000pt}%
\definecolor{currentstroke}{rgb}{0.000000,0.000000,0.000000}%
\pgfsetstrokecolor{currentstroke}%
\pgfsetstrokeopacity{0.700000}%
\pgfsetdash{}{0pt}%
\pgfpathmoveto{\pgfqpoint{1.877486in}{0.550000in}}%
\pgfpathlineto{\pgfqpoint{1.905007in}{0.550000in}}%
\pgfpathlineto{\pgfqpoint{1.905007in}{0.550000in}}%
\pgfpathlineto{\pgfqpoint{1.877486in}{0.550000in}}%
\pgfpathlineto{\pgfqpoint{1.877486in}{0.550000in}}%
\pgfpathclose%
\pgfusepath{fill}%
\end{pgfscope}%
\begin{pgfscope}%
\pgfpathrectangle{\pgfqpoint{1.250000in}{0.550000in}}{\pgfqpoint{7.750000in}{3.850000in}}%
\pgfusepath{clip}%
\pgfsetbuttcap%
\pgfsetmiterjoin%
\definecolor{currentfill}{rgb}{0.501961,0.501961,0.501961}%
\pgfsetfillcolor{currentfill}%
\pgfsetfillopacity{0.700000}%
\pgfsetlinewidth{0.000000pt}%
\definecolor{currentstroke}{rgb}{0.000000,0.000000,0.000000}%
\pgfsetstrokecolor{currentstroke}%
\pgfsetstrokeopacity{0.700000}%
\pgfsetdash{}{0pt}%
\pgfpathmoveto{\pgfqpoint{1.905007in}{0.550000in}}%
\pgfpathlineto{\pgfqpoint{1.932528in}{0.550000in}}%
\pgfpathlineto{\pgfqpoint{1.932528in}{0.550000in}}%
\pgfpathlineto{\pgfqpoint{1.905007in}{0.550000in}}%
\pgfpathlineto{\pgfqpoint{1.905007in}{0.550000in}}%
\pgfpathclose%
\pgfusepath{fill}%
\end{pgfscope}%
\begin{pgfscope}%
\pgfpathrectangle{\pgfqpoint{1.250000in}{0.550000in}}{\pgfqpoint{7.750000in}{3.850000in}}%
\pgfusepath{clip}%
\pgfsetbuttcap%
\pgfsetmiterjoin%
\definecolor{currentfill}{rgb}{0.501961,0.501961,0.501961}%
\pgfsetfillcolor{currentfill}%
\pgfsetfillopacity{0.700000}%
\pgfsetlinewidth{0.000000pt}%
\definecolor{currentstroke}{rgb}{0.000000,0.000000,0.000000}%
\pgfsetstrokecolor{currentstroke}%
\pgfsetstrokeopacity{0.700000}%
\pgfsetdash{}{0pt}%
\pgfpathmoveto{\pgfqpoint{1.932528in}{0.550000in}}%
\pgfpathlineto{\pgfqpoint{1.960050in}{0.550000in}}%
\pgfpathlineto{\pgfqpoint{1.960050in}{0.550000in}}%
\pgfpathlineto{\pgfqpoint{1.932528in}{0.550000in}}%
\pgfpathlineto{\pgfqpoint{1.932528in}{0.550000in}}%
\pgfpathclose%
\pgfusepath{fill}%
\end{pgfscope}%
\begin{pgfscope}%
\pgfpathrectangle{\pgfqpoint{1.250000in}{0.550000in}}{\pgfqpoint{7.750000in}{3.850000in}}%
\pgfusepath{clip}%
\pgfsetbuttcap%
\pgfsetmiterjoin%
\definecolor{currentfill}{rgb}{0.501961,0.501961,0.501961}%
\pgfsetfillcolor{currentfill}%
\pgfsetfillopacity{0.700000}%
\pgfsetlinewidth{0.000000pt}%
\definecolor{currentstroke}{rgb}{0.000000,0.000000,0.000000}%
\pgfsetstrokecolor{currentstroke}%
\pgfsetstrokeopacity{0.700000}%
\pgfsetdash{}{0pt}%
\pgfpathmoveto{\pgfqpoint{1.960050in}{0.550000in}}%
\pgfpathlineto{\pgfqpoint{1.987571in}{0.550000in}}%
\pgfpathlineto{\pgfqpoint{1.987571in}{0.550000in}}%
\pgfpathlineto{\pgfqpoint{1.960050in}{0.550000in}}%
\pgfpathlineto{\pgfqpoint{1.960050in}{0.550000in}}%
\pgfpathclose%
\pgfusepath{fill}%
\end{pgfscope}%
\begin{pgfscope}%
\pgfpathrectangle{\pgfqpoint{1.250000in}{0.550000in}}{\pgfqpoint{7.750000in}{3.850000in}}%
\pgfusepath{clip}%
\pgfsetbuttcap%
\pgfsetmiterjoin%
\definecolor{currentfill}{rgb}{0.501961,0.501961,0.501961}%
\pgfsetfillcolor{currentfill}%
\pgfsetfillopacity{0.700000}%
\pgfsetlinewidth{0.000000pt}%
\definecolor{currentstroke}{rgb}{0.000000,0.000000,0.000000}%
\pgfsetstrokecolor{currentstroke}%
\pgfsetstrokeopacity{0.700000}%
\pgfsetdash{}{0pt}%
\pgfpathmoveto{\pgfqpoint{1.987571in}{0.550000in}}%
\pgfpathlineto{\pgfqpoint{2.015092in}{0.550000in}}%
\pgfpathlineto{\pgfqpoint{2.015092in}{0.550000in}}%
\pgfpathlineto{\pgfqpoint{1.987571in}{0.550000in}}%
\pgfpathlineto{\pgfqpoint{1.987571in}{0.550000in}}%
\pgfpathclose%
\pgfusepath{fill}%
\end{pgfscope}%
\begin{pgfscope}%
\pgfpathrectangle{\pgfqpoint{1.250000in}{0.550000in}}{\pgfqpoint{7.750000in}{3.850000in}}%
\pgfusepath{clip}%
\pgfsetbuttcap%
\pgfsetmiterjoin%
\definecolor{currentfill}{rgb}{0.501961,0.501961,0.501961}%
\pgfsetfillcolor{currentfill}%
\pgfsetfillopacity{0.700000}%
\pgfsetlinewidth{0.000000pt}%
\definecolor{currentstroke}{rgb}{0.000000,0.000000,0.000000}%
\pgfsetstrokecolor{currentstroke}%
\pgfsetstrokeopacity{0.700000}%
\pgfsetdash{}{0pt}%
\pgfpathmoveto{\pgfqpoint{2.015092in}{0.550000in}}%
\pgfpathlineto{\pgfqpoint{2.042614in}{0.550000in}}%
\pgfpathlineto{\pgfqpoint{2.042614in}{0.550000in}}%
\pgfpathlineto{\pgfqpoint{2.015092in}{0.550000in}}%
\pgfpathlineto{\pgfqpoint{2.015092in}{0.550000in}}%
\pgfpathclose%
\pgfusepath{fill}%
\end{pgfscope}%
\begin{pgfscope}%
\pgfpathrectangle{\pgfqpoint{1.250000in}{0.550000in}}{\pgfqpoint{7.750000in}{3.850000in}}%
\pgfusepath{clip}%
\pgfsetbuttcap%
\pgfsetmiterjoin%
\definecolor{currentfill}{rgb}{0.501961,0.501961,0.501961}%
\pgfsetfillcolor{currentfill}%
\pgfsetfillopacity{0.700000}%
\pgfsetlinewidth{0.000000pt}%
\definecolor{currentstroke}{rgb}{0.000000,0.000000,0.000000}%
\pgfsetstrokecolor{currentstroke}%
\pgfsetstrokeopacity{0.700000}%
\pgfsetdash{}{0pt}%
\pgfpathmoveto{\pgfqpoint{2.042614in}{0.550000in}}%
\pgfpathlineto{\pgfqpoint{2.070135in}{0.550000in}}%
\pgfpathlineto{\pgfqpoint{2.070135in}{0.550000in}}%
\pgfpathlineto{\pgfqpoint{2.042614in}{0.550000in}}%
\pgfpathlineto{\pgfqpoint{2.042614in}{0.550000in}}%
\pgfpathclose%
\pgfusepath{fill}%
\end{pgfscope}%
\begin{pgfscope}%
\pgfpathrectangle{\pgfqpoint{1.250000in}{0.550000in}}{\pgfqpoint{7.750000in}{3.850000in}}%
\pgfusepath{clip}%
\pgfsetbuttcap%
\pgfsetmiterjoin%
\definecolor{currentfill}{rgb}{0.501961,0.501961,0.501961}%
\pgfsetfillcolor{currentfill}%
\pgfsetfillopacity{0.700000}%
\pgfsetlinewidth{0.000000pt}%
\definecolor{currentstroke}{rgb}{0.000000,0.000000,0.000000}%
\pgfsetstrokecolor{currentstroke}%
\pgfsetstrokeopacity{0.700000}%
\pgfsetdash{}{0pt}%
\pgfpathmoveto{\pgfqpoint{2.070135in}{0.550000in}}%
\pgfpathlineto{\pgfqpoint{2.097656in}{0.550000in}}%
\pgfpathlineto{\pgfqpoint{2.097656in}{0.550000in}}%
\pgfpathlineto{\pgfqpoint{2.070135in}{0.550000in}}%
\pgfpathlineto{\pgfqpoint{2.070135in}{0.550000in}}%
\pgfpathclose%
\pgfusepath{fill}%
\end{pgfscope}%
\begin{pgfscope}%
\pgfpathrectangle{\pgfqpoint{1.250000in}{0.550000in}}{\pgfqpoint{7.750000in}{3.850000in}}%
\pgfusepath{clip}%
\pgfsetbuttcap%
\pgfsetmiterjoin%
\definecolor{currentfill}{rgb}{0.501961,0.501961,0.501961}%
\pgfsetfillcolor{currentfill}%
\pgfsetfillopacity{0.700000}%
\pgfsetlinewidth{0.000000pt}%
\definecolor{currentstroke}{rgb}{0.000000,0.000000,0.000000}%
\pgfsetstrokecolor{currentstroke}%
\pgfsetstrokeopacity{0.700000}%
\pgfsetdash{}{0pt}%
\pgfpathmoveto{\pgfqpoint{2.097656in}{0.550000in}}%
\pgfpathlineto{\pgfqpoint{2.125178in}{0.550000in}}%
\pgfpathlineto{\pgfqpoint{2.125178in}{0.550000in}}%
\pgfpathlineto{\pgfqpoint{2.097656in}{0.550000in}}%
\pgfpathlineto{\pgfqpoint{2.097656in}{0.550000in}}%
\pgfpathclose%
\pgfusepath{fill}%
\end{pgfscope}%
\begin{pgfscope}%
\pgfpathrectangle{\pgfqpoint{1.250000in}{0.550000in}}{\pgfqpoint{7.750000in}{3.850000in}}%
\pgfusepath{clip}%
\pgfsetbuttcap%
\pgfsetmiterjoin%
\definecolor{currentfill}{rgb}{0.501961,0.501961,0.501961}%
\pgfsetfillcolor{currentfill}%
\pgfsetfillopacity{0.700000}%
\pgfsetlinewidth{0.000000pt}%
\definecolor{currentstroke}{rgb}{0.000000,0.000000,0.000000}%
\pgfsetstrokecolor{currentstroke}%
\pgfsetstrokeopacity{0.700000}%
\pgfsetdash{}{0pt}%
\pgfpathmoveto{\pgfqpoint{2.125178in}{0.550000in}}%
\pgfpathlineto{\pgfqpoint{2.152699in}{0.550000in}}%
\pgfpathlineto{\pgfqpoint{2.152699in}{0.550000in}}%
\pgfpathlineto{\pgfqpoint{2.125178in}{0.550000in}}%
\pgfpathlineto{\pgfqpoint{2.125178in}{0.550000in}}%
\pgfpathclose%
\pgfusepath{fill}%
\end{pgfscope}%
\begin{pgfscope}%
\pgfpathrectangle{\pgfqpoint{1.250000in}{0.550000in}}{\pgfqpoint{7.750000in}{3.850000in}}%
\pgfusepath{clip}%
\pgfsetbuttcap%
\pgfsetmiterjoin%
\definecolor{currentfill}{rgb}{0.501961,0.501961,0.501961}%
\pgfsetfillcolor{currentfill}%
\pgfsetfillopacity{0.700000}%
\pgfsetlinewidth{0.000000pt}%
\definecolor{currentstroke}{rgb}{0.000000,0.000000,0.000000}%
\pgfsetstrokecolor{currentstroke}%
\pgfsetstrokeopacity{0.700000}%
\pgfsetdash{}{0pt}%
\pgfpathmoveto{\pgfqpoint{2.152699in}{0.550000in}}%
\pgfpathlineto{\pgfqpoint{2.180220in}{0.550000in}}%
\pgfpathlineto{\pgfqpoint{2.180220in}{3.683059in}}%
\pgfpathlineto{\pgfqpoint{2.152699in}{3.683059in}}%
\pgfpathlineto{\pgfqpoint{2.152699in}{0.550000in}}%
\pgfpathclose%
\pgfusepath{fill}%
\end{pgfscope}%
\begin{pgfscope}%
\pgfpathrectangle{\pgfqpoint{1.250000in}{0.550000in}}{\pgfqpoint{7.750000in}{3.850000in}}%
\pgfusepath{clip}%
\pgfsetbuttcap%
\pgfsetmiterjoin%
\definecolor{currentfill}{rgb}{0.501961,0.501961,0.501961}%
\pgfsetfillcolor{currentfill}%
\pgfsetfillopacity{0.700000}%
\pgfsetlinewidth{0.000000pt}%
\definecolor{currentstroke}{rgb}{0.000000,0.000000,0.000000}%
\pgfsetstrokecolor{currentstroke}%
\pgfsetstrokeopacity{0.700000}%
\pgfsetdash{}{0pt}%
\pgfpathmoveto{\pgfqpoint{2.180220in}{0.550000in}}%
\pgfpathlineto{\pgfqpoint{2.207741in}{0.550000in}}%
\pgfpathlineto{\pgfqpoint{2.207741in}{0.550000in}}%
\pgfpathlineto{\pgfqpoint{2.180220in}{0.550000in}}%
\pgfpathlineto{\pgfqpoint{2.180220in}{0.550000in}}%
\pgfpathclose%
\pgfusepath{fill}%
\end{pgfscope}%
\begin{pgfscope}%
\pgfpathrectangle{\pgfqpoint{1.250000in}{0.550000in}}{\pgfqpoint{7.750000in}{3.850000in}}%
\pgfusepath{clip}%
\pgfsetbuttcap%
\pgfsetmiterjoin%
\definecolor{currentfill}{rgb}{0.501961,0.501961,0.501961}%
\pgfsetfillcolor{currentfill}%
\pgfsetfillopacity{0.700000}%
\pgfsetlinewidth{0.000000pt}%
\definecolor{currentstroke}{rgb}{0.000000,0.000000,0.000000}%
\pgfsetstrokecolor{currentstroke}%
\pgfsetstrokeopacity{0.700000}%
\pgfsetdash{}{0pt}%
\pgfpathmoveto{\pgfqpoint{2.207741in}{0.550000in}}%
\pgfpathlineto{\pgfqpoint{2.235263in}{0.550000in}}%
\pgfpathlineto{\pgfqpoint{2.235263in}{0.550000in}}%
\pgfpathlineto{\pgfqpoint{2.207741in}{0.550000in}}%
\pgfpathlineto{\pgfqpoint{2.207741in}{0.550000in}}%
\pgfpathclose%
\pgfusepath{fill}%
\end{pgfscope}%
\begin{pgfscope}%
\pgfpathrectangle{\pgfqpoint{1.250000in}{0.550000in}}{\pgfqpoint{7.750000in}{3.850000in}}%
\pgfusepath{clip}%
\pgfsetbuttcap%
\pgfsetmiterjoin%
\definecolor{currentfill}{rgb}{0.501961,0.501961,0.501961}%
\pgfsetfillcolor{currentfill}%
\pgfsetfillopacity{0.700000}%
\pgfsetlinewidth{0.000000pt}%
\definecolor{currentstroke}{rgb}{0.000000,0.000000,0.000000}%
\pgfsetstrokecolor{currentstroke}%
\pgfsetstrokeopacity{0.700000}%
\pgfsetdash{}{0pt}%
\pgfpathmoveto{\pgfqpoint{2.235263in}{0.550000in}}%
\pgfpathlineto{\pgfqpoint{2.262784in}{0.550000in}}%
\pgfpathlineto{\pgfqpoint{2.262784in}{0.550000in}}%
\pgfpathlineto{\pgfqpoint{2.235263in}{0.550000in}}%
\pgfpathlineto{\pgfqpoint{2.235263in}{0.550000in}}%
\pgfpathclose%
\pgfusepath{fill}%
\end{pgfscope}%
\begin{pgfscope}%
\pgfpathrectangle{\pgfqpoint{1.250000in}{0.550000in}}{\pgfqpoint{7.750000in}{3.850000in}}%
\pgfusepath{clip}%
\pgfsetbuttcap%
\pgfsetmiterjoin%
\definecolor{currentfill}{rgb}{0.501961,0.501961,0.501961}%
\pgfsetfillcolor{currentfill}%
\pgfsetfillopacity{0.700000}%
\pgfsetlinewidth{0.000000pt}%
\definecolor{currentstroke}{rgb}{0.000000,0.000000,0.000000}%
\pgfsetstrokecolor{currentstroke}%
\pgfsetstrokeopacity{0.700000}%
\pgfsetdash{}{0pt}%
\pgfpathmoveto{\pgfqpoint{2.262784in}{0.550000in}}%
\pgfpathlineto{\pgfqpoint{2.290305in}{0.550000in}}%
\pgfpathlineto{\pgfqpoint{2.290305in}{0.550000in}}%
\pgfpathlineto{\pgfqpoint{2.262784in}{0.550000in}}%
\pgfpathlineto{\pgfqpoint{2.262784in}{0.550000in}}%
\pgfpathclose%
\pgfusepath{fill}%
\end{pgfscope}%
\begin{pgfscope}%
\pgfpathrectangle{\pgfqpoint{1.250000in}{0.550000in}}{\pgfqpoint{7.750000in}{3.850000in}}%
\pgfusepath{clip}%
\pgfsetbuttcap%
\pgfsetmiterjoin%
\definecolor{currentfill}{rgb}{0.501961,0.501961,0.501961}%
\pgfsetfillcolor{currentfill}%
\pgfsetfillopacity{0.700000}%
\pgfsetlinewidth{0.000000pt}%
\definecolor{currentstroke}{rgb}{0.000000,0.000000,0.000000}%
\pgfsetstrokecolor{currentstroke}%
\pgfsetstrokeopacity{0.700000}%
\pgfsetdash{}{0pt}%
\pgfpathmoveto{\pgfqpoint{2.290305in}{0.550000in}}%
\pgfpathlineto{\pgfqpoint{2.317827in}{0.550000in}}%
\pgfpathlineto{\pgfqpoint{2.317827in}{0.550000in}}%
\pgfpathlineto{\pgfqpoint{2.290305in}{0.550000in}}%
\pgfpathlineto{\pgfqpoint{2.290305in}{0.550000in}}%
\pgfpathclose%
\pgfusepath{fill}%
\end{pgfscope}%
\begin{pgfscope}%
\pgfpathrectangle{\pgfqpoint{1.250000in}{0.550000in}}{\pgfqpoint{7.750000in}{3.850000in}}%
\pgfusepath{clip}%
\pgfsetbuttcap%
\pgfsetmiterjoin%
\definecolor{currentfill}{rgb}{0.501961,0.501961,0.501961}%
\pgfsetfillcolor{currentfill}%
\pgfsetfillopacity{0.700000}%
\pgfsetlinewidth{0.000000pt}%
\definecolor{currentstroke}{rgb}{0.000000,0.000000,0.000000}%
\pgfsetstrokecolor{currentstroke}%
\pgfsetstrokeopacity{0.700000}%
\pgfsetdash{}{0pt}%
\pgfpathmoveto{\pgfqpoint{2.317827in}{0.550000in}}%
\pgfpathlineto{\pgfqpoint{2.345348in}{0.550000in}}%
\pgfpathlineto{\pgfqpoint{2.345348in}{0.550000in}}%
\pgfpathlineto{\pgfqpoint{2.317827in}{0.550000in}}%
\pgfpathlineto{\pgfqpoint{2.317827in}{0.550000in}}%
\pgfpathclose%
\pgfusepath{fill}%
\end{pgfscope}%
\begin{pgfscope}%
\pgfpathrectangle{\pgfqpoint{1.250000in}{0.550000in}}{\pgfqpoint{7.750000in}{3.850000in}}%
\pgfusepath{clip}%
\pgfsetbuttcap%
\pgfsetmiterjoin%
\definecolor{currentfill}{rgb}{0.501961,0.501961,0.501961}%
\pgfsetfillcolor{currentfill}%
\pgfsetfillopacity{0.700000}%
\pgfsetlinewidth{0.000000pt}%
\definecolor{currentstroke}{rgb}{0.000000,0.000000,0.000000}%
\pgfsetstrokecolor{currentstroke}%
\pgfsetstrokeopacity{0.700000}%
\pgfsetdash{}{0pt}%
\pgfpathmoveto{\pgfqpoint{2.345348in}{0.550000in}}%
\pgfpathlineto{\pgfqpoint{2.372869in}{0.550000in}}%
\pgfpathlineto{\pgfqpoint{2.372869in}{0.550000in}}%
\pgfpathlineto{\pgfqpoint{2.345348in}{0.550000in}}%
\pgfpathlineto{\pgfqpoint{2.345348in}{0.550000in}}%
\pgfpathclose%
\pgfusepath{fill}%
\end{pgfscope}%
\begin{pgfscope}%
\pgfpathrectangle{\pgfqpoint{1.250000in}{0.550000in}}{\pgfqpoint{7.750000in}{3.850000in}}%
\pgfusepath{clip}%
\pgfsetbuttcap%
\pgfsetmiterjoin%
\definecolor{currentfill}{rgb}{0.501961,0.501961,0.501961}%
\pgfsetfillcolor{currentfill}%
\pgfsetfillopacity{0.700000}%
\pgfsetlinewidth{0.000000pt}%
\definecolor{currentstroke}{rgb}{0.000000,0.000000,0.000000}%
\pgfsetstrokecolor{currentstroke}%
\pgfsetstrokeopacity{0.700000}%
\pgfsetdash{}{0pt}%
\pgfpathmoveto{\pgfqpoint{2.372869in}{0.550000in}}%
\pgfpathlineto{\pgfqpoint{2.400391in}{0.550000in}}%
\pgfpathlineto{\pgfqpoint{2.400391in}{0.550000in}}%
\pgfpathlineto{\pgfqpoint{2.372869in}{0.550000in}}%
\pgfpathlineto{\pgfqpoint{2.372869in}{0.550000in}}%
\pgfpathclose%
\pgfusepath{fill}%
\end{pgfscope}%
\begin{pgfscope}%
\pgfpathrectangle{\pgfqpoint{1.250000in}{0.550000in}}{\pgfqpoint{7.750000in}{3.850000in}}%
\pgfusepath{clip}%
\pgfsetbuttcap%
\pgfsetmiterjoin%
\definecolor{currentfill}{rgb}{0.501961,0.501961,0.501961}%
\pgfsetfillcolor{currentfill}%
\pgfsetfillopacity{0.700000}%
\pgfsetlinewidth{0.000000pt}%
\definecolor{currentstroke}{rgb}{0.000000,0.000000,0.000000}%
\pgfsetstrokecolor{currentstroke}%
\pgfsetstrokeopacity{0.700000}%
\pgfsetdash{}{0pt}%
\pgfpathmoveto{\pgfqpoint{2.400391in}{0.550000in}}%
\pgfpathlineto{\pgfqpoint{2.427912in}{0.550000in}}%
\pgfpathlineto{\pgfqpoint{2.427912in}{0.550000in}}%
\pgfpathlineto{\pgfqpoint{2.400391in}{0.550000in}}%
\pgfpathlineto{\pgfqpoint{2.400391in}{0.550000in}}%
\pgfpathclose%
\pgfusepath{fill}%
\end{pgfscope}%
\begin{pgfscope}%
\pgfpathrectangle{\pgfqpoint{1.250000in}{0.550000in}}{\pgfqpoint{7.750000in}{3.850000in}}%
\pgfusepath{clip}%
\pgfsetbuttcap%
\pgfsetmiterjoin%
\definecolor{currentfill}{rgb}{0.501961,0.501961,0.501961}%
\pgfsetfillcolor{currentfill}%
\pgfsetfillopacity{0.700000}%
\pgfsetlinewidth{0.000000pt}%
\definecolor{currentstroke}{rgb}{0.000000,0.000000,0.000000}%
\pgfsetstrokecolor{currentstroke}%
\pgfsetstrokeopacity{0.700000}%
\pgfsetdash{}{0pt}%
\pgfpathmoveto{\pgfqpoint{2.427912in}{0.550000in}}%
\pgfpathlineto{\pgfqpoint{2.455433in}{0.550000in}}%
\pgfpathlineto{\pgfqpoint{2.455433in}{0.550000in}}%
\pgfpathlineto{\pgfqpoint{2.427912in}{0.550000in}}%
\pgfpathlineto{\pgfqpoint{2.427912in}{0.550000in}}%
\pgfpathclose%
\pgfusepath{fill}%
\end{pgfscope}%
\begin{pgfscope}%
\pgfpathrectangle{\pgfqpoint{1.250000in}{0.550000in}}{\pgfqpoint{7.750000in}{3.850000in}}%
\pgfusepath{clip}%
\pgfsetbuttcap%
\pgfsetmiterjoin%
\definecolor{currentfill}{rgb}{0.501961,0.501961,0.501961}%
\pgfsetfillcolor{currentfill}%
\pgfsetfillopacity{0.700000}%
\pgfsetlinewidth{0.000000pt}%
\definecolor{currentstroke}{rgb}{0.000000,0.000000,0.000000}%
\pgfsetstrokecolor{currentstroke}%
\pgfsetstrokeopacity{0.700000}%
\pgfsetdash{}{0pt}%
\pgfpathmoveto{\pgfqpoint{2.455433in}{0.550000in}}%
\pgfpathlineto{\pgfqpoint{2.482955in}{0.550000in}}%
\pgfpathlineto{\pgfqpoint{2.482955in}{2.356000in}}%
\pgfpathlineto{\pgfqpoint{2.455433in}{2.356000in}}%
\pgfpathlineto{\pgfqpoint{2.455433in}{0.550000in}}%
\pgfpathclose%
\pgfusepath{fill}%
\end{pgfscope}%
\begin{pgfscope}%
\pgfpathrectangle{\pgfqpoint{1.250000in}{0.550000in}}{\pgfqpoint{7.750000in}{3.850000in}}%
\pgfusepath{clip}%
\pgfsetbuttcap%
\pgfsetmiterjoin%
\definecolor{currentfill}{rgb}{0.501961,0.501961,0.501961}%
\pgfsetfillcolor{currentfill}%
\pgfsetfillopacity{0.700000}%
\pgfsetlinewidth{0.000000pt}%
\definecolor{currentstroke}{rgb}{0.000000,0.000000,0.000000}%
\pgfsetstrokecolor{currentstroke}%
\pgfsetstrokeopacity{0.700000}%
\pgfsetdash{}{0pt}%
\pgfpathmoveto{\pgfqpoint{2.482955in}{0.550000in}}%
\pgfpathlineto{\pgfqpoint{2.510476in}{0.550000in}}%
\pgfpathlineto{\pgfqpoint{2.510476in}{0.550000in}}%
\pgfpathlineto{\pgfqpoint{2.482955in}{0.550000in}}%
\pgfpathlineto{\pgfqpoint{2.482955in}{0.550000in}}%
\pgfpathclose%
\pgfusepath{fill}%
\end{pgfscope}%
\begin{pgfscope}%
\pgfpathrectangle{\pgfqpoint{1.250000in}{0.550000in}}{\pgfqpoint{7.750000in}{3.850000in}}%
\pgfusepath{clip}%
\pgfsetbuttcap%
\pgfsetmiterjoin%
\definecolor{currentfill}{rgb}{0.501961,0.501961,0.501961}%
\pgfsetfillcolor{currentfill}%
\pgfsetfillopacity{0.700000}%
\pgfsetlinewidth{0.000000pt}%
\definecolor{currentstroke}{rgb}{0.000000,0.000000,0.000000}%
\pgfsetstrokecolor{currentstroke}%
\pgfsetstrokeopacity{0.700000}%
\pgfsetdash{}{0pt}%
\pgfpathmoveto{\pgfqpoint{2.510476in}{0.550000in}}%
\pgfpathlineto{\pgfqpoint{2.537997in}{0.550000in}}%
\pgfpathlineto{\pgfqpoint{2.537997in}{0.550000in}}%
\pgfpathlineto{\pgfqpoint{2.510476in}{0.550000in}}%
\pgfpathlineto{\pgfqpoint{2.510476in}{0.550000in}}%
\pgfpathclose%
\pgfusepath{fill}%
\end{pgfscope}%
\begin{pgfscope}%
\pgfpathrectangle{\pgfqpoint{1.250000in}{0.550000in}}{\pgfqpoint{7.750000in}{3.850000in}}%
\pgfusepath{clip}%
\pgfsetbuttcap%
\pgfsetmiterjoin%
\definecolor{currentfill}{rgb}{0.501961,0.501961,0.501961}%
\pgfsetfillcolor{currentfill}%
\pgfsetfillopacity{0.700000}%
\pgfsetlinewidth{0.000000pt}%
\definecolor{currentstroke}{rgb}{0.000000,0.000000,0.000000}%
\pgfsetstrokecolor{currentstroke}%
\pgfsetstrokeopacity{0.700000}%
\pgfsetdash{}{0pt}%
\pgfpathmoveto{\pgfqpoint{2.537997in}{0.550000in}}%
\pgfpathlineto{\pgfqpoint{2.565518in}{0.550000in}}%
\pgfpathlineto{\pgfqpoint{2.565518in}{0.550000in}}%
\pgfpathlineto{\pgfqpoint{2.537997in}{0.550000in}}%
\pgfpathlineto{\pgfqpoint{2.537997in}{0.550000in}}%
\pgfpathclose%
\pgfusepath{fill}%
\end{pgfscope}%
\begin{pgfscope}%
\pgfpathrectangle{\pgfqpoint{1.250000in}{0.550000in}}{\pgfqpoint{7.750000in}{3.850000in}}%
\pgfusepath{clip}%
\pgfsetbuttcap%
\pgfsetmiterjoin%
\definecolor{currentfill}{rgb}{0.501961,0.501961,0.501961}%
\pgfsetfillcolor{currentfill}%
\pgfsetfillopacity{0.700000}%
\pgfsetlinewidth{0.000000pt}%
\definecolor{currentstroke}{rgb}{0.000000,0.000000,0.000000}%
\pgfsetstrokecolor{currentstroke}%
\pgfsetstrokeopacity{0.700000}%
\pgfsetdash{}{0pt}%
\pgfpathmoveto{\pgfqpoint{2.565518in}{0.550000in}}%
\pgfpathlineto{\pgfqpoint{2.593040in}{0.550000in}}%
\pgfpathlineto{\pgfqpoint{2.593040in}{0.550000in}}%
\pgfpathlineto{\pgfqpoint{2.565518in}{0.550000in}}%
\pgfpathlineto{\pgfqpoint{2.565518in}{0.550000in}}%
\pgfpathclose%
\pgfusepath{fill}%
\end{pgfscope}%
\begin{pgfscope}%
\pgfpathrectangle{\pgfqpoint{1.250000in}{0.550000in}}{\pgfqpoint{7.750000in}{3.850000in}}%
\pgfusepath{clip}%
\pgfsetbuttcap%
\pgfsetmiterjoin%
\definecolor{currentfill}{rgb}{0.501961,0.501961,0.501961}%
\pgfsetfillcolor{currentfill}%
\pgfsetfillopacity{0.700000}%
\pgfsetlinewidth{0.000000pt}%
\definecolor{currentstroke}{rgb}{0.000000,0.000000,0.000000}%
\pgfsetstrokecolor{currentstroke}%
\pgfsetstrokeopacity{0.700000}%
\pgfsetdash{}{0pt}%
\pgfpathmoveto{\pgfqpoint{2.593040in}{0.550000in}}%
\pgfpathlineto{\pgfqpoint{2.620561in}{0.550000in}}%
\pgfpathlineto{\pgfqpoint{2.620561in}{0.550000in}}%
\pgfpathlineto{\pgfqpoint{2.593040in}{0.550000in}}%
\pgfpathlineto{\pgfqpoint{2.593040in}{0.550000in}}%
\pgfpathclose%
\pgfusepath{fill}%
\end{pgfscope}%
\begin{pgfscope}%
\pgfpathrectangle{\pgfqpoint{1.250000in}{0.550000in}}{\pgfqpoint{7.750000in}{3.850000in}}%
\pgfusepath{clip}%
\pgfsetbuttcap%
\pgfsetmiterjoin%
\definecolor{currentfill}{rgb}{0.501961,0.501961,0.501961}%
\pgfsetfillcolor{currentfill}%
\pgfsetfillopacity{0.700000}%
\pgfsetlinewidth{0.000000pt}%
\definecolor{currentstroke}{rgb}{0.000000,0.000000,0.000000}%
\pgfsetstrokecolor{currentstroke}%
\pgfsetstrokeopacity{0.700000}%
\pgfsetdash{}{0pt}%
\pgfpathmoveto{\pgfqpoint{2.620561in}{0.550000in}}%
\pgfpathlineto{\pgfqpoint{2.648082in}{0.550000in}}%
\pgfpathlineto{\pgfqpoint{2.648082in}{0.550000in}}%
\pgfpathlineto{\pgfqpoint{2.620561in}{0.550000in}}%
\pgfpathlineto{\pgfqpoint{2.620561in}{0.550000in}}%
\pgfpathclose%
\pgfusepath{fill}%
\end{pgfscope}%
\begin{pgfscope}%
\pgfpathrectangle{\pgfqpoint{1.250000in}{0.550000in}}{\pgfqpoint{7.750000in}{3.850000in}}%
\pgfusepath{clip}%
\pgfsetbuttcap%
\pgfsetmiterjoin%
\definecolor{currentfill}{rgb}{0.501961,0.501961,0.501961}%
\pgfsetfillcolor{currentfill}%
\pgfsetfillopacity{0.700000}%
\pgfsetlinewidth{0.000000pt}%
\definecolor{currentstroke}{rgb}{0.000000,0.000000,0.000000}%
\pgfsetstrokecolor{currentstroke}%
\pgfsetstrokeopacity{0.700000}%
\pgfsetdash{}{0pt}%
\pgfpathmoveto{\pgfqpoint{2.648082in}{0.550000in}}%
\pgfpathlineto{\pgfqpoint{2.675604in}{0.550000in}}%
\pgfpathlineto{\pgfqpoint{2.675604in}{0.550000in}}%
\pgfpathlineto{\pgfqpoint{2.648082in}{0.550000in}}%
\pgfpathlineto{\pgfqpoint{2.648082in}{0.550000in}}%
\pgfpathclose%
\pgfusepath{fill}%
\end{pgfscope}%
\begin{pgfscope}%
\pgfpathrectangle{\pgfqpoint{1.250000in}{0.550000in}}{\pgfqpoint{7.750000in}{3.850000in}}%
\pgfusepath{clip}%
\pgfsetbuttcap%
\pgfsetmiterjoin%
\definecolor{currentfill}{rgb}{0.501961,0.501961,0.501961}%
\pgfsetfillcolor{currentfill}%
\pgfsetfillopacity{0.700000}%
\pgfsetlinewidth{0.000000pt}%
\definecolor{currentstroke}{rgb}{0.000000,0.000000,0.000000}%
\pgfsetstrokecolor{currentstroke}%
\pgfsetstrokeopacity{0.700000}%
\pgfsetdash{}{0pt}%
\pgfpathmoveto{\pgfqpoint{2.675604in}{0.550000in}}%
\pgfpathlineto{\pgfqpoint{2.703125in}{0.550000in}}%
\pgfpathlineto{\pgfqpoint{2.703125in}{0.550000in}}%
\pgfpathlineto{\pgfqpoint{2.675604in}{0.550000in}}%
\pgfpathlineto{\pgfqpoint{2.675604in}{0.550000in}}%
\pgfpathclose%
\pgfusepath{fill}%
\end{pgfscope}%
\begin{pgfscope}%
\pgfpathrectangle{\pgfqpoint{1.250000in}{0.550000in}}{\pgfqpoint{7.750000in}{3.850000in}}%
\pgfusepath{clip}%
\pgfsetbuttcap%
\pgfsetmiterjoin%
\definecolor{currentfill}{rgb}{0.501961,0.501961,0.501961}%
\pgfsetfillcolor{currentfill}%
\pgfsetfillopacity{0.700000}%
\pgfsetlinewidth{0.000000pt}%
\definecolor{currentstroke}{rgb}{0.000000,0.000000,0.000000}%
\pgfsetstrokecolor{currentstroke}%
\pgfsetstrokeopacity{0.700000}%
\pgfsetdash{}{0pt}%
\pgfpathmoveto{\pgfqpoint{2.703125in}{0.550000in}}%
\pgfpathlineto{\pgfqpoint{2.730646in}{0.550000in}}%
\pgfpathlineto{\pgfqpoint{2.730646in}{2.100507in}}%
\pgfpathlineto{\pgfqpoint{2.703125in}{2.100507in}}%
\pgfpathlineto{\pgfqpoint{2.703125in}{0.550000in}}%
\pgfpathclose%
\pgfusepath{fill}%
\end{pgfscope}%
\begin{pgfscope}%
\pgfpathrectangle{\pgfqpoint{1.250000in}{0.550000in}}{\pgfqpoint{7.750000in}{3.850000in}}%
\pgfusepath{clip}%
\pgfsetbuttcap%
\pgfsetmiterjoin%
\definecolor{currentfill}{rgb}{0.501961,0.501961,0.501961}%
\pgfsetfillcolor{currentfill}%
\pgfsetfillopacity{0.700000}%
\pgfsetlinewidth{0.000000pt}%
\definecolor{currentstroke}{rgb}{0.000000,0.000000,0.000000}%
\pgfsetstrokecolor{currentstroke}%
\pgfsetstrokeopacity{0.700000}%
\pgfsetdash{}{0pt}%
\pgfpathmoveto{\pgfqpoint{2.730646in}{0.550000in}}%
\pgfpathlineto{\pgfqpoint{2.758168in}{0.550000in}}%
\pgfpathlineto{\pgfqpoint{2.758168in}{0.550000in}}%
\pgfpathlineto{\pgfqpoint{2.730646in}{0.550000in}}%
\pgfpathlineto{\pgfqpoint{2.730646in}{0.550000in}}%
\pgfpathclose%
\pgfusepath{fill}%
\end{pgfscope}%
\begin{pgfscope}%
\pgfpathrectangle{\pgfqpoint{1.250000in}{0.550000in}}{\pgfqpoint{7.750000in}{3.850000in}}%
\pgfusepath{clip}%
\pgfsetbuttcap%
\pgfsetmiterjoin%
\definecolor{currentfill}{rgb}{0.501961,0.501961,0.501961}%
\pgfsetfillcolor{currentfill}%
\pgfsetfillopacity{0.700000}%
\pgfsetlinewidth{0.000000pt}%
\definecolor{currentstroke}{rgb}{0.000000,0.000000,0.000000}%
\pgfsetstrokecolor{currentstroke}%
\pgfsetstrokeopacity{0.700000}%
\pgfsetdash{}{0pt}%
\pgfpathmoveto{\pgfqpoint{2.758168in}{0.550000in}}%
\pgfpathlineto{\pgfqpoint{2.785689in}{0.550000in}}%
\pgfpathlineto{\pgfqpoint{2.785689in}{0.550000in}}%
\pgfpathlineto{\pgfqpoint{2.758168in}{0.550000in}}%
\pgfpathlineto{\pgfqpoint{2.758168in}{0.550000in}}%
\pgfpathclose%
\pgfusepath{fill}%
\end{pgfscope}%
\begin{pgfscope}%
\pgfpathrectangle{\pgfqpoint{1.250000in}{0.550000in}}{\pgfqpoint{7.750000in}{3.850000in}}%
\pgfusepath{clip}%
\pgfsetbuttcap%
\pgfsetmiterjoin%
\definecolor{currentfill}{rgb}{0.501961,0.501961,0.501961}%
\pgfsetfillcolor{currentfill}%
\pgfsetfillopacity{0.700000}%
\pgfsetlinewidth{0.000000pt}%
\definecolor{currentstroke}{rgb}{0.000000,0.000000,0.000000}%
\pgfsetstrokecolor{currentstroke}%
\pgfsetstrokeopacity{0.700000}%
\pgfsetdash{}{0pt}%
\pgfpathmoveto{\pgfqpoint{2.785689in}{0.550000in}}%
\pgfpathlineto{\pgfqpoint{2.813210in}{0.550000in}}%
\pgfpathlineto{\pgfqpoint{2.813210in}{0.550000in}}%
\pgfpathlineto{\pgfqpoint{2.785689in}{0.550000in}}%
\pgfpathlineto{\pgfqpoint{2.785689in}{0.550000in}}%
\pgfpathclose%
\pgfusepath{fill}%
\end{pgfscope}%
\begin{pgfscope}%
\pgfpathrectangle{\pgfqpoint{1.250000in}{0.550000in}}{\pgfqpoint{7.750000in}{3.850000in}}%
\pgfusepath{clip}%
\pgfsetbuttcap%
\pgfsetmiterjoin%
\definecolor{currentfill}{rgb}{0.501961,0.501961,0.501961}%
\pgfsetfillcolor{currentfill}%
\pgfsetfillopacity{0.700000}%
\pgfsetlinewidth{0.000000pt}%
\definecolor{currentstroke}{rgb}{0.000000,0.000000,0.000000}%
\pgfsetstrokecolor{currentstroke}%
\pgfsetstrokeopacity{0.700000}%
\pgfsetdash{}{0pt}%
\pgfpathmoveto{\pgfqpoint{2.813210in}{0.550000in}}%
\pgfpathlineto{\pgfqpoint{2.840732in}{0.550000in}}%
\pgfpathlineto{\pgfqpoint{2.840732in}{0.550000in}}%
\pgfpathlineto{\pgfqpoint{2.813210in}{0.550000in}}%
\pgfpathlineto{\pgfqpoint{2.813210in}{0.550000in}}%
\pgfpathclose%
\pgfusepath{fill}%
\end{pgfscope}%
\begin{pgfscope}%
\pgfpathrectangle{\pgfqpoint{1.250000in}{0.550000in}}{\pgfqpoint{7.750000in}{3.850000in}}%
\pgfusepath{clip}%
\pgfsetbuttcap%
\pgfsetmiterjoin%
\definecolor{currentfill}{rgb}{0.501961,0.501961,0.501961}%
\pgfsetfillcolor{currentfill}%
\pgfsetfillopacity{0.700000}%
\pgfsetlinewidth{0.000000pt}%
\definecolor{currentstroke}{rgb}{0.000000,0.000000,0.000000}%
\pgfsetstrokecolor{currentstroke}%
\pgfsetstrokeopacity{0.700000}%
\pgfsetdash{}{0pt}%
\pgfpathmoveto{\pgfqpoint{2.840732in}{0.550000in}}%
\pgfpathlineto{\pgfqpoint{2.868253in}{0.550000in}}%
\pgfpathlineto{\pgfqpoint{2.868253in}{0.550000in}}%
\pgfpathlineto{\pgfqpoint{2.840732in}{0.550000in}}%
\pgfpathlineto{\pgfqpoint{2.840732in}{0.550000in}}%
\pgfpathclose%
\pgfusepath{fill}%
\end{pgfscope}%
\begin{pgfscope}%
\pgfpathrectangle{\pgfqpoint{1.250000in}{0.550000in}}{\pgfqpoint{7.750000in}{3.850000in}}%
\pgfusepath{clip}%
\pgfsetbuttcap%
\pgfsetmiterjoin%
\definecolor{currentfill}{rgb}{0.501961,0.501961,0.501961}%
\pgfsetfillcolor{currentfill}%
\pgfsetfillopacity{0.700000}%
\pgfsetlinewidth{0.000000pt}%
\definecolor{currentstroke}{rgb}{0.000000,0.000000,0.000000}%
\pgfsetstrokecolor{currentstroke}%
\pgfsetstrokeopacity{0.700000}%
\pgfsetdash{}{0pt}%
\pgfpathmoveto{\pgfqpoint{2.868253in}{0.550000in}}%
\pgfpathlineto{\pgfqpoint{2.895774in}{0.550000in}}%
\pgfpathlineto{\pgfqpoint{2.895774in}{0.550000in}}%
\pgfpathlineto{\pgfqpoint{2.868253in}{0.550000in}}%
\pgfpathlineto{\pgfqpoint{2.868253in}{0.550000in}}%
\pgfpathclose%
\pgfusepath{fill}%
\end{pgfscope}%
\begin{pgfscope}%
\pgfpathrectangle{\pgfqpoint{1.250000in}{0.550000in}}{\pgfqpoint{7.750000in}{3.850000in}}%
\pgfusepath{clip}%
\pgfsetbuttcap%
\pgfsetmiterjoin%
\definecolor{currentfill}{rgb}{0.501961,0.501961,0.501961}%
\pgfsetfillcolor{currentfill}%
\pgfsetfillopacity{0.700000}%
\pgfsetlinewidth{0.000000pt}%
\definecolor{currentstroke}{rgb}{0.000000,0.000000,0.000000}%
\pgfsetstrokecolor{currentstroke}%
\pgfsetstrokeopacity{0.700000}%
\pgfsetdash{}{0pt}%
\pgfpathmoveto{\pgfqpoint{2.895774in}{0.550000in}}%
\pgfpathlineto{\pgfqpoint{2.923295in}{0.550000in}}%
\pgfpathlineto{\pgfqpoint{2.923295in}{0.550000in}}%
\pgfpathlineto{\pgfqpoint{2.895774in}{0.550000in}}%
\pgfpathlineto{\pgfqpoint{2.895774in}{0.550000in}}%
\pgfpathclose%
\pgfusepath{fill}%
\end{pgfscope}%
\begin{pgfscope}%
\pgfpathrectangle{\pgfqpoint{1.250000in}{0.550000in}}{\pgfqpoint{7.750000in}{3.850000in}}%
\pgfusepath{clip}%
\pgfsetbuttcap%
\pgfsetmiterjoin%
\definecolor{currentfill}{rgb}{0.501961,0.501961,0.501961}%
\pgfsetfillcolor{currentfill}%
\pgfsetfillopacity{0.700000}%
\pgfsetlinewidth{0.000000pt}%
\definecolor{currentstroke}{rgb}{0.000000,0.000000,0.000000}%
\pgfsetstrokecolor{currentstroke}%
\pgfsetstrokeopacity{0.700000}%
\pgfsetdash{}{0pt}%
\pgfpathmoveto{\pgfqpoint{2.923295in}{0.550000in}}%
\pgfpathlineto{\pgfqpoint{2.950817in}{0.550000in}}%
\pgfpathlineto{\pgfqpoint{2.950817in}{0.550000in}}%
\pgfpathlineto{\pgfqpoint{2.923295in}{0.550000in}}%
\pgfpathlineto{\pgfqpoint{2.923295in}{0.550000in}}%
\pgfpathclose%
\pgfusepath{fill}%
\end{pgfscope}%
\begin{pgfscope}%
\pgfpathrectangle{\pgfqpoint{1.250000in}{0.550000in}}{\pgfqpoint{7.750000in}{3.850000in}}%
\pgfusepath{clip}%
\pgfsetbuttcap%
\pgfsetmiterjoin%
\definecolor{currentfill}{rgb}{0.501961,0.501961,0.501961}%
\pgfsetfillcolor{currentfill}%
\pgfsetfillopacity{0.700000}%
\pgfsetlinewidth{0.000000pt}%
\definecolor{currentstroke}{rgb}{0.000000,0.000000,0.000000}%
\pgfsetstrokecolor{currentstroke}%
\pgfsetstrokeopacity{0.700000}%
\pgfsetdash{}{0pt}%
\pgfpathmoveto{\pgfqpoint{2.950817in}{0.550000in}}%
\pgfpathlineto{\pgfqpoint{2.978338in}{0.550000in}}%
\pgfpathlineto{\pgfqpoint{2.978338in}{1.845448in}}%
\pgfpathlineto{\pgfqpoint{2.950817in}{1.845448in}}%
\pgfpathlineto{\pgfqpoint{2.950817in}{0.550000in}}%
\pgfpathclose%
\pgfusepath{fill}%
\end{pgfscope}%
\begin{pgfscope}%
\pgfpathrectangle{\pgfqpoint{1.250000in}{0.550000in}}{\pgfqpoint{7.750000in}{3.850000in}}%
\pgfusepath{clip}%
\pgfsetbuttcap%
\pgfsetmiterjoin%
\definecolor{currentfill}{rgb}{0.501961,0.501961,0.501961}%
\pgfsetfillcolor{currentfill}%
\pgfsetfillopacity{0.700000}%
\pgfsetlinewidth{0.000000pt}%
\definecolor{currentstroke}{rgb}{0.000000,0.000000,0.000000}%
\pgfsetstrokecolor{currentstroke}%
\pgfsetstrokeopacity{0.700000}%
\pgfsetdash{}{0pt}%
\pgfpathmoveto{\pgfqpoint{2.978338in}{0.550000in}}%
\pgfpathlineto{\pgfqpoint{3.005859in}{0.550000in}}%
\pgfpathlineto{\pgfqpoint{3.005859in}{0.550000in}}%
\pgfpathlineto{\pgfqpoint{2.978338in}{0.550000in}}%
\pgfpathlineto{\pgfqpoint{2.978338in}{0.550000in}}%
\pgfpathclose%
\pgfusepath{fill}%
\end{pgfscope}%
\begin{pgfscope}%
\pgfpathrectangle{\pgfqpoint{1.250000in}{0.550000in}}{\pgfqpoint{7.750000in}{3.850000in}}%
\pgfusepath{clip}%
\pgfsetbuttcap%
\pgfsetmiterjoin%
\definecolor{currentfill}{rgb}{0.501961,0.501961,0.501961}%
\pgfsetfillcolor{currentfill}%
\pgfsetfillopacity{0.700000}%
\pgfsetlinewidth{0.000000pt}%
\definecolor{currentstroke}{rgb}{0.000000,0.000000,0.000000}%
\pgfsetstrokecolor{currentstroke}%
\pgfsetstrokeopacity{0.700000}%
\pgfsetdash{}{0pt}%
\pgfpathmoveto{\pgfqpoint{3.005859in}{0.550000in}}%
\pgfpathlineto{\pgfqpoint{3.033381in}{0.550000in}}%
\pgfpathlineto{\pgfqpoint{3.033381in}{0.550000in}}%
\pgfpathlineto{\pgfqpoint{3.005859in}{0.550000in}}%
\pgfpathlineto{\pgfqpoint{3.005859in}{0.550000in}}%
\pgfpathclose%
\pgfusepath{fill}%
\end{pgfscope}%
\begin{pgfscope}%
\pgfpathrectangle{\pgfqpoint{1.250000in}{0.550000in}}{\pgfqpoint{7.750000in}{3.850000in}}%
\pgfusepath{clip}%
\pgfsetbuttcap%
\pgfsetmiterjoin%
\definecolor{currentfill}{rgb}{0.501961,0.501961,0.501961}%
\pgfsetfillcolor{currentfill}%
\pgfsetfillopacity{0.700000}%
\pgfsetlinewidth{0.000000pt}%
\definecolor{currentstroke}{rgb}{0.000000,0.000000,0.000000}%
\pgfsetstrokecolor{currentstroke}%
\pgfsetstrokeopacity{0.700000}%
\pgfsetdash{}{0pt}%
\pgfpathmoveto{\pgfqpoint{3.033381in}{0.550000in}}%
\pgfpathlineto{\pgfqpoint{3.060902in}{0.550000in}}%
\pgfpathlineto{\pgfqpoint{3.060902in}{0.550000in}}%
\pgfpathlineto{\pgfqpoint{3.033381in}{0.550000in}}%
\pgfpathlineto{\pgfqpoint{3.033381in}{0.550000in}}%
\pgfpathclose%
\pgfusepath{fill}%
\end{pgfscope}%
\begin{pgfscope}%
\pgfpathrectangle{\pgfqpoint{1.250000in}{0.550000in}}{\pgfqpoint{7.750000in}{3.850000in}}%
\pgfusepath{clip}%
\pgfsetbuttcap%
\pgfsetmiterjoin%
\definecolor{currentfill}{rgb}{0.501961,0.501961,0.501961}%
\pgfsetfillcolor{currentfill}%
\pgfsetfillopacity{0.700000}%
\pgfsetlinewidth{0.000000pt}%
\definecolor{currentstroke}{rgb}{0.000000,0.000000,0.000000}%
\pgfsetstrokecolor{currentstroke}%
\pgfsetstrokeopacity{0.700000}%
\pgfsetdash{}{0pt}%
\pgfpathmoveto{\pgfqpoint{3.060902in}{0.550000in}}%
\pgfpathlineto{\pgfqpoint{3.088423in}{0.550000in}}%
\pgfpathlineto{\pgfqpoint{3.088423in}{0.550000in}}%
\pgfpathlineto{\pgfqpoint{3.060902in}{0.550000in}}%
\pgfpathlineto{\pgfqpoint{3.060902in}{0.550000in}}%
\pgfpathclose%
\pgfusepath{fill}%
\end{pgfscope}%
\begin{pgfscope}%
\pgfpathrectangle{\pgfqpoint{1.250000in}{0.550000in}}{\pgfqpoint{7.750000in}{3.850000in}}%
\pgfusepath{clip}%
\pgfsetbuttcap%
\pgfsetmiterjoin%
\definecolor{currentfill}{rgb}{0.501961,0.501961,0.501961}%
\pgfsetfillcolor{currentfill}%
\pgfsetfillopacity{0.700000}%
\pgfsetlinewidth{0.000000pt}%
\definecolor{currentstroke}{rgb}{0.000000,0.000000,0.000000}%
\pgfsetstrokecolor{currentstroke}%
\pgfsetstrokeopacity{0.700000}%
\pgfsetdash{}{0pt}%
\pgfpathmoveto{\pgfqpoint{3.088423in}{0.550000in}}%
\pgfpathlineto{\pgfqpoint{3.115945in}{0.550000in}}%
\pgfpathlineto{\pgfqpoint{3.115945in}{0.550000in}}%
\pgfpathlineto{\pgfqpoint{3.088423in}{0.550000in}}%
\pgfpathlineto{\pgfqpoint{3.088423in}{0.550000in}}%
\pgfpathclose%
\pgfusepath{fill}%
\end{pgfscope}%
\begin{pgfscope}%
\pgfpathrectangle{\pgfqpoint{1.250000in}{0.550000in}}{\pgfqpoint{7.750000in}{3.850000in}}%
\pgfusepath{clip}%
\pgfsetbuttcap%
\pgfsetmiterjoin%
\definecolor{currentfill}{rgb}{0.501961,0.501961,0.501961}%
\pgfsetfillcolor{currentfill}%
\pgfsetfillopacity{0.700000}%
\pgfsetlinewidth{0.000000pt}%
\definecolor{currentstroke}{rgb}{0.000000,0.000000,0.000000}%
\pgfsetstrokecolor{currentstroke}%
\pgfsetstrokeopacity{0.700000}%
\pgfsetdash{}{0pt}%
\pgfpathmoveto{\pgfqpoint{3.115945in}{0.550000in}}%
\pgfpathlineto{\pgfqpoint{3.143466in}{0.550000in}}%
\pgfpathlineto{\pgfqpoint{3.143466in}{1.619100in}}%
\pgfpathlineto{\pgfqpoint{3.115945in}{1.619100in}}%
\pgfpathlineto{\pgfqpoint{3.115945in}{0.550000in}}%
\pgfpathclose%
\pgfusepath{fill}%
\end{pgfscope}%
\begin{pgfscope}%
\pgfpathrectangle{\pgfqpoint{1.250000in}{0.550000in}}{\pgfqpoint{7.750000in}{3.850000in}}%
\pgfusepath{clip}%
\pgfsetbuttcap%
\pgfsetmiterjoin%
\definecolor{currentfill}{rgb}{0.501961,0.501961,0.501961}%
\pgfsetfillcolor{currentfill}%
\pgfsetfillopacity{0.700000}%
\pgfsetlinewidth{0.000000pt}%
\definecolor{currentstroke}{rgb}{0.000000,0.000000,0.000000}%
\pgfsetstrokecolor{currentstroke}%
\pgfsetstrokeopacity{0.700000}%
\pgfsetdash{}{0pt}%
\pgfpathmoveto{\pgfqpoint{3.143466in}{0.550000in}}%
\pgfpathlineto{\pgfqpoint{3.170987in}{0.550000in}}%
\pgfpathlineto{\pgfqpoint{3.170987in}{0.550000in}}%
\pgfpathlineto{\pgfqpoint{3.143466in}{0.550000in}}%
\pgfpathlineto{\pgfqpoint{3.143466in}{0.550000in}}%
\pgfpathclose%
\pgfusepath{fill}%
\end{pgfscope}%
\begin{pgfscope}%
\pgfpathrectangle{\pgfqpoint{1.250000in}{0.550000in}}{\pgfqpoint{7.750000in}{3.850000in}}%
\pgfusepath{clip}%
\pgfsetbuttcap%
\pgfsetmiterjoin%
\definecolor{currentfill}{rgb}{0.501961,0.501961,0.501961}%
\pgfsetfillcolor{currentfill}%
\pgfsetfillopacity{0.700000}%
\pgfsetlinewidth{0.000000pt}%
\definecolor{currentstroke}{rgb}{0.000000,0.000000,0.000000}%
\pgfsetstrokecolor{currentstroke}%
\pgfsetstrokeopacity{0.700000}%
\pgfsetdash{}{0pt}%
\pgfpathmoveto{\pgfqpoint{3.170987in}{0.550000in}}%
\pgfpathlineto{\pgfqpoint{3.198509in}{0.550000in}}%
\pgfpathlineto{\pgfqpoint{3.198509in}{0.550000in}}%
\pgfpathlineto{\pgfqpoint{3.170987in}{0.550000in}}%
\pgfpathlineto{\pgfqpoint{3.170987in}{0.550000in}}%
\pgfpathclose%
\pgfusepath{fill}%
\end{pgfscope}%
\begin{pgfscope}%
\pgfpathrectangle{\pgfqpoint{1.250000in}{0.550000in}}{\pgfqpoint{7.750000in}{3.850000in}}%
\pgfusepath{clip}%
\pgfsetbuttcap%
\pgfsetmiterjoin%
\definecolor{currentfill}{rgb}{0.501961,0.501961,0.501961}%
\pgfsetfillcolor{currentfill}%
\pgfsetfillopacity{0.700000}%
\pgfsetlinewidth{0.000000pt}%
\definecolor{currentstroke}{rgb}{0.000000,0.000000,0.000000}%
\pgfsetstrokecolor{currentstroke}%
\pgfsetstrokeopacity{0.700000}%
\pgfsetdash{}{0pt}%
\pgfpathmoveto{\pgfqpoint{3.198509in}{0.550000in}}%
\pgfpathlineto{\pgfqpoint{3.226030in}{0.550000in}}%
\pgfpathlineto{\pgfqpoint{3.226030in}{0.550000in}}%
\pgfpathlineto{\pgfqpoint{3.198509in}{0.550000in}}%
\pgfpathlineto{\pgfqpoint{3.198509in}{0.550000in}}%
\pgfpathclose%
\pgfusepath{fill}%
\end{pgfscope}%
\begin{pgfscope}%
\pgfpathrectangle{\pgfqpoint{1.250000in}{0.550000in}}{\pgfqpoint{7.750000in}{3.850000in}}%
\pgfusepath{clip}%
\pgfsetbuttcap%
\pgfsetmiterjoin%
\definecolor{currentfill}{rgb}{0.501961,0.501961,0.501961}%
\pgfsetfillcolor{currentfill}%
\pgfsetfillopacity{0.700000}%
\pgfsetlinewidth{0.000000pt}%
\definecolor{currentstroke}{rgb}{0.000000,0.000000,0.000000}%
\pgfsetstrokecolor{currentstroke}%
\pgfsetstrokeopacity{0.700000}%
\pgfsetdash{}{0pt}%
\pgfpathmoveto{\pgfqpoint{3.226030in}{0.550000in}}%
\pgfpathlineto{\pgfqpoint{3.253551in}{0.550000in}}%
\pgfpathlineto{\pgfqpoint{3.253551in}{0.550000in}}%
\pgfpathlineto{\pgfqpoint{3.226030in}{0.550000in}}%
\pgfpathlineto{\pgfqpoint{3.226030in}{0.550000in}}%
\pgfpathclose%
\pgfusepath{fill}%
\end{pgfscope}%
\begin{pgfscope}%
\pgfpathrectangle{\pgfqpoint{1.250000in}{0.550000in}}{\pgfqpoint{7.750000in}{3.850000in}}%
\pgfusepath{clip}%
\pgfsetbuttcap%
\pgfsetmiterjoin%
\definecolor{currentfill}{rgb}{0.501961,0.501961,0.501961}%
\pgfsetfillcolor{currentfill}%
\pgfsetfillopacity{0.700000}%
\pgfsetlinewidth{0.000000pt}%
\definecolor{currentstroke}{rgb}{0.000000,0.000000,0.000000}%
\pgfsetstrokecolor{currentstroke}%
\pgfsetstrokeopacity{0.700000}%
\pgfsetdash{}{0pt}%
\pgfpathmoveto{\pgfqpoint{3.253551in}{0.550000in}}%
\pgfpathlineto{\pgfqpoint{3.281072in}{0.550000in}}%
\pgfpathlineto{\pgfqpoint{3.281072in}{0.550000in}}%
\pgfpathlineto{\pgfqpoint{3.253551in}{0.550000in}}%
\pgfpathlineto{\pgfqpoint{3.253551in}{0.550000in}}%
\pgfpathclose%
\pgfusepath{fill}%
\end{pgfscope}%
\begin{pgfscope}%
\pgfpathrectangle{\pgfqpoint{1.250000in}{0.550000in}}{\pgfqpoint{7.750000in}{3.850000in}}%
\pgfusepath{clip}%
\pgfsetbuttcap%
\pgfsetmiterjoin%
\definecolor{currentfill}{rgb}{0.501961,0.501961,0.501961}%
\pgfsetfillcolor{currentfill}%
\pgfsetfillopacity{0.700000}%
\pgfsetlinewidth{0.000000pt}%
\definecolor{currentstroke}{rgb}{0.000000,0.000000,0.000000}%
\pgfsetstrokecolor{currentstroke}%
\pgfsetstrokeopacity{0.700000}%
\pgfsetdash{}{0pt}%
\pgfpathmoveto{\pgfqpoint{3.281072in}{0.550000in}}%
\pgfpathlineto{\pgfqpoint{3.308594in}{0.550000in}}%
\pgfpathlineto{\pgfqpoint{3.308594in}{1.464963in}}%
\pgfpathlineto{\pgfqpoint{3.281072in}{1.464963in}}%
\pgfpathlineto{\pgfqpoint{3.281072in}{0.550000in}}%
\pgfpathclose%
\pgfusepath{fill}%
\end{pgfscope}%
\begin{pgfscope}%
\pgfpathrectangle{\pgfqpoint{1.250000in}{0.550000in}}{\pgfqpoint{7.750000in}{3.850000in}}%
\pgfusepath{clip}%
\pgfsetbuttcap%
\pgfsetmiterjoin%
\definecolor{currentfill}{rgb}{0.501961,0.501961,0.501961}%
\pgfsetfillcolor{currentfill}%
\pgfsetfillopacity{0.700000}%
\pgfsetlinewidth{0.000000pt}%
\definecolor{currentstroke}{rgb}{0.000000,0.000000,0.000000}%
\pgfsetstrokecolor{currentstroke}%
\pgfsetstrokeopacity{0.700000}%
\pgfsetdash{}{0pt}%
\pgfpathmoveto{\pgfqpoint{3.308594in}{0.550000in}}%
\pgfpathlineto{\pgfqpoint{3.336115in}{0.550000in}}%
\pgfpathlineto{\pgfqpoint{3.336115in}{0.550000in}}%
\pgfpathlineto{\pgfqpoint{3.308594in}{0.550000in}}%
\pgfpathlineto{\pgfqpoint{3.308594in}{0.550000in}}%
\pgfpathclose%
\pgfusepath{fill}%
\end{pgfscope}%
\begin{pgfscope}%
\pgfpathrectangle{\pgfqpoint{1.250000in}{0.550000in}}{\pgfqpoint{7.750000in}{3.850000in}}%
\pgfusepath{clip}%
\pgfsetbuttcap%
\pgfsetmiterjoin%
\definecolor{currentfill}{rgb}{0.501961,0.501961,0.501961}%
\pgfsetfillcolor{currentfill}%
\pgfsetfillopacity{0.700000}%
\pgfsetlinewidth{0.000000pt}%
\definecolor{currentstroke}{rgb}{0.000000,0.000000,0.000000}%
\pgfsetstrokecolor{currentstroke}%
\pgfsetstrokeopacity{0.700000}%
\pgfsetdash{}{0pt}%
\pgfpathmoveto{\pgfqpoint{3.336115in}{0.550000in}}%
\pgfpathlineto{\pgfqpoint{3.363636in}{0.550000in}}%
\pgfpathlineto{\pgfqpoint{3.363636in}{0.550000in}}%
\pgfpathlineto{\pgfqpoint{3.336115in}{0.550000in}}%
\pgfpathlineto{\pgfqpoint{3.336115in}{0.550000in}}%
\pgfpathclose%
\pgfusepath{fill}%
\end{pgfscope}%
\begin{pgfscope}%
\pgfpathrectangle{\pgfqpoint{1.250000in}{0.550000in}}{\pgfqpoint{7.750000in}{3.850000in}}%
\pgfusepath{clip}%
\pgfsetbuttcap%
\pgfsetmiterjoin%
\definecolor{currentfill}{rgb}{0.501961,0.501961,0.501961}%
\pgfsetfillcolor{currentfill}%
\pgfsetfillopacity{0.700000}%
\pgfsetlinewidth{0.000000pt}%
\definecolor{currentstroke}{rgb}{0.000000,0.000000,0.000000}%
\pgfsetstrokecolor{currentstroke}%
\pgfsetstrokeopacity{0.700000}%
\pgfsetdash{}{0pt}%
\pgfpathmoveto{\pgfqpoint{3.363636in}{0.550000in}}%
\pgfpathlineto{\pgfqpoint{3.391158in}{0.550000in}}%
\pgfpathlineto{\pgfqpoint{3.391158in}{0.550000in}}%
\pgfpathlineto{\pgfqpoint{3.363636in}{0.550000in}}%
\pgfpathlineto{\pgfqpoint{3.363636in}{0.550000in}}%
\pgfpathclose%
\pgfusepath{fill}%
\end{pgfscope}%
\begin{pgfscope}%
\pgfpathrectangle{\pgfqpoint{1.250000in}{0.550000in}}{\pgfqpoint{7.750000in}{3.850000in}}%
\pgfusepath{clip}%
\pgfsetbuttcap%
\pgfsetmiterjoin%
\definecolor{currentfill}{rgb}{0.501961,0.501961,0.501961}%
\pgfsetfillcolor{currentfill}%
\pgfsetfillopacity{0.700000}%
\pgfsetlinewidth{0.000000pt}%
\definecolor{currentstroke}{rgb}{0.000000,0.000000,0.000000}%
\pgfsetstrokecolor{currentstroke}%
\pgfsetstrokeopacity{0.700000}%
\pgfsetdash{}{0pt}%
\pgfpathmoveto{\pgfqpoint{3.391158in}{0.550000in}}%
\pgfpathlineto{\pgfqpoint{3.418679in}{0.550000in}}%
\pgfpathlineto{\pgfqpoint{3.418679in}{0.550000in}}%
\pgfpathlineto{\pgfqpoint{3.391158in}{0.550000in}}%
\pgfpathlineto{\pgfqpoint{3.391158in}{0.550000in}}%
\pgfpathclose%
\pgfusepath{fill}%
\end{pgfscope}%
\begin{pgfscope}%
\pgfpathrectangle{\pgfqpoint{1.250000in}{0.550000in}}{\pgfqpoint{7.750000in}{3.850000in}}%
\pgfusepath{clip}%
\pgfsetbuttcap%
\pgfsetmiterjoin%
\definecolor{currentfill}{rgb}{0.501961,0.501961,0.501961}%
\pgfsetfillcolor{currentfill}%
\pgfsetfillopacity{0.700000}%
\pgfsetlinewidth{0.000000pt}%
\definecolor{currentstroke}{rgb}{0.000000,0.000000,0.000000}%
\pgfsetstrokecolor{currentstroke}%
\pgfsetstrokeopacity{0.700000}%
\pgfsetdash{}{0pt}%
\pgfpathmoveto{\pgfqpoint{3.418679in}{0.550000in}}%
\pgfpathlineto{\pgfqpoint{3.446200in}{0.550000in}}%
\pgfpathlineto{\pgfqpoint{3.446200in}{1.319671in}}%
\pgfpathlineto{\pgfqpoint{3.418679in}{1.319671in}}%
\pgfpathlineto{\pgfqpoint{3.418679in}{0.550000in}}%
\pgfpathclose%
\pgfusepath{fill}%
\end{pgfscope}%
\begin{pgfscope}%
\pgfpathrectangle{\pgfqpoint{1.250000in}{0.550000in}}{\pgfqpoint{7.750000in}{3.850000in}}%
\pgfusepath{clip}%
\pgfsetbuttcap%
\pgfsetmiterjoin%
\definecolor{currentfill}{rgb}{0.501961,0.501961,0.501961}%
\pgfsetfillcolor{currentfill}%
\pgfsetfillopacity{0.700000}%
\pgfsetlinewidth{0.000000pt}%
\definecolor{currentstroke}{rgb}{0.000000,0.000000,0.000000}%
\pgfsetstrokecolor{currentstroke}%
\pgfsetstrokeopacity{0.700000}%
\pgfsetdash{}{0pt}%
\pgfpathmoveto{\pgfqpoint{3.446200in}{0.550000in}}%
\pgfpathlineto{\pgfqpoint{3.473722in}{0.550000in}}%
\pgfpathlineto{\pgfqpoint{3.473722in}{0.550000in}}%
\pgfpathlineto{\pgfqpoint{3.446200in}{0.550000in}}%
\pgfpathlineto{\pgfqpoint{3.446200in}{0.550000in}}%
\pgfpathclose%
\pgfusepath{fill}%
\end{pgfscope}%
\begin{pgfscope}%
\pgfpathrectangle{\pgfqpoint{1.250000in}{0.550000in}}{\pgfqpoint{7.750000in}{3.850000in}}%
\pgfusepath{clip}%
\pgfsetbuttcap%
\pgfsetmiterjoin%
\definecolor{currentfill}{rgb}{0.501961,0.501961,0.501961}%
\pgfsetfillcolor{currentfill}%
\pgfsetfillopacity{0.700000}%
\pgfsetlinewidth{0.000000pt}%
\definecolor{currentstroke}{rgb}{0.000000,0.000000,0.000000}%
\pgfsetstrokecolor{currentstroke}%
\pgfsetstrokeopacity{0.700000}%
\pgfsetdash{}{0pt}%
\pgfpathmoveto{\pgfqpoint{3.473722in}{0.550000in}}%
\pgfpathlineto{\pgfqpoint{3.501243in}{0.550000in}}%
\pgfpathlineto{\pgfqpoint{3.501243in}{0.550000in}}%
\pgfpathlineto{\pgfqpoint{3.473722in}{0.550000in}}%
\pgfpathlineto{\pgfqpoint{3.473722in}{0.550000in}}%
\pgfpathclose%
\pgfusepath{fill}%
\end{pgfscope}%
\begin{pgfscope}%
\pgfpathrectangle{\pgfqpoint{1.250000in}{0.550000in}}{\pgfqpoint{7.750000in}{3.850000in}}%
\pgfusepath{clip}%
\pgfsetbuttcap%
\pgfsetmiterjoin%
\definecolor{currentfill}{rgb}{0.501961,0.501961,0.501961}%
\pgfsetfillcolor{currentfill}%
\pgfsetfillopacity{0.700000}%
\pgfsetlinewidth{0.000000pt}%
\definecolor{currentstroke}{rgb}{0.000000,0.000000,0.000000}%
\pgfsetstrokecolor{currentstroke}%
\pgfsetstrokeopacity{0.700000}%
\pgfsetdash{}{0pt}%
\pgfpathmoveto{\pgfqpoint{3.501243in}{0.550000in}}%
\pgfpathlineto{\pgfqpoint{3.528764in}{0.550000in}}%
\pgfpathlineto{\pgfqpoint{3.528764in}{0.550000in}}%
\pgfpathlineto{\pgfqpoint{3.501243in}{0.550000in}}%
\pgfpathlineto{\pgfqpoint{3.501243in}{0.550000in}}%
\pgfpathclose%
\pgfusepath{fill}%
\end{pgfscope}%
\begin{pgfscope}%
\pgfpathrectangle{\pgfqpoint{1.250000in}{0.550000in}}{\pgfqpoint{7.750000in}{3.850000in}}%
\pgfusepath{clip}%
\pgfsetbuttcap%
\pgfsetmiterjoin%
\definecolor{currentfill}{rgb}{0.501961,0.501961,0.501961}%
\pgfsetfillcolor{currentfill}%
\pgfsetfillopacity{0.700000}%
\pgfsetlinewidth{0.000000pt}%
\definecolor{currentstroke}{rgb}{0.000000,0.000000,0.000000}%
\pgfsetstrokecolor{currentstroke}%
\pgfsetstrokeopacity{0.700000}%
\pgfsetdash{}{0pt}%
\pgfpathmoveto{\pgfqpoint{3.528764in}{0.550000in}}%
\pgfpathlineto{\pgfqpoint{3.556286in}{0.550000in}}%
\pgfpathlineto{\pgfqpoint{3.556286in}{1.243110in}}%
\pgfpathlineto{\pgfqpoint{3.528764in}{1.243110in}}%
\pgfpathlineto{\pgfqpoint{3.528764in}{0.550000in}}%
\pgfpathclose%
\pgfusepath{fill}%
\end{pgfscope}%
\begin{pgfscope}%
\pgfpathrectangle{\pgfqpoint{1.250000in}{0.550000in}}{\pgfqpoint{7.750000in}{3.850000in}}%
\pgfusepath{clip}%
\pgfsetbuttcap%
\pgfsetmiterjoin%
\definecolor{currentfill}{rgb}{0.501961,0.501961,0.501961}%
\pgfsetfillcolor{currentfill}%
\pgfsetfillopacity{0.700000}%
\pgfsetlinewidth{0.000000pt}%
\definecolor{currentstroke}{rgb}{0.000000,0.000000,0.000000}%
\pgfsetstrokecolor{currentstroke}%
\pgfsetstrokeopacity{0.700000}%
\pgfsetdash{}{0pt}%
\pgfpathmoveto{\pgfqpoint{3.556286in}{0.550000in}}%
\pgfpathlineto{\pgfqpoint{3.583807in}{0.550000in}}%
\pgfpathlineto{\pgfqpoint{3.583807in}{0.550000in}}%
\pgfpathlineto{\pgfqpoint{3.556286in}{0.550000in}}%
\pgfpathlineto{\pgfqpoint{3.556286in}{0.550000in}}%
\pgfpathclose%
\pgfusepath{fill}%
\end{pgfscope}%
\begin{pgfscope}%
\pgfpathrectangle{\pgfqpoint{1.250000in}{0.550000in}}{\pgfqpoint{7.750000in}{3.850000in}}%
\pgfusepath{clip}%
\pgfsetbuttcap%
\pgfsetmiterjoin%
\definecolor{currentfill}{rgb}{0.501961,0.501961,0.501961}%
\pgfsetfillcolor{currentfill}%
\pgfsetfillopacity{0.700000}%
\pgfsetlinewidth{0.000000pt}%
\definecolor{currentstroke}{rgb}{0.000000,0.000000,0.000000}%
\pgfsetstrokecolor{currentstroke}%
\pgfsetstrokeopacity{0.700000}%
\pgfsetdash{}{0pt}%
\pgfpathmoveto{\pgfqpoint{3.583807in}{0.550000in}}%
\pgfpathlineto{\pgfqpoint{3.611328in}{0.550000in}}%
\pgfpathlineto{\pgfqpoint{3.611328in}{0.550000in}}%
\pgfpathlineto{\pgfqpoint{3.583807in}{0.550000in}}%
\pgfpathlineto{\pgfqpoint{3.583807in}{0.550000in}}%
\pgfpathclose%
\pgfusepath{fill}%
\end{pgfscope}%
\begin{pgfscope}%
\pgfpathrectangle{\pgfqpoint{1.250000in}{0.550000in}}{\pgfqpoint{7.750000in}{3.850000in}}%
\pgfusepath{clip}%
\pgfsetbuttcap%
\pgfsetmiterjoin%
\definecolor{currentfill}{rgb}{0.501961,0.501961,0.501961}%
\pgfsetfillcolor{currentfill}%
\pgfsetfillopacity{0.700000}%
\pgfsetlinewidth{0.000000pt}%
\definecolor{currentstroke}{rgb}{0.000000,0.000000,0.000000}%
\pgfsetstrokecolor{currentstroke}%
\pgfsetstrokeopacity{0.700000}%
\pgfsetdash{}{0pt}%
\pgfpathmoveto{\pgfqpoint{3.611328in}{0.550000in}}%
\pgfpathlineto{\pgfqpoint{3.638849in}{0.550000in}}%
\pgfpathlineto{\pgfqpoint{3.638849in}{0.550000in}}%
\pgfpathlineto{\pgfqpoint{3.611328in}{0.550000in}}%
\pgfpathlineto{\pgfqpoint{3.611328in}{0.550000in}}%
\pgfpathclose%
\pgfusepath{fill}%
\end{pgfscope}%
\begin{pgfscope}%
\pgfpathrectangle{\pgfqpoint{1.250000in}{0.550000in}}{\pgfqpoint{7.750000in}{3.850000in}}%
\pgfusepath{clip}%
\pgfsetbuttcap%
\pgfsetmiterjoin%
\definecolor{currentfill}{rgb}{0.501961,0.501961,0.501961}%
\pgfsetfillcolor{currentfill}%
\pgfsetfillopacity{0.700000}%
\pgfsetlinewidth{0.000000pt}%
\definecolor{currentstroke}{rgb}{0.000000,0.000000,0.000000}%
\pgfsetstrokecolor{currentstroke}%
\pgfsetstrokeopacity{0.700000}%
\pgfsetdash{}{0pt}%
\pgfpathmoveto{\pgfqpoint{3.638849in}{0.550000in}}%
\pgfpathlineto{\pgfqpoint{3.666371in}{0.550000in}}%
\pgfpathlineto{\pgfqpoint{3.666371in}{1.179454in}}%
\pgfpathlineto{\pgfqpoint{3.638849in}{1.179454in}}%
\pgfpathlineto{\pgfqpoint{3.638849in}{0.550000in}}%
\pgfpathclose%
\pgfusepath{fill}%
\end{pgfscope}%
\begin{pgfscope}%
\pgfpathrectangle{\pgfqpoint{1.250000in}{0.550000in}}{\pgfqpoint{7.750000in}{3.850000in}}%
\pgfusepath{clip}%
\pgfsetbuttcap%
\pgfsetmiterjoin%
\definecolor{currentfill}{rgb}{0.501961,0.501961,0.501961}%
\pgfsetfillcolor{currentfill}%
\pgfsetfillopacity{0.700000}%
\pgfsetlinewidth{0.000000pt}%
\definecolor{currentstroke}{rgb}{0.000000,0.000000,0.000000}%
\pgfsetstrokecolor{currentstroke}%
\pgfsetstrokeopacity{0.700000}%
\pgfsetdash{}{0pt}%
\pgfpathmoveto{\pgfqpoint{3.666371in}{0.550000in}}%
\pgfpathlineto{\pgfqpoint{3.693892in}{0.550000in}}%
\pgfpathlineto{\pgfqpoint{3.693892in}{0.550000in}}%
\pgfpathlineto{\pgfqpoint{3.666371in}{0.550000in}}%
\pgfpathlineto{\pgfqpoint{3.666371in}{0.550000in}}%
\pgfpathclose%
\pgfusepath{fill}%
\end{pgfscope}%
\begin{pgfscope}%
\pgfpathrectangle{\pgfqpoint{1.250000in}{0.550000in}}{\pgfqpoint{7.750000in}{3.850000in}}%
\pgfusepath{clip}%
\pgfsetbuttcap%
\pgfsetmiterjoin%
\definecolor{currentfill}{rgb}{0.501961,0.501961,0.501961}%
\pgfsetfillcolor{currentfill}%
\pgfsetfillopacity{0.700000}%
\pgfsetlinewidth{0.000000pt}%
\definecolor{currentstroke}{rgb}{0.000000,0.000000,0.000000}%
\pgfsetstrokecolor{currentstroke}%
\pgfsetstrokeopacity{0.700000}%
\pgfsetdash{}{0pt}%
\pgfpathmoveto{\pgfqpoint{3.693892in}{0.550000in}}%
\pgfpathlineto{\pgfqpoint{3.721413in}{0.550000in}}%
\pgfpathlineto{\pgfqpoint{3.721413in}{0.550000in}}%
\pgfpathlineto{\pgfqpoint{3.693892in}{0.550000in}}%
\pgfpathlineto{\pgfqpoint{3.693892in}{0.550000in}}%
\pgfpathclose%
\pgfusepath{fill}%
\end{pgfscope}%
\begin{pgfscope}%
\pgfpathrectangle{\pgfqpoint{1.250000in}{0.550000in}}{\pgfqpoint{7.750000in}{3.850000in}}%
\pgfusepath{clip}%
\pgfsetbuttcap%
\pgfsetmiterjoin%
\definecolor{currentfill}{rgb}{0.501961,0.501961,0.501961}%
\pgfsetfillcolor{currentfill}%
\pgfsetfillopacity{0.700000}%
\pgfsetlinewidth{0.000000pt}%
\definecolor{currentstroke}{rgb}{0.000000,0.000000,0.000000}%
\pgfsetstrokecolor{currentstroke}%
\pgfsetstrokeopacity{0.700000}%
\pgfsetdash{}{0pt}%
\pgfpathmoveto{\pgfqpoint{3.721413in}{0.550000in}}%
\pgfpathlineto{\pgfqpoint{3.748935in}{0.550000in}}%
\pgfpathlineto{\pgfqpoint{3.748935in}{0.550000in}}%
\pgfpathlineto{\pgfqpoint{3.721413in}{0.550000in}}%
\pgfpathlineto{\pgfqpoint{3.721413in}{0.550000in}}%
\pgfpathclose%
\pgfusepath{fill}%
\end{pgfscope}%
\begin{pgfscope}%
\pgfpathrectangle{\pgfqpoint{1.250000in}{0.550000in}}{\pgfqpoint{7.750000in}{3.850000in}}%
\pgfusepath{clip}%
\pgfsetbuttcap%
\pgfsetmiterjoin%
\definecolor{currentfill}{rgb}{0.501961,0.501961,0.501961}%
\pgfsetfillcolor{currentfill}%
\pgfsetfillopacity{0.700000}%
\pgfsetlinewidth{0.000000pt}%
\definecolor{currentstroke}{rgb}{0.000000,0.000000,0.000000}%
\pgfsetstrokecolor{currentstroke}%
\pgfsetstrokeopacity{0.700000}%
\pgfsetdash{}{0pt}%
\pgfpathmoveto{\pgfqpoint{3.748935in}{0.550000in}}%
\pgfpathlineto{\pgfqpoint{3.776456in}{0.550000in}}%
\pgfpathlineto{\pgfqpoint{3.776456in}{1.167274in}}%
\pgfpathlineto{\pgfqpoint{3.748935in}{1.167274in}}%
\pgfpathlineto{\pgfqpoint{3.748935in}{0.550000in}}%
\pgfpathclose%
\pgfusepath{fill}%
\end{pgfscope}%
\begin{pgfscope}%
\pgfpathrectangle{\pgfqpoint{1.250000in}{0.550000in}}{\pgfqpoint{7.750000in}{3.850000in}}%
\pgfusepath{clip}%
\pgfsetbuttcap%
\pgfsetmiterjoin%
\definecolor{currentfill}{rgb}{0.501961,0.501961,0.501961}%
\pgfsetfillcolor{currentfill}%
\pgfsetfillopacity{0.700000}%
\pgfsetlinewidth{0.000000pt}%
\definecolor{currentstroke}{rgb}{0.000000,0.000000,0.000000}%
\pgfsetstrokecolor{currentstroke}%
\pgfsetstrokeopacity{0.700000}%
\pgfsetdash{}{0pt}%
\pgfpathmoveto{\pgfqpoint{3.776456in}{0.550000in}}%
\pgfpathlineto{\pgfqpoint{3.803977in}{0.550000in}}%
\pgfpathlineto{\pgfqpoint{3.803977in}{0.550000in}}%
\pgfpathlineto{\pgfqpoint{3.776456in}{0.550000in}}%
\pgfpathlineto{\pgfqpoint{3.776456in}{0.550000in}}%
\pgfpathclose%
\pgfusepath{fill}%
\end{pgfscope}%
\begin{pgfscope}%
\pgfpathrectangle{\pgfqpoint{1.250000in}{0.550000in}}{\pgfqpoint{7.750000in}{3.850000in}}%
\pgfusepath{clip}%
\pgfsetbuttcap%
\pgfsetmiterjoin%
\definecolor{currentfill}{rgb}{0.501961,0.501961,0.501961}%
\pgfsetfillcolor{currentfill}%
\pgfsetfillopacity{0.700000}%
\pgfsetlinewidth{0.000000pt}%
\definecolor{currentstroke}{rgb}{0.000000,0.000000,0.000000}%
\pgfsetstrokecolor{currentstroke}%
\pgfsetstrokeopacity{0.700000}%
\pgfsetdash{}{0pt}%
\pgfpathmoveto{\pgfqpoint{3.803977in}{0.550000in}}%
\pgfpathlineto{\pgfqpoint{3.831499in}{0.550000in}}%
\pgfpathlineto{\pgfqpoint{3.831499in}{0.550000in}}%
\pgfpathlineto{\pgfqpoint{3.803977in}{0.550000in}}%
\pgfpathlineto{\pgfqpoint{3.803977in}{0.550000in}}%
\pgfpathclose%
\pgfusepath{fill}%
\end{pgfscope}%
\begin{pgfscope}%
\pgfpathrectangle{\pgfqpoint{1.250000in}{0.550000in}}{\pgfqpoint{7.750000in}{3.850000in}}%
\pgfusepath{clip}%
\pgfsetbuttcap%
\pgfsetmiterjoin%
\definecolor{currentfill}{rgb}{0.501961,0.501961,0.501961}%
\pgfsetfillcolor{currentfill}%
\pgfsetfillopacity{0.700000}%
\pgfsetlinewidth{0.000000pt}%
\definecolor{currentstroke}{rgb}{0.000000,0.000000,0.000000}%
\pgfsetstrokecolor{currentstroke}%
\pgfsetstrokeopacity{0.700000}%
\pgfsetdash{}{0pt}%
\pgfpathmoveto{\pgfqpoint{3.831499in}{0.550000in}}%
\pgfpathlineto{\pgfqpoint{3.859020in}{0.550000in}}%
\pgfpathlineto{\pgfqpoint{3.859020in}{1.126383in}}%
\pgfpathlineto{\pgfqpoint{3.831499in}{1.126383in}}%
\pgfpathlineto{\pgfqpoint{3.831499in}{0.550000in}}%
\pgfpathclose%
\pgfusepath{fill}%
\end{pgfscope}%
\begin{pgfscope}%
\pgfpathrectangle{\pgfqpoint{1.250000in}{0.550000in}}{\pgfqpoint{7.750000in}{3.850000in}}%
\pgfusepath{clip}%
\pgfsetbuttcap%
\pgfsetmiterjoin%
\definecolor{currentfill}{rgb}{0.501961,0.501961,0.501961}%
\pgfsetfillcolor{currentfill}%
\pgfsetfillopacity{0.700000}%
\pgfsetlinewidth{0.000000pt}%
\definecolor{currentstroke}{rgb}{0.000000,0.000000,0.000000}%
\pgfsetstrokecolor{currentstroke}%
\pgfsetstrokeopacity{0.700000}%
\pgfsetdash{}{0pt}%
\pgfpathmoveto{\pgfqpoint{3.859020in}{0.550000in}}%
\pgfpathlineto{\pgfqpoint{3.886541in}{0.550000in}}%
\pgfpathlineto{\pgfqpoint{3.886541in}{0.550000in}}%
\pgfpathlineto{\pgfqpoint{3.859020in}{0.550000in}}%
\pgfpathlineto{\pgfqpoint{3.859020in}{0.550000in}}%
\pgfpathclose%
\pgfusepath{fill}%
\end{pgfscope}%
\begin{pgfscope}%
\pgfpathrectangle{\pgfqpoint{1.250000in}{0.550000in}}{\pgfqpoint{7.750000in}{3.850000in}}%
\pgfusepath{clip}%
\pgfsetbuttcap%
\pgfsetmiterjoin%
\definecolor{currentfill}{rgb}{0.501961,0.501961,0.501961}%
\pgfsetfillcolor{currentfill}%
\pgfsetfillopacity{0.700000}%
\pgfsetlinewidth{0.000000pt}%
\definecolor{currentstroke}{rgb}{0.000000,0.000000,0.000000}%
\pgfsetstrokecolor{currentstroke}%
\pgfsetstrokeopacity{0.700000}%
\pgfsetdash{}{0pt}%
\pgfpathmoveto{\pgfqpoint{3.886541in}{0.550000in}}%
\pgfpathlineto{\pgfqpoint{3.914062in}{0.550000in}}%
\pgfpathlineto{\pgfqpoint{3.914062in}{0.550000in}}%
\pgfpathlineto{\pgfqpoint{3.886541in}{0.550000in}}%
\pgfpathlineto{\pgfqpoint{3.886541in}{0.550000in}}%
\pgfpathclose%
\pgfusepath{fill}%
\end{pgfscope}%
\begin{pgfscope}%
\pgfpathrectangle{\pgfqpoint{1.250000in}{0.550000in}}{\pgfqpoint{7.750000in}{3.850000in}}%
\pgfusepath{clip}%
\pgfsetbuttcap%
\pgfsetmiterjoin%
\definecolor{currentfill}{rgb}{0.501961,0.501961,0.501961}%
\pgfsetfillcolor{currentfill}%
\pgfsetfillopacity{0.700000}%
\pgfsetlinewidth{0.000000pt}%
\definecolor{currentstroke}{rgb}{0.000000,0.000000,0.000000}%
\pgfsetstrokecolor{currentstroke}%
\pgfsetstrokeopacity{0.700000}%
\pgfsetdash{}{0pt}%
\pgfpathmoveto{\pgfqpoint{3.914062in}{0.550000in}}%
\pgfpathlineto{\pgfqpoint{3.941584in}{0.550000in}}%
\pgfpathlineto{\pgfqpoint{3.941584in}{0.550000in}}%
\pgfpathlineto{\pgfqpoint{3.914062in}{0.550000in}}%
\pgfpathlineto{\pgfqpoint{3.914062in}{0.550000in}}%
\pgfpathclose%
\pgfusepath{fill}%
\end{pgfscope}%
\begin{pgfscope}%
\pgfpathrectangle{\pgfqpoint{1.250000in}{0.550000in}}{\pgfqpoint{7.750000in}{3.850000in}}%
\pgfusepath{clip}%
\pgfsetbuttcap%
\pgfsetmiterjoin%
\definecolor{currentfill}{rgb}{0.501961,0.501961,0.501961}%
\pgfsetfillcolor{currentfill}%
\pgfsetfillopacity{0.700000}%
\pgfsetlinewidth{0.000000pt}%
\definecolor{currentstroke}{rgb}{0.000000,0.000000,0.000000}%
\pgfsetstrokecolor{currentstroke}%
\pgfsetstrokeopacity{0.700000}%
\pgfsetdash{}{0pt}%
\pgfpathmoveto{\pgfqpoint{3.941584in}{0.550000in}}%
\pgfpathlineto{\pgfqpoint{3.969105in}{0.550000in}}%
\pgfpathlineto{\pgfqpoint{3.969105in}{1.112463in}}%
\pgfpathlineto{\pgfqpoint{3.941584in}{1.112463in}}%
\pgfpathlineto{\pgfqpoint{3.941584in}{0.550000in}}%
\pgfpathclose%
\pgfusepath{fill}%
\end{pgfscope}%
\begin{pgfscope}%
\pgfpathrectangle{\pgfqpoint{1.250000in}{0.550000in}}{\pgfqpoint{7.750000in}{3.850000in}}%
\pgfusepath{clip}%
\pgfsetbuttcap%
\pgfsetmiterjoin%
\definecolor{currentfill}{rgb}{0.501961,0.501961,0.501961}%
\pgfsetfillcolor{currentfill}%
\pgfsetfillopacity{0.700000}%
\pgfsetlinewidth{0.000000pt}%
\definecolor{currentstroke}{rgb}{0.000000,0.000000,0.000000}%
\pgfsetstrokecolor{currentstroke}%
\pgfsetstrokeopacity{0.700000}%
\pgfsetdash{}{0pt}%
\pgfpathmoveto{\pgfqpoint{3.969105in}{0.550000in}}%
\pgfpathlineto{\pgfqpoint{3.996626in}{0.550000in}}%
\pgfpathlineto{\pgfqpoint{3.996626in}{0.550000in}}%
\pgfpathlineto{\pgfqpoint{3.969105in}{0.550000in}}%
\pgfpathlineto{\pgfqpoint{3.969105in}{0.550000in}}%
\pgfpathclose%
\pgfusepath{fill}%
\end{pgfscope}%
\begin{pgfscope}%
\pgfpathrectangle{\pgfqpoint{1.250000in}{0.550000in}}{\pgfqpoint{7.750000in}{3.850000in}}%
\pgfusepath{clip}%
\pgfsetbuttcap%
\pgfsetmiterjoin%
\definecolor{currentfill}{rgb}{0.501961,0.501961,0.501961}%
\pgfsetfillcolor{currentfill}%
\pgfsetfillopacity{0.700000}%
\pgfsetlinewidth{0.000000pt}%
\definecolor{currentstroke}{rgb}{0.000000,0.000000,0.000000}%
\pgfsetstrokecolor{currentstroke}%
\pgfsetstrokeopacity{0.700000}%
\pgfsetdash{}{0pt}%
\pgfpathmoveto{\pgfqpoint{3.996626in}{0.550000in}}%
\pgfpathlineto{\pgfqpoint{4.024148in}{0.550000in}}%
\pgfpathlineto{\pgfqpoint{4.024148in}{0.550000in}}%
\pgfpathlineto{\pgfqpoint{3.996626in}{0.550000in}}%
\pgfpathlineto{\pgfqpoint{3.996626in}{0.550000in}}%
\pgfpathclose%
\pgfusepath{fill}%
\end{pgfscope}%
\begin{pgfscope}%
\pgfpathrectangle{\pgfqpoint{1.250000in}{0.550000in}}{\pgfqpoint{7.750000in}{3.850000in}}%
\pgfusepath{clip}%
\pgfsetbuttcap%
\pgfsetmiterjoin%
\definecolor{currentfill}{rgb}{0.501961,0.501961,0.501961}%
\pgfsetfillcolor{currentfill}%
\pgfsetfillopacity{0.700000}%
\pgfsetlinewidth{0.000000pt}%
\definecolor{currentstroke}{rgb}{0.000000,0.000000,0.000000}%
\pgfsetstrokecolor{currentstroke}%
\pgfsetstrokeopacity{0.700000}%
\pgfsetdash{}{0pt}%
\pgfpathmoveto{\pgfqpoint{4.024148in}{0.550000in}}%
\pgfpathlineto{\pgfqpoint{4.051669in}{0.550000in}}%
\pgfpathlineto{\pgfqpoint{4.051669in}{1.099848in}}%
\pgfpathlineto{\pgfqpoint{4.024148in}{1.099848in}}%
\pgfpathlineto{\pgfqpoint{4.024148in}{0.550000in}}%
\pgfpathclose%
\pgfusepath{fill}%
\end{pgfscope}%
\begin{pgfscope}%
\pgfpathrectangle{\pgfqpoint{1.250000in}{0.550000in}}{\pgfqpoint{7.750000in}{3.850000in}}%
\pgfusepath{clip}%
\pgfsetbuttcap%
\pgfsetmiterjoin%
\definecolor{currentfill}{rgb}{0.501961,0.501961,0.501961}%
\pgfsetfillcolor{currentfill}%
\pgfsetfillopacity{0.700000}%
\pgfsetlinewidth{0.000000pt}%
\definecolor{currentstroke}{rgb}{0.000000,0.000000,0.000000}%
\pgfsetstrokecolor{currentstroke}%
\pgfsetstrokeopacity{0.700000}%
\pgfsetdash{}{0pt}%
\pgfpathmoveto{\pgfqpoint{4.051669in}{0.550000in}}%
\pgfpathlineto{\pgfqpoint{4.079190in}{0.550000in}}%
\pgfpathlineto{\pgfqpoint{4.079190in}{0.550000in}}%
\pgfpathlineto{\pgfqpoint{4.051669in}{0.550000in}}%
\pgfpathlineto{\pgfqpoint{4.051669in}{0.550000in}}%
\pgfpathclose%
\pgfusepath{fill}%
\end{pgfscope}%
\begin{pgfscope}%
\pgfpathrectangle{\pgfqpoint{1.250000in}{0.550000in}}{\pgfqpoint{7.750000in}{3.850000in}}%
\pgfusepath{clip}%
\pgfsetbuttcap%
\pgfsetmiterjoin%
\definecolor{currentfill}{rgb}{0.501961,0.501961,0.501961}%
\pgfsetfillcolor{currentfill}%
\pgfsetfillopacity{0.700000}%
\pgfsetlinewidth{0.000000pt}%
\definecolor{currentstroke}{rgb}{0.000000,0.000000,0.000000}%
\pgfsetstrokecolor{currentstroke}%
\pgfsetstrokeopacity{0.700000}%
\pgfsetdash{}{0pt}%
\pgfpathmoveto{\pgfqpoint{4.079190in}{0.550000in}}%
\pgfpathlineto{\pgfqpoint{4.106712in}{0.550000in}}%
\pgfpathlineto{\pgfqpoint{4.106712in}{0.550000in}}%
\pgfpathlineto{\pgfqpoint{4.079190in}{0.550000in}}%
\pgfpathlineto{\pgfqpoint{4.079190in}{0.550000in}}%
\pgfpathclose%
\pgfusepath{fill}%
\end{pgfscope}%
\begin{pgfscope}%
\pgfpathrectangle{\pgfqpoint{1.250000in}{0.550000in}}{\pgfqpoint{7.750000in}{3.850000in}}%
\pgfusepath{clip}%
\pgfsetbuttcap%
\pgfsetmiterjoin%
\definecolor{currentfill}{rgb}{0.501961,0.501961,0.501961}%
\pgfsetfillcolor{currentfill}%
\pgfsetfillopacity{0.700000}%
\pgfsetlinewidth{0.000000pt}%
\definecolor{currentstroke}{rgb}{0.000000,0.000000,0.000000}%
\pgfsetstrokecolor{currentstroke}%
\pgfsetstrokeopacity{0.700000}%
\pgfsetdash{}{0pt}%
\pgfpathmoveto{\pgfqpoint{4.106712in}{0.550000in}}%
\pgfpathlineto{\pgfqpoint{4.134233in}{0.550000in}}%
\pgfpathlineto{\pgfqpoint{4.134233in}{0.550000in}}%
\pgfpathlineto{\pgfqpoint{4.106712in}{0.550000in}}%
\pgfpathlineto{\pgfqpoint{4.106712in}{0.550000in}}%
\pgfpathclose%
\pgfusepath{fill}%
\end{pgfscope}%
\begin{pgfscope}%
\pgfpathrectangle{\pgfqpoint{1.250000in}{0.550000in}}{\pgfqpoint{7.750000in}{3.850000in}}%
\pgfusepath{clip}%
\pgfsetbuttcap%
\pgfsetmiterjoin%
\definecolor{currentfill}{rgb}{0.501961,0.501961,0.501961}%
\pgfsetfillcolor{currentfill}%
\pgfsetfillopacity{0.700000}%
\pgfsetlinewidth{0.000000pt}%
\definecolor{currentstroke}{rgb}{0.000000,0.000000,0.000000}%
\pgfsetstrokecolor{currentstroke}%
\pgfsetstrokeopacity{0.700000}%
\pgfsetdash{}{0pt}%
\pgfpathmoveto{\pgfqpoint{4.134233in}{0.550000in}}%
\pgfpathlineto{\pgfqpoint{4.161754in}{0.550000in}}%
\pgfpathlineto{\pgfqpoint{4.161754in}{1.064902in}}%
\pgfpathlineto{\pgfqpoint{4.134233in}{1.064902in}}%
\pgfpathlineto{\pgfqpoint{4.134233in}{0.550000in}}%
\pgfpathclose%
\pgfusepath{fill}%
\end{pgfscope}%
\begin{pgfscope}%
\pgfpathrectangle{\pgfqpoint{1.250000in}{0.550000in}}{\pgfqpoint{7.750000in}{3.850000in}}%
\pgfusepath{clip}%
\pgfsetbuttcap%
\pgfsetmiterjoin%
\definecolor{currentfill}{rgb}{0.501961,0.501961,0.501961}%
\pgfsetfillcolor{currentfill}%
\pgfsetfillopacity{0.700000}%
\pgfsetlinewidth{0.000000pt}%
\definecolor{currentstroke}{rgb}{0.000000,0.000000,0.000000}%
\pgfsetstrokecolor{currentstroke}%
\pgfsetstrokeopacity{0.700000}%
\pgfsetdash{}{0pt}%
\pgfpathmoveto{\pgfqpoint{4.161754in}{0.550000in}}%
\pgfpathlineto{\pgfqpoint{4.189276in}{0.550000in}}%
\pgfpathlineto{\pgfqpoint{4.189276in}{0.550000in}}%
\pgfpathlineto{\pgfqpoint{4.161754in}{0.550000in}}%
\pgfpathlineto{\pgfqpoint{4.161754in}{0.550000in}}%
\pgfpathclose%
\pgfusepath{fill}%
\end{pgfscope}%
\begin{pgfscope}%
\pgfpathrectangle{\pgfqpoint{1.250000in}{0.550000in}}{\pgfqpoint{7.750000in}{3.850000in}}%
\pgfusepath{clip}%
\pgfsetbuttcap%
\pgfsetmiterjoin%
\definecolor{currentfill}{rgb}{0.501961,0.501961,0.501961}%
\pgfsetfillcolor{currentfill}%
\pgfsetfillopacity{0.700000}%
\pgfsetlinewidth{0.000000pt}%
\definecolor{currentstroke}{rgb}{0.000000,0.000000,0.000000}%
\pgfsetstrokecolor{currentstroke}%
\pgfsetstrokeopacity{0.700000}%
\pgfsetdash{}{0pt}%
\pgfpathmoveto{\pgfqpoint{4.189276in}{0.550000in}}%
\pgfpathlineto{\pgfqpoint{4.216797in}{0.550000in}}%
\pgfpathlineto{\pgfqpoint{4.216797in}{0.550000in}}%
\pgfpathlineto{\pgfqpoint{4.189276in}{0.550000in}}%
\pgfpathlineto{\pgfqpoint{4.189276in}{0.550000in}}%
\pgfpathclose%
\pgfusepath{fill}%
\end{pgfscope}%
\begin{pgfscope}%
\pgfpathrectangle{\pgfqpoint{1.250000in}{0.550000in}}{\pgfqpoint{7.750000in}{3.850000in}}%
\pgfusepath{clip}%
\pgfsetbuttcap%
\pgfsetmiterjoin%
\definecolor{currentfill}{rgb}{0.501961,0.501961,0.501961}%
\pgfsetfillcolor{currentfill}%
\pgfsetfillopacity{0.700000}%
\pgfsetlinewidth{0.000000pt}%
\definecolor{currentstroke}{rgb}{0.000000,0.000000,0.000000}%
\pgfsetstrokecolor{currentstroke}%
\pgfsetstrokeopacity{0.700000}%
\pgfsetdash{}{0pt}%
\pgfpathmoveto{\pgfqpoint{4.216797in}{0.550000in}}%
\pgfpathlineto{\pgfqpoint{4.244318in}{0.550000in}}%
\pgfpathlineto{\pgfqpoint{4.244318in}{1.074327in}}%
\pgfpathlineto{\pgfqpoint{4.216797in}{1.074327in}}%
\pgfpathlineto{\pgfqpoint{4.216797in}{0.550000in}}%
\pgfpathclose%
\pgfusepath{fill}%
\end{pgfscope}%
\begin{pgfscope}%
\pgfpathrectangle{\pgfqpoint{1.250000in}{0.550000in}}{\pgfqpoint{7.750000in}{3.850000in}}%
\pgfusepath{clip}%
\pgfsetbuttcap%
\pgfsetmiterjoin%
\definecolor{currentfill}{rgb}{0.501961,0.501961,0.501961}%
\pgfsetfillcolor{currentfill}%
\pgfsetfillopacity{0.700000}%
\pgfsetlinewidth{0.000000pt}%
\definecolor{currentstroke}{rgb}{0.000000,0.000000,0.000000}%
\pgfsetstrokecolor{currentstroke}%
\pgfsetstrokeopacity{0.700000}%
\pgfsetdash{}{0pt}%
\pgfpathmoveto{\pgfqpoint{4.244318in}{0.550000in}}%
\pgfpathlineto{\pgfqpoint{4.271839in}{0.550000in}}%
\pgfpathlineto{\pgfqpoint{4.271839in}{0.550000in}}%
\pgfpathlineto{\pgfqpoint{4.244318in}{0.550000in}}%
\pgfpathlineto{\pgfqpoint{4.244318in}{0.550000in}}%
\pgfpathclose%
\pgfusepath{fill}%
\end{pgfscope}%
\begin{pgfscope}%
\pgfpathrectangle{\pgfqpoint{1.250000in}{0.550000in}}{\pgfqpoint{7.750000in}{3.850000in}}%
\pgfusepath{clip}%
\pgfsetbuttcap%
\pgfsetmiterjoin%
\definecolor{currentfill}{rgb}{0.501961,0.501961,0.501961}%
\pgfsetfillcolor{currentfill}%
\pgfsetfillopacity{0.700000}%
\pgfsetlinewidth{0.000000pt}%
\definecolor{currentstroke}{rgb}{0.000000,0.000000,0.000000}%
\pgfsetstrokecolor{currentstroke}%
\pgfsetstrokeopacity{0.700000}%
\pgfsetdash{}{0pt}%
\pgfpathmoveto{\pgfqpoint{4.271839in}{0.550000in}}%
\pgfpathlineto{\pgfqpoint{4.299361in}{0.550000in}}%
\pgfpathlineto{\pgfqpoint{4.299361in}{0.550000in}}%
\pgfpathlineto{\pgfqpoint{4.271839in}{0.550000in}}%
\pgfpathlineto{\pgfqpoint{4.271839in}{0.550000in}}%
\pgfpathclose%
\pgfusepath{fill}%
\end{pgfscope}%
\begin{pgfscope}%
\pgfpathrectangle{\pgfqpoint{1.250000in}{0.550000in}}{\pgfqpoint{7.750000in}{3.850000in}}%
\pgfusepath{clip}%
\pgfsetbuttcap%
\pgfsetmiterjoin%
\definecolor{currentfill}{rgb}{0.501961,0.501961,0.501961}%
\pgfsetfillcolor{currentfill}%
\pgfsetfillopacity{0.700000}%
\pgfsetlinewidth{0.000000pt}%
\definecolor{currentstroke}{rgb}{0.000000,0.000000,0.000000}%
\pgfsetstrokecolor{currentstroke}%
\pgfsetstrokeopacity{0.700000}%
\pgfsetdash{}{0pt}%
\pgfpathmoveto{\pgfqpoint{4.299361in}{0.550000in}}%
\pgfpathlineto{\pgfqpoint{4.326882in}{0.550000in}}%
\pgfpathlineto{\pgfqpoint{4.326882in}{1.044312in}}%
\pgfpathlineto{\pgfqpoint{4.299361in}{1.044312in}}%
\pgfpathlineto{\pgfqpoint{4.299361in}{0.550000in}}%
\pgfpathclose%
\pgfusepath{fill}%
\end{pgfscope}%
\begin{pgfscope}%
\pgfpathrectangle{\pgfqpoint{1.250000in}{0.550000in}}{\pgfqpoint{7.750000in}{3.850000in}}%
\pgfusepath{clip}%
\pgfsetbuttcap%
\pgfsetmiterjoin%
\definecolor{currentfill}{rgb}{0.501961,0.501961,0.501961}%
\pgfsetfillcolor{currentfill}%
\pgfsetfillopacity{0.700000}%
\pgfsetlinewidth{0.000000pt}%
\definecolor{currentstroke}{rgb}{0.000000,0.000000,0.000000}%
\pgfsetstrokecolor{currentstroke}%
\pgfsetstrokeopacity{0.700000}%
\pgfsetdash{}{0pt}%
\pgfpathmoveto{\pgfqpoint{4.326882in}{0.550000in}}%
\pgfpathlineto{\pgfqpoint{4.354403in}{0.550000in}}%
\pgfpathlineto{\pgfqpoint{4.354403in}{0.550000in}}%
\pgfpathlineto{\pgfqpoint{4.326882in}{0.550000in}}%
\pgfpathlineto{\pgfqpoint{4.326882in}{0.550000in}}%
\pgfpathclose%
\pgfusepath{fill}%
\end{pgfscope}%
\begin{pgfscope}%
\pgfpathrectangle{\pgfqpoint{1.250000in}{0.550000in}}{\pgfqpoint{7.750000in}{3.850000in}}%
\pgfusepath{clip}%
\pgfsetbuttcap%
\pgfsetmiterjoin%
\definecolor{currentfill}{rgb}{0.501961,0.501961,0.501961}%
\pgfsetfillcolor{currentfill}%
\pgfsetfillopacity{0.700000}%
\pgfsetlinewidth{0.000000pt}%
\definecolor{currentstroke}{rgb}{0.000000,0.000000,0.000000}%
\pgfsetstrokecolor{currentstroke}%
\pgfsetstrokeopacity{0.700000}%
\pgfsetdash{}{0pt}%
\pgfpathmoveto{\pgfqpoint{4.354403in}{0.550000in}}%
\pgfpathlineto{\pgfqpoint{4.381925in}{0.550000in}}%
\pgfpathlineto{\pgfqpoint{4.381925in}{0.550000in}}%
\pgfpathlineto{\pgfqpoint{4.354403in}{0.550000in}}%
\pgfpathlineto{\pgfqpoint{4.354403in}{0.550000in}}%
\pgfpathclose%
\pgfusepath{fill}%
\end{pgfscope}%
\begin{pgfscope}%
\pgfpathrectangle{\pgfqpoint{1.250000in}{0.550000in}}{\pgfqpoint{7.750000in}{3.850000in}}%
\pgfusepath{clip}%
\pgfsetbuttcap%
\pgfsetmiterjoin%
\definecolor{currentfill}{rgb}{0.501961,0.501961,0.501961}%
\pgfsetfillcolor{currentfill}%
\pgfsetfillopacity{0.700000}%
\pgfsetlinewidth{0.000000pt}%
\definecolor{currentstroke}{rgb}{0.000000,0.000000,0.000000}%
\pgfsetstrokecolor{currentstroke}%
\pgfsetstrokeopacity{0.700000}%
\pgfsetdash{}{0pt}%
\pgfpathmoveto{\pgfqpoint{4.381925in}{0.550000in}}%
\pgfpathlineto{\pgfqpoint{4.409446in}{0.550000in}}%
\pgfpathlineto{\pgfqpoint{4.409446in}{1.041267in}}%
\pgfpathlineto{\pgfqpoint{4.381925in}{1.041267in}}%
\pgfpathlineto{\pgfqpoint{4.381925in}{0.550000in}}%
\pgfpathclose%
\pgfusepath{fill}%
\end{pgfscope}%
\begin{pgfscope}%
\pgfpathrectangle{\pgfqpoint{1.250000in}{0.550000in}}{\pgfqpoint{7.750000in}{3.850000in}}%
\pgfusepath{clip}%
\pgfsetbuttcap%
\pgfsetmiterjoin%
\definecolor{currentfill}{rgb}{0.501961,0.501961,0.501961}%
\pgfsetfillcolor{currentfill}%
\pgfsetfillopacity{0.700000}%
\pgfsetlinewidth{0.000000pt}%
\definecolor{currentstroke}{rgb}{0.000000,0.000000,0.000000}%
\pgfsetstrokecolor{currentstroke}%
\pgfsetstrokeopacity{0.700000}%
\pgfsetdash{}{0pt}%
\pgfpathmoveto{\pgfqpoint{4.409446in}{0.550000in}}%
\pgfpathlineto{\pgfqpoint{4.436967in}{0.550000in}}%
\pgfpathlineto{\pgfqpoint{4.436967in}{0.550000in}}%
\pgfpathlineto{\pgfqpoint{4.409446in}{0.550000in}}%
\pgfpathlineto{\pgfqpoint{4.409446in}{0.550000in}}%
\pgfpathclose%
\pgfusepath{fill}%
\end{pgfscope}%
\begin{pgfscope}%
\pgfpathrectangle{\pgfqpoint{1.250000in}{0.550000in}}{\pgfqpoint{7.750000in}{3.850000in}}%
\pgfusepath{clip}%
\pgfsetbuttcap%
\pgfsetmiterjoin%
\definecolor{currentfill}{rgb}{0.501961,0.501961,0.501961}%
\pgfsetfillcolor{currentfill}%
\pgfsetfillopacity{0.700000}%
\pgfsetlinewidth{0.000000pt}%
\definecolor{currentstroke}{rgb}{0.000000,0.000000,0.000000}%
\pgfsetstrokecolor{currentstroke}%
\pgfsetstrokeopacity{0.700000}%
\pgfsetdash{}{0pt}%
\pgfpathmoveto{\pgfqpoint{4.436967in}{0.550000in}}%
\pgfpathlineto{\pgfqpoint{4.464489in}{0.550000in}}%
\pgfpathlineto{\pgfqpoint{4.464489in}{0.550000in}}%
\pgfpathlineto{\pgfqpoint{4.436967in}{0.550000in}}%
\pgfpathlineto{\pgfqpoint{4.436967in}{0.550000in}}%
\pgfpathclose%
\pgfusepath{fill}%
\end{pgfscope}%
\begin{pgfscope}%
\pgfpathrectangle{\pgfqpoint{1.250000in}{0.550000in}}{\pgfqpoint{7.750000in}{3.850000in}}%
\pgfusepath{clip}%
\pgfsetbuttcap%
\pgfsetmiterjoin%
\definecolor{currentfill}{rgb}{0.501961,0.501961,0.501961}%
\pgfsetfillcolor{currentfill}%
\pgfsetfillopacity{0.700000}%
\pgfsetlinewidth{0.000000pt}%
\definecolor{currentstroke}{rgb}{0.000000,0.000000,0.000000}%
\pgfsetstrokecolor{currentstroke}%
\pgfsetstrokeopacity{0.700000}%
\pgfsetdash{}{0pt}%
\pgfpathmoveto{\pgfqpoint{4.464489in}{0.550000in}}%
\pgfpathlineto{\pgfqpoint{4.492010in}{0.550000in}}%
\pgfpathlineto{\pgfqpoint{4.492010in}{1.042282in}}%
\pgfpathlineto{\pgfqpoint{4.464489in}{1.042282in}}%
\pgfpathlineto{\pgfqpoint{4.464489in}{0.550000in}}%
\pgfpathclose%
\pgfusepath{fill}%
\end{pgfscope}%
\begin{pgfscope}%
\pgfpathrectangle{\pgfqpoint{1.250000in}{0.550000in}}{\pgfqpoint{7.750000in}{3.850000in}}%
\pgfusepath{clip}%
\pgfsetbuttcap%
\pgfsetmiterjoin%
\definecolor{currentfill}{rgb}{0.501961,0.501961,0.501961}%
\pgfsetfillcolor{currentfill}%
\pgfsetfillopacity{0.700000}%
\pgfsetlinewidth{0.000000pt}%
\definecolor{currentstroke}{rgb}{0.000000,0.000000,0.000000}%
\pgfsetstrokecolor{currentstroke}%
\pgfsetstrokeopacity{0.700000}%
\pgfsetdash{}{0pt}%
\pgfpathmoveto{\pgfqpoint{4.492010in}{0.550000in}}%
\pgfpathlineto{\pgfqpoint{4.519531in}{0.550000in}}%
\pgfpathlineto{\pgfqpoint{4.519531in}{0.550000in}}%
\pgfpathlineto{\pgfqpoint{4.492010in}{0.550000in}}%
\pgfpathlineto{\pgfqpoint{4.492010in}{0.550000in}}%
\pgfpathclose%
\pgfusepath{fill}%
\end{pgfscope}%
\begin{pgfscope}%
\pgfpathrectangle{\pgfqpoint{1.250000in}{0.550000in}}{\pgfqpoint{7.750000in}{3.850000in}}%
\pgfusepath{clip}%
\pgfsetbuttcap%
\pgfsetmiterjoin%
\definecolor{currentfill}{rgb}{0.501961,0.501961,0.501961}%
\pgfsetfillcolor{currentfill}%
\pgfsetfillopacity{0.700000}%
\pgfsetlinewidth{0.000000pt}%
\definecolor{currentstroke}{rgb}{0.000000,0.000000,0.000000}%
\pgfsetstrokecolor{currentstroke}%
\pgfsetstrokeopacity{0.700000}%
\pgfsetdash{}{0pt}%
\pgfpathmoveto{\pgfqpoint{4.519531in}{0.550000in}}%
\pgfpathlineto{\pgfqpoint{4.547053in}{0.550000in}}%
\pgfpathlineto{\pgfqpoint{4.547053in}{0.550000in}}%
\pgfpathlineto{\pgfqpoint{4.519531in}{0.550000in}}%
\pgfpathlineto{\pgfqpoint{4.519531in}{0.550000in}}%
\pgfpathclose%
\pgfusepath{fill}%
\end{pgfscope}%
\begin{pgfscope}%
\pgfpathrectangle{\pgfqpoint{1.250000in}{0.550000in}}{\pgfqpoint{7.750000in}{3.850000in}}%
\pgfusepath{clip}%
\pgfsetbuttcap%
\pgfsetmiterjoin%
\definecolor{currentfill}{rgb}{0.501961,0.501961,0.501961}%
\pgfsetfillcolor{currentfill}%
\pgfsetfillopacity{0.700000}%
\pgfsetlinewidth{0.000000pt}%
\definecolor{currentstroke}{rgb}{0.000000,0.000000,0.000000}%
\pgfsetstrokecolor{currentstroke}%
\pgfsetstrokeopacity{0.700000}%
\pgfsetdash{}{0pt}%
\pgfpathmoveto{\pgfqpoint{4.547053in}{0.550000in}}%
\pgfpathlineto{\pgfqpoint{4.574574in}{0.550000in}}%
\pgfpathlineto{\pgfqpoint{4.574574in}{1.032422in}}%
\pgfpathlineto{\pgfqpoint{4.547053in}{1.032422in}}%
\pgfpathlineto{\pgfqpoint{4.547053in}{0.550000in}}%
\pgfpathclose%
\pgfusepath{fill}%
\end{pgfscope}%
\begin{pgfscope}%
\pgfpathrectangle{\pgfqpoint{1.250000in}{0.550000in}}{\pgfqpoint{7.750000in}{3.850000in}}%
\pgfusepath{clip}%
\pgfsetbuttcap%
\pgfsetmiterjoin%
\definecolor{currentfill}{rgb}{0.501961,0.501961,0.501961}%
\pgfsetfillcolor{currentfill}%
\pgfsetfillopacity{0.700000}%
\pgfsetlinewidth{0.000000pt}%
\definecolor{currentstroke}{rgb}{0.000000,0.000000,0.000000}%
\pgfsetstrokecolor{currentstroke}%
\pgfsetstrokeopacity{0.700000}%
\pgfsetdash{}{0pt}%
\pgfpathmoveto{\pgfqpoint{4.574574in}{0.550000in}}%
\pgfpathlineto{\pgfqpoint{4.602095in}{0.550000in}}%
\pgfpathlineto{\pgfqpoint{4.602095in}{0.550000in}}%
\pgfpathlineto{\pgfqpoint{4.574574in}{0.550000in}}%
\pgfpathlineto{\pgfqpoint{4.574574in}{0.550000in}}%
\pgfpathclose%
\pgfusepath{fill}%
\end{pgfscope}%
\begin{pgfscope}%
\pgfpathrectangle{\pgfqpoint{1.250000in}{0.550000in}}{\pgfqpoint{7.750000in}{3.850000in}}%
\pgfusepath{clip}%
\pgfsetbuttcap%
\pgfsetmiterjoin%
\definecolor{currentfill}{rgb}{0.501961,0.501961,0.501961}%
\pgfsetfillcolor{currentfill}%
\pgfsetfillopacity{0.700000}%
\pgfsetlinewidth{0.000000pt}%
\definecolor{currentstroke}{rgb}{0.000000,0.000000,0.000000}%
\pgfsetstrokecolor{currentstroke}%
\pgfsetstrokeopacity{0.700000}%
\pgfsetdash{}{0pt}%
\pgfpathmoveto{\pgfqpoint{4.602095in}{0.550000in}}%
\pgfpathlineto{\pgfqpoint{4.629616in}{0.550000in}}%
\pgfpathlineto{\pgfqpoint{4.629616in}{0.550000in}}%
\pgfpathlineto{\pgfqpoint{4.602095in}{0.550000in}}%
\pgfpathlineto{\pgfqpoint{4.602095in}{0.550000in}}%
\pgfpathclose%
\pgfusepath{fill}%
\end{pgfscope}%
\begin{pgfscope}%
\pgfpathrectangle{\pgfqpoint{1.250000in}{0.550000in}}{\pgfqpoint{7.750000in}{3.850000in}}%
\pgfusepath{clip}%
\pgfsetbuttcap%
\pgfsetmiterjoin%
\definecolor{currentfill}{rgb}{0.501961,0.501961,0.501961}%
\pgfsetfillcolor{currentfill}%
\pgfsetfillopacity{0.700000}%
\pgfsetlinewidth{0.000000pt}%
\definecolor{currentstroke}{rgb}{0.000000,0.000000,0.000000}%
\pgfsetstrokecolor{currentstroke}%
\pgfsetstrokeopacity{0.700000}%
\pgfsetdash{}{0pt}%
\pgfpathmoveto{\pgfqpoint{4.629616in}{0.550000in}}%
\pgfpathlineto{\pgfqpoint{4.657138in}{0.550000in}}%
\pgfpathlineto{\pgfqpoint{4.657138in}{1.008351in}}%
\pgfpathlineto{\pgfqpoint{4.629616in}{1.008351in}}%
\pgfpathlineto{\pgfqpoint{4.629616in}{0.550000in}}%
\pgfpathclose%
\pgfusepath{fill}%
\end{pgfscope}%
\begin{pgfscope}%
\pgfpathrectangle{\pgfqpoint{1.250000in}{0.550000in}}{\pgfqpoint{7.750000in}{3.850000in}}%
\pgfusepath{clip}%
\pgfsetbuttcap%
\pgfsetmiterjoin%
\definecolor{currentfill}{rgb}{0.501961,0.501961,0.501961}%
\pgfsetfillcolor{currentfill}%
\pgfsetfillopacity{0.700000}%
\pgfsetlinewidth{0.000000pt}%
\definecolor{currentstroke}{rgb}{0.000000,0.000000,0.000000}%
\pgfsetstrokecolor{currentstroke}%
\pgfsetstrokeopacity{0.700000}%
\pgfsetdash{}{0pt}%
\pgfpathmoveto{\pgfqpoint{4.657138in}{0.550000in}}%
\pgfpathlineto{\pgfqpoint{4.684659in}{0.550000in}}%
\pgfpathlineto{\pgfqpoint{4.684659in}{0.550000in}}%
\pgfpathlineto{\pgfqpoint{4.657138in}{0.550000in}}%
\pgfpathlineto{\pgfqpoint{4.657138in}{0.550000in}}%
\pgfpathclose%
\pgfusepath{fill}%
\end{pgfscope}%
\begin{pgfscope}%
\pgfpathrectangle{\pgfqpoint{1.250000in}{0.550000in}}{\pgfqpoint{7.750000in}{3.850000in}}%
\pgfusepath{clip}%
\pgfsetbuttcap%
\pgfsetmiterjoin%
\definecolor{currentfill}{rgb}{0.501961,0.501961,0.501961}%
\pgfsetfillcolor{currentfill}%
\pgfsetfillopacity{0.700000}%
\pgfsetlinewidth{0.000000pt}%
\definecolor{currentstroke}{rgb}{0.000000,0.000000,0.000000}%
\pgfsetstrokecolor{currentstroke}%
\pgfsetstrokeopacity{0.700000}%
\pgfsetdash{}{0pt}%
\pgfpathmoveto{\pgfqpoint{4.684659in}{0.550000in}}%
\pgfpathlineto{\pgfqpoint{4.712180in}{0.550000in}}%
\pgfpathlineto{\pgfqpoint{4.712180in}{0.550000in}}%
\pgfpathlineto{\pgfqpoint{4.684659in}{0.550000in}}%
\pgfpathlineto{\pgfqpoint{4.684659in}{0.550000in}}%
\pgfpathclose%
\pgfusepath{fill}%
\end{pgfscope}%
\begin{pgfscope}%
\pgfpathrectangle{\pgfqpoint{1.250000in}{0.550000in}}{\pgfqpoint{7.750000in}{3.850000in}}%
\pgfusepath{clip}%
\pgfsetbuttcap%
\pgfsetmiterjoin%
\definecolor{currentfill}{rgb}{0.501961,0.501961,0.501961}%
\pgfsetfillcolor{currentfill}%
\pgfsetfillopacity{0.700000}%
\pgfsetlinewidth{0.000000pt}%
\definecolor{currentstroke}{rgb}{0.000000,0.000000,0.000000}%
\pgfsetstrokecolor{currentstroke}%
\pgfsetstrokeopacity{0.700000}%
\pgfsetdash{}{0pt}%
\pgfpathmoveto{\pgfqpoint{4.712180in}{0.550000in}}%
\pgfpathlineto{\pgfqpoint{4.739702in}{0.550000in}}%
\pgfpathlineto{\pgfqpoint{4.739702in}{1.021402in}}%
\pgfpathlineto{\pgfqpoint{4.712180in}{1.021402in}}%
\pgfpathlineto{\pgfqpoint{4.712180in}{0.550000in}}%
\pgfpathclose%
\pgfusepath{fill}%
\end{pgfscope}%
\begin{pgfscope}%
\pgfpathrectangle{\pgfqpoint{1.250000in}{0.550000in}}{\pgfqpoint{7.750000in}{3.850000in}}%
\pgfusepath{clip}%
\pgfsetbuttcap%
\pgfsetmiterjoin%
\definecolor{currentfill}{rgb}{0.501961,0.501961,0.501961}%
\pgfsetfillcolor{currentfill}%
\pgfsetfillopacity{0.700000}%
\pgfsetlinewidth{0.000000pt}%
\definecolor{currentstroke}{rgb}{0.000000,0.000000,0.000000}%
\pgfsetstrokecolor{currentstroke}%
\pgfsetstrokeopacity{0.700000}%
\pgfsetdash{}{0pt}%
\pgfpathmoveto{\pgfqpoint{4.739702in}{0.550000in}}%
\pgfpathlineto{\pgfqpoint{4.767223in}{0.550000in}}%
\pgfpathlineto{\pgfqpoint{4.767223in}{0.550000in}}%
\pgfpathlineto{\pgfqpoint{4.739702in}{0.550000in}}%
\pgfpathlineto{\pgfqpoint{4.739702in}{0.550000in}}%
\pgfpathclose%
\pgfusepath{fill}%
\end{pgfscope}%
\begin{pgfscope}%
\pgfpathrectangle{\pgfqpoint{1.250000in}{0.550000in}}{\pgfqpoint{7.750000in}{3.850000in}}%
\pgfusepath{clip}%
\pgfsetbuttcap%
\pgfsetmiterjoin%
\definecolor{currentfill}{rgb}{0.501961,0.501961,0.501961}%
\pgfsetfillcolor{currentfill}%
\pgfsetfillopacity{0.700000}%
\pgfsetlinewidth{0.000000pt}%
\definecolor{currentstroke}{rgb}{0.000000,0.000000,0.000000}%
\pgfsetstrokecolor{currentstroke}%
\pgfsetstrokeopacity{0.700000}%
\pgfsetdash{}{0pt}%
\pgfpathmoveto{\pgfqpoint{4.767223in}{0.550000in}}%
\pgfpathlineto{\pgfqpoint{4.794744in}{0.550000in}}%
\pgfpathlineto{\pgfqpoint{4.794744in}{0.550000in}}%
\pgfpathlineto{\pgfqpoint{4.767223in}{0.550000in}}%
\pgfpathlineto{\pgfqpoint{4.767223in}{0.550000in}}%
\pgfpathclose%
\pgfusepath{fill}%
\end{pgfscope}%
\begin{pgfscope}%
\pgfpathrectangle{\pgfqpoint{1.250000in}{0.550000in}}{\pgfqpoint{7.750000in}{3.850000in}}%
\pgfusepath{clip}%
\pgfsetbuttcap%
\pgfsetmiterjoin%
\definecolor{currentfill}{rgb}{0.501961,0.501961,0.501961}%
\pgfsetfillcolor{currentfill}%
\pgfsetfillopacity{0.700000}%
\pgfsetlinewidth{0.000000pt}%
\definecolor{currentstroke}{rgb}{0.000000,0.000000,0.000000}%
\pgfsetstrokecolor{currentstroke}%
\pgfsetstrokeopacity{0.700000}%
\pgfsetdash{}{0pt}%
\pgfpathmoveto{\pgfqpoint{4.794744in}{0.550000in}}%
\pgfpathlineto{\pgfqpoint{4.822266in}{0.550000in}}%
\pgfpathlineto{\pgfqpoint{4.822266in}{1.002986in}}%
\pgfpathlineto{\pgfqpoint{4.794744in}{1.002986in}}%
\pgfpathlineto{\pgfqpoint{4.794744in}{0.550000in}}%
\pgfpathclose%
\pgfusepath{fill}%
\end{pgfscope}%
\begin{pgfscope}%
\pgfpathrectangle{\pgfqpoint{1.250000in}{0.550000in}}{\pgfqpoint{7.750000in}{3.850000in}}%
\pgfusepath{clip}%
\pgfsetbuttcap%
\pgfsetmiterjoin%
\definecolor{currentfill}{rgb}{0.501961,0.501961,0.501961}%
\pgfsetfillcolor{currentfill}%
\pgfsetfillopacity{0.700000}%
\pgfsetlinewidth{0.000000pt}%
\definecolor{currentstroke}{rgb}{0.000000,0.000000,0.000000}%
\pgfsetstrokecolor{currentstroke}%
\pgfsetstrokeopacity{0.700000}%
\pgfsetdash{}{0pt}%
\pgfpathmoveto{\pgfqpoint{4.822266in}{0.550000in}}%
\pgfpathlineto{\pgfqpoint{4.849787in}{0.550000in}}%
\pgfpathlineto{\pgfqpoint{4.849787in}{0.550000in}}%
\pgfpathlineto{\pgfqpoint{4.822266in}{0.550000in}}%
\pgfpathlineto{\pgfqpoint{4.822266in}{0.550000in}}%
\pgfpathclose%
\pgfusepath{fill}%
\end{pgfscope}%
\begin{pgfscope}%
\pgfpathrectangle{\pgfqpoint{1.250000in}{0.550000in}}{\pgfqpoint{7.750000in}{3.850000in}}%
\pgfusepath{clip}%
\pgfsetbuttcap%
\pgfsetmiterjoin%
\definecolor{currentfill}{rgb}{0.501961,0.501961,0.501961}%
\pgfsetfillcolor{currentfill}%
\pgfsetfillopacity{0.700000}%
\pgfsetlinewidth{0.000000pt}%
\definecolor{currentstroke}{rgb}{0.000000,0.000000,0.000000}%
\pgfsetstrokecolor{currentstroke}%
\pgfsetstrokeopacity{0.700000}%
\pgfsetdash{}{0pt}%
\pgfpathmoveto{\pgfqpoint{4.849787in}{0.550000in}}%
\pgfpathlineto{\pgfqpoint{4.877308in}{0.550000in}}%
\pgfpathlineto{\pgfqpoint{4.877308in}{0.990516in}}%
\pgfpathlineto{\pgfqpoint{4.849787in}{0.990516in}}%
\pgfpathlineto{\pgfqpoint{4.849787in}{0.550000in}}%
\pgfpathclose%
\pgfusepath{fill}%
\end{pgfscope}%
\begin{pgfscope}%
\pgfpathrectangle{\pgfqpoint{1.250000in}{0.550000in}}{\pgfqpoint{7.750000in}{3.850000in}}%
\pgfusepath{clip}%
\pgfsetbuttcap%
\pgfsetmiterjoin%
\definecolor{currentfill}{rgb}{0.501961,0.501961,0.501961}%
\pgfsetfillcolor{currentfill}%
\pgfsetfillopacity{0.700000}%
\pgfsetlinewidth{0.000000pt}%
\definecolor{currentstroke}{rgb}{0.000000,0.000000,0.000000}%
\pgfsetstrokecolor{currentstroke}%
\pgfsetstrokeopacity{0.700000}%
\pgfsetdash{}{0pt}%
\pgfpathmoveto{\pgfqpoint{4.877308in}{0.550000in}}%
\pgfpathlineto{\pgfqpoint{4.904830in}{0.550000in}}%
\pgfpathlineto{\pgfqpoint{4.904830in}{0.550000in}}%
\pgfpathlineto{\pgfqpoint{4.877308in}{0.550000in}}%
\pgfpathlineto{\pgfqpoint{4.877308in}{0.550000in}}%
\pgfpathclose%
\pgfusepath{fill}%
\end{pgfscope}%
\begin{pgfscope}%
\pgfpathrectangle{\pgfqpoint{1.250000in}{0.550000in}}{\pgfqpoint{7.750000in}{3.850000in}}%
\pgfusepath{clip}%
\pgfsetbuttcap%
\pgfsetmiterjoin%
\definecolor{currentfill}{rgb}{0.501961,0.501961,0.501961}%
\pgfsetfillcolor{currentfill}%
\pgfsetfillopacity{0.700000}%
\pgfsetlinewidth{0.000000pt}%
\definecolor{currentstroke}{rgb}{0.000000,0.000000,0.000000}%
\pgfsetstrokecolor{currentstroke}%
\pgfsetstrokeopacity{0.700000}%
\pgfsetdash{}{0pt}%
\pgfpathmoveto{\pgfqpoint{4.904830in}{0.550000in}}%
\pgfpathlineto{\pgfqpoint{4.932351in}{0.550000in}}%
\pgfpathlineto{\pgfqpoint{4.932351in}{0.550000in}}%
\pgfpathlineto{\pgfqpoint{4.904830in}{0.550000in}}%
\pgfpathlineto{\pgfqpoint{4.904830in}{0.550000in}}%
\pgfpathclose%
\pgfusepath{fill}%
\end{pgfscope}%
\begin{pgfscope}%
\pgfpathrectangle{\pgfqpoint{1.250000in}{0.550000in}}{\pgfqpoint{7.750000in}{3.850000in}}%
\pgfusepath{clip}%
\pgfsetbuttcap%
\pgfsetmiterjoin%
\definecolor{currentfill}{rgb}{0.501961,0.501961,0.501961}%
\pgfsetfillcolor{currentfill}%
\pgfsetfillopacity{0.700000}%
\pgfsetlinewidth{0.000000pt}%
\definecolor{currentstroke}{rgb}{0.000000,0.000000,0.000000}%
\pgfsetstrokecolor{currentstroke}%
\pgfsetstrokeopacity{0.700000}%
\pgfsetdash{}{0pt}%
\pgfpathmoveto{\pgfqpoint{4.932351in}{0.550000in}}%
\pgfpathlineto{\pgfqpoint{4.959872in}{0.550000in}}%
\pgfpathlineto{\pgfqpoint{4.959872in}{0.968766in}}%
\pgfpathlineto{\pgfqpoint{4.932351in}{0.968766in}}%
\pgfpathlineto{\pgfqpoint{4.932351in}{0.550000in}}%
\pgfpathclose%
\pgfusepath{fill}%
\end{pgfscope}%
\begin{pgfscope}%
\pgfpathrectangle{\pgfqpoint{1.250000in}{0.550000in}}{\pgfqpoint{7.750000in}{3.850000in}}%
\pgfusepath{clip}%
\pgfsetbuttcap%
\pgfsetmiterjoin%
\definecolor{currentfill}{rgb}{0.501961,0.501961,0.501961}%
\pgfsetfillcolor{currentfill}%
\pgfsetfillopacity{0.700000}%
\pgfsetlinewidth{0.000000pt}%
\definecolor{currentstroke}{rgb}{0.000000,0.000000,0.000000}%
\pgfsetstrokecolor{currentstroke}%
\pgfsetstrokeopacity{0.700000}%
\pgfsetdash{}{0pt}%
\pgfpathmoveto{\pgfqpoint{4.959872in}{0.550000in}}%
\pgfpathlineto{\pgfqpoint{4.987393in}{0.550000in}}%
\pgfpathlineto{\pgfqpoint{4.987393in}{0.550000in}}%
\pgfpathlineto{\pgfqpoint{4.959872in}{0.550000in}}%
\pgfpathlineto{\pgfqpoint{4.959872in}{0.550000in}}%
\pgfpathclose%
\pgfusepath{fill}%
\end{pgfscope}%
\begin{pgfscope}%
\pgfpathrectangle{\pgfqpoint{1.250000in}{0.550000in}}{\pgfqpoint{7.750000in}{3.850000in}}%
\pgfusepath{clip}%
\pgfsetbuttcap%
\pgfsetmiterjoin%
\definecolor{currentfill}{rgb}{0.501961,0.501961,0.501961}%
\pgfsetfillcolor{currentfill}%
\pgfsetfillopacity{0.700000}%
\pgfsetlinewidth{0.000000pt}%
\definecolor{currentstroke}{rgb}{0.000000,0.000000,0.000000}%
\pgfsetstrokecolor{currentstroke}%
\pgfsetstrokeopacity{0.700000}%
\pgfsetdash{}{0pt}%
\pgfpathmoveto{\pgfqpoint{4.987393in}{0.550000in}}%
\pgfpathlineto{\pgfqpoint{5.014915in}{0.550000in}}%
\pgfpathlineto{\pgfqpoint{5.014915in}{0.550000in}}%
\pgfpathlineto{\pgfqpoint{4.987393in}{0.550000in}}%
\pgfpathlineto{\pgfqpoint{4.987393in}{0.550000in}}%
\pgfpathclose%
\pgfusepath{fill}%
\end{pgfscope}%
\begin{pgfscope}%
\pgfpathrectangle{\pgfqpoint{1.250000in}{0.550000in}}{\pgfqpoint{7.750000in}{3.850000in}}%
\pgfusepath{clip}%
\pgfsetbuttcap%
\pgfsetmiterjoin%
\definecolor{currentfill}{rgb}{0.501961,0.501961,0.501961}%
\pgfsetfillcolor{currentfill}%
\pgfsetfillopacity{0.700000}%
\pgfsetlinewidth{0.000000pt}%
\definecolor{currentstroke}{rgb}{0.000000,0.000000,0.000000}%
\pgfsetstrokecolor{currentstroke}%
\pgfsetstrokeopacity{0.700000}%
\pgfsetdash{}{0pt}%
\pgfpathmoveto{\pgfqpoint{5.014915in}{0.550000in}}%
\pgfpathlineto{\pgfqpoint{5.042436in}{0.550000in}}%
\pgfpathlineto{\pgfqpoint{5.042436in}{0.967461in}}%
\pgfpathlineto{\pgfqpoint{5.014915in}{0.967461in}}%
\pgfpathlineto{\pgfqpoint{5.014915in}{0.550000in}}%
\pgfpathclose%
\pgfusepath{fill}%
\end{pgfscope}%
\begin{pgfscope}%
\pgfpathrectangle{\pgfqpoint{1.250000in}{0.550000in}}{\pgfqpoint{7.750000in}{3.850000in}}%
\pgfusepath{clip}%
\pgfsetbuttcap%
\pgfsetmiterjoin%
\definecolor{currentfill}{rgb}{0.501961,0.501961,0.501961}%
\pgfsetfillcolor{currentfill}%
\pgfsetfillopacity{0.700000}%
\pgfsetlinewidth{0.000000pt}%
\definecolor{currentstroke}{rgb}{0.000000,0.000000,0.000000}%
\pgfsetstrokecolor{currentstroke}%
\pgfsetstrokeopacity{0.700000}%
\pgfsetdash{}{0pt}%
\pgfpathmoveto{\pgfqpoint{5.042436in}{0.550000in}}%
\pgfpathlineto{\pgfqpoint{5.069957in}{0.550000in}}%
\pgfpathlineto{\pgfqpoint{5.069957in}{0.550000in}}%
\pgfpathlineto{\pgfqpoint{5.042436in}{0.550000in}}%
\pgfpathlineto{\pgfqpoint{5.042436in}{0.550000in}}%
\pgfpathclose%
\pgfusepath{fill}%
\end{pgfscope}%
\begin{pgfscope}%
\pgfpathrectangle{\pgfqpoint{1.250000in}{0.550000in}}{\pgfqpoint{7.750000in}{3.850000in}}%
\pgfusepath{clip}%
\pgfsetbuttcap%
\pgfsetmiterjoin%
\definecolor{currentfill}{rgb}{0.501961,0.501961,0.501961}%
\pgfsetfillcolor{currentfill}%
\pgfsetfillopacity{0.700000}%
\pgfsetlinewidth{0.000000pt}%
\definecolor{currentstroke}{rgb}{0.000000,0.000000,0.000000}%
\pgfsetstrokecolor{currentstroke}%
\pgfsetstrokeopacity{0.700000}%
\pgfsetdash{}{0pt}%
\pgfpathmoveto{\pgfqpoint{5.069957in}{0.550000in}}%
\pgfpathlineto{\pgfqpoint{5.097479in}{0.550000in}}%
\pgfpathlineto{\pgfqpoint{5.097479in}{0.950206in}}%
\pgfpathlineto{\pgfqpoint{5.069957in}{0.950206in}}%
\pgfpathlineto{\pgfqpoint{5.069957in}{0.550000in}}%
\pgfpathclose%
\pgfusepath{fill}%
\end{pgfscope}%
\begin{pgfscope}%
\pgfpathrectangle{\pgfqpoint{1.250000in}{0.550000in}}{\pgfqpoint{7.750000in}{3.850000in}}%
\pgfusepath{clip}%
\pgfsetbuttcap%
\pgfsetmiterjoin%
\definecolor{currentfill}{rgb}{0.501961,0.501961,0.501961}%
\pgfsetfillcolor{currentfill}%
\pgfsetfillopacity{0.700000}%
\pgfsetlinewidth{0.000000pt}%
\definecolor{currentstroke}{rgb}{0.000000,0.000000,0.000000}%
\pgfsetstrokecolor{currentstroke}%
\pgfsetstrokeopacity{0.700000}%
\pgfsetdash{}{0pt}%
\pgfpathmoveto{\pgfqpoint{5.097479in}{0.550000in}}%
\pgfpathlineto{\pgfqpoint{5.125000in}{0.550000in}}%
\pgfpathlineto{\pgfqpoint{5.125000in}{0.550000in}}%
\pgfpathlineto{\pgfqpoint{5.097479in}{0.550000in}}%
\pgfpathlineto{\pgfqpoint{5.097479in}{0.550000in}}%
\pgfpathclose%
\pgfusepath{fill}%
\end{pgfscope}%
\begin{pgfscope}%
\pgfpathrectangle{\pgfqpoint{1.250000in}{0.550000in}}{\pgfqpoint{7.750000in}{3.850000in}}%
\pgfusepath{clip}%
\pgfsetbuttcap%
\pgfsetmiterjoin%
\definecolor{currentfill}{rgb}{0.501961,0.501961,0.501961}%
\pgfsetfillcolor{currentfill}%
\pgfsetfillopacity{0.700000}%
\pgfsetlinewidth{0.000000pt}%
\definecolor{currentstroke}{rgb}{0.000000,0.000000,0.000000}%
\pgfsetstrokecolor{currentstroke}%
\pgfsetstrokeopacity{0.700000}%
\pgfsetdash{}{0pt}%
\pgfpathmoveto{\pgfqpoint{5.125000in}{0.550000in}}%
\pgfpathlineto{\pgfqpoint{5.152521in}{0.550000in}}%
\pgfpathlineto{\pgfqpoint{5.152521in}{0.550000in}}%
\pgfpathlineto{\pgfqpoint{5.125000in}{0.550000in}}%
\pgfpathlineto{\pgfqpoint{5.125000in}{0.550000in}}%
\pgfpathclose%
\pgfusepath{fill}%
\end{pgfscope}%
\begin{pgfscope}%
\pgfpathrectangle{\pgfqpoint{1.250000in}{0.550000in}}{\pgfqpoint{7.750000in}{3.850000in}}%
\pgfusepath{clip}%
\pgfsetbuttcap%
\pgfsetmiterjoin%
\definecolor{currentfill}{rgb}{0.501961,0.501961,0.501961}%
\pgfsetfillcolor{currentfill}%
\pgfsetfillopacity{0.700000}%
\pgfsetlinewidth{0.000000pt}%
\definecolor{currentstroke}{rgb}{0.000000,0.000000,0.000000}%
\pgfsetstrokecolor{currentstroke}%
\pgfsetstrokeopacity{0.700000}%
\pgfsetdash{}{0pt}%
\pgfpathmoveto{\pgfqpoint{5.152521in}{0.550000in}}%
\pgfpathlineto{\pgfqpoint{5.180043in}{0.550000in}}%
\pgfpathlineto{\pgfqpoint{5.180043in}{0.950061in}}%
\pgfpathlineto{\pgfqpoint{5.152521in}{0.950061in}}%
\pgfpathlineto{\pgfqpoint{5.152521in}{0.550000in}}%
\pgfpathclose%
\pgfusepath{fill}%
\end{pgfscope}%
\begin{pgfscope}%
\pgfpathrectangle{\pgfqpoint{1.250000in}{0.550000in}}{\pgfqpoint{7.750000in}{3.850000in}}%
\pgfusepath{clip}%
\pgfsetbuttcap%
\pgfsetmiterjoin%
\definecolor{currentfill}{rgb}{0.501961,0.501961,0.501961}%
\pgfsetfillcolor{currentfill}%
\pgfsetfillopacity{0.700000}%
\pgfsetlinewidth{0.000000pt}%
\definecolor{currentstroke}{rgb}{0.000000,0.000000,0.000000}%
\pgfsetstrokecolor{currentstroke}%
\pgfsetstrokeopacity{0.700000}%
\pgfsetdash{}{0pt}%
\pgfpathmoveto{\pgfqpoint{5.180043in}{0.550000in}}%
\pgfpathlineto{\pgfqpoint{5.207564in}{0.550000in}}%
\pgfpathlineto{\pgfqpoint{5.207564in}{0.550000in}}%
\pgfpathlineto{\pgfqpoint{5.180043in}{0.550000in}}%
\pgfpathlineto{\pgfqpoint{5.180043in}{0.550000in}}%
\pgfpathclose%
\pgfusepath{fill}%
\end{pgfscope}%
\begin{pgfscope}%
\pgfpathrectangle{\pgfqpoint{1.250000in}{0.550000in}}{\pgfqpoint{7.750000in}{3.850000in}}%
\pgfusepath{clip}%
\pgfsetbuttcap%
\pgfsetmiterjoin%
\definecolor{currentfill}{rgb}{0.501961,0.501961,0.501961}%
\pgfsetfillcolor{currentfill}%
\pgfsetfillopacity{0.700000}%
\pgfsetlinewidth{0.000000pt}%
\definecolor{currentstroke}{rgb}{0.000000,0.000000,0.000000}%
\pgfsetstrokecolor{currentstroke}%
\pgfsetstrokeopacity{0.700000}%
\pgfsetdash{}{0pt}%
\pgfpathmoveto{\pgfqpoint{5.207564in}{0.550000in}}%
\pgfpathlineto{\pgfqpoint{5.235085in}{0.550000in}}%
\pgfpathlineto{\pgfqpoint{5.235085in}{0.932225in}}%
\pgfpathlineto{\pgfqpoint{5.207564in}{0.932225in}}%
\pgfpathlineto{\pgfqpoint{5.207564in}{0.550000in}}%
\pgfpathclose%
\pgfusepath{fill}%
\end{pgfscope}%
\begin{pgfscope}%
\pgfpathrectangle{\pgfqpoint{1.250000in}{0.550000in}}{\pgfqpoint{7.750000in}{3.850000in}}%
\pgfusepath{clip}%
\pgfsetbuttcap%
\pgfsetmiterjoin%
\definecolor{currentfill}{rgb}{0.501961,0.501961,0.501961}%
\pgfsetfillcolor{currentfill}%
\pgfsetfillopacity{0.700000}%
\pgfsetlinewidth{0.000000pt}%
\definecolor{currentstroke}{rgb}{0.000000,0.000000,0.000000}%
\pgfsetstrokecolor{currentstroke}%
\pgfsetstrokeopacity{0.700000}%
\pgfsetdash{}{0pt}%
\pgfpathmoveto{\pgfqpoint{5.235085in}{0.550000in}}%
\pgfpathlineto{\pgfqpoint{5.262607in}{0.550000in}}%
\pgfpathlineto{\pgfqpoint{5.262607in}{0.550000in}}%
\pgfpathlineto{\pgfqpoint{5.235085in}{0.550000in}}%
\pgfpathlineto{\pgfqpoint{5.235085in}{0.550000in}}%
\pgfpathclose%
\pgfusepath{fill}%
\end{pgfscope}%
\begin{pgfscope}%
\pgfpathrectangle{\pgfqpoint{1.250000in}{0.550000in}}{\pgfqpoint{7.750000in}{3.850000in}}%
\pgfusepath{clip}%
\pgfsetbuttcap%
\pgfsetmiterjoin%
\definecolor{currentfill}{rgb}{0.501961,0.501961,0.501961}%
\pgfsetfillcolor{currentfill}%
\pgfsetfillopacity{0.700000}%
\pgfsetlinewidth{0.000000pt}%
\definecolor{currentstroke}{rgb}{0.000000,0.000000,0.000000}%
\pgfsetstrokecolor{currentstroke}%
\pgfsetstrokeopacity{0.700000}%
\pgfsetdash{}{0pt}%
\pgfpathmoveto{\pgfqpoint{5.262607in}{0.550000in}}%
\pgfpathlineto{\pgfqpoint{5.290128in}{0.550000in}}%
\pgfpathlineto{\pgfqpoint{5.290128in}{0.928745in}}%
\pgfpathlineto{\pgfqpoint{5.262607in}{0.928745in}}%
\pgfpathlineto{\pgfqpoint{5.262607in}{0.550000in}}%
\pgfpathclose%
\pgfusepath{fill}%
\end{pgfscope}%
\begin{pgfscope}%
\pgfpathrectangle{\pgfqpoint{1.250000in}{0.550000in}}{\pgfqpoint{7.750000in}{3.850000in}}%
\pgfusepath{clip}%
\pgfsetbuttcap%
\pgfsetmiterjoin%
\definecolor{currentfill}{rgb}{0.501961,0.501961,0.501961}%
\pgfsetfillcolor{currentfill}%
\pgfsetfillopacity{0.700000}%
\pgfsetlinewidth{0.000000pt}%
\definecolor{currentstroke}{rgb}{0.000000,0.000000,0.000000}%
\pgfsetstrokecolor{currentstroke}%
\pgfsetstrokeopacity{0.700000}%
\pgfsetdash{}{0pt}%
\pgfpathmoveto{\pgfqpoint{5.290128in}{0.550000in}}%
\pgfpathlineto{\pgfqpoint{5.317649in}{0.550000in}}%
\pgfpathlineto{\pgfqpoint{5.317649in}{0.550000in}}%
\pgfpathlineto{\pgfqpoint{5.290128in}{0.550000in}}%
\pgfpathlineto{\pgfqpoint{5.290128in}{0.550000in}}%
\pgfpathclose%
\pgfusepath{fill}%
\end{pgfscope}%
\begin{pgfscope}%
\pgfpathrectangle{\pgfqpoint{1.250000in}{0.550000in}}{\pgfqpoint{7.750000in}{3.850000in}}%
\pgfusepath{clip}%
\pgfsetbuttcap%
\pgfsetmiterjoin%
\definecolor{currentfill}{rgb}{0.501961,0.501961,0.501961}%
\pgfsetfillcolor{currentfill}%
\pgfsetfillopacity{0.700000}%
\pgfsetlinewidth{0.000000pt}%
\definecolor{currentstroke}{rgb}{0.000000,0.000000,0.000000}%
\pgfsetstrokecolor{currentstroke}%
\pgfsetstrokeopacity{0.700000}%
\pgfsetdash{}{0pt}%
\pgfpathmoveto{\pgfqpoint{5.317649in}{0.550000in}}%
\pgfpathlineto{\pgfqpoint{5.345170in}{0.550000in}}%
\pgfpathlineto{\pgfqpoint{5.345170in}{0.550000in}}%
\pgfpathlineto{\pgfqpoint{5.317649in}{0.550000in}}%
\pgfpathlineto{\pgfqpoint{5.317649in}{0.550000in}}%
\pgfpathclose%
\pgfusepath{fill}%
\end{pgfscope}%
\begin{pgfscope}%
\pgfpathrectangle{\pgfqpoint{1.250000in}{0.550000in}}{\pgfqpoint{7.750000in}{3.850000in}}%
\pgfusepath{clip}%
\pgfsetbuttcap%
\pgfsetmiterjoin%
\definecolor{currentfill}{rgb}{0.501961,0.501961,0.501961}%
\pgfsetfillcolor{currentfill}%
\pgfsetfillopacity{0.700000}%
\pgfsetlinewidth{0.000000pt}%
\definecolor{currentstroke}{rgb}{0.000000,0.000000,0.000000}%
\pgfsetstrokecolor{currentstroke}%
\pgfsetstrokeopacity{0.700000}%
\pgfsetdash{}{0pt}%
\pgfpathmoveto{\pgfqpoint{5.345170in}{0.550000in}}%
\pgfpathlineto{\pgfqpoint{5.372692in}{0.550000in}}%
\pgfpathlineto{\pgfqpoint{5.372692in}{0.922365in}}%
\pgfpathlineto{\pgfqpoint{5.345170in}{0.922365in}}%
\pgfpathlineto{\pgfqpoint{5.345170in}{0.550000in}}%
\pgfpathclose%
\pgfusepath{fill}%
\end{pgfscope}%
\begin{pgfscope}%
\pgfpathrectangle{\pgfqpoint{1.250000in}{0.550000in}}{\pgfqpoint{7.750000in}{3.850000in}}%
\pgfusepath{clip}%
\pgfsetbuttcap%
\pgfsetmiterjoin%
\definecolor{currentfill}{rgb}{0.501961,0.501961,0.501961}%
\pgfsetfillcolor{currentfill}%
\pgfsetfillopacity{0.700000}%
\pgfsetlinewidth{0.000000pt}%
\definecolor{currentstroke}{rgb}{0.000000,0.000000,0.000000}%
\pgfsetstrokecolor{currentstroke}%
\pgfsetstrokeopacity{0.700000}%
\pgfsetdash{}{0pt}%
\pgfpathmoveto{\pgfqpoint{5.372692in}{0.550000in}}%
\pgfpathlineto{\pgfqpoint{5.400213in}{0.550000in}}%
\pgfpathlineto{\pgfqpoint{5.400213in}{0.550000in}}%
\pgfpathlineto{\pgfqpoint{5.372692in}{0.550000in}}%
\pgfpathlineto{\pgfqpoint{5.372692in}{0.550000in}}%
\pgfpathclose%
\pgfusepath{fill}%
\end{pgfscope}%
\begin{pgfscope}%
\pgfpathrectangle{\pgfqpoint{1.250000in}{0.550000in}}{\pgfqpoint{7.750000in}{3.850000in}}%
\pgfusepath{clip}%
\pgfsetbuttcap%
\pgfsetmiterjoin%
\definecolor{currentfill}{rgb}{0.501961,0.501961,0.501961}%
\pgfsetfillcolor{currentfill}%
\pgfsetfillopacity{0.700000}%
\pgfsetlinewidth{0.000000pt}%
\definecolor{currentstroke}{rgb}{0.000000,0.000000,0.000000}%
\pgfsetstrokecolor{currentstroke}%
\pgfsetstrokeopacity{0.700000}%
\pgfsetdash{}{0pt}%
\pgfpathmoveto{\pgfqpoint{5.400213in}{0.550000in}}%
\pgfpathlineto{\pgfqpoint{5.427734in}{0.550000in}}%
\pgfpathlineto{\pgfqpoint{5.427734in}{0.908445in}}%
\pgfpathlineto{\pgfqpoint{5.400213in}{0.908445in}}%
\pgfpathlineto{\pgfqpoint{5.400213in}{0.550000in}}%
\pgfpathclose%
\pgfusepath{fill}%
\end{pgfscope}%
\begin{pgfscope}%
\pgfpathrectangle{\pgfqpoint{1.250000in}{0.550000in}}{\pgfqpoint{7.750000in}{3.850000in}}%
\pgfusepath{clip}%
\pgfsetbuttcap%
\pgfsetmiterjoin%
\definecolor{currentfill}{rgb}{0.501961,0.501961,0.501961}%
\pgfsetfillcolor{currentfill}%
\pgfsetfillopacity{0.700000}%
\pgfsetlinewidth{0.000000pt}%
\definecolor{currentstroke}{rgb}{0.000000,0.000000,0.000000}%
\pgfsetstrokecolor{currentstroke}%
\pgfsetstrokeopacity{0.700000}%
\pgfsetdash{}{0pt}%
\pgfpathmoveto{\pgfqpoint{5.427734in}{0.550000in}}%
\pgfpathlineto{\pgfqpoint{5.455256in}{0.550000in}}%
\pgfpathlineto{\pgfqpoint{5.455256in}{0.550000in}}%
\pgfpathlineto{\pgfqpoint{5.427734in}{0.550000in}}%
\pgfpathlineto{\pgfqpoint{5.427734in}{0.550000in}}%
\pgfpathclose%
\pgfusepath{fill}%
\end{pgfscope}%
\begin{pgfscope}%
\pgfpathrectangle{\pgfqpoint{1.250000in}{0.550000in}}{\pgfqpoint{7.750000in}{3.850000in}}%
\pgfusepath{clip}%
\pgfsetbuttcap%
\pgfsetmiterjoin%
\definecolor{currentfill}{rgb}{0.501961,0.501961,0.501961}%
\pgfsetfillcolor{currentfill}%
\pgfsetfillopacity{0.700000}%
\pgfsetlinewidth{0.000000pt}%
\definecolor{currentstroke}{rgb}{0.000000,0.000000,0.000000}%
\pgfsetstrokecolor{currentstroke}%
\pgfsetstrokeopacity{0.700000}%
\pgfsetdash{}{0pt}%
\pgfpathmoveto{\pgfqpoint{5.455256in}{0.550000in}}%
\pgfpathlineto{\pgfqpoint{5.482777in}{0.550000in}}%
\pgfpathlineto{\pgfqpoint{5.482777in}{0.896700in}}%
\pgfpathlineto{\pgfqpoint{5.455256in}{0.896700in}}%
\pgfpathlineto{\pgfqpoint{5.455256in}{0.550000in}}%
\pgfpathclose%
\pgfusepath{fill}%
\end{pgfscope}%
\begin{pgfscope}%
\pgfpathrectangle{\pgfqpoint{1.250000in}{0.550000in}}{\pgfqpoint{7.750000in}{3.850000in}}%
\pgfusepath{clip}%
\pgfsetbuttcap%
\pgfsetmiterjoin%
\definecolor{currentfill}{rgb}{0.501961,0.501961,0.501961}%
\pgfsetfillcolor{currentfill}%
\pgfsetfillopacity{0.700000}%
\pgfsetlinewidth{0.000000pt}%
\definecolor{currentstroke}{rgb}{0.000000,0.000000,0.000000}%
\pgfsetstrokecolor{currentstroke}%
\pgfsetstrokeopacity{0.700000}%
\pgfsetdash{}{0pt}%
\pgfpathmoveto{\pgfqpoint{5.482777in}{0.550000in}}%
\pgfpathlineto{\pgfqpoint{5.510298in}{0.550000in}}%
\pgfpathlineto{\pgfqpoint{5.510298in}{0.550000in}}%
\pgfpathlineto{\pgfqpoint{5.482777in}{0.550000in}}%
\pgfpathlineto{\pgfqpoint{5.482777in}{0.550000in}}%
\pgfpathclose%
\pgfusepath{fill}%
\end{pgfscope}%
\begin{pgfscope}%
\pgfpathrectangle{\pgfqpoint{1.250000in}{0.550000in}}{\pgfqpoint{7.750000in}{3.850000in}}%
\pgfusepath{clip}%
\pgfsetbuttcap%
\pgfsetmiterjoin%
\definecolor{currentfill}{rgb}{0.501961,0.501961,0.501961}%
\pgfsetfillcolor{currentfill}%
\pgfsetfillopacity{0.700000}%
\pgfsetlinewidth{0.000000pt}%
\definecolor{currentstroke}{rgb}{0.000000,0.000000,0.000000}%
\pgfsetstrokecolor{currentstroke}%
\pgfsetstrokeopacity{0.700000}%
\pgfsetdash{}{0pt}%
\pgfpathmoveto{\pgfqpoint{5.510298in}{0.550000in}}%
\pgfpathlineto{\pgfqpoint{5.537820in}{0.550000in}}%
\pgfpathlineto{\pgfqpoint{5.537820in}{0.883360in}}%
\pgfpathlineto{\pgfqpoint{5.510298in}{0.883360in}}%
\pgfpathlineto{\pgfqpoint{5.510298in}{0.550000in}}%
\pgfpathclose%
\pgfusepath{fill}%
\end{pgfscope}%
\begin{pgfscope}%
\pgfpathrectangle{\pgfqpoint{1.250000in}{0.550000in}}{\pgfqpoint{7.750000in}{3.850000in}}%
\pgfusepath{clip}%
\pgfsetbuttcap%
\pgfsetmiterjoin%
\definecolor{currentfill}{rgb}{0.501961,0.501961,0.501961}%
\pgfsetfillcolor{currentfill}%
\pgfsetfillopacity{0.700000}%
\pgfsetlinewidth{0.000000pt}%
\definecolor{currentstroke}{rgb}{0.000000,0.000000,0.000000}%
\pgfsetstrokecolor{currentstroke}%
\pgfsetstrokeopacity{0.700000}%
\pgfsetdash{}{0pt}%
\pgfpathmoveto{\pgfqpoint{5.537820in}{0.550000in}}%
\pgfpathlineto{\pgfqpoint{5.565341in}{0.550000in}}%
\pgfpathlineto{\pgfqpoint{5.565341in}{0.550000in}}%
\pgfpathlineto{\pgfqpoint{5.537820in}{0.550000in}}%
\pgfpathlineto{\pgfqpoint{5.537820in}{0.550000in}}%
\pgfpathclose%
\pgfusepath{fill}%
\end{pgfscope}%
\begin{pgfscope}%
\pgfpathrectangle{\pgfqpoint{1.250000in}{0.550000in}}{\pgfqpoint{7.750000in}{3.850000in}}%
\pgfusepath{clip}%
\pgfsetbuttcap%
\pgfsetmiterjoin%
\definecolor{currentfill}{rgb}{0.501961,0.501961,0.501961}%
\pgfsetfillcolor{currentfill}%
\pgfsetfillopacity{0.700000}%
\pgfsetlinewidth{0.000000pt}%
\definecolor{currentstroke}{rgb}{0.000000,0.000000,0.000000}%
\pgfsetstrokecolor{currentstroke}%
\pgfsetstrokeopacity{0.700000}%
\pgfsetdash{}{0pt}%
\pgfpathmoveto{\pgfqpoint{5.565341in}{0.550000in}}%
\pgfpathlineto{\pgfqpoint{5.592862in}{0.550000in}}%
\pgfpathlineto{\pgfqpoint{5.592862in}{0.869149in}}%
\pgfpathlineto{\pgfqpoint{5.565341in}{0.869149in}}%
\pgfpathlineto{\pgfqpoint{5.565341in}{0.550000in}}%
\pgfpathclose%
\pgfusepath{fill}%
\end{pgfscope}%
\begin{pgfscope}%
\pgfpathrectangle{\pgfqpoint{1.250000in}{0.550000in}}{\pgfqpoint{7.750000in}{3.850000in}}%
\pgfusepath{clip}%
\pgfsetbuttcap%
\pgfsetmiterjoin%
\definecolor{currentfill}{rgb}{0.501961,0.501961,0.501961}%
\pgfsetfillcolor{currentfill}%
\pgfsetfillopacity{0.700000}%
\pgfsetlinewidth{0.000000pt}%
\definecolor{currentstroke}{rgb}{0.000000,0.000000,0.000000}%
\pgfsetstrokecolor{currentstroke}%
\pgfsetstrokeopacity{0.700000}%
\pgfsetdash{}{0pt}%
\pgfpathmoveto{\pgfqpoint{5.592862in}{0.550000in}}%
\pgfpathlineto{\pgfqpoint{5.620384in}{0.550000in}}%
\pgfpathlineto{\pgfqpoint{5.620384in}{0.550000in}}%
\pgfpathlineto{\pgfqpoint{5.592862in}{0.550000in}}%
\pgfpathlineto{\pgfqpoint{5.592862in}{0.550000in}}%
\pgfpathclose%
\pgfusepath{fill}%
\end{pgfscope}%
\begin{pgfscope}%
\pgfpathrectangle{\pgfqpoint{1.250000in}{0.550000in}}{\pgfqpoint{7.750000in}{3.850000in}}%
\pgfusepath{clip}%
\pgfsetbuttcap%
\pgfsetmiterjoin%
\definecolor{currentfill}{rgb}{0.501961,0.501961,0.501961}%
\pgfsetfillcolor{currentfill}%
\pgfsetfillopacity{0.700000}%
\pgfsetlinewidth{0.000000pt}%
\definecolor{currentstroke}{rgb}{0.000000,0.000000,0.000000}%
\pgfsetstrokecolor{currentstroke}%
\pgfsetstrokeopacity{0.700000}%
\pgfsetdash{}{0pt}%
\pgfpathmoveto{\pgfqpoint{5.620384in}{0.550000in}}%
\pgfpathlineto{\pgfqpoint{5.647905in}{0.550000in}}%
\pgfpathlineto{\pgfqpoint{5.647905in}{0.854794in}}%
\pgfpathlineto{\pgfqpoint{5.620384in}{0.854794in}}%
\pgfpathlineto{\pgfqpoint{5.620384in}{0.550000in}}%
\pgfpathclose%
\pgfusepath{fill}%
\end{pgfscope}%
\begin{pgfscope}%
\pgfpathrectangle{\pgfqpoint{1.250000in}{0.550000in}}{\pgfqpoint{7.750000in}{3.850000in}}%
\pgfusepath{clip}%
\pgfsetbuttcap%
\pgfsetmiterjoin%
\definecolor{currentfill}{rgb}{0.501961,0.501961,0.501961}%
\pgfsetfillcolor{currentfill}%
\pgfsetfillopacity{0.700000}%
\pgfsetlinewidth{0.000000pt}%
\definecolor{currentstroke}{rgb}{0.000000,0.000000,0.000000}%
\pgfsetstrokecolor{currentstroke}%
\pgfsetstrokeopacity{0.700000}%
\pgfsetdash{}{0pt}%
\pgfpathmoveto{\pgfqpoint{5.647905in}{0.550000in}}%
\pgfpathlineto{\pgfqpoint{5.675426in}{0.550000in}}%
\pgfpathlineto{\pgfqpoint{5.675426in}{0.550000in}}%
\pgfpathlineto{\pgfqpoint{5.647905in}{0.550000in}}%
\pgfpathlineto{\pgfqpoint{5.647905in}{0.550000in}}%
\pgfpathclose%
\pgfusepath{fill}%
\end{pgfscope}%
\begin{pgfscope}%
\pgfpathrectangle{\pgfqpoint{1.250000in}{0.550000in}}{\pgfqpoint{7.750000in}{3.850000in}}%
\pgfusepath{clip}%
\pgfsetbuttcap%
\pgfsetmiterjoin%
\definecolor{currentfill}{rgb}{0.501961,0.501961,0.501961}%
\pgfsetfillcolor{currentfill}%
\pgfsetfillopacity{0.700000}%
\pgfsetlinewidth{0.000000pt}%
\definecolor{currentstroke}{rgb}{0.000000,0.000000,0.000000}%
\pgfsetstrokecolor{currentstroke}%
\pgfsetstrokeopacity{0.700000}%
\pgfsetdash{}{0pt}%
\pgfpathmoveto{\pgfqpoint{5.675426in}{0.550000in}}%
\pgfpathlineto{\pgfqpoint{5.702947in}{0.550000in}}%
\pgfpathlineto{\pgfqpoint{5.702947in}{0.838264in}}%
\pgfpathlineto{\pgfqpoint{5.675426in}{0.838264in}}%
\pgfpathlineto{\pgfqpoint{5.675426in}{0.550000in}}%
\pgfpathclose%
\pgfusepath{fill}%
\end{pgfscope}%
\begin{pgfscope}%
\pgfpathrectangle{\pgfqpoint{1.250000in}{0.550000in}}{\pgfqpoint{7.750000in}{3.850000in}}%
\pgfusepath{clip}%
\pgfsetbuttcap%
\pgfsetmiterjoin%
\definecolor{currentfill}{rgb}{0.501961,0.501961,0.501961}%
\pgfsetfillcolor{currentfill}%
\pgfsetfillopacity{0.700000}%
\pgfsetlinewidth{0.000000pt}%
\definecolor{currentstroke}{rgb}{0.000000,0.000000,0.000000}%
\pgfsetstrokecolor{currentstroke}%
\pgfsetstrokeopacity{0.700000}%
\pgfsetdash{}{0pt}%
\pgfpathmoveto{\pgfqpoint{5.702947in}{0.550000in}}%
\pgfpathlineto{\pgfqpoint{5.730469in}{0.550000in}}%
\pgfpathlineto{\pgfqpoint{5.730469in}{0.550000in}}%
\pgfpathlineto{\pgfqpoint{5.702947in}{0.550000in}}%
\pgfpathlineto{\pgfqpoint{5.702947in}{0.550000in}}%
\pgfpathclose%
\pgfusepath{fill}%
\end{pgfscope}%
\begin{pgfscope}%
\pgfpathrectangle{\pgfqpoint{1.250000in}{0.550000in}}{\pgfqpoint{7.750000in}{3.850000in}}%
\pgfusepath{clip}%
\pgfsetbuttcap%
\pgfsetmiterjoin%
\definecolor{currentfill}{rgb}{0.501961,0.501961,0.501961}%
\pgfsetfillcolor{currentfill}%
\pgfsetfillopacity{0.700000}%
\pgfsetlinewidth{0.000000pt}%
\definecolor{currentstroke}{rgb}{0.000000,0.000000,0.000000}%
\pgfsetstrokecolor{currentstroke}%
\pgfsetstrokeopacity{0.700000}%
\pgfsetdash{}{0pt}%
\pgfpathmoveto{\pgfqpoint{5.730469in}{0.550000in}}%
\pgfpathlineto{\pgfqpoint{5.757990in}{0.550000in}}%
\pgfpathlineto{\pgfqpoint{5.757990in}{0.842469in}}%
\pgfpathlineto{\pgfqpoint{5.730469in}{0.842469in}}%
\pgfpathlineto{\pgfqpoint{5.730469in}{0.550000in}}%
\pgfpathclose%
\pgfusepath{fill}%
\end{pgfscope}%
\begin{pgfscope}%
\pgfpathrectangle{\pgfqpoint{1.250000in}{0.550000in}}{\pgfqpoint{7.750000in}{3.850000in}}%
\pgfusepath{clip}%
\pgfsetbuttcap%
\pgfsetmiterjoin%
\definecolor{currentfill}{rgb}{0.501961,0.501961,0.501961}%
\pgfsetfillcolor{currentfill}%
\pgfsetfillopacity{0.700000}%
\pgfsetlinewidth{0.000000pt}%
\definecolor{currentstroke}{rgb}{0.000000,0.000000,0.000000}%
\pgfsetstrokecolor{currentstroke}%
\pgfsetstrokeopacity{0.700000}%
\pgfsetdash{}{0pt}%
\pgfpathmoveto{\pgfqpoint{5.757990in}{0.550000in}}%
\pgfpathlineto{\pgfqpoint{5.785511in}{0.550000in}}%
\pgfpathlineto{\pgfqpoint{5.785511in}{0.823764in}}%
\pgfpathlineto{\pgfqpoint{5.757990in}{0.823764in}}%
\pgfpathlineto{\pgfqpoint{5.757990in}{0.550000in}}%
\pgfpathclose%
\pgfusepath{fill}%
\end{pgfscope}%
\begin{pgfscope}%
\pgfpathrectangle{\pgfqpoint{1.250000in}{0.550000in}}{\pgfqpoint{7.750000in}{3.850000in}}%
\pgfusepath{clip}%
\pgfsetbuttcap%
\pgfsetmiterjoin%
\definecolor{currentfill}{rgb}{0.501961,0.501961,0.501961}%
\pgfsetfillcolor{currentfill}%
\pgfsetfillopacity{0.700000}%
\pgfsetlinewidth{0.000000pt}%
\definecolor{currentstroke}{rgb}{0.000000,0.000000,0.000000}%
\pgfsetstrokecolor{currentstroke}%
\pgfsetstrokeopacity{0.700000}%
\pgfsetdash{}{0pt}%
\pgfpathmoveto{\pgfqpoint{5.785511in}{0.550000in}}%
\pgfpathlineto{\pgfqpoint{5.813033in}{0.550000in}}%
\pgfpathlineto{\pgfqpoint{5.813033in}{0.550000in}}%
\pgfpathlineto{\pgfqpoint{5.785511in}{0.550000in}}%
\pgfpathlineto{\pgfqpoint{5.785511in}{0.550000in}}%
\pgfpathclose%
\pgfusepath{fill}%
\end{pgfscope}%
\begin{pgfscope}%
\pgfpathrectangle{\pgfqpoint{1.250000in}{0.550000in}}{\pgfqpoint{7.750000in}{3.850000in}}%
\pgfusepath{clip}%
\pgfsetbuttcap%
\pgfsetmiterjoin%
\definecolor{currentfill}{rgb}{0.501961,0.501961,0.501961}%
\pgfsetfillcolor{currentfill}%
\pgfsetfillopacity{0.700000}%
\pgfsetlinewidth{0.000000pt}%
\definecolor{currentstroke}{rgb}{0.000000,0.000000,0.000000}%
\pgfsetstrokecolor{currentstroke}%
\pgfsetstrokeopacity{0.700000}%
\pgfsetdash{}{0pt}%
\pgfpathmoveto{\pgfqpoint{5.813033in}{0.550000in}}%
\pgfpathlineto{\pgfqpoint{5.840554in}{0.550000in}}%
\pgfpathlineto{\pgfqpoint{5.840554in}{0.816949in}}%
\pgfpathlineto{\pgfqpoint{5.813033in}{0.816949in}}%
\pgfpathlineto{\pgfqpoint{5.813033in}{0.550000in}}%
\pgfpathclose%
\pgfusepath{fill}%
\end{pgfscope}%
\begin{pgfscope}%
\pgfpathrectangle{\pgfqpoint{1.250000in}{0.550000in}}{\pgfqpoint{7.750000in}{3.850000in}}%
\pgfusepath{clip}%
\pgfsetbuttcap%
\pgfsetmiterjoin%
\definecolor{currentfill}{rgb}{0.501961,0.501961,0.501961}%
\pgfsetfillcolor{currentfill}%
\pgfsetfillopacity{0.700000}%
\pgfsetlinewidth{0.000000pt}%
\definecolor{currentstroke}{rgb}{0.000000,0.000000,0.000000}%
\pgfsetstrokecolor{currentstroke}%
\pgfsetstrokeopacity{0.700000}%
\pgfsetdash{}{0pt}%
\pgfpathmoveto{\pgfqpoint{5.840554in}{0.550000in}}%
\pgfpathlineto{\pgfqpoint{5.868075in}{0.550000in}}%
\pgfpathlineto{\pgfqpoint{5.868075in}{0.550000in}}%
\pgfpathlineto{\pgfqpoint{5.840554in}{0.550000in}}%
\pgfpathlineto{\pgfqpoint{5.840554in}{0.550000in}}%
\pgfpathclose%
\pgfusepath{fill}%
\end{pgfscope}%
\begin{pgfscope}%
\pgfpathrectangle{\pgfqpoint{1.250000in}{0.550000in}}{\pgfqpoint{7.750000in}{3.850000in}}%
\pgfusepath{clip}%
\pgfsetbuttcap%
\pgfsetmiterjoin%
\definecolor{currentfill}{rgb}{0.501961,0.501961,0.501961}%
\pgfsetfillcolor{currentfill}%
\pgfsetfillopacity{0.700000}%
\pgfsetlinewidth{0.000000pt}%
\definecolor{currentstroke}{rgb}{0.000000,0.000000,0.000000}%
\pgfsetstrokecolor{currentstroke}%
\pgfsetstrokeopacity{0.700000}%
\pgfsetdash{}{0pt}%
\pgfpathmoveto{\pgfqpoint{5.868075in}{0.550000in}}%
\pgfpathlineto{\pgfqpoint{5.895597in}{0.550000in}}%
\pgfpathlineto{\pgfqpoint{5.895597in}{0.804914in}}%
\pgfpathlineto{\pgfqpoint{5.868075in}{0.804914in}}%
\pgfpathlineto{\pgfqpoint{5.868075in}{0.550000in}}%
\pgfpathclose%
\pgfusepath{fill}%
\end{pgfscope}%
\begin{pgfscope}%
\pgfpathrectangle{\pgfqpoint{1.250000in}{0.550000in}}{\pgfqpoint{7.750000in}{3.850000in}}%
\pgfusepath{clip}%
\pgfsetbuttcap%
\pgfsetmiterjoin%
\definecolor{currentfill}{rgb}{0.501961,0.501961,0.501961}%
\pgfsetfillcolor{currentfill}%
\pgfsetfillopacity{0.700000}%
\pgfsetlinewidth{0.000000pt}%
\definecolor{currentstroke}{rgb}{0.000000,0.000000,0.000000}%
\pgfsetstrokecolor{currentstroke}%
\pgfsetstrokeopacity{0.700000}%
\pgfsetdash{}{0pt}%
\pgfpathmoveto{\pgfqpoint{5.895597in}{0.550000in}}%
\pgfpathlineto{\pgfqpoint{5.923118in}{0.550000in}}%
\pgfpathlineto{\pgfqpoint{5.923118in}{0.794763in}}%
\pgfpathlineto{\pgfqpoint{5.895597in}{0.794763in}}%
\pgfpathlineto{\pgfqpoint{5.895597in}{0.550000in}}%
\pgfpathclose%
\pgfusepath{fill}%
\end{pgfscope}%
\begin{pgfscope}%
\pgfpathrectangle{\pgfqpoint{1.250000in}{0.550000in}}{\pgfqpoint{7.750000in}{3.850000in}}%
\pgfusepath{clip}%
\pgfsetbuttcap%
\pgfsetmiterjoin%
\definecolor{currentfill}{rgb}{0.501961,0.501961,0.501961}%
\pgfsetfillcolor{currentfill}%
\pgfsetfillopacity{0.700000}%
\pgfsetlinewidth{0.000000pt}%
\definecolor{currentstroke}{rgb}{0.000000,0.000000,0.000000}%
\pgfsetstrokecolor{currentstroke}%
\pgfsetstrokeopacity{0.700000}%
\pgfsetdash{}{0pt}%
\pgfpathmoveto{\pgfqpoint{5.923118in}{0.550000in}}%
\pgfpathlineto{\pgfqpoint{5.950639in}{0.550000in}}%
\pgfpathlineto{\pgfqpoint{5.950639in}{0.550000in}}%
\pgfpathlineto{\pgfqpoint{5.923118in}{0.550000in}}%
\pgfpathlineto{\pgfqpoint{5.923118in}{0.550000in}}%
\pgfpathclose%
\pgfusepath{fill}%
\end{pgfscope}%
\begin{pgfscope}%
\pgfpathrectangle{\pgfqpoint{1.250000in}{0.550000in}}{\pgfqpoint{7.750000in}{3.850000in}}%
\pgfusepath{clip}%
\pgfsetbuttcap%
\pgfsetmiterjoin%
\definecolor{currentfill}{rgb}{0.501961,0.501961,0.501961}%
\pgfsetfillcolor{currentfill}%
\pgfsetfillopacity{0.700000}%
\pgfsetlinewidth{0.000000pt}%
\definecolor{currentstroke}{rgb}{0.000000,0.000000,0.000000}%
\pgfsetstrokecolor{currentstroke}%
\pgfsetstrokeopacity{0.700000}%
\pgfsetdash{}{0pt}%
\pgfpathmoveto{\pgfqpoint{5.950639in}{0.550000in}}%
\pgfpathlineto{\pgfqpoint{5.978161in}{0.550000in}}%
\pgfpathlineto{\pgfqpoint{5.978161in}{0.796503in}}%
\pgfpathlineto{\pgfqpoint{5.950639in}{0.796503in}}%
\pgfpathlineto{\pgfqpoint{5.950639in}{0.550000in}}%
\pgfpathclose%
\pgfusepath{fill}%
\end{pgfscope}%
\begin{pgfscope}%
\pgfpathrectangle{\pgfqpoint{1.250000in}{0.550000in}}{\pgfqpoint{7.750000in}{3.850000in}}%
\pgfusepath{clip}%
\pgfsetbuttcap%
\pgfsetmiterjoin%
\definecolor{currentfill}{rgb}{0.501961,0.501961,0.501961}%
\pgfsetfillcolor{currentfill}%
\pgfsetfillopacity{0.700000}%
\pgfsetlinewidth{0.000000pt}%
\definecolor{currentstroke}{rgb}{0.000000,0.000000,0.000000}%
\pgfsetstrokecolor{currentstroke}%
\pgfsetstrokeopacity{0.700000}%
\pgfsetdash{}{0pt}%
\pgfpathmoveto{\pgfqpoint{5.978161in}{0.550000in}}%
\pgfpathlineto{\pgfqpoint{6.005682in}{0.550000in}}%
\pgfpathlineto{\pgfqpoint{6.005682in}{0.782728in}}%
\pgfpathlineto{\pgfqpoint{5.978161in}{0.782728in}}%
\pgfpathlineto{\pgfqpoint{5.978161in}{0.550000in}}%
\pgfpathclose%
\pgfusepath{fill}%
\end{pgfscope}%
\begin{pgfscope}%
\pgfpathrectangle{\pgfqpoint{1.250000in}{0.550000in}}{\pgfqpoint{7.750000in}{3.850000in}}%
\pgfusepath{clip}%
\pgfsetbuttcap%
\pgfsetmiterjoin%
\definecolor{currentfill}{rgb}{0.501961,0.501961,0.501961}%
\pgfsetfillcolor{currentfill}%
\pgfsetfillopacity{0.700000}%
\pgfsetlinewidth{0.000000pt}%
\definecolor{currentstroke}{rgb}{0.000000,0.000000,0.000000}%
\pgfsetstrokecolor{currentstroke}%
\pgfsetstrokeopacity{0.700000}%
\pgfsetdash{}{0pt}%
\pgfpathmoveto{\pgfqpoint{6.005682in}{0.550000in}}%
\pgfpathlineto{\pgfqpoint{6.033203in}{0.550000in}}%
\pgfpathlineto{\pgfqpoint{6.033203in}{0.550000in}}%
\pgfpathlineto{\pgfqpoint{6.005682in}{0.550000in}}%
\pgfpathlineto{\pgfqpoint{6.005682in}{0.550000in}}%
\pgfpathclose%
\pgfusepath{fill}%
\end{pgfscope}%
\begin{pgfscope}%
\pgfpathrectangle{\pgfqpoint{1.250000in}{0.550000in}}{\pgfqpoint{7.750000in}{3.850000in}}%
\pgfusepath{clip}%
\pgfsetbuttcap%
\pgfsetmiterjoin%
\definecolor{currentfill}{rgb}{0.501961,0.501961,0.501961}%
\pgfsetfillcolor{currentfill}%
\pgfsetfillopacity{0.700000}%
\pgfsetlinewidth{0.000000pt}%
\definecolor{currentstroke}{rgb}{0.000000,0.000000,0.000000}%
\pgfsetstrokecolor{currentstroke}%
\pgfsetstrokeopacity{0.700000}%
\pgfsetdash{}{0pt}%
\pgfpathmoveto{\pgfqpoint{6.033203in}{0.550000in}}%
\pgfpathlineto{\pgfqpoint{6.060724in}{0.550000in}}%
\pgfpathlineto{\pgfqpoint{6.060724in}{0.779393in}}%
\pgfpathlineto{\pgfqpoint{6.033203in}{0.779393in}}%
\pgfpathlineto{\pgfqpoint{6.033203in}{0.550000in}}%
\pgfpathclose%
\pgfusepath{fill}%
\end{pgfscope}%
\begin{pgfscope}%
\pgfpathrectangle{\pgfqpoint{1.250000in}{0.550000in}}{\pgfqpoint{7.750000in}{3.850000in}}%
\pgfusepath{clip}%
\pgfsetbuttcap%
\pgfsetmiterjoin%
\definecolor{currentfill}{rgb}{0.501961,0.501961,0.501961}%
\pgfsetfillcolor{currentfill}%
\pgfsetfillopacity{0.700000}%
\pgfsetlinewidth{0.000000pt}%
\definecolor{currentstroke}{rgb}{0.000000,0.000000,0.000000}%
\pgfsetstrokecolor{currentstroke}%
\pgfsetstrokeopacity{0.700000}%
\pgfsetdash{}{0pt}%
\pgfpathmoveto{\pgfqpoint{6.060724in}{0.550000in}}%
\pgfpathlineto{\pgfqpoint{6.088246in}{0.550000in}}%
\pgfpathlineto{\pgfqpoint{6.088246in}{0.763733in}}%
\pgfpathlineto{\pgfqpoint{6.060724in}{0.763733in}}%
\pgfpathlineto{\pgfqpoint{6.060724in}{0.550000in}}%
\pgfpathclose%
\pgfusepath{fill}%
\end{pgfscope}%
\begin{pgfscope}%
\pgfpathrectangle{\pgfqpoint{1.250000in}{0.550000in}}{\pgfqpoint{7.750000in}{3.850000in}}%
\pgfusepath{clip}%
\pgfsetbuttcap%
\pgfsetmiterjoin%
\definecolor{currentfill}{rgb}{0.501961,0.501961,0.501961}%
\pgfsetfillcolor{currentfill}%
\pgfsetfillopacity{0.700000}%
\pgfsetlinewidth{0.000000pt}%
\definecolor{currentstroke}{rgb}{0.000000,0.000000,0.000000}%
\pgfsetstrokecolor{currentstroke}%
\pgfsetstrokeopacity{0.700000}%
\pgfsetdash{}{0pt}%
\pgfpathmoveto{\pgfqpoint{6.088246in}{0.550000in}}%
\pgfpathlineto{\pgfqpoint{6.115767in}{0.550000in}}%
\pgfpathlineto{\pgfqpoint{6.115767in}{0.762573in}}%
\pgfpathlineto{\pgfqpoint{6.088246in}{0.762573in}}%
\pgfpathlineto{\pgfqpoint{6.088246in}{0.550000in}}%
\pgfpathclose%
\pgfusepath{fill}%
\end{pgfscope}%
\begin{pgfscope}%
\pgfpathrectangle{\pgfqpoint{1.250000in}{0.550000in}}{\pgfqpoint{7.750000in}{3.850000in}}%
\pgfusepath{clip}%
\pgfsetbuttcap%
\pgfsetmiterjoin%
\definecolor{currentfill}{rgb}{0.501961,0.501961,0.501961}%
\pgfsetfillcolor{currentfill}%
\pgfsetfillopacity{0.700000}%
\pgfsetlinewidth{0.000000pt}%
\definecolor{currentstroke}{rgb}{0.000000,0.000000,0.000000}%
\pgfsetstrokecolor{currentstroke}%
\pgfsetstrokeopacity{0.700000}%
\pgfsetdash{}{0pt}%
\pgfpathmoveto{\pgfqpoint{6.115767in}{0.550000in}}%
\pgfpathlineto{\pgfqpoint{6.143288in}{0.550000in}}%
\pgfpathlineto{\pgfqpoint{6.143288in}{0.550000in}}%
\pgfpathlineto{\pgfqpoint{6.115767in}{0.550000in}}%
\pgfpathlineto{\pgfqpoint{6.115767in}{0.550000in}}%
\pgfpathclose%
\pgfusepath{fill}%
\end{pgfscope}%
\begin{pgfscope}%
\pgfpathrectangle{\pgfqpoint{1.250000in}{0.550000in}}{\pgfqpoint{7.750000in}{3.850000in}}%
\pgfusepath{clip}%
\pgfsetbuttcap%
\pgfsetmiterjoin%
\definecolor{currentfill}{rgb}{0.501961,0.501961,0.501961}%
\pgfsetfillcolor{currentfill}%
\pgfsetfillopacity{0.700000}%
\pgfsetlinewidth{0.000000pt}%
\definecolor{currentstroke}{rgb}{0.000000,0.000000,0.000000}%
\pgfsetstrokecolor{currentstroke}%
\pgfsetstrokeopacity{0.700000}%
\pgfsetdash{}{0pt}%
\pgfpathmoveto{\pgfqpoint{6.143288in}{0.550000in}}%
\pgfpathlineto{\pgfqpoint{6.170810in}{0.550000in}}%
\pgfpathlineto{\pgfqpoint{6.170810in}{0.749958in}}%
\pgfpathlineto{\pgfqpoint{6.143288in}{0.749958in}}%
\pgfpathlineto{\pgfqpoint{6.143288in}{0.550000in}}%
\pgfpathclose%
\pgfusepath{fill}%
\end{pgfscope}%
\begin{pgfscope}%
\pgfpathrectangle{\pgfqpoint{1.250000in}{0.550000in}}{\pgfqpoint{7.750000in}{3.850000in}}%
\pgfusepath{clip}%
\pgfsetbuttcap%
\pgfsetmiterjoin%
\definecolor{currentfill}{rgb}{0.501961,0.501961,0.501961}%
\pgfsetfillcolor{currentfill}%
\pgfsetfillopacity{0.700000}%
\pgfsetlinewidth{0.000000pt}%
\definecolor{currentstroke}{rgb}{0.000000,0.000000,0.000000}%
\pgfsetstrokecolor{currentstroke}%
\pgfsetstrokeopacity{0.700000}%
\pgfsetdash{}{0pt}%
\pgfpathmoveto{\pgfqpoint{6.170810in}{0.550000in}}%
\pgfpathlineto{\pgfqpoint{6.198331in}{0.550000in}}%
\pgfpathlineto{\pgfqpoint{6.198331in}{0.736328in}}%
\pgfpathlineto{\pgfqpoint{6.170810in}{0.736328in}}%
\pgfpathlineto{\pgfqpoint{6.170810in}{0.550000in}}%
\pgfpathclose%
\pgfusepath{fill}%
\end{pgfscope}%
\begin{pgfscope}%
\pgfpathrectangle{\pgfqpoint{1.250000in}{0.550000in}}{\pgfqpoint{7.750000in}{3.850000in}}%
\pgfusepath{clip}%
\pgfsetbuttcap%
\pgfsetmiterjoin%
\definecolor{currentfill}{rgb}{0.501961,0.501961,0.501961}%
\pgfsetfillcolor{currentfill}%
\pgfsetfillopacity{0.700000}%
\pgfsetlinewidth{0.000000pt}%
\definecolor{currentstroke}{rgb}{0.000000,0.000000,0.000000}%
\pgfsetstrokecolor{currentstroke}%
\pgfsetstrokeopacity{0.700000}%
\pgfsetdash{}{0pt}%
\pgfpathmoveto{\pgfqpoint{6.198331in}{0.550000in}}%
\pgfpathlineto{\pgfqpoint{6.225852in}{0.550000in}}%
\pgfpathlineto{\pgfqpoint{6.225852in}{0.726467in}}%
\pgfpathlineto{\pgfqpoint{6.198331in}{0.726467in}}%
\pgfpathlineto{\pgfqpoint{6.198331in}{0.550000in}}%
\pgfpathclose%
\pgfusepath{fill}%
\end{pgfscope}%
\begin{pgfscope}%
\pgfpathrectangle{\pgfqpoint{1.250000in}{0.550000in}}{\pgfqpoint{7.750000in}{3.850000in}}%
\pgfusepath{clip}%
\pgfsetbuttcap%
\pgfsetmiterjoin%
\definecolor{currentfill}{rgb}{0.501961,0.501961,0.501961}%
\pgfsetfillcolor{currentfill}%
\pgfsetfillopacity{0.700000}%
\pgfsetlinewidth{0.000000pt}%
\definecolor{currentstroke}{rgb}{0.000000,0.000000,0.000000}%
\pgfsetstrokecolor{currentstroke}%
\pgfsetstrokeopacity{0.700000}%
\pgfsetdash{}{0pt}%
\pgfpathmoveto{\pgfqpoint{6.225852in}{0.550000in}}%
\pgfpathlineto{\pgfqpoint{6.253374in}{0.550000in}}%
\pgfpathlineto{\pgfqpoint{6.253374in}{0.732413in}}%
\pgfpathlineto{\pgfqpoint{6.225852in}{0.732413in}}%
\pgfpathlineto{\pgfqpoint{6.225852in}{0.550000in}}%
\pgfpathclose%
\pgfusepath{fill}%
\end{pgfscope}%
\begin{pgfscope}%
\pgfpathrectangle{\pgfqpoint{1.250000in}{0.550000in}}{\pgfqpoint{7.750000in}{3.850000in}}%
\pgfusepath{clip}%
\pgfsetbuttcap%
\pgfsetmiterjoin%
\definecolor{currentfill}{rgb}{0.501961,0.501961,0.501961}%
\pgfsetfillcolor{currentfill}%
\pgfsetfillopacity{0.700000}%
\pgfsetlinewidth{0.000000pt}%
\definecolor{currentstroke}{rgb}{0.000000,0.000000,0.000000}%
\pgfsetstrokecolor{currentstroke}%
\pgfsetstrokeopacity{0.700000}%
\pgfsetdash{}{0pt}%
\pgfpathmoveto{\pgfqpoint{6.253374in}{0.550000in}}%
\pgfpathlineto{\pgfqpoint{6.280895in}{0.550000in}}%
\pgfpathlineto{\pgfqpoint{6.280895in}{0.721827in}}%
\pgfpathlineto{\pgfqpoint{6.253374in}{0.721827in}}%
\pgfpathlineto{\pgfqpoint{6.253374in}{0.550000in}}%
\pgfpathclose%
\pgfusepath{fill}%
\end{pgfscope}%
\begin{pgfscope}%
\pgfpathrectangle{\pgfqpoint{1.250000in}{0.550000in}}{\pgfqpoint{7.750000in}{3.850000in}}%
\pgfusepath{clip}%
\pgfsetbuttcap%
\pgfsetmiterjoin%
\definecolor{currentfill}{rgb}{0.501961,0.501961,0.501961}%
\pgfsetfillcolor{currentfill}%
\pgfsetfillopacity{0.700000}%
\pgfsetlinewidth{0.000000pt}%
\definecolor{currentstroke}{rgb}{0.000000,0.000000,0.000000}%
\pgfsetstrokecolor{currentstroke}%
\pgfsetstrokeopacity{0.700000}%
\pgfsetdash{}{0pt}%
\pgfpathmoveto{\pgfqpoint{6.280895in}{0.550000in}}%
\pgfpathlineto{\pgfqpoint{6.308416in}{0.550000in}}%
\pgfpathlineto{\pgfqpoint{6.308416in}{0.708487in}}%
\pgfpathlineto{\pgfqpoint{6.280895in}{0.708487in}}%
\pgfpathlineto{\pgfqpoint{6.280895in}{0.550000in}}%
\pgfpathclose%
\pgfusepath{fill}%
\end{pgfscope}%
\begin{pgfscope}%
\pgfpathrectangle{\pgfqpoint{1.250000in}{0.550000in}}{\pgfqpoint{7.750000in}{3.850000in}}%
\pgfusepath{clip}%
\pgfsetbuttcap%
\pgfsetmiterjoin%
\definecolor{currentfill}{rgb}{0.501961,0.501961,0.501961}%
\pgfsetfillcolor{currentfill}%
\pgfsetfillopacity{0.700000}%
\pgfsetlinewidth{0.000000pt}%
\definecolor{currentstroke}{rgb}{0.000000,0.000000,0.000000}%
\pgfsetstrokecolor{currentstroke}%
\pgfsetstrokeopacity{0.700000}%
\pgfsetdash{}{0pt}%
\pgfpathmoveto{\pgfqpoint{6.308416in}{0.550000in}}%
\pgfpathlineto{\pgfqpoint{6.335937in}{0.550000in}}%
\pgfpathlineto{\pgfqpoint{6.335937in}{0.704862in}}%
\pgfpathlineto{\pgfqpoint{6.308416in}{0.704862in}}%
\pgfpathlineto{\pgfqpoint{6.308416in}{0.550000in}}%
\pgfpathclose%
\pgfusepath{fill}%
\end{pgfscope}%
\begin{pgfscope}%
\pgfpathrectangle{\pgfqpoint{1.250000in}{0.550000in}}{\pgfqpoint{7.750000in}{3.850000in}}%
\pgfusepath{clip}%
\pgfsetbuttcap%
\pgfsetmiterjoin%
\definecolor{currentfill}{rgb}{0.501961,0.501961,0.501961}%
\pgfsetfillcolor{currentfill}%
\pgfsetfillopacity{0.700000}%
\pgfsetlinewidth{0.000000pt}%
\definecolor{currentstroke}{rgb}{0.000000,0.000000,0.000000}%
\pgfsetstrokecolor{currentstroke}%
\pgfsetstrokeopacity{0.700000}%
\pgfsetdash{}{0pt}%
\pgfpathmoveto{\pgfqpoint{6.335938in}{0.550000in}}%
\pgfpathlineto{\pgfqpoint{6.363459in}{0.550000in}}%
\pgfpathlineto{\pgfqpoint{6.363459in}{0.696017in}}%
\pgfpathlineto{\pgfqpoint{6.335938in}{0.696017in}}%
\pgfpathlineto{\pgfqpoint{6.335938in}{0.550000in}}%
\pgfpathclose%
\pgfusepath{fill}%
\end{pgfscope}%
\begin{pgfscope}%
\pgfpathrectangle{\pgfqpoint{1.250000in}{0.550000in}}{\pgfqpoint{7.750000in}{3.850000in}}%
\pgfusepath{clip}%
\pgfsetbuttcap%
\pgfsetmiterjoin%
\definecolor{currentfill}{rgb}{0.501961,0.501961,0.501961}%
\pgfsetfillcolor{currentfill}%
\pgfsetfillopacity{0.700000}%
\pgfsetlinewidth{0.000000pt}%
\definecolor{currentstroke}{rgb}{0.000000,0.000000,0.000000}%
\pgfsetstrokecolor{currentstroke}%
\pgfsetstrokeopacity{0.700000}%
\pgfsetdash{}{0pt}%
\pgfpathmoveto{\pgfqpoint{6.363459in}{0.550000in}}%
\pgfpathlineto{\pgfqpoint{6.390980in}{0.550000in}}%
\pgfpathlineto{\pgfqpoint{6.390980in}{0.699642in}}%
\pgfpathlineto{\pgfqpoint{6.363459in}{0.699642in}}%
\pgfpathlineto{\pgfqpoint{6.363459in}{0.550000in}}%
\pgfpathclose%
\pgfusepath{fill}%
\end{pgfscope}%
\begin{pgfscope}%
\pgfpathrectangle{\pgfqpoint{1.250000in}{0.550000in}}{\pgfqpoint{7.750000in}{3.850000in}}%
\pgfusepath{clip}%
\pgfsetbuttcap%
\pgfsetmiterjoin%
\definecolor{currentfill}{rgb}{0.501961,0.501961,0.501961}%
\pgfsetfillcolor{currentfill}%
\pgfsetfillopacity{0.700000}%
\pgfsetlinewidth{0.000000pt}%
\definecolor{currentstroke}{rgb}{0.000000,0.000000,0.000000}%
\pgfsetstrokecolor{currentstroke}%
\pgfsetstrokeopacity{0.700000}%
\pgfsetdash{}{0pt}%
\pgfpathmoveto{\pgfqpoint{6.390980in}{0.550000in}}%
\pgfpathlineto{\pgfqpoint{6.418501in}{0.550000in}}%
\pgfpathlineto{\pgfqpoint{6.418501in}{0.690797in}}%
\pgfpathlineto{\pgfqpoint{6.390980in}{0.690797in}}%
\pgfpathlineto{\pgfqpoint{6.390980in}{0.550000in}}%
\pgfpathclose%
\pgfusepath{fill}%
\end{pgfscope}%
\begin{pgfscope}%
\pgfpathrectangle{\pgfqpoint{1.250000in}{0.550000in}}{\pgfqpoint{7.750000in}{3.850000in}}%
\pgfusepath{clip}%
\pgfsetbuttcap%
\pgfsetmiterjoin%
\definecolor{currentfill}{rgb}{0.501961,0.501961,0.501961}%
\pgfsetfillcolor{currentfill}%
\pgfsetfillopacity{0.700000}%
\pgfsetlinewidth{0.000000pt}%
\definecolor{currentstroke}{rgb}{0.000000,0.000000,0.000000}%
\pgfsetstrokecolor{currentstroke}%
\pgfsetstrokeopacity{0.700000}%
\pgfsetdash{}{0pt}%
\pgfpathmoveto{\pgfqpoint{6.418501in}{0.550000in}}%
\pgfpathlineto{\pgfqpoint{6.446023in}{0.550000in}}%
\pgfpathlineto{\pgfqpoint{6.446023in}{0.822169in}}%
\pgfpathlineto{\pgfqpoint{6.418501in}{0.822169in}}%
\pgfpathlineto{\pgfqpoint{6.418501in}{0.550000in}}%
\pgfpathclose%
\pgfusepath{fill}%
\end{pgfscope}%
\begin{pgfscope}%
\pgfpathrectangle{\pgfqpoint{1.250000in}{0.550000in}}{\pgfqpoint{7.750000in}{3.850000in}}%
\pgfusepath{clip}%
\pgfsetbuttcap%
\pgfsetmiterjoin%
\definecolor{currentfill}{rgb}{0.501961,0.501961,0.501961}%
\pgfsetfillcolor{currentfill}%
\pgfsetfillopacity{0.700000}%
\pgfsetlinewidth{0.000000pt}%
\definecolor{currentstroke}{rgb}{0.000000,0.000000,0.000000}%
\pgfsetstrokecolor{currentstroke}%
\pgfsetstrokeopacity{0.700000}%
\pgfsetdash{}{0pt}%
\pgfpathmoveto{\pgfqpoint{6.446023in}{0.550000in}}%
\pgfpathlineto{\pgfqpoint{6.473544in}{0.550000in}}%
\pgfpathlineto{\pgfqpoint{6.473544in}{0.677602in}}%
\pgfpathlineto{\pgfqpoint{6.446023in}{0.677602in}}%
\pgfpathlineto{\pgfqpoint{6.446023in}{0.550000in}}%
\pgfpathclose%
\pgfusepath{fill}%
\end{pgfscope}%
\begin{pgfscope}%
\pgfpathrectangle{\pgfqpoint{1.250000in}{0.550000in}}{\pgfqpoint{7.750000in}{3.850000in}}%
\pgfusepath{clip}%
\pgfsetbuttcap%
\pgfsetmiterjoin%
\definecolor{currentfill}{rgb}{0.501961,0.501961,0.501961}%
\pgfsetfillcolor{currentfill}%
\pgfsetfillopacity{0.700000}%
\pgfsetlinewidth{0.000000pt}%
\definecolor{currentstroke}{rgb}{0.000000,0.000000,0.000000}%
\pgfsetstrokecolor{currentstroke}%
\pgfsetstrokeopacity{0.700000}%
\pgfsetdash{}{0pt}%
\pgfpathmoveto{\pgfqpoint{6.473544in}{0.550000in}}%
\pgfpathlineto{\pgfqpoint{6.501065in}{0.550000in}}%
\pgfpathlineto{\pgfqpoint{6.501065in}{0.678617in}}%
\pgfpathlineto{\pgfqpoint{6.473544in}{0.678617in}}%
\pgfpathlineto{\pgfqpoint{6.473544in}{0.550000in}}%
\pgfpathclose%
\pgfusepath{fill}%
\end{pgfscope}%
\begin{pgfscope}%
\pgfpathrectangle{\pgfqpoint{1.250000in}{0.550000in}}{\pgfqpoint{7.750000in}{3.850000in}}%
\pgfusepath{clip}%
\pgfsetbuttcap%
\pgfsetmiterjoin%
\definecolor{currentfill}{rgb}{0.501961,0.501961,0.501961}%
\pgfsetfillcolor{currentfill}%
\pgfsetfillopacity{0.700000}%
\pgfsetlinewidth{0.000000pt}%
\definecolor{currentstroke}{rgb}{0.000000,0.000000,0.000000}%
\pgfsetstrokecolor{currentstroke}%
\pgfsetstrokeopacity{0.700000}%
\pgfsetdash{}{0pt}%
\pgfpathmoveto{\pgfqpoint{6.501065in}{0.550000in}}%
\pgfpathlineto{\pgfqpoint{6.528587in}{0.550000in}}%
\pgfpathlineto{\pgfqpoint{6.528587in}{0.677312in}}%
\pgfpathlineto{\pgfqpoint{6.501065in}{0.677312in}}%
\pgfpathlineto{\pgfqpoint{6.501065in}{0.550000in}}%
\pgfpathclose%
\pgfusepath{fill}%
\end{pgfscope}%
\begin{pgfscope}%
\pgfpathrectangle{\pgfqpoint{1.250000in}{0.550000in}}{\pgfqpoint{7.750000in}{3.850000in}}%
\pgfusepath{clip}%
\pgfsetbuttcap%
\pgfsetmiterjoin%
\definecolor{currentfill}{rgb}{0.501961,0.501961,0.501961}%
\pgfsetfillcolor{currentfill}%
\pgfsetfillopacity{0.700000}%
\pgfsetlinewidth{0.000000pt}%
\definecolor{currentstroke}{rgb}{0.000000,0.000000,0.000000}%
\pgfsetstrokecolor{currentstroke}%
\pgfsetstrokeopacity{0.700000}%
\pgfsetdash{}{0pt}%
\pgfpathmoveto{\pgfqpoint{6.528587in}{0.550000in}}%
\pgfpathlineto{\pgfqpoint{6.556108in}{0.550000in}}%
\pgfpathlineto{\pgfqpoint{6.556108in}{0.799839in}}%
\pgfpathlineto{\pgfqpoint{6.528587in}{0.799839in}}%
\pgfpathlineto{\pgfqpoint{6.528587in}{0.550000in}}%
\pgfpathclose%
\pgfusepath{fill}%
\end{pgfscope}%
\begin{pgfscope}%
\pgfpathrectangle{\pgfqpoint{1.250000in}{0.550000in}}{\pgfqpoint{7.750000in}{3.850000in}}%
\pgfusepath{clip}%
\pgfsetbuttcap%
\pgfsetmiterjoin%
\definecolor{currentfill}{rgb}{0.501961,0.501961,0.501961}%
\pgfsetfillcolor{currentfill}%
\pgfsetfillopacity{0.700000}%
\pgfsetlinewidth{0.000000pt}%
\definecolor{currentstroke}{rgb}{0.000000,0.000000,0.000000}%
\pgfsetstrokecolor{currentstroke}%
\pgfsetstrokeopacity{0.700000}%
\pgfsetdash{}{0pt}%
\pgfpathmoveto{\pgfqpoint{6.556108in}{0.550000in}}%
\pgfpathlineto{\pgfqpoint{6.583629in}{0.550000in}}%
\pgfpathlineto{\pgfqpoint{6.583629in}{0.675282in}}%
\pgfpathlineto{\pgfqpoint{6.556108in}{0.675282in}}%
\pgfpathlineto{\pgfqpoint{6.556108in}{0.550000in}}%
\pgfpathclose%
\pgfusepath{fill}%
\end{pgfscope}%
\begin{pgfscope}%
\pgfpathrectangle{\pgfqpoint{1.250000in}{0.550000in}}{\pgfqpoint{7.750000in}{3.850000in}}%
\pgfusepath{clip}%
\pgfsetbuttcap%
\pgfsetmiterjoin%
\definecolor{currentfill}{rgb}{0.501961,0.501961,0.501961}%
\pgfsetfillcolor{currentfill}%
\pgfsetfillopacity{0.700000}%
\pgfsetlinewidth{0.000000pt}%
\definecolor{currentstroke}{rgb}{0.000000,0.000000,0.000000}%
\pgfsetstrokecolor{currentstroke}%
\pgfsetstrokeopacity{0.700000}%
\pgfsetdash{}{0pt}%
\pgfpathmoveto{\pgfqpoint{6.583629in}{0.550000in}}%
\pgfpathlineto{\pgfqpoint{6.611151in}{0.550000in}}%
\pgfpathlineto{\pgfqpoint{6.611151in}{0.663827in}}%
\pgfpathlineto{\pgfqpoint{6.583629in}{0.663827in}}%
\pgfpathlineto{\pgfqpoint{6.583629in}{0.550000in}}%
\pgfpathclose%
\pgfusepath{fill}%
\end{pgfscope}%
\begin{pgfscope}%
\pgfpathrectangle{\pgfqpoint{1.250000in}{0.550000in}}{\pgfqpoint{7.750000in}{3.850000in}}%
\pgfusepath{clip}%
\pgfsetbuttcap%
\pgfsetmiterjoin%
\definecolor{currentfill}{rgb}{0.501961,0.501961,0.501961}%
\pgfsetfillcolor{currentfill}%
\pgfsetfillopacity{0.700000}%
\pgfsetlinewidth{0.000000pt}%
\definecolor{currentstroke}{rgb}{0.000000,0.000000,0.000000}%
\pgfsetstrokecolor{currentstroke}%
\pgfsetstrokeopacity{0.700000}%
\pgfsetdash{}{0pt}%
\pgfpathmoveto{\pgfqpoint{6.611151in}{0.550000in}}%
\pgfpathlineto{\pgfqpoint{6.638672in}{0.550000in}}%
\pgfpathlineto{\pgfqpoint{6.638672in}{0.779828in}}%
\pgfpathlineto{\pgfqpoint{6.611151in}{0.779828in}}%
\pgfpathlineto{\pgfqpoint{6.611151in}{0.550000in}}%
\pgfpathclose%
\pgfusepath{fill}%
\end{pgfscope}%
\begin{pgfscope}%
\pgfpathrectangle{\pgfqpoint{1.250000in}{0.550000in}}{\pgfqpoint{7.750000in}{3.850000in}}%
\pgfusepath{clip}%
\pgfsetbuttcap%
\pgfsetmiterjoin%
\definecolor{currentfill}{rgb}{0.501961,0.501961,0.501961}%
\pgfsetfillcolor{currentfill}%
\pgfsetfillopacity{0.700000}%
\pgfsetlinewidth{0.000000pt}%
\definecolor{currentstroke}{rgb}{0.000000,0.000000,0.000000}%
\pgfsetstrokecolor{currentstroke}%
\pgfsetstrokeopacity{0.700000}%
\pgfsetdash{}{0pt}%
\pgfpathmoveto{\pgfqpoint{6.638672in}{0.550000in}}%
\pgfpathlineto{\pgfqpoint{6.666193in}{0.550000in}}%
\pgfpathlineto{\pgfqpoint{6.666193in}{0.664697in}}%
\pgfpathlineto{\pgfqpoint{6.638672in}{0.664697in}}%
\pgfpathlineto{\pgfqpoint{6.638672in}{0.550000in}}%
\pgfpathclose%
\pgfusepath{fill}%
\end{pgfscope}%
\begin{pgfscope}%
\pgfpathrectangle{\pgfqpoint{1.250000in}{0.550000in}}{\pgfqpoint{7.750000in}{3.850000in}}%
\pgfusepath{clip}%
\pgfsetbuttcap%
\pgfsetmiterjoin%
\definecolor{currentfill}{rgb}{0.501961,0.501961,0.501961}%
\pgfsetfillcolor{currentfill}%
\pgfsetfillopacity{0.700000}%
\pgfsetlinewidth{0.000000pt}%
\definecolor{currentstroke}{rgb}{0.000000,0.000000,0.000000}%
\pgfsetstrokecolor{currentstroke}%
\pgfsetstrokeopacity{0.700000}%
\pgfsetdash{}{0pt}%
\pgfpathmoveto{\pgfqpoint{6.666193in}{0.550000in}}%
\pgfpathlineto{\pgfqpoint{6.693714in}{0.550000in}}%
\pgfpathlineto{\pgfqpoint{6.693714in}{0.763153in}}%
\pgfpathlineto{\pgfqpoint{6.666193in}{0.763153in}}%
\pgfpathlineto{\pgfqpoint{6.666193in}{0.550000in}}%
\pgfpathclose%
\pgfusepath{fill}%
\end{pgfscope}%
\begin{pgfscope}%
\pgfpathrectangle{\pgfqpoint{1.250000in}{0.550000in}}{\pgfqpoint{7.750000in}{3.850000in}}%
\pgfusepath{clip}%
\pgfsetbuttcap%
\pgfsetmiterjoin%
\definecolor{currentfill}{rgb}{0.501961,0.501961,0.501961}%
\pgfsetfillcolor{currentfill}%
\pgfsetfillopacity{0.700000}%
\pgfsetlinewidth{0.000000pt}%
\definecolor{currentstroke}{rgb}{0.000000,0.000000,0.000000}%
\pgfsetstrokecolor{currentstroke}%
\pgfsetstrokeopacity{0.700000}%
\pgfsetdash{}{0pt}%
\pgfpathmoveto{\pgfqpoint{6.693714in}{0.550000in}}%
\pgfpathlineto{\pgfqpoint{6.721236in}{0.550000in}}%
\pgfpathlineto{\pgfqpoint{6.721236in}{0.658172in}}%
\pgfpathlineto{\pgfqpoint{6.693714in}{0.658172in}}%
\pgfpathlineto{\pgfqpoint{6.693714in}{0.550000in}}%
\pgfpathclose%
\pgfusepath{fill}%
\end{pgfscope}%
\begin{pgfscope}%
\pgfpathrectangle{\pgfqpoint{1.250000in}{0.550000in}}{\pgfqpoint{7.750000in}{3.850000in}}%
\pgfusepath{clip}%
\pgfsetbuttcap%
\pgfsetmiterjoin%
\definecolor{currentfill}{rgb}{0.501961,0.501961,0.501961}%
\pgfsetfillcolor{currentfill}%
\pgfsetfillopacity{0.700000}%
\pgfsetlinewidth{0.000000pt}%
\definecolor{currentstroke}{rgb}{0.000000,0.000000,0.000000}%
\pgfsetstrokecolor{currentstroke}%
\pgfsetstrokeopacity{0.700000}%
\pgfsetdash{}{0pt}%
\pgfpathmoveto{\pgfqpoint{6.721236in}{0.550000in}}%
\pgfpathlineto{\pgfqpoint{6.748757in}{0.550000in}}%
\pgfpathlineto{\pgfqpoint{6.748757in}{0.767503in}}%
\pgfpathlineto{\pgfqpoint{6.721236in}{0.767503in}}%
\pgfpathlineto{\pgfqpoint{6.721236in}{0.550000in}}%
\pgfpathclose%
\pgfusepath{fill}%
\end{pgfscope}%
\begin{pgfscope}%
\pgfpathrectangle{\pgfqpoint{1.250000in}{0.550000in}}{\pgfqpoint{7.750000in}{3.850000in}}%
\pgfusepath{clip}%
\pgfsetbuttcap%
\pgfsetmiterjoin%
\definecolor{currentfill}{rgb}{0.501961,0.501961,0.501961}%
\pgfsetfillcolor{currentfill}%
\pgfsetfillopacity{0.700000}%
\pgfsetlinewidth{0.000000pt}%
\definecolor{currentstroke}{rgb}{0.000000,0.000000,0.000000}%
\pgfsetstrokecolor{currentstroke}%
\pgfsetstrokeopacity{0.700000}%
\pgfsetdash{}{0pt}%
\pgfpathmoveto{\pgfqpoint{6.748757in}{0.550000in}}%
\pgfpathlineto{\pgfqpoint{6.776278in}{0.550000in}}%
\pgfpathlineto{\pgfqpoint{6.776278in}{0.651501in}}%
\pgfpathlineto{\pgfqpoint{6.748757in}{0.651501in}}%
\pgfpathlineto{\pgfqpoint{6.748757in}{0.550000in}}%
\pgfpathclose%
\pgfusepath{fill}%
\end{pgfscope}%
\begin{pgfscope}%
\pgfpathrectangle{\pgfqpoint{1.250000in}{0.550000in}}{\pgfqpoint{7.750000in}{3.850000in}}%
\pgfusepath{clip}%
\pgfsetbuttcap%
\pgfsetmiterjoin%
\definecolor{currentfill}{rgb}{0.501961,0.501961,0.501961}%
\pgfsetfillcolor{currentfill}%
\pgfsetfillopacity{0.700000}%
\pgfsetlinewidth{0.000000pt}%
\definecolor{currentstroke}{rgb}{0.000000,0.000000,0.000000}%
\pgfsetstrokecolor{currentstroke}%
\pgfsetstrokeopacity{0.700000}%
\pgfsetdash{}{0pt}%
\pgfpathmoveto{\pgfqpoint{6.776278in}{0.550000in}}%
\pgfpathlineto{\pgfqpoint{6.803800in}{0.550000in}}%
\pgfpathlineto{\pgfqpoint{6.803800in}{0.767503in}}%
\pgfpathlineto{\pgfqpoint{6.776278in}{0.767503in}}%
\pgfpathlineto{\pgfqpoint{6.776278in}{0.550000in}}%
\pgfpathclose%
\pgfusepath{fill}%
\end{pgfscope}%
\begin{pgfscope}%
\pgfpathrectangle{\pgfqpoint{1.250000in}{0.550000in}}{\pgfqpoint{7.750000in}{3.850000in}}%
\pgfusepath{clip}%
\pgfsetbuttcap%
\pgfsetmiterjoin%
\definecolor{currentfill}{rgb}{0.501961,0.501961,0.501961}%
\pgfsetfillcolor{currentfill}%
\pgfsetfillopacity{0.700000}%
\pgfsetlinewidth{0.000000pt}%
\definecolor{currentstroke}{rgb}{0.000000,0.000000,0.000000}%
\pgfsetstrokecolor{currentstroke}%
\pgfsetstrokeopacity{0.700000}%
\pgfsetdash{}{0pt}%
\pgfpathmoveto{\pgfqpoint{6.803800in}{0.550000in}}%
\pgfpathlineto{\pgfqpoint{6.831321in}{0.550000in}}%
\pgfpathlineto{\pgfqpoint{6.831321in}{0.649906in}}%
\pgfpathlineto{\pgfqpoint{6.803800in}{0.649906in}}%
\pgfpathlineto{\pgfqpoint{6.803800in}{0.550000in}}%
\pgfpathclose%
\pgfusepath{fill}%
\end{pgfscope}%
\begin{pgfscope}%
\pgfpathrectangle{\pgfqpoint{1.250000in}{0.550000in}}{\pgfqpoint{7.750000in}{3.850000in}}%
\pgfusepath{clip}%
\pgfsetbuttcap%
\pgfsetmiterjoin%
\definecolor{currentfill}{rgb}{0.501961,0.501961,0.501961}%
\pgfsetfillcolor{currentfill}%
\pgfsetfillopacity{0.700000}%
\pgfsetlinewidth{0.000000pt}%
\definecolor{currentstroke}{rgb}{0.000000,0.000000,0.000000}%
\pgfsetstrokecolor{currentstroke}%
\pgfsetstrokeopacity{0.700000}%
\pgfsetdash{}{0pt}%
\pgfpathmoveto{\pgfqpoint{6.831321in}{0.550000in}}%
\pgfpathlineto{\pgfqpoint{6.858842in}{0.550000in}}%
\pgfpathlineto{\pgfqpoint{6.858842in}{0.755033in}}%
\pgfpathlineto{\pgfqpoint{6.831321in}{0.755033in}}%
\pgfpathlineto{\pgfqpoint{6.831321in}{0.550000in}}%
\pgfpathclose%
\pgfusepath{fill}%
\end{pgfscope}%
\begin{pgfscope}%
\pgfpathrectangle{\pgfqpoint{1.250000in}{0.550000in}}{\pgfqpoint{7.750000in}{3.850000in}}%
\pgfusepath{clip}%
\pgfsetbuttcap%
\pgfsetmiterjoin%
\definecolor{currentfill}{rgb}{0.501961,0.501961,0.501961}%
\pgfsetfillcolor{currentfill}%
\pgfsetfillopacity{0.700000}%
\pgfsetlinewidth{0.000000pt}%
\definecolor{currentstroke}{rgb}{0.000000,0.000000,0.000000}%
\pgfsetstrokecolor{currentstroke}%
\pgfsetstrokeopacity{0.700000}%
\pgfsetdash{}{0pt}%
\pgfpathmoveto{\pgfqpoint{6.858842in}{0.550000in}}%
\pgfpathlineto{\pgfqpoint{6.886364in}{0.550000in}}%
\pgfpathlineto{\pgfqpoint{6.886364in}{0.650196in}}%
\pgfpathlineto{\pgfqpoint{6.858842in}{0.650196in}}%
\pgfpathlineto{\pgfqpoint{6.858842in}{0.550000in}}%
\pgfpathclose%
\pgfusepath{fill}%
\end{pgfscope}%
\begin{pgfscope}%
\pgfpathrectangle{\pgfqpoint{1.250000in}{0.550000in}}{\pgfqpoint{7.750000in}{3.850000in}}%
\pgfusepath{clip}%
\pgfsetbuttcap%
\pgfsetmiterjoin%
\definecolor{currentfill}{rgb}{0.501961,0.501961,0.501961}%
\pgfsetfillcolor{currentfill}%
\pgfsetfillopacity{0.700000}%
\pgfsetlinewidth{0.000000pt}%
\definecolor{currentstroke}{rgb}{0.000000,0.000000,0.000000}%
\pgfsetstrokecolor{currentstroke}%
\pgfsetstrokeopacity{0.700000}%
\pgfsetdash{}{0pt}%
\pgfpathmoveto{\pgfqpoint{6.886364in}{0.550000in}}%
\pgfpathlineto{\pgfqpoint{6.913885in}{0.550000in}}%
\pgfpathlineto{\pgfqpoint{6.913885in}{0.751408in}}%
\pgfpathlineto{\pgfqpoint{6.886364in}{0.751408in}}%
\pgfpathlineto{\pgfqpoint{6.886364in}{0.550000in}}%
\pgfpathclose%
\pgfusepath{fill}%
\end{pgfscope}%
\begin{pgfscope}%
\pgfpathrectangle{\pgfqpoint{1.250000in}{0.550000in}}{\pgfqpoint{7.750000in}{3.850000in}}%
\pgfusepath{clip}%
\pgfsetbuttcap%
\pgfsetmiterjoin%
\definecolor{currentfill}{rgb}{0.501961,0.501961,0.501961}%
\pgfsetfillcolor{currentfill}%
\pgfsetfillopacity{0.700000}%
\pgfsetlinewidth{0.000000pt}%
\definecolor{currentstroke}{rgb}{0.000000,0.000000,0.000000}%
\pgfsetstrokecolor{currentstroke}%
\pgfsetstrokeopacity{0.700000}%
\pgfsetdash{}{0pt}%
\pgfpathmoveto{\pgfqpoint{6.913885in}{0.550000in}}%
\pgfpathlineto{\pgfqpoint{6.941406in}{0.550000in}}%
\pgfpathlineto{\pgfqpoint{6.941406in}{0.746768in}}%
\pgfpathlineto{\pgfqpoint{6.913885in}{0.746768in}}%
\pgfpathlineto{\pgfqpoint{6.913885in}{0.550000in}}%
\pgfpathclose%
\pgfusepath{fill}%
\end{pgfscope}%
\begin{pgfscope}%
\pgfpathrectangle{\pgfqpoint{1.250000in}{0.550000in}}{\pgfqpoint{7.750000in}{3.850000in}}%
\pgfusepath{clip}%
\pgfsetbuttcap%
\pgfsetmiterjoin%
\definecolor{currentfill}{rgb}{0.501961,0.501961,0.501961}%
\pgfsetfillcolor{currentfill}%
\pgfsetfillopacity{0.700000}%
\pgfsetlinewidth{0.000000pt}%
\definecolor{currentstroke}{rgb}{0.000000,0.000000,0.000000}%
\pgfsetstrokecolor{currentstroke}%
\pgfsetstrokeopacity{0.700000}%
\pgfsetdash{}{0pt}%
\pgfpathmoveto{\pgfqpoint{6.941406in}{0.550000in}}%
\pgfpathlineto{\pgfqpoint{6.968928in}{0.550000in}}%
\pgfpathlineto{\pgfqpoint{6.968928in}{0.642221in}}%
\pgfpathlineto{\pgfqpoint{6.941406in}{0.642221in}}%
\pgfpathlineto{\pgfqpoint{6.941406in}{0.550000in}}%
\pgfpathclose%
\pgfusepath{fill}%
\end{pgfscope}%
\begin{pgfscope}%
\pgfpathrectangle{\pgfqpoint{1.250000in}{0.550000in}}{\pgfqpoint{7.750000in}{3.850000in}}%
\pgfusepath{clip}%
\pgfsetbuttcap%
\pgfsetmiterjoin%
\definecolor{currentfill}{rgb}{0.501961,0.501961,0.501961}%
\pgfsetfillcolor{currentfill}%
\pgfsetfillopacity{0.700000}%
\pgfsetlinewidth{0.000000pt}%
\definecolor{currentstroke}{rgb}{0.000000,0.000000,0.000000}%
\pgfsetstrokecolor{currentstroke}%
\pgfsetstrokeopacity{0.700000}%
\pgfsetdash{}{0pt}%
\pgfpathmoveto{\pgfqpoint{6.968928in}{0.550000in}}%
\pgfpathlineto{\pgfqpoint{6.996449in}{0.550000in}}%
\pgfpathlineto{\pgfqpoint{6.996449in}{0.751118in}}%
\pgfpathlineto{\pgfqpoint{6.968928in}{0.751118in}}%
\pgfpathlineto{\pgfqpoint{6.968928in}{0.550000in}}%
\pgfpathclose%
\pgfusepath{fill}%
\end{pgfscope}%
\begin{pgfscope}%
\pgfpathrectangle{\pgfqpoint{1.250000in}{0.550000in}}{\pgfqpoint{7.750000in}{3.850000in}}%
\pgfusepath{clip}%
\pgfsetbuttcap%
\pgfsetmiterjoin%
\definecolor{currentfill}{rgb}{0.501961,0.501961,0.501961}%
\pgfsetfillcolor{currentfill}%
\pgfsetfillopacity{0.700000}%
\pgfsetlinewidth{0.000000pt}%
\definecolor{currentstroke}{rgb}{0.000000,0.000000,0.000000}%
\pgfsetstrokecolor{currentstroke}%
\pgfsetstrokeopacity{0.700000}%
\pgfsetdash{}{0pt}%
\pgfpathmoveto{\pgfqpoint{6.996449in}{0.550000in}}%
\pgfpathlineto{\pgfqpoint{7.023970in}{0.550000in}}%
\pgfpathlineto{\pgfqpoint{7.023970in}{0.740533in}}%
\pgfpathlineto{\pgfqpoint{6.996449in}{0.740533in}}%
\pgfpathlineto{\pgfqpoint{6.996449in}{0.550000in}}%
\pgfpathclose%
\pgfusepath{fill}%
\end{pgfscope}%
\begin{pgfscope}%
\pgfpathrectangle{\pgfqpoint{1.250000in}{0.550000in}}{\pgfqpoint{7.750000in}{3.850000in}}%
\pgfusepath{clip}%
\pgfsetbuttcap%
\pgfsetmiterjoin%
\definecolor{currentfill}{rgb}{0.501961,0.501961,0.501961}%
\pgfsetfillcolor{currentfill}%
\pgfsetfillopacity{0.700000}%
\pgfsetlinewidth{0.000000pt}%
\definecolor{currentstroke}{rgb}{0.000000,0.000000,0.000000}%
\pgfsetstrokecolor{currentstroke}%
\pgfsetstrokeopacity{0.700000}%
\pgfsetdash{}{0pt}%
\pgfpathmoveto{\pgfqpoint{7.023970in}{0.550000in}}%
\pgfpathlineto{\pgfqpoint{7.051491in}{0.550000in}}%
\pgfpathlineto{\pgfqpoint{7.051491in}{0.647731in}}%
\pgfpathlineto{\pgfqpoint{7.023970in}{0.647731in}}%
\pgfpathlineto{\pgfqpoint{7.023970in}{0.550000in}}%
\pgfpathclose%
\pgfusepath{fill}%
\end{pgfscope}%
\begin{pgfscope}%
\pgfpathrectangle{\pgfqpoint{1.250000in}{0.550000in}}{\pgfqpoint{7.750000in}{3.850000in}}%
\pgfusepath{clip}%
\pgfsetbuttcap%
\pgfsetmiterjoin%
\definecolor{currentfill}{rgb}{0.501961,0.501961,0.501961}%
\pgfsetfillcolor{currentfill}%
\pgfsetfillopacity{0.700000}%
\pgfsetlinewidth{0.000000pt}%
\definecolor{currentstroke}{rgb}{0.000000,0.000000,0.000000}%
\pgfsetstrokecolor{currentstroke}%
\pgfsetstrokeopacity{0.700000}%
\pgfsetdash{}{0pt}%
\pgfpathmoveto{\pgfqpoint{7.051491in}{0.550000in}}%
\pgfpathlineto{\pgfqpoint{7.079013in}{0.550000in}}%
\pgfpathlineto{\pgfqpoint{7.079013in}{0.748508in}}%
\pgfpathlineto{\pgfqpoint{7.051491in}{0.748508in}}%
\pgfpathlineto{\pgfqpoint{7.051491in}{0.550000in}}%
\pgfpathclose%
\pgfusepath{fill}%
\end{pgfscope}%
\begin{pgfscope}%
\pgfpathrectangle{\pgfqpoint{1.250000in}{0.550000in}}{\pgfqpoint{7.750000in}{3.850000in}}%
\pgfusepath{clip}%
\pgfsetbuttcap%
\pgfsetmiterjoin%
\definecolor{currentfill}{rgb}{0.501961,0.501961,0.501961}%
\pgfsetfillcolor{currentfill}%
\pgfsetfillopacity{0.700000}%
\pgfsetlinewidth{0.000000pt}%
\definecolor{currentstroke}{rgb}{0.000000,0.000000,0.000000}%
\pgfsetstrokecolor{currentstroke}%
\pgfsetstrokeopacity{0.700000}%
\pgfsetdash{}{0pt}%
\pgfpathmoveto{\pgfqpoint{7.079013in}{0.550000in}}%
\pgfpathlineto{\pgfqpoint{7.106534in}{0.550000in}}%
\pgfpathlineto{\pgfqpoint{7.106534in}{0.749813in}}%
\pgfpathlineto{\pgfqpoint{7.079013in}{0.749813in}}%
\pgfpathlineto{\pgfqpoint{7.079013in}{0.550000in}}%
\pgfpathclose%
\pgfusepath{fill}%
\end{pgfscope}%
\begin{pgfscope}%
\pgfpathrectangle{\pgfqpoint{1.250000in}{0.550000in}}{\pgfqpoint{7.750000in}{3.850000in}}%
\pgfusepath{clip}%
\pgfsetbuttcap%
\pgfsetmiterjoin%
\definecolor{currentfill}{rgb}{0.501961,0.501961,0.501961}%
\pgfsetfillcolor{currentfill}%
\pgfsetfillopacity{0.700000}%
\pgfsetlinewidth{0.000000pt}%
\definecolor{currentstroke}{rgb}{0.000000,0.000000,0.000000}%
\pgfsetstrokecolor{currentstroke}%
\pgfsetstrokeopacity{0.700000}%
\pgfsetdash{}{0pt}%
\pgfpathmoveto{\pgfqpoint{7.106534in}{0.550000in}}%
\pgfpathlineto{\pgfqpoint{7.134055in}{0.550000in}}%
\pgfpathlineto{\pgfqpoint{7.134055in}{0.657012in}}%
\pgfpathlineto{\pgfqpoint{7.106534in}{0.657012in}}%
\pgfpathlineto{\pgfqpoint{7.106534in}{0.550000in}}%
\pgfpathclose%
\pgfusepath{fill}%
\end{pgfscope}%
\begin{pgfscope}%
\pgfpathrectangle{\pgfqpoint{1.250000in}{0.550000in}}{\pgfqpoint{7.750000in}{3.850000in}}%
\pgfusepath{clip}%
\pgfsetbuttcap%
\pgfsetmiterjoin%
\definecolor{currentfill}{rgb}{0.501961,0.501961,0.501961}%
\pgfsetfillcolor{currentfill}%
\pgfsetfillopacity{0.700000}%
\pgfsetlinewidth{0.000000pt}%
\definecolor{currentstroke}{rgb}{0.000000,0.000000,0.000000}%
\pgfsetstrokecolor{currentstroke}%
\pgfsetstrokeopacity{0.700000}%
\pgfsetdash{}{0pt}%
\pgfpathmoveto{\pgfqpoint{7.134055in}{0.550000in}}%
\pgfpathlineto{\pgfqpoint{7.161577in}{0.550000in}}%
\pgfpathlineto{\pgfqpoint{7.161577in}{0.749233in}}%
\pgfpathlineto{\pgfqpoint{7.134055in}{0.749233in}}%
\pgfpathlineto{\pgfqpoint{7.134055in}{0.550000in}}%
\pgfpathclose%
\pgfusepath{fill}%
\end{pgfscope}%
\begin{pgfscope}%
\pgfpathrectangle{\pgfqpoint{1.250000in}{0.550000in}}{\pgfqpoint{7.750000in}{3.850000in}}%
\pgfusepath{clip}%
\pgfsetbuttcap%
\pgfsetmiterjoin%
\definecolor{currentfill}{rgb}{0.501961,0.501961,0.501961}%
\pgfsetfillcolor{currentfill}%
\pgfsetfillopacity{0.700000}%
\pgfsetlinewidth{0.000000pt}%
\definecolor{currentstroke}{rgb}{0.000000,0.000000,0.000000}%
\pgfsetstrokecolor{currentstroke}%
\pgfsetstrokeopacity{0.700000}%
\pgfsetdash{}{0pt}%
\pgfpathmoveto{\pgfqpoint{7.161577in}{0.550000in}}%
\pgfpathlineto{\pgfqpoint{7.189098in}{0.550000in}}%
\pgfpathlineto{\pgfqpoint{7.189098in}{0.649906in}}%
\pgfpathlineto{\pgfqpoint{7.161577in}{0.649906in}}%
\pgfpathlineto{\pgfqpoint{7.161577in}{0.550000in}}%
\pgfpathclose%
\pgfusepath{fill}%
\end{pgfscope}%
\begin{pgfscope}%
\pgfpathrectangle{\pgfqpoint{1.250000in}{0.550000in}}{\pgfqpoint{7.750000in}{3.850000in}}%
\pgfusepath{clip}%
\pgfsetbuttcap%
\pgfsetmiterjoin%
\definecolor{currentfill}{rgb}{0.501961,0.501961,0.501961}%
\pgfsetfillcolor{currentfill}%
\pgfsetfillopacity{0.700000}%
\pgfsetlinewidth{0.000000pt}%
\definecolor{currentstroke}{rgb}{0.000000,0.000000,0.000000}%
\pgfsetstrokecolor{currentstroke}%
\pgfsetstrokeopacity{0.700000}%
\pgfsetdash{}{0pt}%
\pgfpathmoveto{\pgfqpoint{7.189098in}{0.550000in}}%
\pgfpathlineto{\pgfqpoint{7.216619in}{0.550000in}}%
\pgfpathlineto{\pgfqpoint{7.216619in}{0.752858in}}%
\pgfpathlineto{\pgfqpoint{7.189098in}{0.752858in}}%
\pgfpathlineto{\pgfqpoint{7.189098in}{0.550000in}}%
\pgfpathclose%
\pgfusepath{fill}%
\end{pgfscope}%
\begin{pgfscope}%
\pgfpathrectangle{\pgfqpoint{1.250000in}{0.550000in}}{\pgfqpoint{7.750000in}{3.850000in}}%
\pgfusepath{clip}%
\pgfsetbuttcap%
\pgfsetmiterjoin%
\definecolor{currentfill}{rgb}{0.501961,0.501961,0.501961}%
\pgfsetfillcolor{currentfill}%
\pgfsetfillopacity{0.700000}%
\pgfsetlinewidth{0.000000pt}%
\definecolor{currentstroke}{rgb}{0.000000,0.000000,0.000000}%
\pgfsetstrokecolor{currentstroke}%
\pgfsetstrokeopacity{0.700000}%
\pgfsetdash{}{0pt}%
\pgfpathmoveto{\pgfqpoint{7.216619in}{0.550000in}}%
\pgfpathlineto{\pgfqpoint{7.244141in}{0.550000in}}%
\pgfpathlineto{\pgfqpoint{7.244141in}{0.744158in}}%
\pgfpathlineto{\pgfqpoint{7.216619in}{0.744158in}}%
\pgfpathlineto{\pgfqpoint{7.216619in}{0.550000in}}%
\pgfpathclose%
\pgfusepath{fill}%
\end{pgfscope}%
\begin{pgfscope}%
\pgfpathrectangle{\pgfqpoint{1.250000in}{0.550000in}}{\pgfqpoint{7.750000in}{3.850000in}}%
\pgfusepath{clip}%
\pgfsetbuttcap%
\pgfsetmiterjoin%
\definecolor{currentfill}{rgb}{0.501961,0.501961,0.501961}%
\pgfsetfillcolor{currentfill}%
\pgfsetfillopacity{0.700000}%
\pgfsetlinewidth{0.000000pt}%
\definecolor{currentstroke}{rgb}{0.000000,0.000000,0.000000}%
\pgfsetstrokecolor{currentstroke}%
\pgfsetstrokeopacity{0.700000}%
\pgfsetdash{}{0pt}%
\pgfpathmoveto{\pgfqpoint{7.244141in}{0.550000in}}%
\pgfpathlineto{\pgfqpoint{7.271662in}{0.550000in}}%
\pgfpathlineto{\pgfqpoint{7.271662in}{0.646281in}}%
\pgfpathlineto{\pgfqpoint{7.244141in}{0.646281in}}%
\pgfpathlineto{\pgfqpoint{7.244141in}{0.550000in}}%
\pgfpathclose%
\pgfusepath{fill}%
\end{pgfscope}%
\begin{pgfscope}%
\pgfpathrectangle{\pgfqpoint{1.250000in}{0.550000in}}{\pgfqpoint{7.750000in}{3.850000in}}%
\pgfusepath{clip}%
\pgfsetbuttcap%
\pgfsetmiterjoin%
\definecolor{currentfill}{rgb}{0.501961,0.501961,0.501961}%
\pgfsetfillcolor{currentfill}%
\pgfsetfillopacity{0.700000}%
\pgfsetlinewidth{0.000000pt}%
\definecolor{currentstroke}{rgb}{0.000000,0.000000,0.000000}%
\pgfsetstrokecolor{currentstroke}%
\pgfsetstrokeopacity{0.700000}%
\pgfsetdash{}{0pt}%
\pgfpathmoveto{\pgfqpoint{7.271662in}{0.550000in}}%
\pgfpathlineto{\pgfqpoint{7.299183in}{0.550000in}}%
\pgfpathlineto{\pgfqpoint{7.299183in}{0.744738in}}%
\pgfpathlineto{\pgfqpoint{7.271662in}{0.744738in}}%
\pgfpathlineto{\pgfqpoint{7.271662in}{0.550000in}}%
\pgfpathclose%
\pgfusepath{fill}%
\end{pgfscope}%
\begin{pgfscope}%
\pgfpathrectangle{\pgfqpoint{1.250000in}{0.550000in}}{\pgfqpoint{7.750000in}{3.850000in}}%
\pgfusepath{clip}%
\pgfsetbuttcap%
\pgfsetmiterjoin%
\definecolor{currentfill}{rgb}{0.501961,0.501961,0.501961}%
\pgfsetfillcolor{currentfill}%
\pgfsetfillopacity{0.700000}%
\pgfsetlinewidth{0.000000pt}%
\definecolor{currentstroke}{rgb}{0.000000,0.000000,0.000000}%
\pgfsetstrokecolor{currentstroke}%
\pgfsetstrokeopacity{0.700000}%
\pgfsetdash{}{0pt}%
\pgfpathmoveto{\pgfqpoint{7.299183in}{0.550000in}}%
\pgfpathlineto{\pgfqpoint{7.326705in}{0.550000in}}%
\pgfpathlineto{\pgfqpoint{7.326705in}{0.656141in}}%
\pgfpathlineto{\pgfqpoint{7.299183in}{0.656141in}}%
\pgfpathlineto{\pgfqpoint{7.299183in}{0.550000in}}%
\pgfpathclose%
\pgfusepath{fill}%
\end{pgfscope}%
\begin{pgfscope}%
\pgfpathrectangle{\pgfqpoint{1.250000in}{0.550000in}}{\pgfqpoint{7.750000in}{3.850000in}}%
\pgfusepath{clip}%
\pgfsetbuttcap%
\pgfsetmiterjoin%
\definecolor{currentfill}{rgb}{0.501961,0.501961,0.501961}%
\pgfsetfillcolor{currentfill}%
\pgfsetfillopacity{0.700000}%
\pgfsetlinewidth{0.000000pt}%
\definecolor{currentstroke}{rgb}{0.000000,0.000000,0.000000}%
\pgfsetstrokecolor{currentstroke}%
\pgfsetstrokeopacity{0.700000}%
\pgfsetdash{}{0pt}%
\pgfpathmoveto{\pgfqpoint{7.326705in}{0.550000in}}%
\pgfpathlineto{\pgfqpoint{7.354226in}{0.550000in}}%
\pgfpathlineto{\pgfqpoint{7.354226in}{0.741693in}}%
\pgfpathlineto{\pgfqpoint{7.326705in}{0.741693in}}%
\pgfpathlineto{\pgfqpoint{7.326705in}{0.550000in}}%
\pgfpathclose%
\pgfusepath{fill}%
\end{pgfscope}%
\begin{pgfscope}%
\pgfpathrectangle{\pgfqpoint{1.250000in}{0.550000in}}{\pgfqpoint{7.750000in}{3.850000in}}%
\pgfusepath{clip}%
\pgfsetbuttcap%
\pgfsetmiterjoin%
\definecolor{currentfill}{rgb}{0.501961,0.501961,0.501961}%
\pgfsetfillcolor{currentfill}%
\pgfsetfillopacity{0.700000}%
\pgfsetlinewidth{0.000000pt}%
\definecolor{currentstroke}{rgb}{0.000000,0.000000,0.000000}%
\pgfsetstrokecolor{currentstroke}%
\pgfsetstrokeopacity{0.700000}%
\pgfsetdash{}{0pt}%
\pgfpathmoveto{\pgfqpoint{7.354226in}{0.550000in}}%
\pgfpathlineto{\pgfqpoint{7.381747in}{0.550000in}}%
\pgfpathlineto{\pgfqpoint{7.381747in}{0.737633in}}%
\pgfpathlineto{\pgfqpoint{7.354226in}{0.737633in}}%
\pgfpathlineto{\pgfqpoint{7.354226in}{0.550000in}}%
\pgfpathclose%
\pgfusepath{fill}%
\end{pgfscope}%
\begin{pgfscope}%
\pgfpathrectangle{\pgfqpoint{1.250000in}{0.550000in}}{\pgfqpoint{7.750000in}{3.850000in}}%
\pgfusepath{clip}%
\pgfsetbuttcap%
\pgfsetmiterjoin%
\definecolor{currentfill}{rgb}{0.501961,0.501961,0.501961}%
\pgfsetfillcolor{currentfill}%
\pgfsetfillopacity{0.700000}%
\pgfsetlinewidth{0.000000pt}%
\definecolor{currentstroke}{rgb}{0.000000,0.000000,0.000000}%
\pgfsetstrokecolor{currentstroke}%
\pgfsetstrokeopacity{0.700000}%
\pgfsetdash{}{0pt}%
\pgfpathmoveto{\pgfqpoint{7.381747in}{0.550000in}}%
\pgfpathlineto{\pgfqpoint{7.409268in}{0.550000in}}%
\pgfpathlineto{\pgfqpoint{7.409268in}{0.732848in}}%
\pgfpathlineto{\pgfqpoint{7.381747in}{0.732848in}}%
\pgfpathlineto{\pgfqpoint{7.381747in}{0.550000in}}%
\pgfpathclose%
\pgfusepath{fill}%
\end{pgfscope}%
\begin{pgfscope}%
\pgfpathrectangle{\pgfqpoint{1.250000in}{0.550000in}}{\pgfqpoint{7.750000in}{3.850000in}}%
\pgfusepath{clip}%
\pgfsetbuttcap%
\pgfsetmiterjoin%
\definecolor{currentfill}{rgb}{0.501961,0.501961,0.501961}%
\pgfsetfillcolor{currentfill}%
\pgfsetfillopacity{0.700000}%
\pgfsetlinewidth{0.000000pt}%
\definecolor{currentstroke}{rgb}{0.000000,0.000000,0.000000}%
\pgfsetstrokecolor{currentstroke}%
\pgfsetstrokeopacity{0.700000}%
\pgfsetdash{}{0pt}%
\pgfpathmoveto{\pgfqpoint{7.409268in}{0.550000in}}%
\pgfpathlineto{\pgfqpoint{7.436790in}{0.550000in}}%
\pgfpathlineto{\pgfqpoint{7.436790in}{0.635116in}}%
\pgfpathlineto{\pgfqpoint{7.409268in}{0.635116in}}%
\pgfpathlineto{\pgfqpoint{7.409268in}{0.550000in}}%
\pgfpathclose%
\pgfusepath{fill}%
\end{pgfscope}%
\begin{pgfscope}%
\pgfpathrectangle{\pgfqpoint{1.250000in}{0.550000in}}{\pgfqpoint{7.750000in}{3.850000in}}%
\pgfusepath{clip}%
\pgfsetbuttcap%
\pgfsetmiterjoin%
\definecolor{currentfill}{rgb}{0.501961,0.501961,0.501961}%
\pgfsetfillcolor{currentfill}%
\pgfsetfillopacity{0.700000}%
\pgfsetlinewidth{0.000000pt}%
\definecolor{currentstroke}{rgb}{0.000000,0.000000,0.000000}%
\pgfsetstrokecolor{currentstroke}%
\pgfsetstrokeopacity{0.700000}%
\pgfsetdash{}{0pt}%
\pgfpathmoveto{\pgfqpoint{7.436790in}{0.550000in}}%
\pgfpathlineto{\pgfqpoint{7.464311in}{0.550000in}}%
\pgfpathlineto{\pgfqpoint{7.464311in}{0.722697in}}%
\pgfpathlineto{\pgfqpoint{7.436790in}{0.722697in}}%
\pgfpathlineto{\pgfqpoint{7.436790in}{0.550000in}}%
\pgfpathclose%
\pgfusepath{fill}%
\end{pgfscope}%
\begin{pgfscope}%
\pgfpathrectangle{\pgfqpoint{1.250000in}{0.550000in}}{\pgfqpoint{7.750000in}{3.850000in}}%
\pgfusepath{clip}%
\pgfsetbuttcap%
\pgfsetmiterjoin%
\definecolor{currentfill}{rgb}{0.501961,0.501961,0.501961}%
\pgfsetfillcolor{currentfill}%
\pgfsetfillopacity{0.700000}%
\pgfsetlinewidth{0.000000pt}%
\definecolor{currentstroke}{rgb}{0.000000,0.000000,0.000000}%
\pgfsetstrokecolor{currentstroke}%
\pgfsetstrokeopacity{0.700000}%
\pgfsetdash{}{0pt}%
\pgfpathmoveto{\pgfqpoint{7.464311in}{0.550000in}}%
\pgfpathlineto{\pgfqpoint{7.491832in}{0.550000in}}%
\pgfpathlineto{\pgfqpoint{7.491832in}{0.712402in}}%
\pgfpathlineto{\pgfqpoint{7.464311in}{0.712402in}}%
\pgfpathlineto{\pgfqpoint{7.464311in}{0.550000in}}%
\pgfpathclose%
\pgfusepath{fill}%
\end{pgfscope}%
\begin{pgfscope}%
\pgfpathrectangle{\pgfqpoint{1.250000in}{0.550000in}}{\pgfqpoint{7.750000in}{3.850000in}}%
\pgfusepath{clip}%
\pgfsetbuttcap%
\pgfsetmiterjoin%
\definecolor{currentfill}{rgb}{0.501961,0.501961,0.501961}%
\pgfsetfillcolor{currentfill}%
\pgfsetfillopacity{0.700000}%
\pgfsetlinewidth{0.000000pt}%
\definecolor{currentstroke}{rgb}{0.000000,0.000000,0.000000}%
\pgfsetstrokecolor{currentstroke}%
\pgfsetstrokeopacity{0.700000}%
\pgfsetdash{}{0pt}%
\pgfpathmoveto{\pgfqpoint{7.491832in}{0.550000in}}%
\pgfpathlineto{\pgfqpoint{7.519354in}{0.550000in}}%
\pgfpathlineto{\pgfqpoint{7.519354in}{0.711967in}}%
\pgfpathlineto{\pgfqpoint{7.491832in}{0.711967in}}%
\pgfpathlineto{\pgfqpoint{7.491832in}{0.550000in}}%
\pgfpathclose%
\pgfusepath{fill}%
\end{pgfscope}%
\begin{pgfscope}%
\pgfpathrectangle{\pgfqpoint{1.250000in}{0.550000in}}{\pgfqpoint{7.750000in}{3.850000in}}%
\pgfusepath{clip}%
\pgfsetbuttcap%
\pgfsetmiterjoin%
\definecolor{currentfill}{rgb}{0.501961,0.501961,0.501961}%
\pgfsetfillcolor{currentfill}%
\pgfsetfillopacity{0.700000}%
\pgfsetlinewidth{0.000000pt}%
\definecolor{currentstroke}{rgb}{0.000000,0.000000,0.000000}%
\pgfsetstrokecolor{currentstroke}%
\pgfsetstrokeopacity{0.700000}%
\pgfsetdash{}{0pt}%
\pgfpathmoveto{\pgfqpoint{7.519354in}{0.550000in}}%
\pgfpathlineto{\pgfqpoint{7.546875in}{0.550000in}}%
\pgfpathlineto{\pgfqpoint{7.546875in}{0.767503in}}%
\pgfpathlineto{\pgfqpoint{7.519354in}{0.767503in}}%
\pgfpathlineto{\pgfqpoint{7.519354in}{0.550000in}}%
\pgfpathclose%
\pgfusepath{fill}%
\end{pgfscope}%
\begin{pgfscope}%
\pgfpathrectangle{\pgfqpoint{1.250000in}{0.550000in}}{\pgfqpoint{7.750000in}{3.850000in}}%
\pgfusepath{clip}%
\pgfsetbuttcap%
\pgfsetmiterjoin%
\definecolor{currentfill}{rgb}{0.501961,0.501961,0.501961}%
\pgfsetfillcolor{currentfill}%
\pgfsetfillopacity{0.700000}%
\pgfsetlinewidth{0.000000pt}%
\definecolor{currentstroke}{rgb}{0.000000,0.000000,0.000000}%
\pgfsetstrokecolor{currentstroke}%
\pgfsetstrokeopacity{0.700000}%
\pgfsetdash{}{0pt}%
\pgfpathmoveto{\pgfqpoint{7.546875in}{0.550000in}}%
\pgfpathlineto{\pgfqpoint{7.574396in}{0.550000in}}%
\pgfpathlineto{\pgfqpoint{7.574396in}{0.686012in}}%
\pgfpathlineto{\pgfqpoint{7.546875in}{0.686012in}}%
\pgfpathlineto{\pgfqpoint{7.546875in}{0.550000in}}%
\pgfpathclose%
\pgfusepath{fill}%
\end{pgfscope}%
\begin{pgfscope}%
\pgfpathrectangle{\pgfqpoint{1.250000in}{0.550000in}}{\pgfqpoint{7.750000in}{3.850000in}}%
\pgfusepath{clip}%
\pgfsetbuttcap%
\pgfsetmiterjoin%
\definecolor{currentfill}{rgb}{0.501961,0.501961,0.501961}%
\pgfsetfillcolor{currentfill}%
\pgfsetfillopacity{0.700000}%
\pgfsetlinewidth{0.000000pt}%
\definecolor{currentstroke}{rgb}{0.000000,0.000000,0.000000}%
\pgfsetstrokecolor{currentstroke}%
\pgfsetstrokeopacity{0.700000}%
\pgfsetdash{}{0pt}%
\pgfpathmoveto{\pgfqpoint{7.574396in}{0.550000in}}%
\pgfpathlineto{\pgfqpoint{7.601918in}{0.550000in}}%
\pgfpathlineto{\pgfqpoint{7.601918in}{0.686737in}}%
\pgfpathlineto{\pgfqpoint{7.574396in}{0.686737in}}%
\pgfpathlineto{\pgfqpoint{7.574396in}{0.550000in}}%
\pgfpathclose%
\pgfusepath{fill}%
\end{pgfscope}%
\begin{pgfscope}%
\pgfpathrectangle{\pgfqpoint{1.250000in}{0.550000in}}{\pgfqpoint{7.750000in}{3.850000in}}%
\pgfusepath{clip}%
\pgfsetbuttcap%
\pgfsetmiterjoin%
\definecolor{currentfill}{rgb}{0.501961,0.501961,0.501961}%
\pgfsetfillcolor{currentfill}%
\pgfsetfillopacity{0.700000}%
\pgfsetlinewidth{0.000000pt}%
\definecolor{currentstroke}{rgb}{0.000000,0.000000,0.000000}%
\pgfsetstrokecolor{currentstroke}%
\pgfsetstrokeopacity{0.700000}%
\pgfsetdash{}{0pt}%
\pgfpathmoveto{\pgfqpoint{7.601918in}{0.550000in}}%
\pgfpathlineto{\pgfqpoint{7.629439in}{0.550000in}}%
\pgfpathlineto{\pgfqpoint{7.629439in}{0.741983in}}%
\pgfpathlineto{\pgfqpoint{7.601918in}{0.741983in}}%
\pgfpathlineto{\pgfqpoint{7.601918in}{0.550000in}}%
\pgfpathclose%
\pgfusepath{fill}%
\end{pgfscope}%
\begin{pgfscope}%
\pgfpathrectangle{\pgfqpoint{1.250000in}{0.550000in}}{\pgfqpoint{7.750000in}{3.850000in}}%
\pgfusepath{clip}%
\pgfsetbuttcap%
\pgfsetmiterjoin%
\definecolor{currentfill}{rgb}{0.501961,0.501961,0.501961}%
\pgfsetfillcolor{currentfill}%
\pgfsetfillopacity{0.700000}%
\pgfsetlinewidth{0.000000pt}%
\definecolor{currentstroke}{rgb}{0.000000,0.000000,0.000000}%
\pgfsetstrokecolor{currentstroke}%
\pgfsetstrokeopacity{0.700000}%
\pgfsetdash{}{0pt}%
\pgfpathmoveto{\pgfqpoint{7.629439in}{0.550000in}}%
\pgfpathlineto{\pgfqpoint{7.656960in}{0.550000in}}%
\pgfpathlineto{\pgfqpoint{7.656960in}{0.679052in}}%
\pgfpathlineto{\pgfqpoint{7.629439in}{0.679052in}}%
\pgfpathlineto{\pgfqpoint{7.629439in}{0.550000in}}%
\pgfpathclose%
\pgfusepath{fill}%
\end{pgfscope}%
\begin{pgfscope}%
\pgfpathrectangle{\pgfqpoint{1.250000in}{0.550000in}}{\pgfqpoint{7.750000in}{3.850000in}}%
\pgfusepath{clip}%
\pgfsetbuttcap%
\pgfsetmiterjoin%
\definecolor{currentfill}{rgb}{0.501961,0.501961,0.501961}%
\pgfsetfillcolor{currentfill}%
\pgfsetfillopacity{0.700000}%
\pgfsetlinewidth{0.000000pt}%
\definecolor{currentstroke}{rgb}{0.000000,0.000000,0.000000}%
\pgfsetstrokecolor{currentstroke}%
\pgfsetstrokeopacity{0.700000}%
\pgfsetdash{}{0pt}%
\pgfpathmoveto{\pgfqpoint{7.656960in}{0.550000in}}%
\pgfpathlineto{\pgfqpoint{7.684482in}{0.550000in}}%
\pgfpathlineto{\pgfqpoint{7.684482in}{0.741113in}}%
\pgfpathlineto{\pgfqpoint{7.656960in}{0.741113in}}%
\pgfpathlineto{\pgfqpoint{7.656960in}{0.550000in}}%
\pgfpathclose%
\pgfusepath{fill}%
\end{pgfscope}%
\begin{pgfscope}%
\pgfpathrectangle{\pgfqpoint{1.250000in}{0.550000in}}{\pgfqpoint{7.750000in}{3.850000in}}%
\pgfusepath{clip}%
\pgfsetbuttcap%
\pgfsetmiterjoin%
\definecolor{currentfill}{rgb}{0.501961,0.501961,0.501961}%
\pgfsetfillcolor{currentfill}%
\pgfsetfillopacity{0.700000}%
\pgfsetlinewidth{0.000000pt}%
\definecolor{currentstroke}{rgb}{0.000000,0.000000,0.000000}%
\pgfsetstrokecolor{currentstroke}%
\pgfsetstrokeopacity{0.700000}%
\pgfsetdash{}{0pt}%
\pgfpathmoveto{\pgfqpoint{7.684482in}{0.550000in}}%
\pgfpathlineto{\pgfqpoint{7.712003in}{0.550000in}}%
\pgfpathlineto{\pgfqpoint{7.712003in}{0.732558in}}%
\pgfpathlineto{\pgfqpoint{7.684482in}{0.732558in}}%
\pgfpathlineto{\pgfqpoint{7.684482in}{0.550000in}}%
\pgfpathclose%
\pgfusepath{fill}%
\end{pgfscope}%
\begin{pgfscope}%
\pgfpathrectangle{\pgfqpoint{1.250000in}{0.550000in}}{\pgfqpoint{7.750000in}{3.850000in}}%
\pgfusepath{clip}%
\pgfsetbuttcap%
\pgfsetmiterjoin%
\definecolor{currentfill}{rgb}{0.501961,0.501961,0.501961}%
\pgfsetfillcolor{currentfill}%
\pgfsetfillopacity{0.700000}%
\pgfsetlinewidth{0.000000pt}%
\definecolor{currentstroke}{rgb}{0.000000,0.000000,0.000000}%
\pgfsetstrokecolor{currentstroke}%
\pgfsetstrokeopacity{0.700000}%
\pgfsetdash{}{0pt}%
\pgfpathmoveto{\pgfqpoint{7.712003in}{0.550000in}}%
\pgfpathlineto{\pgfqpoint{7.739524in}{0.550000in}}%
\pgfpathlineto{\pgfqpoint{7.739524in}{0.671367in}}%
\pgfpathlineto{\pgfqpoint{7.712003in}{0.671367in}}%
\pgfpathlineto{\pgfqpoint{7.712003in}{0.550000in}}%
\pgfpathclose%
\pgfusepath{fill}%
\end{pgfscope}%
\begin{pgfscope}%
\pgfpathrectangle{\pgfqpoint{1.250000in}{0.550000in}}{\pgfqpoint{7.750000in}{3.850000in}}%
\pgfusepath{clip}%
\pgfsetbuttcap%
\pgfsetmiterjoin%
\definecolor{currentfill}{rgb}{0.501961,0.501961,0.501961}%
\pgfsetfillcolor{currentfill}%
\pgfsetfillopacity{0.700000}%
\pgfsetlinewidth{0.000000pt}%
\definecolor{currentstroke}{rgb}{0.000000,0.000000,0.000000}%
\pgfsetstrokecolor{currentstroke}%
\pgfsetstrokeopacity{0.700000}%
\pgfsetdash{}{0pt}%
\pgfpathmoveto{\pgfqpoint{7.739524in}{0.550000in}}%
\pgfpathlineto{\pgfqpoint{7.767045in}{0.550000in}}%
\pgfpathlineto{\pgfqpoint{7.767045in}{0.734588in}}%
\pgfpathlineto{\pgfqpoint{7.739524in}{0.734588in}}%
\pgfpathlineto{\pgfqpoint{7.739524in}{0.550000in}}%
\pgfpathclose%
\pgfusepath{fill}%
\end{pgfscope}%
\begin{pgfscope}%
\pgfpathrectangle{\pgfqpoint{1.250000in}{0.550000in}}{\pgfqpoint{7.750000in}{3.850000in}}%
\pgfusepath{clip}%
\pgfsetbuttcap%
\pgfsetmiterjoin%
\definecolor{currentfill}{rgb}{0.501961,0.501961,0.501961}%
\pgfsetfillcolor{currentfill}%
\pgfsetfillopacity{0.700000}%
\pgfsetlinewidth{0.000000pt}%
\definecolor{currentstroke}{rgb}{0.000000,0.000000,0.000000}%
\pgfsetstrokecolor{currentstroke}%
\pgfsetstrokeopacity{0.700000}%
\pgfsetdash{}{0pt}%
\pgfpathmoveto{\pgfqpoint{7.767045in}{0.550000in}}%
\pgfpathlineto{\pgfqpoint{7.794567in}{0.550000in}}%
\pgfpathlineto{\pgfqpoint{7.794567in}{0.726612in}}%
\pgfpathlineto{\pgfqpoint{7.767045in}{0.726612in}}%
\pgfpathlineto{\pgfqpoint{7.767045in}{0.550000in}}%
\pgfpathclose%
\pgfusepath{fill}%
\end{pgfscope}%
\begin{pgfscope}%
\pgfpathrectangle{\pgfqpoint{1.250000in}{0.550000in}}{\pgfqpoint{7.750000in}{3.850000in}}%
\pgfusepath{clip}%
\pgfsetbuttcap%
\pgfsetmiterjoin%
\definecolor{currentfill}{rgb}{0.501961,0.501961,0.501961}%
\pgfsetfillcolor{currentfill}%
\pgfsetfillopacity{0.700000}%
\pgfsetlinewidth{0.000000pt}%
\definecolor{currentstroke}{rgb}{0.000000,0.000000,0.000000}%
\pgfsetstrokecolor{currentstroke}%
\pgfsetstrokeopacity{0.700000}%
\pgfsetdash{}{0pt}%
\pgfpathmoveto{\pgfqpoint{7.794567in}{0.550000in}}%
\pgfpathlineto{\pgfqpoint{7.822088in}{0.550000in}}%
\pgfpathlineto{\pgfqpoint{7.822088in}{0.673252in}}%
\pgfpathlineto{\pgfqpoint{7.794567in}{0.673252in}}%
\pgfpathlineto{\pgfqpoint{7.794567in}{0.550000in}}%
\pgfpathclose%
\pgfusepath{fill}%
\end{pgfscope}%
\begin{pgfscope}%
\pgfpathrectangle{\pgfqpoint{1.250000in}{0.550000in}}{\pgfqpoint{7.750000in}{3.850000in}}%
\pgfusepath{clip}%
\pgfsetbuttcap%
\pgfsetmiterjoin%
\definecolor{currentfill}{rgb}{0.501961,0.501961,0.501961}%
\pgfsetfillcolor{currentfill}%
\pgfsetfillopacity{0.700000}%
\pgfsetlinewidth{0.000000pt}%
\definecolor{currentstroke}{rgb}{0.000000,0.000000,0.000000}%
\pgfsetstrokecolor{currentstroke}%
\pgfsetstrokeopacity{0.700000}%
\pgfsetdash{}{0pt}%
\pgfpathmoveto{\pgfqpoint{7.822088in}{0.550000in}}%
\pgfpathlineto{\pgfqpoint{7.849609in}{0.550000in}}%
\pgfpathlineto{\pgfqpoint{7.849609in}{0.706892in}}%
\pgfpathlineto{\pgfqpoint{7.822088in}{0.706892in}}%
\pgfpathlineto{\pgfqpoint{7.822088in}{0.550000in}}%
\pgfpathclose%
\pgfusepath{fill}%
\end{pgfscope}%
\begin{pgfscope}%
\pgfpathrectangle{\pgfqpoint{1.250000in}{0.550000in}}{\pgfqpoint{7.750000in}{3.850000in}}%
\pgfusepath{clip}%
\pgfsetbuttcap%
\pgfsetmiterjoin%
\definecolor{currentfill}{rgb}{0.501961,0.501961,0.501961}%
\pgfsetfillcolor{currentfill}%
\pgfsetfillopacity{0.700000}%
\pgfsetlinewidth{0.000000pt}%
\definecolor{currentstroke}{rgb}{0.000000,0.000000,0.000000}%
\pgfsetstrokecolor{currentstroke}%
\pgfsetstrokeopacity{0.700000}%
\pgfsetdash{}{0pt}%
\pgfpathmoveto{\pgfqpoint{7.849609in}{0.550000in}}%
\pgfpathlineto{\pgfqpoint{7.877131in}{0.550000in}}%
\pgfpathlineto{\pgfqpoint{7.877131in}{0.745608in}}%
\pgfpathlineto{\pgfqpoint{7.849609in}{0.745608in}}%
\pgfpathlineto{\pgfqpoint{7.849609in}{0.550000in}}%
\pgfpathclose%
\pgfusepath{fill}%
\end{pgfscope}%
\begin{pgfscope}%
\pgfpathrectangle{\pgfqpoint{1.250000in}{0.550000in}}{\pgfqpoint{7.750000in}{3.850000in}}%
\pgfusepath{clip}%
\pgfsetbuttcap%
\pgfsetmiterjoin%
\definecolor{currentfill}{rgb}{0.501961,0.501961,0.501961}%
\pgfsetfillcolor{currentfill}%
\pgfsetfillopacity{0.700000}%
\pgfsetlinewidth{0.000000pt}%
\definecolor{currentstroke}{rgb}{0.000000,0.000000,0.000000}%
\pgfsetstrokecolor{currentstroke}%
\pgfsetstrokeopacity{0.700000}%
\pgfsetdash{}{0pt}%
\pgfpathmoveto{\pgfqpoint{7.877131in}{0.550000in}}%
\pgfpathlineto{\pgfqpoint{7.904652in}{0.550000in}}%
\pgfpathlineto{\pgfqpoint{7.904652in}{0.690072in}}%
\pgfpathlineto{\pgfqpoint{7.877131in}{0.690072in}}%
\pgfpathlineto{\pgfqpoint{7.877131in}{0.550000in}}%
\pgfpathclose%
\pgfusepath{fill}%
\end{pgfscope}%
\begin{pgfscope}%
\pgfpathrectangle{\pgfqpoint{1.250000in}{0.550000in}}{\pgfqpoint{7.750000in}{3.850000in}}%
\pgfusepath{clip}%
\pgfsetbuttcap%
\pgfsetmiterjoin%
\definecolor{currentfill}{rgb}{0.501961,0.501961,0.501961}%
\pgfsetfillcolor{currentfill}%
\pgfsetfillopacity{0.700000}%
\pgfsetlinewidth{0.000000pt}%
\definecolor{currentstroke}{rgb}{0.000000,0.000000,0.000000}%
\pgfsetstrokecolor{currentstroke}%
\pgfsetstrokeopacity{0.700000}%
\pgfsetdash{}{0pt}%
\pgfpathmoveto{\pgfqpoint{7.904652in}{0.550000in}}%
\pgfpathlineto{\pgfqpoint{7.932173in}{0.550000in}}%
\pgfpathlineto{\pgfqpoint{7.932173in}{0.722407in}}%
\pgfpathlineto{\pgfqpoint{7.904652in}{0.722407in}}%
\pgfpathlineto{\pgfqpoint{7.904652in}{0.550000in}}%
\pgfpathclose%
\pgfusepath{fill}%
\end{pgfscope}%
\begin{pgfscope}%
\pgfpathrectangle{\pgfqpoint{1.250000in}{0.550000in}}{\pgfqpoint{7.750000in}{3.850000in}}%
\pgfusepath{clip}%
\pgfsetbuttcap%
\pgfsetmiterjoin%
\definecolor{currentfill}{rgb}{0.501961,0.501961,0.501961}%
\pgfsetfillcolor{currentfill}%
\pgfsetfillopacity{0.700000}%
\pgfsetlinewidth{0.000000pt}%
\definecolor{currentstroke}{rgb}{0.000000,0.000000,0.000000}%
\pgfsetstrokecolor{currentstroke}%
\pgfsetstrokeopacity{0.700000}%
\pgfsetdash{}{0pt}%
\pgfpathmoveto{\pgfqpoint{7.932173in}{0.550000in}}%
\pgfpathlineto{\pgfqpoint{7.959695in}{0.550000in}}%
\pgfpathlineto{\pgfqpoint{7.959695in}{0.706892in}}%
\pgfpathlineto{\pgfqpoint{7.932173in}{0.706892in}}%
\pgfpathlineto{\pgfqpoint{7.932173in}{0.550000in}}%
\pgfpathclose%
\pgfusepath{fill}%
\end{pgfscope}%
\begin{pgfscope}%
\pgfpathrectangle{\pgfqpoint{1.250000in}{0.550000in}}{\pgfqpoint{7.750000in}{3.850000in}}%
\pgfusepath{clip}%
\pgfsetbuttcap%
\pgfsetmiterjoin%
\definecolor{currentfill}{rgb}{0.501961,0.501961,0.501961}%
\pgfsetfillcolor{currentfill}%
\pgfsetfillopacity{0.700000}%
\pgfsetlinewidth{0.000000pt}%
\definecolor{currentstroke}{rgb}{0.000000,0.000000,0.000000}%
\pgfsetstrokecolor{currentstroke}%
\pgfsetstrokeopacity{0.700000}%
\pgfsetdash{}{0pt}%
\pgfpathmoveto{\pgfqpoint{7.959695in}{0.550000in}}%
\pgfpathlineto{\pgfqpoint{7.987216in}{0.550000in}}%
\pgfpathlineto{\pgfqpoint{7.987216in}{0.703992in}}%
\pgfpathlineto{\pgfqpoint{7.959695in}{0.703992in}}%
\pgfpathlineto{\pgfqpoint{7.959695in}{0.550000in}}%
\pgfpathclose%
\pgfusepath{fill}%
\end{pgfscope}%
\begin{pgfscope}%
\pgfpathrectangle{\pgfqpoint{1.250000in}{0.550000in}}{\pgfqpoint{7.750000in}{3.850000in}}%
\pgfusepath{clip}%
\pgfsetbuttcap%
\pgfsetmiterjoin%
\definecolor{currentfill}{rgb}{0.501961,0.501961,0.501961}%
\pgfsetfillcolor{currentfill}%
\pgfsetfillopacity{0.700000}%
\pgfsetlinewidth{0.000000pt}%
\definecolor{currentstroke}{rgb}{0.000000,0.000000,0.000000}%
\pgfsetstrokecolor{currentstroke}%
\pgfsetstrokeopacity{0.700000}%
\pgfsetdash{}{0pt}%
\pgfpathmoveto{\pgfqpoint{7.987216in}{0.550000in}}%
\pgfpathlineto{\pgfqpoint{8.014737in}{0.550000in}}%
\pgfpathlineto{\pgfqpoint{8.014737in}{0.726902in}}%
\pgfpathlineto{\pgfqpoint{7.987216in}{0.726902in}}%
\pgfpathlineto{\pgfqpoint{7.987216in}{0.550000in}}%
\pgfpathclose%
\pgfusepath{fill}%
\end{pgfscope}%
\begin{pgfscope}%
\pgfpathrectangle{\pgfqpoint{1.250000in}{0.550000in}}{\pgfqpoint{7.750000in}{3.850000in}}%
\pgfusepath{clip}%
\pgfsetbuttcap%
\pgfsetmiterjoin%
\definecolor{currentfill}{rgb}{0.501961,0.501961,0.501961}%
\pgfsetfillcolor{currentfill}%
\pgfsetfillopacity{0.700000}%
\pgfsetlinewidth{0.000000pt}%
\definecolor{currentstroke}{rgb}{0.000000,0.000000,0.000000}%
\pgfsetstrokecolor{currentstroke}%
\pgfsetstrokeopacity{0.700000}%
\pgfsetdash{}{0pt}%
\pgfpathmoveto{\pgfqpoint{8.014737in}{0.550000in}}%
\pgfpathlineto{\pgfqpoint{8.042259in}{0.550000in}}%
\pgfpathlineto{\pgfqpoint{8.042259in}{0.692682in}}%
\pgfpathlineto{\pgfqpoint{8.014737in}{0.692682in}}%
\pgfpathlineto{\pgfqpoint{8.014737in}{0.550000in}}%
\pgfpathclose%
\pgfusepath{fill}%
\end{pgfscope}%
\begin{pgfscope}%
\pgfpathrectangle{\pgfqpoint{1.250000in}{0.550000in}}{\pgfqpoint{7.750000in}{3.850000in}}%
\pgfusepath{clip}%
\pgfsetbuttcap%
\pgfsetmiterjoin%
\definecolor{currentfill}{rgb}{0.501961,0.501961,0.501961}%
\pgfsetfillcolor{currentfill}%
\pgfsetfillopacity{0.700000}%
\pgfsetlinewidth{0.000000pt}%
\definecolor{currentstroke}{rgb}{0.000000,0.000000,0.000000}%
\pgfsetstrokecolor{currentstroke}%
\pgfsetstrokeopacity{0.700000}%
\pgfsetdash{}{0pt}%
\pgfpathmoveto{\pgfqpoint{8.042259in}{0.550000in}}%
\pgfpathlineto{\pgfqpoint{8.069780in}{0.550000in}}%
\pgfpathlineto{\pgfqpoint{8.069780in}{0.718637in}}%
\pgfpathlineto{\pgfqpoint{8.042259in}{0.718637in}}%
\pgfpathlineto{\pgfqpoint{8.042259in}{0.550000in}}%
\pgfpathclose%
\pgfusepath{fill}%
\end{pgfscope}%
\begin{pgfscope}%
\pgfpathrectangle{\pgfqpoint{1.250000in}{0.550000in}}{\pgfqpoint{7.750000in}{3.850000in}}%
\pgfusepath{clip}%
\pgfsetbuttcap%
\pgfsetmiterjoin%
\definecolor{currentfill}{rgb}{0.501961,0.501961,0.501961}%
\pgfsetfillcolor{currentfill}%
\pgfsetfillopacity{0.700000}%
\pgfsetlinewidth{0.000000pt}%
\definecolor{currentstroke}{rgb}{0.000000,0.000000,0.000000}%
\pgfsetstrokecolor{currentstroke}%
\pgfsetstrokeopacity{0.700000}%
\pgfsetdash{}{0pt}%
\pgfpathmoveto{\pgfqpoint{8.069780in}{0.550000in}}%
\pgfpathlineto{\pgfqpoint{8.097301in}{0.550000in}}%
\pgfpathlineto{\pgfqpoint{8.097301in}{0.702542in}}%
\pgfpathlineto{\pgfqpoint{8.069780in}{0.702542in}}%
\pgfpathlineto{\pgfqpoint{8.069780in}{0.550000in}}%
\pgfpathclose%
\pgfusepath{fill}%
\end{pgfscope}%
\begin{pgfscope}%
\pgfpathrectangle{\pgfqpoint{1.250000in}{0.550000in}}{\pgfqpoint{7.750000in}{3.850000in}}%
\pgfusepath{clip}%
\pgfsetbuttcap%
\pgfsetmiterjoin%
\definecolor{currentfill}{rgb}{0.501961,0.501961,0.501961}%
\pgfsetfillcolor{currentfill}%
\pgfsetfillopacity{0.700000}%
\pgfsetlinewidth{0.000000pt}%
\definecolor{currentstroke}{rgb}{0.000000,0.000000,0.000000}%
\pgfsetstrokecolor{currentstroke}%
\pgfsetstrokeopacity{0.700000}%
\pgfsetdash{}{0pt}%
\pgfpathmoveto{\pgfqpoint{8.097301in}{0.550000in}}%
\pgfpathlineto{\pgfqpoint{8.124822in}{0.550000in}}%
\pgfpathlineto{\pgfqpoint{8.124822in}{0.712547in}}%
\pgfpathlineto{\pgfqpoint{8.097301in}{0.712547in}}%
\pgfpathlineto{\pgfqpoint{8.097301in}{0.550000in}}%
\pgfpathclose%
\pgfusepath{fill}%
\end{pgfscope}%
\begin{pgfscope}%
\pgfpathrectangle{\pgfqpoint{1.250000in}{0.550000in}}{\pgfqpoint{7.750000in}{3.850000in}}%
\pgfusepath{clip}%
\pgfsetbuttcap%
\pgfsetmiterjoin%
\definecolor{currentfill}{rgb}{0.501961,0.501961,0.501961}%
\pgfsetfillcolor{currentfill}%
\pgfsetfillopacity{0.700000}%
\pgfsetlinewidth{0.000000pt}%
\definecolor{currentstroke}{rgb}{0.000000,0.000000,0.000000}%
\pgfsetstrokecolor{currentstroke}%
\pgfsetstrokeopacity{0.700000}%
\pgfsetdash{}{0pt}%
\pgfpathmoveto{\pgfqpoint{8.124822in}{0.550000in}}%
\pgfpathlineto{\pgfqpoint{8.152344in}{0.550000in}}%
\pgfpathlineto{\pgfqpoint{8.152344in}{0.700657in}}%
\pgfpathlineto{\pgfqpoint{8.124822in}{0.700657in}}%
\pgfpathlineto{\pgfqpoint{8.124822in}{0.550000in}}%
\pgfpathclose%
\pgfusepath{fill}%
\end{pgfscope}%
\begin{pgfscope}%
\pgfpathrectangle{\pgfqpoint{1.250000in}{0.550000in}}{\pgfqpoint{7.750000in}{3.850000in}}%
\pgfusepath{clip}%
\pgfsetbuttcap%
\pgfsetmiterjoin%
\definecolor{currentfill}{rgb}{0.501961,0.501961,0.501961}%
\pgfsetfillcolor{currentfill}%
\pgfsetfillopacity{0.700000}%
\pgfsetlinewidth{0.000000pt}%
\definecolor{currentstroke}{rgb}{0.000000,0.000000,0.000000}%
\pgfsetstrokecolor{currentstroke}%
\pgfsetstrokeopacity{0.700000}%
\pgfsetdash{}{0pt}%
\pgfpathmoveto{\pgfqpoint{8.152344in}{0.550000in}}%
\pgfpathlineto{\pgfqpoint{8.179865in}{0.550000in}}%
\pgfpathlineto{\pgfqpoint{8.179865in}{0.713272in}}%
\pgfpathlineto{\pgfqpoint{8.152344in}{0.713272in}}%
\pgfpathlineto{\pgfqpoint{8.152344in}{0.550000in}}%
\pgfpathclose%
\pgfusepath{fill}%
\end{pgfscope}%
\begin{pgfscope}%
\pgfpathrectangle{\pgfqpoint{1.250000in}{0.550000in}}{\pgfqpoint{7.750000in}{3.850000in}}%
\pgfusepath{clip}%
\pgfsetbuttcap%
\pgfsetmiterjoin%
\definecolor{currentfill}{rgb}{0.501961,0.501961,0.501961}%
\pgfsetfillcolor{currentfill}%
\pgfsetfillopacity{0.700000}%
\pgfsetlinewidth{0.000000pt}%
\definecolor{currentstroke}{rgb}{0.000000,0.000000,0.000000}%
\pgfsetstrokecolor{currentstroke}%
\pgfsetstrokeopacity{0.700000}%
\pgfsetdash{}{0pt}%
\pgfpathmoveto{\pgfqpoint{8.179865in}{0.550000in}}%
\pgfpathlineto{\pgfqpoint{8.207386in}{0.550000in}}%
\pgfpathlineto{\pgfqpoint{8.207386in}{0.722697in}}%
\pgfpathlineto{\pgfqpoint{8.179865in}{0.722697in}}%
\pgfpathlineto{\pgfqpoint{8.179865in}{0.550000in}}%
\pgfpathclose%
\pgfusepath{fill}%
\end{pgfscope}%
\begin{pgfscope}%
\pgfpathrectangle{\pgfqpoint{1.250000in}{0.550000in}}{\pgfqpoint{7.750000in}{3.850000in}}%
\pgfusepath{clip}%
\pgfsetbuttcap%
\pgfsetmiterjoin%
\definecolor{currentfill}{rgb}{0.501961,0.501961,0.501961}%
\pgfsetfillcolor{currentfill}%
\pgfsetfillopacity{0.700000}%
\pgfsetlinewidth{0.000000pt}%
\definecolor{currentstroke}{rgb}{0.000000,0.000000,0.000000}%
\pgfsetstrokecolor{currentstroke}%
\pgfsetstrokeopacity{0.700000}%
\pgfsetdash{}{0pt}%
\pgfpathmoveto{\pgfqpoint{8.207386in}{0.550000in}}%
\pgfpathlineto{\pgfqpoint{8.234908in}{0.550000in}}%
\pgfpathlineto{\pgfqpoint{8.234908in}{0.716462in}}%
\pgfpathlineto{\pgfqpoint{8.207386in}{0.716462in}}%
\pgfpathlineto{\pgfqpoint{8.207386in}{0.550000in}}%
\pgfpathclose%
\pgfusepath{fill}%
\end{pgfscope}%
\begin{pgfscope}%
\pgfpathrectangle{\pgfqpoint{1.250000in}{0.550000in}}{\pgfqpoint{7.750000in}{3.850000in}}%
\pgfusepath{clip}%
\pgfsetbuttcap%
\pgfsetmiterjoin%
\definecolor{currentfill}{rgb}{0.501961,0.501961,0.501961}%
\pgfsetfillcolor{currentfill}%
\pgfsetfillopacity{0.700000}%
\pgfsetlinewidth{0.000000pt}%
\definecolor{currentstroke}{rgb}{0.000000,0.000000,0.000000}%
\pgfsetstrokecolor{currentstroke}%
\pgfsetstrokeopacity{0.700000}%
\pgfsetdash{}{0pt}%
\pgfpathmoveto{\pgfqpoint{8.234908in}{0.550000in}}%
\pgfpathlineto{\pgfqpoint{8.262429in}{0.550000in}}%
\pgfpathlineto{\pgfqpoint{8.262429in}{0.694857in}}%
\pgfpathlineto{\pgfqpoint{8.234908in}{0.694857in}}%
\pgfpathlineto{\pgfqpoint{8.234908in}{0.550000in}}%
\pgfpathclose%
\pgfusepath{fill}%
\end{pgfscope}%
\begin{pgfscope}%
\pgfpathrectangle{\pgfqpoint{1.250000in}{0.550000in}}{\pgfqpoint{7.750000in}{3.850000in}}%
\pgfusepath{clip}%
\pgfsetbuttcap%
\pgfsetmiterjoin%
\definecolor{currentfill}{rgb}{0.501961,0.501961,0.501961}%
\pgfsetfillcolor{currentfill}%
\pgfsetfillopacity{0.700000}%
\pgfsetlinewidth{0.000000pt}%
\definecolor{currentstroke}{rgb}{0.000000,0.000000,0.000000}%
\pgfsetstrokecolor{currentstroke}%
\pgfsetstrokeopacity{0.700000}%
\pgfsetdash{}{0pt}%
\pgfpathmoveto{\pgfqpoint{8.262429in}{0.550000in}}%
\pgfpathlineto{\pgfqpoint{8.289950in}{0.550000in}}%
\pgfpathlineto{\pgfqpoint{8.289950in}{0.732268in}}%
\pgfpathlineto{\pgfqpoint{8.262429in}{0.732268in}}%
\pgfpathlineto{\pgfqpoint{8.262429in}{0.550000in}}%
\pgfpathclose%
\pgfusepath{fill}%
\end{pgfscope}%
\begin{pgfscope}%
\pgfpathrectangle{\pgfqpoint{1.250000in}{0.550000in}}{\pgfqpoint{7.750000in}{3.850000in}}%
\pgfusepath{clip}%
\pgfsetbuttcap%
\pgfsetmiterjoin%
\definecolor{currentfill}{rgb}{0.501961,0.501961,0.501961}%
\pgfsetfillcolor{currentfill}%
\pgfsetfillopacity{0.700000}%
\pgfsetlinewidth{0.000000pt}%
\definecolor{currentstroke}{rgb}{0.000000,0.000000,0.000000}%
\pgfsetstrokecolor{currentstroke}%
\pgfsetstrokeopacity{0.700000}%
\pgfsetdash{}{0pt}%
\pgfpathmoveto{\pgfqpoint{8.289950in}{0.550000in}}%
\pgfpathlineto{\pgfqpoint{8.317472in}{0.550000in}}%
\pgfpathlineto{\pgfqpoint{8.317472in}{0.709212in}}%
\pgfpathlineto{\pgfqpoint{8.289950in}{0.709212in}}%
\pgfpathlineto{\pgfqpoint{8.289950in}{0.550000in}}%
\pgfpathclose%
\pgfusepath{fill}%
\end{pgfscope}%
\begin{pgfscope}%
\pgfpathrectangle{\pgfqpoint{1.250000in}{0.550000in}}{\pgfqpoint{7.750000in}{3.850000in}}%
\pgfusepath{clip}%
\pgfsetbuttcap%
\pgfsetmiterjoin%
\definecolor{currentfill}{rgb}{0.501961,0.501961,0.501961}%
\pgfsetfillcolor{currentfill}%
\pgfsetfillopacity{0.700000}%
\pgfsetlinewidth{0.000000pt}%
\definecolor{currentstroke}{rgb}{0.000000,0.000000,0.000000}%
\pgfsetstrokecolor{currentstroke}%
\pgfsetstrokeopacity{0.700000}%
\pgfsetdash{}{0pt}%
\pgfpathmoveto{\pgfqpoint{8.317472in}{0.550000in}}%
\pgfpathlineto{\pgfqpoint{8.344993in}{0.550000in}}%
\pgfpathlineto{\pgfqpoint{8.344993in}{0.709647in}}%
\pgfpathlineto{\pgfqpoint{8.317472in}{0.709647in}}%
\pgfpathlineto{\pgfqpoint{8.317472in}{0.550000in}}%
\pgfpathclose%
\pgfusepath{fill}%
\end{pgfscope}%
\begin{pgfscope}%
\pgfpathrectangle{\pgfqpoint{1.250000in}{0.550000in}}{\pgfqpoint{7.750000in}{3.850000in}}%
\pgfusepath{clip}%
\pgfsetbuttcap%
\pgfsetmiterjoin%
\definecolor{currentfill}{rgb}{0.501961,0.501961,0.501961}%
\pgfsetfillcolor{currentfill}%
\pgfsetfillopacity{0.700000}%
\pgfsetlinewidth{0.000000pt}%
\definecolor{currentstroke}{rgb}{0.000000,0.000000,0.000000}%
\pgfsetstrokecolor{currentstroke}%
\pgfsetstrokeopacity{0.700000}%
\pgfsetdash{}{0pt}%
\pgfpathmoveto{\pgfqpoint{8.344993in}{0.550000in}}%
\pgfpathlineto{\pgfqpoint{8.372514in}{0.550000in}}%
\pgfpathlineto{\pgfqpoint{8.372514in}{0.711822in}}%
\pgfpathlineto{\pgfqpoint{8.344993in}{0.711822in}}%
\pgfpathlineto{\pgfqpoint{8.344993in}{0.550000in}}%
\pgfpathclose%
\pgfusepath{fill}%
\end{pgfscope}%
\begin{pgfscope}%
\pgfpathrectangle{\pgfqpoint{1.250000in}{0.550000in}}{\pgfqpoint{7.750000in}{3.850000in}}%
\pgfusepath{clip}%
\pgfsetbuttcap%
\pgfsetmiterjoin%
\definecolor{currentfill}{rgb}{0.501961,0.501961,0.501961}%
\pgfsetfillcolor{currentfill}%
\pgfsetfillopacity{0.700000}%
\pgfsetlinewidth{0.000000pt}%
\definecolor{currentstroke}{rgb}{0.000000,0.000000,0.000000}%
\pgfsetstrokecolor{currentstroke}%
\pgfsetstrokeopacity{0.700000}%
\pgfsetdash{}{0pt}%
\pgfpathmoveto{\pgfqpoint{8.372514in}{0.550000in}}%
\pgfpathlineto{\pgfqpoint{8.400036in}{0.550000in}}%
\pgfpathlineto{\pgfqpoint{8.400036in}{0.701962in}}%
\pgfpathlineto{\pgfqpoint{8.372514in}{0.701962in}}%
\pgfpathlineto{\pgfqpoint{8.372514in}{0.550000in}}%
\pgfpathclose%
\pgfusepath{fill}%
\end{pgfscope}%
\begin{pgfscope}%
\pgfpathrectangle{\pgfqpoint{1.250000in}{0.550000in}}{\pgfqpoint{7.750000in}{3.850000in}}%
\pgfusepath{clip}%
\pgfsetbuttcap%
\pgfsetmiterjoin%
\definecolor{currentfill}{rgb}{0.501961,0.501961,0.501961}%
\pgfsetfillcolor{currentfill}%
\pgfsetfillopacity{0.700000}%
\pgfsetlinewidth{0.000000pt}%
\definecolor{currentstroke}{rgb}{0.000000,0.000000,0.000000}%
\pgfsetstrokecolor{currentstroke}%
\pgfsetstrokeopacity{0.700000}%
\pgfsetdash{}{0pt}%
\pgfpathmoveto{\pgfqpoint{8.400036in}{0.550000in}}%
\pgfpathlineto{\pgfqpoint{8.427557in}{0.550000in}}%
\pgfpathlineto{\pgfqpoint{8.427557in}{0.706602in}}%
\pgfpathlineto{\pgfqpoint{8.400036in}{0.706602in}}%
\pgfpathlineto{\pgfqpoint{8.400036in}{0.550000in}}%
\pgfpathclose%
\pgfusepath{fill}%
\end{pgfscope}%
\begin{pgfscope}%
\pgfpathrectangle{\pgfqpoint{1.250000in}{0.550000in}}{\pgfqpoint{7.750000in}{3.850000in}}%
\pgfusepath{clip}%
\pgfsetbuttcap%
\pgfsetmiterjoin%
\definecolor{currentfill}{rgb}{0.501961,0.501961,0.501961}%
\pgfsetfillcolor{currentfill}%
\pgfsetfillopacity{0.700000}%
\pgfsetlinewidth{0.000000pt}%
\definecolor{currentstroke}{rgb}{0.000000,0.000000,0.000000}%
\pgfsetstrokecolor{currentstroke}%
\pgfsetstrokeopacity{0.700000}%
\pgfsetdash{}{0pt}%
\pgfpathmoveto{\pgfqpoint{8.427557in}{0.550000in}}%
\pgfpathlineto{\pgfqpoint{8.455078in}{0.550000in}}%
\pgfpathlineto{\pgfqpoint{8.455078in}{0.710082in}}%
\pgfpathlineto{\pgfqpoint{8.427557in}{0.710082in}}%
\pgfpathlineto{\pgfqpoint{8.427557in}{0.550000in}}%
\pgfpathclose%
\pgfusepath{fill}%
\end{pgfscope}%
\begin{pgfscope}%
\pgfpathrectangle{\pgfqpoint{1.250000in}{0.550000in}}{\pgfqpoint{7.750000in}{3.850000in}}%
\pgfusepath{clip}%
\pgfsetbuttcap%
\pgfsetmiterjoin%
\definecolor{currentfill}{rgb}{0.501961,0.501961,0.501961}%
\pgfsetfillcolor{currentfill}%
\pgfsetfillopacity{0.700000}%
\pgfsetlinewidth{0.000000pt}%
\definecolor{currentstroke}{rgb}{0.000000,0.000000,0.000000}%
\pgfsetstrokecolor{currentstroke}%
\pgfsetstrokeopacity{0.700000}%
\pgfsetdash{}{0pt}%
\pgfpathmoveto{\pgfqpoint{8.455078in}{0.550000in}}%
\pgfpathlineto{\pgfqpoint{8.482599in}{0.550000in}}%
\pgfpathlineto{\pgfqpoint{8.482599in}{0.692972in}}%
\pgfpathlineto{\pgfqpoint{8.455078in}{0.692972in}}%
\pgfpathlineto{\pgfqpoint{8.455078in}{0.550000in}}%
\pgfpathclose%
\pgfusepath{fill}%
\end{pgfscope}%
\begin{pgfscope}%
\pgfpathrectangle{\pgfqpoint{1.250000in}{0.550000in}}{\pgfqpoint{7.750000in}{3.850000in}}%
\pgfusepath{clip}%
\pgfsetbuttcap%
\pgfsetmiterjoin%
\definecolor{currentfill}{rgb}{0.501961,0.501961,0.501961}%
\pgfsetfillcolor{currentfill}%
\pgfsetfillopacity{0.700000}%
\pgfsetlinewidth{0.000000pt}%
\definecolor{currentstroke}{rgb}{0.000000,0.000000,0.000000}%
\pgfsetstrokecolor{currentstroke}%
\pgfsetstrokeopacity{0.700000}%
\pgfsetdash{}{0pt}%
\pgfpathmoveto{\pgfqpoint{8.482599in}{0.550000in}}%
\pgfpathlineto{\pgfqpoint{8.510121in}{0.550000in}}%
\pgfpathlineto{\pgfqpoint{8.510121in}{0.693842in}}%
\pgfpathlineto{\pgfqpoint{8.482599in}{0.693842in}}%
\pgfpathlineto{\pgfqpoint{8.482599in}{0.550000in}}%
\pgfpathclose%
\pgfusepath{fill}%
\end{pgfscope}%
\begin{pgfscope}%
\pgfpathrectangle{\pgfqpoint{1.250000in}{0.550000in}}{\pgfqpoint{7.750000in}{3.850000in}}%
\pgfusepath{clip}%
\pgfsetbuttcap%
\pgfsetmiterjoin%
\definecolor{currentfill}{rgb}{0.501961,0.501961,0.501961}%
\pgfsetfillcolor{currentfill}%
\pgfsetfillopacity{0.700000}%
\pgfsetlinewidth{0.000000pt}%
\definecolor{currentstroke}{rgb}{0.000000,0.000000,0.000000}%
\pgfsetstrokecolor{currentstroke}%
\pgfsetstrokeopacity{0.700000}%
\pgfsetdash{}{0pt}%
\pgfpathmoveto{\pgfqpoint{8.510121in}{0.550000in}}%
\pgfpathlineto{\pgfqpoint{8.537642in}{0.550000in}}%
\pgfpathlineto{\pgfqpoint{8.537642in}{0.655561in}}%
\pgfpathlineto{\pgfqpoint{8.510121in}{0.655561in}}%
\pgfpathlineto{\pgfqpoint{8.510121in}{0.550000in}}%
\pgfpathclose%
\pgfusepath{fill}%
\end{pgfscope}%
\begin{pgfscope}%
\pgfpathrectangle{\pgfqpoint{1.250000in}{0.550000in}}{\pgfqpoint{7.750000in}{3.850000in}}%
\pgfusepath{clip}%
\pgfsetbuttcap%
\pgfsetmiterjoin%
\definecolor{currentfill}{rgb}{0.501961,0.501961,0.501961}%
\pgfsetfillcolor{currentfill}%
\pgfsetfillopacity{0.700000}%
\pgfsetlinewidth{0.000000pt}%
\definecolor{currentstroke}{rgb}{0.000000,0.000000,0.000000}%
\pgfsetstrokecolor{currentstroke}%
\pgfsetstrokeopacity{0.700000}%
\pgfsetdash{}{0pt}%
\pgfpathmoveto{\pgfqpoint{8.537642in}{0.550000in}}%
\pgfpathlineto{\pgfqpoint{8.565163in}{0.550000in}}%
\pgfpathlineto{\pgfqpoint{8.565163in}{0.694857in}}%
\pgfpathlineto{\pgfqpoint{8.537642in}{0.694857in}}%
\pgfpathlineto{\pgfqpoint{8.537642in}{0.550000in}}%
\pgfpathclose%
\pgfusepath{fill}%
\end{pgfscope}%
\begin{pgfscope}%
\pgfpathrectangle{\pgfqpoint{1.250000in}{0.550000in}}{\pgfqpoint{7.750000in}{3.850000in}}%
\pgfusepath{clip}%
\pgfsetbuttcap%
\pgfsetmiterjoin%
\definecolor{currentfill}{rgb}{0.501961,0.501961,0.501961}%
\pgfsetfillcolor{currentfill}%
\pgfsetfillopacity{0.700000}%
\pgfsetlinewidth{0.000000pt}%
\definecolor{currentstroke}{rgb}{0.000000,0.000000,0.000000}%
\pgfsetstrokecolor{currentstroke}%
\pgfsetstrokeopacity{0.700000}%
\pgfsetdash{}{0pt}%
\pgfpathmoveto{\pgfqpoint{8.565163in}{0.550000in}}%
\pgfpathlineto{\pgfqpoint{8.592685in}{0.550000in}}%
\pgfpathlineto{\pgfqpoint{8.592685in}{0.741113in}}%
\pgfpathlineto{\pgfqpoint{8.565163in}{0.741113in}}%
\pgfpathlineto{\pgfqpoint{8.565163in}{0.550000in}}%
\pgfpathclose%
\pgfusepath{fill}%
\end{pgfscope}%
\begin{pgfscope}%
\pgfpathrectangle{\pgfqpoint{1.250000in}{0.550000in}}{\pgfqpoint{7.750000in}{3.850000in}}%
\pgfusepath{clip}%
\pgfsetbuttcap%
\pgfsetmiterjoin%
\definecolor{currentfill}{rgb}{0.501961,0.501961,0.501961}%
\pgfsetfillcolor{currentfill}%
\pgfsetfillopacity{0.700000}%
\pgfsetlinewidth{0.000000pt}%
\definecolor{currentstroke}{rgb}{0.000000,0.000000,0.000000}%
\pgfsetstrokecolor{currentstroke}%
\pgfsetstrokeopacity{0.700000}%
\pgfsetdash{}{0pt}%
\pgfpathmoveto{\pgfqpoint{8.592685in}{0.550000in}}%
\pgfpathlineto{\pgfqpoint{8.620206in}{0.550000in}}%
\pgfpathlineto{\pgfqpoint{8.620206in}{0.550000in}}%
\pgfpathlineto{\pgfqpoint{8.592685in}{0.550000in}}%
\pgfpathlineto{\pgfqpoint{8.592685in}{0.550000in}}%
\pgfpathclose%
\pgfusepath{fill}%
\end{pgfscope}%
\begin{pgfscope}%
\pgfpathrectangle{\pgfqpoint{1.250000in}{0.550000in}}{\pgfqpoint{7.750000in}{3.850000in}}%
\pgfusepath{clip}%
\pgfsetbuttcap%
\pgfsetmiterjoin%
\definecolor{currentfill}{rgb}{0.501961,0.501961,0.501961}%
\pgfsetfillcolor{currentfill}%
\pgfsetfillopacity{0.700000}%
\pgfsetlinewidth{0.000000pt}%
\definecolor{currentstroke}{rgb}{0.000000,0.000000,0.000000}%
\pgfsetstrokecolor{currentstroke}%
\pgfsetstrokeopacity{0.700000}%
\pgfsetdash{}{0pt}%
\pgfpathmoveto{\pgfqpoint{8.620206in}{0.550000in}}%
\pgfpathlineto{\pgfqpoint{8.647727in}{0.550000in}}%
\pgfpathlineto{\pgfqpoint{8.647727in}{0.880025in}}%
\pgfpathlineto{\pgfqpoint{8.620206in}{0.880025in}}%
\pgfpathlineto{\pgfqpoint{8.620206in}{0.550000in}}%
\pgfpathclose%
\pgfusepath{fill}%
\end{pgfscope}%
\begin{pgfscope}%
\pgfsetbuttcap%
\pgfsetroundjoin%
\definecolor{currentfill}{rgb}{0.000000,0.000000,0.000000}%
\pgfsetfillcolor{currentfill}%
\pgfsetlinewidth{0.803000pt}%
\definecolor{currentstroke}{rgb}{0.000000,0.000000,0.000000}%
\pgfsetstrokecolor{currentstroke}%
\pgfsetdash{}{0pt}%
\pgfsys@defobject{currentmarker}{\pgfqpoint{0.000000in}{-0.048611in}}{\pgfqpoint{0.000000in}{0.000000in}}{%
\pgfpathmoveto{\pgfqpoint{0.000000in}{0.000000in}}%
\pgfpathlineto{\pgfqpoint{0.000000in}{-0.048611in}}%
\pgfusepath{stroke,fill}%
}%
\begin{pgfscope}%
\pgfsys@transformshift{1.602273in}{0.550000in}%
\pgfsys@useobject{currentmarker}{}%
\end{pgfscope}%
\end{pgfscope}%
\begin{pgfscope}%
\definecolor{textcolor}{rgb}{0.000000,0.000000,0.000000}%
\pgfsetstrokecolor{textcolor}%
\pgfsetfillcolor{textcolor}%
\pgftext[x=1.602273in,y=0.452778in,,top]{\color{textcolor}{\sffamily\fontsize{10.000000}{12.000000}\selectfont\catcode`\^=\active\def^{\ifmmode\sp\else\^{}\fi}\catcode`\%=\active\def%{\%}0}}%
\end{pgfscope}%
\begin{pgfscope}%
\pgfsetbuttcap%
\pgfsetroundjoin%
\definecolor{currentfill}{rgb}{0.000000,0.000000,0.000000}%
\pgfsetfillcolor{currentfill}%
\pgfsetlinewidth{0.803000pt}%
\definecolor{currentstroke}{rgb}{0.000000,0.000000,0.000000}%
\pgfsetstrokecolor{currentstroke}%
\pgfsetdash{}{0pt}%
\pgfsys@defobject{currentmarker}{\pgfqpoint{0.000000in}{-0.048611in}}{\pgfqpoint{0.000000in}{0.000000in}}{%
\pgfpathmoveto{\pgfqpoint{0.000000in}{0.000000in}}%
\pgfpathlineto{\pgfqpoint{0.000000in}{-0.048611in}}%
\pgfusepath{stroke,fill}%
}%
\begin{pgfscope}%
\pgfsys@transformshift{2.978338in}{0.550000in}%
\pgfsys@useobject{currentmarker}{}%
\end{pgfscope}%
\end{pgfscope}%
\begin{pgfscope}%
\definecolor{textcolor}{rgb}{0.000000,0.000000,0.000000}%
\pgfsetstrokecolor{textcolor}%
\pgfsetfillcolor{textcolor}%
\pgftext[x=2.978338in,y=0.452778in,,top]{\color{textcolor}{\sffamily\fontsize{10.000000}{12.000000}\selectfont\catcode`\^=\active\def^{\ifmmode\sp\else\^{}\fi}\catcode`\%=\active\def%{\%}50}}%
\end{pgfscope}%
\begin{pgfscope}%
\pgfsetbuttcap%
\pgfsetroundjoin%
\definecolor{currentfill}{rgb}{0.000000,0.000000,0.000000}%
\pgfsetfillcolor{currentfill}%
\pgfsetlinewidth{0.803000pt}%
\definecolor{currentstroke}{rgb}{0.000000,0.000000,0.000000}%
\pgfsetstrokecolor{currentstroke}%
\pgfsetdash{}{0pt}%
\pgfsys@defobject{currentmarker}{\pgfqpoint{0.000000in}{-0.048611in}}{\pgfqpoint{0.000000in}{0.000000in}}{%
\pgfpathmoveto{\pgfqpoint{0.000000in}{0.000000in}}%
\pgfpathlineto{\pgfqpoint{0.000000in}{-0.048611in}}%
\pgfusepath{stroke,fill}%
}%
\begin{pgfscope}%
\pgfsys@transformshift{4.354403in}{0.550000in}%
\pgfsys@useobject{currentmarker}{}%
\end{pgfscope}%
\end{pgfscope}%
\begin{pgfscope}%
\definecolor{textcolor}{rgb}{0.000000,0.000000,0.000000}%
\pgfsetstrokecolor{textcolor}%
\pgfsetfillcolor{textcolor}%
\pgftext[x=4.354403in,y=0.452778in,,top]{\color{textcolor}{\sffamily\fontsize{10.000000}{12.000000}\selectfont\catcode`\^=\active\def^{\ifmmode\sp\else\^{}\fi}\catcode`\%=\active\def%{\%}100}}%
\end{pgfscope}%
\begin{pgfscope}%
\pgfsetbuttcap%
\pgfsetroundjoin%
\definecolor{currentfill}{rgb}{0.000000,0.000000,0.000000}%
\pgfsetfillcolor{currentfill}%
\pgfsetlinewidth{0.803000pt}%
\definecolor{currentstroke}{rgb}{0.000000,0.000000,0.000000}%
\pgfsetstrokecolor{currentstroke}%
\pgfsetdash{}{0pt}%
\pgfsys@defobject{currentmarker}{\pgfqpoint{0.000000in}{-0.048611in}}{\pgfqpoint{0.000000in}{0.000000in}}{%
\pgfpathmoveto{\pgfqpoint{0.000000in}{0.000000in}}%
\pgfpathlineto{\pgfqpoint{0.000000in}{-0.048611in}}%
\pgfusepath{stroke,fill}%
}%
\begin{pgfscope}%
\pgfsys@transformshift{5.730469in}{0.550000in}%
\pgfsys@useobject{currentmarker}{}%
\end{pgfscope}%
\end{pgfscope}%
\begin{pgfscope}%
\definecolor{textcolor}{rgb}{0.000000,0.000000,0.000000}%
\pgfsetstrokecolor{textcolor}%
\pgfsetfillcolor{textcolor}%
\pgftext[x=5.730469in,y=0.452778in,,top]{\color{textcolor}{\sffamily\fontsize{10.000000}{12.000000}\selectfont\catcode`\^=\active\def^{\ifmmode\sp\else\^{}\fi}\catcode`\%=\active\def%{\%}150}}%
\end{pgfscope}%
\begin{pgfscope}%
\pgfsetbuttcap%
\pgfsetroundjoin%
\definecolor{currentfill}{rgb}{0.000000,0.000000,0.000000}%
\pgfsetfillcolor{currentfill}%
\pgfsetlinewidth{0.803000pt}%
\definecolor{currentstroke}{rgb}{0.000000,0.000000,0.000000}%
\pgfsetstrokecolor{currentstroke}%
\pgfsetdash{}{0pt}%
\pgfsys@defobject{currentmarker}{\pgfqpoint{0.000000in}{-0.048611in}}{\pgfqpoint{0.000000in}{0.000000in}}{%
\pgfpathmoveto{\pgfqpoint{0.000000in}{0.000000in}}%
\pgfpathlineto{\pgfqpoint{0.000000in}{-0.048611in}}%
\pgfusepath{stroke,fill}%
}%
\begin{pgfscope}%
\pgfsys@transformshift{7.106534in}{0.550000in}%
\pgfsys@useobject{currentmarker}{}%
\end{pgfscope}%
\end{pgfscope}%
\begin{pgfscope}%
\definecolor{textcolor}{rgb}{0.000000,0.000000,0.000000}%
\pgfsetstrokecolor{textcolor}%
\pgfsetfillcolor{textcolor}%
\pgftext[x=7.106534in,y=0.452778in,,top]{\color{textcolor}{\sffamily\fontsize{10.000000}{12.000000}\selectfont\catcode`\^=\active\def^{\ifmmode\sp\else\^{}\fi}\catcode`\%=\active\def%{\%}200}}%
\end{pgfscope}%
\begin{pgfscope}%
\pgfsetbuttcap%
\pgfsetroundjoin%
\definecolor{currentfill}{rgb}{0.000000,0.000000,0.000000}%
\pgfsetfillcolor{currentfill}%
\pgfsetlinewidth{0.803000pt}%
\definecolor{currentstroke}{rgb}{0.000000,0.000000,0.000000}%
\pgfsetstrokecolor{currentstroke}%
\pgfsetdash{}{0pt}%
\pgfsys@defobject{currentmarker}{\pgfqpoint{0.000000in}{-0.048611in}}{\pgfqpoint{0.000000in}{0.000000in}}{%
\pgfpathmoveto{\pgfqpoint{0.000000in}{0.000000in}}%
\pgfpathlineto{\pgfqpoint{0.000000in}{-0.048611in}}%
\pgfusepath{stroke,fill}%
}%
\begin{pgfscope}%
\pgfsys@transformshift{8.482599in}{0.550000in}%
\pgfsys@useobject{currentmarker}{}%
\end{pgfscope}%
\end{pgfscope}%
\begin{pgfscope}%
\definecolor{textcolor}{rgb}{0.000000,0.000000,0.000000}%
\pgfsetstrokecolor{textcolor}%
\pgfsetfillcolor{textcolor}%
\pgftext[x=8.482599in,y=0.452778in,,top]{\color{textcolor}{\sffamily\fontsize{10.000000}{12.000000}\selectfont\catcode`\^=\active\def^{\ifmmode\sp\else\^{}\fi}\catcode`\%=\active\def%{\%}250}}%
\end{pgfscope}%
\begin{pgfscope}%
\definecolor{textcolor}{rgb}{0.000000,0.000000,0.000000}%
\pgfsetstrokecolor{textcolor}%
\pgfsetfillcolor{textcolor}%
\pgftext[x=5.125000in,y=0.262809in,,top]{\color{textcolor}{\sffamily\fontsize{10.000000}{12.000000}\selectfont\catcode`\^=\active\def^{\ifmmode\sp\else\^{}\fi}\catcode`\%=\active\def%{\%}Intensidad}}%
\end{pgfscope}%
\begin{pgfscope}%
\pgfsetbuttcap%
\pgfsetroundjoin%
\definecolor{currentfill}{rgb}{0.000000,0.000000,0.000000}%
\pgfsetfillcolor{currentfill}%
\pgfsetlinewidth{0.803000pt}%
\definecolor{currentstroke}{rgb}{0.000000,0.000000,0.000000}%
\pgfsetstrokecolor{currentstroke}%
\pgfsetdash{}{0pt}%
\pgfsys@defobject{currentmarker}{\pgfqpoint{-0.048611in}{0.000000in}}{\pgfqpoint{-0.000000in}{0.000000in}}{%
\pgfpathmoveto{\pgfqpoint{-0.000000in}{0.000000in}}%
\pgfpathlineto{\pgfqpoint{-0.048611in}{0.000000in}}%
\pgfusepath{stroke,fill}%
}%
\begin{pgfscope}%
\pgfsys@transformshift{1.250000in}{0.550000in}%
\pgfsys@useobject{currentmarker}{}%
\end{pgfscope}%
\end{pgfscope}%
\begin{pgfscope}%
\definecolor{textcolor}{rgb}{0.000000,0.000000,0.000000}%
\pgfsetstrokecolor{textcolor}%
\pgfsetfillcolor{textcolor}%
\pgftext[x=1.064412in, y=0.497238in, left, base]{\color{textcolor}{\sffamily\fontsize{10.000000}{12.000000}\selectfont\catcode`\^=\active\def^{\ifmmode\sp\else\^{}\fi}\catcode`\%=\active\def%{\%}0}}%
\end{pgfscope}%
\begin{pgfscope}%
\pgfsetbuttcap%
\pgfsetroundjoin%
\definecolor{currentfill}{rgb}{0.000000,0.000000,0.000000}%
\pgfsetfillcolor{currentfill}%
\pgfsetlinewidth{0.803000pt}%
\definecolor{currentstroke}{rgb}{0.000000,0.000000,0.000000}%
\pgfsetstrokecolor{currentstroke}%
\pgfsetdash{}{0pt}%
\pgfsys@defobject{currentmarker}{\pgfqpoint{-0.048611in}{0.000000in}}{\pgfqpoint{-0.000000in}{0.000000in}}{%
\pgfpathmoveto{\pgfqpoint{-0.000000in}{0.000000in}}%
\pgfpathlineto{\pgfqpoint{-0.048611in}{0.000000in}}%
\pgfusepath{stroke,fill}%
}%
\begin{pgfscope}%
\pgfsys@transformshift{1.250000in}{1.275010in}%
\pgfsys@useobject{currentmarker}{}%
\end{pgfscope}%
\end{pgfscope}%
\begin{pgfscope}%
\definecolor{textcolor}{rgb}{0.000000,0.000000,0.000000}%
\pgfsetstrokecolor{textcolor}%
\pgfsetfillcolor{textcolor}%
\pgftext[x=0.799316in, y=1.222249in, left, base]{\color{textcolor}{\sffamily\fontsize{10.000000}{12.000000}\selectfont\catcode`\^=\active\def^{\ifmmode\sp\else\^{}\fi}\catcode`\%=\active\def%{\%}5000}}%
\end{pgfscope}%
\begin{pgfscope}%
\pgfsetbuttcap%
\pgfsetroundjoin%
\definecolor{currentfill}{rgb}{0.000000,0.000000,0.000000}%
\pgfsetfillcolor{currentfill}%
\pgfsetlinewidth{0.803000pt}%
\definecolor{currentstroke}{rgb}{0.000000,0.000000,0.000000}%
\pgfsetstrokecolor{currentstroke}%
\pgfsetdash{}{0pt}%
\pgfsys@defobject{currentmarker}{\pgfqpoint{-0.048611in}{0.000000in}}{\pgfqpoint{-0.000000in}{0.000000in}}{%
\pgfpathmoveto{\pgfqpoint{-0.000000in}{0.000000in}}%
\pgfpathlineto{\pgfqpoint{-0.048611in}{0.000000in}}%
\pgfusepath{stroke,fill}%
}%
\begin{pgfscope}%
\pgfsys@transformshift{1.250000in}{2.000020in}%
\pgfsys@useobject{currentmarker}{}%
\end{pgfscope}%
\end{pgfscope}%
\begin{pgfscope}%
\definecolor{textcolor}{rgb}{0.000000,0.000000,0.000000}%
\pgfsetstrokecolor{textcolor}%
\pgfsetfillcolor{textcolor}%
\pgftext[x=0.710951in, y=1.947259in, left, base]{\color{textcolor}{\sffamily\fontsize{10.000000}{12.000000}\selectfont\catcode`\^=\active\def^{\ifmmode\sp\else\^{}\fi}\catcode`\%=\active\def%{\%}10000}}%
\end{pgfscope}%
\begin{pgfscope}%
\pgfsetbuttcap%
\pgfsetroundjoin%
\definecolor{currentfill}{rgb}{0.000000,0.000000,0.000000}%
\pgfsetfillcolor{currentfill}%
\pgfsetlinewidth{0.803000pt}%
\definecolor{currentstroke}{rgb}{0.000000,0.000000,0.000000}%
\pgfsetstrokecolor{currentstroke}%
\pgfsetdash{}{0pt}%
\pgfsys@defobject{currentmarker}{\pgfqpoint{-0.048611in}{0.000000in}}{\pgfqpoint{-0.000000in}{0.000000in}}{%
\pgfpathmoveto{\pgfqpoint{-0.000000in}{0.000000in}}%
\pgfpathlineto{\pgfqpoint{-0.048611in}{0.000000in}}%
\pgfusepath{stroke,fill}%
}%
\begin{pgfscope}%
\pgfsys@transformshift{1.250000in}{2.725031in}%
\pgfsys@useobject{currentmarker}{}%
\end{pgfscope}%
\end{pgfscope}%
\begin{pgfscope}%
\definecolor{textcolor}{rgb}{0.000000,0.000000,0.000000}%
\pgfsetstrokecolor{textcolor}%
\pgfsetfillcolor{textcolor}%
\pgftext[x=0.710951in, y=2.672269in, left, base]{\color{textcolor}{\sffamily\fontsize{10.000000}{12.000000}\selectfont\catcode`\^=\active\def^{\ifmmode\sp\else\^{}\fi}\catcode`\%=\active\def%{\%}15000}}%
\end{pgfscope}%
\begin{pgfscope}%
\pgfsetbuttcap%
\pgfsetroundjoin%
\definecolor{currentfill}{rgb}{0.000000,0.000000,0.000000}%
\pgfsetfillcolor{currentfill}%
\pgfsetlinewidth{0.803000pt}%
\definecolor{currentstroke}{rgb}{0.000000,0.000000,0.000000}%
\pgfsetstrokecolor{currentstroke}%
\pgfsetdash{}{0pt}%
\pgfsys@defobject{currentmarker}{\pgfqpoint{-0.048611in}{0.000000in}}{\pgfqpoint{-0.000000in}{0.000000in}}{%
\pgfpathmoveto{\pgfqpoint{-0.000000in}{0.000000in}}%
\pgfpathlineto{\pgfqpoint{-0.048611in}{0.000000in}}%
\pgfusepath{stroke,fill}%
}%
\begin{pgfscope}%
\pgfsys@transformshift{1.250000in}{3.450041in}%
\pgfsys@useobject{currentmarker}{}%
\end{pgfscope}%
\end{pgfscope}%
\begin{pgfscope}%
\definecolor{textcolor}{rgb}{0.000000,0.000000,0.000000}%
\pgfsetstrokecolor{textcolor}%
\pgfsetfillcolor{textcolor}%
\pgftext[x=0.710951in, y=3.397279in, left, base]{\color{textcolor}{\sffamily\fontsize{10.000000}{12.000000}\selectfont\catcode`\^=\active\def^{\ifmmode\sp\else\^{}\fi}\catcode`\%=\active\def%{\%}20000}}%
\end{pgfscope}%
\begin{pgfscope}%
\pgfsetbuttcap%
\pgfsetroundjoin%
\definecolor{currentfill}{rgb}{0.000000,0.000000,0.000000}%
\pgfsetfillcolor{currentfill}%
\pgfsetlinewidth{0.803000pt}%
\definecolor{currentstroke}{rgb}{0.000000,0.000000,0.000000}%
\pgfsetstrokecolor{currentstroke}%
\pgfsetdash{}{0pt}%
\pgfsys@defobject{currentmarker}{\pgfqpoint{-0.048611in}{0.000000in}}{\pgfqpoint{-0.000000in}{0.000000in}}{%
\pgfpathmoveto{\pgfqpoint{-0.000000in}{0.000000in}}%
\pgfpathlineto{\pgfqpoint{-0.048611in}{0.000000in}}%
\pgfusepath{stroke,fill}%
}%
\begin{pgfscope}%
\pgfsys@transformshift{1.250000in}{4.175051in}%
\pgfsys@useobject{currentmarker}{}%
\end{pgfscope}%
\end{pgfscope}%
\begin{pgfscope}%
\definecolor{textcolor}{rgb}{0.000000,0.000000,0.000000}%
\pgfsetstrokecolor{textcolor}%
\pgfsetfillcolor{textcolor}%
\pgftext[x=0.710951in, y=4.122290in, left, base]{\color{textcolor}{\sffamily\fontsize{10.000000}{12.000000}\selectfont\catcode`\^=\active\def^{\ifmmode\sp\else\^{}\fi}\catcode`\%=\active\def%{\%}25000}}%
\end{pgfscope}%
\begin{pgfscope}%
\definecolor{textcolor}{rgb}{0.000000,0.000000,0.000000}%
\pgfsetstrokecolor{textcolor}%
\pgfsetfillcolor{textcolor}%
\pgftext[x=0.655396in,y=2.475000in,,bottom,rotate=90.000000]{\color{textcolor}{\sffamily\fontsize{10.000000}{12.000000}\selectfont\catcode`\^=\active\def^{\ifmmode\sp\else\^{}\fi}\catcode`\%=\active\def%{\%}Frecuencia}}%
\end{pgfscope}%
\begin{pgfscope}%
\pgfsetrectcap%
\pgfsetmiterjoin%
\pgfsetlinewidth{0.803000pt}%
\definecolor{currentstroke}{rgb}{0.000000,0.000000,0.000000}%
\pgfsetstrokecolor{currentstroke}%
\pgfsetdash{}{0pt}%
\pgfpathmoveto{\pgfqpoint{1.250000in}{0.550000in}}%
\pgfpathlineto{\pgfqpoint{1.250000in}{4.400000in}}%
\pgfusepath{stroke}%
\end{pgfscope}%
\begin{pgfscope}%
\pgfsetrectcap%
\pgfsetmiterjoin%
\pgfsetlinewidth{0.803000pt}%
\definecolor{currentstroke}{rgb}{0.000000,0.000000,0.000000}%
\pgfsetstrokecolor{currentstroke}%
\pgfsetdash{}{0pt}%
\pgfpathmoveto{\pgfqpoint{9.000000in}{0.550000in}}%
\pgfpathlineto{\pgfqpoint{9.000000in}{4.400000in}}%
\pgfusepath{stroke}%
\end{pgfscope}%
\begin{pgfscope}%
\pgfsetrectcap%
\pgfsetmiterjoin%
\pgfsetlinewidth{0.803000pt}%
\definecolor{currentstroke}{rgb}{0.000000,0.000000,0.000000}%
\pgfsetstrokecolor{currentstroke}%
\pgfsetdash{}{0pt}%
\pgfpathmoveto{\pgfqpoint{1.250000in}{0.550000in}}%
\pgfpathlineto{\pgfqpoint{9.000000in}{0.550000in}}%
\pgfusepath{stroke}%
\end{pgfscope}%
\begin{pgfscope}%
\pgfsetrectcap%
\pgfsetmiterjoin%
\pgfsetlinewidth{0.803000pt}%
\definecolor{currentstroke}{rgb}{0.000000,0.000000,0.000000}%
\pgfsetstrokecolor{currentstroke}%
\pgfsetdash{}{0pt}%
\pgfpathmoveto{\pgfqpoint{1.250000in}{4.400000in}}%
\pgfpathlineto{\pgfqpoint{9.000000in}{4.400000in}}%
\pgfusepath{stroke}%
\end{pgfscope}%
\begin{pgfscope}%
\definecolor{textcolor}{rgb}{0.000000,0.000000,0.000000}%
\pgfsetstrokecolor{textcolor}%
\pgfsetfillcolor{textcolor}%
\pgftext[x=5.125000in,y=4.483333in,,base]{\color{textcolor}{\sffamily\fontsize{12.000000}{14.400000}\selectfont\catcode`\^=\active\def^{\ifmmode\sp\else\^{}\fi}\catcode`\%=\active\def%{\%}Histograma de HE}}%
\end{pgfscope}%
\end{pgfpicture}%
\makeatother%
\endgroup%
}
		\caption{HE}
		\label{fig:he_3}
	\end{subfigure}
	\hfill
	\begin{subfigure}[b]{0.45\textwidth}
		\resizebox{\linewidth}{!}{%% Creator: Matplotlib, PGF backend
%%
%% To include the figure in your LaTeX document, write
%%   \input{<filename>.pgf}
%%
%% Make sure the required packages are loaded in your preamble
%%   \usepackage{pgf}
%%
%% Also ensure that all the required font packages are loaded; for instance,
%% the lmodern package is sometimes necessary when using math font.
%%   \usepackage{lmodern}
%%
%% Figures using additional raster images can only be included by \input if
%% they are in the same directory as the main LaTeX file. For loading figures
%% from other directories you can use the `import` package
%%   \usepackage{import}
%%
%% and then include the figures with
%%   \import{<path to file>}{<filename>.pgf}
%%
%% Matplotlib used the following preamble
%%   \def\mathdefault#1{#1}
%%   \everymath=\expandafter{\the\everymath\displaystyle}
%%   \IfFileExists{scrextend.sty}{
%%     \usepackage[fontsize=10.000000pt]{scrextend}
%%   }{
%%     \renewcommand{\normalsize}{\fontsize{10.000000}{12.000000}\selectfont}
%%     \normalsize
%%   }
%%   
%%   \ifdefined\pdftexversion\else  % non-pdftex case.
%%     \usepackage{fontspec}
%%     \setmainfont{DejaVuSerif.ttf}[Path=\detokenize{/home/elias/.virtualenvs/src-pway/lib/python3.13/site-packages/matplotlib/mpl-data/fonts/ttf/}]
%%     \setsansfont{DejaVuSans.ttf}[Path=\detokenize{/home/elias/.virtualenvs/src-pway/lib/python3.13/site-packages/matplotlib/mpl-data/fonts/ttf/}]
%%     \setmonofont{DejaVuSansMono.ttf}[Path=\detokenize{/home/elias/.virtualenvs/src-pway/lib/python3.13/site-packages/matplotlib/mpl-data/fonts/ttf/}]
%%   \fi
%%   \makeatletter\@ifpackageloaded{underscore}{}{\usepackage[strings]{underscore}}\makeatother
%%
\begingroup%
\makeatletter%
\begin{pgfpicture}%
\pgfpathrectangle{\pgfpointorigin}{\pgfqpoint{10.000000in}{5.000000in}}%
\pgfusepath{use as bounding box, clip}%
\begin{pgfscope}%
\pgfsetbuttcap%
\pgfsetmiterjoin%
\definecolor{currentfill}{rgb}{1.000000,1.000000,1.000000}%
\pgfsetfillcolor{currentfill}%
\pgfsetlinewidth{0.000000pt}%
\definecolor{currentstroke}{rgb}{1.000000,1.000000,1.000000}%
\pgfsetstrokecolor{currentstroke}%
\pgfsetdash{}{0pt}%
\pgfpathmoveto{\pgfqpoint{0.000000in}{0.000000in}}%
\pgfpathlineto{\pgfqpoint{10.000000in}{0.000000in}}%
\pgfpathlineto{\pgfqpoint{10.000000in}{5.000000in}}%
\pgfpathlineto{\pgfqpoint{0.000000in}{5.000000in}}%
\pgfpathlineto{\pgfqpoint{0.000000in}{0.000000in}}%
\pgfpathclose%
\pgfusepath{fill}%
\end{pgfscope}%
\begin{pgfscope}%
\pgfsetbuttcap%
\pgfsetmiterjoin%
\definecolor{currentfill}{rgb}{1.000000,1.000000,1.000000}%
\pgfsetfillcolor{currentfill}%
\pgfsetlinewidth{0.000000pt}%
\definecolor{currentstroke}{rgb}{0.000000,0.000000,0.000000}%
\pgfsetstrokecolor{currentstroke}%
\pgfsetstrokeopacity{0.000000}%
\pgfsetdash{}{0pt}%
\pgfpathmoveto{\pgfqpoint{1.250000in}{0.550000in}}%
\pgfpathlineto{\pgfqpoint{9.000000in}{0.550000in}}%
\pgfpathlineto{\pgfqpoint{9.000000in}{4.400000in}}%
\pgfpathlineto{\pgfqpoint{1.250000in}{4.400000in}}%
\pgfpathlineto{\pgfqpoint{1.250000in}{0.550000in}}%
\pgfpathclose%
\pgfusepath{fill}%
\end{pgfscope}%
\begin{pgfscope}%
\pgfpathrectangle{\pgfqpoint{1.250000in}{0.550000in}}{\pgfqpoint{7.750000in}{3.850000in}}%
\pgfusepath{clip}%
\pgfsetbuttcap%
\pgfsetmiterjoin%
\definecolor{currentfill}{rgb}{0.501961,0.501961,0.501961}%
\pgfsetfillcolor{currentfill}%
\pgfsetfillopacity{0.700000}%
\pgfsetlinewidth{0.000000pt}%
\definecolor{currentstroke}{rgb}{0.000000,0.000000,0.000000}%
\pgfsetstrokecolor{currentstroke}%
\pgfsetstrokeopacity{0.700000}%
\pgfsetdash{}{0pt}%
\pgfpathmoveto{\pgfqpoint{1.602273in}{0.550000in}}%
\pgfpathlineto{\pgfqpoint{1.629794in}{0.550000in}}%
\pgfpathlineto{\pgfqpoint{1.629794in}{0.550000in}}%
\pgfpathlineto{\pgfqpoint{1.602273in}{0.550000in}}%
\pgfpathlineto{\pgfqpoint{1.602273in}{0.550000in}}%
\pgfpathclose%
\pgfusepath{fill}%
\end{pgfscope}%
\begin{pgfscope}%
\pgfpathrectangle{\pgfqpoint{1.250000in}{0.550000in}}{\pgfqpoint{7.750000in}{3.850000in}}%
\pgfusepath{clip}%
\pgfsetbuttcap%
\pgfsetmiterjoin%
\definecolor{currentfill}{rgb}{0.501961,0.501961,0.501961}%
\pgfsetfillcolor{currentfill}%
\pgfsetfillopacity{0.700000}%
\pgfsetlinewidth{0.000000pt}%
\definecolor{currentstroke}{rgb}{0.000000,0.000000,0.000000}%
\pgfsetstrokecolor{currentstroke}%
\pgfsetstrokeopacity{0.700000}%
\pgfsetdash{}{0pt}%
\pgfpathmoveto{\pgfqpoint{1.629794in}{0.550000in}}%
\pgfpathlineto{\pgfqpoint{1.657315in}{0.550000in}}%
\pgfpathlineto{\pgfqpoint{1.657315in}{0.555206in}}%
\pgfpathlineto{\pgfqpoint{1.629794in}{0.555206in}}%
\pgfpathlineto{\pgfqpoint{1.629794in}{0.550000in}}%
\pgfpathclose%
\pgfusepath{fill}%
\end{pgfscope}%
\begin{pgfscope}%
\pgfpathrectangle{\pgfqpoint{1.250000in}{0.550000in}}{\pgfqpoint{7.750000in}{3.850000in}}%
\pgfusepath{clip}%
\pgfsetbuttcap%
\pgfsetmiterjoin%
\definecolor{currentfill}{rgb}{0.501961,0.501961,0.501961}%
\pgfsetfillcolor{currentfill}%
\pgfsetfillopacity{0.700000}%
\pgfsetlinewidth{0.000000pt}%
\definecolor{currentstroke}{rgb}{0.000000,0.000000,0.000000}%
\pgfsetstrokecolor{currentstroke}%
\pgfsetstrokeopacity{0.700000}%
\pgfsetdash{}{0pt}%
\pgfpathmoveto{\pgfqpoint{1.657315in}{0.550000in}}%
\pgfpathlineto{\pgfqpoint{1.684837in}{0.550000in}}%
\pgfpathlineto{\pgfqpoint{1.684837in}{0.634516in}}%
\pgfpathlineto{\pgfqpoint{1.657315in}{0.634516in}}%
\pgfpathlineto{\pgfqpoint{1.657315in}{0.550000in}}%
\pgfpathclose%
\pgfusepath{fill}%
\end{pgfscope}%
\begin{pgfscope}%
\pgfpathrectangle{\pgfqpoint{1.250000in}{0.550000in}}{\pgfqpoint{7.750000in}{3.850000in}}%
\pgfusepath{clip}%
\pgfsetbuttcap%
\pgfsetmiterjoin%
\definecolor{currentfill}{rgb}{0.501961,0.501961,0.501961}%
\pgfsetfillcolor{currentfill}%
\pgfsetfillopacity{0.700000}%
\pgfsetlinewidth{0.000000pt}%
\definecolor{currentstroke}{rgb}{0.000000,0.000000,0.000000}%
\pgfsetstrokecolor{currentstroke}%
\pgfsetstrokeopacity{0.700000}%
\pgfsetdash{}{0pt}%
\pgfpathmoveto{\pgfqpoint{1.684837in}{0.550000in}}%
\pgfpathlineto{\pgfqpoint{1.712358in}{0.550000in}}%
\pgfpathlineto{\pgfqpoint{1.712358in}{4.216667in}}%
\pgfpathlineto{\pgfqpoint{1.684837in}{4.216667in}}%
\pgfpathlineto{\pgfqpoint{1.684837in}{0.550000in}}%
\pgfpathclose%
\pgfusepath{fill}%
\end{pgfscope}%
\begin{pgfscope}%
\pgfpathrectangle{\pgfqpoint{1.250000in}{0.550000in}}{\pgfqpoint{7.750000in}{3.850000in}}%
\pgfusepath{clip}%
\pgfsetbuttcap%
\pgfsetmiterjoin%
\definecolor{currentfill}{rgb}{0.501961,0.501961,0.501961}%
\pgfsetfillcolor{currentfill}%
\pgfsetfillopacity{0.700000}%
\pgfsetlinewidth{0.000000pt}%
\definecolor{currentstroke}{rgb}{0.000000,0.000000,0.000000}%
\pgfsetstrokecolor{currentstroke}%
\pgfsetstrokeopacity{0.700000}%
\pgfsetdash{}{0pt}%
\pgfpathmoveto{\pgfqpoint{1.712358in}{0.550000in}}%
\pgfpathlineto{\pgfqpoint{1.739879in}{0.550000in}}%
\pgfpathlineto{\pgfqpoint{1.739879in}{0.597158in}}%
\pgfpathlineto{\pgfqpoint{1.712358in}{0.597158in}}%
\pgfpathlineto{\pgfqpoint{1.712358in}{0.550000in}}%
\pgfpathclose%
\pgfusepath{fill}%
\end{pgfscope}%
\begin{pgfscope}%
\pgfpathrectangle{\pgfqpoint{1.250000in}{0.550000in}}{\pgfqpoint{7.750000in}{3.850000in}}%
\pgfusepath{clip}%
\pgfsetbuttcap%
\pgfsetmiterjoin%
\definecolor{currentfill}{rgb}{0.501961,0.501961,0.501961}%
\pgfsetfillcolor{currentfill}%
\pgfsetfillopacity{0.700000}%
\pgfsetlinewidth{0.000000pt}%
\definecolor{currentstroke}{rgb}{0.000000,0.000000,0.000000}%
\pgfsetstrokecolor{currentstroke}%
\pgfsetstrokeopacity{0.700000}%
\pgfsetdash{}{0pt}%
\pgfpathmoveto{\pgfqpoint{1.739879in}{0.550000in}}%
\pgfpathlineto{\pgfqpoint{1.767401in}{0.550000in}}%
\pgfpathlineto{\pgfqpoint{1.767401in}{1.176524in}}%
\pgfpathlineto{\pgfqpoint{1.739879in}{1.176524in}}%
\pgfpathlineto{\pgfqpoint{1.739879in}{0.550000in}}%
\pgfpathclose%
\pgfusepath{fill}%
\end{pgfscope}%
\begin{pgfscope}%
\pgfpathrectangle{\pgfqpoint{1.250000in}{0.550000in}}{\pgfqpoint{7.750000in}{3.850000in}}%
\pgfusepath{clip}%
\pgfsetbuttcap%
\pgfsetmiterjoin%
\definecolor{currentfill}{rgb}{0.501961,0.501961,0.501961}%
\pgfsetfillcolor{currentfill}%
\pgfsetfillopacity{0.700000}%
\pgfsetlinewidth{0.000000pt}%
\definecolor{currentstroke}{rgb}{0.000000,0.000000,0.000000}%
\pgfsetstrokecolor{currentstroke}%
\pgfsetstrokeopacity{0.700000}%
\pgfsetdash{}{0pt}%
\pgfpathmoveto{\pgfqpoint{1.767401in}{0.550000in}}%
\pgfpathlineto{\pgfqpoint{1.794922in}{0.550000in}}%
\pgfpathlineto{\pgfqpoint{1.794922in}{2.075430in}}%
\pgfpathlineto{\pgfqpoint{1.767401in}{2.075430in}}%
\pgfpathlineto{\pgfqpoint{1.767401in}{0.550000in}}%
\pgfpathclose%
\pgfusepath{fill}%
\end{pgfscope}%
\begin{pgfscope}%
\pgfpathrectangle{\pgfqpoint{1.250000in}{0.550000in}}{\pgfqpoint{7.750000in}{3.850000in}}%
\pgfusepath{clip}%
\pgfsetbuttcap%
\pgfsetmiterjoin%
\definecolor{currentfill}{rgb}{0.501961,0.501961,0.501961}%
\pgfsetfillcolor{currentfill}%
\pgfsetfillopacity{0.700000}%
\pgfsetlinewidth{0.000000pt}%
\definecolor{currentstroke}{rgb}{0.000000,0.000000,0.000000}%
\pgfsetstrokecolor{currentstroke}%
\pgfsetstrokeopacity{0.700000}%
\pgfsetdash{}{0pt}%
\pgfpathmoveto{\pgfqpoint{1.794922in}{0.550000in}}%
\pgfpathlineto{\pgfqpoint{1.822443in}{0.550000in}}%
\pgfpathlineto{\pgfqpoint{1.822443in}{0.638957in}}%
\pgfpathlineto{\pgfqpoint{1.794922in}{0.638957in}}%
\pgfpathlineto{\pgfqpoint{1.794922in}{0.550000in}}%
\pgfpathclose%
\pgfusepath{fill}%
\end{pgfscope}%
\begin{pgfscope}%
\pgfpathrectangle{\pgfqpoint{1.250000in}{0.550000in}}{\pgfqpoint{7.750000in}{3.850000in}}%
\pgfusepath{clip}%
\pgfsetbuttcap%
\pgfsetmiterjoin%
\definecolor{currentfill}{rgb}{0.501961,0.501961,0.501961}%
\pgfsetfillcolor{currentfill}%
\pgfsetfillopacity{0.700000}%
\pgfsetlinewidth{0.000000pt}%
\definecolor{currentstroke}{rgb}{0.000000,0.000000,0.000000}%
\pgfsetstrokecolor{currentstroke}%
\pgfsetstrokeopacity{0.700000}%
\pgfsetdash{}{0pt}%
\pgfpathmoveto{\pgfqpoint{1.822443in}{0.550000in}}%
\pgfpathlineto{\pgfqpoint{1.849964in}{0.550000in}}%
\pgfpathlineto{\pgfqpoint{1.849964in}{1.248026in}}%
\pgfpathlineto{\pgfqpoint{1.822443in}{1.248026in}}%
\pgfpathlineto{\pgfqpoint{1.822443in}{0.550000in}}%
\pgfpathclose%
\pgfusepath{fill}%
\end{pgfscope}%
\begin{pgfscope}%
\pgfpathrectangle{\pgfqpoint{1.250000in}{0.550000in}}{\pgfqpoint{7.750000in}{3.850000in}}%
\pgfusepath{clip}%
\pgfsetbuttcap%
\pgfsetmiterjoin%
\definecolor{currentfill}{rgb}{0.501961,0.501961,0.501961}%
\pgfsetfillcolor{currentfill}%
\pgfsetfillopacity{0.700000}%
\pgfsetlinewidth{0.000000pt}%
\definecolor{currentstroke}{rgb}{0.000000,0.000000,0.000000}%
\pgfsetstrokecolor{currentstroke}%
\pgfsetstrokeopacity{0.700000}%
\pgfsetdash{}{0pt}%
\pgfpathmoveto{\pgfqpoint{1.849964in}{0.550000in}}%
\pgfpathlineto{\pgfqpoint{1.877486in}{0.550000in}}%
\pgfpathlineto{\pgfqpoint{1.877486in}{1.045462in}}%
\pgfpathlineto{\pgfqpoint{1.849964in}{1.045462in}}%
\pgfpathlineto{\pgfqpoint{1.849964in}{0.550000in}}%
\pgfpathclose%
\pgfusepath{fill}%
\end{pgfscope}%
\begin{pgfscope}%
\pgfpathrectangle{\pgfqpoint{1.250000in}{0.550000in}}{\pgfqpoint{7.750000in}{3.850000in}}%
\pgfusepath{clip}%
\pgfsetbuttcap%
\pgfsetmiterjoin%
\definecolor{currentfill}{rgb}{0.501961,0.501961,0.501961}%
\pgfsetfillcolor{currentfill}%
\pgfsetfillopacity{0.700000}%
\pgfsetlinewidth{0.000000pt}%
\definecolor{currentstroke}{rgb}{0.000000,0.000000,0.000000}%
\pgfsetstrokecolor{currentstroke}%
\pgfsetstrokeopacity{0.700000}%
\pgfsetdash{}{0pt}%
\pgfpathmoveto{\pgfqpoint{1.877486in}{0.550000in}}%
\pgfpathlineto{\pgfqpoint{1.905007in}{0.550000in}}%
\pgfpathlineto{\pgfqpoint{1.905007in}{0.679684in}}%
\pgfpathlineto{\pgfqpoint{1.877486in}{0.679684in}}%
\pgfpathlineto{\pgfqpoint{1.877486in}{0.550000in}}%
\pgfpathclose%
\pgfusepath{fill}%
\end{pgfscope}%
\begin{pgfscope}%
\pgfpathrectangle{\pgfqpoint{1.250000in}{0.550000in}}{\pgfqpoint{7.750000in}{3.850000in}}%
\pgfusepath{clip}%
\pgfsetbuttcap%
\pgfsetmiterjoin%
\definecolor{currentfill}{rgb}{0.501961,0.501961,0.501961}%
\pgfsetfillcolor{currentfill}%
\pgfsetfillopacity{0.700000}%
\pgfsetlinewidth{0.000000pt}%
\definecolor{currentstroke}{rgb}{0.000000,0.000000,0.000000}%
\pgfsetstrokecolor{currentstroke}%
\pgfsetstrokeopacity{0.700000}%
\pgfsetdash{}{0pt}%
\pgfpathmoveto{\pgfqpoint{1.905007in}{0.550000in}}%
\pgfpathlineto{\pgfqpoint{1.932528in}{0.550000in}}%
\pgfpathlineto{\pgfqpoint{1.932528in}{1.116046in}}%
\pgfpathlineto{\pgfqpoint{1.905007in}{1.116046in}}%
\pgfpathlineto{\pgfqpoint{1.905007in}{0.550000in}}%
\pgfpathclose%
\pgfusepath{fill}%
\end{pgfscope}%
\begin{pgfscope}%
\pgfpathrectangle{\pgfqpoint{1.250000in}{0.550000in}}{\pgfqpoint{7.750000in}{3.850000in}}%
\pgfusepath{clip}%
\pgfsetbuttcap%
\pgfsetmiterjoin%
\definecolor{currentfill}{rgb}{0.501961,0.501961,0.501961}%
\pgfsetfillcolor{currentfill}%
\pgfsetfillopacity{0.700000}%
\pgfsetlinewidth{0.000000pt}%
\definecolor{currentstroke}{rgb}{0.000000,0.000000,0.000000}%
\pgfsetstrokecolor{currentstroke}%
\pgfsetstrokeopacity{0.700000}%
\pgfsetdash{}{0pt}%
\pgfpathmoveto{\pgfqpoint{1.932528in}{0.550000in}}%
\pgfpathlineto{\pgfqpoint{1.960050in}{0.550000in}}%
\pgfpathlineto{\pgfqpoint{1.960050in}{0.752870in}}%
\pgfpathlineto{\pgfqpoint{1.932528in}{0.752870in}}%
\pgfpathlineto{\pgfqpoint{1.932528in}{0.550000in}}%
\pgfpathclose%
\pgfusepath{fill}%
\end{pgfscope}%
\begin{pgfscope}%
\pgfpathrectangle{\pgfqpoint{1.250000in}{0.550000in}}{\pgfqpoint{7.750000in}{3.850000in}}%
\pgfusepath{clip}%
\pgfsetbuttcap%
\pgfsetmiterjoin%
\definecolor{currentfill}{rgb}{0.501961,0.501961,0.501961}%
\pgfsetfillcolor{currentfill}%
\pgfsetfillopacity{0.700000}%
\pgfsetlinewidth{0.000000pt}%
\definecolor{currentstroke}{rgb}{0.000000,0.000000,0.000000}%
\pgfsetstrokecolor{currentstroke}%
\pgfsetstrokeopacity{0.700000}%
\pgfsetdash{}{0pt}%
\pgfpathmoveto{\pgfqpoint{1.960050in}{0.550000in}}%
\pgfpathlineto{\pgfqpoint{1.987571in}{0.550000in}}%
\pgfpathlineto{\pgfqpoint{1.987571in}{0.732353in}}%
\pgfpathlineto{\pgfqpoint{1.960050in}{0.732353in}}%
\pgfpathlineto{\pgfqpoint{1.960050in}{0.550000in}}%
\pgfpathclose%
\pgfusepath{fill}%
\end{pgfscope}%
\begin{pgfscope}%
\pgfpathrectangle{\pgfqpoint{1.250000in}{0.550000in}}{\pgfqpoint{7.750000in}{3.850000in}}%
\pgfusepath{clip}%
\pgfsetbuttcap%
\pgfsetmiterjoin%
\definecolor{currentfill}{rgb}{0.501961,0.501961,0.501961}%
\pgfsetfillcolor{currentfill}%
\pgfsetfillopacity{0.700000}%
\pgfsetlinewidth{0.000000pt}%
\definecolor{currentstroke}{rgb}{0.000000,0.000000,0.000000}%
\pgfsetstrokecolor{currentstroke}%
\pgfsetstrokeopacity{0.700000}%
\pgfsetdash{}{0pt}%
\pgfpathmoveto{\pgfqpoint{1.987571in}{0.550000in}}%
\pgfpathlineto{\pgfqpoint{2.015092in}{0.550000in}}%
\pgfpathlineto{\pgfqpoint{2.015092in}{1.029998in}}%
\pgfpathlineto{\pgfqpoint{1.987571in}{1.029998in}}%
\pgfpathlineto{\pgfqpoint{1.987571in}{0.550000in}}%
\pgfpathclose%
\pgfusepath{fill}%
\end{pgfscope}%
\begin{pgfscope}%
\pgfpathrectangle{\pgfqpoint{1.250000in}{0.550000in}}{\pgfqpoint{7.750000in}{3.850000in}}%
\pgfusepath{clip}%
\pgfsetbuttcap%
\pgfsetmiterjoin%
\definecolor{currentfill}{rgb}{0.501961,0.501961,0.501961}%
\pgfsetfillcolor{currentfill}%
\pgfsetfillopacity{0.700000}%
\pgfsetlinewidth{0.000000pt}%
\definecolor{currentstroke}{rgb}{0.000000,0.000000,0.000000}%
\pgfsetstrokecolor{currentstroke}%
\pgfsetstrokeopacity{0.700000}%
\pgfsetdash{}{0pt}%
\pgfpathmoveto{\pgfqpoint{2.015092in}{0.550000in}}%
\pgfpathlineto{\pgfqpoint{2.042614in}{0.550000in}}%
\pgfpathlineto{\pgfqpoint{2.042614in}{0.722095in}}%
\pgfpathlineto{\pgfqpoint{2.015092in}{0.722095in}}%
\pgfpathlineto{\pgfqpoint{2.015092in}{0.550000in}}%
\pgfpathclose%
\pgfusepath{fill}%
\end{pgfscope}%
\begin{pgfscope}%
\pgfpathrectangle{\pgfqpoint{1.250000in}{0.550000in}}{\pgfqpoint{7.750000in}{3.850000in}}%
\pgfusepath{clip}%
\pgfsetbuttcap%
\pgfsetmiterjoin%
\definecolor{currentfill}{rgb}{0.501961,0.501961,0.501961}%
\pgfsetfillcolor{currentfill}%
\pgfsetfillopacity{0.700000}%
\pgfsetlinewidth{0.000000pt}%
\definecolor{currentstroke}{rgb}{0.000000,0.000000,0.000000}%
\pgfsetstrokecolor{currentstroke}%
\pgfsetstrokeopacity{0.700000}%
\pgfsetdash{}{0pt}%
\pgfpathmoveto{\pgfqpoint{2.042614in}{0.550000in}}%
\pgfpathlineto{\pgfqpoint{2.070135in}{0.550000in}}%
\pgfpathlineto{\pgfqpoint{2.070135in}{0.843358in}}%
\pgfpathlineto{\pgfqpoint{2.042614in}{0.843358in}}%
\pgfpathlineto{\pgfqpoint{2.042614in}{0.550000in}}%
\pgfpathclose%
\pgfusepath{fill}%
\end{pgfscope}%
\begin{pgfscope}%
\pgfpathrectangle{\pgfqpoint{1.250000in}{0.550000in}}{\pgfqpoint{7.750000in}{3.850000in}}%
\pgfusepath{clip}%
\pgfsetbuttcap%
\pgfsetmiterjoin%
\definecolor{currentfill}{rgb}{0.501961,0.501961,0.501961}%
\pgfsetfillcolor{currentfill}%
\pgfsetfillopacity{0.700000}%
\pgfsetlinewidth{0.000000pt}%
\definecolor{currentstroke}{rgb}{0.000000,0.000000,0.000000}%
\pgfsetstrokecolor{currentstroke}%
\pgfsetstrokeopacity{0.700000}%
\pgfsetdash{}{0pt}%
\pgfpathmoveto{\pgfqpoint{2.070135in}{0.550000in}}%
\pgfpathlineto{\pgfqpoint{2.097656in}{0.550000in}}%
\pgfpathlineto{\pgfqpoint{2.097656in}{0.925884in}}%
\pgfpathlineto{\pgfqpoint{2.070135in}{0.925884in}}%
\pgfpathlineto{\pgfqpoint{2.070135in}{0.550000in}}%
\pgfpathclose%
\pgfusepath{fill}%
\end{pgfscope}%
\begin{pgfscope}%
\pgfpathrectangle{\pgfqpoint{1.250000in}{0.550000in}}{\pgfqpoint{7.750000in}{3.850000in}}%
\pgfusepath{clip}%
\pgfsetbuttcap%
\pgfsetmiterjoin%
\definecolor{currentfill}{rgb}{0.501961,0.501961,0.501961}%
\pgfsetfillcolor{currentfill}%
\pgfsetfillopacity{0.700000}%
\pgfsetlinewidth{0.000000pt}%
\definecolor{currentstroke}{rgb}{0.000000,0.000000,0.000000}%
\pgfsetstrokecolor{currentstroke}%
\pgfsetstrokeopacity{0.700000}%
\pgfsetdash{}{0pt}%
\pgfpathmoveto{\pgfqpoint{2.097656in}{0.550000in}}%
\pgfpathlineto{\pgfqpoint{2.125178in}{0.550000in}}%
\pgfpathlineto{\pgfqpoint{2.125178in}{0.796659in}}%
\pgfpathlineto{\pgfqpoint{2.097656in}{0.796659in}}%
\pgfpathlineto{\pgfqpoint{2.097656in}{0.550000in}}%
\pgfpathclose%
\pgfusepath{fill}%
\end{pgfscope}%
\begin{pgfscope}%
\pgfpathrectangle{\pgfqpoint{1.250000in}{0.550000in}}{\pgfqpoint{7.750000in}{3.850000in}}%
\pgfusepath{clip}%
\pgfsetbuttcap%
\pgfsetmiterjoin%
\definecolor{currentfill}{rgb}{0.501961,0.501961,0.501961}%
\pgfsetfillcolor{currentfill}%
\pgfsetfillopacity{0.700000}%
\pgfsetlinewidth{0.000000pt}%
\definecolor{currentstroke}{rgb}{0.000000,0.000000,0.000000}%
\pgfsetstrokecolor{currentstroke}%
\pgfsetstrokeopacity{0.700000}%
\pgfsetdash{}{0pt}%
\pgfpathmoveto{\pgfqpoint{2.125178in}{0.550000in}}%
\pgfpathlineto{\pgfqpoint{2.152699in}{0.550000in}}%
\pgfpathlineto{\pgfqpoint{2.152699in}{1.010400in}}%
\pgfpathlineto{\pgfqpoint{2.125178in}{1.010400in}}%
\pgfpathlineto{\pgfqpoint{2.125178in}{0.550000in}}%
\pgfpathclose%
\pgfusepath{fill}%
\end{pgfscope}%
\begin{pgfscope}%
\pgfpathrectangle{\pgfqpoint{1.250000in}{0.550000in}}{\pgfqpoint{7.750000in}{3.850000in}}%
\pgfusepath{clip}%
\pgfsetbuttcap%
\pgfsetmiterjoin%
\definecolor{currentfill}{rgb}{0.501961,0.501961,0.501961}%
\pgfsetfillcolor{currentfill}%
\pgfsetfillopacity{0.700000}%
\pgfsetlinewidth{0.000000pt}%
\definecolor{currentstroke}{rgb}{0.000000,0.000000,0.000000}%
\pgfsetstrokecolor{currentstroke}%
\pgfsetstrokeopacity{0.700000}%
\pgfsetdash{}{0pt}%
\pgfpathmoveto{\pgfqpoint{2.152699in}{0.550000in}}%
\pgfpathlineto{\pgfqpoint{2.180220in}{0.550000in}}%
\pgfpathlineto{\pgfqpoint{2.180220in}{0.932927in}}%
\pgfpathlineto{\pgfqpoint{2.152699in}{0.932927in}}%
\pgfpathlineto{\pgfqpoint{2.152699in}{0.550000in}}%
\pgfpathclose%
\pgfusepath{fill}%
\end{pgfscope}%
\begin{pgfscope}%
\pgfpathrectangle{\pgfqpoint{1.250000in}{0.550000in}}{\pgfqpoint{7.750000in}{3.850000in}}%
\pgfusepath{clip}%
\pgfsetbuttcap%
\pgfsetmiterjoin%
\definecolor{currentfill}{rgb}{0.501961,0.501961,0.501961}%
\pgfsetfillcolor{currentfill}%
\pgfsetfillopacity{0.700000}%
\pgfsetlinewidth{0.000000pt}%
\definecolor{currentstroke}{rgb}{0.000000,0.000000,0.000000}%
\pgfsetstrokecolor{currentstroke}%
\pgfsetstrokeopacity{0.700000}%
\pgfsetdash{}{0pt}%
\pgfpathmoveto{\pgfqpoint{2.180220in}{0.550000in}}%
\pgfpathlineto{\pgfqpoint{2.207741in}{0.550000in}}%
\pgfpathlineto{\pgfqpoint{2.207741in}{0.906592in}}%
\pgfpathlineto{\pgfqpoint{2.180220in}{0.906592in}}%
\pgfpathlineto{\pgfqpoint{2.180220in}{0.550000in}}%
\pgfpathclose%
\pgfusepath{fill}%
\end{pgfscope}%
\begin{pgfscope}%
\pgfpathrectangle{\pgfqpoint{1.250000in}{0.550000in}}{\pgfqpoint{7.750000in}{3.850000in}}%
\pgfusepath{clip}%
\pgfsetbuttcap%
\pgfsetmiterjoin%
\definecolor{currentfill}{rgb}{0.501961,0.501961,0.501961}%
\pgfsetfillcolor{currentfill}%
\pgfsetfillopacity{0.700000}%
\pgfsetlinewidth{0.000000pt}%
\definecolor{currentstroke}{rgb}{0.000000,0.000000,0.000000}%
\pgfsetstrokecolor{currentstroke}%
\pgfsetstrokeopacity{0.700000}%
\pgfsetdash{}{0pt}%
\pgfpathmoveto{\pgfqpoint{2.207741in}{0.550000in}}%
\pgfpathlineto{\pgfqpoint{2.235263in}{0.550000in}}%
\pgfpathlineto{\pgfqpoint{2.235263in}{1.190304in}}%
\pgfpathlineto{\pgfqpoint{2.207741in}{1.190304in}}%
\pgfpathlineto{\pgfqpoint{2.207741in}{0.550000in}}%
\pgfpathclose%
\pgfusepath{fill}%
\end{pgfscope}%
\begin{pgfscope}%
\pgfpathrectangle{\pgfqpoint{1.250000in}{0.550000in}}{\pgfqpoint{7.750000in}{3.850000in}}%
\pgfusepath{clip}%
\pgfsetbuttcap%
\pgfsetmiterjoin%
\definecolor{currentfill}{rgb}{0.501961,0.501961,0.501961}%
\pgfsetfillcolor{currentfill}%
\pgfsetfillopacity{0.700000}%
\pgfsetlinewidth{0.000000pt}%
\definecolor{currentstroke}{rgb}{0.000000,0.000000,0.000000}%
\pgfsetstrokecolor{currentstroke}%
\pgfsetstrokeopacity{0.700000}%
\pgfsetdash{}{0pt}%
\pgfpathmoveto{\pgfqpoint{2.235263in}{0.550000in}}%
\pgfpathlineto{\pgfqpoint{2.262784in}{0.550000in}}%
\pgfpathlineto{\pgfqpoint{2.262784in}{1.226285in}}%
\pgfpathlineto{\pgfqpoint{2.235263in}{1.226285in}}%
\pgfpathlineto{\pgfqpoint{2.235263in}{0.550000in}}%
\pgfpathclose%
\pgfusepath{fill}%
\end{pgfscope}%
\begin{pgfscope}%
\pgfpathrectangle{\pgfqpoint{1.250000in}{0.550000in}}{\pgfqpoint{7.750000in}{3.850000in}}%
\pgfusepath{clip}%
\pgfsetbuttcap%
\pgfsetmiterjoin%
\definecolor{currentfill}{rgb}{0.501961,0.501961,0.501961}%
\pgfsetfillcolor{currentfill}%
\pgfsetfillopacity{0.700000}%
\pgfsetlinewidth{0.000000pt}%
\definecolor{currentstroke}{rgb}{0.000000,0.000000,0.000000}%
\pgfsetstrokecolor{currentstroke}%
\pgfsetstrokeopacity{0.700000}%
\pgfsetdash{}{0pt}%
\pgfpathmoveto{\pgfqpoint{2.262784in}{0.550000in}}%
\pgfpathlineto{\pgfqpoint{2.290305in}{0.550000in}}%
\pgfpathlineto{\pgfqpoint{2.290305in}{1.089405in}}%
\pgfpathlineto{\pgfqpoint{2.262784in}{1.089405in}}%
\pgfpathlineto{\pgfqpoint{2.262784in}{0.550000in}}%
\pgfpathclose%
\pgfusepath{fill}%
\end{pgfscope}%
\begin{pgfscope}%
\pgfpathrectangle{\pgfqpoint{1.250000in}{0.550000in}}{\pgfqpoint{7.750000in}{3.850000in}}%
\pgfusepath{clip}%
\pgfsetbuttcap%
\pgfsetmiterjoin%
\definecolor{currentfill}{rgb}{0.501961,0.501961,0.501961}%
\pgfsetfillcolor{currentfill}%
\pgfsetfillopacity{0.700000}%
\pgfsetlinewidth{0.000000pt}%
\definecolor{currentstroke}{rgb}{0.000000,0.000000,0.000000}%
\pgfsetstrokecolor{currentstroke}%
\pgfsetstrokeopacity{0.700000}%
\pgfsetdash{}{0pt}%
\pgfpathmoveto{\pgfqpoint{2.290305in}{0.550000in}}%
\pgfpathlineto{\pgfqpoint{2.317827in}{0.550000in}}%
\pgfpathlineto{\pgfqpoint{2.317827in}{1.623604in}}%
\pgfpathlineto{\pgfqpoint{2.290305in}{1.623604in}}%
\pgfpathlineto{\pgfqpoint{2.290305in}{0.550000in}}%
\pgfpathclose%
\pgfusepath{fill}%
\end{pgfscope}%
\begin{pgfscope}%
\pgfpathrectangle{\pgfqpoint{1.250000in}{0.550000in}}{\pgfqpoint{7.750000in}{3.850000in}}%
\pgfusepath{clip}%
\pgfsetbuttcap%
\pgfsetmiterjoin%
\definecolor{currentfill}{rgb}{0.501961,0.501961,0.501961}%
\pgfsetfillcolor{currentfill}%
\pgfsetfillopacity{0.700000}%
\pgfsetlinewidth{0.000000pt}%
\definecolor{currentstroke}{rgb}{0.000000,0.000000,0.000000}%
\pgfsetstrokecolor{currentstroke}%
\pgfsetstrokeopacity{0.700000}%
\pgfsetdash{}{0pt}%
\pgfpathmoveto{\pgfqpoint{2.317827in}{0.550000in}}%
\pgfpathlineto{\pgfqpoint{2.345348in}{0.550000in}}%
\pgfpathlineto{\pgfqpoint{2.345348in}{1.356122in}}%
\pgfpathlineto{\pgfqpoint{2.317827in}{1.356122in}}%
\pgfpathlineto{\pgfqpoint{2.317827in}{0.550000in}}%
\pgfpathclose%
\pgfusepath{fill}%
\end{pgfscope}%
\begin{pgfscope}%
\pgfpathrectangle{\pgfqpoint{1.250000in}{0.550000in}}{\pgfqpoint{7.750000in}{3.850000in}}%
\pgfusepath{clip}%
\pgfsetbuttcap%
\pgfsetmiterjoin%
\definecolor{currentfill}{rgb}{0.501961,0.501961,0.501961}%
\pgfsetfillcolor{currentfill}%
\pgfsetfillopacity{0.700000}%
\pgfsetlinewidth{0.000000pt}%
\definecolor{currentstroke}{rgb}{0.000000,0.000000,0.000000}%
\pgfsetstrokecolor{currentstroke}%
\pgfsetstrokeopacity{0.700000}%
\pgfsetdash{}{0pt}%
\pgfpathmoveto{\pgfqpoint{2.345348in}{0.550000in}}%
\pgfpathlineto{\pgfqpoint{2.372869in}{0.550000in}}%
\pgfpathlineto{\pgfqpoint{2.372869in}{1.344332in}}%
\pgfpathlineto{\pgfqpoint{2.345348in}{1.344332in}}%
\pgfpathlineto{\pgfqpoint{2.345348in}{0.550000in}}%
\pgfpathclose%
\pgfusepath{fill}%
\end{pgfscope}%
\begin{pgfscope}%
\pgfpathrectangle{\pgfqpoint{1.250000in}{0.550000in}}{\pgfqpoint{7.750000in}{3.850000in}}%
\pgfusepath{clip}%
\pgfsetbuttcap%
\pgfsetmiterjoin%
\definecolor{currentfill}{rgb}{0.501961,0.501961,0.501961}%
\pgfsetfillcolor{currentfill}%
\pgfsetfillopacity{0.700000}%
\pgfsetlinewidth{0.000000pt}%
\definecolor{currentstroke}{rgb}{0.000000,0.000000,0.000000}%
\pgfsetstrokecolor{currentstroke}%
\pgfsetstrokeopacity{0.700000}%
\pgfsetdash{}{0pt}%
\pgfpathmoveto{\pgfqpoint{2.372869in}{0.550000in}}%
\pgfpathlineto{\pgfqpoint{2.400391in}{0.550000in}}%
\pgfpathlineto{\pgfqpoint{2.400391in}{1.777785in}}%
\pgfpathlineto{\pgfqpoint{2.372869in}{1.777785in}}%
\pgfpathlineto{\pgfqpoint{2.372869in}{0.550000in}}%
\pgfpathclose%
\pgfusepath{fill}%
\end{pgfscope}%
\begin{pgfscope}%
\pgfpathrectangle{\pgfqpoint{1.250000in}{0.550000in}}{\pgfqpoint{7.750000in}{3.850000in}}%
\pgfusepath{clip}%
\pgfsetbuttcap%
\pgfsetmiterjoin%
\definecolor{currentfill}{rgb}{0.501961,0.501961,0.501961}%
\pgfsetfillcolor{currentfill}%
\pgfsetfillopacity{0.700000}%
\pgfsetlinewidth{0.000000pt}%
\definecolor{currentstroke}{rgb}{0.000000,0.000000,0.000000}%
\pgfsetstrokecolor{currentstroke}%
\pgfsetstrokeopacity{0.700000}%
\pgfsetdash{}{0pt}%
\pgfpathmoveto{\pgfqpoint{2.400391in}{0.550000in}}%
\pgfpathlineto{\pgfqpoint{2.427912in}{0.550000in}}%
\pgfpathlineto{\pgfqpoint{2.427912in}{1.537403in}}%
\pgfpathlineto{\pgfqpoint{2.400391in}{1.537403in}}%
\pgfpathlineto{\pgfqpoint{2.400391in}{0.550000in}}%
\pgfpathclose%
\pgfusepath{fill}%
\end{pgfscope}%
\begin{pgfscope}%
\pgfpathrectangle{\pgfqpoint{1.250000in}{0.550000in}}{\pgfqpoint{7.750000in}{3.850000in}}%
\pgfusepath{clip}%
\pgfsetbuttcap%
\pgfsetmiterjoin%
\definecolor{currentfill}{rgb}{0.501961,0.501961,0.501961}%
\pgfsetfillcolor{currentfill}%
\pgfsetfillopacity{0.700000}%
\pgfsetlinewidth{0.000000pt}%
\definecolor{currentstroke}{rgb}{0.000000,0.000000,0.000000}%
\pgfsetstrokecolor{currentstroke}%
\pgfsetstrokeopacity{0.700000}%
\pgfsetdash{}{0pt}%
\pgfpathmoveto{\pgfqpoint{2.427912in}{0.550000in}}%
\pgfpathlineto{\pgfqpoint{2.455433in}{0.550000in}}%
\pgfpathlineto{\pgfqpoint{2.455433in}{1.513978in}}%
\pgfpathlineto{\pgfqpoint{2.427912in}{1.513978in}}%
\pgfpathlineto{\pgfqpoint{2.427912in}{0.550000in}}%
\pgfpathclose%
\pgfusepath{fill}%
\end{pgfscope}%
\begin{pgfscope}%
\pgfpathrectangle{\pgfqpoint{1.250000in}{0.550000in}}{\pgfqpoint{7.750000in}{3.850000in}}%
\pgfusepath{clip}%
\pgfsetbuttcap%
\pgfsetmiterjoin%
\definecolor{currentfill}{rgb}{0.501961,0.501961,0.501961}%
\pgfsetfillcolor{currentfill}%
\pgfsetfillopacity{0.700000}%
\pgfsetlinewidth{0.000000pt}%
\definecolor{currentstroke}{rgb}{0.000000,0.000000,0.000000}%
\pgfsetstrokecolor{currentstroke}%
\pgfsetstrokeopacity{0.700000}%
\pgfsetdash{}{0pt}%
\pgfpathmoveto{\pgfqpoint{2.455433in}{0.550000in}}%
\pgfpathlineto{\pgfqpoint{2.482955in}{0.550000in}}%
\pgfpathlineto{\pgfqpoint{2.482955in}{1.840101in}}%
\pgfpathlineto{\pgfqpoint{2.455433in}{1.840101in}}%
\pgfpathlineto{\pgfqpoint{2.455433in}{0.550000in}}%
\pgfpathclose%
\pgfusepath{fill}%
\end{pgfscope}%
\begin{pgfscope}%
\pgfpathrectangle{\pgfqpoint{1.250000in}{0.550000in}}{\pgfqpoint{7.750000in}{3.850000in}}%
\pgfusepath{clip}%
\pgfsetbuttcap%
\pgfsetmiterjoin%
\definecolor{currentfill}{rgb}{0.501961,0.501961,0.501961}%
\pgfsetfillcolor{currentfill}%
\pgfsetfillopacity{0.700000}%
\pgfsetlinewidth{0.000000pt}%
\definecolor{currentstroke}{rgb}{0.000000,0.000000,0.000000}%
\pgfsetstrokecolor{currentstroke}%
\pgfsetstrokeopacity{0.700000}%
\pgfsetdash{}{0pt}%
\pgfpathmoveto{\pgfqpoint{2.482955in}{0.550000in}}%
\pgfpathlineto{\pgfqpoint{2.510476in}{0.550000in}}%
\pgfpathlineto{\pgfqpoint{2.510476in}{1.723125in}}%
\pgfpathlineto{\pgfqpoint{2.482955in}{1.723125in}}%
\pgfpathlineto{\pgfqpoint{2.482955in}{0.550000in}}%
\pgfpathclose%
\pgfusepath{fill}%
\end{pgfscope}%
\begin{pgfscope}%
\pgfpathrectangle{\pgfqpoint{1.250000in}{0.550000in}}{\pgfqpoint{7.750000in}{3.850000in}}%
\pgfusepath{clip}%
\pgfsetbuttcap%
\pgfsetmiterjoin%
\definecolor{currentfill}{rgb}{0.501961,0.501961,0.501961}%
\pgfsetfillcolor{currentfill}%
\pgfsetfillopacity{0.700000}%
\pgfsetlinewidth{0.000000pt}%
\definecolor{currentstroke}{rgb}{0.000000,0.000000,0.000000}%
\pgfsetstrokecolor{currentstroke}%
\pgfsetstrokeopacity{0.700000}%
\pgfsetdash{}{0pt}%
\pgfpathmoveto{\pgfqpoint{2.510476in}{0.550000in}}%
\pgfpathlineto{\pgfqpoint{2.537997in}{0.550000in}}%
\pgfpathlineto{\pgfqpoint{2.537997in}{1.680867in}}%
\pgfpathlineto{\pgfqpoint{2.510476in}{1.680867in}}%
\pgfpathlineto{\pgfqpoint{2.510476in}{0.550000in}}%
\pgfpathclose%
\pgfusepath{fill}%
\end{pgfscope}%
\begin{pgfscope}%
\pgfpathrectangle{\pgfqpoint{1.250000in}{0.550000in}}{\pgfqpoint{7.750000in}{3.850000in}}%
\pgfusepath{clip}%
\pgfsetbuttcap%
\pgfsetmiterjoin%
\definecolor{currentfill}{rgb}{0.501961,0.501961,0.501961}%
\pgfsetfillcolor{currentfill}%
\pgfsetfillopacity{0.700000}%
\pgfsetlinewidth{0.000000pt}%
\definecolor{currentstroke}{rgb}{0.000000,0.000000,0.000000}%
\pgfsetstrokecolor{currentstroke}%
\pgfsetstrokeopacity{0.700000}%
\pgfsetdash{}{0pt}%
\pgfpathmoveto{\pgfqpoint{2.537997in}{0.550000in}}%
\pgfpathlineto{\pgfqpoint{2.565518in}{0.550000in}}%
\pgfpathlineto{\pgfqpoint{2.565518in}{1.612580in}}%
\pgfpathlineto{\pgfqpoint{2.537997in}{1.612580in}}%
\pgfpathlineto{\pgfqpoint{2.537997in}{0.550000in}}%
\pgfpathclose%
\pgfusepath{fill}%
\end{pgfscope}%
\begin{pgfscope}%
\pgfpathrectangle{\pgfqpoint{1.250000in}{0.550000in}}{\pgfqpoint{7.750000in}{3.850000in}}%
\pgfusepath{clip}%
\pgfsetbuttcap%
\pgfsetmiterjoin%
\definecolor{currentfill}{rgb}{0.501961,0.501961,0.501961}%
\pgfsetfillcolor{currentfill}%
\pgfsetfillopacity{0.700000}%
\pgfsetlinewidth{0.000000pt}%
\definecolor{currentstroke}{rgb}{0.000000,0.000000,0.000000}%
\pgfsetstrokecolor{currentstroke}%
\pgfsetstrokeopacity{0.700000}%
\pgfsetdash{}{0pt}%
\pgfpathmoveto{\pgfqpoint{2.565518in}{0.550000in}}%
\pgfpathlineto{\pgfqpoint{2.593040in}{0.550000in}}%
\pgfpathlineto{\pgfqpoint{2.593040in}{1.534341in}}%
\pgfpathlineto{\pgfqpoint{2.565518in}{1.534341in}}%
\pgfpathlineto{\pgfqpoint{2.565518in}{0.550000in}}%
\pgfpathclose%
\pgfusepath{fill}%
\end{pgfscope}%
\begin{pgfscope}%
\pgfpathrectangle{\pgfqpoint{1.250000in}{0.550000in}}{\pgfqpoint{7.750000in}{3.850000in}}%
\pgfusepath{clip}%
\pgfsetbuttcap%
\pgfsetmiterjoin%
\definecolor{currentfill}{rgb}{0.501961,0.501961,0.501961}%
\pgfsetfillcolor{currentfill}%
\pgfsetfillopacity{0.700000}%
\pgfsetlinewidth{0.000000pt}%
\definecolor{currentstroke}{rgb}{0.000000,0.000000,0.000000}%
\pgfsetstrokecolor{currentstroke}%
\pgfsetstrokeopacity{0.700000}%
\pgfsetdash{}{0pt}%
\pgfpathmoveto{\pgfqpoint{2.593040in}{0.550000in}}%
\pgfpathlineto{\pgfqpoint{2.620561in}{0.550000in}}%
\pgfpathlineto{\pgfqpoint{2.620561in}{1.638609in}}%
\pgfpathlineto{\pgfqpoint{2.593040in}{1.638609in}}%
\pgfpathlineto{\pgfqpoint{2.593040in}{0.550000in}}%
\pgfpathclose%
\pgfusepath{fill}%
\end{pgfscope}%
\begin{pgfscope}%
\pgfpathrectangle{\pgfqpoint{1.250000in}{0.550000in}}{\pgfqpoint{7.750000in}{3.850000in}}%
\pgfusepath{clip}%
\pgfsetbuttcap%
\pgfsetmiterjoin%
\definecolor{currentfill}{rgb}{0.501961,0.501961,0.501961}%
\pgfsetfillcolor{currentfill}%
\pgfsetfillopacity{0.700000}%
\pgfsetlinewidth{0.000000pt}%
\definecolor{currentstroke}{rgb}{0.000000,0.000000,0.000000}%
\pgfsetstrokecolor{currentstroke}%
\pgfsetstrokeopacity{0.700000}%
\pgfsetdash{}{0pt}%
\pgfpathmoveto{\pgfqpoint{2.620561in}{0.550000in}}%
\pgfpathlineto{\pgfqpoint{2.648082in}{0.550000in}}%
\pgfpathlineto{\pgfqpoint{2.648082in}{1.574915in}}%
\pgfpathlineto{\pgfqpoint{2.620561in}{1.574915in}}%
\pgfpathlineto{\pgfqpoint{2.620561in}{0.550000in}}%
\pgfpathclose%
\pgfusepath{fill}%
\end{pgfscope}%
\begin{pgfscope}%
\pgfpathrectangle{\pgfqpoint{1.250000in}{0.550000in}}{\pgfqpoint{7.750000in}{3.850000in}}%
\pgfusepath{clip}%
\pgfsetbuttcap%
\pgfsetmiterjoin%
\definecolor{currentfill}{rgb}{0.501961,0.501961,0.501961}%
\pgfsetfillcolor{currentfill}%
\pgfsetfillopacity{0.700000}%
\pgfsetlinewidth{0.000000pt}%
\definecolor{currentstroke}{rgb}{0.000000,0.000000,0.000000}%
\pgfsetstrokecolor{currentstroke}%
\pgfsetstrokeopacity{0.700000}%
\pgfsetdash{}{0pt}%
\pgfpathmoveto{\pgfqpoint{2.648082in}{0.550000in}}%
\pgfpathlineto{\pgfqpoint{2.675604in}{0.550000in}}%
\pgfpathlineto{\pgfqpoint{2.675604in}{1.368677in}}%
\pgfpathlineto{\pgfqpoint{2.648082in}{1.368677in}}%
\pgfpathlineto{\pgfqpoint{2.648082in}{0.550000in}}%
\pgfpathclose%
\pgfusepath{fill}%
\end{pgfscope}%
\begin{pgfscope}%
\pgfpathrectangle{\pgfqpoint{1.250000in}{0.550000in}}{\pgfqpoint{7.750000in}{3.850000in}}%
\pgfusepath{clip}%
\pgfsetbuttcap%
\pgfsetmiterjoin%
\definecolor{currentfill}{rgb}{0.501961,0.501961,0.501961}%
\pgfsetfillcolor{currentfill}%
\pgfsetfillopacity{0.700000}%
\pgfsetlinewidth{0.000000pt}%
\definecolor{currentstroke}{rgb}{0.000000,0.000000,0.000000}%
\pgfsetstrokecolor{currentstroke}%
\pgfsetstrokeopacity{0.700000}%
\pgfsetdash{}{0pt}%
\pgfpathmoveto{\pgfqpoint{2.675604in}{0.550000in}}%
\pgfpathlineto{\pgfqpoint{2.703125in}{0.550000in}}%
\pgfpathlineto{\pgfqpoint{2.703125in}{1.402054in}}%
\pgfpathlineto{\pgfqpoint{2.675604in}{1.402054in}}%
\pgfpathlineto{\pgfqpoint{2.675604in}{0.550000in}}%
\pgfpathclose%
\pgfusepath{fill}%
\end{pgfscope}%
\begin{pgfscope}%
\pgfpathrectangle{\pgfqpoint{1.250000in}{0.550000in}}{\pgfqpoint{7.750000in}{3.850000in}}%
\pgfusepath{clip}%
\pgfsetbuttcap%
\pgfsetmiterjoin%
\definecolor{currentfill}{rgb}{0.501961,0.501961,0.501961}%
\pgfsetfillcolor{currentfill}%
\pgfsetfillopacity{0.700000}%
\pgfsetlinewidth{0.000000pt}%
\definecolor{currentstroke}{rgb}{0.000000,0.000000,0.000000}%
\pgfsetstrokecolor{currentstroke}%
\pgfsetstrokeopacity{0.700000}%
\pgfsetdash{}{0pt}%
\pgfpathmoveto{\pgfqpoint{2.703125in}{0.550000in}}%
\pgfpathlineto{\pgfqpoint{2.730646in}{0.550000in}}%
\pgfpathlineto{\pgfqpoint{2.730646in}{1.303605in}}%
\pgfpathlineto{\pgfqpoint{2.703125in}{1.303605in}}%
\pgfpathlineto{\pgfqpoint{2.703125in}{0.550000in}}%
\pgfpathclose%
\pgfusepath{fill}%
\end{pgfscope}%
\begin{pgfscope}%
\pgfpathrectangle{\pgfqpoint{1.250000in}{0.550000in}}{\pgfqpoint{7.750000in}{3.850000in}}%
\pgfusepath{clip}%
\pgfsetbuttcap%
\pgfsetmiterjoin%
\definecolor{currentfill}{rgb}{0.501961,0.501961,0.501961}%
\pgfsetfillcolor{currentfill}%
\pgfsetfillopacity{0.700000}%
\pgfsetlinewidth{0.000000pt}%
\definecolor{currentstroke}{rgb}{0.000000,0.000000,0.000000}%
\pgfsetstrokecolor{currentstroke}%
\pgfsetstrokeopacity{0.700000}%
\pgfsetdash{}{0pt}%
\pgfpathmoveto{\pgfqpoint{2.730646in}{0.550000in}}%
\pgfpathlineto{\pgfqpoint{2.758168in}{0.550000in}}%
\pgfpathlineto{\pgfqpoint{2.758168in}{1.292734in}}%
\pgfpathlineto{\pgfqpoint{2.730646in}{1.292734in}}%
\pgfpathlineto{\pgfqpoint{2.730646in}{0.550000in}}%
\pgfpathclose%
\pgfusepath{fill}%
\end{pgfscope}%
\begin{pgfscope}%
\pgfpathrectangle{\pgfqpoint{1.250000in}{0.550000in}}{\pgfqpoint{7.750000in}{3.850000in}}%
\pgfusepath{clip}%
\pgfsetbuttcap%
\pgfsetmiterjoin%
\definecolor{currentfill}{rgb}{0.501961,0.501961,0.501961}%
\pgfsetfillcolor{currentfill}%
\pgfsetfillopacity{0.700000}%
\pgfsetlinewidth{0.000000pt}%
\definecolor{currentstroke}{rgb}{0.000000,0.000000,0.000000}%
\pgfsetstrokecolor{currentstroke}%
\pgfsetstrokeopacity{0.700000}%
\pgfsetdash{}{0pt}%
\pgfpathmoveto{\pgfqpoint{2.758168in}{0.550000in}}%
\pgfpathlineto{\pgfqpoint{2.785689in}{0.550000in}}%
\pgfpathlineto{\pgfqpoint{2.785689in}{1.114974in}}%
\pgfpathlineto{\pgfqpoint{2.758168in}{1.114974in}}%
\pgfpathlineto{\pgfqpoint{2.758168in}{0.550000in}}%
\pgfpathclose%
\pgfusepath{fill}%
\end{pgfscope}%
\begin{pgfscope}%
\pgfpathrectangle{\pgfqpoint{1.250000in}{0.550000in}}{\pgfqpoint{7.750000in}{3.850000in}}%
\pgfusepath{clip}%
\pgfsetbuttcap%
\pgfsetmiterjoin%
\definecolor{currentfill}{rgb}{0.501961,0.501961,0.501961}%
\pgfsetfillcolor{currentfill}%
\pgfsetfillopacity{0.700000}%
\pgfsetlinewidth{0.000000pt}%
\definecolor{currentstroke}{rgb}{0.000000,0.000000,0.000000}%
\pgfsetstrokecolor{currentstroke}%
\pgfsetstrokeopacity{0.700000}%
\pgfsetdash{}{0pt}%
\pgfpathmoveto{\pgfqpoint{2.785689in}{0.550000in}}%
\pgfpathlineto{\pgfqpoint{2.813210in}{0.550000in}}%
\pgfpathlineto{\pgfqpoint{2.813210in}{1.037194in}}%
\pgfpathlineto{\pgfqpoint{2.785689in}{1.037194in}}%
\pgfpathlineto{\pgfqpoint{2.785689in}{0.550000in}}%
\pgfpathclose%
\pgfusepath{fill}%
\end{pgfscope}%
\begin{pgfscope}%
\pgfpathrectangle{\pgfqpoint{1.250000in}{0.550000in}}{\pgfqpoint{7.750000in}{3.850000in}}%
\pgfusepath{clip}%
\pgfsetbuttcap%
\pgfsetmiterjoin%
\definecolor{currentfill}{rgb}{0.501961,0.501961,0.501961}%
\pgfsetfillcolor{currentfill}%
\pgfsetfillopacity{0.700000}%
\pgfsetlinewidth{0.000000pt}%
\definecolor{currentstroke}{rgb}{0.000000,0.000000,0.000000}%
\pgfsetstrokecolor{currentstroke}%
\pgfsetstrokeopacity{0.700000}%
\pgfsetdash{}{0pt}%
\pgfpathmoveto{\pgfqpoint{2.813210in}{0.550000in}}%
\pgfpathlineto{\pgfqpoint{2.840732in}{0.550000in}}%
\pgfpathlineto{\pgfqpoint{2.840732in}{1.109462in}}%
\pgfpathlineto{\pgfqpoint{2.813210in}{1.109462in}}%
\pgfpathlineto{\pgfqpoint{2.813210in}{0.550000in}}%
\pgfpathclose%
\pgfusepath{fill}%
\end{pgfscope}%
\begin{pgfscope}%
\pgfpathrectangle{\pgfqpoint{1.250000in}{0.550000in}}{\pgfqpoint{7.750000in}{3.850000in}}%
\pgfusepath{clip}%
\pgfsetbuttcap%
\pgfsetmiterjoin%
\definecolor{currentfill}{rgb}{0.501961,0.501961,0.501961}%
\pgfsetfillcolor{currentfill}%
\pgfsetfillopacity{0.700000}%
\pgfsetlinewidth{0.000000pt}%
\definecolor{currentstroke}{rgb}{0.000000,0.000000,0.000000}%
\pgfsetstrokecolor{currentstroke}%
\pgfsetstrokeopacity{0.700000}%
\pgfsetdash{}{0pt}%
\pgfpathmoveto{\pgfqpoint{2.840732in}{0.550000in}}%
\pgfpathlineto{\pgfqpoint{2.868253in}{0.550000in}}%
\pgfpathlineto{\pgfqpoint{2.868253in}{0.884085in}}%
\pgfpathlineto{\pgfqpoint{2.840732in}{0.884085in}}%
\pgfpathlineto{\pgfqpoint{2.840732in}{0.550000in}}%
\pgfpathclose%
\pgfusepath{fill}%
\end{pgfscope}%
\begin{pgfscope}%
\pgfpathrectangle{\pgfqpoint{1.250000in}{0.550000in}}{\pgfqpoint{7.750000in}{3.850000in}}%
\pgfusepath{clip}%
\pgfsetbuttcap%
\pgfsetmiterjoin%
\definecolor{currentfill}{rgb}{0.501961,0.501961,0.501961}%
\pgfsetfillcolor{currentfill}%
\pgfsetfillopacity{0.700000}%
\pgfsetlinewidth{0.000000pt}%
\definecolor{currentstroke}{rgb}{0.000000,0.000000,0.000000}%
\pgfsetstrokecolor{currentstroke}%
\pgfsetstrokeopacity{0.700000}%
\pgfsetdash{}{0pt}%
\pgfpathmoveto{\pgfqpoint{2.868253in}{0.550000in}}%
\pgfpathlineto{\pgfqpoint{2.895774in}{0.550000in}}%
\pgfpathlineto{\pgfqpoint{2.895774in}{0.881023in}}%
\pgfpathlineto{\pgfqpoint{2.868253in}{0.881023in}}%
\pgfpathlineto{\pgfqpoint{2.868253in}{0.550000in}}%
\pgfpathclose%
\pgfusepath{fill}%
\end{pgfscope}%
\begin{pgfscope}%
\pgfpathrectangle{\pgfqpoint{1.250000in}{0.550000in}}{\pgfqpoint{7.750000in}{3.850000in}}%
\pgfusepath{clip}%
\pgfsetbuttcap%
\pgfsetmiterjoin%
\definecolor{currentfill}{rgb}{0.501961,0.501961,0.501961}%
\pgfsetfillcolor{currentfill}%
\pgfsetfillopacity{0.700000}%
\pgfsetlinewidth{0.000000pt}%
\definecolor{currentstroke}{rgb}{0.000000,0.000000,0.000000}%
\pgfsetstrokecolor{currentstroke}%
\pgfsetstrokeopacity{0.700000}%
\pgfsetdash{}{0pt}%
\pgfpathmoveto{\pgfqpoint{2.895774in}{0.550000in}}%
\pgfpathlineto{\pgfqpoint{2.923295in}{0.550000in}}%
\pgfpathlineto{\pgfqpoint{2.923295in}{0.901846in}}%
\pgfpathlineto{\pgfqpoint{2.895774in}{0.901846in}}%
\pgfpathlineto{\pgfqpoint{2.895774in}{0.550000in}}%
\pgfpathclose%
\pgfusepath{fill}%
\end{pgfscope}%
\begin{pgfscope}%
\pgfpathrectangle{\pgfqpoint{1.250000in}{0.550000in}}{\pgfqpoint{7.750000in}{3.850000in}}%
\pgfusepath{clip}%
\pgfsetbuttcap%
\pgfsetmiterjoin%
\definecolor{currentfill}{rgb}{0.501961,0.501961,0.501961}%
\pgfsetfillcolor{currentfill}%
\pgfsetfillopacity{0.700000}%
\pgfsetlinewidth{0.000000pt}%
\definecolor{currentstroke}{rgb}{0.000000,0.000000,0.000000}%
\pgfsetstrokecolor{currentstroke}%
\pgfsetstrokeopacity{0.700000}%
\pgfsetdash{}{0pt}%
\pgfpathmoveto{\pgfqpoint{2.923295in}{0.550000in}}%
\pgfpathlineto{\pgfqpoint{2.950817in}{0.550000in}}%
\pgfpathlineto{\pgfqpoint{2.950817in}{0.780430in}}%
\pgfpathlineto{\pgfqpoint{2.923295in}{0.780430in}}%
\pgfpathlineto{\pgfqpoint{2.923295in}{0.550000in}}%
\pgfpathclose%
\pgfusepath{fill}%
\end{pgfscope}%
\begin{pgfscope}%
\pgfpathrectangle{\pgfqpoint{1.250000in}{0.550000in}}{\pgfqpoint{7.750000in}{3.850000in}}%
\pgfusepath{clip}%
\pgfsetbuttcap%
\pgfsetmiterjoin%
\definecolor{currentfill}{rgb}{0.501961,0.501961,0.501961}%
\pgfsetfillcolor{currentfill}%
\pgfsetfillopacity{0.700000}%
\pgfsetlinewidth{0.000000pt}%
\definecolor{currentstroke}{rgb}{0.000000,0.000000,0.000000}%
\pgfsetstrokecolor{currentstroke}%
\pgfsetstrokeopacity{0.700000}%
\pgfsetdash{}{0pt}%
\pgfpathmoveto{\pgfqpoint{2.950817in}{0.550000in}}%
\pgfpathlineto{\pgfqpoint{2.978338in}{0.550000in}}%
\pgfpathlineto{\pgfqpoint{2.978338in}{0.798803in}}%
\pgfpathlineto{\pgfqpoint{2.950817in}{0.798803in}}%
\pgfpathlineto{\pgfqpoint{2.950817in}{0.550000in}}%
\pgfpathclose%
\pgfusepath{fill}%
\end{pgfscope}%
\begin{pgfscope}%
\pgfpathrectangle{\pgfqpoint{1.250000in}{0.550000in}}{\pgfqpoint{7.750000in}{3.850000in}}%
\pgfusepath{clip}%
\pgfsetbuttcap%
\pgfsetmiterjoin%
\definecolor{currentfill}{rgb}{0.501961,0.501961,0.501961}%
\pgfsetfillcolor{currentfill}%
\pgfsetfillopacity{0.700000}%
\pgfsetlinewidth{0.000000pt}%
\definecolor{currentstroke}{rgb}{0.000000,0.000000,0.000000}%
\pgfsetstrokecolor{currentstroke}%
\pgfsetstrokeopacity{0.700000}%
\pgfsetdash{}{0pt}%
\pgfpathmoveto{\pgfqpoint{2.978338in}{0.550000in}}%
\pgfpathlineto{\pgfqpoint{3.005859in}{0.550000in}}%
\pgfpathlineto{\pgfqpoint{3.005859in}{0.760219in}}%
\pgfpathlineto{\pgfqpoint{2.978338in}{0.760219in}}%
\pgfpathlineto{\pgfqpoint{2.978338in}{0.550000in}}%
\pgfpathclose%
\pgfusepath{fill}%
\end{pgfscope}%
\begin{pgfscope}%
\pgfpathrectangle{\pgfqpoint{1.250000in}{0.550000in}}{\pgfqpoint{7.750000in}{3.850000in}}%
\pgfusepath{clip}%
\pgfsetbuttcap%
\pgfsetmiterjoin%
\definecolor{currentfill}{rgb}{0.501961,0.501961,0.501961}%
\pgfsetfillcolor{currentfill}%
\pgfsetfillopacity{0.700000}%
\pgfsetlinewidth{0.000000pt}%
\definecolor{currentstroke}{rgb}{0.000000,0.000000,0.000000}%
\pgfsetstrokecolor{currentstroke}%
\pgfsetstrokeopacity{0.700000}%
\pgfsetdash{}{0pt}%
\pgfpathmoveto{\pgfqpoint{3.005859in}{0.550000in}}%
\pgfpathlineto{\pgfqpoint{3.033381in}{0.550000in}}%
\pgfpathlineto{\pgfqpoint{3.033381in}{0.721330in}}%
\pgfpathlineto{\pgfqpoint{3.005859in}{0.721330in}}%
\pgfpathlineto{\pgfqpoint{3.005859in}{0.550000in}}%
\pgfpathclose%
\pgfusepath{fill}%
\end{pgfscope}%
\begin{pgfscope}%
\pgfpathrectangle{\pgfqpoint{1.250000in}{0.550000in}}{\pgfqpoint{7.750000in}{3.850000in}}%
\pgfusepath{clip}%
\pgfsetbuttcap%
\pgfsetmiterjoin%
\definecolor{currentfill}{rgb}{0.501961,0.501961,0.501961}%
\pgfsetfillcolor{currentfill}%
\pgfsetfillopacity{0.700000}%
\pgfsetlinewidth{0.000000pt}%
\definecolor{currentstroke}{rgb}{0.000000,0.000000,0.000000}%
\pgfsetstrokecolor{currentstroke}%
\pgfsetstrokeopacity{0.700000}%
\pgfsetdash{}{0pt}%
\pgfpathmoveto{\pgfqpoint{3.033381in}{0.550000in}}%
\pgfpathlineto{\pgfqpoint{3.060902in}{0.550000in}}%
\pgfpathlineto{\pgfqpoint{3.060902in}{0.761291in}}%
\pgfpathlineto{\pgfqpoint{3.033381in}{0.761291in}}%
\pgfpathlineto{\pgfqpoint{3.033381in}{0.550000in}}%
\pgfpathclose%
\pgfusepath{fill}%
\end{pgfscope}%
\begin{pgfscope}%
\pgfpathrectangle{\pgfqpoint{1.250000in}{0.550000in}}{\pgfqpoint{7.750000in}{3.850000in}}%
\pgfusepath{clip}%
\pgfsetbuttcap%
\pgfsetmiterjoin%
\definecolor{currentfill}{rgb}{0.501961,0.501961,0.501961}%
\pgfsetfillcolor{currentfill}%
\pgfsetfillopacity{0.700000}%
\pgfsetlinewidth{0.000000pt}%
\definecolor{currentstroke}{rgb}{0.000000,0.000000,0.000000}%
\pgfsetstrokecolor{currentstroke}%
\pgfsetstrokeopacity{0.700000}%
\pgfsetdash{}{0pt}%
\pgfpathmoveto{\pgfqpoint{3.060902in}{0.550000in}}%
\pgfpathlineto{\pgfqpoint{3.088423in}{0.550000in}}%
\pgfpathlineto{\pgfqpoint{3.088423in}{0.709693in}}%
\pgfpathlineto{\pgfqpoint{3.060902in}{0.709693in}}%
\pgfpathlineto{\pgfqpoint{3.060902in}{0.550000in}}%
\pgfpathclose%
\pgfusepath{fill}%
\end{pgfscope}%
\begin{pgfscope}%
\pgfpathrectangle{\pgfqpoint{1.250000in}{0.550000in}}{\pgfqpoint{7.750000in}{3.850000in}}%
\pgfusepath{clip}%
\pgfsetbuttcap%
\pgfsetmiterjoin%
\definecolor{currentfill}{rgb}{0.501961,0.501961,0.501961}%
\pgfsetfillcolor{currentfill}%
\pgfsetfillopacity{0.700000}%
\pgfsetlinewidth{0.000000pt}%
\definecolor{currentstroke}{rgb}{0.000000,0.000000,0.000000}%
\pgfsetstrokecolor{currentstroke}%
\pgfsetstrokeopacity{0.700000}%
\pgfsetdash{}{0pt}%
\pgfpathmoveto{\pgfqpoint{3.088423in}{0.550000in}}%
\pgfpathlineto{\pgfqpoint{3.115945in}{0.550000in}}%
\pgfpathlineto{\pgfqpoint{3.115945in}{0.738631in}}%
\pgfpathlineto{\pgfqpoint{3.088423in}{0.738631in}}%
\pgfpathlineto{\pgfqpoint{3.088423in}{0.550000in}}%
\pgfpathclose%
\pgfusepath{fill}%
\end{pgfscope}%
\begin{pgfscope}%
\pgfpathrectangle{\pgfqpoint{1.250000in}{0.550000in}}{\pgfqpoint{7.750000in}{3.850000in}}%
\pgfusepath{clip}%
\pgfsetbuttcap%
\pgfsetmiterjoin%
\definecolor{currentfill}{rgb}{0.501961,0.501961,0.501961}%
\pgfsetfillcolor{currentfill}%
\pgfsetfillopacity{0.700000}%
\pgfsetlinewidth{0.000000pt}%
\definecolor{currentstroke}{rgb}{0.000000,0.000000,0.000000}%
\pgfsetstrokecolor{currentstroke}%
\pgfsetstrokeopacity{0.700000}%
\pgfsetdash{}{0pt}%
\pgfpathmoveto{\pgfqpoint{3.115945in}{0.550000in}}%
\pgfpathlineto{\pgfqpoint{3.143466in}{0.550000in}}%
\pgfpathlineto{\pgfqpoint{3.143466in}{0.707090in}}%
\pgfpathlineto{\pgfqpoint{3.115945in}{0.707090in}}%
\pgfpathlineto{\pgfqpoint{3.115945in}{0.550000in}}%
\pgfpathclose%
\pgfusepath{fill}%
\end{pgfscope}%
\begin{pgfscope}%
\pgfpathrectangle{\pgfqpoint{1.250000in}{0.550000in}}{\pgfqpoint{7.750000in}{3.850000in}}%
\pgfusepath{clip}%
\pgfsetbuttcap%
\pgfsetmiterjoin%
\definecolor{currentfill}{rgb}{0.501961,0.501961,0.501961}%
\pgfsetfillcolor{currentfill}%
\pgfsetfillopacity{0.700000}%
\pgfsetlinewidth{0.000000pt}%
\definecolor{currentstroke}{rgb}{0.000000,0.000000,0.000000}%
\pgfsetstrokecolor{currentstroke}%
\pgfsetstrokeopacity{0.700000}%
\pgfsetdash{}{0pt}%
\pgfpathmoveto{\pgfqpoint{3.143466in}{0.550000in}}%
\pgfpathlineto{\pgfqpoint{3.170987in}{0.550000in}}%
\pgfpathlineto{\pgfqpoint{3.170987in}{0.702650in}}%
\pgfpathlineto{\pgfqpoint{3.143466in}{0.702650in}}%
\pgfpathlineto{\pgfqpoint{3.143466in}{0.550000in}}%
\pgfpathclose%
\pgfusepath{fill}%
\end{pgfscope}%
\begin{pgfscope}%
\pgfpathrectangle{\pgfqpoint{1.250000in}{0.550000in}}{\pgfqpoint{7.750000in}{3.850000in}}%
\pgfusepath{clip}%
\pgfsetbuttcap%
\pgfsetmiterjoin%
\definecolor{currentfill}{rgb}{0.501961,0.501961,0.501961}%
\pgfsetfillcolor{currentfill}%
\pgfsetfillopacity{0.700000}%
\pgfsetlinewidth{0.000000pt}%
\definecolor{currentstroke}{rgb}{0.000000,0.000000,0.000000}%
\pgfsetstrokecolor{currentstroke}%
\pgfsetstrokeopacity{0.700000}%
\pgfsetdash{}{0pt}%
\pgfpathmoveto{\pgfqpoint{3.170987in}{0.550000in}}%
\pgfpathlineto{\pgfqpoint{3.198509in}{0.550000in}}%
\pgfpathlineto{\pgfqpoint{3.198509in}{0.733731in}}%
\pgfpathlineto{\pgfqpoint{3.170987in}{0.733731in}}%
\pgfpathlineto{\pgfqpoint{3.170987in}{0.550000in}}%
\pgfpathclose%
\pgfusepath{fill}%
\end{pgfscope}%
\begin{pgfscope}%
\pgfpathrectangle{\pgfqpoint{1.250000in}{0.550000in}}{\pgfqpoint{7.750000in}{3.850000in}}%
\pgfusepath{clip}%
\pgfsetbuttcap%
\pgfsetmiterjoin%
\definecolor{currentfill}{rgb}{0.501961,0.501961,0.501961}%
\pgfsetfillcolor{currentfill}%
\pgfsetfillopacity{0.700000}%
\pgfsetlinewidth{0.000000pt}%
\definecolor{currentstroke}{rgb}{0.000000,0.000000,0.000000}%
\pgfsetstrokecolor{currentstroke}%
\pgfsetstrokeopacity{0.700000}%
\pgfsetdash{}{0pt}%
\pgfpathmoveto{\pgfqpoint{3.198509in}{0.550000in}}%
\pgfpathlineto{\pgfqpoint{3.226030in}{0.550000in}}%
\pgfpathlineto{\pgfqpoint{3.226030in}{0.702191in}}%
\pgfpathlineto{\pgfqpoint{3.198509in}{0.702191in}}%
\pgfpathlineto{\pgfqpoint{3.198509in}{0.550000in}}%
\pgfpathclose%
\pgfusepath{fill}%
\end{pgfscope}%
\begin{pgfscope}%
\pgfpathrectangle{\pgfqpoint{1.250000in}{0.550000in}}{\pgfqpoint{7.750000in}{3.850000in}}%
\pgfusepath{clip}%
\pgfsetbuttcap%
\pgfsetmiterjoin%
\definecolor{currentfill}{rgb}{0.501961,0.501961,0.501961}%
\pgfsetfillcolor{currentfill}%
\pgfsetfillopacity{0.700000}%
\pgfsetlinewidth{0.000000pt}%
\definecolor{currentstroke}{rgb}{0.000000,0.000000,0.000000}%
\pgfsetstrokecolor{currentstroke}%
\pgfsetstrokeopacity{0.700000}%
\pgfsetdash{}{0pt}%
\pgfpathmoveto{\pgfqpoint{3.226030in}{0.550000in}}%
\pgfpathlineto{\pgfqpoint{3.253551in}{0.550000in}}%
\pgfpathlineto{\pgfqpoint{3.253551in}{0.725923in}}%
\pgfpathlineto{\pgfqpoint{3.226030in}{0.725923in}}%
\pgfpathlineto{\pgfqpoint{3.226030in}{0.550000in}}%
\pgfpathclose%
\pgfusepath{fill}%
\end{pgfscope}%
\begin{pgfscope}%
\pgfpathrectangle{\pgfqpoint{1.250000in}{0.550000in}}{\pgfqpoint{7.750000in}{3.850000in}}%
\pgfusepath{clip}%
\pgfsetbuttcap%
\pgfsetmiterjoin%
\definecolor{currentfill}{rgb}{0.501961,0.501961,0.501961}%
\pgfsetfillcolor{currentfill}%
\pgfsetfillopacity{0.700000}%
\pgfsetlinewidth{0.000000pt}%
\definecolor{currentstroke}{rgb}{0.000000,0.000000,0.000000}%
\pgfsetstrokecolor{currentstroke}%
\pgfsetstrokeopacity{0.700000}%
\pgfsetdash{}{0pt}%
\pgfpathmoveto{\pgfqpoint{3.253551in}{0.550000in}}%
\pgfpathlineto{\pgfqpoint{3.281072in}{0.550000in}}%
\pgfpathlineto{\pgfqpoint{3.281072in}{0.718727in}}%
\pgfpathlineto{\pgfqpoint{3.253551in}{0.718727in}}%
\pgfpathlineto{\pgfqpoint{3.253551in}{0.550000in}}%
\pgfpathclose%
\pgfusepath{fill}%
\end{pgfscope}%
\begin{pgfscope}%
\pgfpathrectangle{\pgfqpoint{1.250000in}{0.550000in}}{\pgfqpoint{7.750000in}{3.850000in}}%
\pgfusepath{clip}%
\pgfsetbuttcap%
\pgfsetmiterjoin%
\definecolor{currentfill}{rgb}{0.501961,0.501961,0.501961}%
\pgfsetfillcolor{currentfill}%
\pgfsetfillopacity{0.700000}%
\pgfsetlinewidth{0.000000pt}%
\definecolor{currentstroke}{rgb}{0.000000,0.000000,0.000000}%
\pgfsetstrokecolor{currentstroke}%
\pgfsetstrokeopacity{0.700000}%
\pgfsetdash{}{0pt}%
\pgfpathmoveto{\pgfqpoint{3.281072in}{0.550000in}}%
\pgfpathlineto{\pgfqpoint{3.308594in}{0.550000in}}%
\pgfpathlineto{\pgfqpoint{3.308594in}{0.711531in}}%
\pgfpathlineto{\pgfqpoint{3.281072in}{0.711531in}}%
\pgfpathlineto{\pgfqpoint{3.281072in}{0.550000in}}%
\pgfpathclose%
\pgfusepath{fill}%
\end{pgfscope}%
\begin{pgfscope}%
\pgfpathrectangle{\pgfqpoint{1.250000in}{0.550000in}}{\pgfqpoint{7.750000in}{3.850000in}}%
\pgfusepath{clip}%
\pgfsetbuttcap%
\pgfsetmiterjoin%
\definecolor{currentfill}{rgb}{0.501961,0.501961,0.501961}%
\pgfsetfillcolor{currentfill}%
\pgfsetfillopacity{0.700000}%
\pgfsetlinewidth{0.000000pt}%
\definecolor{currentstroke}{rgb}{0.000000,0.000000,0.000000}%
\pgfsetstrokecolor{currentstroke}%
\pgfsetstrokeopacity{0.700000}%
\pgfsetdash{}{0pt}%
\pgfpathmoveto{\pgfqpoint{3.308594in}{0.550000in}}%
\pgfpathlineto{\pgfqpoint{3.336115in}{0.550000in}}%
\pgfpathlineto{\pgfqpoint{3.336115in}{0.681674in}}%
\pgfpathlineto{\pgfqpoint{3.308594in}{0.681674in}}%
\pgfpathlineto{\pgfqpoint{3.308594in}{0.550000in}}%
\pgfpathclose%
\pgfusepath{fill}%
\end{pgfscope}%
\begin{pgfscope}%
\pgfpathrectangle{\pgfqpoint{1.250000in}{0.550000in}}{\pgfqpoint{7.750000in}{3.850000in}}%
\pgfusepath{clip}%
\pgfsetbuttcap%
\pgfsetmiterjoin%
\definecolor{currentfill}{rgb}{0.501961,0.501961,0.501961}%
\pgfsetfillcolor{currentfill}%
\pgfsetfillopacity{0.700000}%
\pgfsetlinewidth{0.000000pt}%
\definecolor{currentstroke}{rgb}{0.000000,0.000000,0.000000}%
\pgfsetstrokecolor{currentstroke}%
\pgfsetstrokeopacity{0.700000}%
\pgfsetdash{}{0pt}%
\pgfpathmoveto{\pgfqpoint{3.336115in}{0.550000in}}%
\pgfpathlineto{\pgfqpoint{3.363636in}{0.550000in}}%
\pgfpathlineto{\pgfqpoint{3.363636in}{0.704641in}}%
\pgfpathlineto{\pgfqpoint{3.336115in}{0.704641in}}%
\pgfpathlineto{\pgfqpoint{3.336115in}{0.550000in}}%
\pgfpathclose%
\pgfusepath{fill}%
\end{pgfscope}%
\begin{pgfscope}%
\pgfpathrectangle{\pgfqpoint{1.250000in}{0.550000in}}{\pgfqpoint{7.750000in}{3.850000in}}%
\pgfusepath{clip}%
\pgfsetbuttcap%
\pgfsetmiterjoin%
\definecolor{currentfill}{rgb}{0.501961,0.501961,0.501961}%
\pgfsetfillcolor{currentfill}%
\pgfsetfillopacity{0.700000}%
\pgfsetlinewidth{0.000000pt}%
\definecolor{currentstroke}{rgb}{0.000000,0.000000,0.000000}%
\pgfsetstrokecolor{currentstroke}%
\pgfsetstrokeopacity{0.700000}%
\pgfsetdash{}{0pt}%
\pgfpathmoveto{\pgfqpoint{3.363636in}{0.550000in}}%
\pgfpathlineto{\pgfqpoint{3.391158in}{0.550000in}}%
\pgfpathlineto{\pgfqpoint{3.391158in}{0.689483in}}%
\pgfpathlineto{\pgfqpoint{3.363636in}{0.689483in}}%
\pgfpathlineto{\pgfqpoint{3.363636in}{0.550000in}}%
\pgfpathclose%
\pgfusepath{fill}%
\end{pgfscope}%
\begin{pgfscope}%
\pgfpathrectangle{\pgfqpoint{1.250000in}{0.550000in}}{\pgfqpoint{7.750000in}{3.850000in}}%
\pgfusepath{clip}%
\pgfsetbuttcap%
\pgfsetmiterjoin%
\definecolor{currentfill}{rgb}{0.501961,0.501961,0.501961}%
\pgfsetfillcolor{currentfill}%
\pgfsetfillopacity{0.700000}%
\pgfsetlinewidth{0.000000pt}%
\definecolor{currentstroke}{rgb}{0.000000,0.000000,0.000000}%
\pgfsetstrokecolor{currentstroke}%
\pgfsetstrokeopacity{0.700000}%
\pgfsetdash{}{0pt}%
\pgfpathmoveto{\pgfqpoint{3.391158in}{0.550000in}}%
\pgfpathlineto{\pgfqpoint{3.418679in}{0.550000in}}%
\pgfpathlineto{\pgfqpoint{3.418679in}{0.674937in}}%
\pgfpathlineto{\pgfqpoint{3.391158in}{0.674937in}}%
\pgfpathlineto{\pgfqpoint{3.391158in}{0.550000in}}%
\pgfpathclose%
\pgfusepath{fill}%
\end{pgfscope}%
\begin{pgfscope}%
\pgfpathrectangle{\pgfqpoint{1.250000in}{0.550000in}}{\pgfqpoint{7.750000in}{3.850000in}}%
\pgfusepath{clip}%
\pgfsetbuttcap%
\pgfsetmiterjoin%
\definecolor{currentfill}{rgb}{0.501961,0.501961,0.501961}%
\pgfsetfillcolor{currentfill}%
\pgfsetfillopacity{0.700000}%
\pgfsetlinewidth{0.000000pt}%
\definecolor{currentstroke}{rgb}{0.000000,0.000000,0.000000}%
\pgfsetstrokecolor{currentstroke}%
\pgfsetstrokeopacity{0.700000}%
\pgfsetdash{}{0pt}%
\pgfpathmoveto{\pgfqpoint{3.418679in}{0.550000in}}%
\pgfpathlineto{\pgfqpoint{3.446200in}{0.550000in}}%
\pgfpathlineto{\pgfqpoint{3.446200in}{0.669272in}}%
\pgfpathlineto{\pgfqpoint{3.418679in}{0.669272in}}%
\pgfpathlineto{\pgfqpoint{3.418679in}{0.550000in}}%
\pgfpathclose%
\pgfusepath{fill}%
\end{pgfscope}%
\begin{pgfscope}%
\pgfpathrectangle{\pgfqpoint{1.250000in}{0.550000in}}{\pgfqpoint{7.750000in}{3.850000in}}%
\pgfusepath{clip}%
\pgfsetbuttcap%
\pgfsetmiterjoin%
\definecolor{currentfill}{rgb}{0.501961,0.501961,0.501961}%
\pgfsetfillcolor{currentfill}%
\pgfsetfillopacity{0.700000}%
\pgfsetlinewidth{0.000000pt}%
\definecolor{currentstroke}{rgb}{0.000000,0.000000,0.000000}%
\pgfsetstrokecolor{currentstroke}%
\pgfsetstrokeopacity{0.700000}%
\pgfsetdash{}{0pt}%
\pgfpathmoveto{\pgfqpoint{3.446200in}{0.550000in}}%
\pgfpathlineto{\pgfqpoint{3.473722in}{0.550000in}}%
\pgfpathlineto{\pgfqpoint{3.473722in}{0.671110in}}%
\pgfpathlineto{\pgfqpoint{3.446200in}{0.671110in}}%
\pgfpathlineto{\pgfqpoint{3.446200in}{0.550000in}}%
\pgfpathclose%
\pgfusepath{fill}%
\end{pgfscope}%
\begin{pgfscope}%
\pgfpathrectangle{\pgfqpoint{1.250000in}{0.550000in}}{\pgfqpoint{7.750000in}{3.850000in}}%
\pgfusepath{clip}%
\pgfsetbuttcap%
\pgfsetmiterjoin%
\definecolor{currentfill}{rgb}{0.501961,0.501961,0.501961}%
\pgfsetfillcolor{currentfill}%
\pgfsetfillopacity{0.700000}%
\pgfsetlinewidth{0.000000pt}%
\definecolor{currentstroke}{rgb}{0.000000,0.000000,0.000000}%
\pgfsetstrokecolor{currentstroke}%
\pgfsetstrokeopacity{0.700000}%
\pgfsetdash{}{0pt}%
\pgfpathmoveto{\pgfqpoint{3.473722in}{0.550000in}}%
\pgfpathlineto{\pgfqpoint{3.501243in}{0.550000in}}%
\pgfpathlineto{\pgfqpoint{3.501243in}{0.673100in}}%
\pgfpathlineto{\pgfqpoint{3.473722in}{0.673100in}}%
\pgfpathlineto{\pgfqpoint{3.473722in}{0.550000in}}%
\pgfpathclose%
\pgfusepath{fill}%
\end{pgfscope}%
\begin{pgfscope}%
\pgfpathrectangle{\pgfqpoint{1.250000in}{0.550000in}}{\pgfqpoint{7.750000in}{3.850000in}}%
\pgfusepath{clip}%
\pgfsetbuttcap%
\pgfsetmiterjoin%
\definecolor{currentfill}{rgb}{0.501961,0.501961,0.501961}%
\pgfsetfillcolor{currentfill}%
\pgfsetfillopacity{0.700000}%
\pgfsetlinewidth{0.000000pt}%
\definecolor{currentstroke}{rgb}{0.000000,0.000000,0.000000}%
\pgfsetstrokecolor{currentstroke}%
\pgfsetstrokeopacity{0.700000}%
\pgfsetdash{}{0pt}%
\pgfpathmoveto{\pgfqpoint{3.501243in}{0.550000in}}%
\pgfpathlineto{\pgfqpoint{3.528764in}{0.550000in}}%
\pgfpathlineto{\pgfqpoint{3.528764in}{0.683358in}}%
\pgfpathlineto{\pgfqpoint{3.501243in}{0.683358in}}%
\pgfpathlineto{\pgfqpoint{3.501243in}{0.550000in}}%
\pgfpathclose%
\pgfusepath{fill}%
\end{pgfscope}%
\begin{pgfscope}%
\pgfpathrectangle{\pgfqpoint{1.250000in}{0.550000in}}{\pgfqpoint{7.750000in}{3.850000in}}%
\pgfusepath{clip}%
\pgfsetbuttcap%
\pgfsetmiterjoin%
\definecolor{currentfill}{rgb}{0.501961,0.501961,0.501961}%
\pgfsetfillcolor{currentfill}%
\pgfsetfillopacity{0.700000}%
\pgfsetlinewidth{0.000000pt}%
\definecolor{currentstroke}{rgb}{0.000000,0.000000,0.000000}%
\pgfsetstrokecolor{currentstroke}%
\pgfsetstrokeopacity{0.700000}%
\pgfsetdash{}{0pt}%
\pgfpathmoveto{\pgfqpoint{3.528764in}{0.550000in}}%
\pgfpathlineto{\pgfqpoint{3.556286in}{0.550000in}}%
\pgfpathlineto{\pgfqpoint{3.556286in}{0.668201in}}%
\pgfpathlineto{\pgfqpoint{3.528764in}{0.668201in}}%
\pgfpathlineto{\pgfqpoint{3.528764in}{0.550000in}}%
\pgfpathclose%
\pgfusepath{fill}%
\end{pgfscope}%
\begin{pgfscope}%
\pgfpathrectangle{\pgfqpoint{1.250000in}{0.550000in}}{\pgfqpoint{7.750000in}{3.850000in}}%
\pgfusepath{clip}%
\pgfsetbuttcap%
\pgfsetmiterjoin%
\definecolor{currentfill}{rgb}{0.501961,0.501961,0.501961}%
\pgfsetfillcolor{currentfill}%
\pgfsetfillopacity{0.700000}%
\pgfsetlinewidth{0.000000pt}%
\definecolor{currentstroke}{rgb}{0.000000,0.000000,0.000000}%
\pgfsetstrokecolor{currentstroke}%
\pgfsetstrokeopacity{0.700000}%
\pgfsetdash{}{0pt}%
\pgfpathmoveto{\pgfqpoint{3.556286in}{0.550000in}}%
\pgfpathlineto{\pgfqpoint{3.583807in}{0.550000in}}%
\pgfpathlineto{\pgfqpoint{3.583807in}{0.653655in}}%
\pgfpathlineto{\pgfqpoint{3.556286in}{0.653655in}}%
\pgfpathlineto{\pgfqpoint{3.556286in}{0.550000in}}%
\pgfpathclose%
\pgfusepath{fill}%
\end{pgfscope}%
\begin{pgfscope}%
\pgfpathrectangle{\pgfqpoint{1.250000in}{0.550000in}}{\pgfqpoint{7.750000in}{3.850000in}}%
\pgfusepath{clip}%
\pgfsetbuttcap%
\pgfsetmiterjoin%
\definecolor{currentfill}{rgb}{0.501961,0.501961,0.501961}%
\pgfsetfillcolor{currentfill}%
\pgfsetfillopacity{0.700000}%
\pgfsetlinewidth{0.000000pt}%
\definecolor{currentstroke}{rgb}{0.000000,0.000000,0.000000}%
\pgfsetstrokecolor{currentstroke}%
\pgfsetstrokeopacity{0.700000}%
\pgfsetdash{}{0pt}%
\pgfpathmoveto{\pgfqpoint{3.583807in}{0.550000in}}%
\pgfpathlineto{\pgfqpoint{3.611328in}{0.550000in}}%
\pgfpathlineto{\pgfqpoint{3.611328in}{0.664067in}}%
\pgfpathlineto{\pgfqpoint{3.583807in}{0.664067in}}%
\pgfpathlineto{\pgfqpoint{3.583807in}{0.550000in}}%
\pgfpathclose%
\pgfusepath{fill}%
\end{pgfscope}%
\begin{pgfscope}%
\pgfpathrectangle{\pgfqpoint{1.250000in}{0.550000in}}{\pgfqpoint{7.750000in}{3.850000in}}%
\pgfusepath{clip}%
\pgfsetbuttcap%
\pgfsetmiterjoin%
\definecolor{currentfill}{rgb}{0.501961,0.501961,0.501961}%
\pgfsetfillcolor{currentfill}%
\pgfsetfillopacity{0.700000}%
\pgfsetlinewidth{0.000000pt}%
\definecolor{currentstroke}{rgb}{0.000000,0.000000,0.000000}%
\pgfsetstrokecolor{currentstroke}%
\pgfsetstrokeopacity{0.700000}%
\pgfsetdash{}{0pt}%
\pgfpathmoveto{\pgfqpoint{3.611328in}{0.550000in}}%
\pgfpathlineto{\pgfqpoint{3.638849in}{0.550000in}}%
\pgfpathlineto{\pgfqpoint{3.638849in}{0.652124in}}%
\pgfpathlineto{\pgfqpoint{3.611328in}{0.652124in}}%
\pgfpathlineto{\pgfqpoint{3.611328in}{0.550000in}}%
\pgfpathclose%
\pgfusepath{fill}%
\end{pgfscope}%
\begin{pgfscope}%
\pgfpathrectangle{\pgfqpoint{1.250000in}{0.550000in}}{\pgfqpoint{7.750000in}{3.850000in}}%
\pgfusepath{clip}%
\pgfsetbuttcap%
\pgfsetmiterjoin%
\definecolor{currentfill}{rgb}{0.501961,0.501961,0.501961}%
\pgfsetfillcolor{currentfill}%
\pgfsetfillopacity{0.700000}%
\pgfsetlinewidth{0.000000pt}%
\definecolor{currentstroke}{rgb}{0.000000,0.000000,0.000000}%
\pgfsetstrokecolor{currentstroke}%
\pgfsetstrokeopacity{0.700000}%
\pgfsetdash{}{0pt}%
\pgfpathmoveto{\pgfqpoint{3.638849in}{0.550000in}}%
\pgfpathlineto{\pgfqpoint{3.666371in}{0.550000in}}%
\pgfpathlineto{\pgfqpoint{3.666371in}{0.657942in}}%
\pgfpathlineto{\pgfqpoint{3.638849in}{0.657942in}}%
\pgfpathlineto{\pgfqpoint{3.638849in}{0.550000in}}%
\pgfpathclose%
\pgfusepath{fill}%
\end{pgfscope}%
\begin{pgfscope}%
\pgfpathrectangle{\pgfqpoint{1.250000in}{0.550000in}}{\pgfqpoint{7.750000in}{3.850000in}}%
\pgfusepath{clip}%
\pgfsetbuttcap%
\pgfsetmiterjoin%
\definecolor{currentfill}{rgb}{0.501961,0.501961,0.501961}%
\pgfsetfillcolor{currentfill}%
\pgfsetfillopacity{0.700000}%
\pgfsetlinewidth{0.000000pt}%
\definecolor{currentstroke}{rgb}{0.000000,0.000000,0.000000}%
\pgfsetstrokecolor{currentstroke}%
\pgfsetstrokeopacity{0.700000}%
\pgfsetdash{}{0pt}%
\pgfpathmoveto{\pgfqpoint{3.666371in}{0.550000in}}%
\pgfpathlineto{\pgfqpoint{3.693892in}{0.550000in}}%
\pgfpathlineto{\pgfqpoint{3.693892in}{0.644009in}}%
\pgfpathlineto{\pgfqpoint{3.666371in}{0.644009in}}%
\pgfpathlineto{\pgfqpoint{3.666371in}{0.550000in}}%
\pgfpathclose%
\pgfusepath{fill}%
\end{pgfscope}%
\begin{pgfscope}%
\pgfpathrectangle{\pgfqpoint{1.250000in}{0.550000in}}{\pgfqpoint{7.750000in}{3.850000in}}%
\pgfusepath{clip}%
\pgfsetbuttcap%
\pgfsetmiterjoin%
\definecolor{currentfill}{rgb}{0.501961,0.501961,0.501961}%
\pgfsetfillcolor{currentfill}%
\pgfsetfillopacity{0.700000}%
\pgfsetlinewidth{0.000000pt}%
\definecolor{currentstroke}{rgb}{0.000000,0.000000,0.000000}%
\pgfsetstrokecolor{currentstroke}%
\pgfsetstrokeopacity{0.700000}%
\pgfsetdash{}{0pt}%
\pgfpathmoveto{\pgfqpoint{3.693892in}{0.550000in}}%
\pgfpathlineto{\pgfqpoint{3.721413in}{0.550000in}}%
\pgfpathlineto{\pgfqpoint{3.721413in}{0.650593in}}%
\pgfpathlineto{\pgfqpoint{3.693892in}{0.650593in}}%
\pgfpathlineto{\pgfqpoint{3.693892in}{0.550000in}}%
\pgfpathclose%
\pgfusepath{fill}%
\end{pgfscope}%
\begin{pgfscope}%
\pgfpathrectangle{\pgfqpoint{1.250000in}{0.550000in}}{\pgfqpoint{7.750000in}{3.850000in}}%
\pgfusepath{clip}%
\pgfsetbuttcap%
\pgfsetmiterjoin%
\definecolor{currentfill}{rgb}{0.501961,0.501961,0.501961}%
\pgfsetfillcolor{currentfill}%
\pgfsetfillopacity{0.700000}%
\pgfsetlinewidth{0.000000pt}%
\definecolor{currentstroke}{rgb}{0.000000,0.000000,0.000000}%
\pgfsetstrokecolor{currentstroke}%
\pgfsetstrokeopacity{0.700000}%
\pgfsetdash{}{0pt}%
\pgfpathmoveto{\pgfqpoint{3.721413in}{0.550000in}}%
\pgfpathlineto{\pgfqpoint{3.748935in}{0.550000in}}%
\pgfpathlineto{\pgfqpoint{3.748935in}{0.640641in}}%
\pgfpathlineto{\pgfqpoint{3.721413in}{0.640641in}}%
\pgfpathlineto{\pgfqpoint{3.721413in}{0.550000in}}%
\pgfpathclose%
\pgfusepath{fill}%
\end{pgfscope}%
\begin{pgfscope}%
\pgfpathrectangle{\pgfqpoint{1.250000in}{0.550000in}}{\pgfqpoint{7.750000in}{3.850000in}}%
\pgfusepath{clip}%
\pgfsetbuttcap%
\pgfsetmiterjoin%
\definecolor{currentfill}{rgb}{0.501961,0.501961,0.501961}%
\pgfsetfillcolor{currentfill}%
\pgfsetfillopacity{0.700000}%
\pgfsetlinewidth{0.000000pt}%
\definecolor{currentstroke}{rgb}{0.000000,0.000000,0.000000}%
\pgfsetstrokecolor{currentstroke}%
\pgfsetstrokeopacity{0.700000}%
\pgfsetdash{}{0pt}%
\pgfpathmoveto{\pgfqpoint{3.748935in}{0.550000in}}%
\pgfpathlineto{\pgfqpoint{3.776456in}{0.550000in}}%
\pgfpathlineto{\pgfqpoint{3.776456in}{0.645387in}}%
\pgfpathlineto{\pgfqpoint{3.748935in}{0.645387in}}%
\pgfpathlineto{\pgfqpoint{3.748935in}{0.550000in}}%
\pgfpathclose%
\pgfusepath{fill}%
\end{pgfscope}%
\begin{pgfscope}%
\pgfpathrectangle{\pgfqpoint{1.250000in}{0.550000in}}{\pgfqpoint{7.750000in}{3.850000in}}%
\pgfusepath{clip}%
\pgfsetbuttcap%
\pgfsetmiterjoin%
\definecolor{currentfill}{rgb}{0.501961,0.501961,0.501961}%
\pgfsetfillcolor{currentfill}%
\pgfsetfillopacity{0.700000}%
\pgfsetlinewidth{0.000000pt}%
\definecolor{currentstroke}{rgb}{0.000000,0.000000,0.000000}%
\pgfsetstrokecolor{currentstroke}%
\pgfsetstrokeopacity{0.700000}%
\pgfsetdash{}{0pt}%
\pgfpathmoveto{\pgfqpoint{3.776456in}{0.550000in}}%
\pgfpathlineto{\pgfqpoint{3.803977in}{0.550000in}}%
\pgfpathlineto{\pgfqpoint{3.803977in}{0.632373in}}%
\pgfpathlineto{\pgfqpoint{3.776456in}{0.632373in}}%
\pgfpathlineto{\pgfqpoint{3.776456in}{0.550000in}}%
\pgfpathclose%
\pgfusepath{fill}%
\end{pgfscope}%
\begin{pgfscope}%
\pgfpathrectangle{\pgfqpoint{1.250000in}{0.550000in}}{\pgfqpoint{7.750000in}{3.850000in}}%
\pgfusepath{clip}%
\pgfsetbuttcap%
\pgfsetmiterjoin%
\definecolor{currentfill}{rgb}{0.501961,0.501961,0.501961}%
\pgfsetfillcolor{currentfill}%
\pgfsetfillopacity{0.700000}%
\pgfsetlinewidth{0.000000pt}%
\definecolor{currentstroke}{rgb}{0.000000,0.000000,0.000000}%
\pgfsetstrokecolor{currentstroke}%
\pgfsetstrokeopacity{0.700000}%
\pgfsetdash{}{0pt}%
\pgfpathmoveto{\pgfqpoint{3.803977in}{0.550000in}}%
\pgfpathlineto{\pgfqpoint{3.831499in}{0.550000in}}%
\pgfpathlineto{\pgfqpoint{3.831499in}{0.642019in}}%
\pgfpathlineto{\pgfqpoint{3.803977in}{0.642019in}}%
\pgfpathlineto{\pgfqpoint{3.803977in}{0.550000in}}%
\pgfpathclose%
\pgfusepath{fill}%
\end{pgfscope}%
\begin{pgfscope}%
\pgfpathrectangle{\pgfqpoint{1.250000in}{0.550000in}}{\pgfqpoint{7.750000in}{3.850000in}}%
\pgfusepath{clip}%
\pgfsetbuttcap%
\pgfsetmiterjoin%
\definecolor{currentfill}{rgb}{0.501961,0.501961,0.501961}%
\pgfsetfillcolor{currentfill}%
\pgfsetfillopacity{0.700000}%
\pgfsetlinewidth{0.000000pt}%
\definecolor{currentstroke}{rgb}{0.000000,0.000000,0.000000}%
\pgfsetstrokecolor{currentstroke}%
\pgfsetstrokeopacity{0.700000}%
\pgfsetdash{}{0pt}%
\pgfpathmoveto{\pgfqpoint{3.831499in}{0.550000in}}%
\pgfpathlineto{\pgfqpoint{3.859020in}{0.550000in}}%
\pgfpathlineto{\pgfqpoint{3.859020in}{0.634976in}}%
\pgfpathlineto{\pgfqpoint{3.831499in}{0.634976in}}%
\pgfpathlineto{\pgfqpoint{3.831499in}{0.550000in}}%
\pgfpathclose%
\pgfusepath{fill}%
\end{pgfscope}%
\begin{pgfscope}%
\pgfpathrectangle{\pgfqpoint{1.250000in}{0.550000in}}{\pgfqpoint{7.750000in}{3.850000in}}%
\pgfusepath{clip}%
\pgfsetbuttcap%
\pgfsetmiterjoin%
\definecolor{currentfill}{rgb}{0.501961,0.501961,0.501961}%
\pgfsetfillcolor{currentfill}%
\pgfsetfillopacity{0.700000}%
\pgfsetlinewidth{0.000000pt}%
\definecolor{currentstroke}{rgb}{0.000000,0.000000,0.000000}%
\pgfsetstrokecolor{currentstroke}%
\pgfsetstrokeopacity{0.700000}%
\pgfsetdash{}{0pt}%
\pgfpathmoveto{\pgfqpoint{3.859020in}{0.550000in}}%
\pgfpathlineto{\pgfqpoint{3.886541in}{0.550000in}}%
\pgfpathlineto{\pgfqpoint{3.886541in}{0.641559in}}%
\pgfpathlineto{\pgfqpoint{3.859020in}{0.641559in}}%
\pgfpathlineto{\pgfqpoint{3.859020in}{0.550000in}}%
\pgfpathclose%
\pgfusepath{fill}%
\end{pgfscope}%
\begin{pgfscope}%
\pgfpathrectangle{\pgfqpoint{1.250000in}{0.550000in}}{\pgfqpoint{7.750000in}{3.850000in}}%
\pgfusepath{clip}%
\pgfsetbuttcap%
\pgfsetmiterjoin%
\definecolor{currentfill}{rgb}{0.501961,0.501961,0.501961}%
\pgfsetfillcolor{currentfill}%
\pgfsetfillopacity{0.700000}%
\pgfsetlinewidth{0.000000pt}%
\definecolor{currentstroke}{rgb}{0.000000,0.000000,0.000000}%
\pgfsetstrokecolor{currentstroke}%
\pgfsetstrokeopacity{0.700000}%
\pgfsetdash{}{0pt}%
\pgfpathmoveto{\pgfqpoint{3.886541in}{0.550000in}}%
\pgfpathlineto{\pgfqpoint{3.914062in}{0.550000in}}%
\pgfpathlineto{\pgfqpoint{3.914062in}{0.634363in}}%
\pgfpathlineto{\pgfqpoint{3.886541in}{0.634363in}}%
\pgfpathlineto{\pgfqpoint{3.886541in}{0.550000in}}%
\pgfpathclose%
\pgfusepath{fill}%
\end{pgfscope}%
\begin{pgfscope}%
\pgfpathrectangle{\pgfqpoint{1.250000in}{0.550000in}}{\pgfqpoint{7.750000in}{3.850000in}}%
\pgfusepath{clip}%
\pgfsetbuttcap%
\pgfsetmiterjoin%
\definecolor{currentfill}{rgb}{0.501961,0.501961,0.501961}%
\pgfsetfillcolor{currentfill}%
\pgfsetfillopacity{0.700000}%
\pgfsetlinewidth{0.000000pt}%
\definecolor{currentstroke}{rgb}{0.000000,0.000000,0.000000}%
\pgfsetstrokecolor{currentstroke}%
\pgfsetstrokeopacity{0.700000}%
\pgfsetdash{}{0pt}%
\pgfpathmoveto{\pgfqpoint{3.914062in}{0.550000in}}%
\pgfpathlineto{\pgfqpoint{3.941584in}{0.550000in}}%
\pgfpathlineto{\pgfqpoint{3.941584in}{0.632985in}}%
\pgfpathlineto{\pgfqpoint{3.914062in}{0.632985in}}%
\pgfpathlineto{\pgfqpoint{3.914062in}{0.550000in}}%
\pgfpathclose%
\pgfusepath{fill}%
\end{pgfscope}%
\begin{pgfscope}%
\pgfpathrectangle{\pgfqpoint{1.250000in}{0.550000in}}{\pgfqpoint{7.750000in}{3.850000in}}%
\pgfusepath{clip}%
\pgfsetbuttcap%
\pgfsetmiterjoin%
\definecolor{currentfill}{rgb}{0.501961,0.501961,0.501961}%
\pgfsetfillcolor{currentfill}%
\pgfsetfillopacity{0.700000}%
\pgfsetlinewidth{0.000000pt}%
\definecolor{currentstroke}{rgb}{0.000000,0.000000,0.000000}%
\pgfsetstrokecolor{currentstroke}%
\pgfsetstrokeopacity{0.700000}%
\pgfsetdash{}{0pt}%
\pgfpathmoveto{\pgfqpoint{3.941584in}{0.550000in}}%
\pgfpathlineto{\pgfqpoint{3.969105in}{0.550000in}}%
\pgfpathlineto{\pgfqpoint{3.969105in}{0.628545in}}%
\pgfpathlineto{\pgfqpoint{3.941584in}{0.628545in}}%
\pgfpathlineto{\pgfqpoint{3.941584in}{0.550000in}}%
\pgfpathclose%
\pgfusepath{fill}%
\end{pgfscope}%
\begin{pgfscope}%
\pgfpathrectangle{\pgfqpoint{1.250000in}{0.550000in}}{\pgfqpoint{7.750000in}{3.850000in}}%
\pgfusepath{clip}%
\pgfsetbuttcap%
\pgfsetmiterjoin%
\definecolor{currentfill}{rgb}{0.501961,0.501961,0.501961}%
\pgfsetfillcolor{currentfill}%
\pgfsetfillopacity{0.700000}%
\pgfsetlinewidth{0.000000pt}%
\definecolor{currentstroke}{rgb}{0.000000,0.000000,0.000000}%
\pgfsetstrokecolor{currentstroke}%
\pgfsetstrokeopacity{0.700000}%
\pgfsetdash{}{0pt}%
\pgfpathmoveto{\pgfqpoint{3.969105in}{0.550000in}}%
\pgfpathlineto{\pgfqpoint{3.996626in}{0.550000in}}%
\pgfpathlineto{\pgfqpoint{3.996626in}{0.637426in}}%
\pgfpathlineto{\pgfqpoint{3.969105in}{0.637426in}}%
\pgfpathlineto{\pgfqpoint{3.969105in}{0.550000in}}%
\pgfpathclose%
\pgfusepath{fill}%
\end{pgfscope}%
\begin{pgfscope}%
\pgfpathrectangle{\pgfqpoint{1.250000in}{0.550000in}}{\pgfqpoint{7.750000in}{3.850000in}}%
\pgfusepath{clip}%
\pgfsetbuttcap%
\pgfsetmiterjoin%
\definecolor{currentfill}{rgb}{0.501961,0.501961,0.501961}%
\pgfsetfillcolor{currentfill}%
\pgfsetfillopacity{0.700000}%
\pgfsetlinewidth{0.000000pt}%
\definecolor{currentstroke}{rgb}{0.000000,0.000000,0.000000}%
\pgfsetstrokecolor{currentstroke}%
\pgfsetstrokeopacity{0.700000}%
\pgfsetdash{}{0pt}%
\pgfpathmoveto{\pgfqpoint{3.996626in}{0.550000in}}%
\pgfpathlineto{\pgfqpoint{4.024148in}{0.550000in}}%
\pgfpathlineto{\pgfqpoint{4.024148in}{0.624717in}}%
\pgfpathlineto{\pgfqpoint{3.996626in}{0.624717in}}%
\pgfpathlineto{\pgfqpoint{3.996626in}{0.550000in}}%
\pgfpathclose%
\pgfusepath{fill}%
\end{pgfscope}%
\begin{pgfscope}%
\pgfpathrectangle{\pgfqpoint{1.250000in}{0.550000in}}{\pgfqpoint{7.750000in}{3.850000in}}%
\pgfusepath{clip}%
\pgfsetbuttcap%
\pgfsetmiterjoin%
\definecolor{currentfill}{rgb}{0.501961,0.501961,0.501961}%
\pgfsetfillcolor{currentfill}%
\pgfsetfillopacity{0.700000}%
\pgfsetlinewidth{0.000000pt}%
\definecolor{currentstroke}{rgb}{0.000000,0.000000,0.000000}%
\pgfsetstrokecolor{currentstroke}%
\pgfsetstrokeopacity{0.700000}%
\pgfsetdash{}{0pt}%
\pgfpathmoveto{\pgfqpoint{4.024148in}{0.550000in}}%
\pgfpathlineto{\pgfqpoint{4.051669in}{0.550000in}}%
\pgfpathlineto{\pgfqpoint{4.051669in}{0.641866in}}%
\pgfpathlineto{\pgfqpoint{4.024148in}{0.641866in}}%
\pgfpathlineto{\pgfqpoint{4.024148in}{0.550000in}}%
\pgfpathclose%
\pgfusepath{fill}%
\end{pgfscope}%
\begin{pgfscope}%
\pgfpathrectangle{\pgfqpoint{1.250000in}{0.550000in}}{\pgfqpoint{7.750000in}{3.850000in}}%
\pgfusepath{clip}%
\pgfsetbuttcap%
\pgfsetmiterjoin%
\definecolor{currentfill}{rgb}{0.501961,0.501961,0.501961}%
\pgfsetfillcolor{currentfill}%
\pgfsetfillopacity{0.700000}%
\pgfsetlinewidth{0.000000pt}%
\definecolor{currentstroke}{rgb}{0.000000,0.000000,0.000000}%
\pgfsetstrokecolor{currentstroke}%
\pgfsetstrokeopacity{0.700000}%
\pgfsetdash{}{0pt}%
\pgfpathmoveto{\pgfqpoint{4.051669in}{0.550000in}}%
\pgfpathlineto{\pgfqpoint{4.079190in}{0.550000in}}%
\pgfpathlineto{\pgfqpoint{4.079190in}{0.629464in}}%
\pgfpathlineto{\pgfqpoint{4.051669in}{0.629464in}}%
\pgfpathlineto{\pgfqpoint{4.051669in}{0.550000in}}%
\pgfpathclose%
\pgfusepath{fill}%
\end{pgfscope}%
\begin{pgfscope}%
\pgfpathrectangle{\pgfqpoint{1.250000in}{0.550000in}}{\pgfqpoint{7.750000in}{3.850000in}}%
\pgfusepath{clip}%
\pgfsetbuttcap%
\pgfsetmiterjoin%
\definecolor{currentfill}{rgb}{0.501961,0.501961,0.501961}%
\pgfsetfillcolor{currentfill}%
\pgfsetfillopacity{0.700000}%
\pgfsetlinewidth{0.000000pt}%
\definecolor{currentstroke}{rgb}{0.000000,0.000000,0.000000}%
\pgfsetstrokecolor{currentstroke}%
\pgfsetstrokeopacity{0.700000}%
\pgfsetdash{}{0pt}%
\pgfpathmoveto{\pgfqpoint{4.079190in}{0.550000in}}%
\pgfpathlineto{\pgfqpoint{4.106712in}{0.550000in}}%
\pgfpathlineto{\pgfqpoint{4.106712in}{0.630382in}}%
\pgfpathlineto{\pgfqpoint{4.079190in}{0.630382in}}%
\pgfpathlineto{\pgfqpoint{4.079190in}{0.550000in}}%
\pgfpathclose%
\pgfusepath{fill}%
\end{pgfscope}%
\begin{pgfscope}%
\pgfpathrectangle{\pgfqpoint{1.250000in}{0.550000in}}{\pgfqpoint{7.750000in}{3.850000in}}%
\pgfusepath{clip}%
\pgfsetbuttcap%
\pgfsetmiterjoin%
\definecolor{currentfill}{rgb}{0.501961,0.501961,0.501961}%
\pgfsetfillcolor{currentfill}%
\pgfsetfillopacity{0.700000}%
\pgfsetlinewidth{0.000000pt}%
\definecolor{currentstroke}{rgb}{0.000000,0.000000,0.000000}%
\pgfsetstrokecolor{currentstroke}%
\pgfsetstrokeopacity{0.700000}%
\pgfsetdash{}{0pt}%
\pgfpathmoveto{\pgfqpoint{4.106712in}{0.550000in}}%
\pgfpathlineto{\pgfqpoint{4.134233in}{0.550000in}}%
\pgfpathlineto{\pgfqpoint{4.134233in}{0.628239in}}%
\pgfpathlineto{\pgfqpoint{4.106712in}{0.628239in}}%
\pgfpathlineto{\pgfqpoint{4.106712in}{0.550000in}}%
\pgfpathclose%
\pgfusepath{fill}%
\end{pgfscope}%
\begin{pgfscope}%
\pgfpathrectangle{\pgfqpoint{1.250000in}{0.550000in}}{\pgfqpoint{7.750000in}{3.850000in}}%
\pgfusepath{clip}%
\pgfsetbuttcap%
\pgfsetmiterjoin%
\definecolor{currentfill}{rgb}{0.501961,0.501961,0.501961}%
\pgfsetfillcolor{currentfill}%
\pgfsetfillopacity{0.700000}%
\pgfsetlinewidth{0.000000pt}%
\definecolor{currentstroke}{rgb}{0.000000,0.000000,0.000000}%
\pgfsetstrokecolor{currentstroke}%
\pgfsetstrokeopacity{0.700000}%
\pgfsetdash{}{0pt}%
\pgfpathmoveto{\pgfqpoint{4.134233in}{0.550000in}}%
\pgfpathlineto{\pgfqpoint{4.161754in}{0.550000in}}%
\pgfpathlineto{\pgfqpoint{4.161754in}{0.632985in}}%
\pgfpathlineto{\pgfqpoint{4.134233in}{0.632985in}}%
\pgfpathlineto{\pgfqpoint{4.134233in}{0.550000in}}%
\pgfpathclose%
\pgfusepath{fill}%
\end{pgfscope}%
\begin{pgfscope}%
\pgfpathrectangle{\pgfqpoint{1.250000in}{0.550000in}}{\pgfqpoint{7.750000in}{3.850000in}}%
\pgfusepath{clip}%
\pgfsetbuttcap%
\pgfsetmiterjoin%
\definecolor{currentfill}{rgb}{0.501961,0.501961,0.501961}%
\pgfsetfillcolor{currentfill}%
\pgfsetfillopacity{0.700000}%
\pgfsetlinewidth{0.000000pt}%
\definecolor{currentstroke}{rgb}{0.000000,0.000000,0.000000}%
\pgfsetstrokecolor{currentstroke}%
\pgfsetstrokeopacity{0.700000}%
\pgfsetdash{}{0pt}%
\pgfpathmoveto{\pgfqpoint{4.161754in}{0.550000in}}%
\pgfpathlineto{\pgfqpoint{4.189276in}{0.550000in}}%
\pgfpathlineto{\pgfqpoint{4.189276in}{0.634516in}}%
\pgfpathlineto{\pgfqpoint{4.161754in}{0.634516in}}%
\pgfpathlineto{\pgfqpoint{4.161754in}{0.550000in}}%
\pgfpathclose%
\pgfusepath{fill}%
\end{pgfscope}%
\begin{pgfscope}%
\pgfpathrectangle{\pgfqpoint{1.250000in}{0.550000in}}{\pgfqpoint{7.750000in}{3.850000in}}%
\pgfusepath{clip}%
\pgfsetbuttcap%
\pgfsetmiterjoin%
\definecolor{currentfill}{rgb}{0.501961,0.501961,0.501961}%
\pgfsetfillcolor{currentfill}%
\pgfsetfillopacity{0.700000}%
\pgfsetlinewidth{0.000000pt}%
\definecolor{currentstroke}{rgb}{0.000000,0.000000,0.000000}%
\pgfsetstrokecolor{currentstroke}%
\pgfsetstrokeopacity{0.700000}%
\pgfsetdash{}{0pt}%
\pgfpathmoveto{\pgfqpoint{4.189276in}{0.550000in}}%
\pgfpathlineto{\pgfqpoint{4.216797in}{0.550000in}}%
\pgfpathlineto{\pgfqpoint{4.216797in}{0.638038in}}%
\pgfpathlineto{\pgfqpoint{4.189276in}{0.638038in}}%
\pgfpathlineto{\pgfqpoint{4.189276in}{0.550000in}}%
\pgfpathclose%
\pgfusepath{fill}%
\end{pgfscope}%
\begin{pgfscope}%
\pgfpathrectangle{\pgfqpoint{1.250000in}{0.550000in}}{\pgfqpoint{7.750000in}{3.850000in}}%
\pgfusepath{clip}%
\pgfsetbuttcap%
\pgfsetmiterjoin%
\definecolor{currentfill}{rgb}{0.501961,0.501961,0.501961}%
\pgfsetfillcolor{currentfill}%
\pgfsetfillopacity{0.700000}%
\pgfsetlinewidth{0.000000pt}%
\definecolor{currentstroke}{rgb}{0.000000,0.000000,0.000000}%
\pgfsetstrokecolor{currentstroke}%
\pgfsetstrokeopacity{0.700000}%
\pgfsetdash{}{0pt}%
\pgfpathmoveto{\pgfqpoint{4.216797in}{0.550000in}}%
\pgfpathlineto{\pgfqpoint{4.244318in}{0.550000in}}%
\pgfpathlineto{\pgfqpoint{4.244318in}{0.629923in}}%
\pgfpathlineto{\pgfqpoint{4.216797in}{0.629923in}}%
\pgfpathlineto{\pgfqpoint{4.216797in}{0.550000in}}%
\pgfpathclose%
\pgfusepath{fill}%
\end{pgfscope}%
\begin{pgfscope}%
\pgfpathrectangle{\pgfqpoint{1.250000in}{0.550000in}}{\pgfqpoint{7.750000in}{3.850000in}}%
\pgfusepath{clip}%
\pgfsetbuttcap%
\pgfsetmiterjoin%
\definecolor{currentfill}{rgb}{0.501961,0.501961,0.501961}%
\pgfsetfillcolor{currentfill}%
\pgfsetfillopacity{0.700000}%
\pgfsetlinewidth{0.000000pt}%
\definecolor{currentstroke}{rgb}{0.000000,0.000000,0.000000}%
\pgfsetstrokecolor{currentstroke}%
\pgfsetstrokeopacity{0.700000}%
\pgfsetdash{}{0pt}%
\pgfpathmoveto{\pgfqpoint{4.244318in}{0.550000in}}%
\pgfpathlineto{\pgfqpoint{4.271839in}{0.550000in}}%
\pgfpathlineto{\pgfqpoint{4.271839in}{0.643397in}}%
\pgfpathlineto{\pgfqpoint{4.244318in}{0.643397in}}%
\pgfpathlineto{\pgfqpoint{4.244318in}{0.550000in}}%
\pgfpathclose%
\pgfusepath{fill}%
\end{pgfscope}%
\begin{pgfscope}%
\pgfpathrectangle{\pgfqpoint{1.250000in}{0.550000in}}{\pgfqpoint{7.750000in}{3.850000in}}%
\pgfusepath{clip}%
\pgfsetbuttcap%
\pgfsetmiterjoin%
\definecolor{currentfill}{rgb}{0.501961,0.501961,0.501961}%
\pgfsetfillcolor{currentfill}%
\pgfsetfillopacity{0.700000}%
\pgfsetlinewidth{0.000000pt}%
\definecolor{currentstroke}{rgb}{0.000000,0.000000,0.000000}%
\pgfsetstrokecolor{currentstroke}%
\pgfsetstrokeopacity{0.700000}%
\pgfsetdash{}{0pt}%
\pgfpathmoveto{\pgfqpoint{4.271839in}{0.550000in}}%
\pgfpathlineto{\pgfqpoint{4.299361in}{0.550000in}}%
\pgfpathlineto{\pgfqpoint{4.299361in}{0.633598in}}%
\pgfpathlineto{\pgfqpoint{4.271839in}{0.633598in}}%
\pgfpathlineto{\pgfqpoint{4.271839in}{0.550000in}}%
\pgfpathclose%
\pgfusepath{fill}%
\end{pgfscope}%
\begin{pgfscope}%
\pgfpathrectangle{\pgfqpoint{1.250000in}{0.550000in}}{\pgfqpoint{7.750000in}{3.850000in}}%
\pgfusepath{clip}%
\pgfsetbuttcap%
\pgfsetmiterjoin%
\definecolor{currentfill}{rgb}{0.501961,0.501961,0.501961}%
\pgfsetfillcolor{currentfill}%
\pgfsetfillopacity{0.700000}%
\pgfsetlinewidth{0.000000pt}%
\definecolor{currentstroke}{rgb}{0.000000,0.000000,0.000000}%
\pgfsetstrokecolor{currentstroke}%
\pgfsetstrokeopacity{0.700000}%
\pgfsetdash{}{0pt}%
\pgfpathmoveto{\pgfqpoint{4.299361in}{0.550000in}}%
\pgfpathlineto{\pgfqpoint{4.326882in}{0.550000in}}%
\pgfpathlineto{\pgfqpoint{4.326882in}{0.627473in}}%
\pgfpathlineto{\pgfqpoint{4.299361in}{0.627473in}}%
\pgfpathlineto{\pgfqpoint{4.299361in}{0.550000in}}%
\pgfpathclose%
\pgfusepath{fill}%
\end{pgfscope}%
\begin{pgfscope}%
\pgfpathrectangle{\pgfqpoint{1.250000in}{0.550000in}}{\pgfqpoint{7.750000in}{3.850000in}}%
\pgfusepath{clip}%
\pgfsetbuttcap%
\pgfsetmiterjoin%
\definecolor{currentfill}{rgb}{0.501961,0.501961,0.501961}%
\pgfsetfillcolor{currentfill}%
\pgfsetfillopacity{0.700000}%
\pgfsetlinewidth{0.000000pt}%
\definecolor{currentstroke}{rgb}{0.000000,0.000000,0.000000}%
\pgfsetstrokecolor{currentstroke}%
\pgfsetstrokeopacity{0.700000}%
\pgfsetdash{}{0pt}%
\pgfpathmoveto{\pgfqpoint{4.326882in}{0.550000in}}%
\pgfpathlineto{\pgfqpoint{4.354403in}{0.550000in}}%
\pgfpathlineto{\pgfqpoint{4.354403in}{0.644622in}}%
\pgfpathlineto{\pgfqpoint{4.326882in}{0.644622in}}%
\pgfpathlineto{\pgfqpoint{4.326882in}{0.550000in}}%
\pgfpathclose%
\pgfusepath{fill}%
\end{pgfscope}%
\begin{pgfscope}%
\pgfpathrectangle{\pgfqpoint{1.250000in}{0.550000in}}{\pgfqpoint{7.750000in}{3.850000in}}%
\pgfusepath{clip}%
\pgfsetbuttcap%
\pgfsetmiterjoin%
\definecolor{currentfill}{rgb}{0.501961,0.501961,0.501961}%
\pgfsetfillcolor{currentfill}%
\pgfsetfillopacity{0.700000}%
\pgfsetlinewidth{0.000000pt}%
\definecolor{currentstroke}{rgb}{0.000000,0.000000,0.000000}%
\pgfsetstrokecolor{currentstroke}%
\pgfsetstrokeopacity{0.700000}%
\pgfsetdash{}{0pt}%
\pgfpathmoveto{\pgfqpoint{4.354403in}{0.550000in}}%
\pgfpathlineto{\pgfqpoint{4.381925in}{0.550000in}}%
\pgfpathlineto{\pgfqpoint{4.381925in}{0.637426in}}%
\pgfpathlineto{\pgfqpoint{4.354403in}{0.637426in}}%
\pgfpathlineto{\pgfqpoint{4.354403in}{0.550000in}}%
\pgfpathclose%
\pgfusepath{fill}%
\end{pgfscope}%
\begin{pgfscope}%
\pgfpathrectangle{\pgfqpoint{1.250000in}{0.550000in}}{\pgfqpoint{7.750000in}{3.850000in}}%
\pgfusepath{clip}%
\pgfsetbuttcap%
\pgfsetmiterjoin%
\definecolor{currentfill}{rgb}{0.501961,0.501961,0.501961}%
\pgfsetfillcolor{currentfill}%
\pgfsetfillopacity{0.700000}%
\pgfsetlinewidth{0.000000pt}%
\definecolor{currentstroke}{rgb}{0.000000,0.000000,0.000000}%
\pgfsetstrokecolor{currentstroke}%
\pgfsetstrokeopacity{0.700000}%
\pgfsetdash{}{0pt}%
\pgfpathmoveto{\pgfqpoint{4.381925in}{0.550000in}}%
\pgfpathlineto{\pgfqpoint{4.409446in}{0.550000in}}%
\pgfpathlineto{\pgfqpoint{4.409446in}{0.628851in}}%
\pgfpathlineto{\pgfqpoint{4.381925in}{0.628851in}}%
\pgfpathlineto{\pgfqpoint{4.381925in}{0.550000in}}%
\pgfpathclose%
\pgfusepath{fill}%
\end{pgfscope}%
\begin{pgfscope}%
\pgfpathrectangle{\pgfqpoint{1.250000in}{0.550000in}}{\pgfqpoint{7.750000in}{3.850000in}}%
\pgfusepath{clip}%
\pgfsetbuttcap%
\pgfsetmiterjoin%
\definecolor{currentfill}{rgb}{0.501961,0.501961,0.501961}%
\pgfsetfillcolor{currentfill}%
\pgfsetfillopacity{0.700000}%
\pgfsetlinewidth{0.000000pt}%
\definecolor{currentstroke}{rgb}{0.000000,0.000000,0.000000}%
\pgfsetstrokecolor{currentstroke}%
\pgfsetstrokeopacity{0.700000}%
\pgfsetdash{}{0pt}%
\pgfpathmoveto{\pgfqpoint{4.409446in}{0.550000in}}%
\pgfpathlineto{\pgfqpoint{4.436967in}{0.550000in}}%
\pgfpathlineto{\pgfqpoint{4.436967in}{0.638957in}}%
\pgfpathlineto{\pgfqpoint{4.409446in}{0.638957in}}%
\pgfpathlineto{\pgfqpoint{4.409446in}{0.550000in}}%
\pgfpathclose%
\pgfusepath{fill}%
\end{pgfscope}%
\begin{pgfscope}%
\pgfpathrectangle{\pgfqpoint{1.250000in}{0.550000in}}{\pgfqpoint{7.750000in}{3.850000in}}%
\pgfusepath{clip}%
\pgfsetbuttcap%
\pgfsetmiterjoin%
\definecolor{currentfill}{rgb}{0.501961,0.501961,0.501961}%
\pgfsetfillcolor{currentfill}%
\pgfsetfillopacity{0.700000}%
\pgfsetlinewidth{0.000000pt}%
\definecolor{currentstroke}{rgb}{0.000000,0.000000,0.000000}%
\pgfsetstrokecolor{currentstroke}%
\pgfsetstrokeopacity{0.700000}%
\pgfsetdash{}{0pt}%
\pgfpathmoveto{\pgfqpoint{4.436967in}{0.550000in}}%
\pgfpathlineto{\pgfqpoint{4.464489in}{0.550000in}}%
\pgfpathlineto{\pgfqpoint{4.464489in}{0.632067in}}%
\pgfpathlineto{\pgfqpoint{4.436967in}{0.632067in}}%
\pgfpathlineto{\pgfqpoint{4.436967in}{0.550000in}}%
\pgfpathclose%
\pgfusepath{fill}%
\end{pgfscope}%
\begin{pgfscope}%
\pgfpathrectangle{\pgfqpoint{1.250000in}{0.550000in}}{\pgfqpoint{7.750000in}{3.850000in}}%
\pgfusepath{clip}%
\pgfsetbuttcap%
\pgfsetmiterjoin%
\definecolor{currentfill}{rgb}{0.501961,0.501961,0.501961}%
\pgfsetfillcolor{currentfill}%
\pgfsetfillopacity{0.700000}%
\pgfsetlinewidth{0.000000pt}%
\definecolor{currentstroke}{rgb}{0.000000,0.000000,0.000000}%
\pgfsetstrokecolor{currentstroke}%
\pgfsetstrokeopacity{0.700000}%
\pgfsetdash{}{0pt}%
\pgfpathmoveto{\pgfqpoint{4.464489in}{0.550000in}}%
\pgfpathlineto{\pgfqpoint{4.492010in}{0.550000in}}%
\pgfpathlineto{\pgfqpoint{4.492010in}{0.622727in}}%
\pgfpathlineto{\pgfqpoint{4.464489in}{0.622727in}}%
\pgfpathlineto{\pgfqpoint{4.464489in}{0.550000in}}%
\pgfpathclose%
\pgfusepath{fill}%
\end{pgfscope}%
\begin{pgfscope}%
\pgfpathrectangle{\pgfqpoint{1.250000in}{0.550000in}}{\pgfqpoint{7.750000in}{3.850000in}}%
\pgfusepath{clip}%
\pgfsetbuttcap%
\pgfsetmiterjoin%
\definecolor{currentfill}{rgb}{0.501961,0.501961,0.501961}%
\pgfsetfillcolor{currentfill}%
\pgfsetfillopacity{0.700000}%
\pgfsetlinewidth{0.000000pt}%
\definecolor{currentstroke}{rgb}{0.000000,0.000000,0.000000}%
\pgfsetstrokecolor{currentstroke}%
\pgfsetstrokeopacity{0.700000}%
\pgfsetdash{}{0pt}%
\pgfpathmoveto{\pgfqpoint{4.492010in}{0.550000in}}%
\pgfpathlineto{\pgfqpoint{4.519531in}{0.550000in}}%
\pgfpathlineto{\pgfqpoint{4.519531in}{0.641713in}}%
\pgfpathlineto{\pgfqpoint{4.492010in}{0.641713in}}%
\pgfpathlineto{\pgfqpoint{4.492010in}{0.550000in}}%
\pgfpathclose%
\pgfusepath{fill}%
\end{pgfscope}%
\begin{pgfscope}%
\pgfpathrectangle{\pgfqpoint{1.250000in}{0.550000in}}{\pgfqpoint{7.750000in}{3.850000in}}%
\pgfusepath{clip}%
\pgfsetbuttcap%
\pgfsetmiterjoin%
\definecolor{currentfill}{rgb}{0.501961,0.501961,0.501961}%
\pgfsetfillcolor{currentfill}%
\pgfsetfillopacity{0.700000}%
\pgfsetlinewidth{0.000000pt}%
\definecolor{currentstroke}{rgb}{0.000000,0.000000,0.000000}%
\pgfsetstrokecolor{currentstroke}%
\pgfsetstrokeopacity{0.700000}%
\pgfsetdash{}{0pt}%
\pgfpathmoveto{\pgfqpoint{4.519531in}{0.550000in}}%
\pgfpathlineto{\pgfqpoint{4.547053in}{0.550000in}}%
\pgfpathlineto{\pgfqpoint{4.547053in}{0.624411in}}%
\pgfpathlineto{\pgfqpoint{4.519531in}{0.624411in}}%
\pgfpathlineto{\pgfqpoint{4.519531in}{0.550000in}}%
\pgfpathclose%
\pgfusepath{fill}%
\end{pgfscope}%
\begin{pgfscope}%
\pgfpathrectangle{\pgfqpoint{1.250000in}{0.550000in}}{\pgfqpoint{7.750000in}{3.850000in}}%
\pgfusepath{clip}%
\pgfsetbuttcap%
\pgfsetmiterjoin%
\definecolor{currentfill}{rgb}{0.501961,0.501961,0.501961}%
\pgfsetfillcolor{currentfill}%
\pgfsetfillopacity{0.700000}%
\pgfsetlinewidth{0.000000pt}%
\definecolor{currentstroke}{rgb}{0.000000,0.000000,0.000000}%
\pgfsetstrokecolor{currentstroke}%
\pgfsetstrokeopacity{0.700000}%
\pgfsetdash{}{0pt}%
\pgfpathmoveto{\pgfqpoint{4.547053in}{0.550000in}}%
\pgfpathlineto{\pgfqpoint{4.574574in}{0.550000in}}%
\pgfpathlineto{\pgfqpoint{4.574574in}{0.637579in}}%
\pgfpathlineto{\pgfqpoint{4.547053in}{0.637579in}}%
\pgfpathlineto{\pgfqpoint{4.547053in}{0.550000in}}%
\pgfpathclose%
\pgfusepath{fill}%
\end{pgfscope}%
\begin{pgfscope}%
\pgfpathrectangle{\pgfqpoint{1.250000in}{0.550000in}}{\pgfqpoint{7.750000in}{3.850000in}}%
\pgfusepath{clip}%
\pgfsetbuttcap%
\pgfsetmiterjoin%
\definecolor{currentfill}{rgb}{0.501961,0.501961,0.501961}%
\pgfsetfillcolor{currentfill}%
\pgfsetfillopacity{0.700000}%
\pgfsetlinewidth{0.000000pt}%
\definecolor{currentstroke}{rgb}{0.000000,0.000000,0.000000}%
\pgfsetstrokecolor{currentstroke}%
\pgfsetstrokeopacity{0.700000}%
\pgfsetdash{}{0pt}%
\pgfpathmoveto{\pgfqpoint{4.574574in}{0.550000in}}%
\pgfpathlineto{\pgfqpoint{4.602095in}{0.550000in}}%
\pgfpathlineto{\pgfqpoint{4.602095in}{0.631760in}}%
\pgfpathlineto{\pgfqpoint{4.574574in}{0.631760in}}%
\pgfpathlineto{\pgfqpoint{4.574574in}{0.550000in}}%
\pgfpathclose%
\pgfusepath{fill}%
\end{pgfscope}%
\begin{pgfscope}%
\pgfpathrectangle{\pgfqpoint{1.250000in}{0.550000in}}{\pgfqpoint{7.750000in}{3.850000in}}%
\pgfusepath{clip}%
\pgfsetbuttcap%
\pgfsetmiterjoin%
\definecolor{currentfill}{rgb}{0.501961,0.501961,0.501961}%
\pgfsetfillcolor{currentfill}%
\pgfsetfillopacity{0.700000}%
\pgfsetlinewidth{0.000000pt}%
\definecolor{currentstroke}{rgb}{0.000000,0.000000,0.000000}%
\pgfsetstrokecolor{currentstroke}%
\pgfsetstrokeopacity{0.700000}%
\pgfsetdash{}{0pt}%
\pgfpathmoveto{\pgfqpoint{4.602095in}{0.550000in}}%
\pgfpathlineto{\pgfqpoint{4.629616in}{0.550000in}}%
\pgfpathlineto{\pgfqpoint{4.629616in}{0.633598in}}%
\pgfpathlineto{\pgfqpoint{4.602095in}{0.633598in}}%
\pgfpathlineto{\pgfqpoint{4.602095in}{0.550000in}}%
\pgfpathclose%
\pgfusepath{fill}%
\end{pgfscope}%
\begin{pgfscope}%
\pgfpathrectangle{\pgfqpoint{1.250000in}{0.550000in}}{\pgfqpoint{7.750000in}{3.850000in}}%
\pgfusepath{clip}%
\pgfsetbuttcap%
\pgfsetmiterjoin%
\definecolor{currentfill}{rgb}{0.501961,0.501961,0.501961}%
\pgfsetfillcolor{currentfill}%
\pgfsetfillopacity{0.700000}%
\pgfsetlinewidth{0.000000pt}%
\definecolor{currentstroke}{rgb}{0.000000,0.000000,0.000000}%
\pgfsetstrokecolor{currentstroke}%
\pgfsetstrokeopacity{0.700000}%
\pgfsetdash{}{0pt}%
\pgfpathmoveto{\pgfqpoint{4.629616in}{0.550000in}}%
\pgfpathlineto{\pgfqpoint{4.657138in}{0.550000in}}%
\pgfpathlineto{\pgfqpoint{4.657138in}{0.642937in}}%
\pgfpathlineto{\pgfqpoint{4.629616in}{0.642937in}}%
\pgfpathlineto{\pgfqpoint{4.629616in}{0.550000in}}%
\pgfpathclose%
\pgfusepath{fill}%
\end{pgfscope}%
\begin{pgfscope}%
\pgfpathrectangle{\pgfqpoint{1.250000in}{0.550000in}}{\pgfqpoint{7.750000in}{3.850000in}}%
\pgfusepath{clip}%
\pgfsetbuttcap%
\pgfsetmiterjoin%
\definecolor{currentfill}{rgb}{0.501961,0.501961,0.501961}%
\pgfsetfillcolor{currentfill}%
\pgfsetfillopacity{0.700000}%
\pgfsetlinewidth{0.000000pt}%
\definecolor{currentstroke}{rgb}{0.000000,0.000000,0.000000}%
\pgfsetstrokecolor{currentstroke}%
\pgfsetstrokeopacity{0.700000}%
\pgfsetdash{}{0pt}%
\pgfpathmoveto{\pgfqpoint{4.657138in}{0.550000in}}%
\pgfpathlineto{\pgfqpoint{4.684659in}{0.550000in}}%
\pgfpathlineto{\pgfqpoint{4.684659in}{0.624411in}}%
\pgfpathlineto{\pgfqpoint{4.657138in}{0.624411in}}%
\pgfpathlineto{\pgfqpoint{4.657138in}{0.550000in}}%
\pgfpathclose%
\pgfusepath{fill}%
\end{pgfscope}%
\begin{pgfscope}%
\pgfpathrectangle{\pgfqpoint{1.250000in}{0.550000in}}{\pgfqpoint{7.750000in}{3.850000in}}%
\pgfusepath{clip}%
\pgfsetbuttcap%
\pgfsetmiterjoin%
\definecolor{currentfill}{rgb}{0.501961,0.501961,0.501961}%
\pgfsetfillcolor{currentfill}%
\pgfsetfillopacity{0.700000}%
\pgfsetlinewidth{0.000000pt}%
\definecolor{currentstroke}{rgb}{0.000000,0.000000,0.000000}%
\pgfsetstrokecolor{currentstroke}%
\pgfsetstrokeopacity{0.700000}%
\pgfsetdash{}{0pt}%
\pgfpathmoveto{\pgfqpoint{4.684659in}{0.550000in}}%
\pgfpathlineto{\pgfqpoint{4.712180in}{0.550000in}}%
\pgfpathlineto{\pgfqpoint{4.712180in}{0.627627in}}%
\pgfpathlineto{\pgfqpoint{4.684659in}{0.627627in}}%
\pgfpathlineto{\pgfqpoint{4.684659in}{0.550000in}}%
\pgfpathclose%
\pgfusepath{fill}%
\end{pgfscope}%
\begin{pgfscope}%
\pgfpathrectangle{\pgfqpoint{1.250000in}{0.550000in}}{\pgfqpoint{7.750000in}{3.850000in}}%
\pgfusepath{clip}%
\pgfsetbuttcap%
\pgfsetmiterjoin%
\definecolor{currentfill}{rgb}{0.501961,0.501961,0.501961}%
\pgfsetfillcolor{currentfill}%
\pgfsetfillopacity{0.700000}%
\pgfsetlinewidth{0.000000pt}%
\definecolor{currentstroke}{rgb}{0.000000,0.000000,0.000000}%
\pgfsetstrokecolor{currentstroke}%
\pgfsetstrokeopacity{0.700000}%
\pgfsetdash{}{0pt}%
\pgfpathmoveto{\pgfqpoint{4.712180in}{0.550000in}}%
\pgfpathlineto{\pgfqpoint{4.739702in}{0.550000in}}%
\pgfpathlineto{\pgfqpoint{4.739702in}{0.636048in}}%
\pgfpathlineto{\pgfqpoint{4.712180in}{0.636048in}}%
\pgfpathlineto{\pgfqpoint{4.712180in}{0.550000in}}%
\pgfpathclose%
\pgfusepath{fill}%
\end{pgfscope}%
\begin{pgfscope}%
\pgfpathrectangle{\pgfqpoint{1.250000in}{0.550000in}}{\pgfqpoint{7.750000in}{3.850000in}}%
\pgfusepath{clip}%
\pgfsetbuttcap%
\pgfsetmiterjoin%
\definecolor{currentfill}{rgb}{0.501961,0.501961,0.501961}%
\pgfsetfillcolor{currentfill}%
\pgfsetfillopacity{0.700000}%
\pgfsetlinewidth{0.000000pt}%
\definecolor{currentstroke}{rgb}{0.000000,0.000000,0.000000}%
\pgfsetstrokecolor{currentstroke}%
\pgfsetstrokeopacity{0.700000}%
\pgfsetdash{}{0pt}%
\pgfpathmoveto{\pgfqpoint{4.739702in}{0.550000in}}%
\pgfpathlineto{\pgfqpoint{4.767223in}{0.550000in}}%
\pgfpathlineto{\pgfqpoint{4.767223in}{0.619052in}}%
\pgfpathlineto{\pgfqpoint{4.739702in}{0.619052in}}%
\pgfpathlineto{\pgfqpoint{4.739702in}{0.550000in}}%
\pgfpathclose%
\pgfusepath{fill}%
\end{pgfscope}%
\begin{pgfscope}%
\pgfpathrectangle{\pgfqpoint{1.250000in}{0.550000in}}{\pgfqpoint{7.750000in}{3.850000in}}%
\pgfusepath{clip}%
\pgfsetbuttcap%
\pgfsetmiterjoin%
\definecolor{currentfill}{rgb}{0.501961,0.501961,0.501961}%
\pgfsetfillcolor{currentfill}%
\pgfsetfillopacity{0.700000}%
\pgfsetlinewidth{0.000000pt}%
\definecolor{currentstroke}{rgb}{0.000000,0.000000,0.000000}%
\pgfsetstrokecolor{currentstroke}%
\pgfsetstrokeopacity{0.700000}%
\pgfsetdash{}{0pt}%
\pgfpathmoveto{\pgfqpoint{4.767223in}{0.550000in}}%
\pgfpathlineto{\pgfqpoint{4.794744in}{0.550000in}}%
\pgfpathlineto{\pgfqpoint{4.794744in}{0.637426in}}%
\pgfpathlineto{\pgfqpoint{4.767223in}{0.637426in}}%
\pgfpathlineto{\pgfqpoint{4.767223in}{0.550000in}}%
\pgfpathclose%
\pgfusepath{fill}%
\end{pgfscope}%
\begin{pgfscope}%
\pgfpathrectangle{\pgfqpoint{1.250000in}{0.550000in}}{\pgfqpoint{7.750000in}{3.850000in}}%
\pgfusepath{clip}%
\pgfsetbuttcap%
\pgfsetmiterjoin%
\definecolor{currentfill}{rgb}{0.501961,0.501961,0.501961}%
\pgfsetfillcolor{currentfill}%
\pgfsetfillopacity{0.700000}%
\pgfsetlinewidth{0.000000pt}%
\definecolor{currentstroke}{rgb}{0.000000,0.000000,0.000000}%
\pgfsetstrokecolor{currentstroke}%
\pgfsetstrokeopacity{0.700000}%
\pgfsetdash{}{0pt}%
\pgfpathmoveto{\pgfqpoint{4.794744in}{0.550000in}}%
\pgfpathlineto{\pgfqpoint{4.822266in}{0.550000in}}%
\pgfpathlineto{\pgfqpoint{4.822266in}{0.622421in}}%
\pgfpathlineto{\pgfqpoint{4.794744in}{0.622421in}}%
\pgfpathlineto{\pgfqpoint{4.794744in}{0.550000in}}%
\pgfpathclose%
\pgfusepath{fill}%
\end{pgfscope}%
\begin{pgfscope}%
\pgfpathrectangle{\pgfqpoint{1.250000in}{0.550000in}}{\pgfqpoint{7.750000in}{3.850000in}}%
\pgfusepath{clip}%
\pgfsetbuttcap%
\pgfsetmiterjoin%
\definecolor{currentfill}{rgb}{0.501961,0.501961,0.501961}%
\pgfsetfillcolor{currentfill}%
\pgfsetfillopacity{0.700000}%
\pgfsetlinewidth{0.000000pt}%
\definecolor{currentstroke}{rgb}{0.000000,0.000000,0.000000}%
\pgfsetstrokecolor{currentstroke}%
\pgfsetstrokeopacity{0.700000}%
\pgfsetdash{}{0pt}%
\pgfpathmoveto{\pgfqpoint{4.822266in}{0.550000in}}%
\pgfpathlineto{\pgfqpoint{4.849787in}{0.550000in}}%
\pgfpathlineto{\pgfqpoint{4.849787in}{0.633904in}}%
\pgfpathlineto{\pgfqpoint{4.822266in}{0.633904in}}%
\pgfpathlineto{\pgfqpoint{4.822266in}{0.550000in}}%
\pgfpathclose%
\pgfusepath{fill}%
\end{pgfscope}%
\begin{pgfscope}%
\pgfpathrectangle{\pgfqpoint{1.250000in}{0.550000in}}{\pgfqpoint{7.750000in}{3.850000in}}%
\pgfusepath{clip}%
\pgfsetbuttcap%
\pgfsetmiterjoin%
\definecolor{currentfill}{rgb}{0.501961,0.501961,0.501961}%
\pgfsetfillcolor{currentfill}%
\pgfsetfillopacity{0.700000}%
\pgfsetlinewidth{0.000000pt}%
\definecolor{currentstroke}{rgb}{0.000000,0.000000,0.000000}%
\pgfsetstrokecolor{currentstroke}%
\pgfsetstrokeopacity{0.700000}%
\pgfsetdash{}{0pt}%
\pgfpathmoveto{\pgfqpoint{4.849787in}{0.550000in}}%
\pgfpathlineto{\pgfqpoint{4.877308in}{0.550000in}}%
\pgfpathlineto{\pgfqpoint{4.877308in}{0.619052in}}%
\pgfpathlineto{\pgfqpoint{4.849787in}{0.619052in}}%
\pgfpathlineto{\pgfqpoint{4.849787in}{0.550000in}}%
\pgfpathclose%
\pgfusepath{fill}%
\end{pgfscope}%
\begin{pgfscope}%
\pgfpathrectangle{\pgfqpoint{1.250000in}{0.550000in}}{\pgfqpoint{7.750000in}{3.850000in}}%
\pgfusepath{clip}%
\pgfsetbuttcap%
\pgfsetmiterjoin%
\definecolor{currentfill}{rgb}{0.501961,0.501961,0.501961}%
\pgfsetfillcolor{currentfill}%
\pgfsetfillopacity{0.700000}%
\pgfsetlinewidth{0.000000pt}%
\definecolor{currentstroke}{rgb}{0.000000,0.000000,0.000000}%
\pgfsetstrokecolor{currentstroke}%
\pgfsetstrokeopacity{0.700000}%
\pgfsetdash{}{0pt}%
\pgfpathmoveto{\pgfqpoint{4.877308in}{0.550000in}}%
\pgfpathlineto{\pgfqpoint{4.904830in}{0.550000in}}%
\pgfpathlineto{\pgfqpoint{4.904830in}{0.622574in}}%
\pgfpathlineto{\pgfqpoint{4.877308in}{0.622574in}}%
\pgfpathlineto{\pgfqpoint{4.877308in}{0.550000in}}%
\pgfpathclose%
\pgfusepath{fill}%
\end{pgfscope}%
\begin{pgfscope}%
\pgfpathrectangle{\pgfqpoint{1.250000in}{0.550000in}}{\pgfqpoint{7.750000in}{3.850000in}}%
\pgfusepath{clip}%
\pgfsetbuttcap%
\pgfsetmiterjoin%
\definecolor{currentfill}{rgb}{0.501961,0.501961,0.501961}%
\pgfsetfillcolor{currentfill}%
\pgfsetfillopacity{0.700000}%
\pgfsetlinewidth{0.000000pt}%
\definecolor{currentstroke}{rgb}{0.000000,0.000000,0.000000}%
\pgfsetstrokecolor{currentstroke}%
\pgfsetstrokeopacity{0.700000}%
\pgfsetdash{}{0pt}%
\pgfpathmoveto{\pgfqpoint{4.904830in}{0.550000in}}%
\pgfpathlineto{\pgfqpoint{4.932351in}{0.550000in}}%
\pgfpathlineto{\pgfqpoint{4.932351in}{0.628545in}}%
\pgfpathlineto{\pgfqpoint{4.904830in}{0.628545in}}%
\pgfpathlineto{\pgfqpoint{4.904830in}{0.550000in}}%
\pgfpathclose%
\pgfusepath{fill}%
\end{pgfscope}%
\begin{pgfscope}%
\pgfpathrectangle{\pgfqpoint{1.250000in}{0.550000in}}{\pgfqpoint{7.750000in}{3.850000in}}%
\pgfusepath{clip}%
\pgfsetbuttcap%
\pgfsetmiterjoin%
\definecolor{currentfill}{rgb}{0.501961,0.501961,0.501961}%
\pgfsetfillcolor{currentfill}%
\pgfsetfillopacity{0.700000}%
\pgfsetlinewidth{0.000000pt}%
\definecolor{currentstroke}{rgb}{0.000000,0.000000,0.000000}%
\pgfsetstrokecolor{currentstroke}%
\pgfsetstrokeopacity{0.700000}%
\pgfsetdash{}{0pt}%
\pgfpathmoveto{\pgfqpoint{4.932351in}{0.550000in}}%
\pgfpathlineto{\pgfqpoint{4.959872in}{0.550000in}}%
\pgfpathlineto{\pgfqpoint{4.959872in}{0.616143in}}%
\pgfpathlineto{\pgfqpoint{4.932351in}{0.616143in}}%
\pgfpathlineto{\pgfqpoint{4.932351in}{0.550000in}}%
\pgfpathclose%
\pgfusepath{fill}%
\end{pgfscope}%
\begin{pgfscope}%
\pgfpathrectangle{\pgfqpoint{1.250000in}{0.550000in}}{\pgfqpoint{7.750000in}{3.850000in}}%
\pgfusepath{clip}%
\pgfsetbuttcap%
\pgfsetmiterjoin%
\definecolor{currentfill}{rgb}{0.501961,0.501961,0.501961}%
\pgfsetfillcolor{currentfill}%
\pgfsetfillopacity{0.700000}%
\pgfsetlinewidth{0.000000pt}%
\definecolor{currentstroke}{rgb}{0.000000,0.000000,0.000000}%
\pgfsetstrokecolor{currentstroke}%
\pgfsetstrokeopacity{0.700000}%
\pgfsetdash{}{0pt}%
\pgfpathmoveto{\pgfqpoint{4.959872in}{0.550000in}}%
\pgfpathlineto{\pgfqpoint{4.987393in}{0.550000in}}%
\pgfpathlineto{\pgfqpoint{4.987393in}{0.620583in}}%
\pgfpathlineto{\pgfqpoint{4.959872in}{0.620583in}}%
\pgfpathlineto{\pgfqpoint{4.959872in}{0.550000in}}%
\pgfpathclose%
\pgfusepath{fill}%
\end{pgfscope}%
\begin{pgfscope}%
\pgfpathrectangle{\pgfqpoint{1.250000in}{0.550000in}}{\pgfqpoint{7.750000in}{3.850000in}}%
\pgfusepath{clip}%
\pgfsetbuttcap%
\pgfsetmiterjoin%
\definecolor{currentfill}{rgb}{0.501961,0.501961,0.501961}%
\pgfsetfillcolor{currentfill}%
\pgfsetfillopacity{0.700000}%
\pgfsetlinewidth{0.000000pt}%
\definecolor{currentstroke}{rgb}{0.000000,0.000000,0.000000}%
\pgfsetstrokecolor{currentstroke}%
\pgfsetstrokeopacity{0.700000}%
\pgfsetdash{}{0pt}%
\pgfpathmoveto{\pgfqpoint{4.987393in}{0.550000in}}%
\pgfpathlineto{\pgfqpoint{5.014915in}{0.550000in}}%
\pgfpathlineto{\pgfqpoint{5.014915in}{0.619971in}}%
\pgfpathlineto{\pgfqpoint{4.987393in}{0.619971in}}%
\pgfpathlineto{\pgfqpoint{4.987393in}{0.550000in}}%
\pgfpathclose%
\pgfusepath{fill}%
\end{pgfscope}%
\begin{pgfscope}%
\pgfpathrectangle{\pgfqpoint{1.250000in}{0.550000in}}{\pgfqpoint{7.750000in}{3.850000in}}%
\pgfusepath{clip}%
\pgfsetbuttcap%
\pgfsetmiterjoin%
\definecolor{currentfill}{rgb}{0.501961,0.501961,0.501961}%
\pgfsetfillcolor{currentfill}%
\pgfsetfillopacity{0.700000}%
\pgfsetlinewidth{0.000000pt}%
\definecolor{currentstroke}{rgb}{0.000000,0.000000,0.000000}%
\pgfsetstrokecolor{currentstroke}%
\pgfsetstrokeopacity{0.700000}%
\pgfsetdash{}{0pt}%
\pgfpathmoveto{\pgfqpoint{5.014915in}{0.550000in}}%
\pgfpathlineto{\pgfqpoint{5.042436in}{0.550000in}}%
\pgfpathlineto{\pgfqpoint{5.042436in}{0.610784in}}%
\pgfpathlineto{\pgfqpoint{5.014915in}{0.610784in}}%
\pgfpathlineto{\pgfqpoint{5.014915in}{0.550000in}}%
\pgfpathclose%
\pgfusepath{fill}%
\end{pgfscope}%
\begin{pgfscope}%
\pgfpathrectangle{\pgfqpoint{1.250000in}{0.550000in}}{\pgfqpoint{7.750000in}{3.850000in}}%
\pgfusepath{clip}%
\pgfsetbuttcap%
\pgfsetmiterjoin%
\definecolor{currentfill}{rgb}{0.501961,0.501961,0.501961}%
\pgfsetfillcolor{currentfill}%
\pgfsetfillopacity{0.700000}%
\pgfsetlinewidth{0.000000pt}%
\definecolor{currentstroke}{rgb}{0.000000,0.000000,0.000000}%
\pgfsetstrokecolor{currentstroke}%
\pgfsetstrokeopacity{0.700000}%
\pgfsetdash{}{0pt}%
\pgfpathmoveto{\pgfqpoint{5.042436in}{0.550000in}}%
\pgfpathlineto{\pgfqpoint{5.069957in}{0.550000in}}%
\pgfpathlineto{\pgfqpoint{5.069957in}{0.604201in}}%
\pgfpathlineto{\pgfqpoint{5.042436in}{0.604201in}}%
\pgfpathlineto{\pgfqpoint{5.042436in}{0.550000in}}%
\pgfpathclose%
\pgfusepath{fill}%
\end{pgfscope}%
\begin{pgfscope}%
\pgfpathrectangle{\pgfqpoint{1.250000in}{0.550000in}}{\pgfqpoint{7.750000in}{3.850000in}}%
\pgfusepath{clip}%
\pgfsetbuttcap%
\pgfsetmiterjoin%
\definecolor{currentfill}{rgb}{0.501961,0.501961,0.501961}%
\pgfsetfillcolor{currentfill}%
\pgfsetfillopacity{0.700000}%
\pgfsetlinewidth{0.000000pt}%
\definecolor{currentstroke}{rgb}{0.000000,0.000000,0.000000}%
\pgfsetstrokecolor{currentstroke}%
\pgfsetstrokeopacity{0.700000}%
\pgfsetdash{}{0pt}%
\pgfpathmoveto{\pgfqpoint{5.069957in}{0.550000in}}%
\pgfpathlineto{\pgfqpoint{5.097479in}{0.550000in}}%
\pgfpathlineto{\pgfqpoint{5.097479in}{0.601751in}}%
\pgfpathlineto{\pgfqpoint{5.069957in}{0.601751in}}%
\pgfpathlineto{\pgfqpoint{5.069957in}{0.550000in}}%
\pgfpathclose%
\pgfusepath{fill}%
\end{pgfscope}%
\begin{pgfscope}%
\pgfpathrectangle{\pgfqpoint{1.250000in}{0.550000in}}{\pgfqpoint{7.750000in}{3.850000in}}%
\pgfusepath{clip}%
\pgfsetbuttcap%
\pgfsetmiterjoin%
\definecolor{currentfill}{rgb}{0.501961,0.501961,0.501961}%
\pgfsetfillcolor{currentfill}%
\pgfsetfillopacity{0.700000}%
\pgfsetlinewidth{0.000000pt}%
\definecolor{currentstroke}{rgb}{0.000000,0.000000,0.000000}%
\pgfsetstrokecolor{currentstroke}%
\pgfsetstrokeopacity{0.700000}%
\pgfsetdash{}{0pt}%
\pgfpathmoveto{\pgfqpoint{5.097479in}{0.550000in}}%
\pgfpathlineto{\pgfqpoint{5.125000in}{0.550000in}}%
\pgfpathlineto{\pgfqpoint{5.125000in}{0.600679in}}%
\pgfpathlineto{\pgfqpoint{5.097479in}{0.600679in}}%
\pgfpathlineto{\pgfqpoint{5.097479in}{0.550000in}}%
\pgfpathclose%
\pgfusepath{fill}%
\end{pgfscope}%
\begin{pgfscope}%
\pgfpathrectangle{\pgfqpoint{1.250000in}{0.550000in}}{\pgfqpoint{7.750000in}{3.850000in}}%
\pgfusepath{clip}%
\pgfsetbuttcap%
\pgfsetmiterjoin%
\definecolor{currentfill}{rgb}{0.501961,0.501961,0.501961}%
\pgfsetfillcolor{currentfill}%
\pgfsetfillopacity{0.700000}%
\pgfsetlinewidth{0.000000pt}%
\definecolor{currentstroke}{rgb}{0.000000,0.000000,0.000000}%
\pgfsetstrokecolor{currentstroke}%
\pgfsetstrokeopacity{0.700000}%
\pgfsetdash{}{0pt}%
\pgfpathmoveto{\pgfqpoint{5.125000in}{0.550000in}}%
\pgfpathlineto{\pgfqpoint{5.152521in}{0.550000in}}%
\pgfpathlineto{\pgfqpoint{5.152521in}{0.600526in}}%
\pgfpathlineto{\pgfqpoint{5.125000in}{0.600526in}}%
\pgfpathlineto{\pgfqpoint{5.125000in}{0.550000in}}%
\pgfpathclose%
\pgfusepath{fill}%
\end{pgfscope}%
\begin{pgfscope}%
\pgfpathrectangle{\pgfqpoint{1.250000in}{0.550000in}}{\pgfqpoint{7.750000in}{3.850000in}}%
\pgfusepath{clip}%
\pgfsetbuttcap%
\pgfsetmiterjoin%
\definecolor{currentfill}{rgb}{0.501961,0.501961,0.501961}%
\pgfsetfillcolor{currentfill}%
\pgfsetfillopacity{0.700000}%
\pgfsetlinewidth{0.000000pt}%
\definecolor{currentstroke}{rgb}{0.000000,0.000000,0.000000}%
\pgfsetstrokecolor{currentstroke}%
\pgfsetstrokeopacity{0.700000}%
\pgfsetdash{}{0pt}%
\pgfpathmoveto{\pgfqpoint{5.152521in}{0.550000in}}%
\pgfpathlineto{\pgfqpoint{5.180043in}{0.550000in}}%
\pgfpathlineto{\pgfqpoint{5.180043in}{0.589196in}}%
\pgfpathlineto{\pgfqpoint{5.152521in}{0.589196in}}%
\pgfpathlineto{\pgfqpoint{5.152521in}{0.550000in}}%
\pgfpathclose%
\pgfusepath{fill}%
\end{pgfscope}%
\begin{pgfscope}%
\pgfpathrectangle{\pgfqpoint{1.250000in}{0.550000in}}{\pgfqpoint{7.750000in}{3.850000in}}%
\pgfusepath{clip}%
\pgfsetbuttcap%
\pgfsetmiterjoin%
\definecolor{currentfill}{rgb}{0.501961,0.501961,0.501961}%
\pgfsetfillcolor{currentfill}%
\pgfsetfillopacity{0.700000}%
\pgfsetlinewidth{0.000000pt}%
\definecolor{currentstroke}{rgb}{0.000000,0.000000,0.000000}%
\pgfsetstrokecolor{currentstroke}%
\pgfsetstrokeopacity{0.700000}%
\pgfsetdash{}{0pt}%
\pgfpathmoveto{\pgfqpoint{5.180043in}{0.550000in}}%
\pgfpathlineto{\pgfqpoint{5.207564in}{0.550000in}}%
\pgfpathlineto{\pgfqpoint{5.207564in}{0.597464in}}%
\pgfpathlineto{\pgfqpoint{5.180043in}{0.597464in}}%
\pgfpathlineto{\pgfqpoint{5.180043in}{0.550000in}}%
\pgfpathclose%
\pgfusepath{fill}%
\end{pgfscope}%
\begin{pgfscope}%
\pgfpathrectangle{\pgfqpoint{1.250000in}{0.550000in}}{\pgfqpoint{7.750000in}{3.850000in}}%
\pgfusepath{clip}%
\pgfsetbuttcap%
\pgfsetmiterjoin%
\definecolor{currentfill}{rgb}{0.501961,0.501961,0.501961}%
\pgfsetfillcolor{currentfill}%
\pgfsetfillopacity{0.700000}%
\pgfsetlinewidth{0.000000pt}%
\definecolor{currentstroke}{rgb}{0.000000,0.000000,0.000000}%
\pgfsetstrokecolor{currentstroke}%
\pgfsetstrokeopacity{0.700000}%
\pgfsetdash{}{0pt}%
\pgfpathmoveto{\pgfqpoint{5.207564in}{0.550000in}}%
\pgfpathlineto{\pgfqpoint{5.235085in}{0.550000in}}%
\pgfpathlineto{\pgfqpoint{5.235085in}{0.587818in}}%
\pgfpathlineto{\pgfqpoint{5.207564in}{0.587818in}}%
\pgfpathlineto{\pgfqpoint{5.207564in}{0.550000in}}%
\pgfpathclose%
\pgfusepath{fill}%
\end{pgfscope}%
\begin{pgfscope}%
\pgfpathrectangle{\pgfqpoint{1.250000in}{0.550000in}}{\pgfqpoint{7.750000in}{3.850000in}}%
\pgfusepath{clip}%
\pgfsetbuttcap%
\pgfsetmiterjoin%
\definecolor{currentfill}{rgb}{0.501961,0.501961,0.501961}%
\pgfsetfillcolor{currentfill}%
\pgfsetfillopacity{0.700000}%
\pgfsetlinewidth{0.000000pt}%
\definecolor{currentstroke}{rgb}{0.000000,0.000000,0.000000}%
\pgfsetstrokecolor{currentstroke}%
\pgfsetstrokeopacity{0.700000}%
\pgfsetdash{}{0pt}%
\pgfpathmoveto{\pgfqpoint{5.235085in}{0.550000in}}%
\pgfpathlineto{\pgfqpoint{5.262607in}{0.550000in}}%
\pgfpathlineto{\pgfqpoint{5.262607in}{0.592564in}}%
\pgfpathlineto{\pgfqpoint{5.235085in}{0.592564in}}%
\pgfpathlineto{\pgfqpoint{5.235085in}{0.550000in}}%
\pgfpathclose%
\pgfusepath{fill}%
\end{pgfscope}%
\begin{pgfscope}%
\pgfpathrectangle{\pgfqpoint{1.250000in}{0.550000in}}{\pgfqpoint{7.750000in}{3.850000in}}%
\pgfusepath{clip}%
\pgfsetbuttcap%
\pgfsetmiterjoin%
\definecolor{currentfill}{rgb}{0.501961,0.501961,0.501961}%
\pgfsetfillcolor{currentfill}%
\pgfsetfillopacity{0.700000}%
\pgfsetlinewidth{0.000000pt}%
\definecolor{currentstroke}{rgb}{0.000000,0.000000,0.000000}%
\pgfsetstrokecolor{currentstroke}%
\pgfsetstrokeopacity{0.700000}%
\pgfsetdash{}{0pt}%
\pgfpathmoveto{\pgfqpoint{5.262607in}{0.550000in}}%
\pgfpathlineto{\pgfqpoint{5.290128in}{0.550000in}}%
\pgfpathlineto{\pgfqpoint{5.290128in}{0.584909in}}%
\pgfpathlineto{\pgfqpoint{5.262607in}{0.584909in}}%
\pgfpathlineto{\pgfqpoint{5.262607in}{0.550000in}}%
\pgfpathclose%
\pgfusepath{fill}%
\end{pgfscope}%
\begin{pgfscope}%
\pgfpathrectangle{\pgfqpoint{1.250000in}{0.550000in}}{\pgfqpoint{7.750000in}{3.850000in}}%
\pgfusepath{clip}%
\pgfsetbuttcap%
\pgfsetmiterjoin%
\definecolor{currentfill}{rgb}{0.501961,0.501961,0.501961}%
\pgfsetfillcolor{currentfill}%
\pgfsetfillopacity{0.700000}%
\pgfsetlinewidth{0.000000pt}%
\definecolor{currentstroke}{rgb}{0.000000,0.000000,0.000000}%
\pgfsetstrokecolor{currentstroke}%
\pgfsetstrokeopacity{0.700000}%
\pgfsetdash{}{0pt}%
\pgfpathmoveto{\pgfqpoint{5.290128in}{0.550000in}}%
\pgfpathlineto{\pgfqpoint{5.317649in}{0.550000in}}%
\pgfpathlineto{\pgfqpoint{5.317649in}{0.579091in}}%
\pgfpathlineto{\pgfqpoint{5.290128in}{0.579091in}}%
\pgfpathlineto{\pgfqpoint{5.290128in}{0.550000in}}%
\pgfpathclose%
\pgfusepath{fill}%
\end{pgfscope}%
\begin{pgfscope}%
\pgfpathrectangle{\pgfqpoint{1.250000in}{0.550000in}}{\pgfqpoint{7.750000in}{3.850000in}}%
\pgfusepath{clip}%
\pgfsetbuttcap%
\pgfsetmiterjoin%
\definecolor{currentfill}{rgb}{0.501961,0.501961,0.501961}%
\pgfsetfillcolor{currentfill}%
\pgfsetfillopacity{0.700000}%
\pgfsetlinewidth{0.000000pt}%
\definecolor{currentstroke}{rgb}{0.000000,0.000000,0.000000}%
\pgfsetstrokecolor{currentstroke}%
\pgfsetstrokeopacity{0.700000}%
\pgfsetdash{}{0pt}%
\pgfpathmoveto{\pgfqpoint{5.317649in}{0.550000in}}%
\pgfpathlineto{\pgfqpoint{5.345170in}{0.550000in}}%
\pgfpathlineto{\pgfqpoint{5.345170in}{0.579550in}}%
\pgfpathlineto{\pgfqpoint{5.317649in}{0.579550in}}%
\pgfpathlineto{\pgfqpoint{5.317649in}{0.550000in}}%
\pgfpathclose%
\pgfusepath{fill}%
\end{pgfscope}%
\begin{pgfscope}%
\pgfpathrectangle{\pgfqpoint{1.250000in}{0.550000in}}{\pgfqpoint{7.750000in}{3.850000in}}%
\pgfusepath{clip}%
\pgfsetbuttcap%
\pgfsetmiterjoin%
\definecolor{currentfill}{rgb}{0.501961,0.501961,0.501961}%
\pgfsetfillcolor{currentfill}%
\pgfsetfillopacity{0.700000}%
\pgfsetlinewidth{0.000000pt}%
\definecolor{currentstroke}{rgb}{0.000000,0.000000,0.000000}%
\pgfsetstrokecolor{currentstroke}%
\pgfsetstrokeopacity{0.700000}%
\pgfsetdash{}{0pt}%
\pgfpathmoveto{\pgfqpoint{5.345170in}{0.550000in}}%
\pgfpathlineto{\pgfqpoint{5.372692in}{0.550000in}}%
\pgfpathlineto{\pgfqpoint{5.372692in}{0.575263in}}%
\pgfpathlineto{\pgfqpoint{5.345170in}{0.575263in}}%
\pgfpathlineto{\pgfqpoint{5.345170in}{0.550000in}}%
\pgfpathclose%
\pgfusepath{fill}%
\end{pgfscope}%
\begin{pgfscope}%
\pgfpathrectangle{\pgfqpoint{1.250000in}{0.550000in}}{\pgfqpoint{7.750000in}{3.850000in}}%
\pgfusepath{clip}%
\pgfsetbuttcap%
\pgfsetmiterjoin%
\definecolor{currentfill}{rgb}{0.501961,0.501961,0.501961}%
\pgfsetfillcolor{currentfill}%
\pgfsetfillopacity{0.700000}%
\pgfsetlinewidth{0.000000pt}%
\definecolor{currentstroke}{rgb}{0.000000,0.000000,0.000000}%
\pgfsetstrokecolor{currentstroke}%
\pgfsetstrokeopacity{0.700000}%
\pgfsetdash{}{0pt}%
\pgfpathmoveto{\pgfqpoint{5.372692in}{0.550000in}}%
\pgfpathlineto{\pgfqpoint{5.400213in}{0.550000in}}%
\pgfpathlineto{\pgfqpoint{5.400213in}{0.574498in}}%
\pgfpathlineto{\pgfqpoint{5.372692in}{0.574498in}}%
\pgfpathlineto{\pgfqpoint{5.372692in}{0.550000in}}%
\pgfpathclose%
\pgfusepath{fill}%
\end{pgfscope}%
\begin{pgfscope}%
\pgfpathrectangle{\pgfqpoint{1.250000in}{0.550000in}}{\pgfqpoint{7.750000in}{3.850000in}}%
\pgfusepath{clip}%
\pgfsetbuttcap%
\pgfsetmiterjoin%
\definecolor{currentfill}{rgb}{0.501961,0.501961,0.501961}%
\pgfsetfillcolor{currentfill}%
\pgfsetfillopacity{0.700000}%
\pgfsetlinewidth{0.000000pt}%
\definecolor{currentstroke}{rgb}{0.000000,0.000000,0.000000}%
\pgfsetstrokecolor{currentstroke}%
\pgfsetstrokeopacity{0.700000}%
\pgfsetdash{}{0pt}%
\pgfpathmoveto{\pgfqpoint{5.400213in}{0.550000in}}%
\pgfpathlineto{\pgfqpoint{5.427734in}{0.550000in}}%
\pgfpathlineto{\pgfqpoint{5.427734in}{0.574804in}}%
\pgfpathlineto{\pgfqpoint{5.400213in}{0.574804in}}%
\pgfpathlineto{\pgfqpoint{5.400213in}{0.550000in}}%
\pgfpathclose%
\pgfusepath{fill}%
\end{pgfscope}%
\begin{pgfscope}%
\pgfpathrectangle{\pgfqpoint{1.250000in}{0.550000in}}{\pgfqpoint{7.750000in}{3.850000in}}%
\pgfusepath{clip}%
\pgfsetbuttcap%
\pgfsetmiterjoin%
\definecolor{currentfill}{rgb}{0.501961,0.501961,0.501961}%
\pgfsetfillcolor{currentfill}%
\pgfsetfillopacity{0.700000}%
\pgfsetlinewidth{0.000000pt}%
\definecolor{currentstroke}{rgb}{0.000000,0.000000,0.000000}%
\pgfsetstrokecolor{currentstroke}%
\pgfsetstrokeopacity{0.700000}%
\pgfsetdash{}{0pt}%
\pgfpathmoveto{\pgfqpoint{5.427734in}{0.550000in}}%
\pgfpathlineto{\pgfqpoint{5.455256in}{0.550000in}}%
\pgfpathlineto{\pgfqpoint{5.455256in}{0.573426in}}%
\pgfpathlineto{\pgfqpoint{5.427734in}{0.573426in}}%
\pgfpathlineto{\pgfqpoint{5.427734in}{0.550000in}}%
\pgfpathclose%
\pgfusepath{fill}%
\end{pgfscope}%
\begin{pgfscope}%
\pgfpathrectangle{\pgfqpoint{1.250000in}{0.550000in}}{\pgfqpoint{7.750000in}{3.850000in}}%
\pgfusepath{clip}%
\pgfsetbuttcap%
\pgfsetmiterjoin%
\definecolor{currentfill}{rgb}{0.501961,0.501961,0.501961}%
\pgfsetfillcolor{currentfill}%
\pgfsetfillopacity{0.700000}%
\pgfsetlinewidth{0.000000pt}%
\definecolor{currentstroke}{rgb}{0.000000,0.000000,0.000000}%
\pgfsetstrokecolor{currentstroke}%
\pgfsetstrokeopacity{0.700000}%
\pgfsetdash{}{0pt}%
\pgfpathmoveto{\pgfqpoint{5.455256in}{0.550000in}}%
\pgfpathlineto{\pgfqpoint{5.482777in}{0.550000in}}%
\pgfpathlineto{\pgfqpoint{5.482777in}{0.568832in}}%
\pgfpathlineto{\pgfqpoint{5.455256in}{0.568832in}}%
\pgfpathlineto{\pgfqpoint{5.455256in}{0.550000in}}%
\pgfpathclose%
\pgfusepath{fill}%
\end{pgfscope}%
\begin{pgfscope}%
\pgfpathrectangle{\pgfqpoint{1.250000in}{0.550000in}}{\pgfqpoint{7.750000in}{3.850000in}}%
\pgfusepath{clip}%
\pgfsetbuttcap%
\pgfsetmiterjoin%
\definecolor{currentfill}{rgb}{0.501961,0.501961,0.501961}%
\pgfsetfillcolor{currentfill}%
\pgfsetfillopacity{0.700000}%
\pgfsetlinewidth{0.000000pt}%
\definecolor{currentstroke}{rgb}{0.000000,0.000000,0.000000}%
\pgfsetstrokecolor{currentstroke}%
\pgfsetstrokeopacity{0.700000}%
\pgfsetdash{}{0pt}%
\pgfpathmoveto{\pgfqpoint{5.482777in}{0.550000in}}%
\pgfpathlineto{\pgfqpoint{5.510298in}{0.550000in}}%
\pgfpathlineto{\pgfqpoint{5.510298in}{0.566995in}}%
\pgfpathlineto{\pgfqpoint{5.482777in}{0.566995in}}%
\pgfpathlineto{\pgfqpoint{5.482777in}{0.550000in}}%
\pgfpathclose%
\pgfusepath{fill}%
\end{pgfscope}%
\begin{pgfscope}%
\pgfpathrectangle{\pgfqpoint{1.250000in}{0.550000in}}{\pgfqpoint{7.750000in}{3.850000in}}%
\pgfusepath{clip}%
\pgfsetbuttcap%
\pgfsetmiterjoin%
\definecolor{currentfill}{rgb}{0.501961,0.501961,0.501961}%
\pgfsetfillcolor{currentfill}%
\pgfsetfillopacity{0.700000}%
\pgfsetlinewidth{0.000000pt}%
\definecolor{currentstroke}{rgb}{0.000000,0.000000,0.000000}%
\pgfsetstrokecolor{currentstroke}%
\pgfsetstrokeopacity{0.700000}%
\pgfsetdash{}{0pt}%
\pgfpathmoveto{\pgfqpoint{5.510298in}{0.550000in}}%
\pgfpathlineto{\pgfqpoint{5.537820in}{0.550000in}}%
\pgfpathlineto{\pgfqpoint{5.537820in}{0.566383in}}%
\pgfpathlineto{\pgfqpoint{5.510298in}{0.566383in}}%
\pgfpathlineto{\pgfqpoint{5.510298in}{0.550000in}}%
\pgfpathclose%
\pgfusepath{fill}%
\end{pgfscope}%
\begin{pgfscope}%
\pgfpathrectangle{\pgfqpoint{1.250000in}{0.550000in}}{\pgfqpoint{7.750000in}{3.850000in}}%
\pgfusepath{clip}%
\pgfsetbuttcap%
\pgfsetmiterjoin%
\definecolor{currentfill}{rgb}{0.501961,0.501961,0.501961}%
\pgfsetfillcolor{currentfill}%
\pgfsetfillopacity{0.700000}%
\pgfsetlinewidth{0.000000pt}%
\definecolor{currentstroke}{rgb}{0.000000,0.000000,0.000000}%
\pgfsetstrokecolor{currentstroke}%
\pgfsetstrokeopacity{0.700000}%
\pgfsetdash{}{0pt}%
\pgfpathmoveto{\pgfqpoint{5.537820in}{0.550000in}}%
\pgfpathlineto{\pgfqpoint{5.565341in}{0.550000in}}%
\pgfpathlineto{\pgfqpoint{5.565341in}{0.561483in}}%
\pgfpathlineto{\pgfqpoint{5.537820in}{0.561483in}}%
\pgfpathlineto{\pgfqpoint{5.537820in}{0.550000in}}%
\pgfpathclose%
\pgfusepath{fill}%
\end{pgfscope}%
\begin{pgfscope}%
\pgfpathrectangle{\pgfqpoint{1.250000in}{0.550000in}}{\pgfqpoint{7.750000in}{3.850000in}}%
\pgfusepath{clip}%
\pgfsetbuttcap%
\pgfsetmiterjoin%
\definecolor{currentfill}{rgb}{0.501961,0.501961,0.501961}%
\pgfsetfillcolor{currentfill}%
\pgfsetfillopacity{0.700000}%
\pgfsetlinewidth{0.000000pt}%
\definecolor{currentstroke}{rgb}{0.000000,0.000000,0.000000}%
\pgfsetstrokecolor{currentstroke}%
\pgfsetstrokeopacity{0.700000}%
\pgfsetdash{}{0pt}%
\pgfpathmoveto{\pgfqpoint{5.565341in}{0.550000in}}%
\pgfpathlineto{\pgfqpoint{5.592862in}{0.550000in}}%
\pgfpathlineto{\pgfqpoint{5.592862in}{0.563167in}}%
\pgfpathlineto{\pgfqpoint{5.565341in}{0.563167in}}%
\pgfpathlineto{\pgfqpoint{5.565341in}{0.550000in}}%
\pgfpathclose%
\pgfusepath{fill}%
\end{pgfscope}%
\begin{pgfscope}%
\pgfpathrectangle{\pgfqpoint{1.250000in}{0.550000in}}{\pgfqpoint{7.750000in}{3.850000in}}%
\pgfusepath{clip}%
\pgfsetbuttcap%
\pgfsetmiterjoin%
\definecolor{currentfill}{rgb}{0.501961,0.501961,0.501961}%
\pgfsetfillcolor{currentfill}%
\pgfsetfillopacity{0.700000}%
\pgfsetlinewidth{0.000000pt}%
\definecolor{currentstroke}{rgb}{0.000000,0.000000,0.000000}%
\pgfsetstrokecolor{currentstroke}%
\pgfsetstrokeopacity{0.700000}%
\pgfsetdash{}{0pt}%
\pgfpathmoveto{\pgfqpoint{5.592862in}{0.550000in}}%
\pgfpathlineto{\pgfqpoint{5.620384in}{0.550000in}}%
\pgfpathlineto{\pgfqpoint{5.620384in}{0.561636in}}%
\pgfpathlineto{\pgfqpoint{5.592862in}{0.561636in}}%
\pgfpathlineto{\pgfqpoint{5.592862in}{0.550000in}}%
\pgfpathclose%
\pgfusepath{fill}%
\end{pgfscope}%
\begin{pgfscope}%
\pgfpathrectangle{\pgfqpoint{1.250000in}{0.550000in}}{\pgfqpoint{7.750000in}{3.850000in}}%
\pgfusepath{clip}%
\pgfsetbuttcap%
\pgfsetmiterjoin%
\definecolor{currentfill}{rgb}{0.501961,0.501961,0.501961}%
\pgfsetfillcolor{currentfill}%
\pgfsetfillopacity{0.700000}%
\pgfsetlinewidth{0.000000pt}%
\definecolor{currentstroke}{rgb}{0.000000,0.000000,0.000000}%
\pgfsetstrokecolor{currentstroke}%
\pgfsetstrokeopacity{0.700000}%
\pgfsetdash{}{0pt}%
\pgfpathmoveto{\pgfqpoint{5.620384in}{0.550000in}}%
\pgfpathlineto{\pgfqpoint{5.647905in}{0.550000in}}%
\pgfpathlineto{\pgfqpoint{5.647905in}{0.560258in}}%
\pgfpathlineto{\pgfqpoint{5.620384in}{0.560258in}}%
\pgfpathlineto{\pgfqpoint{5.620384in}{0.550000in}}%
\pgfpathclose%
\pgfusepath{fill}%
\end{pgfscope}%
\begin{pgfscope}%
\pgfpathrectangle{\pgfqpoint{1.250000in}{0.550000in}}{\pgfqpoint{7.750000in}{3.850000in}}%
\pgfusepath{clip}%
\pgfsetbuttcap%
\pgfsetmiterjoin%
\definecolor{currentfill}{rgb}{0.501961,0.501961,0.501961}%
\pgfsetfillcolor{currentfill}%
\pgfsetfillopacity{0.700000}%
\pgfsetlinewidth{0.000000pt}%
\definecolor{currentstroke}{rgb}{0.000000,0.000000,0.000000}%
\pgfsetstrokecolor{currentstroke}%
\pgfsetstrokeopacity{0.700000}%
\pgfsetdash{}{0pt}%
\pgfpathmoveto{\pgfqpoint{5.647905in}{0.550000in}}%
\pgfpathlineto{\pgfqpoint{5.675426in}{0.550000in}}%
\pgfpathlineto{\pgfqpoint{5.675426in}{0.559799in}}%
\pgfpathlineto{\pgfqpoint{5.647905in}{0.559799in}}%
\pgfpathlineto{\pgfqpoint{5.647905in}{0.550000in}}%
\pgfpathclose%
\pgfusepath{fill}%
\end{pgfscope}%
\begin{pgfscope}%
\pgfpathrectangle{\pgfqpoint{1.250000in}{0.550000in}}{\pgfqpoint{7.750000in}{3.850000in}}%
\pgfusepath{clip}%
\pgfsetbuttcap%
\pgfsetmiterjoin%
\definecolor{currentfill}{rgb}{0.501961,0.501961,0.501961}%
\pgfsetfillcolor{currentfill}%
\pgfsetfillopacity{0.700000}%
\pgfsetlinewidth{0.000000pt}%
\definecolor{currentstroke}{rgb}{0.000000,0.000000,0.000000}%
\pgfsetstrokecolor{currentstroke}%
\pgfsetstrokeopacity{0.700000}%
\pgfsetdash{}{0pt}%
\pgfpathmoveto{\pgfqpoint{5.675426in}{0.550000in}}%
\pgfpathlineto{\pgfqpoint{5.702947in}{0.550000in}}%
\pgfpathlineto{\pgfqpoint{5.702947in}{0.559493in}}%
\pgfpathlineto{\pgfqpoint{5.675426in}{0.559493in}}%
\pgfpathlineto{\pgfqpoint{5.675426in}{0.550000in}}%
\pgfpathclose%
\pgfusepath{fill}%
\end{pgfscope}%
\begin{pgfscope}%
\pgfpathrectangle{\pgfqpoint{1.250000in}{0.550000in}}{\pgfqpoint{7.750000in}{3.850000in}}%
\pgfusepath{clip}%
\pgfsetbuttcap%
\pgfsetmiterjoin%
\definecolor{currentfill}{rgb}{0.501961,0.501961,0.501961}%
\pgfsetfillcolor{currentfill}%
\pgfsetfillopacity{0.700000}%
\pgfsetlinewidth{0.000000pt}%
\definecolor{currentstroke}{rgb}{0.000000,0.000000,0.000000}%
\pgfsetstrokecolor{currentstroke}%
\pgfsetstrokeopacity{0.700000}%
\pgfsetdash{}{0pt}%
\pgfpathmoveto{\pgfqpoint{5.702947in}{0.550000in}}%
\pgfpathlineto{\pgfqpoint{5.730469in}{0.550000in}}%
\pgfpathlineto{\pgfqpoint{5.730469in}{0.559033in}}%
\pgfpathlineto{\pgfqpoint{5.702947in}{0.559033in}}%
\pgfpathlineto{\pgfqpoint{5.702947in}{0.550000in}}%
\pgfpathclose%
\pgfusepath{fill}%
\end{pgfscope}%
\begin{pgfscope}%
\pgfpathrectangle{\pgfqpoint{1.250000in}{0.550000in}}{\pgfqpoint{7.750000in}{3.850000in}}%
\pgfusepath{clip}%
\pgfsetbuttcap%
\pgfsetmiterjoin%
\definecolor{currentfill}{rgb}{0.501961,0.501961,0.501961}%
\pgfsetfillcolor{currentfill}%
\pgfsetfillopacity{0.700000}%
\pgfsetlinewidth{0.000000pt}%
\definecolor{currentstroke}{rgb}{0.000000,0.000000,0.000000}%
\pgfsetstrokecolor{currentstroke}%
\pgfsetstrokeopacity{0.700000}%
\pgfsetdash{}{0pt}%
\pgfpathmoveto{\pgfqpoint{5.730469in}{0.550000in}}%
\pgfpathlineto{\pgfqpoint{5.757990in}{0.550000in}}%
\pgfpathlineto{\pgfqpoint{5.757990in}{0.556124in}}%
\pgfpathlineto{\pgfqpoint{5.730469in}{0.556124in}}%
\pgfpathlineto{\pgfqpoint{5.730469in}{0.550000in}}%
\pgfpathclose%
\pgfusepath{fill}%
\end{pgfscope}%
\begin{pgfscope}%
\pgfpathrectangle{\pgfqpoint{1.250000in}{0.550000in}}{\pgfqpoint{7.750000in}{3.850000in}}%
\pgfusepath{clip}%
\pgfsetbuttcap%
\pgfsetmiterjoin%
\definecolor{currentfill}{rgb}{0.501961,0.501961,0.501961}%
\pgfsetfillcolor{currentfill}%
\pgfsetfillopacity{0.700000}%
\pgfsetlinewidth{0.000000pt}%
\definecolor{currentstroke}{rgb}{0.000000,0.000000,0.000000}%
\pgfsetstrokecolor{currentstroke}%
\pgfsetstrokeopacity{0.700000}%
\pgfsetdash{}{0pt}%
\pgfpathmoveto{\pgfqpoint{5.757990in}{0.550000in}}%
\pgfpathlineto{\pgfqpoint{5.785511in}{0.550000in}}%
\pgfpathlineto{\pgfqpoint{5.785511in}{0.557502in}}%
\pgfpathlineto{\pgfqpoint{5.757990in}{0.557502in}}%
\pgfpathlineto{\pgfqpoint{5.757990in}{0.550000in}}%
\pgfpathclose%
\pgfusepath{fill}%
\end{pgfscope}%
\begin{pgfscope}%
\pgfpathrectangle{\pgfqpoint{1.250000in}{0.550000in}}{\pgfqpoint{7.750000in}{3.850000in}}%
\pgfusepath{clip}%
\pgfsetbuttcap%
\pgfsetmiterjoin%
\definecolor{currentfill}{rgb}{0.501961,0.501961,0.501961}%
\pgfsetfillcolor{currentfill}%
\pgfsetfillopacity{0.700000}%
\pgfsetlinewidth{0.000000pt}%
\definecolor{currentstroke}{rgb}{0.000000,0.000000,0.000000}%
\pgfsetstrokecolor{currentstroke}%
\pgfsetstrokeopacity{0.700000}%
\pgfsetdash{}{0pt}%
\pgfpathmoveto{\pgfqpoint{5.785511in}{0.550000in}}%
\pgfpathlineto{\pgfqpoint{5.813033in}{0.550000in}}%
\pgfpathlineto{\pgfqpoint{5.813033in}{0.556890in}}%
\pgfpathlineto{\pgfqpoint{5.785511in}{0.556890in}}%
\pgfpathlineto{\pgfqpoint{5.785511in}{0.550000in}}%
\pgfpathclose%
\pgfusepath{fill}%
\end{pgfscope}%
\begin{pgfscope}%
\pgfpathrectangle{\pgfqpoint{1.250000in}{0.550000in}}{\pgfqpoint{7.750000in}{3.850000in}}%
\pgfusepath{clip}%
\pgfsetbuttcap%
\pgfsetmiterjoin%
\definecolor{currentfill}{rgb}{0.501961,0.501961,0.501961}%
\pgfsetfillcolor{currentfill}%
\pgfsetfillopacity{0.700000}%
\pgfsetlinewidth{0.000000pt}%
\definecolor{currentstroke}{rgb}{0.000000,0.000000,0.000000}%
\pgfsetstrokecolor{currentstroke}%
\pgfsetstrokeopacity{0.700000}%
\pgfsetdash{}{0pt}%
\pgfpathmoveto{\pgfqpoint{5.813033in}{0.550000in}}%
\pgfpathlineto{\pgfqpoint{5.840554in}{0.550000in}}%
\pgfpathlineto{\pgfqpoint{5.840554in}{0.558574in}}%
\pgfpathlineto{\pgfqpoint{5.813033in}{0.558574in}}%
\pgfpathlineto{\pgfqpoint{5.813033in}{0.550000in}}%
\pgfpathclose%
\pgfusepath{fill}%
\end{pgfscope}%
\begin{pgfscope}%
\pgfpathrectangle{\pgfqpoint{1.250000in}{0.550000in}}{\pgfqpoint{7.750000in}{3.850000in}}%
\pgfusepath{clip}%
\pgfsetbuttcap%
\pgfsetmiterjoin%
\definecolor{currentfill}{rgb}{0.501961,0.501961,0.501961}%
\pgfsetfillcolor{currentfill}%
\pgfsetfillopacity{0.700000}%
\pgfsetlinewidth{0.000000pt}%
\definecolor{currentstroke}{rgb}{0.000000,0.000000,0.000000}%
\pgfsetstrokecolor{currentstroke}%
\pgfsetstrokeopacity{0.700000}%
\pgfsetdash{}{0pt}%
\pgfpathmoveto{\pgfqpoint{5.840554in}{0.550000in}}%
\pgfpathlineto{\pgfqpoint{5.868075in}{0.550000in}}%
\pgfpathlineto{\pgfqpoint{5.868075in}{0.556431in}}%
\pgfpathlineto{\pgfqpoint{5.840554in}{0.556431in}}%
\pgfpathlineto{\pgfqpoint{5.840554in}{0.550000in}}%
\pgfpathclose%
\pgfusepath{fill}%
\end{pgfscope}%
\begin{pgfscope}%
\pgfpathrectangle{\pgfqpoint{1.250000in}{0.550000in}}{\pgfqpoint{7.750000in}{3.850000in}}%
\pgfusepath{clip}%
\pgfsetbuttcap%
\pgfsetmiterjoin%
\definecolor{currentfill}{rgb}{0.501961,0.501961,0.501961}%
\pgfsetfillcolor{currentfill}%
\pgfsetfillopacity{0.700000}%
\pgfsetlinewidth{0.000000pt}%
\definecolor{currentstroke}{rgb}{0.000000,0.000000,0.000000}%
\pgfsetstrokecolor{currentstroke}%
\pgfsetstrokeopacity{0.700000}%
\pgfsetdash{}{0pt}%
\pgfpathmoveto{\pgfqpoint{5.868075in}{0.550000in}}%
\pgfpathlineto{\pgfqpoint{5.895597in}{0.550000in}}%
\pgfpathlineto{\pgfqpoint{5.895597in}{0.558115in}}%
\pgfpathlineto{\pgfqpoint{5.868075in}{0.558115in}}%
\pgfpathlineto{\pgfqpoint{5.868075in}{0.550000in}}%
\pgfpathclose%
\pgfusepath{fill}%
\end{pgfscope}%
\begin{pgfscope}%
\pgfpathrectangle{\pgfqpoint{1.250000in}{0.550000in}}{\pgfqpoint{7.750000in}{3.850000in}}%
\pgfusepath{clip}%
\pgfsetbuttcap%
\pgfsetmiterjoin%
\definecolor{currentfill}{rgb}{0.501961,0.501961,0.501961}%
\pgfsetfillcolor{currentfill}%
\pgfsetfillopacity{0.700000}%
\pgfsetlinewidth{0.000000pt}%
\definecolor{currentstroke}{rgb}{0.000000,0.000000,0.000000}%
\pgfsetstrokecolor{currentstroke}%
\pgfsetstrokeopacity{0.700000}%
\pgfsetdash{}{0pt}%
\pgfpathmoveto{\pgfqpoint{5.895597in}{0.550000in}}%
\pgfpathlineto{\pgfqpoint{5.923118in}{0.550000in}}%
\pgfpathlineto{\pgfqpoint{5.923118in}{0.557502in}}%
\pgfpathlineto{\pgfqpoint{5.895597in}{0.557502in}}%
\pgfpathlineto{\pgfqpoint{5.895597in}{0.550000in}}%
\pgfpathclose%
\pgfusepath{fill}%
\end{pgfscope}%
\begin{pgfscope}%
\pgfpathrectangle{\pgfqpoint{1.250000in}{0.550000in}}{\pgfqpoint{7.750000in}{3.850000in}}%
\pgfusepath{clip}%
\pgfsetbuttcap%
\pgfsetmiterjoin%
\definecolor{currentfill}{rgb}{0.501961,0.501961,0.501961}%
\pgfsetfillcolor{currentfill}%
\pgfsetfillopacity{0.700000}%
\pgfsetlinewidth{0.000000pt}%
\definecolor{currentstroke}{rgb}{0.000000,0.000000,0.000000}%
\pgfsetstrokecolor{currentstroke}%
\pgfsetstrokeopacity{0.700000}%
\pgfsetdash{}{0pt}%
\pgfpathmoveto{\pgfqpoint{5.923118in}{0.550000in}}%
\pgfpathlineto{\pgfqpoint{5.950639in}{0.550000in}}%
\pgfpathlineto{\pgfqpoint{5.950639in}{0.556584in}}%
\pgfpathlineto{\pgfqpoint{5.923118in}{0.556584in}}%
\pgfpathlineto{\pgfqpoint{5.923118in}{0.550000in}}%
\pgfpathclose%
\pgfusepath{fill}%
\end{pgfscope}%
\begin{pgfscope}%
\pgfpathrectangle{\pgfqpoint{1.250000in}{0.550000in}}{\pgfqpoint{7.750000in}{3.850000in}}%
\pgfusepath{clip}%
\pgfsetbuttcap%
\pgfsetmiterjoin%
\definecolor{currentfill}{rgb}{0.501961,0.501961,0.501961}%
\pgfsetfillcolor{currentfill}%
\pgfsetfillopacity{0.700000}%
\pgfsetlinewidth{0.000000pt}%
\definecolor{currentstroke}{rgb}{0.000000,0.000000,0.000000}%
\pgfsetstrokecolor{currentstroke}%
\pgfsetstrokeopacity{0.700000}%
\pgfsetdash{}{0pt}%
\pgfpathmoveto{\pgfqpoint{5.950639in}{0.550000in}}%
\pgfpathlineto{\pgfqpoint{5.978161in}{0.550000in}}%
\pgfpathlineto{\pgfqpoint{5.978161in}{0.555818in}}%
\pgfpathlineto{\pgfqpoint{5.950639in}{0.555818in}}%
\pgfpathlineto{\pgfqpoint{5.950639in}{0.550000in}}%
\pgfpathclose%
\pgfusepath{fill}%
\end{pgfscope}%
\begin{pgfscope}%
\pgfpathrectangle{\pgfqpoint{1.250000in}{0.550000in}}{\pgfqpoint{7.750000in}{3.850000in}}%
\pgfusepath{clip}%
\pgfsetbuttcap%
\pgfsetmiterjoin%
\definecolor{currentfill}{rgb}{0.501961,0.501961,0.501961}%
\pgfsetfillcolor{currentfill}%
\pgfsetfillopacity{0.700000}%
\pgfsetlinewidth{0.000000pt}%
\definecolor{currentstroke}{rgb}{0.000000,0.000000,0.000000}%
\pgfsetstrokecolor{currentstroke}%
\pgfsetstrokeopacity{0.700000}%
\pgfsetdash{}{0pt}%
\pgfpathmoveto{\pgfqpoint{5.978161in}{0.550000in}}%
\pgfpathlineto{\pgfqpoint{6.005682in}{0.550000in}}%
\pgfpathlineto{\pgfqpoint{6.005682in}{0.556890in}}%
\pgfpathlineto{\pgfqpoint{5.978161in}{0.556890in}}%
\pgfpathlineto{\pgfqpoint{5.978161in}{0.550000in}}%
\pgfpathclose%
\pgfusepath{fill}%
\end{pgfscope}%
\begin{pgfscope}%
\pgfpathrectangle{\pgfqpoint{1.250000in}{0.550000in}}{\pgfqpoint{7.750000in}{3.850000in}}%
\pgfusepath{clip}%
\pgfsetbuttcap%
\pgfsetmiterjoin%
\definecolor{currentfill}{rgb}{0.501961,0.501961,0.501961}%
\pgfsetfillcolor{currentfill}%
\pgfsetfillopacity{0.700000}%
\pgfsetlinewidth{0.000000pt}%
\definecolor{currentstroke}{rgb}{0.000000,0.000000,0.000000}%
\pgfsetstrokecolor{currentstroke}%
\pgfsetstrokeopacity{0.700000}%
\pgfsetdash{}{0pt}%
\pgfpathmoveto{\pgfqpoint{6.005682in}{0.550000in}}%
\pgfpathlineto{\pgfqpoint{6.033203in}{0.550000in}}%
\pgfpathlineto{\pgfqpoint{6.033203in}{0.555206in}}%
\pgfpathlineto{\pgfqpoint{6.005682in}{0.555206in}}%
\pgfpathlineto{\pgfqpoint{6.005682in}{0.550000in}}%
\pgfpathclose%
\pgfusepath{fill}%
\end{pgfscope}%
\begin{pgfscope}%
\pgfpathrectangle{\pgfqpoint{1.250000in}{0.550000in}}{\pgfqpoint{7.750000in}{3.850000in}}%
\pgfusepath{clip}%
\pgfsetbuttcap%
\pgfsetmiterjoin%
\definecolor{currentfill}{rgb}{0.501961,0.501961,0.501961}%
\pgfsetfillcolor{currentfill}%
\pgfsetfillopacity{0.700000}%
\pgfsetlinewidth{0.000000pt}%
\definecolor{currentstroke}{rgb}{0.000000,0.000000,0.000000}%
\pgfsetstrokecolor{currentstroke}%
\pgfsetstrokeopacity{0.700000}%
\pgfsetdash{}{0pt}%
\pgfpathmoveto{\pgfqpoint{6.033203in}{0.550000in}}%
\pgfpathlineto{\pgfqpoint{6.060724in}{0.550000in}}%
\pgfpathlineto{\pgfqpoint{6.060724in}{0.556584in}}%
\pgfpathlineto{\pgfqpoint{6.033203in}{0.556584in}}%
\pgfpathlineto{\pgfqpoint{6.033203in}{0.550000in}}%
\pgfpathclose%
\pgfusepath{fill}%
\end{pgfscope}%
\begin{pgfscope}%
\pgfpathrectangle{\pgfqpoint{1.250000in}{0.550000in}}{\pgfqpoint{7.750000in}{3.850000in}}%
\pgfusepath{clip}%
\pgfsetbuttcap%
\pgfsetmiterjoin%
\definecolor{currentfill}{rgb}{0.501961,0.501961,0.501961}%
\pgfsetfillcolor{currentfill}%
\pgfsetfillopacity{0.700000}%
\pgfsetlinewidth{0.000000pt}%
\definecolor{currentstroke}{rgb}{0.000000,0.000000,0.000000}%
\pgfsetstrokecolor{currentstroke}%
\pgfsetstrokeopacity{0.700000}%
\pgfsetdash{}{0pt}%
\pgfpathmoveto{\pgfqpoint{6.060724in}{0.550000in}}%
\pgfpathlineto{\pgfqpoint{6.088246in}{0.550000in}}%
\pgfpathlineto{\pgfqpoint{6.088246in}{0.557502in}}%
\pgfpathlineto{\pgfqpoint{6.060724in}{0.557502in}}%
\pgfpathlineto{\pgfqpoint{6.060724in}{0.550000in}}%
\pgfpathclose%
\pgfusepath{fill}%
\end{pgfscope}%
\begin{pgfscope}%
\pgfpathrectangle{\pgfqpoint{1.250000in}{0.550000in}}{\pgfqpoint{7.750000in}{3.850000in}}%
\pgfusepath{clip}%
\pgfsetbuttcap%
\pgfsetmiterjoin%
\definecolor{currentfill}{rgb}{0.501961,0.501961,0.501961}%
\pgfsetfillcolor{currentfill}%
\pgfsetfillopacity{0.700000}%
\pgfsetlinewidth{0.000000pt}%
\definecolor{currentstroke}{rgb}{0.000000,0.000000,0.000000}%
\pgfsetstrokecolor{currentstroke}%
\pgfsetstrokeopacity{0.700000}%
\pgfsetdash{}{0pt}%
\pgfpathmoveto{\pgfqpoint{6.088246in}{0.550000in}}%
\pgfpathlineto{\pgfqpoint{6.115767in}{0.550000in}}%
\pgfpathlineto{\pgfqpoint{6.115767in}{0.555053in}}%
\pgfpathlineto{\pgfqpoint{6.088246in}{0.555053in}}%
\pgfpathlineto{\pgfqpoint{6.088246in}{0.550000in}}%
\pgfpathclose%
\pgfusepath{fill}%
\end{pgfscope}%
\begin{pgfscope}%
\pgfpathrectangle{\pgfqpoint{1.250000in}{0.550000in}}{\pgfqpoint{7.750000in}{3.850000in}}%
\pgfusepath{clip}%
\pgfsetbuttcap%
\pgfsetmiterjoin%
\definecolor{currentfill}{rgb}{0.501961,0.501961,0.501961}%
\pgfsetfillcolor{currentfill}%
\pgfsetfillopacity{0.700000}%
\pgfsetlinewidth{0.000000pt}%
\definecolor{currentstroke}{rgb}{0.000000,0.000000,0.000000}%
\pgfsetstrokecolor{currentstroke}%
\pgfsetstrokeopacity{0.700000}%
\pgfsetdash{}{0pt}%
\pgfpathmoveto{\pgfqpoint{6.115767in}{0.550000in}}%
\pgfpathlineto{\pgfqpoint{6.143288in}{0.550000in}}%
\pgfpathlineto{\pgfqpoint{6.143288in}{0.556890in}}%
\pgfpathlineto{\pgfqpoint{6.115767in}{0.556890in}}%
\pgfpathlineto{\pgfqpoint{6.115767in}{0.550000in}}%
\pgfpathclose%
\pgfusepath{fill}%
\end{pgfscope}%
\begin{pgfscope}%
\pgfpathrectangle{\pgfqpoint{1.250000in}{0.550000in}}{\pgfqpoint{7.750000in}{3.850000in}}%
\pgfusepath{clip}%
\pgfsetbuttcap%
\pgfsetmiterjoin%
\definecolor{currentfill}{rgb}{0.501961,0.501961,0.501961}%
\pgfsetfillcolor{currentfill}%
\pgfsetfillopacity{0.700000}%
\pgfsetlinewidth{0.000000pt}%
\definecolor{currentstroke}{rgb}{0.000000,0.000000,0.000000}%
\pgfsetstrokecolor{currentstroke}%
\pgfsetstrokeopacity{0.700000}%
\pgfsetdash{}{0pt}%
\pgfpathmoveto{\pgfqpoint{6.143288in}{0.550000in}}%
\pgfpathlineto{\pgfqpoint{6.170810in}{0.550000in}}%
\pgfpathlineto{\pgfqpoint{6.170810in}{0.553981in}}%
\pgfpathlineto{\pgfqpoint{6.143288in}{0.553981in}}%
\pgfpathlineto{\pgfqpoint{6.143288in}{0.550000in}}%
\pgfpathclose%
\pgfusepath{fill}%
\end{pgfscope}%
\begin{pgfscope}%
\pgfpathrectangle{\pgfqpoint{1.250000in}{0.550000in}}{\pgfqpoint{7.750000in}{3.850000in}}%
\pgfusepath{clip}%
\pgfsetbuttcap%
\pgfsetmiterjoin%
\definecolor{currentfill}{rgb}{0.501961,0.501961,0.501961}%
\pgfsetfillcolor{currentfill}%
\pgfsetfillopacity{0.700000}%
\pgfsetlinewidth{0.000000pt}%
\definecolor{currentstroke}{rgb}{0.000000,0.000000,0.000000}%
\pgfsetstrokecolor{currentstroke}%
\pgfsetstrokeopacity{0.700000}%
\pgfsetdash{}{0pt}%
\pgfpathmoveto{\pgfqpoint{6.170810in}{0.550000in}}%
\pgfpathlineto{\pgfqpoint{6.198331in}{0.550000in}}%
\pgfpathlineto{\pgfqpoint{6.198331in}{0.554900in}}%
\pgfpathlineto{\pgfqpoint{6.170810in}{0.554900in}}%
\pgfpathlineto{\pgfqpoint{6.170810in}{0.550000in}}%
\pgfpathclose%
\pgfusepath{fill}%
\end{pgfscope}%
\begin{pgfscope}%
\pgfpathrectangle{\pgfqpoint{1.250000in}{0.550000in}}{\pgfqpoint{7.750000in}{3.850000in}}%
\pgfusepath{clip}%
\pgfsetbuttcap%
\pgfsetmiterjoin%
\definecolor{currentfill}{rgb}{0.501961,0.501961,0.501961}%
\pgfsetfillcolor{currentfill}%
\pgfsetfillopacity{0.700000}%
\pgfsetlinewidth{0.000000pt}%
\definecolor{currentstroke}{rgb}{0.000000,0.000000,0.000000}%
\pgfsetstrokecolor{currentstroke}%
\pgfsetstrokeopacity{0.700000}%
\pgfsetdash{}{0pt}%
\pgfpathmoveto{\pgfqpoint{6.198331in}{0.550000in}}%
\pgfpathlineto{\pgfqpoint{6.225852in}{0.550000in}}%
\pgfpathlineto{\pgfqpoint{6.225852in}{0.555359in}}%
\pgfpathlineto{\pgfqpoint{6.198331in}{0.555359in}}%
\pgfpathlineto{\pgfqpoint{6.198331in}{0.550000in}}%
\pgfpathclose%
\pgfusepath{fill}%
\end{pgfscope}%
\begin{pgfscope}%
\pgfpathrectangle{\pgfqpoint{1.250000in}{0.550000in}}{\pgfqpoint{7.750000in}{3.850000in}}%
\pgfusepath{clip}%
\pgfsetbuttcap%
\pgfsetmiterjoin%
\definecolor{currentfill}{rgb}{0.501961,0.501961,0.501961}%
\pgfsetfillcolor{currentfill}%
\pgfsetfillopacity{0.700000}%
\pgfsetlinewidth{0.000000pt}%
\definecolor{currentstroke}{rgb}{0.000000,0.000000,0.000000}%
\pgfsetstrokecolor{currentstroke}%
\pgfsetstrokeopacity{0.700000}%
\pgfsetdash{}{0pt}%
\pgfpathmoveto{\pgfqpoint{6.225852in}{0.550000in}}%
\pgfpathlineto{\pgfqpoint{6.253374in}{0.550000in}}%
\pgfpathlineto{\pgfqpoint{6.253374in}{0.557043in}}%
\pgfpathlineto{\pgfqpoint{6.225852in}{0.557043in}}%
\pgfpathlineto{\pgfqpoint{6.225852in}{0.550000in}}%
\pgfpathclose%
\pgfusepath{fill}%
\end{pgfscope}%
\begin{pgfscope}%
\pgfpathrectangle{\pgfqpoint{1.250000in}{0.550000in}}{\pgfqpoint{7.750000in}{3.850000in}}%
\pgfusepath{clip}%
\pgfsetbuttcap%
\pgfsetmiterjoin%
\definecolor{currentfill}{rgb}{0.501961,0.501961,0.501961}%
\pgfsetfillcolor{currentfill}%
\pgfsetfillopacity{0.700000}%
\pgfsetlinewidth{0.000000pt}%
\definecolor{currentstroke}{rgb}{0.000000,0.000000,0.000000}%
\pgfsetstrokecolor{currentstroke}%
\pgfsetstrokeopacity{0.700000}%
\pgfsetdash{}{0pt}%
\pgfpathmoveto{\pgfqpoint{6.253374in}{0.550000in}}%
\pgfpathlineto{\pgfqpoint{6.280895in}{0.550000in}}%
\pgfpathlineto{\pgfqpoint{6.280895in}{0.555512in}}%
\pgfpathlineto{\pgfqpoint{6.253374in}{0.555512in}}%
\pgfpathlineto{\pgfqpoint{6.253374in}{0.550000in}}%
\pgfpathclose%
\pgfusepath{fill}%
\end{pgfscope}%
\begin{pgfscope}%
\pgfpathrectangle{\pgfqpoint{1.250000in}{0.550000in}}{\pgfqpoint{7.750000in}{3.850000in}}%
\pgfusepath{clip}%
\pgfsetbuttcap%
\pgfsetmiterjoin%
\definecolor{currentfill}{rgb}{0.501961,0.501961,0.501961}%
\pgfsetfillcolor{currentfill}%
\pgfsetfillopacity{0.700000}%
\pgfsetlinewidth{0.000000pt}%
\definecolor{currentstroke}{rgb}{0.000000,0.000000,0.000000}%
\pgfsetstrokecolor{currentstroke}%
\pgfsetstrokeopacity{0.700000}%
\pgfsetdash{}{0pt}%
\pgfpathmoveto{\pgfqpoint{6.280895in}{0.550000in}}%
\pgfpathlineto{\pgfqpoint{6.308416in}{0.550000in}}%
\pgfpathlineto{\pgfqpoint{6.308416in}{0.554746in}}%
\pgfpathlineto{\pgfqpoint{6.280895in}{0.554746in}}%
\pgfpathlineto{\pgfqpoint{6.280895in}{0.550000in}}%
\pgfpathclose%
\pgfusepath{fill}%
\end{pgfscope}%
\begin{pgfscope}%
\pgfpathrectangle{\pgfqpoint{1.250000in}{0.550000in}}{\pgfqpoint{7.750000in}{3.850000in}}%
\pgfusepath{clip}%
\pgfsetbuttcap%
\pgfsetmiterjoin%
\definecolor{currentfill}{rgb}{0.501961,0.501961,0.501961}%
\pgfsetfillcolor{currentfill}%
\pgfsetfillopacity{0.700000}%
\pgfsetlinewidth{0.000000pt}%
\definecolor{currentstroke}{rgb}{0.000000,0.000000,0.000000}%
\pgfsetstrokecolor{currentstroke}%
\pgfsetstrokeopacity{0.700000}%
\pgfsetdash{}{0pt}%
\pgfpathmoveto{\pgfqpoint{6.308416in}{0.550000in}}%
\pgfpathlineto{\pgfqpoint{6.335937in}{0.550000in}}%
\pgfpathlineto{\pgfqpoint{6.335937in}{0.553675in}}%
\pgfpathlineto{\pgfqpoint{6.308416in}{0.553675in}}%
\pgfpathlineto{\pgfqpoint{6.308416in}{0.550000in}}%
\pgfpathclose%
\pgfusepath{fill}%
\end{pgfscope}%
\begin{pgfscope}%
\pgfpathrectangle{\pgfqpoint{1.250000in}{0.550000in}}{\pgfqpoint{7.750000in}{3.850000in}}%
\pgfusepath{clip}%
\pgfsetbuttcap%
\pgfsetmiterjoin%
\definecolor{currentfill}{rgb}{0.501961,0.501961,0.501961}%
\pgfsetfillcolor{currentfill}%
\pgfsetfillopacity{0.700000}%
\pgfsetlinewidth{0.000000pt}%
\definecolor{currentstroke}{rgb}{0.000000,0.000000,0.000000}%
\pgfsetstrokecolor{currentstroke}%
\pgfsetstrokeopacity{0.700000}%
\pgfsetdash{}{0pt}%
\pgfpathmoveto{\pgfqpoint{6.335938in}{0.550000in}}%
\pgfpathlineto{\pgfqpoint{6.363459in}{0.550000in}}%
\pgfpathlineto{\pgfqpoint{6.363459in}{0.553368in}}%
\pgfpathlineto{\pgfqpoint{6.335938in}{0.553368in}}%
\pgfpathlineto{\pgfqpoint{6.335938in}{0.550000in}}%
\pgfpathclose%
\pgfusepath{fill}%
\end{pgfscope}%
\begin{pgfscope}%
\pgfpathrectangle{\pgfqpoint{1.250000in}{0.550000in}}{\pgfqpoint{7.750000in}{3.850000in}}%
\pgfusepath{clip}%
\pgfsetbuttcap%
\pgfsetmiterjoin%
\definecolor{currentfill}{rgb}{0.501961,0.501961,0.501961}%
\pgfsetfillcolor{currentfill}%
\pgfsetfillopacity{0.700000}%
\pgfsetlinewidth{0.000000pt}%
\definecolor{currentstroke}{rgb}{0.000000,0.000000,0.000000}%
\pgfsetstrokecolor{currentstroke}%
\pgfsetstrokeopacity{0.700000}%
\pgfsetdash{}{0pt}%
\pgfpathmoveto{\pgfqpoint{6.363459in}{0.550000in}}%
\pgfpathlineto{\pgfqpoint{6.390980in}{0.550000in}}%
\pgfpathlineto{\pgfqpoint{6.390980in}{0.554746in}}%
\pgfpathlineto{\pgfqpoint{6.363459in}{0.554746in}}%
\pgfpathlineto{\pgfqpoint{6.363459in}{0.550000in}}%
\pgfpathclose%
\pgfusepath{fill}%
\end{pgfscope}%
\begin{pgfscope}%
\pgfpathrectangle{\pgfqpoint{1.250000in}{0.550000in}}{\pgfqpoint{7.750000in}{3.850000in}}%
\pgfusepath{clip}%
\pgfsetbuttcap%
\pgfsetmiterjoin%
\definecolor{currentfill}{rgb}{0.501961,0.501961,0.501961}%
\pgfsetfillcolor{currentfill}%
\pgfsetfillopacity{0.700000}%
\pgfsetlinewidth{0.000000pt}%
\definecolor{currentstroke}{rgb}{0.000000,0.000000,0.000000}%
\pgfsetstrokecolor{currentstroke}%
\pgfsetstrokeopacity{0.700000}%
\pgfsetdash{}{0pt}%
\pgfpathmoveto{\pgfqpoint{6.390980in}{0.550000in}}%
\pgfpathlineto{\pgfqpoint{6.418501in}{0.550000in}}%
\pgfpathlineto{\pgfqpoint{6.418501in}{0.554746in}}%
\pgfpathlineto{\pgfqpoint{6.390980in}{0.554746in}}%
\pgfpathlineto{\pgfqpoint{6.390980in}{0.550000in}}%
\pgfpathclose%
\pgfusepath{fill}%
\end{pgfscope}%
\begin{pgfscope}%
\pgfpathrectangle{\pgfqpoint{1.250000in}{0.550000in}}{\pgfqpoint{7.750000in}{3.850000in}}%
\pgfusepath{clip}%
\pgfsetbuttcap%
\pgfsetmiterjoin%
\definecolor{currentfill}{rgb}{0.501961,0.501961,0.501961}%
\pgfsetfillcolor{currentfill}%
\pgfsetfillopacity{0.700000}%
\pgfsetlinewidth{0.000000pt}%
\definecolor{currentstroke}{rgb}{0.000000,0.000000,0.000000}%
\pgfsetstrokecolor{currentstroke}%
\pgfsetstrokeopacity{0.700000}%
\pgfsetdash{}{0pt}%
\pgfpathmoveto{\pgfqpoint{6.418501in}{0.550000in}}%
\pgfpathlineto{\pgfqpoint{6.446023in}{0.550000in}}%
\pgfpathlineto{\pgfqpoint{6.446023in}{0.555818in}}%
\pgfpathlineto{\pgfqpoint{6.418501in}{0.555818in}}%
\pgfpathlineto{\pgfqpoint{6.418501in}{0.550000in}}%
\pgfpathclose%
\pgfusepath{fill}%
\end{pgfscope}%
\begin{pgfscope}%
\pgfpathrectangle{\pgfqpoint{1.250000in}{0.550000in}}{\pgfqpoint{7.750000in}{3.850000in}}%
\pgfusepath{clip}%
\pgfsetbuttcap%
\pgfsetmiterjoin%
\definecolor{currentfill}{rgb}{0.501961,0.501961,0.501961}%
\pgfsetfillcolor{currentfill}%
\pgfsetfillopacity{0.700000}%
\pgfsetlinewidth{0.000000pt}%
\definecolor{currentstroke}{rgb}{0.000000,0.000000,0.000000}%
\pgfsetstrokecolor{currentstroke}%
\pgfsetstrokeopacity{0.700000}%
\pgfsetdash{}{0pt}%
\pgfpathmoveto{\pgfqpoint{6.446023in}{0.550000in}}%
\pgfpathlineto{\pgfqpoint{6.473544in}{0.550000in}}%
\pgfpathlineto{\pgfqpoint{6.473544in}{0.552756in}}%
\pgfpathlineto{\pgfqpoint{6.446023in}{0.552756in}}%
\pgfpathlineto{\pgfqpoint{6.446023in}{0.550000in}}%
\pgfpathclose%
\pgfusepath{fill}%
\end{pgfscope}%
\begin{pgfscope}%
\pgfpathrectangle{\pgfqpoint{1.250000in}{0.550000in}}{\pgfqpoint{7.750000in}{3.850000in}}%
\pgfusepath{clip}%
\pgfsetbuttcap%
\pgfsetmiterjoin%
\definecolor{currentfill}{rgb}{0.501961,0.501961,0.501961}%
\pgfsetfillcolor{currentfill}%
\pgfsetfillopacity{0.700000}%
\pgfsetlinewidth{0.000000pt}%
\definecolor{currentstroke}{rgb}{0.000000,0.000000,0.000000}%
\pgfsetstrokecolor{currentstroke}%
\pgfsetstrokeopacity{0.700000}%
\pgfsetdash{}{0pt}%
\pgfpathmoveto{\pgfqpoint{6.473544in}{0.550000in}}%
\pgfpathlineto{\pgfqpoint{6.501065in}{0.550000in}}%
\pgfpathlineto{\pgfqpoint{6.501065in}{0.553675in}}%
\pgfpathlineto{\pgfqpoint{6.473544in}{0.553675in}}%
\pgfpathlineto{\pgfqpoint{6.473544in}{0.550000in}}%
\pgfpathclose%
\pgfusepath{fill}%
\end{pgfscope}%
\begin{pgfscope}%
\pgfpathrectangle{\pgfqpoint{1.250000in}{0.550000in}}{\pgfqpoint{7.750000in}{3.850000in}}%
\pgfusepath{clip}%
\pgfsetbuttcap%
\pgfsetmiterjoin%
\definecolor{currentfill}{rgb}{0.501961,0.501961,0.501961}%
\pgfsetfillcolor{currentfill}%
\pgfsetfillopacity{0.700000}%
\pgfsetlinewidth{0.000000pt}%
\definecolor{currentstroke}{rgb}{0.000000,0.000000,0.000000}%
\pgfsetstrokecolor{currentstroke}%
\pgfsetstrokeopacity{0.700000}%
\pgfsetdash{}{0pt}%
\pgfpathmoveto{\pgfqpoint{6.501065in}{0.550000in}}%
\pgfpathlineto{\pgfqpoint{6.528587in}{0.550000in}}%
\pgfpathlineto{\pgfqpoint{6.528587in}{0.554746in}}%
\pgfpathlineto{\pgfqpoint{6.501065in}{0.554746in}}%
\pgfpathlineto{\pgfqpoint{6.501065in}{0.550000in}}%
\pgfpathclose%
\pgfusepath{fill}%
\end{pgfscope}%
\begin{pgfscope}%
\pgfpathrectangle{\pgfqpoint{1.250000in}{0.550000in}}{\pgfqpoint{7.750000in}{3.850000in}}%
\pgfusepath{clip}%
\pgfsetbuttcap%
\pgfsetmiterjoin%
\definecolor{currentfill}{rgb}{0.501961,0.501961,0.501961}%
\pgfsetfillcolor{currentfill}%
\pgfsetfillopacity{0.700000}%
\pgfsetlinewidth{0.000000pt}%
\definecolor{currentstroke}{rgb}{0.000000,0.000000,0.000000}%
\pgfsetstrokecolor{currentstroke}%
\pgfsetstrokeopacity{0.700000}%
\pgfsetdash{}{0pt}%
\pgfpathmoveto{\pgfqpoint{6.528587in}{0.550000in}}%
\pgfpathlineto{\pgfqpoint{6.556108in}{0.550000in}}%
\pgfpathlineto{\pgfqpoint{6.556108in}{0.553828in}}%
\pgfpathlineto{\pgfqpoint{6.528587in}{0.553828in}}%
\pgfpathlineto{\pgfqpoint{6.528587in}{0.550000in}}%
\pgfpathclose%
\pgfusepath{fill}%
\end{pgfscope}%
\begin{pgfscope}%
\pgfpathrectangle{\pgfqpoint{1.250000in}{0.550000in}}{\pgfqpoint{7.750000in}{3.850000in}}%
\pgfusepath{clip}%
\pgfsetbuttcap%
\pgfsetmiterjoin%
\definecolor{currentfill}{rgb}{0.501961,0.501961,0.501961}%
\pgfsetfillcolor{currentfill}%
\pgfsetfillopacity{0.700000}%
\pgfsetlinewidth{0.000000pt}%
\definecolor{currentstroke}{rgb}{0.000000,0.000000,0.000000}%
\pgfsetstrokecolor{currentstroke}%
\pgfsetstrokeopacity{0.700000}%
\pgfsetdash{}{0pt}%
\pgfpathmoveto{\pgfqpoint{6.556108in}{0.550000in}}%
\pgfpathlineto{\pgfqpoint{6.583629in}{0.550000in}}%
\pgfpathlineto{\pgfqpoint{6.583629in}{0.553215in}}%
\pgfpathlineto{\pgfqpoint{6.556108in}{0.553215in}}%
\pgfpathlineto{\pgfqpoint{6.556108in}{0.550000in}}%
\pgfpathclose%
\pgfusepath{fill}%
\end{pgfscope}%
\begin{pgfscope}%
\pgfpathrectangle{\pgfqpoint{1.250000in}{0.550000in}}{\pgfqpoint{7.750000in}{3.850000in}}%
\pgfusepath{clip}%
\pgfsetbuttcap%
\pgfsetmiterjoin%
\definecolor{currentfill}{rgb}{0.501961,0.501961,0.501961}%
\pgfsetfillcolor{currentfill}%
\pgfsetfillopacity{0.700000}%
\pgfsetlinewidth{0.000000pt}%
\definecolor{currentstroke}{rgb}{0.000000,0.000000,0.000000}%
\pgfsetstrokecolor{currentstroke}%
\pgfsetstrokeopacity{0.700000}%
\pgfsetdash{}{0pt}%
\pgfpathmoveto{\pgfqpoint{6.583629in}{0.550000in}}%
\pgfpathlineto{\pgfqpoint{6.611151in}{0.550000in}}%
\pgfpathlineto{\pgfqpoint{6.611151in}{0.552603in}}%
\pgfpathlineto{\pgfqpoint{6.583629in}{0.552603in}}%
\pgfpathlineto{\pgfqpoint{6.583629in}{0.550000in}}%
\pgfpathclose%
\pgfusepath{fill}%
\end{pgfscope}%
\begin{pgfscope}%
\pgfpathrectangle{\pgfqpoint{1.250000in}{0.550000in}}{\pgfqpoint{7.750000in}{3.850000in}}%
\pgfusepath{clip}%
\pgfsetbuttcap%
\pgfsetmiterjoin%
\definecolor{currentfill}{rgb}{0.501961,0.501961,0.501961}%
\pgfsetfillcolor{currentfill}%
\pgfsetfillopacity{0.700000}%
\pgfsetlinewidth{0.000000pt}%
\definecolor{currentstroke}{rgb}{0.000000,0.000000,0.000000}%
\pgfsetstrokecolor{currentstroke}%
\pgfsetstrokeopacity{0.700000}%
\pgfsetdash{}{0pt}%
\pgfpathmoveto{\pgfqpoint{6.611151in}{0.550000in}}%
\pgfpathlineto{\pgfqpoint{6.638672in}{0.550000in}}%
\pgfpathlineto{\pgfqpoint{6.638672in}{0.552297in}}%
\pgfpathlineto{\pgfqpoint{6.611151in}{0.552297in}}%
\pgfpathlineto{\pgfqpoint{6.611151in}{0.550000in}}%
\pgfpathclose%
\pgfusepath{fill}%
\end{pgfscope}%
\begin{pgfscope}%
\pgfpathrectangle{\pgfqpoint{1.250000in}{0.550000in}}{\pgfqpoint{7.750000in}{3.850000in}}%
\pgfusepath{clip}%
\pgfsetbuttcap%
\pgfsetmiterjoin%
\definecolor{currentfill}{rgb}{0.501961,0.501961,0.501961}%
\pgfsetfillcolor{currentfill}%
\pgfsetfillopacity{0.700000}%
\pgfsetlinewidth{0.000000pt}%
\definecolor{currentstroke}{rgb}{0.000000,0.000000,0.000000}%
\pgfsetstrokecolor{currentstroke}%
\pgfsetstrokeopacity{0.700000}%
\pgfsetdash{}{0pt}%
\pgfpathmoveto{\pgfqpoint{6.638672in}{0.550000in}}%
\pgfpathlineto{\pgfqpoint{6.666193in}{0.550000in}}%
\pgfpathlineto{\pgfqpoint{6.666193in}{0.552603in}}%
\pgfpathlineto{\pgfqpoint{6.638672in}{0.552603in}}%
\pgfpathlineto{\pgfqpoint{6.638672in}{0.550000in}}%
\pgfpathclose%
\pgfusepath{fill}%
\end{pgfscope}%
\begin{pgfscope}%
\pgfpathrectangle{\pgfqpoint{1.250000in}{0.550000in}}{\pgfqpoint{7.750000in}{3.850000in}}%
\pgfusepath{clip}%
\pgfsetbuttcap%
\pgfsetmiterjoin%
\definecolor{currentfill}{rgb}{0.501961,0.501961,0.501961}%
\pgfsetfillcolor{currentfill}%
\pgfsetfillopacity{0.700000}%
\pgfsetlinewidth{0.000000pt}%
\definecolor{currentstroke}{rgb}{0.000000,0.000000,0.000000}%
\pgfsetstrokecolor{currentstroke}%
\pgfsetstrokeopacity{0.700000}%
\pgfsetdash{}{0pt}%
\pgfpathmoveto{\pgfqpoint{6.666193in}{0.550000in}}%
\pgfpathlineto{\pgfqpoint{6.693714in}{0.550000in}}%
\pgfpathlineto{\pgfqpoint{6.693714in}{0.552450in}}%
\pgfpathlineto{\pgfqpoint{6.666193in}{0.552450in}}%
\pgfpathlineto{\pgfqpoint{6.666193in}{0.550000in}}%
\pgfpathclose%
\pgfusepath{fill}%
\end{pgfscope}%
\begin{pgfscope}%
\pgfpathrectangle{\pgfqpoint{1.250000in}{0.550000in}}{\pgfqpoint{7.750000in}{3.850000in}}%
\pgfusepath{clip}%
\pgfsetbuttcap%
\pgfsetmiterjoin%
\definecolor{currentfill}{rgb}{0.501961,0.501961,0.501961}%
\pgfsetfillcolor{currentfill}%
\pgfsetfillopacity{0.700000}%
\pgfsetlinewidth{0.000000pt}%
\definecolor{currentstroke}{rgb}{0.000000,0.000000,0.000000}%
\pgfsetstrokecolor{currentstroke}%
\pgfsetstrokeopacity{0.700000}%
\pgfsetdash{}{0pt}%
\pgfpathmoveto{\pgfqpoint{6.693714in}{0.550000in}}%
\pgfpathlineto{\pgfqpoint{6.721236in}{0.550000in}}%
\pgfpathlineto{\pgfqpoint{6.721236in}{0.553981in}}%
\pgfpathlineto{\pgfqpoint{6.693714in}{0.553981in}}%
\pgfpathlineto{\pgfqpoint{6.693714in}{0.550000in}}%
\pgfpathclose%
\pgfusepath{fill}%
\end{pgfscope}%
\begin{pgfscope}%
\pgfpathrectangle{\pgfqpoint{1.250000in}{0.550000in}}{\pgfqpoint{7.750000in}{3.850000in}}%
\pgfusepath{clip}%
\pgfsetbuttcap%
\pgfsetmiterjoin%
\definecolor{currentfill}{rgb}{0.501961,0.501961,0.501961}%
\pgfsetfillcolor{currentfill}%
\pgfsetfillopacity{0.700000}%
\pgfsetlinewidth{0.000000pt}%
\definecolor{currentstroke}{rgb}{0.000000,0.000000,0.000000}%
\pgfsetstrokecolor{currentstroke}%
\pgfsetstrokeopacity{0.700000}%
\pgfsetdash{}{0pt}%
\pgfpathmoveto{\pgfqpoint{6.721236in}{0.550000in}}%
\pgfpathlineto{\pgfqpoint{6.748757in}{0.550000in}}%
\pgfpathlineto{\pgfqpoint{6.748757in}{0.553368in}}%
\pgfpathlineto{\pgfqpoint{6.721236in}{0.553368in}}%
\pgfpathlineto{\pgfqpoint{6.721236in}{0.550000in}}%
\pgfpathclose%
\pgfusepath{fill}%
\end{pgfscope}%
\begin{pgfscope}%
\pgfpathrectangle{\pgfqpoint{1.250000in}{0.550000in}}{\pgfqpoint{7.750000in}{3.850000in}}%
\pgfusepath{clip}%
\pgfsetbuttcap%
\pgfsetmiterjoin%
\definecolor{currentfill}{rgb}{0.501961,0.501961,0.501961}%
\pgfsetfillcolor{currentfill}%
\pgfsetfillopacity{0.700000}%
\pgfsetlinewidth{0.000000pt}%
\definecolor{currentstroke}{rgb}{0.000000,0.000000,0.000000}%
\pgfsetstrokecolor{currentstroke}%
\pgfsetstrokeopacity{0.700000}%
\pgfsetdash{}{0pt}%
\pgfpathmoveto{\pgfqpoint{6.748757in}{0.550000in}}%
\pgfpathlineto{\pgfqpoint{6.776278in}{0.550000in}}%
\pgfpathlineto{\pgfqpoint{6.776278in}{0.553062in}}%
\pgfpathlineto{\pgfqpoint{6.748757in}{0.553062in}}%
\pgfpathlineto{\pgfqpoint{6.748757in}{0.550000in}}%
\pgfpathclose%
\pgfusepath{fill}%
\end{pgfscope}%
\begin{pgfscope}%
\pgfpathrectangle{\pgfqpoint{1.250000in}{0.550000in}}{\pgfqpoint{7.750000in}{3.850000in}}%
\pgfusepath{clip}%
\pgfsetbuttcap%
\pgfsetmiterjoin%
\definecolor{currentfill}{rgb}{0.501961,0.501961,0.501961}%
\pgfsetfillcolor{currentfill}%
\pgfsetfillopacity{0.700000}%
\pgfsetlinewidth{0.000000pt}%
\definecolor{currentstroke}{rgb}{0.000000,0.000000,0.000000}%
\pgfsetstrokecolor{currentstroke}%
\pgfsetstrokeopacity{0.700000}%
\pgfsetdash{}{0pt}%
\pgfpathmoveto{\pgfqpoint{6.776278in}{0.550000in}}%
\pgfpathlineto{\pgfqpoint{6.803800in}{0.550000in}}%
\pgfpathlineto{\pgfqpoint{6.803800in}{0.551684in}}%
\pgfpathlineto{\pgfqpoint{6.776278in}{0.551684in}}%
\pgfpathlineto{\pgfqpoint{6.776278in}{0.550000in}}%
\pgfpathclose%
\pgfusepath{fill}%
\end{pgfscope}%
\begin{pgfscope}%
\pgfpathrectangle{\pgfqpoint{1.250000in}{0.550000in}}{\pgfqpoint{7.750000in}{3.850000in}}%
\pgfusepath{clip}%
\pgfsetbuttcap%
\pgfsetmiterjoin%
\definecolor{currentfill}{rgb}{0.501961,0.501961,0.501961}%
\pgfsetfillcolor{currentfill}%
\pgfsetfillopacity{0.700000}%
\pgfsetlinewidth{0.000000pt}%
\definecolor{currentstroke}{rgb}{0.000000,0.000000,0.000000}%
\pgfsetstrokecolor{currentstroke}%
\pgfsetstrokeopacity{0.700000}%
\pgfsetdash{}{0pt}%
\pgfpathmoveto{\pgfqpoint{6.803800in}{0.550000in}}%
\pgfpathlineto{\pgfqpoint{6.831321in}{0.550000in}}%
\pgfpathlineto{\pgfqpoint{6.831321in}{0.552450in}}%
\pgfpathlineto{\pgfqpoint{6.803800in}{0.552450in}}%
\pgfpathlineto{\pgfqpoint{6.803800in}{0.550000in}}%
\pgfpathclose%
\pgfusepath{fill}%
\end{pgfscope}%
\begin{pgfscope}%
\pgfpathrectangle{\pgfqpoint{1.250000in}{0.550000in}}{\pgfqpoint{7.750000in}{3.850000in}}%
\pgfusepath{clip}%
\pgfsetbuttcap%
\pgfsetmiterjoin%
\definecolor{currentfill}{rgb}{0.501961,0.501961,0.501961}%
\pgfsetfillcolor{currentfill}%
\pgfsetfillopacity{0.700000}%
\pgfsetlinewidth{0.000000pt}%
\definecolor{currentstroke}{rgb}{0.000000,0.000000,0.000000}%
\pgfsetstrokecolor{currentstroke}%
\pgfsetstrokeopacity{0.700000}%
\pgfsetdash{}{0pt}%
\pgfpathmoveto{\pgfqpoint{6.831321in}{0.550000in}}%
\pgfpathlineto{\pgfqpoint{6.858842in}{0.550000in}}%
\pgfpathlineto{\pgfqpoint{6.858842in}{0.551378in}}%
\pgfpathlineto{\pgfqpoint{6.831321in}{0.551378in}}%
\pgfpathlineto{\pgfqpoint{6.831321in}{0.550000in}}%
\pgfpathclose%
\pgfusepath{fill}%
\end{pgfscope}%
\begin{pgfscope}%
\pgfpathrectangle{\pgfqpoint{1.250000in}{0.550000in}}{\pgfqpoint{7.750000in}{3.850000in}}%
\pgfusepath{clip}%
\pgfsetbuttcap%
\pgfsetmiterjoin%
\definecolor{currentfill}{rgb}{0.501961,0.501961,0.501961}%
\pgfsetfillcolor{currentfill}%
\pgfsetfillopacity{0.700000}%
\pgfsetlinewidth{0.000000pt}%
\definecolor{currentstroke}{rgb}{0.000000,0.000000,0.000000}%
\pgfsetstrokecolor{currentstroke}%
\pgfsetstrokeopacity{0.700000}%
\pgfsetdash{}{0pt}%
\pgfpathmoveto{\pgfqpoint{6.858842in}{0.550000in}}%
\pgfpathlineto{\pgfqpoint{6.886364in}{0.550000in}}%
\pgfpathlineto{\pgfqpoint{6.886364in}{0.551684in}}%
\pgfpathlineto{\pgfqpoint{6.858842in}{0.551684in}}%
\pgfpathlineto{\pgfqpoint{6.858842in}{0.550000in}}%
\pgfpathclose%
\pgfusepath{fill}%
\end{pgfscope}%
\begin{pgfscope}%
\pgfpathrectangle{\pgfqpoint{1.250000in}{0.550000in}}{\pgfqpoint{7.750000in}{3.850000in}}%
\pgfusepath{clip}%
\pgfsetbuttcap%
\pgfsetmiterjoin%
\definecolor{currentfill}{rgb}{0.501961,0.501961,0.501961}%
\pgfsetfillcolor{currentfill}%
\pgfsetfillopacity{0.700000}%
\pgfsetlinewidth{0.000000pt}%
\definecolor{currentstroke}{rgb}{0.000000,0.000000,0.000000}%
\pgfsetstrokecolor{currentstroke}%
\pgfsetstrokeopacity{0.700000}%
\pgfsetdash{}{0pt}%
\pgfpathmoveto{\pgfqpoint{6.886364in}{0.550000in}}%
\pgfpathlineto{\pgfqpoint{6.913885in}{0.550000in}}%
\pgfpathlineto{\pgfqpoint{6.913885in}{0.551990in}}%
\pgfpathlineto{\pgfqpoint{6.886364in}{0.551990in}}%
\pgfpathlineto{\pgfqpoint{6.886364in}{0.550000in}}%
\pgfpathclose%
\pgfusepath{fill}%
\end{pgfscope}%
\begin{pgfscope}%
\pgfpathrectangle{\pgfqpoint{1.250000in}{0.550000in}}{\pgfqpoint{7.750000in}{3.850000in}}%
\pgfusepath{clip}%
\pgfsetbuttcap%
\pgfsetmiterjoin%
\definecolor{currentfill}{rgb}{0.501961,0.501961,0.501961}%
\pgfsetfillcolor{currentfill}%
\pgfsetfillopacity{0.700000}%
\pgfsetlinewidth{0.000000pt}%
\definecolor{currentstroke}{rgb}{0.000000,0.000000,0.000000}%
\pgfsetstrokecolor{currentstroke}%
\pgfsetstrokeopacity{0.700000}%
\pgfsetdash{}{0pt}%
\pgfpathmoveto{\pgfqpoint{6.913885in}{0.550000in}}%
\pgfpathlineto{\pgfqpoint{6.941406in}{0.550000in}}%
\pgfpathlineto{\pgfqpoint{6.941406in}{0.552909in}}%
\pgfpathlineto{\pgfqpoint{6.913885in}{0.552909in}}%
\pgfpathlineto{\pgfqpoint{6.913885in}{0.550000in}}%
\pgfpathclose%
\pgfusepath{fill}%
\end{pgfscope}%
\begin{pgfscope}%
\pgfpathrectangle{\pgfqpoint{1.250000in}{0.550000in}}{\pgfqpoint{7.750000in}{3.850000in}}%
\pgfusepath{clip}%
\pgfsetbuttcap%
\pgfsetmiterjoin%
\definecolor{currentfill}{rgb}{0.501961,0.501961,0.501961}%
\pgfsetfillcolor{currentfill}%
\pgfsetfillopacity{0.700000}%
\pgfsetlinewidth{0.000000pt}%
\definecolor{currentstroke}{rgb}{0.000000,0.000000,0.000000}%
\pgfsetstrokecolor{currentstroke}%
\pgfsetstrokeopacity{0.700000}%
\pgfsetdash{}{0pt}%
\pgfpathmoveto{\pgfqpoint{6.941406in}{0.550000in}}%
\pgfpathlineto{\pgfqpoint{6.968928in}{0.550000in}}%
\pgfpathlineto{\pgfqpoint{6.968928in}{0.552450in}}%
\pgfpathlineto{\pgfqpoint{6.941406in}{0.552450in}}%
\pgfpathlineto{\pgfqpoint{6.941406in}{0.550000in}}%
\pgfpathclose%
\pgfusepath{fill}%
\end{pgfscope}%
\begin{pgfscope}%
\pgfpathrectangle{\pgfqpoint{1.250000in}{0.550000in}}{\pgfqpoint{7.750000in}{3.850000in}}%
\pgfusepath{clip}%
\pgfsetbuttcap%
\pgfsetmiterjoin%
\definecolor{currentfill}{rgb}{0.501961,0.501961,0.501961}%
\pgfsetfillcolor{currentfill}%
\pgfsetfillopacity{0.700000}%
\pgfsetlinewidth{0.000000pt}%
\definecolor{currentstroke}{rgb}{0.000000,0.000000,0.000000}%
\pgfsetstrokecolor{currentstroke}%
\pgfsetstrokeopacity{0.700000}%
\pgfsetdash{}{0pt}%
\pgfpathmoveto{\pgfqpoint{6.968928in}{0.550000in}}%
\pgfpathlineto{\pgfqpoint{6.996449in}{0.550000in}}%
\pgfpathlineto{\pgfqpoint{6.996449in}{0.551072in}}%
\pgfpathlineto{\pgfqpoint{6.968928in}{0.551072in}}%
\pgfpathlineto{\pgfqpoint{6.968928in}{0.550000in}}%
\pgfpathclose%
\pgfusepath{fill}%
\end{pgfscope}%
\begin{pgfscope}%
\pgfpathrectangle{\pgfqpoint{1.250000in}{0.550000in}}{\pgfqpoint{7.750000in}{3.850000in}}%
\pgfusepath{clip}%
\pgfsetbuttcap%
\pgfsetmiterjoin%
\definecolor{currentfill}{rgb}{0.501961,0.501961,0.501961}%
\pgfsetfillcolor{currentfill}%
\pgfsetfillopacity{0.700000}%
\pgfsetlinewidth{0.000000pt}%
\definecolor{currentstroke}{rgb}{0.000000,0.000000,0.000000}%
\pgfsetstrokecolor{currentstroke}%
\pgfsetstrokeopacity{0.700000}%
\pgfsetdash{}{0pt}%
\pgfpathmoveto{\pgfqpoint{6.996449in}{0.550000in}}%
\pgfpathlineto{\pgfqpoint{7.023970in}{0.550000in}}%
\pgfpathlineto{\pgfqpoint{7.023970in}{0.552297in}}%
\pgfpathlineto{\pgfqpoint{6.996449in}{0.552297in}}%
\pgfpathlineto{\pgfqpoint{6.996449in}{0.550000in}}%
\pgfpathclose%
\pgfusepath{fill}%
\end{pgfscope}%
\begin{pgfscope}%
\pgfpathrectangle{\pgfqpoint{1.250000in}{0.550000in}}{\pgfqpoint{7.750000in}{3.850000in}}%
\pgfusepath{clip}%
\pgfsetbuttcap%
\pgfsetmiterjoin%
\definecolor{currentfill}{rgb}{0.501961,0.501961,0.501961}%
\pgfsetfillcolor{currentfill}%
\pgfsetfillopacity{0.700000}%
\pgfsetlinewidth{0.000000pt}%
\definecolor{currentstroke}{rgb}{0.000000,0.000000,0.000000}%
\pgfsetstrokecolor{currentstroke}%
\pgfsetstrokeopacity{0.700000}%
\pgfsetdash{}{0pt}%
\pgfpathmoveto{\pgfqpoint{7.023970in}{0.550000in}}%
\pgfpathlineto{\pgfqpoint{7.051491in}{0.550000in}}%
\pgfpathlineto{\pgfqpoint{7.051491in}{0.552144in}}%
\pgfpathlineto{\pgfqpoint{7.023970in}{0.552144in}}%
\pgfpathlineto{\pgfqpoint{7.023970in}{0.550000in}}%
\pgfpathclose%
\pgfusepath{fill}%
\end{pgfscope}%
\begin{pgfscope}%
\pgfpathrectangle{\pgfqpoint{1.250000in}{0.550000in}}{\pgfqpoint{7.750000in}{3.850000in}}%
\pgfusepath{clip}%
\pgfsetbuttcap%
\pgfsetmiterjoin%
\definecolor{currentfill}{rgb}{0.501961,0.501961,0.501961}%
\pgfsetfillcolor{currentfill}%
\pgfsetfillopacity{0.700000}%
\pgfsetlinewidth{0.000000pt}%
\definecolor{currentstroke}{rgb}{0.000000,0.000000,0.000000}%
\pgfsetstrokecolor{currentstroke}%
\pgfsetstrokeopacity{0.700000}%
\pgfsetdash{}{0pt}%
\pgfpathmoveto{\pgfqpoint{7.051491in}{0.550000in}}%
\pgfpathlineto{\pgfqpoint{7.079013in}{0.550000in}}%
\pgfpathlineto{\pgfqpoint{7.079013in}{0.552756in}}%
\pgfpathlineto{\pgfqpoint{7.051491in}{0.552756in}}%
\pgfpathlineto{\pgfqpoint{7.051491in}{0.550000in}}%
\pgfpathclose%
\pgfusepath{fill}%
\end{pgfscope}%
\begin{pgfscope}%
\pgfpathrectangle{\pgfqpoint{1.250000in}{0.550000in}}{\pgfqpoint{7.750000in}{3.850000in}}%
\pgfusepath{clip}%
\pgfsetbuttcap%
\pgfsetmiterjoin%
\definecolor{currentfill}{rgb}{0.501961,0.501961,0.501961}%
\pgfsetfillcolor{currentfill}%
\pgfsetfillopacity{0.700000}%
\pgfsetlinewidth{0.000000pt}%
\definecolor{currentstroke}{rgb}{0.000000,0.000000,0.000000}%
\pgfsetstrokecolor{currentstroke}%
\pgfsetstrokeopacity{0.700000}%
\pgfsetdash{}{0pt}%
\pgfpathmoveto{\pgfqpoint{7.079013in}{0.550000in}}%
\pgfpathlineto{\pgfqpoint{7.106534in}{0.550000in}}%
\pgfpathlineto{\pgfqpoint{7.106534in}{0.551684in}}%
\pgfpathlineto{\pgfqpoint{7.079013in}{0.551684in}}%
\pgfpathlineto{\pgfqpoint{7.079013in}{0.550000in}}%
\pgfpathclose%
\pgfusepath{fill}%
\end{pgfscope}%
\begin{pgfscope}%
\pgfpathrectangle{\pgfqpoint{1.250000in}{0.550000in}}{\pgfqpoint{7.750000in}{3.850000in}}%
\pgfusepath{clip}%
\pgfsetbuttcap%
\pgfsetmiterjoin%
\definecolor{currentfill}{rgb}{0.501961,0.501961,0.501961}%
\pgfsetfillcolor{currentfill}%
\pgfsetfillopacity{0.700000}%
\pgfsetlinewidth{0.000000pt}%
\definecolor{currentstroke}{rgb}{0.000000,0.000000,0.000000}%
\pgfsetstrokecolor{currentstroke}%
\pgfsetstrokeopacity{0.700000}%
\pgfsetdash{}{0pt}%
\pgfpathmoveto{\pgfqpoint{7.106534in}{0.550000in}}%
\pgfpathlineto{\pgfqpoint{7.134055in}{0.550000in}}%
\pgfpathlineto{\pgfqpoint{7.134055in}{0.551837in}}%
\pgfpathlineto{\pgfqpoint{7.106534in}{0.551837in}}%
\pgfpathlineto{\pgfqpoint{7.106534in}{0.550000in}}%
\pgfpathclose%
\pgfusepath{fill}%
\end{pgfscope}%
\begin{pgfscope}%
\pgfpathrectangle{\pgfqpoint{1.250000in}{0.550000in}}{\pgfqpoint{7.750000in}{3.850000in}}%
\pgfusepath{clip}%
\pgfsetbuttcap%
\pgfsetmiterjoin%
\definecolor{currentfill}{rgb}{0.501961,0.501961,0.501961}%
\pgfsetfillcolor{currentfill}%
\pgfsetfillopacity{0.700000}%
\pgfsetlinewidth{0.000000pt}%
\definecolor{currentstroke}{rgb}{0.000000,0.000000,0.000000}%
\pgfsetstrokecolor{currentstroke}%
\pgfsetstrokeopacity{0.700000}%
\pgfsetdash{}{0pt}%
\pgfpathmoveto{\pgfqpoint{7.134055in}{0.550000in}}%
\pgfpathlineto{\pgfqpoint{7.161577in}{0.550000in}}%
\pgfpathlineto{\pgfqpoint{7.161577in}{0.551072in}}%
\pgfpathlineto{\pgfqpoint{7.134055in}{0.551072in}}%
\pgfpathlineto{\pgfqpoint{7.134055in}{0.550000in}}%
\pgfpathclose%
\pgfusepath{fill}%
\end{pgfscope}%
\begin{pgfscope}%
\pgfpathrectangle{\pgfqpoint{1.250000in}{0.550000in}}{\pgfqpoint{7.750000in}{3.850000in}}%
\pgfusepath{clip}%
\pgfsetbuttcap%
\pgfsetmiterjoin%
\definecolor{currentfill}{rgb}{0.501961,0.501961,0.501961}%
\pgfsetfillcolor{currentfill}%
\pgfsetfillopacity{0.700000}%
\pgfsetlinewidth{0.000000pt}%
\definecolor{currentstroke}{rgb}{0.000000,0.000000,0.000000}%
\pgfsetstrokecolor{currentstroke}%
\pgfsetstrokeopacity{0.700000}%
\pgfsetdash{}{0pt}%
\pgfpathmoveto{\pgfqpoint{7.161577in}{0.550000in}}%
\pgfpathlineto{\pgfqpoint{7.189098in}{0.550000in}}%
\pgfpathlineto{\pgfqpoint{7.189098in}{0.551225in}}%
\pgfpathlineto{\pgfqpoint{7.161577in}{0.551225in}}%
\pgfpathlineto{\pgfqpoint{7.161577in}{0.550000in}}%
\pgfpathclose%
\pgfusepath{fill}%
\end{pgfscope}%
\begin{pgfscope}%
\pgfpathrectangle{\pgfqpoint{1.250000in}{0.550000in}}{\pgfqpoint{7.750000in}{3.850000in}}%
\pgfusepath{clip}%
\pgfsetbuttcap%
\pgfsetmiterjoin%
\definecolor{currentfill}{rgb}{0.501961,0.501961,0.501961}%
\pgfsetfillcolor{currentfill}%
\pgfsetfillopacity{0.700000}%
\pgfsetlinewidth{0.000000pt}%
\definecolor{currentstroke}{rgb}{0.000000,0.000000,0.000000}%
\pgfsetstrokecolor{currentstroke}%
\pgfsetstrokeopacity{0.700000}%
\pgfsetdash{}{0pt}%
\pgfpathmoveto{\pgfqpoint{7.189098in}{0.550000in}}%
\pgfpathlineto{\pgfqpoint{7.216619in}{0.550000in}}%
\pgfpathlineto{\pgfqpoint{7.216619in}{0.550459in}}%
\pgfpathlineto{\pgfqpoint{7.189098in}{0.550459in}}%
\pgfpathlineto{\pgfqpoint{7.189098in}{0.550000in}}%
\pgfpathclose%
\pgfusepath{fill}%
\end{pgfscope}%
\begin{pgfscope}%
\pgfpathrectangle{\pgfqpoint{1.250000in}{0.550000in}}{\pgfqpoint{7.750000in}{3.850000in}}%
\pgfusepath{clip}%
\pgfsetbuttcap%
\pgfsetmiterjoin%
\definecolor{currentfill}{rgb}{0.501961,0.501961,0.501961}%
\pgfsetfillcolor{currentfill}%
\pgfsetfillopacity{0.700000}%
\pgfsetlinewidth{0.000000pt}%
\definecolor{currentstroke}{rgb}{0.000000,0.000000,0.000000}%
\pgfsetstrokecolor{currentstroke}%
\pgfsetstrokeopacity{0.700000}%
\pgfsetdash{}{0pt}%
\pgfpathmoveto{\pgfqpoint{7.216619in}{0.550000in}}%
\pgfpathlineto{\pgfqpoint{7.244141in}{0.550000in}}%
\pgfpathlineto{\pgfqpoint{7.244141in}{0.550919in}}%
\pgfpathlineto{\pgfqpoint{7.216619in}{0.550919in}}%
\pgfpathlineto{\pgfqpoint{7.216619in}{0.550000in}}%
\pgfpathclose%
\pgfusepath{fill}%
\end{pgfscope}%
\begin{pgfscope}%
\pgfpathrectangle{\pgfqpoint{1.250000in}{0.550000in}}{\pgfqpoint{7.750000in}{3.850000in}}%
\pgfusepath{clip}%
\pgfsetbuttcap%
\pgfsetmiterjoin%
\definecolor{currentfill}{rgb}{0.501961,0.501961,0.501961}%
\pgfsetfillcolor{currentfill}%
\pgfsetfillopacity{0.700000}%
\pgfsetlinewidth{0.000000pt}%
\definecolor{currentstroke}{rgb}{0.000000,0.000000,0.000000}%
\pgfsetstrokecolor{currentstroke}%
\pgfsetstrokeopacity{0.700000}%
\pgfsetdash{}{0pt}%
\pgfpathmoveto{\pgfqpoint{7.244141in}{0.550000in}}%
\pgfpathlineto{\pgfqpoint{7.271662in}{0.550000in}}%
\pgfpathlineto{\pgfqpoint{7.271662in}{0.550000in}}%
\pgfpathlineto{\pgfqpoint{7.244141in}{0.550000in}}%
\pgfpathlineto{\pgfqpoint{7.244141in}{0.550000in}}%
\pgfpathclose%
\pgfusepath{fill}%
\end{pgfscope}%
\begin{pgfscope}%
\pgfpathrectangle{\pgfqpoint{1.250000in}{0.550000in}}{\pgfqpoint{7.750000in}{3.850000in}}%
\pgfusepath{clip}%
\pgfsetbuttcap%
\pgfsetmiterjoin%
\definecolor{currentfill}{rgb}{0.501961,0.501961,0.501961}%
\pgfsetfillcolor{currentfill}%
\pgfsetfillopacity{0.700000}%
\pgfsetlinewidth{0.000000pt}%
\definecolor{currentstroke}{rgb}{0.000000,0.000000,0.000000}%
\pgfsetstrokecolor{currentstroke}%
\pgfsetstrokeopacity{0.700000}%
\pgfsetdash{}{0pt}%
\pgfpathmoveto{\pgfqpoint{7.271662in}{0.550000in}}%
\pgfpathlineto{\pgfqpoint{7.299183in}{0.550000in}}%
\pgfpathlineto{\pgfqpoint{7.299183in}{0.550153in}}%
\pgfpathlineto{\pgfqpoint{7.271662in}{0.550153in}}%
\pgfpathlineto{\pgfqpoint{7.271662in}{0.550000in}}%
\pgfpathclose%
\pgfusepath{fill}%
\end{pgfscope}%
\begin{pgfscope}%
\pgfpathrectangle{\pgfqpoint{1.250000in}{0.550000in}}{\pgfqpoint{7.750000in}{3.850000in}}%
\pgfusepath{clip}%
\pgfsetbuttcap%
\pgfsetmiterjoin%
\definecolor{currentfill}{rgb}{0.501961,0.501961,0.501961}%
\pgfsetfillcolor{currentfill}%
\pgfsetfillopacity{0.700000}%
\pgfsetlinewidth{0.000000pt}%
\definecolor{currentstroke}{rgb}{0.000000,0.000000,0.000000}%
\pgfsetstrokecolor{currentstroke}%
\pgfsetstrokeopacity{0.700000}%
\pgfsetdash{}{0pt}%
\pgfpathmoveto{\pgfqpoint{7.299183in}{0.550000in}}%
\pgfpathlineto{\pgfqpoint{7.326705in}{0.550000in}}%
\pgfpathlineto{\pgfqpoint{7.326705in}{0.550306in}}%
\pgfpathlineto{\pgfqpoint{7.299183in}{0.550306in}}%
\pgfpathlineto{\pgfqpoint{7.299183in}{0.550000in}}%
\pgfpathclose%
\pgfusepath{fill}%
\end{pgfscope}%
\begin{pgfscope}%
\pgfpathrectangle{\pgfqpoint{1.250000in}{0.550000in}}{\pgfqpoint{7.750000in}{3.850000in}}%
\pgfusepath{clip}%
\pgfsetbuttcap%
\pgfsetmiterjoin%
\definecolor{currentfill}{rgb}{0.501961,0.501961,0.501961}%
\pgfsetfillcolor{currentfill}%
\pgfsetfillopacity{0.700000}%
\pgfsetlinewidth{0.000000pt}%
\definecolor{currentstroke}{rgb}{0.000000,0.000000,0.000000}%
\pgfsetstrokecolor{currentstroke}%
\pgfsetstrokeopacity{0.700000}%
\pgfsetdash{}{0pt}%
\pgfpathmoveto{\pgfqpoint{7.326705in}{0.550000in}}%
\pgfpathlineto{\pgfqpoint{7.354226in}{0.550000in}}%
\pgfpathlineto{\pgfqpoint{7.354226in}{0.550000in}}%
\pgfpathlineto{\pgfqpoint{7.326705in}{0.550000in}}%
\pgfpathlineto{\pgfqpoint{7.326705in}{0.550000in}}%
\pgfpathclose%
\pgfusepath{fill}%
\end{pgfscope}%
\begin{pgfscope}%
\pgfpathrectangle{\pgfqpoint{1.250000in}{0.550000in}}{\pgfqpoint{7.750000in}{3.850000in}}%
\pgfusepath{clip}%
\pgfsetbuttcap%
\pgfsetmiterjoin%
\definecolor{currentfill}{rgb}{0.501961,0.501961,0.501961}%
\pgfsetfillcolor{currentfill}%
\pgfsetfillopacity{0.700000}%
\pgfsetlinewidth{0.000000pt}%
\definecolor{currentstroke}{rgb}{0.000000,0.000000,0.000000}%
\pgfsetstrokecolor{currentstroke}%
\pgfsetstrokeopacity{0.700000}%
\pgfsetdash{}{0pt}%
\pgfpathmoveto{\pgfqpoint{7.354226in}{0.550000in}}%
\pgfpathlineto{\pgfqpoint{7.381747in}{0.550000in}}%
\pgfpathlineto{\pgfqpoint{7.381747in}{0.550000in}}%
\pgfpathlineto{\pgfqpoint{7.354226in}{0.550000in}}%
\pgfpathlineto{\pgfqpoint{7.354226in}{0.550000in}}%
\pgfpathclose%
\pgfusepath{fill}%
\end{pgfscope}%
\begin{pgfscope}%
\pgfpathrectangle{\pgfqpoint{1.250000in}{0.550000in}}{\pgfqpoint{7.750000in}{3.850000in}}%
\pgfusepath{clip}%
\pgfsetbuttcap%
\pgfsetmiterjoin%
\definecolor{currentfill}{rgb}{0.501961,0.501961,0.501961}%
\pgfsetfillcolor{currentfill}%
\pgfsetfillopacity{0.700000}%
\pgfsetlinewidth{0.000000pt}%
\definecolor{currentstroke}{rgb}{0.000000,0.000000,0.000000}%
\pgfsetstrokecolor{currentstroke}%
\pgfsetstrokeopacity{0.700000}%
\pgfsetdash{}{0pt}%
\pgfpathmoveto{\pgfqpoint{7.381747in}{0.550000in}}%
\pgfpathlineto{\pgfqpoint{7.409268in}{0.550000in}}%
\pgfpathlineto{\pgfqpoint{7.409268in}{0.550000in}}%
\pgfpathlineto{\pgfqpoint{7.381747in}{0.550000in}}%
\pgfpathlineto{\pgfqpoint{7.381747in}{0.550000in}}%
\pgfpathclose%
\pgfusepath{fill}%
\end{pgfscope}%
\begin{pgfscope}%
\pgfpathrectangle{\pgfqpoint{1.250000in}{0.550000in}}{\pgfqpoint{7.750000in}{3.850000in}}%
\pgfusepath{clip}%
\pgfsetbuttcap%
\pgfsetmiterjoin%
\definecolor{currentfill}{rgb}{0.501961,0.501961,0.501961}%
\pgfsetfillcolor{currentfill}%
\pgfsetfillopacity{0.700000}%
\pgfsetlinewidth{0.000000pt}%
\definecolor{currentstroke}{rgb}{0.000000,0.000000,0.000000}%
\pgfsetstrokecolor{currentstroke}%
\pgfsetstrokeopacity{0.700000}%
\pgfsetdash{}{0pt}%
\pgfpathmoveto{\pgfqpoint{7.409268in}{0.550000in}}%
\pgfpathlineto{\pgfqpoint{7.436790in}{0.550000in}}%
\pgfpathlineto{\pgfqpoint{7.436790in}{0.550153in}}%
\pgfpathlineto{\pgfqpoint{7.409268in}{0.550153in}}%
\pgfpathlineto{\pgfqpoint{7.409268in}{0.550000in}}%
\pgfpathclose%
\pgfusepath{fill}%
\end{pgfscope}%
\begin{pgfscope}%
\pgfpathrectangle{\pgfqpoint{1.250000in}{0.550000in}}{\pgfqpoint{7.750000in}{3.850000in}}%
\pgfusepath{clip}%
\pgfsetbuttcap%
\pgfsetmiterjoin%
\definecolor{currentfill}{rgb}{0.501961,0.501961,0.501961}%
\pgfsetfillcolor{currentfill}%
\pgfsetfillopacity{0.700000}%
\pgfsetlinewidth{0.000000pt}%
\definecolor{currentstroke}{rgb}{0.000000,0.000000,0.000000}%
\pgfsetstrokecolor{currentstroke}%
\pgfsetstrokeopacity{0.700000}%
\pgfsetdash{}{0pt}%
\pgfpathmoveto{\pgfqpoint{7.436790in}{0.550000in}}%
\pgfpathlineto{\pgfqpoint{7.464311in}{0.550000in}}%
\pgfpathlineto{\pgfqpoint{7.464311in}{0.550000in}}%
\pgfpathlineto{\pgfqpoint{7.436790in}{0.550000in}}%
\pgfpathlineto{\pgfqpoint{7.436790in}{0.550000in}}%
\pgfpathclose%
\pgfusepath{fill}%
\end{pgfscope}%
\begin{pgfscope}%
\pgfpathrectangle{\pgfqpoint{1.250000in}{0.550000in}}{\pgfqpoint{7.750000in}{3.850000in}}%
\pgfusepath{clip}%
\pgfsetbuttcap%
\pgfsetmiterjoin%
\definecolor{currentfill}{rgb}{0.501961,0.501961,0.501961}%
\pgfsetfillcolor{currentfill}%
\pgfsetfillopacity{0.700000}%
\pgfsetlinewidth{0.000000pt}%
\definecolor{currentstroke}{rgb}{0.000000,0.000000,0.000000}%
\pgfsetstrokecolor{currentstroke}%
\pgfsetstrokeopacity{0.700000}%
\pgfsetdash{}{0pt}%
\pgfpathmoveto{\pgfqpoint{7.464311in}{0.550000in}}%
\pgfpathlineto{\pgfqpoint{7.491832in}{0.550000in}}%
\pgfpathlineto{\pgfqpoint{7.491832in}{0.550153in}}%
\pgfpathlineto{\pgfqpoint{7.464311in}{0.550153in}}%
\pgfpathlineto{\pgfqpoint{7.464311in}{0.550000in}}%
\pgfpathclose%
\pgfusepath{fill}%
\end{pgfscope}%
\begin{pgfscope}%
\pgfpathrectangle{\pgfqpoint{1.250000in}{0.550000in}}{\pgfqpoint{7.750000in}{3.850000in}}%
\pgfusepath{clip}%
\pgfsetbuttcap%
\pgfsetmiterjoin%
\definecolor{currentfill}{rgb}{0.501961,0.501961,0.501961}%
\pgfsetfillcolor{currentfill}%
\pgfsetfillopacity{0.700000}%
\pgfsetlinewidth{0.000000pt}%
\definecolor{currentstroke}{rgb}{0.000000,0.000000,0.000000}%
\pgfsetstrokecolor{currentstroke}%
\pgfsetstrokeopacity{0.700000}%
\pgfsetdash{}{0pt}%
\pgfpathmoveto{\pgfqpoint{7.491832in}{0.550000in}}%
\pgfpathlineto{\pgfqpoint{7.519354in}{0.550000in}}%
\pgfpathlineto{\pgfqpoint{7.519354in}{0.550000in}}%
\pgfpathlineto{\pgfqpoint{7.491832in}{0.550000in}}%
\pgfpathlineto{\pgfqpoint{7.491832in}{0.550000in}}%
\pgfpathclose%
\pgfusepath{fill}%
\end{pgfscope}%
\begin{pgfscope}%
\pgfpathrectangle{\pgfqpoint{1.250000in}{0.550000in}}{\pgfqpoint{7.750000in}{3.850000in}}%
\pgfusepath{clip}%
\pgfsetbuttcap%
\pgfsetmiterjoin%
\definecolor{currentfill}{rgb}{0.501961,0.501961,0.501961}%
\pgfsetfillcolor{currentfill}%
\pgfsetfillopacity{0.700000}%
\pgfsetlinewidth{0.000000pt}%
\definecolor{currentstroke}{rgb}{0.000000,0.000000,0.000000}%
\pgfsetstrokecolor{currentstroke}%
\pgfsetstrokeopacity{0.700000}%
\pgfsetdash{}{0pt}%
\pgfpathmoveto{\pgfqpoint{7.519354in}{0.550000in}}%
\pgfpathlineto{\pgfqpoint{7.546875in}{0.550000in}}%
\pgfpathlineto{\pgfqpoint{7.546875in}{0.550000in}}%
\pgfpathlineto{\pgfqpoint{7.519354in}{0.550000in}}%
\pgfpathlineto{\pgfqpoint{7.519354in}{0.550000in}}%
\pgfpathclose%
\pgfusepath{fill}%
\end{pgfscope}%
\begin{pgfscope}%
\pgfpathrectangle{\pgfqpoint{1.250000in}{0.550000in}}{\pgfqpoint{7.750000in}{3.850000in}}%
\pgfusepath{clip}%
\pgfsetbuttcap%
\pgfsetmiterjoin%
\definecolor{currentfill}{rgb}{0.501961,0.501961,0.501961}%
\pgfsetfillcolor{currentfill}%
\pgfsetfillopacity{0.700000}%
\pgfsetlinewidth{0.000000pt}%
\definecolor{currentstroke}{rgb}{0.000000,0.000000,0.000000}%
\pgfsetstrokecolor{currentstroke}%
\pgfsetstrokeopacity{0.700000}%
\pgfsetdash{}{0pt}%
\pgfpathmoveto{\pgfqpoint{7.546875in}{0.550000in}}%
\pgfpathlineto{\pgfqpoint{7.574396in}{0.550000in}}%
\pgfpathlineto{\pgfqpoint{7.574396in}{0.550000in}}%
\pgfpathlineto{\pgfqpoint{7.546875in}{0.550000in}}%
\pgfpathlineto{\pgfqpoint{7.546875in}{0.550000in}}%
\pgfpathclose%
\pgfusepath{fill}%
\end{pgfscope}%
\begin{pgfscope}%
\pgfpathrectangle{\pgfqpoint{1.250000in}{0.550000in}}{\pgfqpoint{7.750000in}{3.850000in}}%
\pgfusepath{clip}%
\pgfsetbuttcap%
\pgfsetmiterjoin%
\definecolor{currentfill}{rgb}{0.501961,0.501961,0.501961}%
\pgfsetfillcolor{currentfill}%
\pgfsetfillopacity{0.700000}%
\pgfsetlinewidth{0.000000pt}%
\definecolor{currentstroke}{rgb}{0.000000,0.000000,0.000000}%
\pgfsetstrokecolor{currentstroke}%
\pgfsetstrokeopacity{0.700000}%
\pgfsetdash{}{0pt}%
\pgfpathmoveto{\pgfqpoint{7.574396in}{0.550000in}}%
\pgfpathlineto{\pgfqpoint{7.601918in}{0.550000in}}%
\pgfpathlineto{\pgfqpoint{7.601918in}{0.550000in}}%
\pgfpathlineto{\pgfqpoint{7.574396in}{0.550000in}}%
\pgfpathlineto{\pgfqpoint{7.574396in}{0.550000in}}%
\pgfpathclose%
\pgfusepath{fill}%
\end{pgfscope}%
\begin{pgfscope}%
\pgfpathrectangle{\pgfqpoint{1.250000in}{0.550000in}}{\pgfqpoint{7.750000in}{3.850000in}}%
\pgfusepath{clip}%
\pgfsetbuttcap%
\pgfsetmiterjoin%
\definecolor{currentfill}{rgb}{0.501961,0.501961,0.501961}%
\pgfsetfillcolor{currentfill}%
\pgfsetfillopacity{0.700000}%
\pgfsetlinewidth{0.000000pt}%
\definecolor{currentstroke}{rgb}{0.000000,0.000000,0.000000}%
\pgfsetstrokecolor{currentstroke}%
\pgfsetstrokeopacity{0.700000}%
\pgfsetdash{}{0pt}%
\pgfpathmoveto{\pgfqpoint{7.601918in}{0.550000in}}%
\pgfpathlineto{\pgfqpoint{7.629439in}{0.550000in}}%
\pgfpathlineto{\pgfqpoint{7.629439in}{0.550000in}}%
\pgfpathlineto{\pgfqpoint{7.601918in}{0.550000in}}%
\pgfpathlineto{\pgfqpoint{7.601918in}{0.550000in}}%
\pgfpathclose%
\pgfusepath{fill}%
\end{pgfscope}%
\begin{pgfscope}%
\pgfpathrectangle{\pgfqpoint{1.250000in}{0.550000in}}{\pgfqpoint{7.750000in}{3.850000in}}%
\pgfusepath{clip}%
\pgfsetbuttcap%
\pgfsetmiterjoin%
\definecolor{currentfill}{rgb}{0.501961,0.501961,0.501961}%
\pgfsetfillcolor{currentfill}%
\pgfsetfillopacity{0.700000}%
\pgfsetlinewidth{0.000000pt}%
\definecolor{currentstroke}{rgb}{0.000000,0.000000,0.000000}%
\pgfsetstrokecolor{currentstroke}%
\pgfsetstrokeopacity{0.700000}%
\pgfsetdash{}{0pt}%
\pgfpathmoveto{\pgfqpoint{7.629439in}{0.550000in}}%
\pgfpathlineto{\pgfqpoint{7.656960in}{0.550000in}}%
\pgfpathlineto{\pgfqpoint{7.656960in}{0.550000in}}%
\pgfpathlineto{\pgfqpoint{7.629439in}{0.550000in}}%
\pgfpathlineto{\pgfqpoint{7.629439in}{0.550000in}}%
\pgfpathclose%
\pgfusepath{fill}%
\end{pgfscope}%
\begin{pgfscope}%
\pgfpathrectangle{\pgfqpoint{1.250000in}{0.550000in}}{\pgfqpoint{7.750000in}{3.850000in}}%
\pgfusepath{clip}%
\pgfsetbuttcap%
\pgfsetmiterjoin%
\definecolor{currentfill}{rgb}{0.501961,0.501961,0.501961}%
\pgfsetfillcolor{currentfill}%
\pgfsetfillopacity{0.700000}%
\pgfsetlinewidth{0.000000pt}%
\definecolor{currentstroke}{rgb}{0.000000,0.000000,0.000000}%
\pgfsetstrokecolor{currentstroke}%
\pgfsetstrokeopacity{0.700000}%
\pgfsetdash{}{0pt}%
\pgfpathmoveto{\pgfqpoint{7.656960in}{0.550000in}}%
\pgfpathlineto{\pgfqpoint{7.684482in}{0.550000in}}%
\pgfpathlineto{\pgfqpoint{7.684482in}{0.550306in}}%
\pgfpathlineto{\pgfqpoint{7.656960in}{0.550306in}}%
\pgfpathlineto{\pgfqpoint{7.656960in}{0.550000in}}%
\pgfpathclose%
\pgfusepath{fill}%
\end{pgfscope}%
\begin{pgfscope}%
\pgfpathrectangle{\pgfqpoint{1.250000in}{0.550000in}}{\pgfqpoint{7.750000in}{3.850000in}}%
\pgfusepath{clip}%
\pgfsetbuttcap%
\pgfsetmiterjoin%
\definecolor{currentfill}{rgb}{0.501961,0.501961,0.501961}%
\pgfsetfillcolor{currentfill}%
\pgfsetfillopacity{0.700000}%
\pgfsetlinewidth{0.000000pt}%
\definecolor{currentstroke}{rgb}{0.000000,0.000000,0.000000}%
\pgfsetstrokecolor{currentstroke}%
\pgfsetstrokeopacity{0.700000}%
\pgfsetdash{}{0pt}%
\pgfpathmoveto{\pgfqpoint{7.684482in}{0.550000in}}%
\pgfpathlineto{\pgfqpoint{7.712003in}{0.550000in}}%
\pgfpathlineto{\pgfqpoint{7.712003in}{0.550000in}}%
\pgfpathlineto{\pgfqpoint{7.684482in}{0.550000in}}%
\pgfpathlineto{\pgfqpoint{7.684482in}{0.550000in}}%
\pgfpathclose%
\pgfusepath{fill}%
\end{pgfscope}%
\begin{pgfscope}%
\pgfpathrectangle{\pgfqpoint{1.250000in}{0.550000in}}{\pgfqpoint{7.750000in}{3.850000in}}%
\pgfusepath{clip}%
\pgfsetbuttcap%
\pgfsetmiterjoin%
\definecolor{currentfill}{rgb}{0.501961,0.501961,0.501961}%
\pgfsetfillcolor{currentfill}%
\pgfsetfillopacity{0.700000}%
\pgfsetlinewidth{0.000000pt}%
\definecolor{currentstroke}{rgb}{0.000000,0.000000,0.000000}%
\pgfsetstrokecolor{currentstroke}%
\pgfsetstrokeopacity{0.700000}%
\pgfsetdash{}{0pt}%
\pgfpathmoveto{\pgfqpoint{7.712003in}{0.550000in}}%
\pgfpathlineto{\pgfqpoint{7.739524in}{0.550000in}}%
\pgfpathlineto{\pgfqpoint{7.739524in}{0.550153in}}%
\pgfpathlineto{\pgfqpoint{7.712003in}{0.550153in}}%
\pgfpathlineto{\pgfqpoint{7.712003in}{0.550000in}}%
\pgfpathclose%
\pgfusepath{fill}%
\end{pgfscope}%
\begin{pgfscope}%
\pgfpathrectangle{\pgfqpoint{1.250000in}{0.550000in}}{\pgfqpoint{7.750000in}{3.850000in}}%
\pgfusepath{clip}%
\pgfsetbuttcap%
\pgfsetmiterjoin%
\definecolor{currentfill}{rgb}{0.501961,0.501961,0.501961}%
\pgfsetfillcolor{currentfill}%
\pgfsetfillopacity{0.700000}%
\pgfsetlinewidth{0.000000pt}%
\definecolor{currentstroke}{rgb}{0.000000,0.000000,0.000000}%
\pgfsetstrokecolor{currentstroke}%
\pgfsetstrokeopacity{0.700000}%
\pgfsetdash{}{0pt}%
\pgfpathmoveto{\pgfqpoint{7.739524in}{0.550000in}}%
\pgfpathlineto{\pgfqpoint{7.767045in}{0.550000in}}%
\pgfpathlineto{\pgfqpoint{7.767045in}{0.550000in}}%
\pgfpathlineto{\pgfqpoint{7.739524in}{0.550000in}}%
\pgfpathlineto{\pgfqpoint{7.739524in}{0.550000in}}%
\pgfpathclose%
\pgfusepath{fill}%
\end{pgfscope}%
\begin{pgfscope}%
\pgfpathrectangle{\pgfqpoint{1.250000in}{0.550000in}}{\pgfqpoint{7.750000in}{3.850000in}}%
\pgfusepath{clip}%
\pgfsetbuttcap%
\pgfsetmiterjoin%
\definecolor{currentfill}{rgb}{0.501961,0.501961,0.501961}%
\pgfsetfillcolor{currentfill}%
\pgfsetfillopacity{0.700000}%
\pgfsetlinewidth{0.000000pt}%
\definecolor{currentstroke}{rgb}{0.000000,0.000000,0.000000}%
\pgfsetstrokecolor{currentstroke}%
\pgfsetstrokeopacity{0.700000}%
\pgfsetdash{}{0pt}%
\pgfpathmoveto{\pgfqpoint{7.767045in}{0.550000in}}%
\pgfpathlineto{\pgfqpoint{7.794567in}{0.550000in}}%
\pgfpathlineto{\pgfqpoint{7.794567in}{0.550153in}}%
\pgfpathlineto{\pgfqpoint{7.767045in}{0.550153in}}%
\pgfpathlineto{\pgfqpoint{7.767045in}{0.550000in}}%
\pgfpathclose%
\pgfusepath{fill}%
\end{pgfscope}%
\begin{pgfscope}%
\pgfpathrectangle{\pgfqpoint{1.250000in}{0.550000in}}{\pgfqpoint{7.750000in}{3.850000in}}%
\pgfusepath{clip}%
\pgfsetbuttcap%
\pgfsetmiterjoin%
\definecolor{currentfill}{rgb}{0.501961,0.501961,0.501961}%
\pgfsetfillcolor{currentfill}%
\pgfsetfillopacity{0.700000}%
\pgfsetlinewidth{0.000000pt}%
\definecolor{currentstroke}{rgb}{0.000000,0.000000,0.000000}%
\pgfsetstrokecolor{currentstroke}%
\pgfsetstrokeopacity{0.700000}%
\pgfsetdash{}{0pt}%
\pgfpathmoveto{\pgfqpoint{7.794567in}{0.550000in}}%
\pgfpathlineto{\pgfqpoint{7.822088in}{0.550000in}}%
\pgfpathlineto{\pgfqpoint{7.822088in}{0.550153in}}%
\pgfpathlineto{\pgfqpoint{7.794567in}{0.550153in}}%
\pgfpathlineto{\pgfqpoint{7.794567in}{0.550000in}}%
\pgfpathclose%
\pgfusepath{fill}%
\end{pgfscope}%
\begin{pgfscope}%
\pgfpathrectangle{\pgfqpoint{1.250000in}{0.550000in}}{\pgfqpoint{7.750000in}{3.850000in}}%
\pgfusepath{clip}%
\pgfsetbuttcap%
\pgfsetmiterjoin%
\definecolor{currentfill}{rgb}{0.501961,0.501961,0.501961}%
\pgfsetfillcolor{currentfill}%
\pgfsetfillopacity{0.700000}%
\pgfsetlinewidth{0.000000pt}%
\definecolor{currentstroke}{rgb}{0.000000,0.000000,0.000000}%
\pgfsetstrokecolor{currentstroke}%
\pgfsetstrokeopacity{0.700000}%
\pgfsetdash{}{0pt}%
\pgfpathmoveto{\pgfqpoint{7.822088in}{0.550000in}}%
\pgfpathlineto{\pgfqpoint{7.849609in}{0.550000in}}%
\pgfpathlineto{\pgfqpoint{7.849609in}{0.550153in}}%
\pgfpathlineto{\pgfqpoint{7.822088in}{0.550153in}}%
\pgfpathlineto{\pgfqpoint{7.822088in}{0.550000in}}%
\pgfpathclose%
\pgfusepath{fill}%
\end{pgfscope}%
\begin{pgfscope}%
\pgfpathrectangle{\pgfqpoint{1.250000in}{0.550000in}}{\pgfqpoint{7.750000in}{3.850000in}}%
\pgfusepath{clip}%
\pgfsetbuttcap%
\pgfsetmiterjoin%
\definecolor{currentfill}{rgb}{0.501961,0.501961,0.501961}%
\pgfsetfillcolor{currentfill}%
\pgfsetfillopacity{0.700000}%
\pgfsetlinewidth{0.000000pt}%
\definecolor{currentstroke}{rgb}{0.000000,0.000000,0.000000}%
\pgfsetstrokecolor{currentstroke}%
\pgfsetstrokeopacity{0.700000}%
\pgfsetdash{}{0pt}%
\pgfpathmoveto{\pgfqpoint{7.849609in}{0.550000in}}%
\pgfpathlineto{\pgfqpoint{7.877131in}{0.550000in}}%
\pgfpathlineto{\pgfqpoint{7.877131in}{0.550000in}}%
\pgfpathlineto{\pgfqpoint{7.849609in}{0.550000in}}%
\pgfpathlineto{\pgfqpoint{7.849609in}{0.550000in}}%
\pgfpathclose%
\pgfusepath{fill}%
\end{pgfscope}%
\begin{pgfscope}%
\pgfpathrectangle{\pgfqpoint{1.250000in}{0.550000in}}{\pgfqpoint{7.750000in}{3.850000in}}%
\pgfusepath{clip}%
\pgfsetbuttcap%
\pgfsetmiterjoin%
\definecolor{currentfill}{rgb}{0.501961,0.501961,0.501961}%
\pgfsetfillcolor{currentfill}%
\pgfsetfillopacity{0.700000}%
\pgfsetlinewidth{0.000000pt}%
\definecolor{currentstroke}{rgb}{0.000000,0.000000,0.000000}%
\pgfsetstrokecolor{currentstroke}%
\pgfsetstrokeopacity{0.700000}%
\pgfsetdash{}{0pt}%
\pgfpathmoveto{\pgfqpoint{7.877131in}{0.550000in}}%
\pgfpathlineto{\pgfqpoint{7.904652in}{0.550000in}}%
\pgfpathlineto{\pgfqpoint{7.904652in}{0.550000in}}%
\pgfpathlineto{\pgfqpoint{7.877131in}{0.550000in}}%
\pgfpathlineto{\pgfqpoint{7.877131in}{0.550000in}}%
\pgfpathclose%
\pgfusepath{fill}%
\end{pgfscope}%
\begin{pgfscope}%
\pgfpathrectangle{\pgfqpoint{1.250000in}{0.550000in}}{\pgfqpoint{7.750000in}{3.850000in}}%
\pgfusepath{clip}%
\pgfsetbuttcap%
\pgfsetmiterjoin%
\definecolor{currentfill}{rgb}{0.501961,0.501961,0.501961}%
\pgfsetfillcolor{currentfill}%
\pgfsetfillopacity{0.700000}%
\pgfsetlinewidth{0.000000pt}%
\definecolor{currentstroke}{rgb}{0.000000,0.000000,0.000000}%
\pgfsetstrokecolor{currentstroke}%
\pgfsetstrokeopacity{0.700000}%
\pgfsetdash{}{0pt}%
\pgfpathmoveto{\pgfqpoint{7.904652in}{0.550000in}}%
\pgfpathlineto{\pgfqpoint{7.932173in}{0.550000in}}%
\pgfpathlineto{\pgfqpoint{7.932173in}{0.550153in}}%
\pgfpathlineto{\pgfqpoint{7.904652in}{0.550153in}}%
\pgfpathlineto{\pgfqpoint{7.904652in}{0.550000in}}%
\pgfpathclose%
\pgfusepath{fill}%
\end{pgfscope}%
\begin{pgfscope}%
\pgfpathrectangle{\pgfqpoint{1.250000in}{0.550000in}}{\pgfqpoint{7.750000in}{3.850000in}}%
\pgfusepath{clip}%
\pgfsetbuttcap%
\pgfsetmiterjoin%
\definecolor{currentfill}{rgb}{0.501961,0.501961,0.501961}%
\pgfsetfillcolor{currentfill}%
\pgfsetfillopacity{0.700000}%
\pgfsetlinewidth{0.000000pt}%
\definecolor{currentstroke}{rgb}{0.000000,0.000000,0.000000}%
\pgfsetstrokecolor{currentstroke}%
\pgfsetstrokeopacity{0.700000}%
\pgfsetdash{}{0pt}%
\pgfpathmoveto{\pgfqpoint{7.932173in}{0.550000in}}%
\pgfpathlineto{\pgfqpoint{7.959695in}{0.550000in}}%
\pgfpathlineto{\pgfqpoint{7.959695in}{0.550306in}}%
\pgfpathlineto{\pgfqpoint{7.932173in}{0.550306in}}%
\pgfpathlineto{\pgfqpoint{7.932173in}{0.550000in}}%
\pgfpathclose%
\pgfusepath{fill}%
\end{pgfscope}%
\begin{pgfscope}%
\pgfpathrectangle{\pgfqpoint{1.250000in}{0.550000in}}{\pgfqpoint{7.750000in}{3.850000in}}%
\pgfusepath{clip}%
\pgfsetbuttcap%
\pgfsetmiterjoin%
\definecolor{currentfill}{rgb}{0.501961,0.501961,0.501961}%
\pgfsetfillcolor{currentfill}%
\pgfsetfillopacity{0.700000}%
\pgfsetlinewidth{0.000000pt}%
\definecolor{currentstroke}{rgb}{0.000000,0.000000,0.000000}%
\pgfsetstrokecolor{currentstroke}%
\pgfsetstrokeopacity{0.700000}%
\pgfsetdash{}{0pt}%
\pgfpathmoveto{\pgfqpoint{7.959695in}{0.550000in}}%
\pgfpathlineto{\pgfqpoint{7.987216in}{0.550000in}}%
\pgfpathlineto{\pgfqpoint{7.987216in}{0.550000in}}%
\pgfpathlineto{\pgfqpoint{7.959695in}{0.550000in}}%
\pgfpathlineto{\pgfqpoint{7.959695in}{0.550000in}}%
\pgfpathclose%
\pgfusepath{fill}%
\end{pgfscope}%
\begin{pgfscope}%
\pgfpathrectangle{\pgfqpoint{1.250000in}{0.550000in}}{\pgfqpoint{7.750000in}{3.850000in}}%
\pgfusepath{clip}%
\pgfsetbuttcap%
\pgfsetmiterjoin%
\definecolor{currentfill}{rgb}{0.501961,0.501961,0.501961}%
\pgfsetfillcolor{currentfill}%
\pgfsetfillopacity{0.700000}%
\pgfsetlinewidth{0.000000pt}%
\definecolor{currentstroke}{rgb}{0.000000,0.000000,0.000000}%
\pgfsetstrokecolor{currentstroke}%
\pgfsetstrokeopacity{0.700000}%
\pgfsetdash{}{0pt}%
\pgfpathmoveto{\pgfqpoint{7.987216in}{0.550000in}}%
\pgfpathlineto{\pgfqpoint{8.014737in}{0.550000in}}%
\pgfpathlineto{\pgfqpoint{8.014737in}{0.550153in}}%
\pgfpathlineto{\pgfqpoint{7.987216in}{0.550153in}}%
\pgfpathlineto{\pgfqpoint{7.987216in}{0.550000in}}%
\pgfpathclose%
\pgfusepath{fill}%
\end{pgfscope}%
\begin{pgfscope}%
\pgfpathrectangle{\pgfqpoint{1.250000in}{0.550000in}}{\pgfqpoint{7.750000in}{3.850000in}}%
\pgfusepath{clip}%
\pgfsetbuttcap%
\pgfsetmiterjoin%
\definecolor{currentfill}{rgb}{0.501961,0.501961,0.501961}%
\pgfsetfillcolor{currentfill}%
\pgfsetfillopacity{0.700000}%
\pgfsetlinewidth{0.000000pt}%
\definecolor{currentstroke}{rgb}{0.000000,0.000000,0.000000}%
\pgfsetstrokecolor{currentstroke}%
\pgfsetstrokeopacity{0.700000}%
\pgfsetdash{}{0pt}%
\pgfpathmoveto{\pgfqpoint{8.014737in}{0.550000in}}%
\pgfpathlineto{\pgfqpoint{8.042259in}{0.550000in}}%
\pgfpathlineto{\pgfqpoint{8.042259in}{0.550153in}}%
\pgfpathlineto{\pgfqpoint{8.014737in}{0.550153in}}%
\pgfpathlineto{\pgfqpoint{8.014737in}{0.550000in}}%
\pgfpathclose%
\pgfusepath{fill}%
\end{pgfscope}%
\begin{pgfscope}%
\pgfpathrectangle{\pgfqpoint{1.250000in}{0.550000in}}{\pgfqpoint{7.750000in}{3.850000in}}%
\pgfusepath{clip}%
\pgfsetbuttcap%
\pgfsetmiterjoin%
\definecolor{currentfill}{rgb}{0.501961,0.501961,0.501961}%
\pgfsetfillcolor{currentfill}%
\pgfsetfillopacity{0.700000}%
\pgfsetlinewidth{0.000000pt}%
\definecolor{currentstroke}{rgb}{0.000000,0.000000,0.000000}%
\pgfsetstrokecolor{currentstroke}%
\pgfsetstrokeopacity{0.700000}%
\pgfsetdash{}{0pt}%
\pgfpathmoveto{\pgfqpoint{8.042259in}{0.550000in}}%
\pgfpathlineto{\pgfqpoint{8.069780in}{0.550000in}}%
\pgfpathlineto{\pgfqpoint{8.069780in}{0.550000in}}%
\pgfpathlineto{\pgfqpoint{8.042259in}{0.550000in}}%
\pgfpathlineto{\pgfqpoint{8.042259in}{0.550000in}}%
\pgfpathclose%
\pgfusepath{fill}%
\end{pgfscope}%
\begin{pgfscope}%
\pgfpathrectangle{\pgfqpoint{1.250000in}{0.550000in}}{\pgfqpoint{7.750000in}{3.850000in}}%
\pgfusepath{clip}%
\pgfsetbuttcap%
\pgfsetmiterjoin%
\definecolor{currentfill}{rgb}{0.501961,0.501961,0.501961}%
\pgfsetfillcolor{currentfill}%
\pgfsetfillopacity{0.700000}%
\pgfsetlinewidth{0.000000pt}%
\definecolor{currentstroke}{rgb}{0.000000,0.000000,0.000000}%
\pgfsetstrokecolor{currentstroke}%
\pgfsetstrokeopacity{0.700000}%
\pgfsetdash{}{0pt}%
\pgfpathmoveto{\pgfqpoint{8.069780in}{0.550000in}}%
\pgfpathlineto{\pgfqpoint{8.097301in}{0.550000in}}%
\pgfpathlineto{\pgfqpoint{8.097301in}{0.550000in}}%
\pgfpathlineto{\pgfqpoint{8.069780in}{0.550000in}}%
\pgfpathlineto{\pgfqpoint{8.069780in}{0.550000in}}%
\pgfpathclose%
\pgfusepath{fill}%
\end{pgfscope}%
\begin{pgfscope}%
\pgfpathrectangle{\pgfqpoint{1.250000in}{0.550000in}}{\pgfqpoint{7.750000in}{3.850000in}}%
\pgfusepath{clip}%
\pgfsetbuttcap%
\pgfsetmiterjoin%
\definecolor{currentfill}{rgb}{0.501961,0.501961,0.501961}%
\pgfsetfillcolor{currentfill}%
\pgfsetfillopacity{0.700000}%
\pgfsetlinewidth{0.000000pt}%
\definecolor{currentstroke}{rgb}{0.000000,0.000000,0.000000}%
\pgfsetstrokecolor{currentstroke}%
\pgfsetstrokeopacity{0.700000}%
\pgfsetdash{}{0pt}%
\pgfpathmoveto{\pgfqpoint{8.097301in}{0.550000in}}%
\pgfpathlineto{\pgfqpoint{8.124822in}{0.550000in}}%
\pgfpathlineto{\pgfqpoint{8.124822in}{0.550153in}}%
\pgfpathlineto{\pgfqpoint{8.097301in}{0.550153in}}%
\pgfpathlineto{\pgfqpoint{8.097301in}{0.550000in}}%
\pgfpathclose%
\pgfusepath{fill}%
\end{pgfscope}%
\begin{pgfscope}%
\pgfpathrectangle{\pgfqpoint{1.250000in}{0.550000in}}{\pgfqpoint{7.750000in}{3.850000in}}%
\pgfusepath{clip}%
\pgfsetbuttcap%
\pgfsetmiterjoin%
\definecolor{currentfill}{rgb}{0.501961,0.501961,0.501961}%
\pgfsetfillcolor{currentfill}%
\pgfsetfillopacity{0.700000}%
\pgfsetlinewidth{0.000000pt}%
\definecolor{currentstroke}{rgb}{0.000000,0.000000,0.000000}%
\pgfsetstrokecolor{currentstroke}%
\pgfsetstrokeopacity{0.700000}%
\pgfsetdash{}{0pt}%
\pgfpathmoveto{\pgfqpoint{8.124822in}{0.550000in}}%
\pgfpathlineto{\pgfqpoint{8.152344in}{0.550000in}}%
\pgfpathlineto{\pgfqpoint{8.152344in}{0.550306in}}%
\pgfpathlineto{\pgfqpoint{8.124822in}{0.550306in}}%
\pgfpathlineto{\pgfqpoint{8.124822in}{0.550000in}}%
\pgfpathclose%
\pgfusepath{fill}%
\end{pgfscope}%
\begin{pgfscope}%
\pgfpathrectangle{\pgfqpoint{1.250000in}{0.550000in}}{\pgfqpoint{7.750000in}{3.850000in}}%
\pgfusepath{clip}%
\pgfsetbuttcap%
\pgfsetmiterjoin%
\definecolor{currentfill}{rgb}{0.501961,0.501961,0.501961}%
\pgfsetfillcolor{currentfill}%
\pgfsetfillopacity{0.700000}%
\pgfsetlinewidth{0.000000pt}%
\definecolor{currentstroke}{rgb}{0.000000,0.000000,0.000000}%
\pgfsetstrokecolor{currentstroke}%
\pgfsetstrokeopacity{0.700000}%
\pgfsetdash{}{0pt}%
\pgfpathmoveto{\pgfqpoint{8.152344in}{0.550000in}}%
\pgfpathlineto{\pgfqpoint{8.179865in}{0.550000in}}%
\pgfpathlineto{\pgfqpoint{8.179865in}{0.550000in}}%
\pgfpathlineto{\pgfqpoint{8.152344in}{0.550000in}}%
\pgfpathlineto{\pgfqpoint{8.152344in}{0.550000in}}%
\pgfpathclose%
\pgfusepath{fill}%
\end{pgfscope}%
\begin{pgfscope}%
\pgfpathrectangle{\pgfqpoint{1.250000in}{0.550000in}}{\pgfqpoint{7.750000in}{3.850000in}}%
\pgfusepath{clip}%
\pgfsetbuttcap%
\pgfsetmiterjoin%
\definecolor{currentfill}{rgb}{0.501961,0.501961,0.501961}%
\pgfsetfillcolor{currentfill}%
\pgfsetfillopacity{0.700000}%
\pgfsetlinewidth{0.000000pt}%
\definecolor{currentstroke}{rgb}{0.000000,0.000000,0.000000}%
\pgfsetstrokecolor{currentstroke}%
\pgfsetstrokeopacity{0.700000}%
\pgfsetdash{}{0pt}%
\pgfpathmoveto{\pgfqpoint{8.179865in}{0.550000in}}%
\pgfpathlineto{\pgfqpoint{8.207386in}{0.550000in}}%
\pgfpathlineto{\pgfqpoint{8.207386in}{0.550000in}}%
\pgfpathlineto{\pgfqpoint{8.179865in}{0.550000in}}%
\pgfpathlineto{\pgfqpoint{8.179865in}{0.550000in}}%
\pgfpathclose%
\pgfusepath{fill}%
\end{pgfscope}%
\begin{pgfscope}%
\pgfpathrectangle{\pgfqpoint{1.250000in}{0.550000in}}{\pgfqpoint{7.750000in}{3.850000in}}%
\pgfusepath{clip}%
\pgfsetbuttcap%
\pgfsetmiterjoin%
\definecolor{currentfill}{rgb}{0.501961,0.501961,0.501961}%
\pgfsetfillcolor{currentfill}%
\pgfsetfillopacity{0.700000}%
\pgfsetlinewidth{0.000000pt}%
\definecolor{currentstroke}{rgb}{0.000000,0.000000,0.000000}%
\pgfsetstrokecolor{currentstroke}%
\pgfsetstrokeopacity{0.700000}%
\pgfsetdash{}{0pt}%
\pgfpathmoveto{\pgfqpoint{8.207386in}{0.550000in}}%
\pgfpathlineto{\pgfqpoint{8.234908in}{0.550000in}}%
\pgfpathlineto{\pgfqpoint{8.234908in}{0.550000in}}%
\pgfpathlineto{\pgfqpoint{8.207386in}{0.550000in}}%
\pgfpathlineto{\pgfqpoint{8.207386in}{0.550000in}}%
\pgfpathclose%
\pgfusepath{fill}%
\end{pgfscope}%
\begin{pgfscope}%
\pgfpathrectangle{\pgfqpoint{1.250000in}{0.550000in}}{\pgfqpoint{7.750000in}{3.850000in}}%
\pgfusepath{clip}%
\pgfsetbuttcap%
\pgfsetmiterjoin%
\definecolor{currentfill}{rgb}{0.501961,0.501961,0.501961}%
\pgfsetfillcolor{currentfill}%
\pgfsetfillopacity{0.700000}%
\pgfsetlinewidth{0.000000pt}%
\definecolor{currentstroke}{rgb}{0.000000,0.000000,0.000000}%
\pgfsetstrokecolor{currentstroke}%
\pgfsetstrokeopacity{0.700000}%
\pgfsetdash{}{0pt}%
\pgfpathmoveto{\pgfqpoint{8.234908in}{0.550000in}}%
\pgfpathlineto{\pgfqpoint{8.262429in}{0.550000in}}%
\pgfpathlineto{\pgfqpoint{8.262429in}{0.550000in}}%
\pgfpathlineto{\pgfqpoint{8.234908in}{0.550000in}}%
\pgfpathlineto{\pgfqpoint{8.234908in}{0.550000in}}%
\pgfpathclose%
\pgfusepath{fill}%
\end{pgfscope}%
\begin{pgfscope}%
\pgfpathrectangle{\pgfqpoint{1.250000in}{0.550000in}}{\pgfqpoint{7.750000in}{3.850000in}}%
\pgfusepath{clip}%
\pgfsetbuttcap%
\pgfsetmiterjoin%
\definecolor{currentfill}{rgb}{0.501961,0.501961,0.501961}%
\pgfsetfillcolor{currentfill}%
\pgfsetfillopacity{0.700000}%
\pgfsetlinewidth{0.000000pt}%
\definecolor{currentstroke}{rgb}{0.000000,0.000000,0.000000}%
\pgfsetstrokecolor{currentstroke}%
\pgfsetstrokeopacity{0.700000}%
\pgfsetdash{}{0pt}%
\pgfpathmoveto{\pgfqpoint{8.262429in}{0.550000in}}%
\pgfpathlineto{\pgfqpoint{8.289950in}{0.550000in}}%
\pgfpathlineto{\pgfqpoint{8.289950in}{0.550153in}}%
\pgfpathlineto{\pgfqpoint{8.262429in}{0.550153in}}%
\pgfpathlineto{\pgfqpoint{8.262429in}{0.550000in}}%
\pgfpathclose%
\pgfusepath{fill}%
\end{pgfscope}%
\begin{pgfscope}%
\pgfpathrectangle{\pgfqpoint{1.250000in}{0.550000in}}{\pgfqpoint{7.750000in}{3.850000in}}%
\pgfusepath{clip}%
\pgfsetbuttcap%
\pgfsetmiterjoin%
\definecolor{currentfill}{rgb}{0.501961,0.501961,0.501961}%
\pgfsetfillcolor{currentfill}%
\pgfsetfillopacity{0.700000}%
\pgfsetlinewidth{0.000000pt}%
\definecolor{currentstroke}{rgb}{0.000000,0.000000,0.000000}%
\pgfsetstrokecolor{currentstroke}%
\pgfsetstrokeopacity{0.700000}%
\pgfsetdash{}{0pt}%
\pgfpathmoveto{\pgfqpoint{8.289950in}{0.550000in}}%
\pgfpathlineto{\pgfqpoint{8.317472in}{0.550000in}}%
\pgfpathlineto{\pgfqpoint{8.317472in}{0.550000in}}%
\pgfpathlineto{\pgfqpoint{8.289950in}{0.550000in}}%
\pgfpathlineto{\pgfqpoint{8.289950in}{0.550000in}}%
\pgfpathclose%
\pgfusepath{fill}%
\end{pgfscope}%
\begin{pgfscope}%
\pgfpathrectangle{\pgfqpoint{1.250000in}{0.550000in}}{\pgfqpoint{7.750000in}{3.850000in}}%
\pgfusepath{clip}%
\pgfsetbuttcap%
\pgfsetmiterjoin%
\definecolor{currentfill}{rgb}{0.501961,0.501961,0.501961}%
\pgfsetfillcolor{currentfill}%
\pgfsetfillopacity{0.700000}%
\pgfsetlinewidth{0.000000pt}%
\definecolor{currentstroke}{rgb}{0.000000,0.000000,0.000000}%
\pgfsetstrokecolor{currentstroke}%
\pgfsetstrokeopacity{0.700000}%
\pgfsetdash{}{0pt}%
\pgfpathmoveto{\pgfqpoint{8.317472in}{0.550000in}}%
\pgfpathlineto{\pgfqpoint{8.344993in}{0.550000in}}%
\pgfpathlineto{\pgfqpoint{8.344993in}{0.550000in}}%
\pgfpathlineto{\pgfqpoint{8.317472in}{0.550000in}}%
\pgfpathlineto{\pgfqpoint{8.317472in}{0.550000in}}%
\pgfpathclose%
\pgfusepath{fill}%
\end{pgfscope}%
\begin{pgfscope}%
\pgfpathrectangle{\pgfqpoint{1.250000in}{0.550000in}}{\pgfqpoint{7.750000in}{3.850000in}}%
\pgfusepath{clip}%
\pgfsetbuttcap%
\pgfsetmiterjoin%
\definecolor{currentfill}{rgb}{0.501961,0.501961,0.501961}%
\pgfsetfillcolor{currentfill}%
\pgfsetfillopacity{0.700000}%
\pgfsetlinewidth{0.000000pt}%
\definecolor{currentstroke}{rgb}{0.000000,0.000000,0.000000}%
\pgfsetstrokecolor{currentstroke}%
\pgfsetstrokeopacity{0.700000}%
\pgfsetdash{}{0pt}%
\pgfpathmoveto{\pgfqpoint{8.344993in}{0.550000in}}%
\pgfpathlineto{\pgfqpoint{8.372514in}{0.550000in}}%
\pgfpathlineto{\pgfqpoint{8.372514in}{0.550000in}}%
\pgfpathlineto{\pgfqpoint{8.344993in}{0.550000in}}%
\pgfpathlineto{\pgfqpoint{8.344993in}{0.550000in}}%
\pgfpathclose%
\pgfusepath{fill}%
\end{pgfscope}%
\begin{pgfscope}%
\pgfpathrectangle{\pgfqpoint{1.250000in}{0.550000in}}{\pgfqpoint{7.750000in}{3.850000in}}%
\pgfusepath{clip}%
\pgfsetbuttcap%
\pgfsetmiterjoin%
\definecolor{currentfill}{rgb}{0.501961,0.501961,0.501961}%
\pgfsetfillcolor{currentfill}%
\pgfsetfillopacity{0.700000}%
\pgfsetlinewidth{0.000000pt}%
\definecolor{currentstroke}{rgb}{0.000000,0.000000,0.000000}%
\pgfsetstrokecolor{currentstroke}%
\pgfsetstrokeopacity{0.700000}%
\pgfsetdash{}{0pt}%
\pgfpathmoveto{\pgfqpoint{8.372514in}{0.550000in}}%
\pgfpathlineto{\pgfqpoint{8.400036in}{0.550000in}}%
\pgfpathlineto{\pgfqpoint{8.400036in}{0.550000in}}%
\pgfpathlineto{\pgfqpoint{8.372514in}{0.550000in}}%
\pgfpathlineto{\pgfqpoint{8.372514in}{0.550000in}}%
\pgfpathclose%
\pgfusepath{fill}%
\end{pgfscope}%
\begin{pgfscope}%
\pgfpathrectangle{\pgfqpoint{1.250000in}{0.550000in}}{\pgfqpoint{7.750000in}{3.850000in}}%
\pgfusepath{clip}%
\pgfsetbuttcap%
\pgfsetmiterjoin%
\definecolor{currentfill}{rgb}{0.501961,0.501961,0.501961}%
\pgfsetfillcolor{currentfill}%
\pgfsetfillopacity{0.700000}%
\pgfsetlinewidth{0.000000pt}%
\definecolor{currentstroke}{rgb}{0.000000,0.000000,0.000000}%
\pgfsetstrokecolor{currentstroke}%
\pgfsetstrokeopacity{0.700000}%
\pgfsetdash{}{0pt}%
\pgfpathmoveto{\pgfqpoint{8.400036in}{0.550000in}}%
\pgfpathlineto{\pgfqpoint{8.427557in}{0.550000in}}%
\pgfpathlineto{\pgfqpoint{8.427557in}{0.550000in}}%
\pgfpathlineto{\pgfqpoint{8.400036in}{0.550000in}}%
\pgfpathlineto{\pgfqpoint{8.400036in}{0.550000in}}%
\pgfpathclose%
\pgfusepath{fill}%
\end{pgfscope}%
\begin{pgfscope}%
\pgfpathrectangle{\pgfqpoint{1.250000in}{0.550000in}}{\pgfqpoint{7.750000in}{3.850000in}}%
\pgfusepath{clip}%
\pgfsetbuttcap%
\pgfsetmiterjoin%
\definecolor{currentfill}{rgb}{0.501961,0.501961,0.501961}%
\pgfsetfillcolor{currentfill}%
\pgfsetfillopacity{0.700000}%
\pgfsetlinewidth{0.000000pt}%
\definecolor{currentstroke}{rgb}{0.000000,0.000000,0.000000}%
\pgfsetstrokecolor{currentstroke}%
\pgfsetstrokeopacity{0.700000}%
\pgfsetdash{}{0pt}%
\pgfpathmoveto{\pgfqpoint{8.427557in}{0.550000in}}%
\pgfpathlineto{\pgfqpoint{8.455078in}{0.550000in}}%
\pgfpathlineto{\pgfqpoint{8.455078in}{0.550000in}}%
\pgfpathlineto{\pgfqpoint{8.427557in}{0.550000in}}%
\pgfpathlineto{\pgfqpoint{8.427557in}{0.550000in}}%
\pgfpathclose%
\pgfusepath{fill}%
\end{pgfscope}%
\begin{pgfscope}%
\pgfpathrectangle{\pgfqpoint{1.250000in}{0.550000in}}{\pgfqpoint{7.750000in}{3.850000in}}%
\pgfusepath{clip}%
\pgfsetbuttcap%
\pgfsetmiterjoin%
\definecolor{currentfill}{rgb}{0.501961,0.501961,0.501961}%
\pgfsetfillcolor{currentfill}%
\pgfsetfillopacity{0.700000}%
\pgfsetlinewidth{0.000000pt}%
\definecolor{currentstroke}{rgb}{0.000000,0.000000,0.000000}%
\pgfsetstrokecolor{currentstroke}%
\pgfsetstrokeopacity{0.700000}%
\pgfsetdash{}{0pt}%
\pgfpathmoveto{\pgfqpoint{8.455078in}{0.550000in}}%
\pgfpathlineto{\pgfqpoint{8.482599in}{0.550000in}}%
\pgfpathlineto{\pgfqpoint{8.482599in}{0.550000in}}%
\pgfpathlineto{\pgfqpoint{8.455078in}{0.550000in}}%
\pgfpathlineto{\pgfqpoint{8.455078in}{0.550000in}}%
\pgfpathclose%
\pgfusepath{fill}%
\end{pgfscope}%
\begin{pgfscope}%
\pgfpathrectangle{\pgfqpoint{1.250000in}{0.550000in}}{\pgfqpoint{7.750000in}{3.850000in}}%
\pgfusepath{clip}%
\pgfsetbuttcap%
\pgfsetmiterjoin%
\definecolor{currentfill}{rgb}{0.501961,0.501961,0.501961}%
\pgfsetfillcolor{currentfill}%
\pgfsetfillopacity{0.700000}%
\pgfsetlinewidth{0.000000pt}%
\definecolor{currentstroke}{rgb}{0.000000,0.000000,0.000000}%
\pgfsetstrokecolor{currentstroke}%
\pgfsetstrokeopacity{0.700000}%
\pgfsetdash{}{0pt}%
\pgfpathmoveto{\pgfqpoint{8.482599in}{0.550000in}}%
\pgfpathlineto{\pgfqpoint{8.510121in}{0.550000in}}%
\pgfpathlineto{\pgfqpoint{8.510121in}{0.550000in}}%
\pgfpathlineto{\pgfqpoint{8.482599in}{0.550000in}}%
\pgfpathlineto{\pgfqpoint{8.482599in}{0.550000in}}%
\pgfpathclose%
\pgfusepath{fill}%
\end{pgfscope}%
\begin{pgfscope}%
\pgfpathrectangle{\pgfqpoint{1.250000in}{0.550000in}}{\pgfqpoint{7.750000in}{3.850000in}}%
\pgfusepath{clip}%
\pgfsetbuttcap%
\pgfsetmiterjoin%
\definecolor{currentfill}{rgb}{0.501961,0.501961,0.501961}%
\pgfsetfillcolor{currentfill}%
\pgfsetfillopacity{0.700000}%
\pgfsetlinewidth{0.000000pt}%
\definecolor{currentstroke}{rgb}{0.000000,0.000000,0.000000}%
\pgfsetstrokecolor{currentstroke}%
\pgfsetstrokeopacity{0.700000}%
\pgfsetdash{}{0pt}%
\pgfpathmoveto{\pgfqpoint{8.510121in}{0.550000in}}%
\pgfpathlineto{\pgfqpoint{8.537642in}{0.550000in}}%
\pgfpathlineto{\pgfqpoint{8.537642in}{0.550000in}}%
\pgfpathlineto{\pgfqpoint{8.510121in}{0.550000in}}%
\pgfpathlineto{\pgfqpoint{8.510121in}{0.550000in}}%
\pgfpathclose%
\pgfusepath{fill}%
\end{pgfscope}%
\begin{pgfscope}%
\pgfpathrectangle{\pgfqpoint{1.250000in}{0.550000in}}{\pgfqpoint{7.750000in}{3.850000in}}%
\pgfusepath{clip}%
\pgfsetbuttcap%
\pgfsetmiterjoin%
\definecolor{currentfill}{rgb}{0.501961,0.501961,0.501961}%
\pgfsetfillcolor{currentfill}%
\pgfsetfillopacity{0.700000}%
\pgfsetlinewidth{0.000000pt}%
\definecolor{currentstroke}{rgb}{0.000000,0.000000,0.000000}%
\pgfsetstrokecolor{currentstroke}%
\pgfsetstrokeopacity{0.700000}%
\pgfsetdash{}{0pt}%
\pgfpathmoveto{\pgfqpoint{8.537642in}{0.550000in}}%
\pgfpathlineto{\pgfqpoint{8.565163in}{0.550000in}}%
\pgfpathlineto{\pgfqpoint{8.565163in}{0.550000in}}%
\pgfpathlineto{\pgfqpoint{8.537642in}{0.550000in}}%
\pgfpathlineto{\pgfqpoint{8.537642in}{0.550000in}}%
\pgfpathclose%
\pgfusepath{fill}%
\end{pgfscope}%
\begin{pgfscope}%
\pgfpathrectangle{\pgfqpoint{1.250000in}{0.550000in}}{\pgfqpoint{7.750000in}{3.850000in}}%
\pgfusepath{clip}%
\pgfsetbuttcap%
\pgfsetmiterjoin%
\definecolor{currentfill}{rgb}{0.501961,0.501961,0.501961}%
\pgfsetfillcolor{currentfill}%
\pgfsetfillopacity{0.700000}%
\pgfsetlinewidth{0.000000pt}%
\definecolor{currentstroke}{rgb}{0.000000,0.000000,0.000000}%
\pgfsetstrokecolor{currentstroke}%
\pgfsetstrokeopacity{0.700000}%
\pgfsetdash{}{0pt}%
\pgfpathmoveto{\pgfqpoint{8.565163in}{0.550000in}}%
\pgfpathlineto{\pgfqpoint{8.592685in}{0.550000in}}%
\pgfpathlineto{\pgfqpoint{8.592685in}{0.550306in}}%
\pgfpathlineto{\pgfqpoint{8.565163in}{0.550306in}}%
\pgfpathlineto{\pgfqpoint{8.565163in}{0.550000in}}%
\pgfpathclose%
\pgfusepath{fill}%
\end{pgfscope}%
\begin{pgfscope}%
\pgfpathrectangle{\pgfqpoint{1.250000in}{0.550000in}}{\pgfqpoint{7.750000in}{3.850000in}}%
\pgfusepath{clip}%
\pgfsetbuttcap%
\pgfsetmiterjoin%
\definecolor{currentfill}{rgb}{0.501961,0.501961,0.501961}%
\pgfsetfillcolor{currentfill}%
\pgfsetfillopacity{0.700000}%
\pgfsetlinewidth{0.000000pt}%
\definecolor{currentstroke}{rgb}{0.000000,0.000000,0.000000}%
\pgfsetstrokecolor{currentstroke}%
\pgfsetstrokeopacity{0.700000}%
\pgfsetdash{}{0pt}%
\pgfpathmoveto{\pgfqpoint{8.592685in}{0.550000in}}%
\pgfpathlineto{\pgfqpoint{8.620206in}{0.550000in}}%
\pgfpathlineto{\pgfqpoint{8.620206in}{0.550153in}}%
\pgfpathlineto{\pgfqpoint{8.592685in}{0.550153in}}%
\pgfpathlineto{\pgfqpoint{8.592685in}{0.550000in}}%
\pgfpathclose%
\pgfusepath{fill}%
\end{pgfscope}%
\begin{pgfscope}%
\pgfpathrectangle{\pgfqpoint{1.250000in}{0.550000in}}{\pgfqpoint{7.750000in}{3.850000in}}%
\pgfusepath{clip}%
\pgfsetbuttcap%
\pgfsetmiterjoin%
\definecolor{currentfill}{rgb}{0.501961,0.501961,0.501961}%
\pgfsetfillcolor{currentfill}%
\pgfsetfillopacity{0.700000}%
\pgfsetlinewidth{0.000000pt}%
\definecolor{currentstroke}{rgb}{0.000000,0.000000,0.000000}%
\pgfsetstrokecolor{currentstroke}%
\pgfsetstrokeopacity{0.700000}%
\pgfsetdash{}{0pt}%
\pgfpathmoveto{\pgfqpoint{8.620206in}{0.550000in}}%
\pgfpathlineto{\pgfqpoint{8.647727in}{0.550000in}}%
\pgfpathlineto{\pgfqpoint{8.647727in}{0.550612in}}%
\pgfpathlineto{\pgfqpoint{8.620206in}{0.550612in}}%
\pgfpathlineto{\pgfqpoint{8.620206in}{0.550000in}}%
\pgfpathclose%
\pgfusepath{fill}%
\end{pgfscope}%
\begin{pgfscope}%
\pgfsetbuttcap%
\pgfsetroundjoin%
\definecolor{currentfill}{rgb}{0.000000,0.000000,0.000000}%
\pgfsetfillcolor{currentfill}%
\pgfsetlinewidth{0.803000pt}%
\definecolor{currentstroke}{rgb}{0.000000,0.000000,0.000000}%
\pgfsetstrokecolor{currentstroke}%
\pgfsetdash{}{0pt}%
\pgfsys@defobject{currentmarker}{\pgfqpoint{0.000000in}{-0.048611in}}{\pgfqpoint{0.000000in}{0.000000in}}{%
\pgfpathmoveto{\pgfqpoint{0.000000in}{0.000000in}}%
\pgfpathlineto{\pgfqpoint{0.000000in}{-0.048611in}}%
\pgfusepath{stroke,fill}%
}%
\begin{pgfscope}%
\pgfsys@transformshift{1.602273in}{0.550000in}%
\pgfsys@useobject{currentmarker}{}%
\end{pgfscope}%
\end{pgfscope}%
\begin{pgfscope}%
\definecolor{textcolor}{rgb}{0.000000,0.000000,0.000000}%
\pgfsetstrokecolor{textcolor}%
\pgfsetfillcolor{textcolor}%
\pgftext[x=1.602273in,y=0.452778in,,top]{\color{textcolor}{\sffamily\fontsize{10.000000}{12.000000}\selectfont\catcode`\^=\active\def^{\ifmmode\sp\else\^{}\fi}\catcode`\%=\active\def%{\%}0}}%
\end{pgfscope}%
\begin{pgfscope}%
\pgfsetbuttcap%
\pgfsetroundjoin%
\definecolor{currentfill}{rgb}{0.000000,0.000000,0.000000}%
\pgfsetfillcolor{currentfill}%
\pgfsetlinewidth{0.803000pt}%
\definecolor{currentstroke}{rgb}{0.000000,0.000000,0.000000}%
\pgfsetstrokecolor{currentstroke}%
\pgfsetdash{}{0pt}%
\pgfsys@defobject{currentmarker}{\pgfqpoint{0.000000in}{-0.048611in}}{\pgfqpoint{0.000000in}{0.000000in}}{%
\pgfpathmoveto{\pgfqpoint{0.000000in}{0.000000in}}%
\pgfpathlineto{\pgfqpoint{0.000000in}{-0.048611in}}%
\pgfusepath{stroke,fill}%
}%
\begin{pgfscope}%
\pgfsys@transformshift{2.978338in}{0.550000in}%
\pgfsys@useobject{currentmarker}{}%
\end{pgfscope}%
\end{pgfscope}%
\begin{pgfscope}%
\definecolor{textcolor}{rgb}{0.000000,0.000000,0.000000}%
\pgfsetstrokecolor{textcolor}%
\pgfsetfillcolor{textcolor}%
\pgftext[x=2.978338in,y=0.452778in,,top]{\color{textcolor}{\sffamily\fontsize{10.000000}{12.000000}\selectfont\catcode`\^=\active\def^{\ifmmode\sp\else\^{}\fi}\catcode`\%=\active\def%{\%}50}}%
\end{pgfscope}%
\begin{pgfscope}%
\pgfsetbuttcap%
\pgfsetroundjoin%
\definecolor{currentfill}{rgb}{0.000000,0.000000,0.000000}%
\pgfsetfillcolor{currentfill}%
\pgfsetlinewidth{0.803000pt}%
\definecolor{currentstroke}{rgb}{0.000000,0.000000,0.000000}%
\pgfsetstrokecolor{currentstroke}%
\pgfsetdash{}{0pt}%
\pgfsys@defobject{currentmarker}{\pgfqpoint{0.000000in}{-0.048611in}}{\pgfqpoint{0.000000in}{0.000000in}}{%
\pgfpathmoveto{\pgfqpoint{0.000000in}{0.000000in}}%
\pgfpathlineto{\pgfqpoint{0.000000in}{-0.048611in}}%
\pgfusepath{stroke,fill}%
}%
\begin{pgfscope}%
\pgfsys@transformshift{4.354403in}{0.550000in}%
\pgfsys@useobject{currentmarker}{}%
\end{pgfscope}%
\end{pgfscope}%
\begin{pgfscope}%
\definecolor{textcolor}{rgb}{0.000000,0.000000,0.000000}%
\pgfsetstrokecolor{textcolor}%
\pgfsetfillcolor{textcolor}%
\pgftext[x=4.354403in,y=0.452778in,,top]{\color{textcolor}{\sffamily\fontsize{10.000000}{12.000000}\selectfont\catcode`\^=\active\def^{\ifmmode\sp\else\^{}\fi}\catcode`\%=\active\def%{\%}100}}%
\end{pgfscope}%
\begin{pgfscope}%
\pgfsetbuttcap%
\pgfsetroundjoin%
\definecolor{currentfill}{rgb}{0.000000,0.000000,0.000000}%
\pgfsetfillcolor{currentfill}%
\pgfsetlinewidth{0.803000pt}%
\definecolor{currentstroke}{rgb}{0.000000,0.000000,0.000000}%
\pgfsetstrokecolor{currentstroke}%
\pgfsetdash{}{0pt}%
\pgfsys@defobject{currentmarker}{\pgfqpoint{0.000000in}{-0.048611in}}{\pgfqpoint{0.000000in}{0.000000in}}{%
\pgfpathmoveto{\pgfqpoint{0.000000in}{0.000000in}}%
\pgfpathlineto{\pgfqpoint{0.000000in}{-0.048611in}}%
\pgfusepath{stroke,fill}%
}%
\begin{pgfscope}%
\pgfsys@transformshift{5.730469in}{0.550000in}%
\pgfsys@useobject{currentmarker}{}%
\end{pgfscope}%
\end{pgfscope}%
\begin{pgfscope}%
\definecolor{textcolor}{rgb}{0.000000,0.000000,0.000000}%
\pgfsetstrokecolor{textcolor}%
\pgfsetfillcolor{textcolor}%
\pgftext[x=5.730469in,y=0.452778in,,top]{\color{textcolor}{\sffamily\fontsize{10.000000}{12.000000}\selectfont\catcode`\^=\active\def^{\ifmmode\sp\else\^{}\fi}\catcode`\%=\active\def%{\%}150}}%
\end{pgfscope}%
\begin{pgfscope}%
\pgfsetbuttcap%
\pgfsetroundjoin%
\definecolor{currentfill}{rgb}{0.000000,0.000000,0.000000}%
\pgfsetfillcolor{currentfill}%
\pgfsetlinewidth{0.803000pt}%
\definecolor{currentstroke}{rgb}{0.000000,0.000000,0.000000}%
\pgfsetstrokecolor{currentstroke}%
\pgfsetdash{}{0pt}%
\pgfsys@defobject{currentmarker}{\pgfqpoint{0.000000in}{-0.048611in}}{\pgfqpoint{0.000000in}{0.000000in}}{%
\pgfpathmoveto{\pgfqpoint{0.000000in}{0.000000in}}%
\pgfpathlineto{\pgfqpoint{0.000000in}{-0.048611in}}%
\pgfusepath{stroke,fill}%
}%
\begin{pgfscope}%
\pgfsys@transformshift{7.106534in}{0.550000in}%
\pgfsys@useobject{currentmarker}{}%
\end{pgfscope}%
\end{pgfscope}%
\begin{pgfscope}%
\definecolor{textcolor}{rgb}{0.000000,0.000000,0.000000}%
\pgfsetstrokecolor{textcolor}%
\pgfsetfillcolor{textcolor}%
\pgftext[x=7.106534in,y=0.452778in,,top]{\color{textcolor}{\sffamily\fontsize{10.000000}{12.000000}\selectfont\catcode`\^=\active\def^{\ifmmode\sp\else\^{}\fi}\catcode`\%=\active\def%{\%}200}}%
\end{pgfscope}%
\begin{pgfscope}%
\pgfsetbuttcap%
\pgfsetroundjoin%
\definecolor{currentfill}{rgb}{0.000000,0.000000,0.000000}%
\pgfsetfillcolor{currentfill}%
\pgfsetlinewidth{0.803000pt}%
\definecolor{currentstroke}{rgb}{0.000000,0.000000,0.000000}%
\pgfsetstrokecolor{currentstroke}%
\pgfsetdash{}{0pt}%
\pgfsys@defobject{currentmarker}{\pgfqpoint{0.000000in}{-0.048611in}}{\pgfqpoint{0.000000in}{0.000000in}}{%
\pgfpathmoveto{\pgfqpoint{0.000000in}{0.000000in}}%
\pgfpathlineto{\pgfqpoint{0.000000in}{-0.048611in}}%
\pgfusepath{stroke,fill}%
}%
\begin{pgfscope}%
\pgfsys@transformshift{8.482599in}{0.550000in}%
\pgfsys@useobject{currentmarker}{}%
\end{pgfscope}%
\end{pgfscope}%
\begin{pgfscope}%
\definecolor{textcolor}{rgb}{0.000000,0.000000,0.000000}%
\pgfsetstrokecolor{textcolor}%
\pgfsetfillcolor{textcolor}%
\pgftext[x=8.482599in,y=0.452778in,,top]{\color{textcolor}{\sffamily\fontsize{10.000000}{12.000000}\selectfont\catcode`\^=\active\def^{\ifmmode\sp\else\^{}\fi}\catcode`\%=\active\def%{\%}250}}%
\end{pgfscope}%
\begin{pgfscope}%
\definecolor{textcolor}{rgb}{0.000000,0.000000,0.000000}%
\pgfsetstrokecolor{textcolor}%
\pgfsetfillcolor{textcolor}%
\pgftext[x=5.125000in,y=0.262809in,,top]{\color{textcolor}{\sffamily\fontsize{10.000000}{12.000000}\selectfont\catcode`\^=\active\def^{\ifmmode\sp\else\^{}\fi}\catcode`\%=\active\def%{\%}Intensidad}}%
\end{pgfscope}%
\begin{pgfscope}%
\pgfsetbuttcap%
\pgfsetroundjoin%
\definecolor{currentfill}{rgb}{0.000000,0.000000,0.000000}%
\pgfsetfillcolor{currentfill}%
\pgfsetlinewidth{0.803000pt}%
\definecolor{currentstroke}{rgb}{0.000000,0.000000,0.000000}%
\pgfsetstrokecolor{currentstroke}%
\pgfsetdash{}{0pt}%
\pgfsys@defobject{currentmarker}{\pgfqpoint{-0.048611in}{0.000000in}}{\pgfqpoint{-0.000000in}{0.000000in}}{%
\pgfpathmoveto{\pgfqpoint{-0.000000in}{0.000000in}}%
\pgfpathlineto{\pgfqpoint{-0.048611in}{0.000000in}}%
\pgfusepath{stroke,fill}%
}%
\begin{pgfscope}%
\pgfsys@transformshift{1.250000in}{0.550000in}%
\pgfsys@useobject{currentmarker}{}%
\end{pgfscope}%
\end{pgfscope}%
\begin{pgfscope}%
\definecolor{textcolor}{rgb}{0.000000,0.000000,0.000000}%
\pgfsetstrokecolor{textcolor}%
\pgfsetfillcolor{textcolor}%
\pgftext[x=1.064412in, y=0.497238in, left, base]{\color{textcolor}{\sffamily\fontsize{10.000000}{12.000000}\selectfont\catcode`\^=\active\def^{\ifmmode\sp\else\^{}\fi}\catcode`\%=\active\def%{\%}0}}%
\end{pgfscope}%
\begin{pgfscope}%
\pgfsetbuttcap%
\pgfsetroundjoin%
\definecolor{currentfill}{rgb}{0.000000,0.000000,0.000000}%
\pgfsetfillcolor{currentfill}%
\pgfsetlinewidth{0.803000pt}%
\definecolor{currentstroke}{rgb}{0.000000,0.000000,0.000000}%
\pgfsetstrokecolor{currentstroke}%
\pgfsetdash{}{0pt}%
\pgfsys@defobject{currentmarker}{\pgfqpoint{-0.048611in}{0.000000in}}{\pgfqpoint{-0.000000in}{0.000000in}}{%
\pgfpathmoveto{\pgfqpoint{-0.000000in}{0.000000in}}%
\pgfpathlineto{\pgfqpoint{-0.048611in}{0.000000in}}%
\pgfusepath{stroke,fill}%
}%
\begin{pgfscope}%
\pgfsys@transformshift{1.250000in}{1.315548in}%
\pgfsys@useobject{currentmarker}{}%
\end{pgfscope}%
\end{pgfscope}%
\begin{pgfscope}%
\definecolor{textcolor}{rgb}{0.000000,0.000000,0.000000}%
\pgfsetstrokecolor{textcolor}%
\pgfsetfillcolor{textcolor}%
\pgftext[x=0.799316in, y=1.262786in, left, base]{\color{textcolor}{\sffamily\fontsize{10.000000}{12.000000}\selectfont\catcode`\^=\active\def^{\ifmmode\sp\else\^{}\fi}\catcode`\%=\active\def%{\%}5000}}%
\end{pgfscope}%
\begin{pgfscope}%
\pgfsetbuttcap%
\pgfsetroundjoin%
\definecolor{currentfill}{rgb}{0.000000,0.000000,0.000000}%
\pgfsetfillcolor{currentfill}%
\pgfsetlinewidth{0.803000pt}%
\definecolor{currentstroke}{rgb}{0.000000,0.000000,0.000000}%
\pgfsetstrokecolor{currentstroke}%
\pgfsetdash{}{0pt}%
\pgfsys@defobject{currentmarker}{\pgfqpoint{-0.048611in}{0.000000in}}{\pgfqpoint{-0.000000in}{0.000000in}}{%
\pgfpathmoveto{\pgfqpoint{-0.000000in}{0.000000in}}%
\pgfpathlineto{\pgfqpoint{-0.048611in}{0.000000in}}%
\pgfusepath{stroke,fill}%
}%
\begin{pgfscope}%
\pgfsys@transformshift{1.250000in}{2.081095in}%
\pgfsys@useobject{currentmarker}{}%
\end{pgfscope}%
\end{pgfscope}%
\begin{pgfscope}%
\definecolor{textcolor}{rgb}{0.000000,0.000000,0.000000}%
\pgfsetstrokecolor{textcolor}%
\pgfsetfillcolor{textcolor}%
\pgftext[x=0.710951in, y=2.028334in, left, base]{\color{textcolor}{\sffamily\fontsize{10.000000}{12.000000}\selectfont\catcode`\^=\active\def^{\ifmmode\sp\else\^{}\fi}\catcode`\%=\active\def%{\%}10000}}%
\end{pgfscope}%
\begin{pgfscope}%
\pgfsetbuttcap%
\pgfsetroundjoin%
\definecolor{currentfill}{rgb}{0.000000,0.000000,0.000000}%
\pgfsetfillcolor{currentfill}%
\pgfsetlinewidth{0.803000pt}%
\definecolor{currentstroke}{rgb}{0.000000,0.000000,0.000000}%
\pgfsetstrokecolor{currentstroke}%
\pgfsetdash{}{0pt}%
\pgfsys@defobject{currentmarker}{\pgfqpoint{-0.048611in}{0.000000in}}{\pgfqpoint{-0.000000in}{0.000000in}}{%
\pgfpathmoveto{\pgfqpoint{-0.000000in}{0.000000in}}%
\pgfpathlineto{\pgfqpoint{-0.048611in}{0.000000in}}%
\pgfusepath{stroke,fill}%
}%
\begin{pgfscope}%
\pgfsys@transformshift{1.250000in}{2.846643in}%
\pgfsys@useobject{currentmarker}{}%
\end{pgfscope}%
\end{pgfscope}%
\begin{pgfscope}%
\definecolor{textcolor}{rgb}{0.000000,0.000000,0.000000}%
\pgfsetstrokecolor{textcolor}%
\pgfsetfillcolor{textcolor}%
\pgftext[x=0.710951in, y=2.793881in, left, base]{\color{textcolor}{\sffamily\fontsize{10.000000}{12.000000}\selectfont\catcode`\^=\active\def^{\ifmmode\sp\else\^{}\fi}\catcode`\%=\active\def%{\%}15000}}%
\end{pgfscope}%
\begin{pgfscope}%
\pgfsetbuttcap%
\pgfsetroundjoin%
\definecolor{currentfill}{rgb}{0.000000,0.000000,0.000000}%
\pgfsetfillcolor{currentfill}%
\pgfsetlinewidth{0.803000pt}%
\definecolor{currentstroke}{rgb}{0.000000,0.000000,0.000000}%
\pgfsetstrokecolor{currentstroke}%
\pgfsetdash{}{0pt}%
\pgfsys@defobject{currentmarker}{\pgfqpoint{-0.048611in}{0.000000in}}{\pgfqpoint{-0.000000in}{0.000000in}}{%
\pgfpathmoveto{\pgfqpoint{-0.000000in}{0.000000in}}%
\pgfpathlineto{\pgfqpoint{-0.048611in}{0.000000in}}%
\pgfusepath{stroke,fill}%
}%
\begin{pgfscope}%
\pgfsys@transformshift{1.250000in}{3.612190in}%
\pgfsys@useobject{currentmarker}{}%
\end{pgfscope}%
\end{pgfscope}%
\begin{pgfscope}%
\definecolor{textcolor}{rgb}{0.000000,0.000000,0.000000}%
\pgfsetstrokecolor{textcolor}%
\pgfsetfillcolor{textcolor}%
\pgftext[x=0.710951in, y=3.559429in, left, base]{\color{textcolor}{\sffamily\fontsize{10.000000}{12.000000}\selectfont\catcode`\^=\active\def^{\ifmmode\sp\else\^{}\fi}\catcode`\%=\active\def%{\%}20000}}%
\end{pgfscope}%
\begin{pgfscope}%
\pgfsetbuttcap%
\pgfsetroundjoin%
\definecolor{currentfill}{rgb}{0.000000,0.000000,0.000000}%
\pgfsetfillcolor{currentfill}%
\pgfsetlinewidth{0.803000pt}%
\definecolor{currentstroke}{rgb}{0.000000,0.000000,0.000000}%
\pgfsetstrokecolor{currentstroke}%
\pgfsetdash{}{0pt}%
\pgfsys@defobject{currentmarker}{\pgfqpoint{-0.048611in}{0.000000in}}{\pgfqpoint{-0.000000in}{0.000000in}}{%
\pgfpathmoveto{\pgfqpoint{-0.000000in}{0.000000in}}%
\pgfpathlineto{\pgfqpoint{-0.048611in}{0.000000in}}%
\pgfusepath{stroke,fill}%
}%
\begin{pgfscope}%
\pgfsys@transformshift{1.250000in}{4.377738in}%
\pgfsys@useobject{currentmarker}{}%
\end{pgfscope}%
\end{pgfscope}%
\begin{pgfscope}%
\definecolor{textcolor}{rgb}{0.000000,0.000000,0.000000}%
\pgfsetstrokecolor{textcolor}%
\pgfsetfillcolor{textcolor}%
\pgftext[x=0.710951in, y=4.324976in, left, base]{\color{textcolor}{\sffamily\fontsize{10.000000}{12.000000}\selectfont\catcode`\^=\active\def^{\ifmmode\sp\else\^{}\fi}\catcode`\%=\active\def%{\%}25000}}%
\end{pgfscope}%
\begin{pgfscope}%
\definecolor{textcolor}{rgb}{0.000000,0.000000,0.000000}%
\pgfsetstrokecolor{textcolor}%
\pgfsetfillcolor{textcolor}%
\pgftext[x=0.655396in,y=2.475000in,,bottom,rotate=90.000000]{\color{textcolor}{\sffamily\fontsize{10.000000}{12.000000}\selectfont\catcode`\^=\active\def^{\ifmmode\sp\else\^{}\fi}\catcode`\%=\active\def%{\%}Frecuencia}}%
\end{pgfscope}%
\begin{pgfscope}%
\pgfsetrectcap%
\pgfsetmiterjoin%
\pgfsetlinewidth{0.803000pt}%
\definecolor{currentstroke}{rgb}{0.000000,0.000000,0.000000}%
\pgfsetstrokecolor{currentstroke}%
\pgfsetdash{}{0pt}%
\pgfpathmoveto{\pgfqpoint{1.250000in}{0.550000in}}%
\pgfpathlineto{\pgfqpoint{1.250000in}{4.400000in}}%
\pgfusepath{stroke}%
\end{pgfscope}%
\begin{pgfscope}%
\pgfsetrectcap%
\pgfsetmiterjoin%
\pgfsetlinewidth{0.803000pt}%
\definecolor{currentstroke}{rgb}{0.000000,0.000000,0.000000}%
\pgfsetstrokecolor{currentstroke}%
\pgfsetdash{}{0pt}%
\pgfpathmoveto{\pgfqpoint{9.000000in}{0.550000in}}%
\pgfpathlineto{\pgfqpoint{9.000000in}{4.400000in}}%
\pgfusepath{stroke}%
\end{pgfscope}%
\begin{pgfscope}%
\pgfsetrectcap%
\pgfsetmiterjoin%
\pgfsetlinewidth{0.803000pt}%
\definecolor{currentstroke}{rgb}{0.000000,0.000000,0.000000}%
\pgfsetstrokecolor{currentstroke}%
\pgfsetdash{}{0pt}%
\pgfpathmoveto{\pgfqpoint{1.250000in}{0.550000in}}%
\pgfpathlineto{\pgfqpoint{9.000000in}{0.550000in}}%
\pgfusepath{stroke}%
\end{pgfscope}%
\begin{pgfscope}%
\pgfsetrectcap%
\pgfsetmiterjoin%
\pgfsetlinewidth{0.803000pt}%
\definecolor{currentstroke}{rgb}{0.000000,0.000000,0.000000}%
\pgfsetstrokecolor{currentstroke}%
\pgfsetdash{}{0pt}%
\pgfpathmoveto{\pgfqpoint{1.250000in}{4.400000in}}%
\pgfpathlineto{\pgfqpoint{9.000000in}{4.400000in}}%
\pgfusepath{stroke}%
\end{pgfscope}%
\begin{pgfscope}%
\definecolor{textcolor}{rgb}{0.000000,0.000000,0.000000}%
\pgfsetstrokecolor{textcolor}%
\pgfsetfillcolor{textcolor}%
\pgftext[x=5.125000in,y=4.483333in,,base]{\color{textcolor}{\sffamily\fontsize{12.000000}{14.400000}\selectfont\catcode`\^=\active\def^{\ifmmode\sp\else\^{}\fi}\catcode`\%=\active\def%{\%}Histograma de CLAHE}}%
\end{pgfscope}%
\end{pgfpicture}%
\makeatother%
\endgroup%
}
		\caption{CLAHE}
		\label{fig:clahe_3}
	\end{subfigure}

	\vspace{5mm}

	\centering
	\begin{subfigure}[b]{0.45\textwidth}
		\resizebox{\linewidth}{!}{%% Creator: Matplotlib, PGF backend
%%
%% To include the figure in your LaTeX document, write
%%   \input{<filename>.pgf}
%%
%% Make sure the required packages are loaded in your preamble
%%   \usepackage{pgf}
%%
%% Also ensure that all the required font packages are loaded; for instance,
%% the lmodern package is sometimes necessary when using math font.
%%   \usepackage{lmodern}
%%
%% Figures using additional raster images can only be included by \input if
%% they are in the same directory as the main LaTeX file. For loading figures
%% from other directories you can use the `import` package
%%   \usepackage{import}
%%
%% and then include the figures with
%%   \import{<path to file>}{<filename>.pgf}
%%
%% Matplotlib used the following preamble
%%   \def\mathdefault#1{#1}
%%   \everymath=\expandafter{\the\everymath\displaystyle}
%%   \IfFileExists{scrextend.sty}{
%%     \usepackage[fontsize=10.000000pt]{scrextend}
%%   }{
%%     \renewcommand{\normalsize}{\fontsize{10.000000}{12.000000}\selectfont}
%%     \normalsize
%%   }
%%   
%%   \ifdefined\pdftexversion\else  % non-pdftex case.
%%     \usepackage{fontspec}
%%     \setmainfont{DejaVuSerif.ttf}[Path=\detokenize{/home/elias/.virtualenvs/src-pway/lib/python3.13/site-packages/matplotlib/mpl-data/fonts/ttf/}]
%%     \setsansfont{DejaVuSans.ttf}[Path=\detokenize{/home/elias/.virtualenvs/src-pway/lib/python3.13/site-packages/matplotlib/mpl-data/fonts/ttf/}]
%%     \setmonofont{DejaVuSansMono.ttf}[Path=\detokenize{/home/elias/.virtualenvs/src-pway/lib/python3.13/site-packages/matplotlib/mpl-data/fonts/ttf/}]
%%   \fi
%%   \makeatletter\@ifpackageloaded{underscore}{}{\usepackage[strings]{underscore}}\makeatother
%%
\begingroup%
\makeatletter%
\begin{pgfpicture}%
\pgfpathrectangle{\pgfpointorigin}{\pgfqpoint{10.000000in}{5.000000in}}%
\pgfusepath{use as bounding box, clip}%
\begin{pgfscope}%
\pgfsetbuttcap%
\pgfsetmiterjoin%
\definecolor{currentfill}{rgb}{1.000000,1.000000,1.000000}%
\pgfsetfillcolor{currentfill}%
\pgfsetlinewidth{0.000000pt}%
\definecolor{currentstroke}{rgb}{1.000000,1.000000,1.000000}%
\pgfsetstrokecolor{currentstroke}%
\pgfsetdash{}{0pt}%
\pgfpathmoveto{\pgfqpoint{0.000000in}{0.000000in}}%
\pgfpathlineto{\pgfqpoint{10.000000in}{0.000000in}}%
\pgfpathlineto{\pgfqpoint{10.000000in}{5.000000in}}%
\pgfpathlineto{\pgfqpoint{0.000000in}{5.000000in}}%
\pgfpathlineto{\pgfqpoint{0.000000in}{0.000000in}}%
\pgfpathclose%
\pgfusepath{fill}%
\end{pgfscope}%
\begin{pgfscope}%
\pgfsetbuttcap%
\pgfsetmiterjoin%
\definecolor{currentfill}{rgb}{1.000000,1.000000,1.000000}%
\pgfsetfillcolor{currentfill}%
\pgfsetlinewidth{0.000000pt}%
\definecolor{currentstroke}{rgb}{0.000000,0.000000,0.000000}%
\pgfsetstrokecolor{currentstroke}%
\pgfsetstrokeopacity{0.000000}%
\pgfsetdash{}{0pt}%
\pgfpathmoveto{\pgfqpoint{1.250000in}{0.550000in}}%
\pgfpathlineto{\pgfqpoint{9.000000in}{0.550000in}}%
\pgfpathlineto{\pgfqpoint{9.000000in}{4.400000in}}%
\pgfpathlineto{\pgfqpoint{1.250000in}{4.400000in}}%
\pgfpathlineto{\pgfqpoint{1.250000in}{0.550000in}}%
\pgfpathclose%
\pgfusepath{fill}%
\end{pgfscope}%
\begin{pgfscope}%
\pgfpathrectangle{\pgfqpoint{1.250000in}{0.550000in}}{\pgfqpoint{7.750000in}{3.850000in}}%
\pgfusepath{clip}%
\pgfsetbuttcap%
\pgfsetmiterjoin%
\definecolor{currentfill}{rgb}{0.501961,0.501961,0.501961}%
\pgfsetfillcolor{currentfill}%
\pgfsetfillopacity{0.700000}%
\pgfsetlinewidth{0.000000pt}%
\definecolor{currentstroke}{rgb}{0.000000,0.000000,0.000000}%
\pgfsetstrokecolor{currentstroke}%
\pgfsetstrokeopacity{0.700000}%
\pgfsetdash{}{0pt}%
\pgfpathmoveto{\pgfqpoint{1.602273in}{0.550000in}}%
\pgfpathlineto{\pgfqpoint{1.629794in}{0.550000in}}%
\pgfpathlineto{\pgfqpoint{1.629794in}{0.550000in}}%
\pgfpathlineto{\pgfqpoint{1.602273in}{0.550000in}}%
\pgfpathlineto{\pgfqpoint{1.602273in}{0.550000in}}%
\pgfpathclose%
\pgfusepath{fill}%
\end{pgfscope}%
\begin{pgfscope}%
\pgfpathrectangle{\pgfqpoint{1.250000in}{0.550000in}}{\pgfqpoint{7.750000in}{3.850000in}}%
\pgfusepath{clip}%
\pgfsetbuttcap%
\pgfsetmiterjoin%
\definecolor{currentfill}{rgb}{0.501961,0.501961,0.501961}%
\pgfsetfillcolor{currentfill}%
\pgfsetfillopacity{0.700000}%
\pgfsetlinewidth{0.000000pt}%
\definecolor{currentstroke}{rgb}{0.000000,0.000000,0.000000}%
\pgfsetstrokecolor{currentstroke}%
\pgfsetstrokeopacity{0.700000}%
\pgfsetdash{}{0pt}%
\pgfpathmoveto{\pgfqpoint{1.629794in}{0.550000in}}%
\pgfpathlineto{\pgfqpoint{1.657315in}{0.550000in}}%
\pgfpathlineto{\pgfqpoint{1.657315in}{1.579420in}}%
\pgfpathlineto{\pgfqpoint{1.629794in}{1.579420in}}%
\pgfpathlineto{\pgfqpoint{1.629794in}{0.550000in}}%
\pgfpathclose%
\pgfusepath{fill}%
\end{pgfscope}%
\begin{pgfscope}%
\pgfpathrectangle{\pgfqpoint{1.250000in}{0.550000in}}{\pgfqpoint{7.750000in}{3.850000in}}%
\pgfusepath{clip}%
\pgfsetbuttcap%
\pgfsetmiterjoin%
\definecolor{currentfill}{rgb}{0.501961,0.501961,0.501961}%
\pgfsetfillcolor{currentfill}%
\pgfsetfillopacity{0.700000}%
\pgfsetlinewidth{0.000000pt}%
\definecolor{currentstroke}{rgb}{0.000000,0.000000,0.000000}%
\pgfsetstrokecolor{currentstroke}%
\pgfsetstrokeopacity{0.700000}%
\pgfsetdash{}{0pt}%
\pgfpathmoveto{\pgfqpoint{1.657315in}{0.550000in}}%
\pgfpathlineto{\pgfqpoint{1.684837in}{0.550000in}}%
\pgfpathlineto{\pgfqpoint{1.684837in}{4.216667in}}%
\pgfpathlineto{\pgfqpoint{1.657315in}{4.216667in}}%
\pgfpathlineto{\pgfqpoint{1.657315in}{0.550000in}}%
\pgfpathclose%
\pgfusepath{fill}%
\end{pgfscope}%
\begin{pgfscope}%
\pgfpathrectangle{\pgfqpoint{1.250000in}{0.550000in}}{\pgfqpoint{7.750000in}{3.850000in}}%
\pgfusepath{clip}%
\pgfsetbuttcap%
\pgfsetmiterjoin%
\definecolor{currentfill}{rgb}{0.501961,0.501961,0.501961}%
\pgfsetfillcolor{currentfill}%
\pgfsetfillopacity{0.700000}%
\pgfsetlinewidth{0.000000pt}%
\definecolor{currentstroke}{rgb}{0.000000,0.000000,0.000000}%
\pgfsetstrokecolor{currentstroke}%
\pgfsetstrokeopacity{0.700000}%
\pgfsetdash{}{0pt}%
\pgfpathmoveto{\pgfqpoint{1.684837in}{0.550000in}}%
\pgfpathlineto{\pgfqpoint{1.712358in}{0.550000in}}%
\pgfpathlineto{\pgfqpoint{1.712358in}{2.976377in}}%
\pgfpathlineto{\pgfqpoint{1.684837in}{2.976377in}}%
\pgfpathlineto{\pgfqpoint{1.684837in}{0.550000in}}%
\pgfpathclose%
\pgfusepath{fill}%
\end{pgfscope}%
\begin{pgfscope}%
\pgfpathrectangle{\pgfqpoint{1.250000in}{0.550000in}}{\pgfqpoint{7.750000in}{3.850000in}}%
\pgfusepath{clip}%
\pgfsetbuttcap%
\pgfsetmiterjoin%
\definecolor{currentfill}{rgb}{0.501961,0.501961,0.501961}%
\pgfsetfillcolor{currentfill}%
\pgfsetfillopacity{0.700000}%
\pgfsetlinewidth{0.000000pt}%
\definecolor{currentstroke}{rgb}{0.000000,0.000000,0.000000}%
\pgfsetstrokecolor{currentstroke}%
\pgfsetstrokeopacity{0.700000}%
\pgfsetdash{}{0pt}%
\pgfpathmoveto{\pgfqpoint{1.712358in}{0.550000in}}%
\pgfpathlineto{\pgfqpoint{1.739879in}{0.550000in}}%
\pgfpathlineto{\pgfqpoint{1.739879in}{1.805323in}}%
\pgfpathlineto{\pgfqpoint{1.712358in}{1.805323in}}%
\pgfpathlineto{\pgfqpoint{1.712358in}{0.550000in}}%
\pgfpathclose%
\pgfusepath{fill}%
\end{pgfscope}%
\begin{pgfscope}%
\pgfpathrectangle{\pgfqpoint{1.250000in}{0.550000in}}{\pgfqpoint{7.750000in}{3.850000in}}%
\pgfusepath{clip}%
\pgfsetbuttcap%
\pgfsetmiterjoin%
\definecolor{currentfill}{rgb}{0.501961,0.501961,0.501961}%
\pgfsetfillcolor{currentfill}%
\pgfsetfillopacity{0.700000}%
\pgfsetlinewidth{0.000000pt}%
\definecolor{currentstroke}{rgb}{0.000000,0.000000,0.000000}%
\pgfsetstrokecolor{currentstroke}%
\pgfsetstrokeopacity{0.700000}%
\pgfsetdash{}{0pt}%
\pgfpathmoveto{\pgfqpoint{1.739879in}{0.550000in}}%
\pgfpathlineto{\pgfqpoint{1.767401in}{0.550000in}}%
\pgfpathlineto{\pgfqpoint{1.767401in}{1.526011in}}%
\pgfpathlineto{\pgfqpoint{1.739879in}{1.526011in}}%
\pgfpathlineto{\pgfqpoint{1.739879in}{0.550000in}}%
\pgfpathclose%
\pgfusepath{fill}%
\end{pgfscope}%
\begin{pgfscope}%
\pgfpathrectangle{\pgfqpoint{1.250000in}{0.550000in}}{\pgfqpoint{7.750000in}{3.850000in}}%
\pgfusepath{clip}%
\pgfsetbuttcap%
\pgfsetmiterjoin%
\definecolor{currentfill}{rgb}{0.501961,0.501961,0.501961}%
\pgfsetfillcolor{currentfill}%
\pgfsetfillopacity{0.700000}%
\pgfsetlinewidth{0.000000pt}%
\definecolor{currentstroke}{rgb}{0.000000,0.000000,0.000000}%
\pgfsetstrokecolor{currentstroke}%
\pgfsetstrokeopacity{0.700000}%
\pgfsetdash{}{0pt}%
\pgfpathmoveto{\pgfqpoint{1.767401in}{0.550000in}}%
\pgfpathlineto{\pgfqpoint{1.794922in}{0.550000in}}%
\pgfpathlineto{\pgfqpoint{1.794922in}{1.324637in}}%
\pgfpathlineto{\pgfqpoint{1.767401in}{1.324637in}}%
\pgfpathlineto{\pgfqpoint{1.767401in}{0.550000in}}%
\pgfpathclose%
\pgfusepath{fill}%
\end{pgfscope}%
\begin{pgfscope}%
\pgfpathrectangle{\pgfqpoint{1.250000in}{0.550000in}}{\pgfqpoint{7.750000in}{3.850000in}}%
\pgfusepath{clip}%
\pgfsetbuttcap%
\pgfsetmiterjoin%
\definecolor{currentfill}{rgb}{0.501961,0.501961,0.501961}%
\pgfsetfillcolor{currentfill}%
\pgfsetfillopacity{0.700000}%
\pgfsetlinewidth{0.000000pt}%
\definecolor{currentstroke}{rgb}{0.000000,0.000000,0.000000}%
\pgfsetstrokecolor{currentstroke}%
\pgfsetstrokeopacity{0.700000}%
\pgfsetdash{}{0pt}%
\pgfpathmoveto{\pgfqpoint{1.794922in}{0.550000in}}%
\pgfpathlineto{\pgfqpoint{1.822443in}{0.550000in}}%
\pgfpathlineto{\pgfqpoint{1.822443in}{1.264106in}}%
\pgfpathlineto{\pgfqpoint{1.794922in}{1.264106in}}%
\pgfpathlineto{\pgfqpoint{1.794922in}{0.550000in}}%
\pgfpathclose%
\pgfusepath{fill}%
\end{pgfscope}%
\begin{pgfscope}%
\pgfpathrectangle{\pgfqpoint{1.250000in}{0.550000in}}{\pgfqpoint{7.750000in}{3.850000in}}%
\pgfusepath{clip}%
\pgfsetbuttcap%
\pgfsetmiterjoin%
\definecolor{currentfill}{rgb}{0.501961,0.501961,0.501961}%
\pgfsetfillcolor{currentfill}%
\pgfsetfillopacity{0.700000}%
\pgfsetlinewidth{0.000000pt}%
\definecolor{currentstroke}{rgb}{0.000000,0.000000,0.000000}%
\pgfsetstrokecolor{currentstroke}%
\pgfsetstrokeopacity{0.700000}%
\pgfsetdash{}{0pt}%
\pgfpathmoveto{\pgfqpoint{1.822443in}{0.550000in}}%
\pgfpathlineto{\pgfqpoint{1.849964in}{0.550000in}}%
\pgfpathlineto{\pgfqpoint{1.849964in}{1.248281in}}%
\pgfpathlineto{\pgfqpoint{1.822443in}{1.248281in}}%
\pgfpathlineto{\pgfqpoint{1.822443in}{0.550000in}}%
\pgfpathclose%
\pgfusepath{fill}%
\end{pgfscope}%
\begin{pgfscope}%
\pgfpathrectangle{\pgfqpoint{1.250000in}{0.550000in}}{\pgfqpoint{7.750000in}{3.850000in}}%
\pgfusepath{clip}%
\pgfsetbuttcap%
\pgfsetmiterjoin%
\definecolor{currentfill}{rgb}{0.501961,0.501961,0.501961}%
\pgfsetfillcolor{currentfill}%
\pgfsetfillopacity{0.700000}%
\pgfsetlinewidth{0.000000pt}%
\definecolor{currentstroke}{rgb}{0.000000,0.000000,0.000000}%
\pgfsetstrokecolor{currentstroke}%
\pgfsetstrokeopacity{0.700000}%
\pgfsetdash{}{0pt}%
\pgfpathmoveto{\pgfqpoint{1.849964in}{0.550000in}}%
\pgfpathlineto{\pgfqpoint{1.877486in}{0.550000in}}%
\pgfpathlineto{\pgfqpoint{1.877486in}{1.290217in}}%
\pgfpathlineto{\pgfqpoint{1.849964in}{1.290217in}}%
\pgfpathlineto{\pgfqpoint{1.849964in}{0.550000in}}%
\pgfpathclose%
\pgfusepath{fill}%
\end{pgfscope}%
\begin{pgfscope}%
\pgfpathrectangle{\pgfqpoint{1.250000in}{0.550000in}}{\pgfqpoint{7.750000in}{3.850000in}}%
\pgfusepath{clip}%
\pgfsetbuttcap%
\pgfsetmiterjoin%
\definecolor{currentfill}{rgb}{0.501961,0.501961,0.501961}%
\pgfsetfillcolor{currentfill}%
\pgfsetfillopacity{0.700000}%
\pgfsetlinewidth{0.000000pt}%
\definecolor{currentstroke}{rgb}{0.000000,0.000000,0.000000}%
\pgfsetstrokecolor{currentstroke}%
\pgfsetstrokeopacity{0.700000}%
\pgfsetdash{}{0pt}%
\pgfpathmoveto{\pgfqpoint{1.877486in}{0.550000in}}%
\pgfpathlineto{\pgfqpoint{1.905007in}{0.550000in}}%
\pgfpathlineto{\pgfqpoint{1.905007in}{0.550000in}}%
\pgfpathlineto{\pgfqpoint{1.877486in}{0.550000in}}%
\pgfpathlineto{\pgfqpoint{1.877486in}{0.550000in}}%
\pgfpathclose%
\pgfusepath{fill}%
\end{pgfscope}%
\begin{pgfscope}%
\pgfpathrectangle{\pgfqpoint{1.250000in}{0.550000in}}{\pgfqpoint{7.750000in}{3.850000in}}%
\pgfusepath{clip}%
\pgfsetbuttcap%
\pgfsetmiterjoin%
\definecolor{currentfill}{rgb}{0.501961,0.501961,0.501961}%
\pgfsetfillcolor{currentfill}%
\pgfsetfillopacity{0.700000}%
\pgfsetlinewidth{0.000000pt}%
\definecolor{currentstroke}{rgb}{0.000000,0.000000,0.000000}%
\pgfsetstrokecolor{currentstroke}%
\pgfsetstrokeopacity{0.700000}%
\pgfsetdash{}{0pt}%
\pgfpathmoveto{\pgfqpoint{1.905007in}{0.550000in}}%
\pgfpathlineto{\pgfqpoint{1.932528in}{0.550000in}}%
\pgfpathlineto{\pgfqpoint{1.932528in}{1.256193in}}%
\pgfpathlineto{\pgfqpoint{1.905007in}{1.256193in}}%
\pgfpathlineto{\pgfqpoint{1.905007in}{0.550000in}}%
\pgfpathclose%
\pgfusepath{fill}%
\end{pgfscope}%
\begin{pgfscope}%
\pgfpathrectangle{\pgfqpoint{1.250000in}{0.550000in}}{\pgfqpoint{7.750000in}{3.850000in}}%
\pgfusepath{clip}%
\pgfsetbuttcap%
\pgfsetmiterjoin%
\definecolor{currentfill}{rgb}{0.501961,0.501961,0.501961}%
\pgfsetfillcolor{currentfill}%
\pgfsetfillopacity{0.700000}%
\pgfsetlinewidth{0.000000pt}%
\definecolor{currentstroke}{rgb}{0.000000,0.000000,0.000000}%
\pgfsetstrokecolor{currentstroke}%
\pgfsetstrokeopacity{0.700000}%
\pgfsetdash{}{0pt}%
\pgfpathmoveto{\pgfqpoint{1.932528in}{0.550000in}}%
\pgfpathlineto{\pgfqpoint{1.960050in}{0.550000in}}%
\pgfpathlineto{\pgfqpoint{1.960050in}{1.258963in}}%
\pgfpathlineto{\pgfqpoint{1.932528in}{1.258963in}}%
\pgfpathlineto{\pgfqpoint{1.932528in}{0.550000in}}%
\pgfpathclose%
\pgfusepath{fill}%
\end{pgfscope}%
\begin{pgfscope}%
\pgfpathrectangle{\pgfqpoint{1.250000in}{0.550000in}}{\pgfqpoint{7.750000in}{3.850000in}}%
\pgfusepath{clip}%
\pgfsetbuttcap%
\pgfsetmiterjoin%
\definecolor{currentfill}{rgb}{0.501961,0.501961,0.501961}%
\pgfsetfillcolor{currentfill}%
\pgfsetfillopacity{0.700000}%
\pgfsetlinewidth{0.000000pt}%
\definecolor{currentstroke}{rgb}{0.000000,0.000000,0.000000}%
\pgfsetstrokecolor{currentstroke}%
\pgfsetstrokeopacity{0.700000}%
\pgfsetdash{}{0pt}%
\pgfpathmoveto{\pgfqpoint{1.960050in}{0.550000in}}%
\pgfpathlineto{\pgfqpoint{1.987571in}{0.550000in}}%
\pgfpathlineto{\pgfqpoint{1.987571in}{1.819961in}}%
\pgfpathlineto{\pgfqpoint{1.960050in}{1.819961in}}%
\pgfpathlineto{\pgfqpoint{1.960050in}{0.550000in}}%
\pgfpathclose%
\pgfusepath{fill}%
\end{pgfscope}%
\begin{pgfscope}%
\pgfpathrectangle{\pgfqpoint{1.250000in}{0.550000in}}{\pgfqpoint{7.750000in}{3.850000in}}%
\pgfusepath{clip}%
\pgfsetbuttcap%
\pgfsetmiterjoin%
\definecolor{currentfill}{rgb}{0.501961,0.501961,0.501961}%
\pgfsetfillcolor{currentfill}%
\pgfsetfillopacity{0.700000}%
\pgfsetlinewidth{0.000000pt}%
\definecolor{currentstroke}{rgb}{0.000000,0.000000,0.000000}%
\pgfsetstrokecolor{currentstroke}%
\pgfsetstrokeopacity{0.700000}%
\pgfsetdash{}{0pt}%
\pgfpathmoveto{\pgfqpoint{1.987571in}{0.550000in}}%
\pgfpathlineto{\pgfqpoint{2.015092in}{0.550000in}}%
\pgfpathlineto{\pgfqpoint{2.015092in}{1.143044in}}%
\pgfpathlineto{\pgfqpoint{1.987571in}{1.143044in}}%
\pgfpathlineto{\pgfqpoint{1.987571in}{0.550000in}}%
\pgfpathclose%
\pgfusepath{fill}%
\end{pgfscope}%
\begin{pgfscope}%
\pgfpathrectangle{\pgfqpoint{1.250000in}{0.550000in}}{\pgfqpoint{7.750000in}{3.850000in}}%
\pgfusepath{clip}%
\pgfsetbuttcap%
\pgfsetmiterjoin%
\definecolor{currentfill}{rgb}{0.501961,0.501961,0.501961}%
\pgfsetfillcolor{currentfill}%
\pgfsetfillopacity{0.700000}%
\pgfsetlinewidth{0.000000pt}%
\definecolor{currentstroke}{rgb}{0.000000,0.000000,0.000000}%
\pgfsetstrokecolor{currentstroke}%
\pgfsetstrokeopacity{0.700000}%
\pgfsetdash{}{0pt}%
\pgfpathmoveto{\pgfqpoint{2.015092in}{0.550000in}}%
\pgfpathlineto{\pgfqpoint{2.042614in}{0.550000in}}%
\pgfpathlineto{\pgfqpoint{2.042614in}{1.129593in}}%
\pgfpathlineto{\pgfqpoint{2.015092in}{1.129593in}}%
\pgfpathlineto{\pgfqpoint{2.015092in}{0.550000in}}%
\pgfpathclose%
\pgfusepath{fill}%
\end{pgfscope}%
\begin{pgfscope}%
\pgfpathrectangle{\pgfqpoint{1.250000in}{0.550000in}}{\pgfqpoint{7.750000in}{3.850000in}}%
\pgfusepath{clip}%
\pgfsetbuttcap%
\pgfsetmiterjoin%
\definecolor{currentfill}{rgb}{0.501961,0.501961,0.501961}%
\pgfsetfillcolor{currentfill}%
\pgfsetfillopacity{0.700000}%
\pgfsetlinewidth{0.000000pt}%
\definecolor{currentstroke}{rgb}{0.000000,0.000000,0.000000}%
\pgfsetstrokecolor{currentstroke}%
\pgfsetstrokeopacity{0.700000}%
\pgfsetdash{}{0pt}%
\pgfpathmoveto{\pgfqpoint{2.042614in}{0.550000in}}%
\pgfpathlineto{\pgfqpoint{2.070135in}{0.550000in}}%
\pgfpathlineto{\pgfqpoint{2.070135in}{1.086074in}}%
\pgfpathlineto{\pgfqpoint{2.042614in}{1.086074in}}%
\pgfpathlineto{\pgfqpoint{2.042614in}{0.550000in}}%
\pgfpathclose%
\pgfusepath{fill}%
\end{pgfscope}%
\begin{pgfscope}%
\pgfpathrectangle{\pgfqpoint{1.250000in}{0.550000in}}{\pgfqpoint{7.750000in}{3.850000in}}%
\pgfusepath{clip}%
\pgfsetbuttcap%
\pgfsetmiterjoin%
\definecolor{currentfill}{rgb}{0.501961,0.501961,0.501961}%
\pgfsetfillcolor{currentfill}%
\pgfsetfillopacity{0.700000}%
\pgfsetlinewidth{0.000000pt}%
\definecolor{currentstroke}{rgb}{0.000000,0.000000,0.000000}%
\pgfsetstrokecolor{currentstroke}%
\pgfsetstrokeopacity{0.700000}%
\pgfsetdash{}{0pt}%
\pgfpathmoveto{\pgfqpoint{2.070135in}{0.550000in}}%
\pgfpathlineto{\pgfqpoint{2.097656in}{0.550000in}}%
\pgfpathlineto{\pgfqpoint{2.097656in}{1.580211in}}%
\pgfpathlineto{\pgfqpoint{2.070135in}{1.580211in}}%
\pgfpathlineto{\pgfqpoint{2.070135in}{0.550000in}}%
\pgfpathclose%
\pgfusepath{fill}%
\end{pgfscope}%
\begin{pgfscope}%
\pgfpathrectangle{\pgfqpoint{1.250000in}{0.550000in}}{\pgfqpoint{7.750000in}{3.850000in}}%
\pgfusepath{clip}%
\pgfsetbuttcap%
\pgfsetmiterjoin%
\definecolor{currentfill}{rgb}{0.501961,0.501961,0.501961}%
\pgfsetfillcolor{currentfill}%
\pgfsetfillopacity{0.700000}%
\pgfsetlinewidth{0.000000pt}%
\definecolor{currentstroke}{rgb}{0.000000,0.000000,0.000000}%
\pgfsetstrokecolor{currentstroke}%
\pgfsetstrokeopacity{0.700000}%
\pgfsetdash{}{0pt}%
\pgfpathmoveto{\pgfqpoint{2.097656in}{0.550000in}}%
\pgfpathlineto{\pgfqpoint{2.125178in}{0.550000in}}%
\pgfpathlineto{\pgfqpoint{2.125178in}{1.048885in}}%
\pgfpathlineto{\pgfqpoint{2.097656in}{1.048885in}}%
\pgfpathlineto{\pgfqpoint{2.097656in}{0.550000in}}%
\pgfpathclose%
\pgfusepath{fill}%
\end{pgfscope}%
\begin{pgfscope}%
\pgfpathrectangle{\pgfqpoint{1.250000in}{0.550000in}}{\pgfqpoint{7.750000in}{3.850000in}}%
\pgfusepath{clip}%
\pgfsetbuttcap%
\pgfsetmiterjoin%
\definecolor{currentfill}{rgb}{0.501961,0.501961,0.501961}%
\pgfsetfillcolor{currentfill}%
\pgfsetfillopacity{0.700000}%
\pgfsetlinewidth{0.000000pt}%
\definecolor{currentstroke}{rgb}{0.000000,0.000000,0.000000}%
\pgfsetstrokecolor{currentstroke}%
\pgfsetstrokeopacity{0.700000}%
\pgfsetdash{}{0pt}%
\pgfpathmoveto{\pgfqpoint{2.125178in}{0.550000in}}%
\pgfpathlineto{\pgfqpoint{2.152699in}{0.550000in}}%
\pgfpathlineto{\pgfqpoint{2.152699in}{1.068666in}}%
\pgfpathlineto{\pgfqpoint{2.125178in}{1.068666in}}%
\pgfpathlineto{\pgfqpoint{2.125178in}{0.550000in}}%
\pgfpathclose%
\pgfusepath{fill}%
\end{pgfscope}%
\begin{pgfscope}%
\pgfpathrectangle{\pgfqpoint{1.250000in}{0.550000in}}{\pgfqpoint{7.750000in}{3.850000in}}%
\pgfusepath{clip}%
\pgfsetbuttcap%
\pgfsetmiterjoin%
\definecolor{currentfill}{rgb}{0.501961,0.501961,0.501961}%
\pgfsetfillcolor{currentfill}%
\pgfsetfillopacity{0.700000}%
\pgfsetlinewidth{0.000000pt}%
\definecolor{currentstroke}{rgb}{0.000000,0.000000,0.000000}%
\pgfsetstrokecolor{currentstroke}%
\pgfsetstrokeopacity{0.700000}%
\pgfsetdash{}{0pt}%
\pgfpathmoveto{\pgfqpoint{2.152699in}{0.550000in}}%
\pgfpathlineto{\pgfqpoint{2.180220in}{0.550000in}}%
\pgfpathlineto{\pgfqpoint{2.180220in}{1.102690in}}%
\pgfpathlineto{\pgfqpoint{2.152699in}{1.102690in}}%
\pgfpathlineto{\pgfqpoint{2.152699in}{0.550000in}}%
\pgfpathclose%
\pgfusepath{fill}%
\end{pgfscope}%
\begin{pgfscope}%
\pgfpathrectangle{\pgfqpoint{1.250000in}{0.550000in}}{\pgfqpoint{7.750000in}{3.850000in}}%
\pgfusepath{clip}%
\pgfsetbuttcap%
\pgfsetmiterjoin%
\definecolor{currentfill}{rgb}{0.501961,0.501961,0.501961}%
\pgfsetfillcolor{currentfill}%
\pgfsetfillopacity{0.700000}%
\pgfsetlinewidth{0.000000pt}%
\definecolor{currentstroke}{rgb}{0.000000,0.000000,0.000000}%
\pgfsetstrokecolor{currentstroke}%
\pgfsetstrokeopacity{0.700000}%
\pgfsetdash{}{0pt}%
\pgfpathmoveto{\pgfqpoint{2.180220in}{0.550000in}}%
\pgfpathlineto{\pgfqpoint{2.207741in}{0.550000in}}%
\pgfpathlineto{\pgfqpoint{2.207741in}{1.743210in}}%
\pgfpathlineto{\pgfqpoint{2.180220in}{1.743210in}}%
\pgfpathlineto{\pgfqpoint{2.180220in}{0.550000in}}%
\pgfpathclose%
\pgfusepath{fill}%
\end{pgfscope}%
\begin{pgfscope}%
\pgfpathrectangle{\pgfqpoint{1.250000in}{0.550000in}}{\pgfqpoint{7.750000in}{3.850000in}}%
\pgfusepath{clip}%
\pgfsetbuttcap%
\pgfsetmiterjoin%
\definecolor{currentfill}{rgb}{0.501961,0.501961,0.501961}%
\pgfsetfillcolor{currentfill}%
\pgfsetfillopacity{0.700000}%
\pgfsetlinewidth{0.000000pt}%
\definecolor{currentstroke}{rgb}{0.000000,0.000000,0.000000}%
\pgfsetstrokecolor{currentstroke}%
\pgfsetstrokeopacity{0.700000}%
\pgfsetdash{}{0pt}%
\pgfpathmoveto{\pgfqpoint{2.207741in}{0.550000in}}%
\pgfpathlineto{\pgfqpoint{2.235263in}{0.550000in}}%
\pgfpathlineto{\pgfqpoint{2.235263in}{1.135132in}}%
\pgfpathlineto{\pgfqpoint{2.207741in}{1.135132in}}%
\pgfpathlineto{\pgfqpoint{2.207741in}{0.550000in}}%
\pgfpathclose%
\pgfusepath{fill}%
\end{pgfscope}%
\begin{pgfscope}%
\pgfpathrectangle{\pgfqpoint{1.250000in}{0.550000in}}{\pgfqpoint{7.750000in}{3.850000in}}%
\pgfusepath{clip}%
\pgfsetbuttcap%
\pgfsetmiterjoin%
\definecolor{currentfill}{rgb}{0.501961,0.501961,0.501961}%
\pgfsetfillcolor{currentfill}%
\pgfsetfillopacity{0.700000}%
\pgfsetlinewidth{0.000000pt}%
\definecolor{currentstroke}{rgb}{0.000000,0.000000,0.000000}%
\pgfsetstrokecolor{currentstroke}%
\pgfsetstrokeopacity{0.700000}%
\pgfsetdash{}{0pt}%
\pgfpathmoveto{\pgfqpoint{2.235263in}{0.550000in}}%
\pgfpathlineto{\pgfqpoint{2.262784in}{0.550000in}}%
\pgfpathlineto{\pgfqpoint{2.262784in}{1.118120in}}%
\pgfpathlineto{\pgfqpoint{2.235263in}{1.118120in}}%
\pgfpathlineto{\pgfqpoint{2.235263in}{0.550000in}}%
\pgfpathclose%
\pgfusepath{fill}%
\end{pgfscope}%
\begin{pgfscope}%
\pgfpathrectangle{\pgfqpoint{1.250000in}{0.550000in}}{\pgfqpoint{7.750000in}{3.850000in}}%
\pgfusepath{clip}%
\pgfsetbuttcap%
\pgfsetmiterjoin%
\definecolor{currentfill}{rgb}{0.501961,0.501961,0.501961}%
\pgfsetfillcolor{currentfill}%
\pgfsetfillopacity{0.700000}%
\pgfsetlinewidth{0.000000pt}%
\definecolor{currentstroke}{rgb}{0.000000,0.000000,0.000000}%
\pgfsetstrokecolor{currentstroke}%
\pgfsetstrokeopacity{0.700000}%
\pgfsetdash{}{0pt}%
\pgfpathmoveto{\pgfqpoint{2.262784in}{0.550000in}}%
\pgfpathlineto{\pgfqpoint{2.290305in}{0.550000in}}%
\pgfpathlineto{\pgfqpoint{2.290305in}{1.145418in}}%
\pgfpathlineto{\pgfqpoint{2.262784in}{1.145418in}}%
\pgfpathlineto{\pgfqpoint{2.262784in}{0.550000in}}%
\pgfpathclose%
\pgfusepath{fill}%
\end{pgfscope}%
\begin{pgfscope}%
\pgfpathrectangle{\pgfqpoint{1.250000in}{0.550000in}}{\pgfqpoint{7.750000in}{3.850000in}}%
\pgfusepath{clip}%
\pgfsetbuttcap%
\pgfsetmiterjoin%
\definecolor{currentfill}{rgb}{0.501961,0.501961,0.501961}%
\pgfsetfillcolor{currentfill}%
\pgfsetfillopacity{0.700000}%
\pgfsetlinewidth{0.000000pt}%
\definecolor{currentstroke}{rgb}{0.000000,0.000000,0.000000}%
\pgfsetstrokecolor{currentstroke}%
\pgfsetstrokeopacity{0.700000}%
\pgfsetdash{}{0pt}%
\pgfpathmoveto{\pgfqpoint{2.290305in}{0.550000in}}%
\pgfpathlineto{\pgfqpoint{2.317827in}{0.550000in}}%
\pgfpathlineto{\pgfqpoint{2.317827in}{1.164012in}}%
\pgfpathlineto{\pgfqpoint{2.290305in}{1.164012in}}%
\pgfpathlineto{\pgfqpoint{2.290305in}{0.550000in}}%
\pgfpathclose%
\pgfusepath{fill}%
\end{pgfscope}%
\begin{pgfscope}%
\pgfpathrectangle{\pgfqpoint{1.250000in}{0.550000in}}{\pgfqpoint{7.750000in}{3.850000in}}%
\pgfusepath{clip}%
\pgfsetbuttcap%
\pgfsetmiterjoin%
\definecolor{currentfill}{rgb}{0.501961,0.501961,0.501961}%
\pgfsetfillcolor{currentfill}%
\pgfsetfillopacity{0.700000}%
\pgfsetlinewidth{0.000000pt}%
\definecolor{currentstroke}{rgb}{0.000000,0.000000,0.000000}%
\pgfsetstrokecolor{currentstroke}%
\pgfsetstrokeopacity{0.700000}%
\pgfsetdash{}{0pt}%
\pgfpathmoveto{\pgfqpoint{2.317827in}{0.550000in}}%
\pgfpathlineto{\pgfqpoint{2.345348in}{0.550000in}}%
\pgfpathlineto{\pgfqpoint{2.345348in}{1.191311in}}%
\pgfpathlineto{\pgfqpoint{2.317827in}{1.191311in}}%
\pgfpathlineto{\pgfqpoint{2.317827in}{0.550000in}}%
\pgfpathclose%
\pgfusepath{fill}%
\end{pgfscope}%
\begin{pgfscope}%
\pgfpathrectangle{\pgfqpoint{1.250000in}{0.550000in}}{\pgfqpoint{7.750000in}{3.850000in}}%
\pgfusepath{clip}%
\pgfsetbuttcap%
\pgfsetmiterjoin%
\definecolor{currentfill}{rgb}{0.501961,0.501961,0.501961}%
\pgfsetfillcolor{currentfill}%
\pgfsetfillopacity{0.700000}%
\pgfsetlinewidth{0.000000pt}%
\definecolor{currentstroke}{rgb}{0.000000,0.000000,0.000000}%
\pgfsetstrokecolor{currentstroke}%
\pgfsetstrokeopacity{0.700000}%
\pgfsetdash{}{0pt}%
\pgfpathmoveto{\pgfqpoint{2.345348in}{0.550000in}}%
\pgfpathlineto{\pgfqpoint{2.372869in}{0.550000in}}%
\pgfpathlineto{\pgfqpoint{2.372869in}{1.294569in}}%
\pgfpathlineto{\pgfqpoint{2.345348in}{1.294569in}}%
\pgfpathlineto{\pgfqpoint{2.345348in}{0.550000in}}%
\pgfpathclose%
\pgfusepath{fill}%
\end{pgfscope}%
\begin{pgfscope}%
\pgfpathrectangle{\pgfqpoint{1.250000in}{0.550000in}}{\pgfqpoint{7.750000in}{3.850000in}}%
\pgfusepath{clip}%
\pgfsetbuttcap%
\pgfsetmiterjoin%
\definecolor{currentfill}{rgb}{0.501961,0.501961,0.501961}%
\pgfsetfillcolor{currentfill}%
\pgfsetfillopacity{0.700000}%
\pgfsetlinewidth{0.000000pt}%
\definecolor{currentstroke}{rgb}{0.000000,0.000000,0.000000}%
\pgfsetstrokecolor{currentstroke}%
\pgfsetstrokeopacity{0.700000}%
\pgfsetdash{}{0pt}%
\pgfpathmoveto{\pgfqpoint{2.372869in}{0.550000in}}%
\pgfpathlineto{\pgfqpoint{2.400391in}{0.550000in}}%
\pgfpathlineto{\pgfqpoint{2.400391in}{1.368156in}}%
\pgfpathlineto{\pgfqpoint{2.372869in}{1.368156in}}%
\pgfpathlineto{\pgfqpoint{2.372869in}{0.550000in}}%
\pgfpathclose%
\pgfusepath{fill}%
\end{pgfscope}%
\begin{pgfscope}%
\pgfpathrectangle{\pgfqpoint{1.250000in}{0.550000in}}{\pgfqpoint{7.750000in}{3.850000in}}%
\pgfusepath{clip}%
\pgfsetbuttcap%
\pgfsetmiterjoin%
\definecolor{currentfill}{rgb}{0.501961,0.501961,0.501961}%
\pgfsetfillcolor{currentfill}%
\pgfsetfillopacity{0.700000}%
\pgfsetlinewidth{0.000000pt}%
\definecolor{currentstroke}{rgb}{0.000000,0.000000,0.000000}%
\pgfsetstrokecolor{currentstroke}%
\pgfsetstrokeopacity{0.700000}%
\pgfsetdash{}{0pt}%
\pgfpathmoveto{\pgfqpoint{2.400391in}{0.550000in}}%
\pgfpathlineto{\pgfqpoint{2.427912in}{0.550000in}}%
\pgfpathlineto{\pgfqpoint{2.427912in}{1.476557in}}%
\pgfpathlineto{\pgfqpoint{2.400391in}{1.476557in}}%
\pgfpathlineto{\pgfqpoint{2.400391in}{0.550000in}}%
\pgfpathclose%
\pgfusepath{fill}%
\end{pgfscope}%
\begin{pgfscope}%
\pgfpathrectangle{\pgfqpoint{1.250000in}{0.550000in}}{\pgfqpoint{7.750000in}{3.850000in}}%
\pgfusepath{clip}%
\pgfsetbuttcap%
\pgfsetmiterjoin%
\definecolor{currentfill}{rgb}{0.501961,0.501961,0.501961}%
\pgfsetfillcolor{currentfill}%
\pgfsetfillopacity{0.700000}%
\pgfsetlinewidth{0.000000pt}%
\definecolor{currentstroke}{rgb}{0.000000,0.000000,0.000000}%
\pgfsetstrokecolor{currentstroke}%
\pgfsetstrokeopacity{0.700000}%
\pgfsetdash{}{0pt}%
\pgfpathmoveto{\pgfqpoint{2.427912in}{0.550000in}}%
\pgfpathlineto{\pgfqpoint{2.455433in}{0.550000in}}%
\pgfpathlineto{\pgfqpoint{2.455433in}{1.641929in}}%
\pgfpathlineto{\pgfqpoint{2.427912in}{1.641929in}}%
\pgfpathlineto{\pgfqpoint{2.427912in}{0.550000in}}%
\pgfpathclose%
\pgfusepath{fill}%
\end{pgfscope}%
\begin{pgfscope}%
\pgfpathrectangle{\pgfqpoint{1.250000in}{0.550000in}}{\pgfqpoint{7.750000in}{3.850000in}}%
\pgfusepath{clip}%
\pgfsetbuttcap%
\pgfsetmiterjoin%
\definecolor{currentfill}{rgb}{0.501961,0.501961,0.501961}%
\pgfsetfillcolor{currentfill}%
\pgfsetfillopacity{0.700000}%
\pgfsetlinewidth{0.000000pt}%
\definecolor{currentstroke}{rgb}{0.000000,0.000000,0.000000}%
\pgfsetstrokecolor{currentstroke}%
\pgfsetstrokeopacity{0.700000}%
\pgfsetdash{}{0pt}%
\pgfpathmoveto{\pgfqpoint{2.455433in}{0.550000in}}%
\pgfpathlineto{\pgfqpoint{2.482955in}{0.550000in}}%
\pgfpathlineto{\pgfqpoint{2.482955in}{1.565573in}}%
\pgfpathlineto{\pgfqpoint{2.455433in}{1.565573in}}%
\pgfpathlineto{\pgfqpoint{2.455433in}{0.550000in}}%
\pgfpathclose%
\pgfusepath{fill}%
\end{pgfscope}%
\begin{pgfscope}%
\pgfpathrectangle{\pgfqpoint{1.250000in}{0.550000in}}{\pgfqpoint{7.750000in}{3.850000in}}%
\pgfusepath{clip}%
\pgfsetbuttcap%
\pgfsetmiterjoin%
\definecolor{currentfill}{rgb}{0.501961,0.501961,0.501961}%
\pgfsetfillcolor{currentfill}%
\pgfsetfillopacity{0.700000}%
\pgfsetlinewidth{0.000000pt}%
\definecolor{currentstroke}{rgb}{0.000000,0.000000,0.000000}%
\pgfsetstrokecolor{currentstroke}%
\pgfsetstrokeopacity{0.700000}%
\pgfsetdash{}{0pt}%
\pgfpathmoveto{\pgfqpoint{2.482955in}{0.550000in}}%
\pgfpathlineto{\pgfqpoint{2.510476in}{0.550000in}}%
\pgfpathlineto{\pgfqpoint{2.510476in}{1.613048in}}%
\pgfpathlineto{\pgfqpoint{2.482955in}{1.613048in}}%
\pgfpathlineto{\pgfqpoint{2.482955in}{0.550000in}}%
\pgfpathclose%
\pgfusepath{fill}%
\end{pgfscope}%
\begin{pgfscope}%
\pgfpathrectangle{\pgfqpoint{1.250000in}{0.550000in}}{\pgfqpoint{7.750000in}{3.850000in}}%
\pgfusepath{clip}%
\pgfsetbuttcap%
\pgfsetmiterjoin%
\definecolor{currentfill}{rgb}{0.501961,0.501961,0.501961}%
\pgfsetfillcolor{currentfill}%
\pgfsetfillopacity{0.700000}%
\pgfsetlinewidth{0.000000pt}%
\definecolor{currentstroke}{rgb}{0.000000,0.000000,0.000000}%
\pgfsetstrokecolor{currentstroke}%
\pgfsetstrokeopacity{0.700000}%
\pgfsetdash{}{0pt}%
\pgfpathmoveto{\pgfqpoint{2.510476in}{0.550000in}}%
\pgfpathlineto{\pgfqpoint{2.537997in}{0.550000in}}%
\pgfpathlineto{\pgfqpoint{2.537997in}{1.790289in}}%
\pgfpathlineto{\pgfqpoint{2.510476in}{1.790289in}}%
\pgfpathlineto{\pgfqpoint{2.510476in}{0.550000in}}%
\pgfpathclose%
\pgfusepath{fill}%
\end{pgfscope}%
\begin{pgfscope}%
\pgfpathrectangle{\pgfqpoint{1.250000in}{0.550000in}}{\pgfqpoint{7.750000in}{3.850000in}}%
\pgfusepath{clip}%
\pgfsetbuttcap%
\pgfsetmiterjoin%
\definecolor{currentfill}{rgb}{0.501961,0.501961,0.501961}%
\pgfsetfillcolor{currentfill}%
\pgfsetfillopacity{0.700000}%
\pgfsetlinewidth{0.000000pt}%
\definecolor{currentstroke}{rgb}{0.000000,0.000000,0.000000}%
\pgfsetstrokecolor{currentstroke}%
\pgfsetstrokeopacity{0.700000}%
\pgfsetdash{}{0pt}%
\pgfpathmoveto{\pgfqpoint{2.537997in}{0.550000in}}%
\pgfpathlineto{\pgfqpoint{2.565518in}{0.550000in}}%
\pgfpathlineto{\pgfqpoint{2.565518in}{1.673975in}}%
\pgfpathlineto{\pgfqpoint{2.537997in}{1.673975in}}%
\pgfpathlineto{\pgfqpoint{2.537997in}{0.550000in}}%
\pgfpathclose%
\pgfusepath{fill}%
\end{pgfscope}%
\begin{pgfscope}%
\pgfpathrectangle{\pgfqpoint{1.250000in}{0.550000in}}{\pgfqpoint{7.750000in}{3.850000in}}%
\pgfusepath{clip}%
\pgfsetbuttcap%
\pgfsetmiterjoin%
\definecolor{currentfill}{rgb}{0.501961,0.501961,0.501961}%
\pgfsetfillcolor{currentfill}%
\pgfsetfillopacity{0.700000}%
\pgfsetlinewidth{0.000000pt}%
\definecolor{currentstroke}{rgb}{0.000000,0.000000,0.000000}%
\pgfsetstrokecolor{currentstroke}%
\pgfsetstrokeopacity{0.700000}%
\pgfsetdash{}{0pt}%
\pgfpathmoveto{\pgfqpoint{2.565518in}{0.550000in}}%
\pgfpathlineto{\pgfqpoint{2.593040in}{0.550000in}}%
\pgfpathlineto{\pgfqpoint{2.593040in}{1.753892in}}%
\pgfpathlineto{\pgfqpoint{2.565518in}{1.753892in}}%
\pgfpathlineto{\pgfqpoint{2.565518in}{0.550000in}}%
\pgfpathclose%
\pgfusepath{fill}%
\end{pgfscope}%
\begin{pgfscope}%
\pgfpathrectangle{\pgfqpoint{1.250000in}{0.550000in}}{\pgfqpoint{7.750000in}{3.850000in}}%
\pgfusepath{clip}%
\pgfsetbuttcap%
\pgfsetmiterjoin%
\definecolor{currentfill}{rgb}{0.501961,0.501961,0.501961}%
\pgfsetfillcolor{currentfill}%
\pgfsetfillopacity{0.700000}%
\pgfsetlinewidth{0.000000pt}%
\definecolor{currentstroke}{rgb}{0.000000,0.000000,0.000000}%
\pgfsetstrokecolor{currentstroke}%
\pgfsetstrokeopacity{0.700000}%
\pgfsetdash{}{0pt}%
\pgfpathmoveto{\pgfqpoint{2.593040in}{0.550000in}}%
\pgfpathlineto{\pgfqpoint{2.620561in}{0.550000in}}%
\pgfpathlineto{\pgfqpoint{2.620561in}{1.768134in}}%
\pgfpathlineto{\pgfqpoint{2.593040in}{1.768134in}}%
\pgfpathlineto{\pgfqpoint{2.593040in}{0.550000in}}%
\pgfpathclose%
\pgfusepath{fill}%
\end{pgfscope}%
\begin{pgfscope}%
\pgfpathrectangle{\pgfqpoint{1.250000in}{0.550000in}}{\pgfqpoint{7.750000in}{3.850000in}}%
\pgfusepath{clip}%
\pgfsetbuttcap%
\pgfsetmiterjoin%
\definecolor{currentfill}{rgb}{0.501961,0.501961,0.501961}%
\pgfsetfillcolor{currentfill}%
\pgfsetfillopacity{0.700000}%
\pgfsetlinewidth{0.000000pt}%
\definecolor{currentstroke}{rgb}{0.000000,0.000000,0.000000}%
\pgfsetstrokecolor{currentstroke}%
\pgfsetstrokeopacity{0.700000}%
\pgfsetdash{}{0pt}%
\pgfpathmoveto{\pgfqpoint{2.620561in}{0.550000in}}%
\pgfpathlineto{\pgfqpoint{2.648082in}{0.550000in}}%
\pgfpathlineto{\pgfqpoint{2.648082in}{0.550000in}}%
\pgfpathlineto{\pgfqpoint{2.620561in}{0.550000in}}%
\pgfpathlineto{\pgfqpoint{2.620561in}{0.550000in}}%
\pgfpathclose%
\pgfusepath{fill}%
\end{pgfscope}%
\begin{pgfscope}%
\pgfpathrectangle{\pgfqpoint{1.250000in}{0.550000in}}{\pgfqpoint{7.750000in}{3.850000in}}%
\pgfusepath{clip}%
\pgfsetbuttcap%
\pgfsetmiterjoin%
\definecolor{currentfill}{rgb}{0.501961,0.501961,0.501961}%
\pgfsetfillcolor{currentfill}%
\pgfsetfillopacity{0.700000}%
\pgfsetlinewidth{0.000000pt}%
\definecolor{currentstroke}{rgb}{0.000000,0.000000,0.000000}%
\pgfsetstrokecolor{currentstroke}%
\pgfsetstrokeopacity{0.700000}%
\pgfsetdash{}{0pt}%
\pgfpathmoveto{\pgfqpoint{2.648082in}{0.550000in}}%
\pgfpathlineto{\pgfqpoint{2.675604in}{0.550000in}}%
\pgfpathlineto{\pgfqpoint{2.675604in}{1.801367in}}%
\pgfpathlineto{\pgfqpoint{2.648082in}{1.801367in}}%
\pgfpathlineto{\pgfqpoint{2.648082in}{0.550000in}}%
\pgfpathclose%
\pgfusepath{fill}%
\end{pgfscope}%
\begin{pgfscope}%
\pgfpathrectangle{\pgfqpoint{1.250000in}{0.550000in}}{\pgfqpoint{7.750000in}{3.850000in}}%
\pgfusepath{clip}%
\pgfsetbuttcap%
\pgfsetmiterjoin%
\definecolor{currentfill}{rgb}{0.501961,0.501961,0.501961}%
\pgfsetfillcolor{currentfill}%
\pgfsetfillopacity{0.700000}%
\pgfsetlinewidth{0.000000pt}%
\definecolor{currentstroke}{rgb}{0.000000,0.000000,0.000000}%
\pgfsetstrokecolor{currentstroke}%
\pgfsetstrokeopacity{0.700000}%
\pgfsetdash{}{0pt}%
\pgfpathmoveto{\pgfqpoint{2.675604in}{0.550000in}}%
\pgfpathlineto{\pgfqpoint{2.703125in}{0.550000in}}%
\pgfpathlineto{\pgfqpoint{2.703125in}{1.788707in}}%
\pgfpathlineto{\pgfqpoint{2.675604in}{1.788707in}}%
\pgfpathlineto{\pgfqpoint{2.675604in}{0.550000in}}%
\pgfpathclose%
\pgfusepath{fill}%
\end{pgfscope}%
\begin{pgfscope}%
\pgfpathrectangle{\pgfqpoint{1.250000in}{0.550000in}}{\pgfqpoint{7.750000in}{3.850000in}}%
\pgfusepath{clip}%
\pgfsetbuttcap%
\pgfsetmiterjoin%
\definecolor{currentfill}{rgb}{0.501961,0.501961,0.501961}%
\pgfsetfillcolor{currentfill}%
\pgfsetfillopacity{0.700000}%
\pgfsetlinewidth{0.000000pt}%
\definecolor{currentstroke}{rgb}{0.000000,0.000000,0.000000}%
\pgfsetstrokecolor{currentstroke}%
\pgfsetstrokeopacity{0.700000}%
\pgfsetdash{}{0pt}%
\pgfpathmoveto{\pgfqpoint{2.703125in}{0.550000in}}%
\pgfpathlineto{\pgfqpoint{2.730646in}{0.550000in}}%
\pgfpathlineto{\pgfqpoint{2.730646in}{1.716703in}}%
\pgfpathlineto{\pgfqpoint{2.703125in}{1.716703in}}%
\pgfpathlineto{\pgfqpoint{2.703125in}{0.550000in}}%
\pgfpathclose%
\pgfusepath{fill}%
\end{pgfscope}%
\begin{pgfscope}%
\pgfpathrectangle{\pgfqpoint{1.250000in}{0.550000in}}{\pgfqpoint{7.750000in}{3.850000in}}%
\pgfusepath{clip}%
\pgfsetbuttcap%
\pgfsetmiterjoin%
\definecolor{currentfill}{rgb}{0.501961,0.501961,0.501961}%
\pgfsetfillcolor{currentfill}%
\pgfsetfillopacity{0.700000}%
\pgfsetlinewidth{0.000000pt}%
\definecolor{currentstroke}{rgb}{0.000000,0.000000,0.000000}%
\pgfsetstrokecolor{currentstroke}%
\pgfsetstrokeopacity{0.700000}%
\pgfsetdash{}{0pt}%
\pgfpathmoveto{\pgfqpoint{2.730646in}{0.550000in}}%
\pgfpathlineto{\pgfqpoint{2.758168in}{0.550000in}}%
\pgfpathlineto{\pgfqpoint{2.758168in}{1.641534in}}%
\pgfpathlineto{\pgfqpoint{2.730646in}{1.641534in}}%
\pgfpathlineto{\pgfqpoint{2.730646in}{0.550000in}}%
\pgfpathclose%
\pgfusepath{fill}%
\end{pgfscope}%
\begin{pgfscope}%
\pgfpathrectangle{\pgfqpoint{1.250000in}{0.550000in}}{\pgfqpoint{7.750000in}{3.850000in}}%
\pgfusepath{clip}%
\pgfsetbuttcap%
\pgfsetmiterjoin%
\definecolor{currentfill}{rgb}{0.501961,0.501961,0.501961}%
\pgfsetfillcolor{currentfill}%
\pgfsetfillopacity{0.700000}%
\pgfsetlinewidth{0.000000pt}%
\definecolor{currentstroke}{rgb}{0.000000,0.000000,0.000000}%
\pgfsetstrokecolor{currentstroke}%
\pgfsetstrokeopacity{0.700000}%
\pgfsetdash{}{0pt}%
\pgfpathmoveto{\pgfqpoint{2.758168in}{0.550000in}}%
\pgfpathlineto{\pgfqpoint{2.785689in}{0.550000in}}%
\pgfpathlineto{\pgfqpoint{2.785689in}{1.477349in}}%
\pgfpathlineto{\pgfqpoint{2.758168in}{1.477349in}}%
\pgfpathlineto{\pgfqpoint{2.758168in}{0.550000in}}%
\pgfpathclose%
\pgfusepath{fill}%
\end{pgfscope}%
\begin{pgfscope}%
\pgfpathrectangle{\pgfqpoint{1.250000in}{0.550000in}}{\pgfqpoint{7.750000in}{3.850000in}}%
\pgfusepath{clip}%
\pgfsetbuttcap%
\pgfsetmiterjoin%
\definecolor{currentfill}{rgb}{0.501961,0.501961,0.501961}%
\pgfsetfillcolor{currentfill}%
\pgfsetfillopacity{0.700000}%
\pgfsetlinewidth{0.000000pt}%
\definecolor{currentstroke}{rgb}{0.000000,0.000000,0.000000}%
\pgfsetstrokecolor{currentstroke}%
\pgfsetstrokeopacity{0.700000}%
\pgfsetdash{}{0pt}%
\pgfpathmoveto{\pgfqpoint{2.785689in}{0.550000in}}%
\pgfpathlineto{\pgfqpoint{2.813210in}{0.550000in}}%
\pgfpathlineto{\pgfqpoint{2.813210in}{1.431060in}}%
\pgfpathlineto{\pgfqpoint{2.785689in}{1.431060in}}%
\pgfpathlineto{\pgfqpoint{2.785689in}{0.550000in}}%
\pgfpathclose%
\pgfusepath{fill}%
\end{pgfscope}%
\begin{pgfscope}%
\pgfpathrectangle{\pgfqpoint{1.250000in}{0.550000in}}{\pgfqpoint{7.750000in}{3.850000in}}%
\pgfusepath{clip}%
\pgfsetbuttcap%
\pgfsetmiterjoin%
\definecolor{currentfill}{rgb}{0.501961,0.501961,0.501961}%
\pgfsetfillcolor{currentfill}%
\pgfsetfillopacity{0.700000}%
\pgfsetlinewidth{0.000000pt}%
\definecolor{currentstroke}{rgb}{0.000000,0.000000,0.000000}%
\pgfsetstrokecolor{currentstroke}%
\pgfsetstrokeopacity{0.700000}%
\pgfsetdash{}{0pt}%
\pgfpathmoveto{\pgfqpoint{2.813210in}{0.550000in}}%
\pgfpathlineto{\pgfqpoint{2.840732in}{0.550000in}}%
\pgfpathlineto{\pgfqpoint{2.840732in}{1.289426in}}%
\pgfpathlineto{\pgfqpoint{2.813210in}{1.289426in}}%
\pgfpathlineto{\pgfqpoint{2.813210in}{0.550000in}}%
\pgfpathclose%
\pgfusepath{fill}%
\end{pgfscope}%
\begin{pgfscope}%
\pgfpathrectangle{\pgfqpoint{1.250000in}{0.550000in}}{\pgfqpoint{7.750000in}{3.850000in}}%
\pgfusepath{clip}%
\pgfsetbuttcap%
\pgfsetmiterjoin%
\definecolor{currentfill}{rgb}{0.501961,0.501961,0.501961}%
\pgfsetfillcolor{currentfill}%
\pgfsetfillopacity{0.700000}%
\pgfsetlinewidth{0.000000pt}%
\definecolor{currentstroke}{rgb}{0.000000,0.000000,0.000000}%
\pgfsetstrokecolor{currentstroke}%
\pgfsetstrokeopacity{0.700000}%
\pgfsetdash{}{0pt}%
\pgfpathmoveto{\pgfqpoint{2.840732in}{0.550000in}}%
\pgfpathlineto{\pgfqpoint{2.868253in}{0.550000in}}%
\pgfpathlineto{\pgfqpoint{2.868253in}{1.277162in}}%
\pgfpathlineto{\pgfqpoint{2.840732in}{1.277162in}}%
\pgfpathlineto{\pgfqpoint{2.840732in}{0.550000in}}%
\pgfpathclose%
\pgfusepath{fill}%
\end{pgfscope}%
\begin{pgfscope}%
\pgfpathrectangle{\pgfqpoint{1.250000in}{0.550000in}}{\pgfqpoint{7.750000in}{3.850000in}}%
\pgfusepath{clip}%
\pgfsetbuttcap%
\pgfsetmiterjoin%
\definecolor{currentfill}{rgb}{0.501961,0.501961,0.501961}%
\pgfsetfillcolor{currentfill}%
\pgfsetfillopacity{0.700000}%
\pgfsetlinewidth{0.000000pt}%
\definecolor{currentstroke}{rgb}{0.000000,0.000000,0.000000}%
\pgfsetstrokecolor{currentstroke}%
\pgfsetstrokeopacity{0.700000}%
\pgfsetdash{}{0pt}%
\pgfpathmoveto{\pgfqpoint{2.868253in}{0.550000in}}%
\pgfpathlineto{\pgfqpoint{2.895774in}{0.550000in}}%
\pgfpathlineto{\pgfqpoint{2.895774in}{1.270040in}}%
\pgfpathlineto{\pgfqpoint{2.868253in}{1.270040in}}%
\pgfpathlineto{\pgfqpoint{2.868253in}{0.550000in}}%
\pgfpathclose%
\pgfusepath{fill}%
\end{pgfscope}%
\begin{pgfscope}%
\pgfpathrectangle{\pgfqpoint{1.250000in}{0.550000in}}{\pgfqpoint{7.750000in}{3.850000in}}%
\pgfusepath{clip}%
\pgfsetbuttcap%
\pgfsetmiterjoin%
\definecolor{currentfill}{rgb}{0.501961,0.501961,0.501961}%
\pgfsetfillcolor{currentfill}%
\pgfsetfillopacity{0.700000}%
\pgfsetlinewidth{0.000000pt}%
\definecolor{currentstroke}{rgb}{0.000000,0.000000,0.000000}%
\pgfsetstrokecolor{currentstroke}%
\pgfsetstrokeopacity{0.700000}%
\pgfsetdash{}{0pt}%
\pgfpathmoveto{\pgfqpoint{2.895774in}{0.550000in}}%
\pgfpathlineto{\pgfqpoint{2.923295in}{0.550000in}}%
\pgfpathlineto{\pgfqpoint{2.923295in}{1.246303in}}%
\pgfpathlineto{\pgfqpoint{2.895774in}{1.246303in}}%
\pgfpathlineto{\pgfqpoint{2.895774in}{0.550000in}}%
\pgfpathclose%
\pgfusepath{fill}%
\end{pgfscope}%
\begin{pgfscope}%
\pgfpathrectangle{\pgfqpoint{1.250000in}{0.550000in}}{\pgfqpoint{7.750000in}{3.850000in}}%
\pgfusepath{clip}%
\pgfsetbuttcap%
\pgfsetmiterjoin%
\definecolor{currentfill}{rgb}{0.501961,0.501961,0.501961}%
\pgfsetfillcolor{currentfill}%
\pgfsetfillopacity{0.700000}%
\pgfsetlinewidth{0.000000pt}%
\definecolor{currentstroke}{rgb}{0.000000,0.000000,0.000000}%
\pgfsetstrokecolor{currentstroke}%
\pgfsetstrokeopacity{0.700000}%
\pgfsetdash{}{0pt}%
\pgfpathmoveto{\pgfqpoint{2.923295in}{0.550000in}}%
\pgfpathlineto{\pgfqpoint{2.950817in}{0.550000in}}%
\pgfpathlineto{\pgfqpoint{2.950817in}{1.208718in}}%
\pgfpathlineto{\pgfqpoint{2.923295in}{1.208718in}}%
\pgfpathlineto{\pgfqpoint{2.923295in}{0.550000in}}%
\pgfpathclose%
\pgfusepath{fill}%
\end{pgfscope}%
\begin{pgfscope}%
\pgfpathrectangle{\pgfqpoint{1.250000in}{0.550000in}}{\pgfqpoint{7.750000in}{3.850000in}}%
\pgfusepath{clip}%
\pgfsetbuttcap%
\pgfsetmiterjoin%
\definecolor{currentfill}{rgb}{0.501961,0.501961,0.501961}%
\pgfsetfillcolor{currentfill}%
\pgfsetfillopacity{0.700000}%
\pgfsetlinewidth{0.000000pt}%
\definecolor{currentstroke}{rgb}{0.000000,0.000000,0.000000}%
\pgfsetstrokecolor{currentstroke}%
\pgfsetstrokeopacity{0.700000}%
\pgfsetdash{}{0pt}%
\pgfpathmoveto{\pgfqpoint{2.950817in}{0.550000in}}%
\pgfpathlineto{\pgfqpoint{2.978338in}{0.550000in}}%
\pgfpathlineto{\pgfqpoint{2.978338in}{0.550000in}}%
\pgfpathlineto{\pgfqpoint{2.950817in}{0.550000in}}%
\pgfpathlineto{\pgfqpoint{2.950817in}{0.550000in}}%
\pgfpathclose%
\pgfusepath{fill}%
\end{pgfscope}%
\begin{pgfscope}%
\pgfpathrectangle{\pgfqpoint{1.250000in}{0.550000in}}{\pgfqpoint{7.750000in}{3.850000in}}%
\pgfusepath{clip}%
\pgfsetbuttcap%
\pgfsetmiterjoin%
\definecolor{currentfill}{rgb}{0.501961,0.501961,0.501961}%
\pgfsetfillcolor{currentfill}%
\pgfsetfillopacity{0.700000}%
\pgfsetlinewidth{0.000000pt}%
\definecolor{currentstroke}{rgb}{0.000000,0.000000,0.000000}%
\pgfsetstrokecolor{currentstroke}%
\pgfsetstrokeopacity{0.700000}%
\pgfsetdash{}{0pt}%
\pgfpathmoveto{\pgfqpoint{2.978338in}{0.550000in}}%
\pgfpathlineto{\pgfqpoint{3.005859in}{0.550000in}}%
\pgfpathlineto{\pgfqpoint{3.005859in}{1.212674in}}%
\pgfpathlineto{\pgfqpoint{2.978338in}{1.212674in}}%
\pgfpathlineto{\pgfqpoint{2.978338in}{0.550000in}}%
\pgfpathclose%
\pgfusepath{fill}%
\end{pgfscope}%
\begin{pgfscope}%
\pgfpathrectangle{\pgfqpoint{1.250000in}{0.550000in}}{\pgfqpoint{7.750000in}{3.850000in}}%
\pgfusepath{clip}%
\pgfsetbuttcap%
\pgfsetmiterjoin%
\definecolor{currentfill}{rgb}{0.501961,0.501961,0.501961}%
\pgfsetfillcolor{currentfill}%
\pgfsetfillopacity{0.700000}%
\pgfsetlinewidth{0.000000pt}%
\definecolor{currentstroke}{rgb}{0.000000,0.000000,0.000000}%
\pgfsetstrokecolor{currentstroke}%
\pgfsetstrokeopacity{0.700000}%
\pgfsetdash{}{0pt}%
\pgfpathmoveto{\pgfqpoint{3.005859in}{0.550000in}}%
\pgfpathlineto{\pgfqpoint{3.033381in}{0.550000in}}%
\pgfpathlineto{\pgfqpoint{3.033381in}{0.550000in}}%
\pgfpathlineto{\pgfqpoint{3.005859in}{0.550000in}}%
\pgfpathlineto{\pgfqpoint{3.005859in}{0.550000in}}%
\pgfpathclose%
\pgfusepath{fill}%
\end{pgfscope}%
\begin{pgfscope}%
\pgfpathrectangle{\pgfqpoint{1.250000in}{0.550000in}}{\pgfqpoint{7.750000in}{3.850000in}}%
\pgfusepath{clip}%
\pgfsetbuttcap%
\pgfsetmiterjoin%
\definecolor{currentfill}{rgb}{0.501961,0.501961,0.501961}%
\pgfsetfillcolor{currentfill}%
\pgfsetfillopacity{0.700000}%
\pgfsetlinewidth{0.000000pt}%
\definecolor{currentstroke}{rgb}{0.000000,0.000000,0.000000}%
\pgfsetstrokecolor{currentstroke}%
\pgfsetstrokeopacity{0.700000}%
\pgfsetdash{}{0pt}%
\pgfpathmoveto{\pgfqpoint{3.033381in}{0.550000in}}%
\pgfpathlineto{\pgfqpoint{3.060902in}{0.550000in}}%
\pgfpathlineto{\pgfqpoint{3.060902in}{1.162034in}}%
\pgfpathlineto{\pgfqpoint{3.033381in}{1.162034in}}%
\pgfpathlineto{\pgfqpoint{3.033381in}{0.550000in}}%
\pgfpathclose%
\pgfusepath{fill}%
\end{pgfscope}%
\begin{pgfscope}%
\pgfpathrectangle{\pgfqpoint{1.250000in}{0.550000in}}{\pgfqpoint{7.750000in}{3.850000in}}%
\pgfusepath{clip}%
\pgfsetbuttcap%
\pgfsetmiterjoin%
\definecolor{currentfill}{rgb}{0.501961,0.501961,0.501961}%
\pgfsetfillcolor{currentfill}%
\pgfsetfillopacity{0.700000}%
\pgfsetlinewidth{0.000000pt}%
\definecolor{currentstroke}{rgb}{0.000000,0.000000,0.000000}%
\pgfsetstrokecolor{currentstroke}%
\pgfsetstrokeopacity{0.700000}%
\pgfsetdash{}{0pt}%
\pgfpathmoveto{\pgfqpoint{3.060902in}{0.550000in}}%
\pgfpathlineto{\pgfqpoint{3.088423in}{0.550000in}}%
\pgfpathlineto{\pgfqpoint{3.088423in}{1.161243in}}%
\pgfpathlineto{\pgfqpoint{3.060902in}{1.161243in}}%
\pgfpathlineto{\pgfqpoint{3.060902in}{0.550000in}}%
\pgfpathclose%
\pgfusepath{fill}%
\end{pgfscope}%
\begin{pgfscope}%
\pgfpathrectangle{\pgfqpoint{1.250000in}{0.550000in}}{\pgfqpoint{7.750000in}{3.850000in}}%
\pgfusepath{clip}%
\pgfsetbuttcap%
\pgfsetmiterjoin%
\definecolor{currentfill}{rgb}{0.501961,0.501961,0.501961}%
\pgfsetfillcolor{currentfill}%
\pgfsetfillopacity{0.700000}%
\pgfsetlinewidth{0.000000pt}%
\definecolor{currentstroke}{rgb}{0.000000,0.000000,0.000000}%
\pgfsetstrokecolor{currentstroke}%
\pgfsetstrokeopacity{0.700000}%
\pgfsetdash{}{0pt}%
\pgfpathmoveto{\pgfqpoint{3.088423in}{0.550000in}}%
\pgfpathlineto{\pgfqpoint{3.115945in}{0.550000in}}%
\pgfpathlineto{\pgfqpoint{3.115945in}{1.144231in}}%
\pgfpathlineto{\pgfqpoint{3.088423in}{1.144231in}}%
\pgfpathlineto{\pgfqpoint{3.088423in}{0.550000in}}%
\pgfpathclose%
\pgfusepath{fill}%
\end{pgfscope}%
\begin{pgfscope}%
\pgfpathrectangle{\pgfqpoint{1.250000in}{0.550000in}}{\pgfqpoint{7.750000in}{3.850000in}}%
\pgfusepath{clip}%
\pgfsetbuttcap%
\pgfsetmiterjoin%
\definecolor{currentfill}{rgb}{0.501961,0.501961,0.501961}%
\pgfsetfillcolor{currentfill}%
\pgfsetfillopacity{0.700000}%
\pgfsetlinewidth{0.000000pt}%
\definecolor{currentstroke}{rgb}{0.000000,0.000000,0.000000}%
\pgfsetstrokecolor{currentstroke}%
\pgfsetstrokeopacity{0.700000}%
\pgfsetdash{}{0pt}%
\pgfpathmoveto{\pgfqpoint{3.115945in}{0.550000in}}%
\pgfpathlineto{\pgfqpoint{3.143466in}{0.550000in}}%
\pgfpathlineto{\pgfqpoint{3.143466in}{1.059171in}}%
\pgfpathlineto{\pgfqpoint{3.115945in}{1.059171in}}%
\pgfpathlineto{\pgfqpoint{3.115945in}{0.550000in}}%
\pgfpathclose%
\pgfusepath{fill}%
\end{pgfscope}%
\begin{pgfscope}%
\pgfpathrectangle{\pgfqpoint{1.250000in}{0.550000in}}{\pgfqpoint{7.750000in}{3.850000in}}%
\pgfusepath{clip}%
\pgfsetbuttcap%
\pgfsetmiterjoin%
\definecolor{currentfill}{rgb}{0.501961,0.501961,0.501961}%
\pgfsetfillcolor{currentfill}%
\pgfsetfillopacity{0.700000}%
\pgfsetlinewidth{0.000000pt}%
\definecolor{currentstroke}{rgb}{0.000000,0.000000,0.000000}%
\pgfsetstrokecolor{currentstroke}%
\pgfsetstrokeopacity{0.700000}%
\pgfsetdash{}{0pt}%
\pgfpathmoveto{\pgfqpoint{3.143466in}{0.550000in}}%
\pgfpathlineto{\pgfqpoint{3.170987in}{0.550000in}}%
\pgfpathlineto{\pgfqpoint{3.170987in}{0.550000in}}%
\pgfpathlineto{\pgfqpoint{3.143466in}{0.550000in}}%
\pgfpathlineto{\pgfqpoint{3.143466in}{0.550000in}}%
\pgfpathclose%
\pgfusepath{fill}%
\end{pgfscope}%
\begin{pgfscope}%
\pgfpathrectangle{\pgfqpoint{1.250000in}{0.550000in}}{\pgfqpoint{7.750000in}{3.850000in}}%
\pgfusepath{clip}%
\pgfsetbuttcap%
\pgfsetmiterjoin%
\definecolor{currentfill}{rgb}{0.501961,0.501961,0.501961}%
\pgfsetfillcolor{currentfill}%
\pgfsetfillopacity{0.700000}%
\pgfsetlinewidth{0.000000pt}%
\definecolor{currentstroke}{rgb}{0.000000,0.000000,0.000000}%
\pgfsetstrokecolor{currentstroke}%
\pgfsetstrokeopacity{0.700000}%
\pgfsetdash{}{0pt}%
\pgfpathmoveto{\pgfqpoint{3.170987in}{0.550000in}}%
\pgfpathlineto{\pgfqpoint{3.198509in}{0.550000in}}%
\pgfpathlineto{\pgfqpoint{3.198509in}{1.025147in}}%
\pgfpathlineto{\pgfqpoint{3.170987in}{1.025147in}}%
\pgfpathlineto{\pgfqpoint{3.170987in}{0.550000in}}%
\pgfpathclose%
\pgfusepath{fill}%
\end{pgfscope}%
\begin{pgfscope}%
\pgfpathrectangle{\pgfqpoint{1.250000in}{0.550000in}}{\pgfqpoint{7.750000in}{3.850000in}}%
\pgfusepath{clip}%
\pgfsetbuttcap%
\pgfsetmiterjoin%
\definecolor{currentfill}{rgb}{0.501961,0.501961,0.501961}%
\pgfsetfillcolor{currentfill}%
\pgfsetfillopacity{0.700000}%
\pgfsetlinewidth{0.000000pt}%
\definecolor{currentstroke}{rgb}{0.000000,0.000000,0.000000}%
\pgfsetstrokecolor{currentstroke}%
\pgfsetstrokeopacity{0.700000}%
\pgfsetdash{}{0pt}%
\pgfpathmoveto{\pgfqpoint{3.198509in}{0.550000in}}%
\pgfpathlineto{\pgfqpoint{3.226030in}{0.550000in}}%
\pgfpathlineto{\pgfqpoint{3.226030in}{1.054819in}}%
\pgfpathlineto{\pgfqpoint{3.198509in}{1.054819in}}%
\pgfpathlineto{\pgfqpoint{3.198509in}{0.550000in}}%
\pgfpathclose%
\pgfusepath{fill}%
\end{pgfscope}%
\begin{pgfscope}%
\pgfpathrectangle{\pgfqpoint{1.250000in}{0.550000in}}{\pgfqpoint{7.750000in}{3.850000in}}%
\pgfusepath{clip}%
\pgfsetbuttcap%
\pgfsetmiterjoin%
\definecolor{currentfill}{rgb}{0.501961,0.501961,0.501961}%
\pgfsetfillcolor{currentfill}%
\pgfsetfillopacity{0.700000}%
\pgfsetlinewidth{0.000000pt}%
\definecolor{currentstroke}{rgb}{0.000000,0.000000,0.000000}%
\pgfsetstrokecolor{currentstroke}%
\pgfsetstrokeopacity{0.700000}%
\pgfsetdash{}{0pt}%
\pgfpathmoveto{\pgfqpoint{3.226030in}{0.550000in}}%
\pgfpathlineto{\pgfqpoint{3.253551in}{0.550000in}}%
\pgfpathlineto{\pgfqpoint{3.253551in}{1.013279in}}%
\pgfpathlineto{\pgfqpoint{3.226030in}{1.013279in}}%
\pgfpathlineto{\pgfqpoint{3.226030in}{0.550000in}}%
\pgfpathclose%
\pgfusepath{fill}%
\end{pgfscope}%
\begin{pgfscope}%
\pgfpathrectangle{\pgfqpoint{1.250000in}{0.550000in}}{\pgfqpoint{7.750000in}{3.850000in}}%
\pgfusepath{clip}%
\pgfsetbuttcap%
\pgfsetmiterjoin%
\definecolor{currentfill}{rgb}{0.501961,0.501961,0.501961}%
\pgfsetfillcolor{currentfill}%
\pgfsetfillopacity{0.700000}%
\pgfsetlinewidth{0.000000pt}%
\definecolor{currentstroke}{rgb}{0.000000,0.000000,0.000000}%
\pgfsetstrokecolor{currentstroke}%
\pgfsetstrokeopacity{0.700000}%
\pgfsetdash{}{0pt}%
\pgfpathmoveto{\pgfqpoint{3.253551in}{0.550000in}}%
\pgfpathlineto{\pgfqpoint{3.281072in}{0.550000in}}%
\pgfpathlineto{\pgfqpoint{3.281072in}{0.986772in}}%
\pgfpathlineto{\pgfqpoint{3.253551in}{0.986772in}}%
\pgfpathlineto{\pgfqpoint{3.253551in}{0.550000in}}%
\pgfpathclose%
\pgfusepath{fill}%
\end{pgfscope}%
\begin{pgfscope}%
\pgfpathrectangle{\pgfqpoint{1.250000in}{0.550000in}}{\pgfqpoint{7.750000in}{3.850000in}}%
\pgfusepath{clip}%
\pgfsetbuttcap%
\pgfsetmiterjoin%
\definecolor{currentfill}{rgb}{0.501961,0.501961,0.501961}%
\pgfsetfillcolor{currentfill}%
\pgfsetfillopacity{0.700000}%
\pgfsetlinewidth{0.000000pt}%
\definecolor{currentstroke}{rgb}{0.000000,0.000000,0.000000}%
\pgfsetstrokecolor{currentstroke}%
\pgfsetstrokeopacity{0.700000}%
\pgfsetdash{}{0pt}%
\pgfpathmoveto{\pgfqpoint{3.281072in}{0.550000in}}%
\pgfpathlineto{\pgfqpoint{3.308594in}{0.550000in}}%
\pgfpathlineto{\pgfqpoint{3.308594in}{0.550000in}}%
\pgfpathlineto{\pgfqpoint{3.281072in}{0.550000in}}%
\pgfpathlineto{\pgfqpoint{3.281072in}{0.550000in}}%
\pgfpathclose%
\pgfusepath{fill}%
\end{pgfscope}%
\begin{pgfscope}%
\pgfpathrectangle{\pgfqpoint{1.250000in}{0.550000in}}{\pgfqpoint{7.750000in}{3.850000in}}%
\pgfusepath{clip}%
\pgfsetbuttcap%
\pgfsetmiterjoin%
\definecolor{currentfill}{rgb}{0.501961,0.501961,0.501961}%
\pgfsetfillcolor{currentfill}%
\pgfsetfillopacity{0.700000}%
\pgfsetlinewidth{0.000000pt}%
\definecolor{currentstroke}{rgb}{0.000000,0.000000,0.000000}%
\pgfsetstrokecolor{currentstroke}%
\pgfsetstrokeopacity{0.700000}%
\pgfsetdash{}{0pt}%
\pgfpathmoveto{\pgfqpoint{3.308594in}{0.550000in}}%
\pgfpathlineto{\pgfqpoint{3.336115in}{0.550000in}}%
\pgfpathlineto{\pgfqpoint{3.336115in}{0.974903in}}%
\pgfpathlineto{\pgfqpoint{3.308594in}{0.974903in}}%
\pgfpathlineto{\pgfqpoint{3.308594in}{0.550000in}}%
\pgfpathclose%
\pgfusepath{fill}%
\end{pgfscope}%
\begin{pgfscope}%
\pgfpathrectangle{\pgfqpoint{1.250000in}{0.550000in}}{\pgfqpoint{7.750000in}{3.850000in}}%
\pgfusepath{clip}%
\pgfsetbuttcap%
\pgfsetmiterjoin%
\definecolor{currentfill}{rgb}{0.501961,0.501961,0.501961}%
\pgfsetfillcolor{currentfill}%
\pgfsetfillopacity{0.700000}%
\pgfsetlinewidth{0.000000pt}%
\definecolor{currentstroke}{rgb}{0.000000,0.000000,0.000000}%
\pgfsetstrokecolor{currentstroke}%
\pgfsetstrokeopacity{0.700000}%
\pgfsetdash{}{0pt}%
\pgfpathmoveto{\pgfqpoint{3.336115in}{0.550000in}}%
\pgfpathlineto{\pgfqpoint{3.363636in}{0.550000in}}%
\pgfpathlineto{\pgfqpoint{3.363636in}{0.948396in}}%
\pgfpathlineto{\pgfqpoint{3.336115in}{0.948396in}}%
\pgfpathlineto{\pgfqpoint{3.336115in}{0.550000in}}%
\pgfpathclose%
\pgfusepath{fill}%
\end{pgfscope}%
\begin{pgfscope}%
\pgfpathrectangle{\pgfqpoint{1.250000in}{0.550000in}}{\pgfqpoint{7.750000in}{3.850000in}}%
\pgfusepath{clip}%
\pgfsetbuttcap%
\pgfsetmiterjoin%
\definecolor{currentfill}{rgb}{0.501961,0.501961,0.501961}%
\pgfsetfillcolor{currentfill}%
\pgfsetfillopacity{0.700000}%
\pgfsetlinewidth{0.000000pt}%
\definecolor{currentstroke}{rgb}{0.000000,0.000000,0.000000}%
\pgfsetstrokecolor{currentstroke}%
\pgfsetstrokeopacity{0.700000}%
\pgfsetdash{}{0pt}%
\pgfpathmoveto{\pgfqpoint{3.363636in}{0.550000in}}%
\pgfpathlineto{\pgfqpoint{3.391158in}{0.550000in}}%
\pgfpathlineto{\pgfqpoint{3.391158in}{0.914768in}}%
\pgfpathlineto{\pgfqpoint{3.363636in}{0.914768in}}%
\pgfpathlineto{\pgfqpoint{3.363636in}{0.550000in}}%
\pgfpathclose%
\pgfusepath{fill}%
\end{pgfscope}%
\begin{pgfscope}%
\pgfpathrectangle{\pgfqpoint{1.250000in}{0.550000in}}{\pgfqpoint{7.750000in}{3.850000in}}%
\pgfusepath{clip}%
\pgfsetbuttcap%
\pgfsetmiterjoin%
\definecolor{currentfill}{rgb}{0.501961,0.501961,0.501961}%
\pgfsetfillcolor{currentfill}%
\pgfsetfillopacity{0.700000}%
\pgfsetlinewidth{0.000000pt}%
\definecolor{currentstroke}{rgb}{0.000000,0.000000,0.000000}%
\pgfsetstrokecolor{currentstroke}%
\pgfsetstrokeopacity{0.700000}%
\pgfsetdash{}{0pt}%
\pgfpathmoveto{\pgfqpoint{3.391158in}{0.550000in}}%
\pgfpathlineto{\pgfqpoint{3.418679in}{0.550000in}}%
\pgfpathlineto{\pgfqpoint{3.418679in}{0.925845in}}%
\pgfpathlineto{\pgfqpoint{3.391158in}{0.925845in}}%
\pgfpathlineto{\pgfqpoint{3.391158in}{0.550000in}}%
\pgfpathclose%
\pgfusepath{fill}%
\end{pgfscope}%
\begin{pgfscope}%
\pgfpathrectangle{\pgfqpoint{1.250000in}{0.550000in}}{\pgfqpoint{7.750000in}{3.850000in}}%
\pgfusepath{clip}%
\pgfsetbuttcap%
\pgfsetmiterjoin%
\definecolor{currentfill}{rgb}{0.501961,0.501961,0.501961}%
\pgfsetfillcolor{currentfill}%
\pgfsetfillopacity{0.700000}%
\pgfsetlinewidth{0.000000pt}%
\definecolor{currentstroke}{rgb}{0.000000,0.000000,0.000000}%
\pgfsetstrokecolor{currentstroke}%
\pgfsetstrokeopacity{0.700000}%
\pgfsetdash{}{0pt}%
\pgfpathmoveto{\pgfqpoint{3.418679in}{0.550000in}}%
\pgfpathlineto{\pgfqpoint{3.446200in}{0.550000in}}%
\pgfpathlineto{\pgfqpoint{3.446200in}{0.550000in}}%
\pgfpathlineto{\pgfqpoint{3.418679in}{0.550000in}}%
\pgfpathlineto{\pgfqpoint{3.418679in}{0.550000in}}%
\pgfpathclose%
\pgfusepath{fill}%
\end{pgfscope}%
\begin{pgfscope}%
\pgfpathrectangle{\pgfqpoint{1.250000in}{0.550000in}}{\pgfqpoint{7.750000in}{3.850000in}}%
\pgfusepath{clip}%
\pgfsetbuttcap%
\pgfsetmiterjoin%
\definecolor{currentfill}{rgb}{0.501961,0.501961,0.501961}%
\pgfsetfillcolor{currentfill}%
\pgfsetfillopacity{0.700000}%
\pgfsetlinewidth{0.000000pt}%
\definecolor{currentstroke}{rgb}{0.000000,0.000000,0.000000}%
\pgfsetstrokecolor{currentstroke}%
\pgfsetstrokeopacity{0.700000}%
\pgfsetdash{}{0pt}%
\pgfpathmoveto{\pgfqpoint{3.446200in}{0.550000in}}%
\pgfpathlineto{\pgfqpoint{3.473722in}{0.550000in}}%
\pgfpathlineto{\pgfqpoint{3.473722in}{0.927032in}}%
\pgfpathlineto{\pgfqpoint{3.446200in}{0.927032in}}%
\pgfpathlineto{\pgfqpoint{3.446200in}{0.550000in}}%
\pgfpathclose%
\pgfusepath{fill}%
\end{pgfscope}%
\begin{pgfscope}%
\pgfpathrectangle{\pgfqpoint{1.250000in}{0.550000in}}{\pgfqpoint{7.750000in}{3.850000in}}%
\pgfusepath{clip}%
\pgfsetbuttcap%
\pgfsetmiterjoin%
\definecolor{currentfill}{rgb}{0.501961,0.501961,0.501961}%
\pgfsetfillcolor{currentfill}%
\pgfsetfillopacity{0.700000}%
\pgfsetlinewidth{0.000000pt}%
\definecolor{currentstroke}{rgb}{0.000000,0.000000,0.000000}%
\pgfsetstrokecolor{currentstroke}%
\pgfsetstrokeopacity{0.700000}%
\pgfsetdash{}{0pt}%
\pgfpathmoveto{\pgfqpoint{3.473722in}{0.550000in}}%
\pgfpathlineto{\pgfqpoint{3.501243in}{0.550000in}}%
\pgfpathlineto{\pgfqpoint{3.501243in}{0.900921in}}%
\pgfpathlineto{\pgfqpoint{3.473722in}{0.900921in}}%
\pgfpathlineto{\pgfqpoint{3.473722in}{0.550000in}}%
\pgfpathclose%
\pgfusepath{fill}%
\end{pgfscope}%
\begin{pgfscope}%
\pgfpathrectangle{\pgfqpoint{1.250000in}{0.550000in}}{\pgfqpoint{7.750000in}{3.850000in}}%
\pgfusepath{clip}%
\pgfsetbuttcap%
\pgfsetmiterjoin%
\definecolor{currentfill}{rgb}{0.501961,0.501961,0.501961}%
\pgfsetfillcolor{currentfill}%
\pgfsetfillopacity{0.700000}%
\pgfsetlinewidth{0.000000pt}%
\definecolor{currentstroke}{rgb}{0.000000,0.000000,0.000000}%
\pgfsetstrokecolor{currentstroke}%
\pgfsetstrokeopacity{0.700000}%
\pgfsetdash{}{0pt}%
\pgfpathmoveto{\pgfqpoint{3.501243in}{0.550000in}}%
\pgfpathlineto{\pgfqpoint{3.528764in}{0.550000in}}%
\pgfpathlineto{\pgfqpoint{3.528764in}{0.904086in}}%
\pgfpathlineto{\pgfqpoint{3.501243in}{0.904086in}}%
\pgfpathlineto{\pgfqpoint{3.501243in}{0.550000in}}%
\pgfpathclose%
\pgfusepath{fill}%
\end{pgfscope}%
\begin{pgfscope}%
\pgfpathrectangle{\pgfqpoint{1.250000in}{0.550000in}}{\pgfqpoint{7.750000in}{3.850000in}}%
\pgfusepath{clip}%
\pgfsetbuttcap%
\pgfsetmiterjoin%
\definecolor{currentfill}{rgb}{0.501961,0.501961,0.501961}%
\pgfsetfillcolor{currentfill}%
\pgfsetfillopacity{0.700000}%
\pgfsetlinewidth{0.000000pt}%
\definecolor{currentstroke}{rgb}{0.000000,0.000000,0.000000}%
\pgfsetstrokecolor{currentstroke}%
\pgfsetstrokeopacity{0.700000}%
\pgfsetdash{}{0pt}%
\pgfpathmoveto{\pgfqpoint{3.528764in}{0.550000in}}%
\pgfpathlineto{\pgfqpoint{3.556286in}{0.550000in}}%
\pgfpathlineto{\pgfqpoint{3.556286in}{0.911207in}}%
\pgfpathlineto{\pgfqpoint{3.528764in}{0.911207in}}%
\pgfpathlineto{\pgfqpoint{3.528764in}{0.550000in}}%
\pgfpathclose%
\pgfusepath{fill}%
\end{pgfscope}%
\begin{pgfscope}%
\pgfpathrectangle{\pgfqpoint{1.250000in}{0.550000in}}{\pgfqpoint{7.750000in}{3.850000in}}%
\pgfusepath{clip}%
\pgfsetbuttcap%
\pgfsetmiterjoin%
\definecolor{currentfill}{rgb}{0.501961,0.501961,0.501961}%
\pgfsetfillcolor{currentfill}%
\pgfsetfillopacity{0.700000}%
\pgfsetlinewidth{0.000000pt}%
\definecolor{currentstroke}{rgb}{0.000000,0.000000,0.000000}%
\pgfsetstrokecolor{currentstroke}%
\pgfsetstrokeopacity{0.700000}%
\pgfsetdash{}{0pt}%
\pgfpathmoveto{\pgfqpoint{3.556286in}{0.550000in}}%
\pgfpathlineto{\pgfqpoint{3.583807in}{0.550000in}}%
\pgfpathlineto{\pgfqpoint{3.583807in}{0.550000in}}%
\pgfpathlineto{\pgfqpoint{3.556286in}{0.550000in}}%
\pgfpathlineto{\pgfqpoint{3.556286in}{0.550000in}}%
\pgfpathclose%
\pgfusepath{fill}%
\end{pgfscope}%
\begin{pgfscope}%
\pgfpathrectangle{\pgfqpoint{1.250000in}{0.550000in}}{\pgfqpoint{7.750000in}{3.850000in}}%
\pgfusepath{clip}%
\pgfsetbuttcap%
\pgfsetmiterjoin%
\definecolor{currentfill}{rgb}{0.501961,0.501961,0.501961}%
\pgfsetfillcolor{currentfill}%
\pgfsetfillopacity{0.700000}%
\pgfsetlinewidth{0.000000pt}%
\definecolor{currentstroke}{rgb}{0.000000,0.000000,0.000000}%
\pgfsetstrokecolor{currentstroke}%
\pgfsetstrokeopacity{0.700000}%
\pgfsetdash{}{0pt}%
\pgfpathmoveto{\pgfqpoint{3.583807in}{0.550000in}}%
\pgfpathlineto{\pgfqpoint{3.611328in}{0.550000in}}%
\pgfpathlineto{\pgfqpoint{3.611328in}{0.910416in}}%
\pgfpathlineto{\pgfqpoint{3.583807in}{0.910416in}}%
\pgfpathlineto{\pgfqpoint{3.583807in}{0.550000in}}%
\pgfpathclose%
\pgfusepath{fill}%
\end{pgfscope}%
\begin{pgfscope}%
\pgfpathrectangle{\pgfqpoint{1.250000in}{0.550000in}}{\pgfqpoint{7.750000in}{3.850000in}}%
\pgfusepath{clip}%
\pgfsetbuttcap%
\pgfsetmiterjoin%
\definecolor{currentfill}{rgb}{0.501961,0.501961,0.501961}%
\pgfsetfillcolor{currentfill}%
\pgfsetfillopacity{0.700000}%
\pgfsetlinewidth{0.000000pt}%
\definecolor{currentstroke}{rgb}{0.000000,0.000000,0.000000}%
\pgfsetstrokecolor{currentstroke}%
\pgfsetstrokeopacity{0.700000}%
\pgfsetdash{}{0pt}%
\pgfpathmoveto{\pgfqpoint{3.611328in}{0.550000in}}%
\pgfpathlineto{\pgfqpoint{3.638849in}{0.550000in}}%
\pgfpathlineto{\pgfqpoint{3.638849in}{0.893008in}}%
\pgfpathlineto{\pgfqpoint{3.611328in}{0.893008in}}%
\pgfpathlineto{\pgfqpoint{3.611328in}{0.550000in}}%
\pgfpathclose%
\pgfusepath{fill}%
\end{pgfscope}%
\begin{pgfscope}%
\pgfpathrectangle{\pgfqpoint{1.250000in}{0.550000in}}{\pgfqpoint{7.750000in}{3.850000in}}%
\pgfusepath{clip}%
\pgfsetbuttcap%
\pgfsetmiterjoin%
\definecolor{currentfill}{rgb}{0.501961,0.501961,0.501961}%
\pgfsetfillcolor{currentfill}%
\pgfsetfillopacity{0.700000}%
\pgfsetlinewidth{0.000000pt}%
\definecolor{currentstroke}{rgb}{0.000000,0.000000,0.000000}%
\pgfsetstrokecolor{currentstroke}%
\pgfsetstrokeopacity{0.700000}%
\pgfsetdash{}{0pt}%
\pgfpathmoveto{\pgfqpoint{3.638849in}{0.550000in}}%
\pgfpathlineto{\pgfqpoint{3.666371in}{0.550000in}}%
\pgfpathlineto{\pgfqpoint{3.666371in}{0.896964in}}%
\pgfpathlineto{\pgfqpoint{3.638849in}{0.896964in}}%
\pgfpathlineto{\pgfqpoint{3.638849in}{0.550000in}}%
\pgfpathclose%
\pgfusepath{fill}%
\end{pgfscope}%
\begin{pgfscope}%
\pgfpathrectangle{\pgfqpoint{1.250000in}{0.550000in}}{\pgfqpoint{7.750000in}{3.850000in}}%
\pgfusepath{clip}%
\pgfsetbuttcap%
\pgfsetmiterjoin%
\definecolor{currentfill}{rgb}{0.501961,0.501961,0.501961}%
\pgfsetfillcolor{currentfill}%
\pgfsetfillopacity{0.700000}%
\pgfsetlinewidth{0.000000pt}%
\definecolor{currentstroke}{rgb}{0.000000,0.000000,0.000000}%
\pgfsetstrokecolor{currentstroke}%
\pgfsetstrokeopacity{0.700000}%
\pgfsetdash{}{0pt}%
\pgfpathmoveto{\pgfqpoint{3.666371in}{0.550000in}}%
\pgfpathlineto{\pgfqpoint{3.693892in}{0.550000in}}%
\pgfpathlineto{\pgfqpoint{3.693892in}{0.893799in}}%
\pgfpathlineto{\pgfqpoint{3.666371in}{0.893799in}}%
\pgfpathlineto{\pgfqpoint{3.666371in}{0.550000in}}%
\pgfpathclose%
\pgfusepath{fill}%
\end{pgfscope}%
\begin{pgfscope}%
\pgfpathrectangle{\pgfqpoint{1.250000in}{0.550000in}}{\pgfqpoint{7.750000in}{3.850000in}}%
\pgfusepath{clip}%
\pgfsetbuttcap%
\pgfsetmiterjoin%
\definecolor{currentfill}{rgb}{0.501961,0.501961,0.501961}%
\pgfsetfillcolor{currentfill}%
\pgfsetfillopacity{0.700000}%
\pgfsetlinewidth{0.000000pt}%
\definecolor{currentstroke}{rgb}{0.000000,0.000000,0.000000}%
\pgfsetstrokecolor{currentstroke}%
\pgfsetstrokeopacity{0.700000}%
\pgfsetdash{}{0pt}%
\pgfpathmoveto{\pgfqpoint{3.693892in}{0.550000in}}%
\pgfpathlineto{\pgfqpoint{3.721413in}{0.550000in}}%
\pgfpathlineto{\pgfqpoint{3.721413in}{0.550000in}}%
\pgfpathlineto{\pgfqpoint{3.693892in}{0.550000in}}%
\pgfpathlineto{\pgfqpoint{3.693892in}{0.550000in}}%
\pgfpathclose%
\pgfusepath{fill}%
\end{pgfscope}%
\begin{pgfscope}%
\pgfpathrectangle{\pgfqpoint{1.250000in}{0.550000in}}{\pgfqpoint{7.750000in}{3.850000in}}%
\pgfusepath{clip}%
\pgfsetbuttcap%
\pgfsetmiterjoin%
\definecolor{currentfill}{rgb}{0.501961,0.501961,0.501961}%
\pgfsetfillcolor{currentfill}%
\pgfsetfillopacity{0.700000}%
\pgfsetlinewidth{0.000000pt}%
\definecolor{currentstroke}{rgb}{0.000000,0.000000,0.000000}%
\pgfsetstrokecolor{currentstroke}%
\pgfsetstrokeopacity{0.700000}%
\pgfsetdash{}{0pt}%
\pgfpathmoveto{\pgfqpoint{3.721413in}{0.550000in}}%
\pgfpathlineto{\pgfqpoint{3.748935in}{0.550000in}}%
\pgfpathlineto{\pgfqpoint{3.748935in}{0.875996in}}%
\pgfpathlineto{\pgfqpoint{3.721413in}{0.875996in}}%
\pgfpathlineto{\pgfqpoint{3.721413in}{0.550000in}}%
\pgfpathclose%
\pgfusepath{fill}%
\end{pgfscope}%
\begin{pgfscope}%
\pgfpathrectangle{\pgfqpoint{1.250000in}{0.550000in}}{\pgfqpoint{7.750000in}{3.850000in}}%
\pgfusepath{clip}%
\pgfsetbuttcap%
\pgfsetmiterjoin%
\definecolor{currentfill}{rgb}{0.501961,0.501961,0.501961}%
\pgfsetfillcolor{currentfill}%
\pgfsetfillopacity{0.700000}%
\pgfsetlinewidth{0.000000pt}%
\definecolor{currentstroke}{rgb}{0.000000,0.000000,0.000000}%
\pgfsetstrokecolor{currentstroke}%
\pgfsetstrokeopacity{0.700000}%
\pgfsetdash{}{0pt}%
\pgfpathmoveto{\pgfqpoint{3.748935in}{0.550000in}}%
\pgfpathlineto{\pgfqpoint{3.776456in}{0.550000in}}%
\pgfpathlineto{\pgfqpoint{3.776456in}{0.877579in}}%
\pgfpathlineto{\pgfqpoint{3.748935in}{0.877579in}}%
\pgfpathlineto{\pgfqpoint{3.748935in}{0.550000in}}%
\pgfpathclose%
\pgfusepath{fill}%
\end{pgfscope}%
\begin{pgfscope}%
\pgfpathrectangle{\pgfqpoint{1.250000in}{0.550000in}}{\pgfqpoint{7.750000in}{3.850000in}}%
\pgfusepath{clip}%
\pgfsetbuttcap%
\pgfsetmiterjoin%
\definecolor{currentfill}{rgb}{0.501961,0.501961,0.501961}%
\pgfsetfillcolor{currentfill}%
\pgfsetfillopacity{0.700000}%
\pgfsetlinewidth{0.000000pt}%
\definecolor{currentstroke}{rgb}{0.000000,0.000000,0.000000}%
\pgfsetstrokecolor{currentstroke}%
\pgfsetstrokeopacity{0.700000}%
\pgfsetdash{}{0pt}%
\pgfpathmoveto{\pgfqpoint{3.776456in}{0.550000in}}%
\pgfpathlineto{\pgfqpoint{3.803977in}{0.550000in}}%
\pgfpathlineto{\pgfqpoint{3.803977in}{0.868084in}}%
\pgfpathlineto{\pgfqpoint{3.776456in}{0.868084in}}%
\pgfpathlineto{\pgfqpoint{3.776456in}{0.550000in}}%
\pgfpathclose%
\pgfusepath{fill}%
\end{pgfscope}%
\begin{pgfscope}%
\pgfpathrectangle{\pgfqpoint{1.250000in}{0.550000in}}{\pgfqpoint{7.750000in}{3.850000in}}%
\pgfusepath{clip}%
\pgfsetbuttcap%
\pgfsetmiterjoin%
\definecolor{currentfill}{rgb}{0.501961,0.501961,0.501961}%
\pgfsetfillcolor{currentfill}%
\pgfsetfillopacity{0.700000}%
\pgfsetlinewidth{0.000000pt}%
\definecolor{currentstroke}{rgb}{0.000000,0.000000,0.000000}%
\pgfsetstrokecolor{currentstroke}%
\pgfsetstrokeopacity{0.700000}%
\pgfsetdash{}{0pt}%
\pgfpathmoveto{\pgfqpoint{3.803977in}{0.550000in}}%
\pgfpathlineto{\pgfqpoint{3.831499in}{0.550000in}}%
\pgfpathlineto{\pgfqpoint{3.831499in}{0.866501in}}%
\pgfpathlineto{\pgfqpoint{3.803977in}{0.866501in}}%
\pgfpathlineto{\pgfqpoint{3.803977in}{0.550000in}}%
\pgfpathclose%
\pgfusepath{fill}%
\end{pgfscope}%
\begin{pgfscope}%
\pgfpathrectangle{\pgfqpoint{1.250000in}{0.550000in}}{\pgfqpoint{7.750000in}{3.850000in}}%
\pgfusepath{clip}%
\pgfsetbuttcap%
\pgfsetmiterjoin%
\definecolor{currentfill}{rgb}{0.501961,0.501961,0.501961}%
\pgfsetfillcolor{currentfill}%
\pgfsetfillopacity{0.700000}%
\pgfsetlinewidth{0.000000pt}%
\definecolor{currentstroke}{rgb}{0.000000,0.000000,0.000000}%
\pgfsetstrokecolor{currentstroke}%
\pgfsetstrokeopacity{0.700000}%
\pgfsetdash{}{0pt}%
\pgfpathmoveto{\pgfqpoint{3.831499in}{0.550000in}}%
\pgfpathlineto{\pgfqpoint{3.859020in}{0.550000in}}%
\pgfpathlineto{\pgfqpoint{3.859020in}{0.550000in}}%
\pgfpathlineto{\pgfqpoint{3.831499in}{0.550000in}}%
\pgfpathlineto{\pgfqpoint{3.831499in}{0.550000in}}%
\pgfpathclose%
\pgfusepath{fill}%
\end{pgfscope}%
\begin{pgfscope}%
\pgfpathrectangle{\pgfqpoint{1.250000in}{0.550000in}}{\pgfqpoint{7.750000in}{3.850000in}}%
\pgfusepath{clip}%
\pgfsetbuttcap%
\pgfsetmiterjoin%
\definecolor{currentfill}{rgb}{0.501961,0.501961,0.501961}%
\pgfsetfillcolor{currentfill}%
\pgfsetfillopacity{0.700000}%
\pgfsetlinewidth{0.000000pt}%
\definecolor{currentstroke}{rgb}{0.000000,0.000000,0.000000}%
\pgfsetstrokecolor{currentstroke}%
\pgfsetstrokeopacity{0.700000}%
\pgfsetdash{}{0pt}%
\pgfpathmoveto{\pgfqpoint{3.859020in}{0.550000in}}%
\pgfpathlineto{\pgfqpoint{3.886541in}{0.550000in}}%
\pgfpathlineto{\pgfqpoint{3.886541in}{0.841577in}}%
\pgfpathlineto{\pgfqpoint{3.859020in}{0.841577in}}%
\pgfpathlineto{\pgfqpoint{3.859020in}{0.550000in}}%
\pgfpathclose%
\pgfusepath{fill}%
\end{pgfscope}%
\begin{pgfscope}%
\pgfpathrectangle{\pgfqpoint{1.250000in}{0.550000in}}{\pgfqpoint{7.750000in}{3.850000in}}%
\pgfusepath{clip}%
\pgfsetbuttcap%
\pgfsetmiterjoin%
\definecolor{currentfill}{rgb}{0.501961,0.501961,0.501961}%
\pgfsetfillcolor{currentfill}%
\pgfsetfillopacity{0.700000}%
\pgfsetlinewidth{0.000000pt}%
\definecolor{currentstroke}{rgb}{0.000000,0.000000,0.000000}%
\pgfsetstrokecolor{currentstroke}%
\pgfsetstrokeopacity{0.700000}%
\pgfsetdash{}{0pt}%
\pgfpathmoveto{\pgfqpoint{3.886541in}{0.550000in}}%
\pgfpathlineto{\pgfqpoint{3.914062in}{0.550000in}}%
\pgfpathlineto{\pgfqpoint{3.914062in}{0.837620in}}%
\pgfpathlineto{\pgfqpoint{3.886541in}{0.837620in}}%
\pgfpathlineto{\pgfqpoint{3.886541in}{0.550000in}}%
\pgfpathclose%
\pgfusepath{fill}%
\end{pgfscope}%
\begin{pgfscope}%
\pgfpathrectangle{\pgfqpoint{1.250000in}{0.550000in}}{\pgfqpoint{7.750000in}{3.850000in}}%
\pgfusepath{clip}%
\pgfsetbuttcap%
\pgfsetmiterjoin%
\definecolor{currentfill}{rgb}{0.501961,0.501961,0.501961}%
\pgfsetfillcolor{currentfill}%
\pgfsetfillopacity{0.700000}%
\pgfsetlinewidth{0.000000pt}%
\definecolor{currentstroke}{rgb}{0.000000,0.000000,0.000000}%
\pgfsetstrokecolor{currentstroke}%
\pgfsetstrokeopacity{0.700000}%
\pgfsetdash{}{0pt}%
\pgfpathmoveto{\pgfqpoint{3.914062in}{0.550000in}}%
\pgfpathlineto{\pgfqpoint{3.941584in}{0.550000in}}%
\pgfpathlineto{\pgfqpoint{3.941584in}{0.827730in}}%
\pgfpathlineto{\pgfqpoint{3.914062in}{0.827730in}}%
\pgfpathlineto{\pgfqpoint{3.914062in}{0.550000in}}%
\pgfpathclose%
\pgfusepath{fill}%
\end{pgfscope}%
\begin{pgfscope}%
\pgfpathrectangle{\pgfqpoint{1.250000in}{0.550000in}}{\pgfqpoint{7.750000in}{3.850000in}}%
\pgfusepath{clip}%
\pgfsetbuttcap%
\pgfsetmiterjoin%
\definecolor{currentfill}{rgb}{0.501961,0.501961,0.501961}%
\pgfsetfillcolor{currentfill}%
\pgfsetfillopacity{0.700000}%
\pgfsetlinewidth{0.000000pt}%
\definecolor{currentstroke}{rgb}{0.000000,0.000000,0.000000}%
\pgfsetstrokecolor{currentstroke}%
\pgfsetstrokeopacity{0.700000}%
\pgfsetdash{}{0pt}%
\pgfpathmoveto{\pgfqpoint{3.941584in}{0.550000in}}%
\pgfpathlineto{\pgfqpoint{3.969105in}{0.550000in}}%
\pgfpathlineto{\pgfqpoint{3.969105in}{0.550000in}}%
\pgfpathlineto{\pgfqpoint{3.941584in}{0.550000in}}%
\pgfpathlineto{\pgfqpoint{3.941584in}{0.550000in}}%
\pgfpathclose%
\pgfusepath{fill}%
\end{pgfscope}%
\begin{pgfscope}%
\pgfpathrectangle{\pgfqpoint{1.250000in}{0.550000in}}{\pgfqpoint{7.750000in}{3.850000in}}%
\pgfusepath{clip}%
\pgfsetbuttcap%
\pgfsetmiterjoin%
\definecolor{currentfill}{rgb}{0.501961,0.501961,0.501961}%
\pgfsetfillcolor{currentfill}%
\pgfsetfillopacity{0.700000}%
\pgfsetlinewidth{0.000000pt}%
\definecolor{currentstroke}{rgb}{0.000000,0.000000,0.000000}%
\pgfsetstrokecolor{currentstroke}%
\pgfsetstrokeopacity{0.700000}%
\pgfsetdash{}{0pt}%
\pgfpathmoveto{\pgfqpoint{3.969105in}{0.550000in}}%
\pgfpathlineto{\pgfqpoint{3.996626in}{0.550000in}}%
\pgfpathlineto{\pgfqpoint{3.996626in}{0.813092in}}%
\pgfpathlineto{\pgfqpoint{3.969105in}{0.813092in}}%
\pgfpathlineto{\pgfqpoint{3.969105in}{0.550000in}}%
\pgfpathclose%
\pgfusepath{fill}%
\end{pgfscope}%
\begin{pgfscope}%
\pgfpathrectangle{\pgfqpoint{1.250000in}{0.550000in}}{\pgfqpoint{7.750000in}{3.850000in}}%
\pgfusepath{clip}%
\pgfsetbuttcap%
\pgfsetmiterjoin%
\definecolor{currentfill}{rgb}{0.501961,0.501961,0.501961}%
\pgfsetfillcolor{currentfill}%
\pgfsetfillopacity{0.700000}%
\pgfsetlinewidth{0.000000pt}%
\definecolor{currentstroke}{rgb}{0.000000,0.000000,0.000000}%
\pgfsetstrokecolor{currentstroke}%
\pgfsetstrokeopacity{0.700000}%
\pgfsetdash{}{0pt}%
\pgfpathmoveto{\pgfqpoint{3.996626in}{0.550000in}}%
\pgfpathlineto{\pgfqpoint{4.024148in}{0.550000in}}%
\pgfpathlineto{\pgfqpoint{4.024148in}{0.815070in}}%
\pgfpathlineto{\pgfqpoint{3.996626in}{0.815070in}}%
\pgfpathlineto{\pgfqpoint{3.996626in}{0.550000in}}%
\pgfpathclose%
\pgfusepath{fill}%
\end{pgfscope}%
\begin{pgfscope}%
\pgfpathrectangle{\pgfqpoint{1.250000in}{0.550000in}}{\pgfqpoint{7.750000in}{3.850000in}}%
\pgfusepath{clip}%
\pgfsetbuttcap%
\pgfsetmiterjoin%
\definecolor{currentfill}{rgb}{0.501961,0.501961,0.501961}%
\pgfsetfillcolor{currentfill}%
\pgfsetfillopacity{0.700000}%
\pgfsetlinewidth{0.000000pt}%
\definecolor{currentstroke}{rgb}{0.000000,0.000000,0.000000}%
\pgfsetstrokecolor{currentstroke}%
\pgfsetstrokeopacity{0.700000}%
\pgfsetdash{}{0pt}%
\pgfpathmoveto{\pgfqpoint{4.024148in}{0.550000in}}%
\pgfpathlineto{\pgfqpoint{4.051669in}{0.550000in}}%
\pgfpathlineto{\pgfqpoint{4.051669in}{0.804388in}}%
\pgfpathlineto{\pgfqpoint{4.024148in}{0.804388in}}%
\pgfpathlineto{\pgfqpoint{4.024148in}{0.550000in}}%
\pgfpathclose%
\pgfusepath{fill}%
\end{pgfscope}%
\begin{pgfscope}%
\pgfpathrectangle{\pgfqpoint{1.250000in}{0.550000in}}{\pgfqpoint{7.750000in}{3.850000in}}%
\pgfusepath{clip}%
\pgfsetbuttcap%
\pgfsetmiterjoin%
\definecolor{currentfill}{rgb}{0.501961,0.501961,0.501961}%
\pgfsetfillcolor{currentfill}%
\pgfsetfillopacity{0.700000}%
\pgfsetlinewidth{0.000000pt}%
\definecolor{currentstroke}{rgb}{0.000000,0.000000,0.000000}%
\pgfsetstrokecolor{currentstroke}%
\pgfsetstrokeopacity{0.700000}%
\pgfsetdash{}{0pt}%
\pgfpathmoveto{\pgfqpoint{4.051669in}{0.550000in}}%
\pgfpathlineto{\pgfqpoint{4.079190in}{0.550000in}}%
\pgfpathlineto{\pgfqpoint{4.079190in}{0.794893in}}%
\pgfpathlineto{\pgfqpoint{4.051669in}{0.794893in}}%
\pgfpathlineto{\pgfqpoint{4.051669in}{0.550000in}}%
\pgfpathclose%
\pgfusepath{fill}%
\end{pgfscope}%
\begin{pgfscope}%
\pgfpathrectangle{\pgfqpoint{1.250000in}{0.550000in}}{\pgfqpoint{7.750000in}{3.850000in}}%
\pgfusepath{clip}%
\pgfsetbuttcap%
\pgfsetmiterjoin%
\definecolor{currentfill}{rgb}{0.501961,0.501961,0.501961}%
\pgfsetfillcolor{currentfill}%
\pgfsetfillopacity{0.700000}%
\pgfsetlinewidth{0.000000pt}%
\definecolor{currentstroke}{rgb}{0.000000,0.000000,0.000000}%
\pgfsetstrokecolor{currentstroke}%
\pgfsetstrokeopacity{0.700000}%
\pgfsetdash{}{0pt}%
\pgfpathmoveto{\pgfqpoint{4.079190in}{0.550000in}}%
\pgfpathlineto{\pgfqpoint{4.106712in}{0.550000in}}%
\pgfpathlineto{\pgfqpoint{4.106712in}{0.550000in}}%
\pgfpathlineto{\pgfqpoint{4.079190in}{0.550000in}}%
\pgfpathlineto{\pgfqpoint{4.079190in}{0.550000in}}%
\pgfpathclose%
\pgfusepath{fill}%
\end{pgfscope}%
\begin{pgfscope}%
\pgfpathrectangle{\pgfqpoint{1.250000in}{0.550000in}}{\pgfqpoint{7.750000in}{3.850000in}}%
\pgfusepath{clip}%
\pgfsetbuttcap%
\pgfsetmiterjoin%
\definecolor{currentfill}{rgb}{0.501961,0.501961,0.501961}%
\pgfsetfillcolor{currentfill}%
\pgfsetfillopacity{0.700000}%
\pgfsetlinewidth{0.000000pt}%
\definecolor{currentstroke}{rgb}{0.000000,0.000000,0.000000}%
\pgfsetstrokecolor{currentstroke}%
\pgfsetstrokeopacity{0.700000}%
\pgfsetdash{}{0pt}%
\pgfpathmoveto{\pgfqpoint{4.106712in}{0.550000in}}%
\pgfpathlineto{\pgfqpoint{4.134233in}{0.550000in}}%
\pgfpathlineto{\pgfqpoint{4.134233in}{0.799640in}}%
\pgfpathlineto{\pgfqpoint{4.106712in}{0.799640in}}%
\pgfpathlineto{\pgfqpoint{4.106712in}{0.550000in}}%
\pgfpathclose%
\pgfusepath{fill}%
\end{pgfscope}%
\begin{pgfscope}%
\pgfpathrectangle{\pgfqpoint{1.250000in}{0.550000in}}{\pgfqpoint{7.750000in}{3.850000in}}%
\pgfusepath{clip}%
\pgfsetbuttcap%
\pgfsetmiterjoin%
\definecolor{currentfill}{rgb}{0.501961,0.501961,0.501961}%
\pgfsetfillcolor{currentfill}%
\pgfsetfillopacity{0.700000}%
\pgfsetlinewidth{0.000000pt}%
\definecolor{currentstroke}{rgb}{0.000000,0.000000,0.000000}%
\pgfsetstrokecolor{currentstroke}%
\pgfsetstrokeopacity{0.700000}%
\pgfsetdash{}{0pt}%
\pgfpathmoveto{\pgfqpoint{4.134233in}{0.550000in}}%
\pgfpathlineto{\pgfqpoint{4.161754in}{0.550000in}}%
\pgfpathlineto{\pgfqpoint{4.161754in}{0.813487in}}%
\pgfpathlineto{\pgfqpoint{4.134233in}{0.813487in}}%
\pgfpathlineto{\pgfqpoint{4.134233in}{0.550000in}}%
\pgfpathclose%
\pgfusepath{fill}%
\end{pgfscope}%
\begin{pgfscope}%
\pgfpathrectangle{\pgfqpoint{1.250000in}{0.550000in}}{\pgfqpoint{7.750000in}{3.850000in}}%
\pgfusepath{clip}%
\pgfsetbuttcap%
\pgfsetmiterjoin%
\definecolor{currentfill}{rgb}{0.501961,0.501961,0.501961}%
\pgfsetfillcolor{currentfill}%
\pgfsetfillopacity{0.700000}%
\pgfsetlinewidth{0.000000pt}%
\definecolor{currentstroke}{rgb}{0.000000,0.000000,0.000000}%
\pgfsetstrokecolor{currentstroke}%
\pgfsetstrokeopacity{0.700000}%
\pgfsetdash{}{0pt}%
\pgfpathmoveto{\pgfqpoint{4.161754in}{0.550000in}}%
\pgfpathlineto{\pgfqpoint{4.189276in}{0.550000in}}%
\pgfpathlineto{\pgfqpoint{4.189276in}{0.819422in}}%
\pgfpathlineto{\pgfqpoint{4.161754in}{0.819422in}}%
\pgfpathlineto{\pgfqpoint{4.161754in}{0.550000in}}%
\pgfpathclose%
\pgfusepath{fill}%
\end{pgfscope}%
\begin{pgfscope}%
\pgfpathrectangle{\pgfqpoint{1.250000in}{0.550000in}}{\pgfqpoint{7.750000in}{3.850000in}}%
\pgfusepath{clip}%
\pgfsetbuttcap%
\pgfsetmiterjoin%
\definecolor{currentfill}{rgb}{0.501961,0.501961,0.501961}%
\pgfsetfillcolor{currentfill}%
\pgfsetfillopacity{0.700000}%
\pgfsetlinewidth{0.000000pt}%
\definecolor{currentstroke}{rgb}{0.000000,0.000000,0.000000}%
\pgfsetstrokecolor{currentstroke}%
\pgfsetstrokeopacity{0.700000}%
\pgfsetdash{}{0pt}%
\pgfpathmoveto{\pgfqpoint{4.189276in}{0.550000in}}%
\pgfpathlineto{\pgfqpoint{4.216797in}{0.550000in}}%
\pgfpathlineto{\pgfqpoint{4.216797in}{0.790937in}}%
\pgfpathlineto{\pgfqpoint{4.189276in}{0.790937in}}%
\pgfpathlineto{\pgfqpoint{4.189276in}{0.550000in}}%
\pgfpathclose%
\pgfusepath{fill}%
\end{pgfscope}%
\begin{pgfscope}%
\pgfpathrectangle{\pgfqpoint{1.250000in}{0.550000in}}{\pgfqpoint{7.750000in}{3.850000in}}%
\pgfusepath{clip}%
\pgfsetbuttcap%
\pgfsetmiterjoin%
\definecolor{currentfill}{rgb}{0.501961,0.501961,0.501961}%
\pgfsetfillcolor{currentfill}%
\pgfsetfillopacity{0.700000}%
\pgfsetlinewidth{0.000000pt}%
\definecolor{currentstroke}{rgb}{0.000000,0.000000,0.000000}%
\pgfsetstrokecolor{currentstroke}%
\pgfsetstrokeopacity{0.700000}%
\pgfsetdash{}{0pt}%
\pgfpathmoveto{\pgfqpoint{4.216797in}{0.550000in}}%
\pgfpathlineto{\pgfqpoint{4.244318in}{0.550000in}}%
\pgfpathlineto{\pgfqpoint{4.244318in}{0.550000in}}%
\pgfpathlineto{\pgfqpoint{4.216797in}{0.550000in}}%
\pgfpathlineto{\pgfqpoint{4.216797in}{0.550000in}}%
\pgfpathclose%
\pgfusepath{fill}%
\end{pgfscope}%
\begin{pgfscope}%
\pgfpathrectangle{\pgfqpoint{1.250000in}{0.550000in}}{\pgfqpoint{7.750000in}{3.850000in}}%
\pgfusepath{clip}%
\pgfsetbuttcap%
\pgfsetmiterjoin%
\definecolor{currentfill}{rgb}{0.501961,0.501961,0.501961}%
\pgfsetfillcolor{currentfill}%
\pgfsetfillopacity{0.700000}%
\pgfsetlinewidth{0.000000pt}%
\definecolor{currentstroke}{rgb}{0.000000,0.000000,0.000000}%
\pgfsetstrokecolor{currentstroke}%
\pgfsetstrokeopacity{0.700000}%
\pgfsetdash{}{0pt}%
\pgfpathmoveto{\pgfqpoint{4.244318in}{0.550000in}}%
\pgfpathlineto{\pgfqpoint{4.271839in}{0.550000in}}%
\pgfpathlineto{\pgfqpoint{4.271839in}{0.788958in}}%
\pgfpathlineto{\pgfqpoint{4.244318in}{0.788958in}}%
\pgfpathlineto{\pgfqpoint{4.244318in}{0.550000in}}%
\pgfpathclose%
\pgfusepath{fill}%
\end{pgfscope}%
\begin{pgfscope}%
\pgfpathrectangle{\pgfqpoint{1.250000in}{0.550000in}}{\pgfqpoint{7.750000in}{3.850000in}}%
\pgfusepath{clip}%
\pgfsetbuttcap%
\pgfsetmiterjoin%
\definecolor{currentfill}{rgb}{0.501961,0.501961,0.501961}%
\pgfsetfillcolor{currentfill}%
\pgfsetfillopacity{0.700000}%
\pgfsetlinewidth{0.000000pt}%
\definecolor{currentstroke}{rgb}{0.000000,0.000000,0.000000}%
\pgfsetstrokecolor{currentstroke}%
\pgfsetstrokeopacity{0.700000}%
\pgfsetdash{}{0pt}%
\pgfpathmoveto{\pgfqpoint{4.271839in}{0.550000in}}%
\pgfpathlineto{\pgfqpoint{4.299361in}{0.550000in}}%
\pgfpathlineto{\pgfqpoint{4.299361in}{0.779859in}}%
\pgfpathlineto{\pgfqpoint{4.271839in}{0.779859in}}%
\pgfpathlineto{\pgfqpoint{4.271839in}{0.550000in}}%
\pgfpathclose%
\pgfusepath{fill}%
\end{pgfscope}%
\begin{pgfscope}%
\pgfpathrectangle{\pgfqpoint{1.250000in}{0.550000in}}{\pgfqpoint{7.750000in}{3.850000in}}%
\pgfusepath{clip}%
\pgfsetbuttcap%
\pgfsetmiterjoin%
\definecolor{currentfill}{rgb}{0.501961,0.501961,0.501961}%
\pgfsetfillcolor{currentfill}%
\pgfsetfillopacity{0.700000}%
\pgfsetlinewidth{0.000000pt}%
\definecolor{currentstroke}{rgb}{0.000000,0.000000,0.000000}%
\pgfsetstrokecolor{currentstroke}%
\pgfsetstrokeopacity{0.700000}%
\pgfsetdash{}{0pt}%
\pgfpathmoveto{\pgfqpoint{4.299361in}{0.550000in}}%
\pgfpathlineto{\pgfqpoint{4.326882in}{0.550000in}}%
\pgfpathlineto{\pgfqpoint{4.326882in}{0.763243in}}%
\pgfpathlineto{\pgfqpoint{4.299361in}{0.763243in}}%
\pgfpathlineto{\pgfqpoint{4.299361in}{0.550000in}}%
\pgfpathclose%
\pgfusepath{fill}%
\end{pgfscope}%
\begin{pgfscope}%
\pgfpathrectangle{\pgfqpoint{1.250000in}{0.550000in}}{\pgfqpoint{7.750000in}{3.850000in}}%
\pgfusepath{clip}%
\pgfsetbuttcap%
\pgfsetmiterjoin%
\definecolor{currentfill}{rgb}{0.501961,0.501961,0.501961}%
\pgfsetfillcolor{currentfill}%
\pgfsetfillopacity{0.700000}%
\pgfsetlinewidth{0.000000pt}%
\definecolor{currentstroke}{rgb}{0.000000,0.000000,0.000000}%
\pgfsetstrokecolor{currentstroke}%
\pgfsetstrokeopacity{0.700000}%
\pgfsetdash{}{0pt}%
\pgfpathmoveto{\pgfqpoint{4.326882in}{0.550000in}}%
\pgfpathlineto{\pgfqpoint{4.354403in}{0.550000in}}%
\pgfpathlineto{\pgfqpoint{4.354403in}{0.756517in}}%
\pgfpathlineto{\pgfqpoint{4.326882in}{0.756517in}}%
\pgfpathlineto{\pgfqpoint{4.326882in}{0.550000in}}%
\pgfpathclose%
\pgfusepath{fill}%
\end{pgfscope}%
\begin{pgfscope}%
\pgfpathrectangle{\pgfqpoint{1.250000in}{0.550000in}}{\pgfqpoint{7.750000in}{3.850000in}}%
\pgfusepath{clip}%
\pgfsetbuttcap%
\pgfsetmiterjoin%
\definecolor{currentfill}{rgb}{0.501961,0.501961,0.501961}%
\pgfsetfillcolor{currentfill}%
\pgfsetfillopacity{0.700000}%
\pgfsetlinewidth{0.000000pt}%
\definecolor{currentstroke}{rgb}{0.000000,0.000000,0.000000}%
\pgfsetstrokecolor{currentstroke}%
\pgfsetstrokeopacity{0.700000}%
\pgfsetdash{}{0pt}%
\pgfpathmoveto{\pgfqpoint{4.354403in}{0.550000in}}%
\pgfpathlineto{\pgfqpoint{4.381925in}{0.550000in}}%
\pgfpathlineto{\pgfqpoint{4.381925in}{0.550000in}}%
\pgfpathlineto{\pgfqpoint{4.354403in}{0.550000in}}%
\pgfpathlineto{\pgfqpoint{4.354403in}{0.550000in}}%
\pgfpathclose%
\pgfusepath{fill}%
\end{pgfscope}%
\begin{pgfscope}%
\pgfpathrectangle{\pgfqpoint{1.250000in}{0.550000in}}{\pgfqpoint{7.750000in}{3.850000in}}%
\pgfusepath{clip}%
\pgfsetbuttcap%
\pgfsetmiterjoin%
\definecolor{currentfill}{rgb}{0.501961,0.501961,0.501961}%
\pgfsetfillcolor{currentfill}%
\pgfsetfillopacity{0.700000}%
\pgfsetlinewidth{0.000000pt}%
\definecolor{currentstroke}{rgb}{0.000000,0.000000,0.000000}%
\pgfsetstrokecolor{currentstroke}%
\pgfsetstrokeopacity{0.700000}%
\pgfsetdash{}{0pt}%
\pgfpathmoveto{\pgfqpoint{4.381925in}{0.550000in}}%
\pgfpathlineto{\pgfqpoint{4.409446in}{0.550000in}}%
\pgfpathlineto{\pgfqpoint{4.409446in}{0.744253in}}%
\pgfpathlineto{\pgfqpoint{4.381925in}{0.744253in}}%
\pgfpathlineto{\pgfqpoint{4.381925in}{0.550000in}}%
\pgfpathclose%
\pgfusepath{fill}%
\end{pgfscope}%
\begin{pgfscope}%
\pgfpathrectangle{\pgfqpoint{1.250000in}{0.550000in}}{\pgfqpoint{7.750000in}{3.850000in}}%
\pgfusepath{clip}%
\pgfsetbuttcap%
\pgfsetmiterjoin%
\definecolor{currentfill}{rgb}{0.501961,0.501961,0.501961}%
\pgfsetfillcolor{currentfill}%
\pgfsetfillopacity{0.700000}%
\pgfsetlinewidth{0.000000pt}%
\definecolor{currentstroke}{rgb}{0.000000,0.000000,0.000000}%
\pgfsetstrokecolor{currentstroke}%
\pgfsetstrokeopacity{0.700000}%
\pgfsetdash{}{0pt}%
\pgfpathmoveto{\pgfqpoint{4.409446in}{0.550000in}}%
\pgfpathlineto{\pgfqpoint{4.436967in}{0.550000in}}%
\pgfpathlineto{\pgfqpoint{4.436967in}{0.737923in}}%
\pgfpathlineto{\pgfqpoint{4.409446in}{0.737923in}}%
\pgfpathlineto{\pgfqpoint{4.409446in}{0.550000in}}%
\pgfpathclose%
\pgfusepath{fill}%
\end{pgfscope}%
\begin{pgfscope}%
\pgfpathrectangle{\pgfqpoint{1.250000in}{0.550000in}}{\pgfqpoint{7.750000in}{3.850000in}}%
\pgfusepath{clip}%
\pgfsetbuttcap%
\pgfsetmiterjoin%
\definecolor{currentfill}{rgb}{0.501961,0.501961,0.501961}%
\pgfsetfillcolor{currentfill}%
\pgfsetfillopacity{0.700000}%
\pgfsetlinewidth{0.000000pt}%
\definecolor{currentstroke}{rgb}{0.000000,0.000000,0.000000}%
\pgfsetstrokecolor{currentstroke}%
\pgfsetstrokeopacity{0.700000}%
\pgfsetdash{}{0pt}%
\pgfpathmoveto{\pgfqpoint{4.436967in}{0.550000in}}%
\pgfpathlineto{\pgfqpoint{4.464489in}{0.550000in}}%
\pgfpathlineto{\pgfqpoint{4.464489in}{0.731593in}}%
\pgfpathlineto{\pgfqpoint{4.436967in}{0.731593in}}%
\pgfpathlineto{\pgfqpoint{4.436967in}{0.550000in}}%
\pgfpathclose%
\pgfusepath{fill}%
\end{pgfscope}%
\begin{pgfscope}%
\pgfpathrectangle{\pgfqpoint{1.250000in}{0.550000in}}{\pgfqpoint{7.750000in}{3.850000in}}%
\pgfusepath{clip}%
\pgfsetbuttcap%
\pgfsetmiterjoin%
\definecolor{currentfill}{rgb}{0.501961,0.501961,0.501961}%
\pgfsetfillcolor{currentfill}%
\pgfsetfillopacity{0.700000}%
\pgfsetlinewidth{0.000000pt}%
\definecolor{currentstroke}{rgb}{0.000000,0.000000,0.000000}%
\pgfsetstrokecolor{currentstroke}%
\pgfsetstrokeopacity{0.700000}%
\pgfsetdash{}{0pt}%
\pgfpathmoveto{\pgfqpoint{4.464489in}{0.550000in}}%
\pgfpathlineto{\pgfqpoint{4.492010in}{0.550000in}}%
\pgfpathlineto{\pgfqpoint{4.492010in}{0.723284in}}%
\pgfpathlineto{\pgfqpoint{4.464489in}{0.723284in}}%
\pgfpathlineto{\pgfqpoint{4.464489in}{0.550000in}}%
\pgfpathclose%
\pgfusepath{fill}%
\end{pgfscope}%
\begin{pgfscope}%
\pgfpathrectangle{\pgfqpoint{1.250000in}{0.550000in}}{\pgfqpoint{7.750000in}{3.850000in}}%
\pgfusepath{clip}%
\pgfsetbuttcap%
\pgfsetmiterjoin%
\definecolor{currentfill}{rgb}{0.501961,0.501961,0.501961}%
\pgfsetfillcolor{currentfill}%
\pgfsetfillopacity{0.700000}%
\pgfsetlinewidth{0.000000pt}%
\definecolor{currentstroke}{rgb}{0.000000,0.000000,0.000000}%
\pgfsetstrokecolor{currentstroke}%
\pgfsetstrokeopacity{0.700000}%
\pgfsetdash{}{0pt}%
\pgfpathmoveto{\pgfqpoint{4.492010in}{0.550000in}}%
\pgfpathlineto{\pgfqpoint{4.519531in}{0.550000in}}%
\pgfpathlineto{\pgfqpoint{4.519531in}{0.550000in}}%
\pgfpathlineto{\pgfqpoint{4.492010in}{0.550000in}}%
\pgfpathlineto{\pgfqpoint{4.492010in}{0.550000in}}%
\pgfpathclose%
\pgfusepath{fill}%
\end{pgfscope}%
\begin{pgfscope}%
\pgfpathrectangle{\pgfqpoint{1.250000in}{0.550000in}}{\pgfqpoint{7.750000in}{3.850000in}}%
\pgfusepath{clip}%
\pgfsetbuttcap%
\pgfsetmiterjoin%
\definecolor{currentfill}{rgb}{0.501961,0.501961,0.501961}%
\pgfsetfillcolor{currentfill}%
\pgfsetfillopacity{0.700000}%
\pgfsetlinewidth{0.000000pt}%
\definecolor{currentstroke}{rgb}{0.000000,0.000000,0.000000}%
\pgfsetstrokecolor{currentstroke}%
\pgfsetstrokeopacity{0.700000}%
\pgfsetdash{}{0pt}%
\pgfpathmoveto{\pgfqpoint{4.519531in}{0.550000in}}%
\pgfpathlineto{\pgfqpoint{4.547053in}{0.550000in}}%
\pgfpathlineto{\pgfqpoint{4.547053in}{0.726449in}}%
\pgfpathlineto{\pgfqpoint{4.519531in}{0.726449in}}%
\pgfpathlineto{\pgfqpoint{4.519531in}{0.550000in}}%
\pgfpathclose%
\pgfusepath{fill}%
\end{pgfscope}%
\begin{pgfscope}%
\pgfpathrectangle{\pgfqpoint{1.250000in}{0.550000in}}{\pgfqpoint{7.750000in}{3.850000in}}%
\pgfusepath{clip}%
\pgfsetbuttcap%
\pgfsetmiterjoin%
\definecolor{currentfill}{rgb}{0.501961,0.501961,0.501961}%
\pgfsetfillcolor{currentfill}%
\pgfsetfillopacity{0.700000}%
\pgfsetlinewidth{0.000000pt}%
\definecolor{currentstroke}{rgb}{0.000000,0.000000,0.000000}%
\pgfsetstrokecolor{currentstroke}%
\pgfsetstrokeopacity{0.700000}%
\pgfsetdash{}{0pt}%
\pgfpathmoveto{\pgfqpoint{4.547053in}{0.550000in}}%
\pgfpathlineto{\pgfqpoint{4.574574in}{0.550000in}}%
\pgfpathlineto{\pgfqpoint{4.574574in}{0.727636in}}%
\pgfpathlineto{\pgfqpoint{4.547053in}{0.727636in}}%
\pgfpathlineto{\pgfqpoint{4.547053in}{0.550000in}}%
\pgfpathclose%
\pgfusepath{fill}%
\end{pgfscope}%
\begin{pgfscope}%
\pgfpathrectangle{\pgfqpoint{1.250000in}{0.550000in}}{\pgfqpoint{7.750000in}{3.850000in}}%
\pgfusepath{clip}%
\pgfsetbuttcap%
\pgfsetmiterjoin%
\definecolor{currentfill}{rgb}{0.501961,0.501961,0.501961}%
\pgfsetfillcolor{currentfill}%
\pgfsetfillopacity{0.700000}%
\pgfsetlinewidth{0.000000pt}%
\definecolor{currentstroke}{rgb}{0.000000,0.000000,0.000000}%
\pgfsetstrokecolor{currentstroke}%
\pgfsetstrokeopacity{0.700000}%
\pgfsetdash{}{0pt}%
\pgfpathmoveto{\pgfqpoint{4.574574in}{0.550000in}}%
\pgfpathlineto{\pgfqpoint{4.602095in}{0.550000in}}%
\pgfpathlineto{\pgfqpoint{4.602095in}{0.726054in}}%
\pgfpathlineto{\pgfqpoint{4.574574in}{0.726054in}}%
\pgfpathlineto{\pgfqpoint{4.574574in}{0.550000in}}%
\pgfpathclose%
\pgfusepath{fill}%
\end{pgfscope}%
\begin{pgfscope}%
\pgfpathrectangle{\pgfqpoint{1.250000in}{0.550000in}}{\pgfqpoint{7.750000in}{3.850000in}}%
\pgfusepath{clip}%
\pgfsetbuttcap%
\pgfsetmiterjoin%
\definecolor{currentfill}{rgb}{0.501961,0.501961,0.501961}%
\pgfsetfillcolor{currentfill}%
\pgfsetfillopacity{0.700000}%
\pgfsetlinewidth{0.000000pt}%
\definecolor{currentstroke}{rgb}{0.000000,0.000000,0.000000}%
\pgfsetstrokecolor{currentstroke}%
\pgfsetstrokeopacity{0.700000}%
\pgfsetdash{}{0pt}%
\pgfpathmoveto{\pgfqpoint{4.602095in}{0.550000in}}%
\pgfpathlineto{\pgfqpoint{4.629616in}{0.550000in}}%
\pgfpathlineto{\pgfqpoint{4.629616in}{0.729219in}}%
\pgfpathlineto{\pgfqpoint{4.602095in}{0.729219in}}%
\pgfpathlineto{\pgfqpoint{4.602095in}{0.550000in}}%
\pgfpathclose%
\pgfusepath{fill}%
\end{pgfscope}%
\begin{pgfscope}%
\pgfpathrectangle{\pgfqpoint{1.250000in}{0.550000in}}{\pgfqpoint{7.750000in}{3.850000in}}%
\pgfusepath{clip}%
\pgfsetbuttcap%
\pgfsetmiterjoin%
\definecolor{currentfill}{rgb}{0.501961,0.501961,0.501961}%
\pgfsetfillcolor{currentfill}%
\pgfsetfillopacity{0.700000}%
\pgfsetlinewidth{0.000000pt}%
\definecolor{currentstroke}{rgb}{0.000000,0.000000,0.000000}%
\pgfsetstrokecolor{currentstroke}%
\pgfsetstrokeopacity{0.700000}%
\pgfsetdash{}{0pt}%
\pgfpathmoveto{\pgfqpoint{4.629616in}{0.550000in}}%
\pgfpathlineto{\pgfqpoint{4.657138in}{0.550000in}}%
\pgfpathlineto{\pgfqpoint{4.657138in}{0.550000in}}%
\pgfpathlineto{\pgfqpoint{4.629616in}{0.550000in}}%
\pgfpathlineto{\pgfqpoint{4.629616in}{0.550000in}}%
\pgfpathclose%
\pgfusepath{fill}%
\end{pgfscope}%
\begin{pgfscope}%
\pgfpathrectangle{\pgfqpoint{1.250000in}{0.550000in}}{\pgfqpoint{7.750000in}{3.850000in}}%
\pgfusepath{clip}%
\pgfsetbuttcap%
\pgfsetmiterjoin%
\definecolor{currentfill}{rgb}{0.501961,0.501961,0.501961}%
\pgfsetfillcolor{currentfill}%
\pgfsetfillopacity{0.700000}%
\pgfsetlinewidth{0.000000pt}%
\definecolor{currentstroke}{rgb}{0.000000,0.000000,0.000000}%
\pgfsetstrokecolor{currentstroke}%
\pgfsetstrokeopacity{0.700000}%
\pgfsetdash{}{0pt}%
\pgfpathmoveto{\pgfqpoint{4.657138in}{0.550000in}}%
\pgfpathlineto{\pgfqpoint{4.684659in}{0.550000in}}%
\pgfpathlineto{\pgfqpoint{4.684659in}{0.726845in}}%
\pgfpathlineto{\pgfqpoint{4.657138in}{0.726845in}}%
\pgfpathlineto{\pgfqpoint{4.657138in}{0.550000in}}%
\pgfpathclose%
\pgfusepath{fill}%
\end{pgfscope}%
\begin{pgfscope}%
\pgfpathrectangle{\pgfqpoint{1.250000in}{0.550000in}}{\pgfqpoint{7.750000in}{3.850000in}}%
\pgfusepath{clip}%
\pgfsetbuttcap%
\pgfsetmiterjoin%
\definecolor{currentfill}{rgb}{0.501961,0.501961,0.501961}%
\pgfsetfillcolor{currentfill}%
\pgfsetfillopacity{0.700000}%
\pgfsetlinewidth{0.000000pt}%
\definecolor{currentstroke}{rgb}{0.000000,0.000000,0.000000}%
\pgfsetstrokecolor{currentstroke}%
\pgfsetstrokeopacity{0.700000}%
\pgfsetdash{}{0pt}%
\pgfpathmoveto{\pgfqpoint{4.684659in}{0.550000in}}%
\pgfpathlineto{\pgfqpoint{4.712180in}{0.550000in}}%
\pgfpathlineto{\pgfqpoint{4.712180in}{0.707855in}}%
\pgfpathlineto{\pgfqpoint{4.684659in}{0.707855in}}%
\pgfpathlineto{\pgfqpoint{4.684659in}{0.550000in}}%
\pgfpathclose%
\pgfusepath{fill}%
\end{pgfscope}%
\begin{pgfscope}%
\pgfpathrectangle{\pgfqpoint{1.250000in}{0.550000in}}{\pgfqpoint{7.750000in}{3.850000in}}%
\pgfusepath{clip}%
\pgfsetbuttcap%
\pgfsetmiterjoin%
\definecolor{currentfill}{rgb}{0.501961,0.501961,0.501961}%
\pgfsetfillcolor{currentfill}%
\pgfsetfillopacity{0.700000}%
\pgfsetlinewidth{0.000000pt}%
\definecolor{currentstroke}{rgb}{0.000000,0.000000,0.000000}%
\pgfsetstrokecolor{currentstroke}%
\pgfsetstrokeopacity{0.700000}%
\pgfsetdash{}{0pt}%
\pgfpathmoveto{\pgfqpoint{4.712180in}{0.550000in}}%
\pgfpathlineto{\pgfqpoint{4.739702in}{0.550000in}}%
\pgfpathlineto{\pgfqpoint{4.739702in}{0.705877in}}%
\pgfpathlineto{\pgfqpoint{4.712180in}{0.705877in}}%
\pgfpathlineto{\pgfqpoint{4.712180in}{0.550000in}}%
\pgfpathclose%
\pgfusepath{fill}%
\end{pgfscope}%
\begin{pgfscope}%
\pgfpathrectangle{\pgfqpoint{1.250000in}{0.550000in}}{\pgfqpoint{7.750000in}{3.850000in}}%
\pgfusepath{clip}%
\pgfsetbuttcap%
\pgfsetmiterjoin%
\definecolor{currentfill}{rgb}{0.501961,0.501961,0.501961}%
\pgfsetfillcolor{currentfill}%
\pgfsetfillopacity{0.700000}%
\pgfsetlinewidth{0.000000pt}%
\definecolor{currentstroke}{rgb}{0.000000,0.000000,0.000000}%
\pgfsetstrokecolor{currentstroke}%
\pgfsetstrokeopacity{0.700000}%
\pgfsetdash{}{0pt}%
\pgfpathmoveto{\pgfqpoint{4.739702in}{0.550000in}}%
\pgfpathlineto{\pgfqpoint{4.767223in}{0.550000in}}%
\pgfpathlineto{\pgfqpoint{4.767223in}{0.709833in}}%
\pgfpathlineto{\pgfqpoint{4.739702in}{0.709833in}}%
\pgfpathlineto{\pgfqpoint{4.739702in}{0.550000in}}%
\pgfpathclose%
\pgfusepath{fill}%
\end{pgfscope}%
\begin{pgfscope}%
\pgfpathrectangle{\pgfqpoint{1.250000in}{0.550000in}}{\pgfqpoint{7.750000in}{3.850000in}}%
\pgfusepath{clip}%
\pgfsetbuttcap%
\pgfsetmiterjoin%
\definecolor{currentfill}{rgb}{0.501961,0.501961,0.501961}%
\pgfsetfillcolor{currentfill}%
\pgfsetfillopacity{0.700000}%
\pgfsetlinewidth{0.000000pt}%
\definecolor{currentstroke}{rgb}{0.000000,0.000000,0.000000}%
\pgfsetstrokecolor{currentstroke}%
\pgfsetstrokeopacity{0.700000}%
\pgfsetdash{}{0pt}%
\pgfpathmoveto{\pgfqpoint{4.767223in}{0.550000in}}%
\pgfpathlineto{\pgfqpoint{4.794744in}{0.550000in}}%
\pgfpathlineto{\pgfqpoint{4.794744in}{0.550000in}}%
\pgfpathlineto{\pgfqpoint{4.767223in}{0.550000in}}%
\pgfpathlineto{\pgfqpoint{4.767223in}{0.550000in}}%
\pgfpathclose%
\pgfusepath{fill}%
\end{pgfscope}%
\begin{pgfscope}%
\pgfpathrectangle{\pgfqpoint{1.250000in}{0.550000in}}{\pgfqpoint{7.750000in}{3.850000in}}%
\pgfusepath{clip}%
\pgfsetbuttcap%
\pgfsetmiterjoin%
\definecolor{currentfill}{rgb}{0.501961,0.501961,0.501961}%
\pgfsetfillcolor{currentfill}%
\pgfsetfillopacity{0.700000}%
\pgfsetlinewidth{0.000000pt}%
\definecolor{currentstroke}{rgb}{0.000000,0.000000,0.000000}%
\pgfsetstrokecolor{currentstroke}%
\pgfsetstrokeopacity{0.700000}%
\pgfsetdash{}{0pt}%
\pgfpathmoveto{\pgfqpoint{4.794744in}{0.550000in}}%
\pgfpathlineto{\pgfqpoint{4.822266in}{0.550000in}}%
\pgfpathlineto{\pgfqpoint{4.822266in}{0.699942in}}%
\pgfpathlineto{\pgfqpoint{4.794744in}{0.699942in}}%
\pgfpathlineto{\pgfqpoint{4.794744in}{0.550000in}}%
\pgfpathclose%
\pgfusepath{fill}%
\end{pgfscope}%
\begin{pgfscope}%
\pgfpathrectangle{\pgfqpoint{1.250000in}{0.550000in}}{\pgfqpoint{7.750000in}{3.850000in}}%
\pgfusepath{clip}%
\pgfsetbuttcap%
\pgfsetmiterjoin%
\definecolor{currentfill}{rgb}{0.501961,0.501961,0.501961}%
\pgfsetfillcolor{currentfill}%
\pgfsetfillopacity{0.700000}%
\pgfsetlinewidth{0.000000pt}%
\definecolor{currentstroke}{rgb}{0.000000,0.000000,0.000000}%
\pgfsetstrokecolor{currentstroke}%
\pgfsetstrokeopacity{0.700000}%
\pgfsetdash{}{0pt}%
\pgfpathmoveto{\pgfqpoint{4.822266in}{0.550000in}}%
\pgfpathlineto{\pgfqpoint{4.849787in}{0.550000in}}%
\pgfpathlineto{\pgfqpoint{4.849787in}{0.689261in}}%
\pgfpathlineto{\pgfqpoint{4.822266in}{0.689261in}}%
\pgfpathlineto{\pgfqpoint{4.822266in}{0.550000in}}%
\pgfpathclose%
\pgfusepath{fill}%
\end{pgfscope}%
\begin{pgfscope}%
\pgfpathrectangle{\pgfqpoint{1.250000in}{0.550000in}}{\pgfqpoint{7.750000in}{3.850000in}}%
\pgfusepath{clip}%
\pgfsetbuttcap%
\pgfsetmiterjoin%
\definecolor{currentfill}{rgb}{0.501961,0.501961,0.501961}%
\pgfsetfillcolor{currentfill}%
\pgfsetfillopacity{0.700000}%
\pgfsetlinewidth{0.000000pt}%
\definecolor{currentstroke}{rgb}{0.000000,0.000000,0.000000}%
\pgfsetstrokecolor{currentstroke}%
\pgfsetstrokeopacity{0.700000}%
\pgfsetdash{}{0pt}%
\pgfpathmoveto{\pgfqpoint{4.849787in}{0.550000in}}%
\pgfpathlineto{\pgfqpoint{4.877308in}{0.550000in}}%
\pgfpathlineto{\pgfqpoint{4.877308in}{0.690843in}}%
\pgfpathlineto{\pgfqpoint{4.849787in}{0.690843in}}%
\pgfpathlineto{\pgfqpoint{4.849787in}{0.550000in}}%
\pgfpathclose%
\pgfusepath{fill}%
\end{pgfscope}%
\begin{pgfscope}%
\pgfpathrectangle{\pgfqpoint{1.250000in}{0.550000in}}{\pgfqpoint{7.750000in}{3.850000in}}%
\pgfusepath{clip}%
\pgfsetbuttcap%
\pgfsetmiterjoin%
\definecolor{currentfill}{rgb}{0.501961,0.501961,0.501961}%
\pgfsetfillcolor{currentfill}%
\pgfsetfillopacity{0.700000}%
\pgfsetlinewidth{0.000000pt}%
\definecolor{currentstroke}{rgb}{0.000000,0.000000,0.000000}%
\pgfsetstrokecolor{currentstroke}%
\pgfsetstrokeopacity{0.700000}%
\pgfsetdash{}{0pt}%
\pgfpathmoveto{\pgfqpoint{4.877308in}{0.550000in}}%
\pgfpathlineto{\pgfqpoint{4.904830in}{0.550000in}}%
\pgfpathlineto{\pgfqpoint{4.904830in}{0.550000in}}%
\pgfpathlineto{\pgfqpoint{4.877308in}{0.550000in}}%
\pgfpathlineto{\pgfqpoint{4.877308in}{0.550000in}}%
\pgfpathclose%
\pgfusepath{fill}%
\end{pgfscope}%
\begin{pgfscope}%
\pgfpathrectangle{\pgfqpoint{1.250000in}{0.550000in}}{\pgfqpoint{7.750000in}{3.850000in}}%
\pgfusepath{clip}%
\pgfsetbuttcap%
\pgfsetmiterjoin%
\definecolor{currentfill}{rgb}{0.501961,0.501961,0.501961}%
\pgfsetfillcolor{currentfill}%
\pgfsetfillopacity{0.700000}%
\pgfsetlinewidth{0.000000pt}%
\definecolor{currentstroke}{rgb}{0.000000,0.000000,0.000000}%
\pgfsetstrokecolor{currentstroke}%
\pgfsetstrokeopacity{0.700000}%
\pgfsetdash{}{0pt}%
\pgfpathmoveto{\pgfqpoint{4.904830in}{0.550000in}}%
\pgfpathlineto{\pgfqpoint{4.932351in}{0.550000in}}%
\pgfpathlineto{\pgfqpoint{4.932351in}{0.663149in}}%
\pgfpathlineto{\pgfqpoint{4.904830in}{0.663149in}}%
\pgfpathlineto{\pgfqpoint{4.904830in}{0.550000in}}%
\pgfpathclose%
\pgfusepath{fill}%
\end{pgfscope}%
\begin{pgfscope}%
\pgfpathrectangle{\pgfqpoint{1.250000in}{0.550000in}}{\pgfqpoint{7.750000in}{3.850000in}}%
\pgfusepath{clip}%
\pgfsetbuttcap%
\pgfsetmiterjoin%
\definecolor{currentfill}{rgb}{0.501961,0.501961,0.501961}%
\pgfsetfillcolor{currentfill}%
\pgfsetfillopacity{0.700000}%
\pgfsetlinewidth{0.000000pt}%
\definecolor{currentstroke}{rgb}{0.000000,0.000000,0.000000}%
\pgfsetstrokecolor{currentstroke}%
\pgfsetstrokeopacity{0.700000}%
\pgfsetdash{}{0pt}%
\pgfpathmoveto{\pgfqpoint{4.932351in}{0.550000in}}%
\pgfpathlineto{\pgfqpoint{4.959872in}{0.550000in}}%
\pgfpathlineto{\pgfqpoint{4.959872in}{0.674622in}}%
\pgfpathlineto{\pgfqpoint{4.932351in}{0.674622in}}%
\pgfpathlineto{\pgfqpoint{4.932351in}{0.550000in}}%
\pgfpathclose%
\pgfusepath{fill}%
\end{pgfscope}%
\begin{pgfscope}%
\pgfpathrectangle{\pgfqpoint{1.250000in}{0.550000in}}{\pgfqpoint{7.750000in}{3.850000in}}%
\pgfusepath{clip}%
\pgfsetbuttcap%
\pgfsetmiterjoin%
\definecolor{currentfill}{rgb}{0.501961,0.501961,0.501961}%
\pgfsetfillcolor{currentfill}%
\pgfsetfillopacity{0.700000}%
\pgfsetlinewidth{0.000000pt}%
\definecolor{currentstroke}{rgb}{0.000000,0.000000,0.000000}%
\pgfsetstrokecolor{currentstroke}%
\pgfsetstrokeopacity{0.700000}%
\pgfsetdash{}{0pt}%
\pgfpathmoveto{\pgfqpoint{4.959872in}{0.550000in}}%
\pgfpathlineto{\pgfqpoint{4.987393in}{0.550000in}}%
\pgfpathlineto{\pgfqpoint{4.987393in}{0.670270in}}%
\pgfpathlineto{\pgfqpoint{4.959872in}{0.670270in}}%
\pgfpathlineto{\pgfqpoint{4.959872in}{0.550000in}}%
\pgfpathclose%
\pgfusepath{fill}%
\end{pgfscope}%
\begin{pgfscope}%
\pgfpathrectangle{\pgfqpoint{1.250000in}{0.550000in}}{\pgfqpoint{7.750000in}{3.850000in}}%
\pgfusepath{clip}%
\pgfsetbuttcap%
\pgfsetmiterjoin%
\definecolor{currentfill}{rgb}{0.501961,0.501961,0.501961}%
\pgfsetfillcolor{currentfill}%
\pgfsetfillopacity{0.700000}%
\pgfsetlinewidth{0.000000pt}%
\definecolor{currentstroke}{rgb}{0.000000,0.000000,0.000000}%
\pgfsetstrokecolor{currentstroke}%
\pgfsetstrokeopacity{0.700000}%
\pgfsetdash{}{0pt}%
\pgfpathmoveto{\pgfqpoint{4.987393in}{0.550000in}}%
\pgfpathlineto{\pgfqpoint{5.014915in}{0.550000in}}%
\pgfpathlineto{\pgfqpoint{5.014915in}{0.688865in}}%
\pgfpathlineto{\pgfqpoint{4.987393in}{0.688865in}}%
\pgfpathlineto{\pgfqpoint{4.987393in}{0.550000in}}%
\pgfpathclose%
\pgfusepath{fill}%
\end{pgfscope}%
\begin{pgfscope}%
\pgfpathrectangle{\pgfqpoint{1.250000in}{0.550000in}}{\pgfqpoint{7.750000in}{3.850000in}}%
\pgfusepath{clip}%
\pgfsetbuttcap%
\pgfsetmiterjoin%
\definecolor{currentfill}{rgb}{0.501961,0.501961,0.501961}%
\pgfsetfillcolor{currentfill}%
\pgfsetfillopacity{0.700000}%
\pgfsetlinewidth{0.000000pt}%
\definecolor{currentstroke}{rgb}{0.000000,0.000000,0.000000}%
\pgfsetstrokecolor{currentstroke}%
\pgfsetstrokeopacity{0.700000}%
\pgfsetdash{}{0pt}%
\pgfpathmoveto{\pgfqpoint{5.014915in}{0.550000in}}%
\pgfpathlineto{\pgfqpoint{5.042436in}{0.550000in}}%
\pgfpathlineto{\pgfqpoint{5.042436in}{0.550000in}}%
\pgfpathlineto{\pgfqpoint{5.014915in}{0.550000in}}%
\pgfpathlineto{\pgfqpoint{5.014915in}{0.550000in}}%
\pgfpathclose%
\pgfusepath{fill}%
\end{pgfscope}%
\begin{pgfscope}%
\pgfpathrectangle{\pgfqpoint{1.250000in}{0.550000in}}{\pgfqpoint{7.750000in}{3.850000in}}%
\pgfusepath{clip}%
\pgfsetbuttcap%
\pgfsetmiterjoin%
\definecolor{currentfill}{rgb}{0.501961,0.501961,0.501961}%
\pgfsetfillcolor{currentfill}%
\pgfsetfillopacity{0.700000}%
\pgfsetlinewidth{0.000000pt}%
\definecolor{currentstroke}{rgb}{0.000000,0.000000,0.000000}%
\pgfsetstrokecolor{currentstroke}%
\pgfsetstrokeopacity{0.700000}%
\pgfsetdash{}{0pt}%
\pgfpathmoveto{\pgfqpoint{5.042436in}{0.550000in}}%
\pgfpathlineto{\pgfqpoint{5.069957in}{0.550000in}}%
\pgfpathlineto{\pgfqpoint{5.069957in}{0.677787in}}%
\pgfpathlineto{\pgfqpoint{5.042436in}{0.677787in}}%
\pgfpathlineto{\pgfqpoint{5.042436in}{0.550000in}}%
\pgfpathclose%
\pgfusepath{fill}%
\end{pgfscope}%
\begin{pgfscope}%
\pgfpathrectangle{\pgfqpoint{1.250000in}{0.550000in}}{\pgfqpoint{7.750000in}{3.850000in}}%
\pgfusepath{clip}%
\pgfsetbuttcap%
\pgfsetmiterjoin%
\definecolor{currentfill}{rgb}{0.501961,0.501961,0.501961}%
\pgfsetfillcolor{currentfill}%
\pgfsetfillopacity{0.700000}%
\pgfsetlinewidth{0.000000pt}%
\definecolor{currentstroke}{rgb}{0.000000,0.000000,0.000000}%
\pgfsetstrokecolor{currentstroke}%
\pgfsetstrokeopacity{0.700000}%
\pgfsetdash{}{0pt}%
\pgfpathmoveto{\pgfqpoint{5.069957in}{0.550000in}}%
\pgfpathlineto{\pgfqpoint{5.097479in}{0.550000in}}%
\pgfpathlineto{\pgfqpoint{5.097479in}{0.670270in}}%
\pgfpathlineto{\pgfqpoint{5.069957in}{0.670270in}}%
\pgfpathlineto{\pgfqpoint{5.069957in}{0.550000in}}%
\pgfpathclose%
\pgfusepath{fill}%
\end{pgfscope}%
\begin{pgfscope}%
\pgfpathrectangle{\pgfqpoint{1.250000in}{0.550000in}}{\pgfqpoint{7.750000in}{3.850000in}}%
\pgfusepath{clip}%
\pgfsetbuttcap%
\pgfsetmiterjoin%
\definecolor{currentfill}{rgb}{0.501961,0.501961,0.501961}%
\pgfsetfillcolor{currentfill}%
\pgfsetfillopacity{0.700000}%
\pgfsetlinewidth{0.000000pt}%
\definecolor{currentstroke}{rgb}{0.000000,0.000000,0.000000}%
\pgfsetstrokecolor{currentstroke}%
\pgfsetstrokeopacity{0.700000}%
\pgfsetdash{}{0pt}%
\pgfpathmoveto{\pgfqpoint{5.097479in}{0.550000in}}%
\pgfpathlineto{\pgfqpoint{5.125000in}{0.550000in}}%
\pgfpathlineto{\pgfqpoint{5.125000in}{0.658402in}}%
\pgfpathlineto{\pgfqpoint{5.097479in}{0.658402in}}%
\pgfpathlineto{\pgfqpoint{5.097479in}{0.550000in}}%
\pgfpathclose%
\pgfusepath{fill}%
\end{pgfscope}%
\begin{pgfscope}%
\pgfpathrectangle{\pgfqpoint{1.250000in}{0.550000in}}{\pgfqpoint{7.750000in}{3.850000in}}%
\pgfusepath{clip}%
\pgfsetbuttcap%
\pgfsetmiterjoin%
\definecolor{currentfill}{rgb}{0.501961,0.501961,0.501961}%
\pgfsetfillcolor{currentfill}%
\pgfsetfillopacity{0.700000}%
\pgfsetlinewidth{0.000000pt}%
\definecolor{currentstroke}{rgb}{0.000000,0.000000,0.000000}%
\pgfsetstrokecolor{currentstroke}%
\pgfsetstrokeopacity{0.700000}%
\pgfsetdash{}{0pt}%
\pgfpathmoveto{\pgfqpoint{5.125000in}{0.550000in}}%
\pgfpathlineto{\pgfqpoint{5.152521in}{0.550000in}}%
\pgfpathlineto{\pgfqpoint{5.152521in}{0.647324in}}%
\pgfpathlineto{\pgfqpoint{5.125000in}{0.647324in}}%
\pgfpathlineto{\pgfqpoint{5.125000in}{0.550000in}}%
\pgfpathclose%
\pgfusepath{fill}%
\end{pgfscope}%
\begin{pgfscope}%
\pgfpathrectangle{\pgfqpoint{1.250000in}{0.550000in}}{\pgfqpoint{7.750000in}{3.850000in}}%
\pgfusepath{clip}%
\pgfsetbuttcap%
\pgfsetmiterjoin%
\definecolor{currentfill}{rgb}{0.501961,0.501961,0.501961}%
\pgfsetfillcolor{currentfill}%
\pgfsetfillopacity{0.700000}%
\pgfsetlinewidth{0.000000pt}%
\definecolor{currentstroke}{rgb}{0.000000,0.000000,0.000000}%
\pgfsetstrokecolor{currentstroke}%
\pgfsetstrokeopacity{0.700000}%
\pgfsetdash{}{0pt}%
\pgfpathmoveto{\pgfqpoint{5.152521in}{0.550000in}}%
\pgfpathlineto{\pgfqpoint{5.180043in}{0.550000in}}%
\pgfpathlineto{\pgfqpoint{5.180043in}{0.550000in}}%
\pgfpathlineto{\pgfqpoint{5.152521in}{0.550000in}}%
\pgfpathlineto{\pgfqpoint{5.152521in}{0.550000in}}%
\pgfpathclose%
\pgfusepath{fill}%
\end{pgfscope}%
\begin{pgfscope}%
\pgfpathrectangle{\pgfqpoint{1.250000in}{0.550000in}}{\pgfqpoint{7.750000in}{3.850000in}}%
\pgfusepath{clip}%
\pgfsetbuttcap%
\pgfsetmiterjoin%
\definecolor{currentfill}{rgb}{0.501961,0.501961,0.501961}%
\pgfsetfillcolor{currentfill}%
\pgfsetfillopacity{0.700000}%
\pgfsetlinewidth{0.000000pt}%
\definecolor{currentstroke}{rgb}{0.000000,0.000000,0.000000}%
\pgfsetstrokecolor{currentstroke}%
\pgfsetstrokeopacity{0.700000}%
\pgfsetdash{}{0pt}%
\pgfpathmoveto{\pgfqpoint{5.180043in}{0.550000in}}%
\pgfpathlineto{\pgfqpoint{5.207564in}{0.550000in}}%
\pgfpathlineto{\pgfqpoint{5.207564in}{0.658006in}}%
\pgfpathlineto{\pgfqpoint{5.180043in}{0.658006in}}%
\pgfpathlineto{\pgfqpoint{5.180043in}{0.550000in}}%
\pgfpathclose%
\pgfusepath{fill}%
\end{pgfscope}%
\begin{pgfscope}%
\pgfpathrectangle{\pgfqpoint{1.250000in}{0.550000in}}{\pgfqpoint{7.750000in}{3.850000in}}%
\pgfusepath{clip}%
\pgfsetbuttcap%
\pgfsetmiterjoin%
\definecolor{currentfill}{rgb}{0.501961,0.501961,0.501961}%
\pgfsetfillcolor{currentfill}%
\pgfsetfillopacity{0.700000}%
\pgfsetlinewidth{0.000000pt}%
\definecolor{currentstroke}{rgb}{0.000000,0.000000,0.000000}%
\pgfsetstrokecolor{currentstroke}%
\pgfsetstrokeopacity{0.700000}%
\pgfsetdash{}{0pt}%
\pgfpathmoveto{\pgfqpoint{5.207564in}{0.550000in}}%
\pgfpathlineto{\pgfqpoint{5.235085in}{0.550000in}}%
\pgfpathlineto{\pgfqpoint{5.235085in}{0.681744in}}%
\pgfpathlineto{\pgfqpoint{5.207564in}{0.681744in}}%
\pgfpathlineto{\pgfqpoint{5.207564in}{0.550000in}}%
\pgfpathclose%
\pgfusepath{fill}%
\end{pgfscope}%
\begin{pgfscope}%
\pgfpathrectangle{\pgfqpoint{1.250000in}{0.550000in}}{\pgfqpoint{7.750000in}{3.850000in}}%
\pgfusepath{clip}%
\pgfsetbuttcap%
\pgfsetmiterjoin%
\definecolor{currentfill}{rgb}{0.501961,0.501961,0.501961}%
\pgfsetfillcolor{currentfill}%
\pgfsetfillopacity{0.700000}%
\pgfsetlinewidth{0.000000pt}%
\definecolor{currentstroke}{rgb}{0.000000,0.000000,0.000000}%
\pgfsetstrokecolor{currentstroke}%
\pgfsetstrokeopacity{0.700000}%
\pgfsetdash{}{0pt}%
\pgfpathmoveto{\pgfqpoint{5.235085in}{0.550000in}}%
\pgfpathlineto{\pgfqpoint{5.262607in}{0.550000in}}%
\pgfpathlineto{\pgfqpoint{5.262607in}{0.665523in}}%
\pgfpathlineto{\pgfqpoint{5.235085in}{0.665523in}}%
\pgfpathlineto{\pgfqpoint{5.235085in}{0.550000in}}%
\pgfpathclose%
\pgfusepath{fill}%
\end{pgfscope}%
\begin{pgfscope}%
\pgfpathrectangle{\pgfqpoint{1.250000in}{0.550000in}}{\pgfqpoint{7.750000in}{3.850000in}}%
\pgfusepath{clip}%
\pgfsetbuttcap%
\pgfsetmiterjoin%
\definecolor{currentfill}{rgb}{0.501961,0.501961,0.501961}%
\pgfsetfillcolor{currentfill}%
\pgfsetfillopacity{0.700000}%
\pgfsetlinewidth{0.000000pt}%
\definecolor{currentstroke}{rgb}{0.000000,0.000000,0.000000}%
\pgfsetstrokecolor{currentstroke}%
\pgfsetstrokeopacity{0.700000}%
\pgfsetdash{}{0pt}%
\pgfpathmoveto{\pgfqpoint{5.262607in}{0.550000in}}%
\pgfpathlineto{\pgfqpoint{5.290128in}{0.550000in}}%
\pgfpathlineto{\pgfqpoint{5.290128in}{0.654050in}}%
\pgfpathlineto{\pgfqpoint{5.262607in}{0.654050in}}%
\pgfpathlineto{\pgfqpoint{5.262607in}{0.550000in}}%
\pgfpathclose%
\pgfusepath{fill}%
\end{pgfscope}%
\begin{pgfscope}%
\pgfpathrectangle{\pgfqpoint{1.250000in}{0.550000in}}{\pgfqpoint{7.750000in}{3.850000in}}%
\pgfusepath{clip}%
\pgfsetbuttcap%
\pgfsetmiterjoin%
\definecolor{currentfill}{rgb}{0.501961,0.501961,0.501961}%
\pgfsetfillcolor{currentfill}%
\pgfsetfillopacity{0.700000}%
\pgfsetlinewidth{0.000000pt}%
\definecolor{currentstroke}{rgb}{0.000000,0.000000,0.000000}%
\pgfsetstrokecolor{currentstroke}%
\pgfsetstrokeopacity{0.700000}%
\pgfsetdash{}{0pt}%
\pgfpathmoveto{\pgfqpoint{5.290128in}{0.550000in}}%
\pgfpathlineto{\pgfqpoint{5.317649in}{0.550000in}}%
\pgfpathlineto{\pgfqpoint{5.317649in}{0.550000in}}%
\pgfpathlineto{\pgfqpoint{5.290128in}{0.550000in}}%
\pgfpathlineto{\pgfqpoint{5.290128in}{0.550000in}}%
\pgfpathclose%
\pgfusepath{fill}%
\end{pgfscope}%
\begin{pgfscope}%
\pgfpathrectangle{\pgfqpoint{1.250000in}{0.550000in}}{\pgfqpoint{7.750000in}{3.850000in}}%
\pgfusepath{clip}%
\pgfsetbuttcap%
\pgfsetmiterjoin%
\definecolor{currentfill}{rgb}{0.501961,0.501961,0.501961}%
\pgfsetfillcolor{currentfill}%
\pgfsetfillopacity{0.700000}%
\pgfsetlinewidth{0.000000pt}%
\definecolor{currentstroke}{rgb}{0.000000,0.000000,0.000000}%
\pgfsetstrokecolor{currentstroke}%
\pgfsetstrokeopacity{0.700000}%
\pgfsetdash{}{0pt}%
\pgfpathmoveto{\pgfqpoint{5.317649in}{0.550000in}}%
\pgfpathlineto{\pgfqpoint{5.345170in}{0.550000in}}%
\pgfpathlineto{\pgfqpoint{5.345170in}{0.648511in}}%
\pgfpathlineto{\pgfqpoint{5.317649in}{0.648511in}}%
\pgfpathlineto{\pgfqpoint{5.317649in}{0.550000in}}%
\pgfpathclose%
\pgfusepath{fill}%
\end{pgfscope}%
\begin{pgfscope}%
\pgfpathrectangle{\pgfqpoint{1.250000in}{0.550000in}}{\pgfqpoint{7.750000in}{3.850000in}}%
\pgfusepath{clip}%
\pgfsetbuttcap%
\pgfsetmiterjoin%
\definecolor{currentfill}{rgb}{0.501961,0.501961,0.501961}%
\pgfsetfillcolor{currentfill}%
\pgfsetfillopacity{0.700000}%
\pgfsetlinewidth{0.000000pt}%
\definecolor{currentstroke}{rgb}{0.000000,0.000000,0.000000}%
\pgfsetstrokecolor{currentstroke}%
\pgfsetstrokeopacity{0.700000}%
\pgfsetdash{}{0pt}%
\pgfpathmoveto{\pgfqpoint{5.345170in}{0.550000in}}%
\pgfpathlineto{\pgfqpoint{5.372692in}{0.550000in}}%
\pgfpathlineto{\pgfqpoint{5.372692in}{0.645346in}}%
\pgfpathlineto{\pgfqpoint{5.345170in}{0.645346in}}%
\pgfpathlineto{\pgfqpoint{5.345170in}{0.550000in}}%
\pgfpathclose%
\pgfusepath{fill}%
\end{pgfscope}%
\begin{pgfscope}%
\pgfpathrectangle{\pgfqpoint{1.250000in}{0.550000in}}{\pgfqpoint{7.750000in}{3.850000in}}%
\pgfusepath{clip}%
\pgfsetbuttcap%
\pgfsetmiterjoin%
\definecolor{currentfill}{rgb}{0.501961,0.501961,0.501961}%
\pgfsetfillcolor{currentfill}%
\pgfsetfillopacity{0.700000}%
\pgfsetlinewidth{0.000000pt}%
\definecolor{currentstroke}{rgb}{0.000000,0.000000,0.000000}%
\pgfsetstrokecolor{currentstroke}%
\pgfsetstrokeopacity{0.700000}%
\pgfsetdash{}{0pt}%
\pgfpathmoveto{\pgfqpoint{5.372692in}{0.550000in}}%
\pgfpathlineto{\pgfqpoint{5.400213in}{0.550000in}}%
\pgfpathlineto{\pgfqpoint{5.400213in}{0.639807in}}%
\pgfpathlineto{\pgfqpoint{5.372692in}{0.639807in}}%
\pgfpathlineto{\pgfqpoint{5.372692in}{0.550000in}}%
\pgfpathclose%
\pgfusepath{fill}%
\end{pgfscope}%
\begin{pgfscope}%
\pgfpathrectangle{\pgfqpoint{1.250000in}{0.550000in}}{\pgfqpoint{7.750000in}{3.850000in}}%
\pgfusepath{clip}%
\pgfsetbuttcap%
\pgfsetmiterjoin%
\definecolor{currentfill}{rgb}{0.501961,0.501961,0.501961}%
\pgfsetfillcolor{currentfill}%
\pgfsetfillopacity{0.700000}%
\pgfsetlinewidth{0.000000pt}%
\definecolor{currentstroke}{rgb}{0.000000,0.000000,0.000000}%
\pgfsetstrokecolor{currentstroke}%
\pgfsetstrokeopacity{0.700000}%
\pgfsetdash{}{0pt}%
\pgfpathmoveto{\pgfqpoint{5.400213in}{0.550000in}}%
\pgfpathlineto{\pgfqpoint{5.427734in}{0.550000in}}%
\pgfpathlineto{\pgfqpoint{5.427734in}{0.639016in}}%
\pgfpathlineto{\pgfqpoint{5.400213in}{0.639016in}}%
\pgfpathlineto{\pgfqpoint{5.400213in}{0.550000in}}%
\pgfpathclose%
\pgfusepath{fill}%
\end{pgfscope}%
\begin{pgfscope}%
\pgfpathrectangle{\pgfqpoint{1.250000in}{0.550000in}}{\pgfqpoint{7.750000in}{3.850000in}}%
\pgfusepath{clip}%
\pgfsetbuttcap%
\pgfsetmiterjoin%
\definecolor{currentfill}{rgb}{0.501961,0.501961,0.501961}%
\pgfsetfillcolor{currentfill}%
\pgfsetfillopacity{0.700000}%
\pgfsetlinewidth{0.000000pt}%
\definecolor{currentstroke}{rgb}{0.000000,0.000000,0.000000}%
\pgfsetstrokecolor{currentstroke}%
\pgfsetstrokeopacity{0.700000}%
\pgfsetdash{}{0pt}%
\pgfpathmoveto{\pgfqpoint{5.427734in}{0.550000in}}%
\pgfpathlineto{\pgfqpoint{5.455256in}{0.550000in}}%
\pgfpathlineto{\pgfqpoint{5.455256in}{0.550000in}}%
\pgfpathlineto{\pgfqpoint{5.427734in}{0.550000in}}%
\pgfpathlineto{\pgfqpoint{5.427734in}{0.550000in}}%
\pgfpathclose%
\pgfusepath{fill}%
\end{pgfscope}%
\begin{pgfscope}%
\pgfpathrectangle{\pgfqpoint{1.250000in}{0.550000in}}{\pgfqpoint{7.750000in}{3.850000in}}%
\pgfusepath{clip}%
\pgfsetbuttcap%
\pgfsetmiterjoin%
\definecolor{currentfill}{rgb}{0.501961,0.501961,0.501961}%
\pgfsetfillcolor{currentfill}%
\pgfsetfillopacity{0.700000}%
\pgfsetlinewidth{0.000000pt}%
\definecolor{currentstroke}{rgb}{0.000000,0.000000,0.000000}%
\pgfsetstrokecolor{currentstroke}%
\pgfsetstrokeopacity{0.700000}%
\pgfsetdash{}{0pt}%
\pgfpathmoveto{\pgfqpoint{5.455256in}{0.550000in}}%
\pgfpathlineto{\pgfqpoint{5.482777in}{0.550000in}}%
\pgfpathlineto{\pgfqpoint{5.482777in}{0.640598in}}%
\pgfpathlineto{\pgfqpoint{5.455256in}{0.640598in}}%
\pgfpathlineto{\pgfqpoint{5.455256in}{0.550000in}}%
\pgfpathclose%
\pgfusepath{fill}%
\end{pgfscope}%
\begin{pgfscope}%
\pgfpathrectangle{\pgfqpoint{1.250000in}{0.550000in}}{\pgfqpoint{7.750000in}{3.850000in}}%
\pgfusepath{clip}%
\pgfsetbuttcap%
\pgfsetmiterjoin%
\definecolor{currentfill}{rgb}{0.501961,0.501961,0.501961}%
\pgfsetfillcolor{currentfill}%
\pgfsetfillopacity{0.700000}%
\pgfsetlinewidth{0.000000pt}%
\definecolor{currentstroke}{rgb}{0.000000,0.000000,0.000000}%
\pgfsetstrokecolor{currentstroke}%
\pgfsetstrokeopacity{0.700000}%
\pgfsetdash{}{0pt}%
\pgfpathmoveto{\pgfqpoint{5.482777in}{0.550000in}}%
\pgfpathlineto{\pgfqpoint{5.510298in}{0.550000in}}%
\pgfpathlineto{\pgfqpoint{5.510298in}{0.631499in}}%
\pgfpathlineto{\pgfqpoint{5.482777in}{0.631499in}}%
\pgfpathlineto{\pgfqpoint{5.482777in}{0.550000in}}%
\pgfpathclose%
\pgfusepath{fill}%
\end{pgfscope}%
\begin{pgfscope}%
\pgfpathrectangle{\pgfqpoint{1.250000in}{0.550000in}}{\pgfqpoint{7.750000in}{3.850000in}}%
\pgfusepath{clip}%
\pgfsetbuttcap%
\pgfsetmiterjoin%
\definecolor{currentfill}{rgb}{0.501961,0.501961,0.501961}%
\pgfsetfillcolor{currentfill}%
\pgfsetfillopacity{0.700000}%
\pgfsetlinewidth{0.000000pt}%
\definecolor{currentstroke}{rgb}{0.000000,0.000000,0.000000}%
\pgfsetstrokecolor{currentstroke}%
\pgfsetstrokeopacity{0.700000}%
\pgfsetdash{}{0pt}%
\pgfpathmoveto{\pgfqpoint{5.510298in}{0.550000in}}%
\pgfpathlineto{\pgfqpoint{5.537820in}{0.550000in}}%
\pgfpathlineto{\pgfqpoint{5.537820in}{0.647720in}}%
\pgfpathlineto{\pgfqpoint{5.510298in}{0.647720in}}%
\pgfpathlineto{\pgfqpoint{5.510298in}{0.550000in}}%
\pgfpathclose%
\pgfusepath{fill}%
\end{pgfscope}%
\begin{pgfscope}%
\pgfpathrectangle{\pgfqpoint{1.250000in}{0.550000in}}{\pgfqpoint{7.750000in}{3.850000in}}%
\pgfusepath{clip}%
\pgfsetbuttcap%
\pgfsetmiterjoin%
\definecolor{currentfill}{rgb}{0.501961,0.501961,0.501961}%
\pgfsetfillcolor{currentfill}%
\pgfsetfillopacity{0.700000}%
\pgfsetlinewidth{0.000000pt}%
\definecolor{currentstroke}{rgb}{0.000000,0.000000,0.000000}%
\pgfsetstrokecolor{currentstroke}%
\pgfsetstrokeopacity{0.700000}%
\pgfsetdash{}{0pt}%
\pgfpathmoveto{\pgfqpoint{5.537820in}{0.550000in}}%
\pgfpathlineto{\pgfqpoint{5.565341in}{0.550000in}}%
\pgfpathlineto{\pgfqpoint{5.565341in}{0.635851in}}%
\pgfpathlineto{\pgfqpoint{5.537820in}{0.635851in}}%
\pgfpathlineto{\pgfqpoint{5.537820in}{0.550000in}}%
\pgfpathclose%
\pgfusepath{fill}%
\end{pgfscope}%
\begin{pgfscope}%
\pgfpathrectangle{\pgfqpoint{1.250000in}{0.550000in}}{\pgfqpoint{7.750000in}{3.850000in}}%
\pgfusepath{clip}%
\pgfsetbuttcap%
\pgfsetmiterjoin%
\definecolor{currentfill}{rgb}{0.501961,0.501961,0.501961}%
\pgfsetfillcolor{currentfill}%
\pgfsetfillopacity{0.700000}%
\pgfsetlinewidth{0.000000pt}%
\definecolor{currentstroke}{rgb}{0.000000,0.000000,0.000000}%
\pgfsetstrokecolor{currentstroke}%
\pgfsetstrokeopacity{0.700000}%
\pgfsetdash{}{0pt}%
\pgfpathmoveto{\pgfqpoint{5.565341in}{0.550000in}}%
\pgfpathlineto{\pgfqpoint{5.592862in}{0.550000in}}%
\pgfpathlineto{\pgfqpoint{5.592862in}{0.550000in}}%
\pgfpathlineto{\pgfqpoint{5.565341in}{0.550000in}}%
\pgfpathlineto{\pgfqpoint{5.565341in}{0.550000in}}%
\pgfpathclose%
\pgfusepath{fill}%
\end{pgfscope}%
\begin{pgfscope}%
\pgfpathrectangle{\pgfqpoint{1.250000in}{0.550000in}}{\pgfqpoint{7.750000in}{3.850000in}}%
\pgfusepath{clip}%
\pgfsetbuttcap%
\pgfsetmiterjoin%
\definecolor{currentfill}{rgb}{0.501961,0.501961,0.501961}%
\pgfsetfillcolor{currentfill}%
\pgfsetfillopacity{0.700000}%
\pgfsetlinewidth{0.000000pt}%
\definecolor{currentstroke}{rgb}{0.000000,0.000000,0.000000}%
\pgfsetstrokecolor{currentstroke}%
\pgfsetstrokeopacity{0.700000}%
\pgfsetdash{}{0pt}%
\pgfpathmoveto{\pgfqpoint{5.592862in}{0.550000in}}%
\pgfpathlineto{\pgfqpoint{5.620384in}{0.550000in}}%
\pgfpathlineto{\pgfqpoint{5.620384in}{0.633082in}}%
\pgfpathlineto{\pgfqpoint{5.592862in}{0.633082in}}%
\pgfpathlineto{\pgfqpoint{5.592862in}{0.550000in}}%
\pgfpathclose%
\pgfusepath{fill}%
\end{pgfscope}%
\begin{pgfscope}%
\pgfpathrectangle{\pgfqpoint{1.250000in}{0.550000in}}{\pgfqpoint{7.750000in}{3.850000in}}%
\pgfusepath{clip}%
\pgfsetbuttcap%
\pgfsetmiterjoin%
\definecolor{currentfill}{rgb}{0.501961,0.501961,0.501961}%
\pgfsetfillcolor{currentfill}%
\pgfsetfillopacity{0.700000}%
\pgfsetlinewidth{0.000000pt}%
\definecolor{currentstroke}{rgb}{0.000000,0.000000,0.000000}%
\pgfsetstrokecolor{currentstroke}%
\pgfsetstrokeopacity{0.700000}%
\pgfsetdash{}{0pt}%
\pgfpathmoveto{\pgfqpoint{5.620384in}{0.550000in}}%
\pgfpathlineto{\pgfqpoint{5.647905in}{0.550000in}}%
\pgfpathlineto{\pgfqpoint{5.647905in}{0.631103in}}%
\pgfpathlineto{\pgfqpoint{5.620384in}{0.631103in}}%
\pgfpathlineto{\pgfqpoint{5.620384in}{0.550000in}}%
\pgfpathclose%
\pgfusepath{fill}%
\end{pgfscope}%
\begin{pgfscope}%
\pgfpathrectangle{\pgfqpoint{1.250000in}{0.550000in}}{\pgfqpoint{7.750000in}{3.850000in}}%
\pgfusepath{clip}%
\pgfsetbuttcap%
\pgfsetmiterjoin%
\definecolor{currentfill}{rgb}{0.501961,0.501961,0.501961}%
\pgfsetfillcolor{currentfill}%
\pgfsetfillopacity{0.700000}%
\pgfsetlinewidth{0.000000pt}%
\definecolor{currentstroke}{rgb}{0.000000,0.000000,0.000000}%
\pgfsetstrokecolor{currentstroke}%
\pgfsetstrokeopacity{0.700000}%
\pgfsetdash{}{0pt}%
\pgfpathmoveto{\pgfqpoint{5.647905in}{0.550000in}}%
\pgfpathlineto{\pgfqpoint{5.675426in}{0.550000in}}%
\pgfpathlineto{\pgfqpoint{5.675426in}{0.626752in}}%
\pgfpathlineto{\pgfqpoint{5.647905in}{0.626752in}}%
\pgfpathlineto{\pgfqpoint{5.647905in}{0.550000in}}%
\pgfpathclose%
\pgfusepath{fill}%
\end{pgfscope}%
\begin{pgfscope}%
\pgfpathrectangle{\pgfqpoint{1.250000in}{0.550000in}}{\pgfqpoint{7.750000in}{3.850000in}}%
\pgfusepath{clip}%
\pgfsetbuttcap%
\pgfsetmiterjoin%
\definecolor{currentfill}{rgb}{0.501961,0.501961,0.501961}%
\pgfsetfillcolor{currentfill}%
\pgfsetfillopacity{0.700000}%
\pgfsetlinewidth{0.000000pt}%
\definecolor{currentstroke}{rgb}{0.000000,0.000000,0.000000}%
\pgfsetstrokecolor{currentstroke}%
\pgfsetstrokeopacity{0.700000}%
\pgfsetdash{}{0pt}%
\pgfpathmoveto{\pgfqpoint{5.675426in}{0.550000in}}%
\pgfpathlineto{\pgfqpoint{5.702947in}{0.550000in}}%
\pgfpathlineto{\pgfqpoint{5.702947in}{0.621608in}}%
\pgfpathlineto{\pgfqpoint{5.675426in}{0.621608in}}%
\pgfpathlineto{\pgfqpoint{5.675426in}{0.550000in}}%
\pgfpathclose%
\pgfusepath{fill}%
\end{pgfscope}%
\begin{pgfscope}%
\pgfpathrectangle{\pgfqpoint{1.250000in}{0.550000in}}{\pgfqpoint{7.750000in}{3.850000in}}%
\pgfusepath{clip}%
\pgfsetbuttcap%
\pgfsetmiterjoin%
\definecolor{currentfill}{rgb}{0.501961,0.501961,0.501961}%
\pgfsetfillcolor{currentfill}%
\pgfsetfillopacity{0.700000}%
\pgfsetlinewidth{0.000000pt}%
\definecolor{currentstroke}{rgb}{0.000000,0.000000,0.000000}%
\pgfsetstrokecolor{currentstroke}%
\pgfsetstrokeopacity{0.700000}%
\pgfsetdash{}{0pt}%
\pgfpathmoveto{\pgfqpoint{5.702947in}{0.550000in}}%
\pgfpathlineto{\pgfqpoint{5.730469in}{0.550000in}}%
\pgfpathlineto{\pgfqpoint{5.730469in}{0.550000in}}%
\pgfpathlineto{\pgfqpoint{5.702947in}{0.550000in}}%
\pgfpathlineto{\pgfqpoint{5.702947in}{0.550000in}}%
\pgfpathclose%
\pgfusepath{fill}%
\end{pgfscope}%
\begin{pgfscope}%
\pgfpathrectangle{\pgfqpoint{1.250000in}{0.550000in}}{\pgfqpoint{7.750000in}{3.850000in}}%
\pgfusepath{clip}%
\pgfsetbuttcap%
\pgfsetmiterjoin%
\definecolor{currentfill}{rgb}{0.501961,0.501961,0.501961}%
\pgfsetfillcolor{currentfill}%
\pgfsetfillopacity{0.700000}%
\pgfsetlinewidth{0.000000pt}%
\definecolor{currentstroke}{rgb}{0.000000,0.000000,0.000000}%
\pgfsetstrokecolor{currentstroke}%
\pgfsetstrokeopacity{0.700000}%
\pgfsetdash{}{0pt}%
\pgfpathmoveto{\pgfqpoint{5.730469in}{0.550000in}}%
\pgfpathlineto{\pgfqpoint{5.757990in}{0.550000in}}%
\pgfpathlineto{\pgfqpoint{5.757990in}{0.627543in}}%
\pgfpathlineto{\pgfqpoint{5.730469in}{0.627543in}}%
\pgfpathlineto{\pgfqpoint{5.730469in}{0.550000in}}%
\pgfpathclose%
\pgfusepath{fill}%
\end{pgfscope}%
\begin{pgfscope}%
\pgfpathrectangle{\pgfqpoint{1.250000in}{0.550000in}}{\pgfqpoint{7.750000in}{3.850000in}}%
\pgfusepath{clip}%
\pgfsetbuttcap%
\pgfsetmiterjoin%
\definecolor{currentfill}{rgb}{0.501961,0.501961,0.501961}%
\pgfsetfillcolor{currentfill}%
\pgfsetfillopacity{0.700000}%
\pgfsetlinewidth{0.000000pt}%
\definecolor{currentstroke}{rgb}{0.000000,0.000000,0.000000}%
\pgfsetstrokecolor{currentstroke}%
\pgfsetstrokeopacity{0.700000}%
\pgfsetdash{}{0pt}%
\pgfpathmoveto{\pgfqpoint{5.757990in}{0.550000in}}%
\pgfpathlineto{\pgfqpoint{5.785511in}{0.550000in}}%
\pgfpathlineto{\pgfqpoint{5.785511in}{0.619235in}}%
\pgfpathlineto{\pgfqpoint{5.757990in}{0.619235in}}%
\pgfpathlineto{\pgfqpoint{5.757990in}{0.550000in}}%
\pgfpathclose%
\pgfusepath{fill}%
\end{pgfscope}%
\begin{pgfscope}%
\pgfpathrectangle{\pgfqpoint{1.250000in}{0.550000in}}{\pgfqpoint{7.750000in}{3.850000in}}%
\pgfusepath{clip}%
\pgfsetbuttcap%
\pgfsetmiterjoin%
\definecolor{currentfill}{rgb}{0.501961,0.501961,0.501961}%
\pgfsetfillcolor{currentfill}%
\pgfsetfillopacity{0.700000}%
\pgfsetlinewidth{0.000000pt}%
\definecolor{currentstroke}{rgb}{0.000000,0.000000,0.000000}%
\pgfsetstrokecolor{currentstroke}%
\pgfsetstrokeopacity{0.700000}%
\pgfsetdash{}{0pt}%
\pgfpathmoveto{\pgfqpoint{5.785511in}{0.550000in}}%
\pgfpathlineto{\pgfqpoint{5.813033in}{0.550000in}}%
\pgfpathlineto{\pgfqpoint{5.813033in}{0.620422in}}%
\pgfpathlineto{\pgfqpoint{5.785511in}{0.620422in}}%
\pgfpathlineto{\pgfqpoint{5.785511in}{0.550000in}}%
\pgfpathclose%
\pgfusepath{fill}%
\end{pgfscope}%
\begin{pgfscope}%
\pgfpathrectangle{\pgfqpoint{1.250000in}{0.550000in}}{\pgfqpoint{7.750000in}{3.850000in}}%
\pgfusepath{clip}%
\pgfsetbuttcap%
\pgfsetmiterjoin%
\definecolor{currentfill}{rgb}{0.501961,0.501961,0.501961}%
\pgfsetfillcolor{currentfill}%
\pgfsetfillopacity{0.700000}%
\pgfsetlinewidth{0.000000pt}%
\definecolor{currentstroke}{rgb}{0.000000,0.000000,0.000000}%
\pgfsetstrokecolor{currentstroke}%
\pgfsetstrokeopacity{0.700000}%
\pgfsetdash{}{0pt}%
\pgfpathmoveto{\pgfqpoint{5.813033in}{0.550000in}}%
\pgfpathlineto{\pgfqpoint{5.840554in}{0.550000in}}%
\pgfpathlineto{\pgfqpoint{5.840554in}{0.550000in}}%
\pgfpathlineto{\pgfqpoint{5.813033in}{0.550000in}}%
\pgfpathlineto{\pgfqpoint{5.813033in}{0.550000in}}%
\pgfpathclose%
\pgfusepath{fill}%
\end{pgfscope}%
\begin{pgfscope}%
\pgfpathrectangle{\pgfqpoint{1.250000in}{0.550000in}}{\pgfqpoint{7.750000in}{3.850000in}}%
\pgfusepath{clip}%
\pgfsetbuttcap%
\pgfsetmiterjoin%
\definecolor{currentfill}{rgb}{0.501961,0.501961,0.501961}%
\pgfsetfillcolor{currentfill}%
\pgfsetfillopacity{0.700000}%
\pgfsetlinewidth{0.000000pt}%
\definecolor{currentstroke}{rgb}{0.000000,0.000000,0.000000}%
\pgfsetstrokecolor{currentstroke}%
\pgfsetstrokeopacity{0.700000}%
\pgfsetdash{}{0pt}%
\pgfpathmoveto{\pgfqpoint{5.840554in}{0.550000in}}%
\pgfpathlineto{\pgfqpoint{5.868075in}{0.550000in}}%
\pgfpathlineto{\pgfqpoint{5.868075in}{0.610135in}}%
\pgfpathlineto{\pgfqpoint{5.840554in}{0.610135in}}%
\pgfpathlineto{\pgfqpoint{5.840554in}{0.550000in}}%
\pgfpathclose%
\pgfusepath{fill}%
\end{pgfscope}%
\begin{pgfscope}%
\pgfpathrectangle{\pgfqpoint{1.250000in}{0.550000in}}{\pgfqpoint{7.750000in}{3.850000in}}%
\pgfusepath{clip}%
\pgfsetbuttcap%
\pgfsetmiterjoin%
\definecolor{currentfill}{rgb}{0.501961,0.501961,0.501961}%
\pgfsetfillcolor{currentfill}%
\pgfsetfillopacity{0.700000}%
\pgfsetlinewidth{0.000000pt}%
\definecolor{currentstroke}{rgb}{0.000000,0.000000,0.000000}%
\pgfsetstrokecolor{currentstroke}%
\pgfsetstrokeopacity{0.700000}%
\pgfsetdash{}{0pt}%
\pgfpathmoveto{\pgfqpoint{5.868075in}{0.550000in}}%
\pgfpathlineto{\pgfqpoint{5.895597in}{0.550000in}}%
\pgfpathlineto{\pgfqpoint{5.895597in}{0.635060in}}%
\pgfpathlineto{\pgfqpoint{5.868075in}{0.635060in}}%
\pgfpathlineto{\pgfqpoint{5.868075in}{0.550000in}}%
\pgfpathclose%
\pgfusepath{fill}%
\end{pgfscope}%
\begin{pgfscope}%
\pgfpathrectangle{\pgfqpoint{1.250000in}{0.550000in}}{\pgfqpoint{7.750000in}{3.850000in}}%
\pgfusepath{clip}%
\pgfsetbuttcap%
\pgfsetmiterjoin%
\definecolor{currentfill}{rgb}{0.501961,0.501961,0.501961}%
\pgfsetfillcolor{currentfill}%
\pgfsetfillopacity{0.700000}%
\pgfsetlinewidth{0.000000pt}%
\definecolor{currentstroke}{rgb}{0.000000,0.000000,0.000000}%
\pgfsetstrokecolor{currentstroke}%
\pgfsetstrokeopacity{0.700000}%
\pgfsetdash{}{0pt}%
\pgfpathmoveto{\pgfqpoint{5.895597in}{0.550000in}}%
\pgfpathlineto{\pgfqpoint{5.923118in}{0.550000in}}%
\pgfpathlineto{\pgfqpoint{5.923118in}{0.621213in}}%
\pgfpathlineto{\pgfqpoint{5.895597in}{0.621213in}}%
\pgfpathlineto{\pgfqpoint{5.895597in}{0.550000in}}%
\pgfpathclose%
\pgfusepath{fill}%
\end{pgfscope}%
\begin{pgfscope}%
\pgfpathrectangle{\pgfqpoint{1.250000in}{0.550000in}}{\pgfqpoint{7.750000in}{3.850000in}}%
\pgfusepath{clip}%
\pgfsetbuttcap%
\pgfsetmiterjoin%
\definecolor{currentfill}{rgb}{0.501961,0.501961,0.501961}%
\pgfsetfillcolor{currentfill}%
\pgfsetfillopacity{0.700000}%
\pgfsetlinewidth{0.000000pt}%
\definecolor{currentstroke}{rgb}{0.000000,0.000000,0.000000}%
\pgfsetstrokecolor{currentstroke}%
\pgfsetstrokeopacity{0.700000}%
\pgfsetdash{}{0pt}%
\pgfpathmoveto{\pgfqpoint{5.923118in}{0.550000in}}%
\pgfpathlineto{\pgfqpoint{5.950639in}{0.550000in}}%
\pgfpathlineto{\pgfqpoint{5.950639in}{0.607366in}}%
\pgfpathlineto{\pgfqpoint{5.923118in}{0.607366in}}%
\pgfpathlineto{\pgfqpoint{5.923118in}{0.550000in}}%
\pgfpathclose%
\pgfusepath{fill}%
\end{pgfscope}%
\begin{pgfscope}%
\pgfpathrectangle{\pgfqpoint{1.250000in}{0.550000in}}{\pgfqpoint{7.750000in}{3.850000in}}%
\pgfusepath{clip}%
\pgfsetbuttcap%
\pgfsetmiterjoin%
\definecolor{currentfill}{rgb}{0.501961,0.501961,0.501961}%
\pgfsetfillcolor{currentfill}%
\pgfsetfillopacity{0.700000}%
\pgfsetlinewidth{0.000000pt}%
\definecolor{currentstroke}{rgb}{0.000000,0.000000,0.000000}%
\pgfsetstrokecolor{currentstroke}%
\pgfsetstrokeopacity{0.700000}%
\pgfsetdash{}{0pt}%
\pgfpathmoveto{\pgfqpoint{5.950639in}{0.550000in}}%
\pgfpathlineto{\pgfqpoint{5.978161in}{0.550000in}}%
\pgfpathlineto{\pgfqpoint{5.978161in}{0.550000in}}%
\pgfpathlineto{\pgfqpoint{5.950639in}{0.550000in}}%
\pgfpathlineto{\pgfqpoint{5.950639in}{0.550000in}}%
\pgfpathclose%
\pgfusepath{fill}%
\end{pgfscope}%
\begin{pgfscope}%
\pgfpathrectangle{\pgfqpoint{1.250000in}{0.550000in}}{\pgfqpoint{7.750000in}{3.850000in}}%
\pgfusepath{clip}%
\pgfsetbuttcap%
\pgfsetmiterjoin%
\definecolor{currentfill}{rgb}{0.501961,0.501961,0.501961}%
\pgfsetfillcolor{currentfill}%
\pgfsetfillopacity{0.700000}%
\pgfsetlinewidth{0.000000pt}%
\definecolor{currentstroke}{rgb}{0.000000,0.000000,0.000000}%
\pgfsetstrokecolor{currentstroke}%
\pgfsetstrokeopacity{0.700000}%
\pgfsetdash{}{0pt}%
\pgfpathmoveto{\pgfqpoint{5.978161in}{0.550000in}}%
\pgfpathlineto{\pgfqpoint{6.005682in}{0.550000in}}%
\pgfpathlineto{\pgfqpoint{6.005682in}{0.616070in}}%
\pgfpathlineto{\pgfqpoint{5.978161in}{0.616070in}}%
\pgfpathlineto{\pgfqpoint{5.978161in}{0.550000in}}%
\pgfpathclose%
\pgfusepath{fill}%
\end{pgfscope}%
\begin{pgfscope}%
\pgfpathrectangle{\pgfqpoint{1.250000in}{0.550000in}}{\pgfqpoint{7.750000in}{3.850000in}}%
\pgfusepath{clip}%
\pgfsetbuttcap%
\pgfsetmiterjoin%
\definecolor{currentfill}{rgb}{0.501961,0.501961,0.501961}%
\pgfsetfillcolor{currentfill}%
\pgfsetfillopacity{0.700000}%
\pgfsetlinewidth{0.000000pt}%
\definecolor{currentstroke}{rgb}{0.000000,0.000000,0.000000}%
\pgfsetstrokecolor{currentstroke}%
\pgfsetstrokeopacity{0.700000}%
\pgfsetdash{}{0pt}%
\pgfpathmoveto{\pgfqpoint{6.005682in}{0.550000in}}%
\pgfpathlineto{\pgfqpoint{6.033203in}{0.550000in}}%
\pgfpathlineto{\pgfqpoint{6.033203in}{0.615674in}}%
\pgfpathlineto{\pgfqpoint{6.005682in}{0.615674in}}%
\pgfpathlineto{\pgfqpoint{6.005682in}{0.550000in}}%
\pgfpathclose%
\pgfusepath{fill}%
\end{pgfscope}%
\begin{pgfscope}%
\pgfpathrectangle{\pgfqpoint{1.250000in}{0.550000in}}{\pgfqpoint{7.750000in}{3.850000in}}%
\pgfusepath{clip}%
\pgfsetbuttcap%
\pgfsetmiterjoin%
\definecolor{currentfill}{rgb}{0.501961,0.501961,0.501961}%
\pgfsetfillcolor{currentfill}%
\pgfsetfillopacity{0.700000}%
\pgfsetlinewidth{0.000000pt}%
\definecolor{currentstroke}{rgb}{0.000000,0.000000,0.000000}%
\pgfsetstrokecolor{currentstroke}%
\pgfsetstrokeopacity{0.700000}%
\pgfsetdash{}{0pt}%
\pgfpathmoveto{\pgfqpoint{6.033203in}{0.550000in}}%
\pgfpathlineto{\pgfqpoint{6.060724in}{0.550000in}}%
\pgfpathlineto{\pgfqpoint{6.060724in}{0.606179in}}%
\pgfpathlineto{\pgfqpoint{6.033203in}{0.606179in}}%
\pgfpathlineto{\pgfqpoint{6.033203in}{0.550000in}}%
\pgfpathclose%
\pgfusepath{fill}%
\end{pgfscope}%
\begin{pgfscope}%
\pgfpathrectangle{\pgfqpoint{1.250000in}{0.550000in}}{\pgfqpoint{7.750000in}{3.850000in}}%
\pgfusepath{clip}%
\pgfsetbuttcap%
\pgfsetmiterjoin%
\definecolor{currentfill}{rgb}{0.501961,0.501961,0.501961}%
\pgfsetfillcolor{currentfill}%
\pgfsetfillopacity{0.700000}%
\pgfsetlinewidth{0.000000pt}%
\definecolor{currentstroke}{rgb}{0.000000,0.000000,0.000000}%
\pgfsetstrokecolor{currentstroke}%
\pgfsetstrokeopacity{0.700000}%
\pgfsetdash{}{0pt}%
\pgfpathmoveto{\pgfqpoint{6.060724in}{0.550000in}}%
\pgfpathlineto{\pgfqpoint{6.088246in}{0.550000in}}%
\pgfpathlineto{\pgfqpoint{6.088246in}{0.602618in}}%
\pgfpathlineto{\pgfqpoint{6.060724in}{0.602618in}}%
\pgfpathlineto{\pgfqpoint{6.060724in}{0.550000in}}%
\pgfpathclose%
\pgfusepath{fill}%
\end{pgfscope}%
\begin{pgfscope}%
\pgfpathrectangle{\pgfqpoint{1.250000in}{0.550000in}}{\pgfqpoint{7.750000in}{3.850000in}}%
\pgfusepath{clip}%
\pgfsetbuttcap%
\pgfsetmiterjoin%
\definecolor{currentfill}{rgb}{0.501961,0.501961,0.501961}%
\pgfsetfillcolor{currentfill}%
\pgfsetfillopacity{0.700000}%
\pgfsetlinewidth{0.000000pt}%
\definecolor{currentstroke}{rgb}{0.000000,0.000000,0.000000}%
\pgfsetstrokecolor{currentstroke}%
\pgfsetstrokeopacity{0.700000}%
\pgfsetdash{}{0pt}%
\pgfpathmoveto{\pgfqpoint{6.088246in}{0.550000in}}%
\pgfpathlineto{\pgfqpoint{6.115767in}{0.550000in}}%
\pgfpathlineto{\pgfqpoint{6.115767in}{0.550000in}}%
\pgfpathlineto{\pgfqpoint{6.088246in}{0.550000in}}%
\pgfpathlineto{\pgfqpoint{6.088246in}{0.550000in}}%
\pgfpathclose%
\pgfusepath{fill}%
\end{pgfscope}%
\begin{pgfscope}%
\pgfpathrectangle{\pgfqpoint{1.250000in}{0.550000in}}{\pgfqpoint{7.750000in}{3.850000in}}%
\pgfusepath{clip}%
\pgfsetbuttcap%
\pgfsetmiterjoin%
\definecolor{currentfill}{rgb}{0.501961,0.501961,0.501961}%
\pgfsetfillcolor{currentfill}%
\pgfsetfillopacity{0.700000}%
\pgfsetlinewidth{0.000000pt}%
\definecolor{currentstroke}{rgb}{0.000000,0.000000,0.000000}%
\pgfsetstrokecolor{currentstroke}%
\pgfsetstrokeopacity{0.700000}%
\pgfsetdash{}{0pt}%
\pgfpathmoveto{\pgfqpoint{6.115767in}{0.550000in}}%
\pgfpathlineto{\pgfqpoint{6.143288in}{0.550000in}}%
\pgfpathlineto{\pgfqpoint{6.143288in}{0.606970in}}%
\pgfpathlineto{\pgfqpoint{6.115767in}{0.606970in}}%
\pgfpathlineto{\pgfqpoint{6.115767in}{0.550000in}}%
\pgfpathclose%
\pgfusepath{fill}%
\end{pgfscope}%
\begin{pgfscope}%
\pgfpathrectangle{\pgfqpoint{1.250000in}{0.550000in}}{\pgfqpoint{7.750000in}{3.850000in}}%
\pgfusepath{clip}%
\pgfsetbuttcap%
\pgfsetmiterjoin%
\definecolor{currentfill}{rgb}{0.501961,0.501961,0.501961}%
\pgfsetfillcolor{currentfill}%
\pgfsetfillopacity{0.700000}%
\pgfsetlinewidth{0.000000pt}%
\definecolor{currentstroke}{rgb}{0.000000,0.000000,0.000000}%
\pgfsetstrokecolor{currentstroke}%
\pgfsetstrokeopacity{0.700000}%
\pgfsetdash{}{0pt}%
\pgfpathmoveto{\pgfqpoint{6.143288in}{0.550000in}}%
\pgfpathlineto{\pgfqpoint{6.170810in}{0.550000in}}%
\pgfpathlineto{\pgfqpoint{6.170810in}{0.617652in}}%
\pgfpathlineto{\pgfqpoint{6.143288in}{0.617652in}}%
\pgfpathlineto{\pgfqpoint{6.143288in}{0.550000in}}%
\pgfpathclose%
\pgfusepath{fill}%
\end{pgfscope}%
\begin{pgfscope}%
\pgfpathrectangle{\pgfqpoint{1.250000in}{0.550000in}}{\pgfqpoint{7.750000in}{3.850000in}}%
\pgfusepath{clip}%
\pgfsetbuttcap%
\pgfsetmiterjoin%
\definecolor{currentfill}{rgb}{0.501961,0.501961,0.501961}%
\pgfsetfillcolor{currentfill}%
\pgfsetfillopacity{0.700000}%
\pgfsetlinewidth{0.000000pt}%
\definecolor{currentstroke}{rgb}{0.000000,0.000000,0.000000}%
\pgfsetstrokecolor{currentstroke}%
\pgfsetstrokeopacity{0.700000}%
\pgfsetdash{}{0pt}%
\pgfpathmoveto{\pgfqpoint{6.170810in}{0.550000in}}%
\pgfpathlineto{\pgfqpoint{6.198331in}{0.550000in}}%
\pgfpathlineto{\pgfqpoint{6.198331in}{0.610135in}}%
\pgfpathlineto{\pgfqpoint{6.170810in}{0.610135in}}%
\pgfpathlineto{\pgfqpoint{6.170810in}{0.550000in}}%
\pgfpathclose%
\pgfusepath{fill}%
\end{pgfscope}%
\begin{pgfscope}%
\pgfpathrectangle{\pgfqpoint{1.250000in}{0.550000in}}{\pgfqpoint{7.750000in}{3.850000in}}%
\pgfusepath{clip}%
\pgfsetbuttcap%
\pgfsetmiterjoin%
\definecolor{currentfill}{rgb}{0.501961,0.501961,0.501961}%
\pgfsetfillcolor{currentfill}%
\pgfsetfillopacity{0.700000}%
\pgfsetlinewidth{0.000000pt}%
\definecolor{currentstroke}{rgb}{0.000000,0.000000,0.000000}%
\pgfsetstrokecolor{currentstroke}%
\pgfsetstrokeopacity{0.700000}%
\pgfsetdash{}{0pt}%
\pgfpathmoveto{\pgfqpoint{6.198331in}{0.550000in}}%
\pgfpathlineto{\pgfqpoint{6.225852in}{0.550000in}}%
\pgfpathlineto{\pgfqpoint{6.225852in}{0.617257in}}%
\pgfpathlineto{\pgfqpoint{6.198331in}{0.617257in}}%
\pgfpathlineto{\pgfqpoint{6.198331in}{0.550000in}}%
\pgfpathclose%
\pgfusepath{fill}%
\end{pgfscope}%
\begin{pgfscope}%
\pgfpathrectangle{\pgfqpoint{1.250000in}{0.550000in}}{\pgfqpoint{7.750000in}{3.850000in}}%
\pgfusepath{clip}%
\pgfsetbuttcap%
\pgfsetmiterjoin%
\definecolor{currentfill}{rgb}{0.501961,0.501961,0.501961}%
\pgfsetfillcolor{currentfill}%
\pgfsetfillopacity{0.700000}%
\pgfsetlinewidth{0.000000pt}%
\definecolor{currentstroke}{rgb}{0.000000,0.000000,0.000000}%
\pgfsetstrokecolor{currentstroke}%
\pgfsetstrokeopacity{0.700000}%
\pgfsetdash{}{0pt}%
\pgfpathmoveto{\pgfqpoint{6.225852in}{0.550000in}}%
\pgfpathlineto{\pgfqpoint{6.253374in}{0.550000in}}%
\pgfpathlineto{\pgfqpoint{6.253374in}{0.550000in}}%
\pgfpathlineto{\pgfqpoint{6.225852in}{0.550000in}}%
\pgfpathlineto{\pgfqpoint{6.225852in}{0.550000in}}%
\pgfpathclose%
\pgfusepath{fill}%
\end{pgfscope}%
\begin{pgfscope}%
\pgfpathrectangle{\pgfqpoint{1.250000in}{0.550000in}}{\pgfqpoint{7.750000in}{3.850000in}}%
\pgfusepath{clip}%
\pgfsetbuttcap%
\pgfsetmiterjoin%
\definecolor{currentfill}{rgb}{0.501961,0.501961,0.501961}%
\pgfsetfillcolor{currentfill}%
\pgfsetfillopacity{0.700000}%
\pgfsetlinewidth{0.000000pt}%
\definecolor{currentstroke}{rgb}{0.000000,0.000000,0.000000}%
\pgfsetstrokecolor{currentstroke}%
\pgfsetstrokeopacity{0.700000}%
\pgfsetdash{}{0pt}%
\pgfpathmoveto{\pgfqpoint{6.253374in}{0.550000in}}%
\pgfpathlineto{\pgfqpoint{6.280895in}{0.550000in}}%
\pgfpathlineto{\pgfqpoint{6.280895in}{0.608948in}}%
\pgfpathlineto{\pgfqpoint{6.253374in}{0.608948in}}%
\pgfpathlineto{\pgfqpoint{6.253374in}{0.550000in}}%
\pgfpathclose%
\pgfusepath{fill}%
\end{pgfscope}%
\begin{pgfscope}%
\pgfpathrectangle{\pgfqpoint{1.250000in}{0.550000in}}{\pgfqpoint{7.750000in}{3.850000in}}%
\pgfusepath{clip}%
\pgfsetbuttcap%
\pgfsetmiterjoin%
\definecolor{currentfill}{rgb}{0.501961,0.501961,0.501961}%
\pgfsetfillcolor{currentfill}%
\pgfsetfillopacity{0.700000}%
\pgfsetlinewidth{0.000000pt}%
\definecolor{currentstroke}{rgb}{0.000000,0.000000,0.000000}%
\pgfsetstrokecolor{currentstroke}%
\pgfsetstrokeopacity{0.700000}%
\pgfsetdash{}{0pt}%
\pgfpathmoveto{\pgfqpoint{6.280895in}{0.550000in}}%
\pgfpathlineto{\pgfqpoint{6.308416in}{0.550000in}}%
\pgfpathlineto{\pgfqpoint{6.308416in}{0.603410in}}%
\pgfpathlineto{\pgfqpoint{6.280895in}{0.603410in}}%
\pgfpathlineto{\pgfqpoint{6.280895in}{0.550000in}}%
\pgfpathclose%
\pgfusepath{fill}%
\end{pgfscope}%
\begin{pgfscope}%
\pgfpathrectangle{\pgfqpoint{1.250000in}{0.550000in}}{\pgfqpoint{7.750000in}{3.850000in}}%
\pgfusepath{clip}%
\pgfsetbuttcap%
\pgfsetmiterjoin%
\definecolor{currentfill}{rgb}{0.501961,0.501961,0.501961}%
\pgfsetfillcolor{currentfill}%
\pgfsetfillopacity{0.700000}%
\pgfsetlinewidth{0.000000pt}%
\definecolor{currentstroke}{rgb}{0.000000,0.000000,0.000000}%
\pgfsetstrokecolor{currentstroke}%
\pgfsetstrokeopacity{0.700000}%
\pgfsetdash{}{0pt}%
\pgfpathmoveto{\pgfqpoint{6.308416in}{0.550000in}}%
\pgfpathlineto{\pgfqpoint{6.335937in}{0.550000in}}%
\pgfpathlineto{\pgfqpoint{6.335937in}{0.597080in}}%
\pgfpathlineto{\pgfqpoint{6.308416in}{0.597080in}}%
\pgfpathlineto{\pgfqpoint{6.308416in}{0.550000in}}%
\pgfpathclose%
\pgfusepath{fill}%
\end{pgfscope}%
\begin{pgfscope}%
\pgfpathrectangle{\pgfqpoint{1.250000in}{0.550000in}}{\pgfqpoint{7.750000in}{3.850000in}}%
\pgfusepath{clip}%
\pgfsetbuttcap%
\pgfsetmiterjoin%
\definecolor{currentfill}{rgb}{0.501961,0.501961,0.501961}%
\pgfsetfillcolor{currentfill}%
\pgfsetfillopacity{0.700000}%
\pgfsetlinewidth{0.000000pt}%
\definecolor{currentstroke}{rgb}{0.000000,0.000000,0.000000}%
\pgfsetstrokecolor{currentstroke}%
\pgfsetstrokeopacity{0.700000}%
\pgfsetdash{}{0pt}%
\pgfpathmoveto{\pgfqpoint{6.335938in}{0.550000in}}%
\pgfpathlineto{\pgfqpoint{6.363459in}{0.550000in}}%
\pgfpathlineto{\pgfqpoint{6.363459in}{0.593519in}}%
\pgfpathlineto{\pgfqpoint{6.335938in}{0.593519in}}%
\pgfpathlineto{\pgfqpoint{6.335938in}{0.550000in}}%
\pgfpathclose%
\pgfusepath{fill}%
\end{pgfscope}%
\begin{pgfscope}%
\pgfpathrectangle{\pgfqpoint{1.250000in}{0.550000in}}{\pgfqpoint{7.750000in}{3.850000in}}%
\pgfusepath{clip}%
\pgfsetbuttcap%
\pgfsetmiterjoin%
\definecolor{currentfill}{rgb}{0.501961,0.501961,0.501961}%
\pgfsetfillcolor{currentfill}%
\pgfsetfillopacity{0.700000}%
\pgfsetlinewidth{0.000000pt}%
\definecolor{currentstroke}{rgb}{0.000000,0.000000,0.000000}%
\pgfsetstrokecolor{currentstroke}%
\pgfsetstrokeopacity{0.700000}%
\pgfsetdash{}{0pt}%
\pgfpathmoveto{\pgfqpoint{6.363459in}{0.550000in}}%
\pgfpathlineto{\pgfqpoint{6.390980in}{0.550000in}}%
\pgfpathlineto{\pgfqpoint{6.390980in}{0.597080in}}%
\pgfpathlineto{\pgfqpoint{6.363459in}{0.597080in}}%
\pgfpathlineto{\pgfqpoint{6.363459in}{0.550000in}}%
\pgfpathclose%
\pgfusepath{fill}%
\end{pgfscope}%
\begin{pgfscope}%
\pgfpathrectangle{\pgfqpoint{1.250000in}{0.550000in}}{\pgfqpoint{7.750000in}{3.850000in}}%
\pgfusepath{clip}%
\pgfsetbuttcap%
\pgfsetmiterjoin%
\definecolor{currentfill}{rgb}{0.501961,0.501961,0.501961}%
\pgfsetfillcolor{currentfill}%
\pgfsetfillopacity{0.700000}%
\pgfsetlinewidth{0.000000pt}%
\definecolor{currentstroke}{rgb}{0.000000,0.000000,0.000000}%
\pgfsetstrokecolor{currentstroke}%
\pgfsetstrokeopacity{0.700000}%
\pgfsetdash{}{0pt}%
\pgfpathmoveto{\pgfqpoint{6.390980in}{0.550000in}}%
\pgfpathlineto{\pgfqpoint{6.418501in}{0.550000in}}%
\pgfpathlineto{\pgfqpoint{6.418501in}{0.590750in}}%
\pgfpathlineto{\pgfqpoint{6.390980in}{0.590750in}}%
\pgfpathlineto{\pgfqpoint{6.390980in}{0.550000in}}%
\pgfpathclose%
\pgfusepath{fill}%
\end{pgfscope}%
\begin{pgfscope}%
\pgfpathrectangle{\pgfqpoint{1.250000in}{0.550000in}}{\pgfqpoint{7.750000in}{3.850000in}}%
\pgfusepath{clip}%
\pgfsetbuttcap%
\pgfsetmiterjoin%
\definecolor{currentfill}{rgb}{0.501961,0.501961,0.501961}%
\pgfsetfillcolor{currentfill}%
\pgfsetfillopacity{0.700000}%
\pgfsetlinewidth{0.000000pt}%
\definecolor{currentstroke}{rgb}{0.000000,0.000000,0.000000}%
\pgfsetstrokecolor{currentstroke}%
\pgfsetstrokeopacity{0.700000}%
\pgfsetdash{}{0pt}%
\pgfpathmoveto{\pgfqpoint{6.418501in}{0.550000in}}%
\pgfpathlineto{\pgfqpoint{6.446023in}{0.550000in}}%
\pgfpathlineto{\pgfqpoint{6.446023in}{0.550000in}}%
\pgfpathlineto{\pgfqpoint{6.418501in}{0.550000in}}%
\pgfpathlineto{\pgfqpoint{6.418501in}{0.550000in}}%
\pgfpathclose%
\pgfusepath{fill}%
\end{pgfscope}%
\begin{pgfscope}%
\pgfpathrectangle{\pgfqpoint{1.250000in}{0.550000in}}{\pgfqpoint{7.750000in}{3.850000in}}%
\pgfusepath{clip}%
\pgfsetbuttcap%
\pgfsetmiterjoin%
\definecolor{currentfill}{rgb}{0.501961,0.501961,0.501961}%
\pgfsetfillcolor{currentfill}%
\pgfsetfillopacity{0.700000}%
\pgfsetlinewidth{0.000000pt}%
\definecolor{currentstroke}{rgb}{0.000000,0.000000,0.000000}%
\pgfsetstrokecolor{currentstroke}%
\pgfsetstrokeopacity{0.700000}%
\pgfsetdash{}{0pt}%
\pgfpathmoveto{\pgfqpoint{6.446023in}{0.550000in}}%
\pgfpathlineto{\pgfqpoint{6.473544in}{0.550000in}}%
\pgfpathlineto{\pgfqpoint{6.473544in}{0.596288in}}%
\pgfpathlineto{\pgfqpoint{6.446023in}{0.596288in}}%
\pgfpathlineto{\pgfqpoint{6.446023in}{0.550000in}}%
\pgfpathclose%
\pgfusepath{fill}%
\end{pgfscope}%
\begin{pgfscope}%
\pgfpathrectangle{\pgfqpoint{1.250000in}{0.550000in}}{\pgfqpoint{7.750000in}{3.850000in}}%
\pgfusepath{clip}%
\pgfsetbuttcap%
\pgfsetmiterjoin%
\definecolor{currentfill}{rgb}{0.501961,0.501961,0.501961}%
\pgfsetfillcolor{currentfill}%
\pgfsetfillopacity{0.700000}%
\pgfsetlinewidth{0.000000pt}%
\definecolor{currentstroke}{rgb}{0.000000,0.000000,0.000000}%
\pgfsetstrokecolor{currentstroke}%
\pgfsetstrokeopacity{0.700000}%
\pgfsetdash{}{0pt}%
\pgfpathmoveto{\pgfqpoint{6.473544in}{0.550000in}}%
\pgfpathlineto{\pgfqpoint{6.501065in}{0.550000in}}%
\pgfpathlineto{\pgfqpoint{6.501065in}{0.594310in}}%
\pgfpathlineto{\pgfqpoint{6.473544in}{0.594310in}}%
\pgfpathlineto{\pgfqpoint{6.473544in}{0.550000in}}%
\pgfpathclose%
\pgfusepath{fill}%
\end{pgfscope}%
\begin{pgfscope}%
\pgfpathrectangle{\pgfqpoint{1.250000in}{0.550000in}}{\pgfqpoint{7.750000in}{3.850000in}}%
\pgfusepath{clip}%
\pgfsetbuttcap%
\pgfsetmiterjoin%
\definecolor{currentfill}{rgb}{0.501961,0.501961,0.501961}%
\pgfsetfillcolor{currentfill}%
\pgfsetfillopacity{0.700000}%
\pgfsetlinewidth{0.000000pt}%
\definecolor{currentstroke}{rgb}{0.000000,0.000000,0.000000}%
\pgfsetstrokecolor{currentstroke}%
\pgfsetstrokeopacity{0.700000}%
\pgfsetdash{}{0pt}%
\pgfpathmoveto{\pgfqpoint{6.501065in}{0.550000in}}%
\pgfpathlineto{\pgfqpoint{6.528587in}{0.550000in}}%
\pgfpathlineto{\pgfqpoint{6.528587in}{0.597080in}}%
\pgfpathlineto{\pgfqpoint{6.501065in}{0.597080in}}%
\pgfpathlineto{\pgfqpoint{6.501065in}{0.550000in}}%
\pgfpathclose%
\pgfusepath{fill}%
\end{pgfscope}%
\begin{pgfscope}%
\pgfpathrectangle{\pgfqpoint{1.250000in}{0.550000in}}{\pgfqpoint{7.750000in}{3.850000in}}%
\pgfusepath{clip}%
\pgfsetbuttcap%
\pgfsetmiterjoin%
\definecolor{currentfill}{rgb}{0.501961,0.501961,0.501961}%
\pgfsetfillcolor{currentfill}%
\pgfsetfillopacity{0.700000}%
\pgfsetlinewidth{0.000000pt}%
\definecolor{currentstroke}{rgb}{0.000000,0.000000,0.000000}%
\pgfsetstrokecolor{currentstroke}%
\pgfsetstrokeopacity{0.700000}%
\pgfsetdash{}{0pt}%
\pgfpathmoveto{\pgfqpoint{6.528587in}{0.550000in}}%
\pgfpathlineto{\pgfqpoint{6.556108in}{0.550000in}}%
\pgfpathlineto{\pgfqpoint{6.556108in}{0.591936in}}%
\pgfpathlineto{\pgfqpoint{6.528587in}{0.591936in}}%
\pgfpathlineto{\pgfqpoint{6.528587in}{0.550000in}}%
\pgfpathclose%
\pgfusepath{fill}%
\end{pgfscope}%
\begin{pgfscope}%
\pgfpathrectangle{\pgfqpoint{1.250000in}{0.550000in}}{\pgfqpoint{7.750000in}{3.850000in}}%
\pgfusepath{clip}%
\pgfsetbuttcap%
\pgfsetmiterjoin%
\definecolor{currentfill}{rgb}{0.501961,0.501961,0.501961}%
\pgfsetfillcolor{currentfill}%
\pgfsetfillopacity{0.700000}%
\pgfsetlinewidth{0.000000pt}%
\definecolor{currentstroke}{rgb}{0.000000,0.000000,0.000000}%
\pgfsetstrokecolor{currentstroke}%
\pgfsetstrokeopacity{0.700000}%
\pgfsetdash{}{0pt}%
\pgfpathmoveto{\pgfqpoint{6.556108in}{0.550000in}}%
\pgfpathlineto{\pgfqpoint{6.583629in}{0.550000in}}%
\pgfpathlineto{\pgfqpoint{6.583629in}{0.592332in}}%
\pgfpathlineto{\pgfqpoint{6.556108in}{0.592332in}}%
\pgfpathlineto{\pgfqpoint{6.556108in}{0.550000in}}%
\pgfpathclose%
\pgfusepath{fill}%
\end{pgfscope}%
\begin{pgfscope}%
\pgfpathrectangle{\pgfqpoint{1.250000in}{0.550000in}}{\pgfqpoint{7.750000in}{3.850000in}}%
\pgfusepath{clip}%
\pgfsetbuttcap%
\pgfsetmiterjoin%
\definecolor{currentfill}{rgb}{0.501961,0.501961,0.501961}%
\pgfsetfillcolor{currentfill}%
\pgfsetfillopacity{0.700000}%
\pgfsetlinewidth{0.000000pt}%
\definecolor{currentstroke}{rgb}{0.000000,0.000000,0.000000}%
\pgfsetstrokecolor{currentstroke}%
\pgfsetstrokeopacity{0.700000}%
\pgfsetdash{}{0pt}%
\pgfpathmoveto{\pgfqpoint{6.583629in}{0.550000in}}%
\pgfpathlineto{\pgfqpoint{6.611151in}{0.550000in}}%
\pgfpathlineto{\pgfqpoint{6.611151in}{0.595497in}}%
\pgfpathlineto{\pgfqpoint{6.583629in}{0.595497in}}%
\pgfpathlineto{\pgfqpoint{6.583629in}{0.550000in}}%
\pgfpathclose%
\pgfusepath{fill}%
\end{pgfscope}%
\begin{pgfscope}%
\pgfpathrectangle{\pgfqpoint{1.250000in}{0.550000in}}{\pgfqpoint{7.750000in}{3.850000in}}%
\pgfusepath{clip}%
\pgfsetbuttcap%
\pgfsetmiterjoin%
\definecolor{currentfill}{rgb}{0.501961,0.501961,0.501961}%
\pgfsetfillcolor{currentfill}%
\pgfsetfillopacity{0.700000}%
\pgfsetlinewidth{0.000000pt}%
\definecolor{currentstroke}{rgb}{0.000000,0.000000,0.000000}%
\pgfsetstrokecolor{currentstroke}%
\pgfsetstrokeopacity{0.700000}%
\pgfsetdash{}{0pt}%
\pgfpathmoveto{\pgfqpoint{6.611151in}{0.550000in}}%
\pgfpathlineto{\pgfqpoint{6.638672in}{0.550000in}}%
\pgfpathlineto{\pgfqpoint{6.638672in}{0.593915in}}%
\pgfpathlineto{\pgfqpoint{6.611151in}{0.593915in}}%
\pgfpathlineto{\pgfqpoint{6.611151in}{0.550000in}}%
\pgfpathclose%
\pgfusepath{fill}%
\end{pgfscope}%
\begin{pgfscope}%
\pgfpathrectangle{\pgfqpoint{1.250000in}{0.550000in}}{\pgfqpoint{7.750000in}{3.850000in}}%
\pgfusepath{clip}%
\pgfsetbuttcap%
\pgfsetmiterjoin%
\definecolor{currentfill}{rgb}{0.501961,0.501961,0.501961}%
\pgfsetfillcolor{currentfill}%
\pgfsetfillopacity{0.700000}%
\pgfsetlinewidth{0.000000pt}%
\definecolor{currentstroke}{rgb}{0.000000,0.000000,0.000000}%
\pgfsetstrokecolor{currentstroke}%
\pgfsetstrokeopacity{0.700000}%
\pgfsetdash{}{0pt}%
\pgfpathmoveto{\pgfqpoint{6.638672in}{0.550000in}}%
\pgfpathlineto{\pgfqpoint{6.666193in}{0.550000in}}%
\pgfpathlineto{\pgfqpoint{6.666193in}{0.586398in}}%
\pgfpathlineto{\pgfqpoint{6.638672in}{0.586398in}}%
\pgfpathlineto{\pgfqpoint{6.638672in}{0.550000in}}%
\pgfpathclose%
\pgfusepath{fill}%
\end{pgfscope}%
\begin{pgfscope}%
\pgfpathrectangle{\pgfqpoint{1.250000in}{0.550000in}}{\pgfqpoint{7.750000in}{3.850000in}}%
\pgfusepath{clip}%
\pgfsetbuttcap%
\pgfsetmiterjoin%
\definecolor{currentfill}{rgb}{0.501961,0.501961,0.501961}%
\pgfsetfillcolor{currentfill}%
\pgfsetfillopacity{0.700000}%
\pgfsetlinewidth{0.000000pt}%
\definecolor{currentstroke}{rgb}{0.000000,0.000000,0.000000}%
\pgfsetstrokecolor{currentstroke}%
\pgfsetstrokeopacity{0.700000}%
\pgfsetdash{}{0pt}%
\pgfpathmoveto{\pgfqpoint{6.666193in}{0.550000in}}%
\pgfpathlineto{\pgfqpoint{6.693714in}{0.550000in}}%
\pgfpathlineto{\pgfqpoint{6.693714in}{0.585606in}}%
\pgfpathlineto{\pgfqpoint{6.666193in}{0.585606in}}%
\pgfpathlineto{\pgfqpoint{6.666193in}{0.550000in}}%
\pgfpathclose%
\pgfusepath{fill}%
\end{pgfscope}%
\begin{pgfscope}%
\pgfpathrectangle{\pgfqpoint{1.250000in}{0.550000in}}{\pgfqpoint{7.750000in}{3.850000in}}%
\pgfusepath{clip}%
\pgfsetbuttcap%
\pgfsetmiterjoin%
\definecolor{currentfill}{rgb}{0.501961,0.501961,0.501961}%
\pgfsetfillcolor{currentfill}%
\pgfsetfillopacity{0.700000}%
\pgfsetlinewidth{0.000000pt}%
\definecolor{currentstroke}{rgb}{0.000000,0.000000,0.000000}%
\pgfsetstrokecolor{currentstroke}%
\pgfsetstrokeopacity{0.700000}%
\pgfsetdash{}{0pt}%
\pgfpathmoveto{\pgfqpoint{6.693714in}{0.550000in}}%
\pgfpathlineto{\pgfqpoint{6.721236in}{0.550000in}}%
\pgfpathlineto{\pgfqpoint{6.721236in}{0.590354in}}%
\pgfpathlineto{\pgfqpoint{6.693714in}{0.590354in}}%
\pgfpathlineto{\pgfqpoint{6.693714in}{0.550000in}}%
\pgfpathclose%
\pgfusepath{fill}%
\end{pgfscope}%
\begin{pgfscope}%
\pgfpathrectangle{\pgfqpoint{1.250000in}{0.550000in}}{\pgfqpoint{7.750000in}{3.850000in}}%
\pgfusepath{clip}%
\pgfsetbuttcap%
\pgfsetmiterjoin%
\definecolor{currentfill}{rgb}{0.501961,0.501961,0.501961}%
\pgfsetfillcolor{currentfill}%
\pgfsetfillopacity{0.700000}%
\pgfsetlinewidth{0.000000pt}%
\definecolor{currentstroke}{rgb}{0.000000,0.000000,0.000000}%
\pgfsetstrokecolor{currentstroke}%
\pgfsetstrokeopacity{0.700000}%
\pgfsetdash{}{0pt}%
\pgfpathmoveto{\pgfqpoint{6.721236in}{0.550000in}}%
\pgfpathlineto{\pgfqpoint{6.748757in}{0.550000in}}%
\pgfpathlineto{\pgfqpoint{6.748757in}{0.591936in}}%
\pgfpathlineto{\pgfqpoint{6.721236in}{0.591936in}}%
\pgfpathlineto{\pgfqpoint{6.721236in}{0.550000in}}%
\pgfpathclose%
\pgfusepath{fill}%
\end{pgfscope}%
\begin{pgfscope}%
\pgfpathrectangle{\pgfqpoint{1.250000in}{0.550000in}}{\pgfqpoint{7.750000in}{3.850000in}}%
\pgfusepath{clip}%
\pgfsetbuttcap%
\pgfsetmiterjoin%
\definecolor{currentfill}{rgb}{0.501961,0.501961,0.501961}%
\pgfsetfillcolor{currentfill}%
\pgfsetfillopacity{0.700000}%
\pgfsetlinewidth{0.000000pt}%
\definecolor{currentstroke}{rgb}{0.000000,0.000000,0.000000}%
\pgfsetstrokecolor{currentstroke}%
\pgfsetstrokeopacity{0.700000}%
\pgfsetdash{}{0pt}%
\pgfpathmoveto{\pgfqpoint{6.748757in}{0.550000in}}%
\pgfpathlineto{\pgfqpoint{6.776278in}{0.550000in}}%
\pgfpathlineto{\pgfqpoint{6.776278in}{0.550000in}}%
\pgfpathlineto{\pgfqpoint{6.748757in}{0.550000in}}%
\pgfpathlineto{\pgfqpoint{6.748757in}{0.550000in}}%
\pgfpathclose%
\pgfusepath{fill}%
\end{pgfscope}%
\begin{pgfscope}%
\pgfpathrectangle{\pgfqpoint{1.250000in}{0.550000in}}{\pgfqpoint{7.750000in}{3.850000in}}%
\pgfusepath{clip}%
\pgfsetbuttcap%
\pgfsetmiterjoin%
\definecolor{currentfill}{rgb}{0.501961,0.501961,0.501961}%
\pgfsetfillcolor{currentfill}%
\pgfsetfillopacity{0.700000}%
\pgfsetlinewidth{0.000000pt}%
\definecolor{currentstroke}{rgb}{0.000000,0.000000,0.000000}%
\pgfsetstrokecolor{currentstroke}%
\pgfsetstrokeopacity{0.700000}%
\pgfsetdash{}{0pt}%
\pgfpathmoveto{\pgfqpoint{6.776278in}{0.550000in}}%
\pgfpathlineto{\pgfqpoint{6.803800in}{0.550000in}}%
\pgfpathlineto{\pgfqpoint{6.803800in}{0.595101in}}%
\pgfpathlineto{\pgfqpoint{6.776278in}{0.595101in}}%
\pgfpathlineto{\pgfqpoint{6.776278in}{0.550000in}}%
\pgfpathclose%
\pgfusepath{fill}%
\end{pgfscope}%
\begin{pgfscope}%
\pgfpathrectangle{\pgfqpoint{1.250000in}{0.550000in}}{\pgfqpoint{7.750000in}{3.850000in}}%
\pgfusepath{clip}%
\pgfsetbuttcap%
\pgfsetmiterjoin%
\definecolor{currentfill}{rgb}{0.501961,0.501961,0.501961}%
\pgfsetfillcolor{currentfill}%
\pgfsetfillopacity{0.700000}%
\pgfsetlinewidth{0.000000pt}%
\definecolor{currentstroke}{rgb}{0.000000,0.000000,0.000000}%
\pgfsetstrokecolor{currentstroke}%
\pgfsetstrokeopacity{0.700000}%
\pgfsetdash{}{0pt}%
\pgfpathmoveto{\pgfqpoint{6.803800in}{0.550000in}}%
\pgfpathlineto{\pgfqpoint{6.831321in}{0.550000in}}%
\pgfpathlineto{\pgfqpoint{6.831321in}{0.592728in}}%
\pgfpathlineto{\pgfqpoint{6.803800in}{0.592728in}}%
\pgfpathlineto{\pgfqpoint{6.803800in}{0.550000in}}%
\pgfpathclose%
\pgfusepath{fill}%
\end{pgfscope}%
\begin{pgfscope}%
\pgfpathrectangle{\pgfqpoint{1.250000in}{0.550000in}}{\pgfqpoint{7.750000in}{3.850000in}}%
\pgfusepath{clip}%
\pgfsetbuttcap%
\pgfsetmiterjoin%
\definecolor{currentfill}{rgb}{0.501961,0.501961,0.501961}%
\pgfsetfillcolor{currentfill}%
\pgfsetfillopacity{0.700000}%
\pgfsetlinewidth{0.000000pt}%
\definecolor{currentstroke}{rgb}{0.000000,0.000000,0.000000}%
\pgfsetstrokecolor{currentstroke}%
\pgfsetstrokeopacity{0.700000}%
\pgfsetdash{}{0pt}%
\pgfpathmoveto{\pgfqpoint{6.831321in}{0.550000in}}%
\pgfpathlineto{\pgfqpoint{6.858842in}{0.550000in}}%
\pgfpathlineto{\pgfqpoint{6.858842in}{0.595101in}}%
\pgfpathlineto{\pgfqpoint{6.831321in}{0.595101in}}%
\pgfpathlineto{\pgfqpoint{6.831321in}{0.550000in}}%
\pgfpathclose%
\pgfusepath{fill}%
\end{pgfscope}%
\begin{pgfscope}%
\pgfpathrectangle{\pgfqpoint{1.250000in}{0.550000in}}{\pgfqpoint{7.750000in}{3.850000in}}%
\pgfusepath{clip}%
\pgfsetbuttcap%
\pgfsetmiterjoin%
\definecolor{currentfill}{rgb}{0.501961,0.501961,0.501961}%
\pgfsetfillcolor{currentfill}%
\pgfsetfillopacity{0.700000}%
\pgfsetlinewidth{0.000000pt}%
\definecolor{currentstroke}{rgb}{0.000000,0.000000,0.000000}%
\pgfsetstrokecolor{currentstroke}%
\pgfsetstrokeopacity{0.700000}%
\pgfsetdash{}{0pt}%
\pgfpathmoveto{\pgfqpoint{6.858842in}{0.550000in}}%
\pgfpathlineto{\pgfqpoint{6.886364in}{0.550000in}}%
\pgfpathlineto{\pgfqpoint{6.886364in}{0.591145in}}%
\pgfpathlineto{\pgfqpoint{6.858842in}{0.591145in}}%
\pgfpathlineto{\pgfqpoint{6.858842in}{0.550000in}}%
\pgfpathclose%
\pgfusepath{fill}%
\end{pgfscope}%
\begin{pgfscope}%
\pgfpathrectangle{\pgfqpoint{1.250000in}{0.550000in}}{\pgfqpoint{7.750000in}{3.850000in}}%
\pgfusepath{clip}%
\pgfsetbuttcap%
\pgfsetmiterjoin%
\definecolor{currentfill}{rgb}{0.501961,0.501961,0.501961}%
\pgfsetfillcolor{currentfill}%
\pgfsetfillopacity{0.700000}%
\pgfsetlinewidth{0.000000pt}%
\definecolor{currentstroke}{rgb}{0.000000,0.000000,0.000000}%
\pgfsetstrokecolor{currentstroke}%
\pgfsetstrokeopacity{0.700000}%
\pgfsetdash{}{0pt}%
\pgfpathmoveto{\pgfqpoint{6.886364in}{0.550000in}}%
\pgfpathlineto{\pgfqpoint{6.913885in}{0.550000in}}%
\pgfpathlineto{\pgfqpoint{6.913885in}{0.590354in}}%
\pgfpathlineto{\pgfqpoint{6.886364in}{0.590354in}}%
\pgfpathlineto{\pgfqpoint{6.886364in}{0.550000in}}%
\pgfpathclose%
\pgfusepath{fill}%
\end{pgfscope}%
\begin{pgfscope}%
\pgfpathrectangle{\pgfqpoint{1.250000in}{0.550000in}}{\pgfqpoint{7.750000in}{3.850000in}}%
\pgfusepath{clip}%
\pgfsetbuttcap%
\pgfsetmiterjoin%
\definecolor{currentfill}{rgb}{0.501961,0.501961,0.501961}%
\pgfsetfillcolor{currentfill}%
\pgfsetfillopacity{0.700000}%
\pgfsetlinewidth{0.000000pt}%
\definecolor{currentstroke}{rgb}{0.000000,0.000000,0.000000}%
\pgfsetstrokecolor{currentstroke}%
\pgfsetstrokeopacity{0.700000}%
\pgfsetdash{}{0pt}%
\pgfpathmoveto{\pgfqpoint{6.913885in}{0.550000in}}%
\pgfpathlineto{\pgfqpoint{6.941406in}{0.550000in}}%
\pgfpathlineto{\pgfqpoint{6.941406in}{0.587980in}}%
\pgfpathlineto{\pgfqpoint{6.913885in}{0.587980in}}%
\pgfpathlineto{\pgfqpoint{6.913885in}{0.550000in}}%
\pgfpathclose%
\pgfusepath{fill}%
\end{pgfscope}%
\begin{pgfscope}%
\pgfpathrectangle{\pgfqpoint{1.250000in}{0.550000in}}{\pgfqpoint{7.750000in}{3.850000in}}%
\pgfusepath{clip}%
\pgfsetbuttcap%
\pgfsetmiterjoin%
\definecolor{currentfill}{rgb}{0.501961,0.501961,0.501961}%
\pgfsetfillcolor{currentfill}%
\pgfsetfillopacity{0.700000}%
\pgfsetlinewidth{0.000000pt}%
\definecolor{currentstroke}{rgb}{0.000000,0.000000,0.000000}%
\pgfsetstrokecolor{currentstroke}%
\pgfsetstrokeopacity{0.700000}%
\pgfsetdash{}{0pt}%
\pgfpathmoveto{\pgfqpoint{6.941406in}{0.550000in}}%
\pgfpathlineto{\pgfqpoint{6.968928in}{0.550000in}}%
\pgfpathlineto{\pgfqpoint{6.968928in}{0.585211in}}%
\pgfpathlineto{\pgfqpoint{6.941406in}{0.585211in}}%
\pgfpathlineto{\pgfqpoint{6.941406in}{0.550000in}}%
\pgfpathclose%
\pgfusepath{fill}%
\end{pgfscope}%
\begin{pgfscope}%
\pgfpathrectangle{\pgfqpoint{1.250000in}{0.550000in}}{\pgfqpoint{7.750000in}{3.850000in}}%
\pgfusepath{clip}%
\pgfsetbuttcap%
\pgfsetmiterjoin%
\definecolor{currentfill}{rgb}{0.501961,0.501961,0.501961}%
\pgfsetfillcolor{currentfill}%
\pgfsetfillopacity{0.700000}%
\pgfsetlinewidth{0.000000pt}%
\definecolor{currentstroke}{rgb}{0.000000,0.000000,0.000000}%
\pgfsetstrokecolor{currentstroke}%
\pgfsetstrokeopacity{0.700000}%
\pgfsetdash{}{0pt}%
\pgfpathmoveto{\pgfqpoint{6.968928in}{0.550000in}}%
\pgfpathlineto{\pgfqpoint{6.996449in}{0.550000in}}%
\pgfpathlineto{\pgfqpoint{6.996449in}{0.587189in}}%
\pgfpathlineto{\pgfqpoint{6.968928in}{0.587189in}}%
\pgfpathlineto{\pgfqpoint{6.968928in}{0.550000in}}%
\pgfpathclose%
\pgfusepath{fill}%
\end{pgfscope}%
\begin{pgfscope}%
\pgfpathrectangle{\pgfqpoint{1.250000in}{0.550000in}}{\pgfqpoint{7.750000in}{3.850000in}}%
\pgfusepath{clip}%
\pgfsetbuttcap%
\pgfsetmiterjoin%
\definecolor{currentfill}{rgb}{0.501961,0.501961,0.501961}%
\pgfsetfillcolor{currentfill}%
\pgfsetfillopacity{0.700000}%
\pgfsetlinewidth{0.000000pt}%
\definecolor{currentstroke}{rgb}{0.000000,0.000000,0.000000}%
\pgfsetstrokecolor{currentstroke}%
\pgfsetstrokeopacity{0.700000}%
\pgfsetdash{}{0pt}%
\pgfpathmoveto{\pgfqpoint{6.996449in}{0.550000in}}%
\pgfpathlineto{\pgfqpoint{7.023970in}{0.550000in}}%
\pgfpathlineto{\pgfqpoint{7.023970in}{0.591936in}}%
\pgfpathlineto{\pgfqpoint{6.996449in}{0.591936in}}%
\pgfpathlineto{\pgfqpoint{6.996449in}{0.550000in}}%
\pgfpathclose%
\pgfusepath{fill}%
\end{pgfscope}%
\begin{pgfscope}%
\pgfpathrectangle{\pgfqpoint{1.250000in}{0.550000in}}{\pgfqpoint{7.750000in}{3.850000in}}%
\pgfusepath{clip}%
\pgfsetbuttcap%
\pgfsetmiterjoin%
\definecolor{currentfill}{rgb}{0.501961,0.501961,0.501961}%
\pgfsetfillcolor{currentfill}%
\pgfsetfillopacity{0.700000}%
\pgfsetlinewidth{0.000000pt}%
\definecolor{currentstroke}{rgb}{0.000000,0.000000,0.000000}%
\pgfsetstrokecolor{currentstroke}%
\pgfsetstrokeopacity{0.700000}%
\pgfsetdash{}{0pt}%
\pgfpathmoveto{\pgfqpoint{7.023970in}{0.550000in}}%
\pgfpathlineto{\pgfqpoint{7.051491in}{0.550000in}}%
\pgfpathlineto{\pgfqpoint{7.051491in}{0.586398in}}%
\pgfpathlineto{\pgfqpoint{7.023970in}{0.586398in}}%
\pgfpathlineto{\pgfqpoint{7.023970in}{0.550000in}}%
\pgfpathclose%
\pgfusepath{fill}%
\end{pgfscope}%
\begin{pgfscope}%
\pgfpathrectangle{\pgfqpoint{1.250000in}{0.550000in}}{\pgfqpoint{7.750000in}{3.850000in}}%
\pgfusepath{clip}%
\pgfsetbuttcap%
\pgfsetmiterjoin%
\definecolor{currentfill}{rgb}{0.501961,0.501961,0.501961}%
\pgfsetfillcolor{currentfill}%
\pgfsetfillopacity{0.700000}%
\pgfsetlinewidth{0.000000pt}%
\definecolor{currentstroke}{rgb}{0.000000,0.000000,0.000000}%
\pgfsetstrokecolor{currentstroke}%
\pgfsetstrokeopacity{0.700000}%
\pgfsetdash{}{0pt}%
\pgfpathmoveto{\pgfqpoint{7.051491in}{0.550000in}}%
\pgfpathlineto{\pgfqpoint{7.079013in}{0.550000in}}%
\pgfpathlineto{\pgfqpoint{7.079013in}{0.593915in}}%
\pgfpathlineto{\pgfqpoint{7.051491in}{0.593915in}}%
\pgfpathlineto{\pgfqpoint{7.051491in}{0.550000in}}%
\pgfpathclose%
\pgfusepath{fill}%
\end{pgfscope}%
\begin{pgfscope}%
\pgfpathrectangle{\pgfqpoint{1.250000in}{0.550000in}}{\pgfqpoint{7.750000in}{3.850000in}}%
\pgfusepath{clip}%
\pgfsetbuttcap%
\pgfsetmiterjoin%
\definecolor{currentfill}{rgb}{0.501961,0.501961,0.501961}%
\pgfsetfillcolor{currentfill}%
\pgfsetfillopacity{0.700000}%
\pgfsetlinewidth{0.000000pt}%
\definecolor{currentstroke}{rgb}{0.000000,0.000000,0.000000}%
\pgfsetstrokecolor{currentstroke}%
\pgfsetstrokeopacity{0.700000}%
\pgfsetdash{}{0pt}%
\pgfpathmoveto{\pgfqpoint{7.079013in}{0.550000in}}%
\pgfpathlineto{\pgfqpoint{7.106534in}{0.550000in}}%
\pgfpathlineto{\pgfqpoint{7.106534in}{0.601431in}}%
\pgfpathlineto{\pgfqpoint{7.079013in}{0.601431in}}%
\pgfpathlineto{\pgfqpoint{7.079013in}{0.550000in}}%
\pgfpathclose%
\pgfusepath{fill}%
\end{pgfscope}%
\begin{pgfscope}%
\pgfpathrectangle{\pgfqpoint{1.250000in}{0.550000in}}{\pgfqpoint{7.750000in}{3.850000in}}%
\pgfusepath{clip}%
\pgfsetbuttcap%
\pgfsetmiterjoin%
\definecolor{currentfill}{rgb}{0.501961,0.501961,0.501961}%
\pgfsetfillcolor{currentfill}%
\pgfsetfillopacity{0.700000}%
\pgfsetlinewidth{0.000000pt}%
\definecolor{currentstroke}{rgb}{0.000000,0.000000,0.000000}%
\pgfsetstrokecolor{currentstroke}%
\pgfsetstrokeopacity{0.700000}%
\pgfsetdash{}{0pt}%
\pgfpathmoveto{\pgfqpoint{7.106534in}{0.550000in}}%
\pgfpathlineto{\pgfqpoint{7.134055in}{0.550000in}}%
\pgfpathlineto{\pgfqpoint{7.134055in}{0.594310in}}%
\pgfpathlineto{\pgfqpoint{7.106534in}{0.594310in}}%
\pgfpathlineto{\pgfqpoint{7.106534in}{0.550000in}}%
\pgfpathclose%
\pgfusepath{fill}%
\end{pgfscope}%
\begin{pgfscope}%
\pgfpathrectangle{\pgfqpoint{1.250000in}{0.550000in}}{\pgfqpoint{7.750000in}{3.850000in}}%
\pgfusepath{clip}%
\pgfsetbuttcap%
\pgfsetmiterjoin%
\definecolor{currentfill}{rgb}{0.501961,0.501961,0.501961}%
\pgfsetfillcolor{currentfill}%
\pgfsetfillopacity{0.700000}%
\pgfsetlinewidth{0.000000pt}%
\definecolor{currentstroke}{rgb}{0.000000,0.000000,0.000000}%
\pgfsetstrokecolor{currentstroke}%
\pgfsetstrokeopacity{0.700000}%
\pgfsetdash{}{0pt}%
\pgfpathmoveto{\pgfqpoint{7.134055in}{0.550000in}}%
\pgfpathlineto{\pgfqpoint{7.161577in}{0.550000in}}%
\pgfpathlineto{\pgfqpoint{7.161577in}{0.550000in}}%
\pgfpathlineto{\pgfqpoint{7.134055in}{0.550000in}}%
\pgfpathlineto{\pgfqpoint{7.134055in}{0.550000in}}%
\pgfpathclose%
\pgfusepath{fill}%
\end{pgfscope}%
\begin{pgfscope}%
\pgfpathrectangle{\pgfqpoint{1.250000in}{0.550000in}}{\pgfqpoint{7.750000in}{3.850000in}}%
\pgfusepath{clip}%
\pgfsetbuttcap%
\pgfsetmiterjoin%
\definecolor{currentfill}{rgb}{0.501961,0.501961,0.501961}%
\pgfsetfillcolor{currentfill}%
\pgfsetfillopacity{0.700000}%
\pgfsetlinewidth{0.000000pt}%
\definecolor{currentstroke}{rgb}{0.000000,0.000000,0.000000}%
\pgfsetstrokecolor{currentstroke}%
\pgfsetstrokeopacity{0.700000}%
\pgfsetdash{}{0pt}%
\pgfpathmoveto{\pgfqpoint{7.161577in}{0.550000in}}%
\pgfpathlineto{\pgfqpoint{7.189098in}{0.550000in}}%
\pgfpathlineto{\pgfqpoint{7.189098in}{0.600245in}}%
\pgfpathlineto{\pgfqpoint{7.161577in}{0.600245in}}%
\pgfpathlineto{\pgfqpoint{7.161577in}{0.550000in}}%
\pgfpathclose%
\pgfusepath{fill}%
\end{pgfscope}%
\begin{pgfscope}%
\pgfpathrectangle{\pgfqpoint{1.250000in}{0.550000in}}{\pgfqpoint{7.750000in}{3.850000in}}%
\pgfusepath{clip}%
\pgfsetbuttcap%
\pgfsetmiterjoin%
\definecolor{currentfill}{rgb}{0.501961,0.501961,0.501961}%
\pgfsetfillcolor{currentfill}%
\pgfsetfillopacity{0.700000}%
\pgfsetlinewidth{0.000000pt}%
\definecolor{currentstroke}{rgb}{0.000000,0.000000,0.000000}%
\pgfsetstrokecolor{currentstroke}%
\pgfsetstrokeopacity{0.700000}%
\pgfsetdash{}{0pt}%
\pgfpathmoveto{\pgfqpoint{7.189098in}{0.550000in}}%
\pgfpathlineto{\pgfqpoint{7.216619in}{0.550000in}}%
\pgfpathlineto{\pgfqpoint{7.216619in}{0.596288in}}%
\pgfpathlineto{\pgfqpoint{7.189098in}{0.596288in}}%
\pgfpathlineto{\pgfqpoint{7.189098in}{0.550000in}}%
\pgfpathclose%
\pgfusepath{fill}%
\end{pgfscope}%
\begin{pgfscope}%
\pgfpathrectangle{\pgfqpoint{1.250000in}{0.550000in}}{\pgfqpoint{7.750000in}{3.850000in}}%
\pgfusepath{clip}%
\pgfsetbuttcap%
\pgfsetmiterjoin%
\definecolor{currentfill}{rgb}{0.501961,0.501961,0.501961}%
\pgfsetfillcolor{currentfill}%
\pgfsetfillopacity{0.700000}%
\pgfsetlinewidth{0.000000pt}%
\definecolor{currentstroke}{rgb}{0.000000,0.000000,0.000000}%
\pgfsetstrokecolor{currentstroke}%
\pgfsetstrokeopacity{0.700000}%
\pgfsetdash{}{0pt}%
\pgfpathmoveto{\pgfqpoint{7.216619in}{0.550000in}}%
\pgfpathlineto{\pgfqpoint{7.244141in}{0.550000in}}%
\pgfpathlineto{\pgfqpoint{7.244141in}{0.591936in}}%
\pgfpathlineto{\pgfqpoint{7.216619in}{0.591936in}}%
\pgfpathlineto{\pgfqpoint{7.216619in}{0.550000in}}%
\pgfpathclose%
\pgfusepath{fill}%
\end{pgfscope}%
\begin{pgfscope}%
\pgfpathrectangle{\pgfqpoint{1.250000in}{0.550000in}}{\pgfqpoint{7.750000in}{3.850000in}}%
\pgfusepath{clip}%
\pgfsetbuttcap%
\pgfsetmiterjoin%
\definecolor{currentfill}{rgb}{0.501961,0.501961,0.501961}%
\pgfsetfillcolor{currentfill}%
\pgfsetfillopacity{0.700000}%
\pgfsetlinewidth{0.000000pt}%
\definecolor{currentstroke}{rgb}{0.000000,0.000000,0.000000}%
\pgfsetstrokecolor{currentstroke}%
\pgfsetstrokeopacity{0.700000}%
\pgfsetdash{}{0pt}%
\pgfpathmoveto{\pgfqpoint{7.244141in}{0.550000in}}%
\pgfpathlineto{\pgfqpoint{7.271662in}{0.550000in}}%
\pgfpathlineto{\pgfqpoint{7.271662in}{0.593519in}}%
\pgfpathlineto{\pgfqpoint{7.244141in}{0.593519in}}%
\pgfpathlineto{\pgfqpoint{7.244141in}{0.550000in}}%
\pgfpathclose%
\pgfusepath{fill}%
\end{pgfscope}%
\begin{pgfscope}%
\pgfpathrectangle{\pgfqpoint{1.250000in}{0.550000in}}{\pgfqpoint{7.750000in}{3.850000in}}%
\pgfusepath{clip}%
\pgfsetbuttcap%
\pgfsetmiterjoin%
\definecolor{currentfill}{rgb}{0.501961,0.501961,0.501961}%
\pgfsetfillcolor{currentfill}%
\pgfsetfillopacity{0.700000}%
\pgfsetlinewidth{0.000000pt}%
\definecolor{currentstroke}{rgb}{0.000000,0.000000,0.000000}%
\pgfsetstrokecolor{currentstroke}%
\pgfsetstrokeopacity{0.700000}%
\pgfsetdash{}{0pt}%
\pgfpathmoveto{\pgfqpoint{7.271662in}{0.550000in}}%
\pgfpathlineto{\pgfqpoint{7.299183in}{0.550000in}}%
\pgfpathlineto{\pgfqpoint{7.299183in}{0.590750in}}%
\pgfpathlineto{\pgfqpoint{7.271662in}{0.590750in}}%
\pgfpathlineto{\pgfqpoint{7.271662in}{0.550000in}}%
\pgfpathclose%
\pgfusepath{fill}%
\end{pgfscope}%
\begin{pgfscope}%
\pgfpathrectangle{\pgfqpoint{1.250000in}{0.550000in}}{\pgfqpoint{7.750000in}{3.850000in}}%
\pgfusepath{clip}%
\pgfsetbuttcap%
\pgfsetmiterjoin%
\definecolor{currentfill}{rgb}{0.501961,0.501961,0.501961}%
\pgfsetfillcolor{currentfill}%
\pgfsetfillopacity{0.700000}%
\pgfsetlinewidth{0.000000pt}%
\definecolor{currentstroke}{rgb}{0.000000,0.000000,0.000000}%
\pgfsetstrokecolor{currentstroke}%
\pgfsetstrokeopacity{0.700000}%
\pgfsetdash{}{0pt}%
\pgfpathmoveto{\pgfqpoint{7.299183in}{0.550000in}}%
\pgfpathlineto{\pgfqpoint{7.326705in}{0.550000in}}%
\pgfpathlineto{\pgfqpoint{7.326705in}{0.586793in}}%
\pgfpathlineto{\pgfqpoint{7.299183in}{0.586793in}}%
\pgfpathlineto{\pgfqpoint{7.299183in}{0.550000in}}%
\pgfpathclose%
\pgfusepath{fill}%
\end{pgfscope}%
\begin{pgfscope}%
\pgfpathrectangle{\pgfqpoint{1.250000in}{0.550000in}}{\pgfqpoint{7.750000in}{3.850000in}}%
\pgfusepath{clip}%
\pgfsetbuttcap%
\pgfsetmiterjoin%
\definecolor{currentfill}{rgb}{0.501961,0.501961,0.501961}%
\pgfsetfillcolor{currentfill}%
\pgfsetfillopacity{0.700000}%
\pgfsetlinewidth{0.000000pt}%
\definecolor{currentstroke}{rgb}{0.000000,0.000000,0.000000}%
\pgfsetstrokecolor{currentstroke}%
\pgfsetstrokeopacity{0.700000}%
\pgfsetdash{}{0pt}%
\pgfpathmoveto{\pgfqpoint{7.326705in}{0.550000in}}%
\pgfpathlineto{\pgfqpoint{7.354226in}{0.550000in}}%
\pgfpathlineto{\pgfqpoint{7.354226in}{0.592332in}}%
\pgfpathlineto{\pgfqpoint{7.326705in}{0.592332in}}%
\pgfpathlineto{\pgfqpoint{7.326705in}{0.550000in}}%
\pgfpathclose%
\pgfusepath{fill}%
\end{pgfscope}%
\begin{pgfscope}%
\pgfpathrectangle{\pgfqpoint{1.250000in}{0.550000in}}{\pgfqpoint{7.750000in}{3.850000in}}%
\pgfusepath{clip}%
\pgfsetbuttcap%
\pgfsetmiterjoin%
\definecolor{currentfill}{rgb}{0.501961,0.501961,0.501961}%
\pgfsetfillcolor{currentfill}%
\pgfsetfillopacity{0.700000}%
\pgfsetlinewidth{0.000000pt}%
\definecolor{currentstroke}{rgb}{0.000000,0.000000,0.000000}%
\pgfsetstrokecolor{currentstroke}%
\pgfsetstrokeopacity{0.700000}%
\pgfsetdash{}{0pt}%
\pgfpathmoveto{\pgfqpoint{7.354226in}{0.550000in}}%
\pgfpathlineto{\pgfqpoint{7.381747in}{0.550000in}}%
\pgfpathlineto{\pgfqpoint{7.381747in}{0.592728in}}%
\pgfpathlineto{\pgfqpoint{7.354226in}{0.592728in}}%
\pgfpathlineto{\pgfqpoint{7.354226in}{0.550000in}}%
\pgfpathclose%
\pgfusepath{fill}%
\end{pgfscope}%
\begin{pgfscope}%
\pgfpathrectangle{\pgfqpoint{1.250000in}{0.550000in}}{\pgfqpoint{7.750000in}{3.850000in}}%
\pgfusepath{clip}%
\pgfsetbuttcap%
\pgfsetmiterjoin%
\definecolor{currentfill}{rgb}{0.501961,0.501961,0.501961}%
\pgfsetfillcolor{currentfill}%
\pgfsetfillopacity{0.700000}%
\pgfsetlinewidth{0.000000pt}%
\definecolor{currentstroke}{rgb}{0.000000,0.000000,0.000000}%
\pgfsetstrokecolor{currentstroke}%
\pgfsetstrokeopacity{0.700000}%
\pgfsetdash{}{0pt}%
\pgfpathmoveto{\pgfqpoint{7.381747in}{0.550000in}}%
\pgfpathlineto{\pgfqpoint{7.409268in}{0.550000in}}%
\pgfpathlineto{\pgfqpoint{7.409268in}{0.593915in}}%
\pgfpathlineto{\pgfqpoint{7.381747in}{0.593915in}}%
\pgfpathlineto{\pgfqpoint{7.381747in}{0.550000in}}%
\pgfpathclose%
\pgfusepath{fill}%
\end{pgfscope}%
\begin{pgfscope}%
\pgfpathrectangle{\pgfqpoint{1.250000in}{0.550000in}}{\pgfqpoint{7.750000in}{3.850000in}}%
\pgfusepath{clip}%
\pgfsetbuttcap%
\pgfsetmiterjoin%
\definecolor{currentfill}{rgb}{0.501961,0.501961,0.501961}%
\pgfsetfillcolor{currentfill}%
\pgfsetfillopacity{0.700000}%
\pgfsetlinewidth{0.000000pt}%
\definecolor{currentstroke}{rgb}{0.000000,0.000000,0.000000}%
\pgfsetstrokecolor{currentstroke}%
\pgfsetstrokeopacity{0.700000}%
\pgfsetdash{}{0pt}%
\pgfpathmoveto{\pgfqpoint{7.409268in}{0.550000in}}%
\pgfpathlineto{\pgfqpoint{7.436790in}{0.550000in}}%
\pgfpathlineto{\pgfqpoint{7.436790in}{0.598662in}}%
\pgfpathlineto{\pgfqpoint{7.409268in}{0.598662in}}%
\pgfpathlineto{\pgfqpoint{7.409268in}{0.550000in}}%
\pgfpathclose%
\pgfusepath{fill}%
\end{pgfscope}%
\begin{pgfscope}%
\pgfpathrectangle{\pgfqpoint{1.250000in}{0.550000in}}{\pgfqpoint{7.750000in}{3.850000in}}%
\pgfusepath{clip}%
\pgfsetbuttcap%
\pgfsetmiterjoin%
\definecolor{currentfill}{rgb}{0.501961,0.501961,0.501961}%
\pgfsetfillcolor{currentfill}%
\pgfsetfillopacity{0.700000}%
\pgfsetlinewidth{0.000000pt}%
\definecolor{currentstroke}{rgb}{0.000000,0.000000,0.000000}%
\pgfsetstrokecolor{currentstroke}%
\pgfsetstrokeopacity{0.700000}%
\pgfsetdash{}{0pt}%
\pgfpathmoveto{\pgfqpoint{7.436790in}{0.550000in}}%
\pgfpathlineto{\pgfqpoint{7.464311in}{0.550000in}}%
\pgfpathlineto{\pgfqpoint{7.464311in}{0.550000in}}%
\pgfpathlineto{\pgfqpoint{7.436790in}{0.550000in}}%
\pgfpathlineto{\pgfqpoint{7.436790in}{0.550000in}}%
\pgfpathclose%
\pgfusepath{fill}%
\end{pgfscope}%
\begin{pgfscope}%
\pgfpathrectangle{\pgfqpoint{1.250000in}{0.550000in}}{\pgfqpoint{7.750000in}{3.850000in}}%
\pgfusepath{clip}%
\pgfsetbuttcap%
\pgfsetmiterjoin%
\definecolor{currentfill}{rgb}{0.501961,0.501961,0.501961}%
\pgfsetfillcolor{currentfill}%
\pgfsetfillopacity{0.700000}%
\pgfsetlinewidth{0.000000pt}%
\definecolor{currentstroke}{rgb}{0.000000,0.000000,0.000000}%
\pgfsetstrokecolor{currentstroke}%
\pgfsetstrokeopacity{0.700000}%
\pgfsetdash{}{0pt}%
\pgfpathmoveto{\pgfqpoint{7.464311in}{0.550000in}}%
\pgfpathlineto{\pgfqpoint{7.491832in}{0.550000in}}%
\pgfpathlineto{\pgfqpoint{7.491832in}{0.598266in}}%
\pgfpathlineto{\pgfqpoint{7.464311in}{0.598266in}}%
\pgfpathlineto{\pgfqpoint{7.464311in}{0.550000in}}%
\pgfpathclose%
\pgfusepath{fill}%
\end{pgfscope}%
\begin{pgfscope}%
\pgfpathrectangle{\pgfqpoint{1.250000in}{0.550000in}}{\pgfqpoint{7.750000in}{3.850000in}}%
\pgfusepath{clip}%
\pgfsetbuttcap%
\pgfsetmiterjoin%
\definecolor{currentfill}{rgb}{0.501961,0.501961,0.501961}%
\pgfsetfillcolor{currentfill}%
\pgfsetfillopacity{0.700000}%
\pgfsetlinewidth{0.000000pt}%
\definecolor{currentstroke}{rgb}{0.000000,0.000000,0.000000}%
\pgfsetstrokecolor{currentstroke}%
\pgfsetstrokeopacity{0.700000}%
\pgfsetdash{}{0pt}%
\pgfpathmoveto{\pgfqpoint{7.491832in}{0.550000in}}%
\pgfpathlineto{\pgfqpoint{7.519354in}{0.550000in}}%
\pgfpathlineto{\pgfqpoint{7.519354in}{0.596684in}}%
\pgfpathlineto{\pgfqpoint{7.491832in}{0.596684in}}%
\pgfpathlineto{\pgfqpoint{7.491832in}{0.550000in}}%
\pgfpathclose%
\pgfusepath{fill}%
\end{pgfscope}%
\begin{pgfscope}%
\pgfpathrectangle{\pgfqpoint{1.250000in}{0.550000in}}{\pgfqpoint{7.750000in}{3.850000in}}%
\pgfusepath{clip}%
\pgfsetbuttcap%
\pgfsetmiterjoin%
\definecolor{currentfill}{rgb}{0.501961,0.501961,0.501961}%
\pgfsetfillcolor{currentfill}%
\pgfsetfillopacity{0.700000}%
\pgfsetlinewidth{0.000000pt}%
\definecolor{currentstroke}{rgb}{0.000000,0.000000,0.000000}%
\pgfsetstrokecolor{currentstroke}%
\pgfsetstrokeopacity{0.700000}%
\pgfsetdash{}{0pt}%
\pgfpathmoveto{\pgfqpoint{7.519354in}{0.550000in}}%
\pgfpathlineto{\pgfqpoint{7.546875in}{0.550000in}}%
\pgfpathlineto{\pgfqpoint{7.546875in}{0.589563in}}%
\pgfpathlineto{\pgfqpoint{7.519354in}{0.589563in}}%
\pgfpathlineto{\pgfqpoint{7.519354in}{0.550000in}}%
\pgfpathclose%
\pgfusepath{fill}%
\end{pgfscope}%
\begin{pgfscope}%
\pgfpathrectangle{\pgfqpoint{1.250000in}{0.550000in}}{\pgfqpoint{7.750000in}{3.850000in}}%
\pgfusepath{clip}%
\pgfsetbuttcap%
\pgfsetmiterjoin%
\definecolor{currentfill}{rgb}{0.501961,0.501961,0.501961}%
\pgfsetfillcolor{currentfill}%
\pgfsetfillopacity{0.700000}%
\pgfsetlinewidth{0.000000pt}%
\definecolor{currentstroke}{rgb}{0.000000,0.000000,0.000000}%
\pgfsetstrokecolor{currentstroke}%
\pgfsetstrokeopacity{0.700000}%
\pgfsetdash{}{0pt}%
\pgfpathmoveto{\pgfqpoint{7.546875in}{0.550000in}}%
\pgfpathlineto{\pgfqpoint{7.574396in}{0.550000in}}%
\pgfpathlineto{\pgfqpoint{7.574396in}{0.594706in}}%
\pgfpathlineto{\pgfqpoint{7.546875in}{0.594706in}}%
\pgfpathlineto{\pgfqpoint{7.546875in}{0.550000in}}%
\pgfpathclose%
\pgfusepath{fill}%
\end{pgfscope}%
\begin{pgfscope}%
\pgfpathrectangle{\pgfqpoint{1.250000in}{0.550000in}}{\pgfqpoint{7.750000in}{3.850000in}}%
\pgfusepath{clip}%
\pgfsetbuttcap%
\pgfsetmiterjoin%
\definecolor{currentfill}{rgb}{0.501961,0.501961,0.501961}%
\pgfsetfillcolor{currentfill}%
\pgfsetfillopacity{0.700000}%
\pgfsetlinewidth{0.000000pt}%
\definecolor{currentstroke}{rgb}{0.000000,0.000000,0.000000}%
\pgfsetstrokecolor{currentstroke}%
\pgfsetstrokeopacity{0.700000}%
\pgfsetdash{}{0pt}%
\pgfpathmoveto{\pgfqpoint{7.574396in}{0.550000in}}%
\pgfpathlineto{\pgfqpoint{7.601918in}{0.550000in}}%
\pgfpathlineto{\pgfqpoint{7.601918in}{0.595893in}}%
\pgfpathlineto{\pgfqpoint{7.574396in}{0.595893in}}%
\pgfpathlineto{\pgfqpoint{7.574396in}{0.550000in}}%
\pgfpathclose%
\pgfusepath{fill}%
\end{pgfscope}%
\begin{pgfscope}%
\pgfpathrectangle{\pgfqpoint{1.250000in}{0.550000in}}{\pgfqpoint{7.750000in}{3.850000in}}%
\pgfusepath{clip}%
\pgfsetbuttcap%
\pgfsetmiterjoin%
\definecolor{currentfill}{rgb}{0.501961,0.501961,0.501961}%
\pgfsetfillcolor{currentfill}%
\pgfsetfillopacity{0.700000}%
\pgfsetlinewidth{0.000000pt}%
\definecolor{currentstroke}{rgb}{0.000000,0.000000,0.000000}%
\pgfsetstrokecolor{currentstroke}%
\pgfsetstrokeopacity{0.700000}%
\pgfsetdash{}{0pt}%
\pgfpathmoveto{\pgfqpoint{7.601918in}{0.550000in}}%
\pgfpathlineto{\pgfqpoint{7.629439in}{0.550000in}}%
\pgfpathlineto{\pgfqpoint{7.629439in}{0.602223in}}%
\pgfpathlineto{\pgfqpoint{7.601918in}{0.602223in}}%
\pgfpathlineto{\pgfqpoint{7.601918in}{0.550000in}}%
\pgfpathclose%
\pgfusepath{fill}%
\end{pgfscope}%
\begin{pgfscope}%
\pgfpathrectangle{\pgfqpoint{1.250000in}{0.550000in}}{\pgfqpoint{7.750000in}{3.850000in}}%
\pgfusepath{clip}%
\pgfsetbuttcap%
\pgfsetmiterjoin%
\definecolor{currentfill}{rgb}{0.501961,0.501961,0.501961}%
\pgfsetfillcolor{currentfill}%
\pgfsetfillopacity{0.700000}%
\pgfsetlinewidth{0.000000pt}%
\definecolor{currentstroke}{rgb}{0.000000,0.000000,0.000000}%
\pgfsetstrokecolor{currentstroke}%
\pgfsetstrokeopacity{0.700000}%
\pgfsetdash{}{0pt}%
\pgfpathmoveto{\pgfqpoint{7.629439in}{0.550000in}}%
\pgfpathlineto{\pgfqpoint{7.656960in}{0.550000in}}%
\pgfpathlineto{\pgfqpoint{7.656960in}{0.550000in}}%
\pgfpathlineto{\pgfqpoint{7.629439in}{0.550000in}}%
\pgfpathlineto{\pgfqpoint{7.629439in}{0.550000in}}%
\pgfpathclose%
\pgfusepath{fill}%
\end{pgfscope}%
\begin{pgfscope}%
\pgfpathrectangle{\pgfqpoint{1.250000in}{0.550000in}}{\pgfqpoint{7.750000in}{3.850000in}}%
\pgfusepath{clip}%
\pgfsetbuttcap%
\pgfsetmiterjoin%
\definecolor{currentfill}{rgb}{0.501961,0.501961,0.501961}%
\pgfsetfillcolor{currentfill}%
\pgfsetfillopacity{0.700000}%
\pgfsetlinewidth{0.000000pt}%
\definecolor{currentstroke}{rgb}{0.000000,0.000000,0.000000}%
\pgfsetstrokecolor{currentstroke}%
\pgfsetstrokeopacity{0.700000}%
\pgfsetdash{}{0pt}%
\pgfpathmoveto{\pgfqpoint{7.656960in}{0.550000in}}%
\pgfpathlineto{\pgfqpoint{7.684482in}{0.550000in}}%
\pgfpathlineto{\pgfqpoint{7.684482in}{0.599453in}}%
\pgfpathlineto{\pgfqpoint{7.656960in}{0.599453in}}%
\pgfpathlineto{\pgfqpoint{7.656960in}{0.550000in}}%
\pgfpathclose%
\pgfusepath{fill}%
\end{pgfscope}%
\begin{pgfscope}%
\pgfpathrectangle{\pgfqpoint{1.250000in}{0.550000in}}{\pgfqpoint{7.750000in}{3.850000in}}%
\pgfusepath{clip}%
\pgfsetbuttcap%
\pgfsetmiterjoin%
\definecolor{currentfill}{rgb}{0.501961,0.501961,0.501961}%
\pgfsetfillcolor{currentfill}%
\pgfsetfillopacity{0.700000}%
\pgfsetlinewidth{0.000000pt}%
\definecolor{currentstroke}{rgb}{0.000000,0.000000,0.000000}%
\pgfsetstrokecolor{currentstroke}%
\pgfsetstrokeopacity{0.700000}%
\pgfsetdash{}{0pt}%
\pgfpathmoveto{\pgfqpoint{7.684482in}{0.550000in}}%
\pgfpathlineto{\pgfqpoint{7.712003in}{0.550000in}}%
\pgfpathlineto{\pgfqpoint{7.712003in}{0.602223in}}%
\pgfpathlineto{\pgfqpoint{7.684482in}{0.602223in}}%
\pgfpathlineto{\pgfqpoint{7.684482in}{0.550000in}}%
\pgfpathclose%
\pgfusepath{fill}%
\end{pgfscope}%
\begin{pgfscope}%
\pgfpathrectangle{\pgfqpoint{1.250000in}{0.550000in}}{\pgfqpoint{7.750000in}{3.850000in}}%
\pgfusepath{clip}%
\pgfsetbuttcap%
\pgfsetmiterjoin%
\definecolor{currentfill}{rgb}{0.501961,0.501961,0.501961}%
\pgfsetfillcolor{currentfill}%
\pgfsetfillopacity{0.700000}%
\pgfsetlinewidth{0.000000pt}%
\definecolor{currentstroke}{rgb}{0.000000,0.000000,0.000000}%
\pgfsetstrokecolor{currentstroke}%
\pgfsetstrokeopacity{0.700000}%
\pgfsetdash{}{0pt}%
\pgfpathmoveto{\pgfqpoint{7.712003in}{0.550000in}}%
\pgfpathlineto{\pgfqpoint{7.739524in}{0.550000in}}%
\pgfpathlineto{\pgfqpoint{7.739524in}{0.614883in}}%
\pgfpathlineto{\pgfqpoint{7.712003in}{0.614883in}}%
\pgfpathlineto{\pgfqpoint{7.712003in}{0.550000in}}%
\pgfpathclose%
\pgfusepath{fill}%
\end{pgfscope}%
\begin{pgfscope}%
\pgfpathrectangle{\pgfqpoint{1.250000in}{0.550000in}}{\pgfqpoint{7.750000in}{3.850000in}}%
\pgfusepath{clip}%
\pgfsetbuttcap%
\pgfsetmiterjoin%
\definecolor{currentfill}{rgb}{0.501961,0.501961,0.501961}%
\pgfsetfillcolor{currentfill}%
\pgfsetfillopacity{0.700000}%
\pgfsetlinewidth{0.000000pt}%
\definecolor{currentstroke}{rgb}{0.000000,0.000000,0.000000}%
\pgfsetstrokecolor{currentstroke}%
\pgfsetstrokeopacity{0.700000}%
\pgfsetdash{}{0pt}%
\pgfpathmoveto{\pgfqpoint{7.739524in}{0.550000in}}%
\pgfpathlineto{\pgfqpoint{7.767045in}{0.550000in}}%
\pgfpathlineto{\pgfqpoint{7.767045in}{0.550000in}}%
\pgfpathlineto{\pgfqpoint{7.739524in}{0.550000in}}%
\pgfpathlineto{\pgfqpoint{7.739524in}{0.550000in}}%
\pgfpathclose%
\pgfusepath{fill}%
\end{pgfscope}%
\begin{pgfscope}%
\pgfpathrectangle{\pgfqpoint{1.250000in}{0.550000in}}{\pgfqpoint{7.750000in}{3.850000in}}%
\pgfusepath{clip}%
\pgfsetbuttcap%
\pgfsetmiterjoin%
\definecolor{currentfill}{rgb}{0.501961,0.501961,0.501961}%
\pgfsetfillcolor{currentfill}%
\pgfsetfillopacity{0.700000}%
\pgfsetlinewidth{0.000000pt}%
\definecolor{currentstroke}{rgb}{0.000000,0.000000,0.000000}%
\pgfsetstrokecolor{currentstroke}%
\pgfsetstrokeopacity{0.700000}%
\pgfsetdash{}{0pt}%
\pgfpathmoveto{\pgfqpoint{7.767045in}{0.550000in}}%
\pgfpathlineto{\pgfqpoint{7.794567in}{0.550000in}}%
\pgfpathlineto{\pgfqpoint{7.794567in}{0.606970in}}%
\pgfpathlineto{\pgfqpoint{7.767045in}{0.606970in}}%
\pgfpathlineto{\pgfqpoint{7.767045in}{0.550000in}}%
\pgfpathclose%
\pgfusepath{fill}%
\end{pgfscope}%
\begin{pgfscope}%
\pgfpathrectangle{\pgfqpoint{1.250000in}{0.550000in}}{\pgfqpoint{7.750000in}{3.850000in}}%
\pgfusepath{clip}%
\pgfsetbuttcap%
\pgfsetmiterjoin%
\definecolor{currentfill}{rgb}{0.501961,0.501961,0.501961}%
\pgfsetfillcolor{currentfill}%
\pgfsetfillopacity{0.700000}%
\pgfsetlinewidth{0.000000pt}%
\definecolor{currentstroke}{rgb}{0.000000,0.000000,0.000000}%
\pgfsetstrokecolor{currentstroke}%
\pgfsetstrokeopacity{0.700000}%
\pgfsetdash{}{0pt}%
\pgfpathmoveto{\pgfqpoint{7.794567in}{0.550000in}}%
\pgfpathlineto{\pgfqpoint{7.822088in}{0.550000in}}%
\pgfpathlineto{\pgfqpoint{7.822088in}{0.599058in}}%
\pgfpathlineto{\pgfqpoint{7.794567in}{0.599058in}}%
\pgfpathlineto{\pgfqpoint{7.794567in}{0.550000in}}%
\pgfpathclose%
\pgfusepath{fill}%
\end{pgfscope}%
\begin{pgfscope}%
\pgfpathrectangle{\pgfqpoint{1.250000in}{0.550000in}}{\pgfqpoint{7.750000in}{3.850000in}}%
\pgfusepath{clip}%
\pgfsetbuttcap%
\pgfsetmiterjoin%
\definecolor{currentfill}{rgb}{0.501961,0.501961,0.501961}%
\pgfsetfillcolor{currentfill}%
\pgfsetfillopacity{0.700000}%
\pgfsetlinewidth{0.000000pt}%
\definecolor{currentstroke}{rgb}{0.000000,0.000000,0.000000}%
\pgfsetstrokecolor{currentstroke}%
\pgfsetstrokeopacity{0.700000}%
\pgfsetdash{}{0pt}%
\pgfpathmoveto{\pgfqpoint{7.822088in}{0.550000in}}%
\pgfpathlineto{\pgfqpoint{7.849609in}{0.550000in}}%
\pgfpathlineto{\pgfqpoint{7.849609in}{0.595893in}}%
\pgfpathlineto{\pgfqpoint{7.822088in}{0.595893in}}%
\pgfpathlineto{\pgfqpoint{7.822088in}{0.550000in}}%
\pgfpathclose%
\pgfusepath{fill}%
\end{pgfscope}%
\begin{pgfscope}%
\pgfpathrectangle{\pgfqpoint{1.250000in}{0.550000in}}{\pgfqpoint{7.750000in}{3.850000in}}%
\pgfusepath{clip}%
\pgfsetbuttcap%
\pgfsetmiterjoin%
\definecolor{currentfill}{rgb}{0.501961,0.501961,0.501961}%
\pgfsetfillcolor{currentfill}%
\pgfsetfillopacity{0.700000}%
\pgfsetlinewidth{0.000000pt}%
\definecolor{currentstroke}{rgb}{0.000000,0.000000,0.000000}%
\pgfsetstrokecolor{currentstroke}%
\pgfsetstrokeopacity{0.700000}%
\pgfsetdash{}{0pt}%
\pgfpathmoveto{\pgfqpoint{7.849609in}{0.550000in}}%
\pgfpathlineto{\pgfqpoint{7.877131in}{0.550000in}}%
\pgfpathlineto{\pgfqpoint{7.877131in}{0.592332in}}%
\pgfpathlineto{\pgfqpoint{7.849609in}{0.592332in}}%
\pgfpathlineto{\pgfqpoint{7.849609in}{0.550000in}}%
\pgfpathclose%
\pgfusepath{fill}%
\end{pgfscope}%
\begin{pgfscope}%
\pgfpathrectangle{\pgfqpoint{1.250000in}{0.550000in}}{\pgfqpoint{7.750000in}{3.850000in}}%
\pgfusepath{clip}%
\pgfsetbuttcap%
\pgfsetmiterjoin%
\definecolor{currentfill}{rgb}{0.501961,0.501961,0.501961}%
\pgfsetfillcolor{currentfill}%
\pgfsetfillopacity{0.700000}%
\pgfsetlinewidth{0.000000pt}%
\definecolor{currentstroke}{rgb}{0.000000,0.000000,0.000000}%
\pgfsetstrokecolor{currentstroke}%
\pgfsetstrokeopacity{0.700000}%
\pgfsetdash{}{0pt}%
\pgfpathmoveto{\pgfqpoint{7.877131in}{0.550000in}}%
\pgfpathlineto{\pgfqpoint{7.904652in}{0.550000in}}%
\pgfpathlineto{\pgfqpoint{7.904652in}{0.593519in}}%
\pgfpathlineto{\pgfqpoint{7.877131in}{0.593519in}}%
\pgfpathlineto{\pgfqpoint{7.877131in}{0.550000in}}%
\pgfpathclose%
\pgfusepath{fill}%
\end{pgfscope}%
\begin{pgfscope}%
\pgfpathrectangle{\pgfqpoint{1.250000in}{0.550000in}}{\pgfqpoint{7.750000in}{3.850000in}}%
\pgfusepath{clip}%
\pgfsetbuttcap%
\pgfsetmiterjoin%
\definecolor{currentfill}{rgb}{0.501961,0.501961,0.501961}%
\pgfsetfillcolor{currentfill}%
\pgfsetfillopacity{0.700000}%
\pgfsetlinewidth{0.000000pt}%
\definecolor{currentstroke}{rgb}{0.000000,0.000000,0.000000}%
\pgfsetstrokecolor{currentstroke}%
\pgfsetstrokeopacity{0.700000}%
\pgfsetdash{}{0pt}%
\pgfpathmoveto{\pgfqpoint{7.904652in}{0.550000in}}%
\pgfpathlineto{\pgfqpoint{7.932173in}{0.550000in}}%
\pgfpathlineto{\pgfqpoint{7.932173in}{0.588376in}}%
\pgfpathlineto{\pgfqpoint{7.904652in}{0.588376in}}%
\pgfpathlineto{\pgfqpoint{7.904652in}{0.550000in}}%
\pgfpathclose%
\pgfusepath{fill}%
\end{pgfscope}%
\begin{pgfscope}%
\pgfpathrectangle{\pgfqpoint{1.250000in}{0.550000in}}{\pgfqpoint{7.750000in}{3.850000in}}%
\pgfusepath{clip}%
\pgfsetbuttcap%
\pgfsetmiterjoin%
\definecolor{currentfill}{rgb}{0.501961,0.501961,0.501961}%
\pgfsetfillcolor{currentfill}%
\pgfsetfillopacity{0.700000}%
\pgfsetlinewidth{0.000000pt}%
\definecolor{currentstroke}{rgb}{0.000000,0.000000,0.000000}%
\pgfsetstrokecolor{currentstroke}%
\pgfsetstrokeopacity{0.700000}%
\pgfsetdash{}{0pt}%
\pgfpathmoveto{\pgfqpoint{7.932173in}{0.550000in}}%
\pgfpathlineto{\pgfqpoint{7.959695in}{0.550000in}}%
\pgfpathlineto{\pgfqpoint{7.959695in}{0.593519in}}%
\pgfpathlineto{\pgfqpoint{7.932173in}{0.593519in}}%
\pgfpathlineto{\pgfqpoint{7.932173in}{0.550000in}}%
\pgfpathclose%
\pgfusepath{fill}%
\end{pgfscope}%
\begin{pgfscope}%
\pgfpathrectangle{\pgfqpoint{1.250000in}{0.550000in}}{\pgfqpoint{7.750000in}{3.850000in}}%
\pgfusepath{clip}%
\pgfsetbuttcap%
\pgfsetmiterjoin%
\definecolor{currentfill}{rgb}{0.501961,0.501961,0.501961}%
\pgfsetfillcolor{currentfill}%
\pgfsetfillopacity{0.700000}%
\pgfsetlinewidth{0.000000pt}%
\definecolor{currentstroke}{rgb}{0.000000,0.000000,0.000000}%
\pgfsetstrokecolor{currentstroke}%
\pgfsetstrokeopacity{0.700000}%
\pgfsetdash{}{0pt}%
\pgfpathmoveto{\pgfqpoint{7.959695in}{0.550000in}}%
\pgfpathlineto{\pgfqpoint{7.987216in}{0.550000in}}%
\pgfpathlineto{\pgfqpoint{7.987216in}{0.597080in}}%
\pgfpathlineto{\pgfqpoint{7.959695in}{0.597080in}}%
\pgfpathlineto{\pgfqpoint{7.959695in}{0.550000in}}%
\pgfpathclose%
\pgfusepath{fill}%
\end{pgfscope}%
\begin{pgfscope}%
\pgfpathrectangle{\pgfqpoint{1.250000in}{0.550000in}}{\pgfqpoint{7.750000in}{3.850000in}}%
\pgfusepath{clip}%
\pgfsetbuttcap%
\pgfsetmiterjoin%
\definecolor{currentfill}{rgb}{0.501961,0.501961,0.501961}%
\pgfsetfillcolor{currentfill}%
\pgfsetfillopacity{0.700000}%
\pgfsetlinewidth{0.000000pt}%
\definecolor{currentstroke}{rgb}{0.000000,0.000000,0.000000}%
\pgfsetstrokecolor{currentstroke}%
\pgfsetstrokeopacity{0.700000}%
\pgfsetdash{}{0pt}%
\pgfpathmoveto{\pgfqpoint{7.987216in}{0.550000in}}%
\pgfpathlineto{\pgfqpoint{8.014737in}{0.550000in}}%
\pgfpathlineto{\pgfqpoint{8.014737in}{0.550000in}}%
\pgfpathlineto{\pgfqpoint{7.987216in}{0.550000in}}%
\pgfpathlineto{\pgfqpoint{7.987216in}{0.550000in}}%
\pgfpathclose%
\pgfusepath{fill}%
\end{pgfscope}%
\begin{pgfscope}%
\pgfpathrectangle{\pgfqpoint{1.250000in}{0.550000in}}{\pgfqpoint{7.750000in}{3.850000in}}%
\pgfusepath{clip}%
\pgfsetbuttcap%
\pgfsetmiterjoin%
\definecolor{currentfill}{rgb}{0.501961,0.501961,0.501961}%
\pgfsetfillcolor{currentfill}%
\pgfsetfillopacity{0.700000}%
\pgfsetlinewidth{0.000000pt}%
\definecolor{currentstroke}{rgb}{0.000000,0.000000,0.000000}%
\pgfsetstrokecolor{currentstroke}%
\pgfsetstrokeopacity{0.700000}%
\pgfsetdash{}{0pt}%
\pgfpathmoveto{\pgfqpoint{8.014737in}{0.550000in}}%
\pgfpathlineto{\pgfqpoint{8.042259in}{0.550000in}}%
\pgfpathlineto{\pgfqpoint{8.042259in}{0.597871in}}%
\pgfpathlineto{\pgfqpoint{8.014737in}{0.597871in}}%
\pgfpathlineto{\pgfqpoint{8.014737in}{0.550000in}}%
\pgfpathclose%
\pgfusepath{fill}%
\end{pgfscope}%
\begin{pgfscope}%
\pgfpathrectangle{\pgfqpoint{1.250000in}{0.550000in}}{\pgfqpoint{7.750000in}{3.850000in}}%
\pgfusepath{clip}%
\pgfsetbuttcap%
\pgfsetmiterjoin%
\definecolor{currentfill}{rgb}{0.501961,0.501961,0.501961}%
\pgfsetfillcolor{currentfill}%
\pgfsetfillopacity{0.700000}%
\pgfsetlinewidth{0.000000pt}%
\definecolor{currentstroke}{rgb}{0.000000,0.000000,0.000000}%
\pgfsetstrokecolor{currentstroke}%
\pgfsetstrokeopacity{0.700000}%
\pgfsetdash{}{0pt}%
\pgfpathmoveto{\pgfqpoint{8.042259in}{0.550000in}}%
\pgfpathlineto{\pgfqpoint{8.069780in}{0.550000in}}%
\pgfpathlineto{\pgfqpoint{8.069780in}{0.594706in}}%
\pgfpathlineto{\pgfqpoint{8.042259in}{0.594706in}}%
\pgfpathlineto{\pgfqpoint{8.042259in}{0.550000in}}%
\pgfpathclose%
\pgfusepath{fill}%
\end{pgfscope}%
\begin{pgfscope}%
\pgfpathrectangle{\pgfqpoint{1.250000in}{0.550000in}}{\pgfqpoint{7.750000in}{3.850000in}}%
\pgfusepath{clip}%
\pgfsetbuttcap%
\pgfsetmiterjoin%
\definecolor{currentfill}{rgb}{0.501961,0.501961,0.501961}%
\pgfsetfillcolor{currentfill}%
\pgfsetfillopacity{0.700000}%
\pgfsetlinewidth{0.000000pt}%
\definecolor{currentstroke}{rgb}{0.000000,0.000000,0.000000}%
\pgfsetstrokecolor{currentstroke}%
\pgfsetstrokeopacity{0.700000}%
\pgfsetdash{}{0pt}%
\pgfpathmoveto{\pgfqpoint{8.069780in}{0.550000in}}%
\pgfpathlineto{\pgfqpoint{8.097301in}{0.550000in}}%
\pgfpathlineto{\pgfqpoint{8.097301in}{0.589958in}}%
\pgfpathlineto{\pgfqpoint{8.069780in}{0.589958in}}%
\pgfpathlineto{\pgfqpoint{8.069780in}{0.550000in}}%
\pgfpathclose%
\pgfusepath{fill}%
\end{pgfscope}%
\begin{pgfscope}%
\pgfpathrectangle{\pgfqpoint{1.250000in}{0.550000in}}{\pgfqpoint{7.750000in}{3.850000in}}%
\pgfusepath{clip}%
\pgfsetbuttcap%
\pgfsetmiterjoin%
\definecolor{currentfill}{rgb}{0.501961,0.501961,0.501961}%
\pgfsetfillcolor{currentfill}%
\pgfsetfillopacity{0.700000}%
\pgfsetlinewidth{0.000000pt}%
\definecolor{currentstroke}{rgb}{0.000000,0.000000,0.000000}%
\pgfsetstrokecolor{currentstroke}%
\pgfsetstrokeopacity{0.700000}%
\pgfsetdash{}{0pt}%
\pgfpathmoveto{\pgfqpoint{8.097301in}{0.550000in}}%
\pgfpathlineto{\pgfqpoint{8.124822in}{0.550000in}}%
\pgfpathlineto{\pgfqpoint{8.124822in}{0.597475in}}%
\pgfpathlineto{\pgfqpoint{8.097301in}{0.597475in}}%
\pgfpathlineto{\pgfqpoint{8.097301in}{0.550000in}}%
\pgfpathclose%
\pgfusepath{fill}%
\end{pgfscope}%
\begin{pgfscope}%
\pgfpathrectangle{\pgfqpoint{1.250000in}{0.550000in}}{\pgfqpoint{7.750000in}{3.850000in}}%
\pgfusepath{clip}%
\pgfsetbuttcap%
\pgfsetmiterjoin%
\definecolor{currentfill}{rgb}{0.501961,0.501961,0.501961}%
\pgfsetfillcolor{currentfill}%
\pgfsetfillopacity{0.700000}%
\pgfsetlinewidth{0.000000pt}%
\definecolor{currentstroke}{rgb}{0.000000,0.000000,0.000000}%
\pgfsetstrokecolor{currentstroke}%
\pgfsetstrokeopacity{0.700000}%
\pgfsetdash{}{0pt}%
\pgfpathmoveto{\pgfqpoint{8.124822in}{0.550000in}}%
\pgfpathlineto{\pgfqpoint{8.152344in}{0.550000in}}%
\pgfpathlineto{\pgfqpoint{8.152344in}{0.589563in}}%
\pgfpathlineto{\pgfqpoint{8.124822in}{0.589563in}}%
\pgfpathlineto{\pgfqpoint{8.124822in}{0.550000in}}%
\pgfpathclose%
\pgfusepath{fill}%
\end{pgfscope}%
\begin{pgfscope}%
\pgfpathrectangle{\pgfqpoint{1.250000in}{0.550000in}}{\pgfqpoint{7.750000in}{3.850000in}}%
\pgfusepath{clip}%
\pgfsetbuttcap%
\pgfsetmiterjoin%
\definecolor{currentfill}{rgb}{0.501961,0.501961,0.501961}%
\pgfsetfillcolor{currentfill}%
\pgfsetfillopacity{0.700000}%
\pgfsetlinewidth{0.000000pt}%
\definecolor{currentstroke}{rgb}{0.000000,0.000000,0.000000}%
\pgfsetstrokecolor{currentstroke}%
\pgfsetstrokeopacity{0.700000}%
\pgfsetdash{}{0pt}%
\pgfpathmoveto{\pgfqpoint{8.152344in}{0.550000in}}%
\pgfpathlineto{\pgfqpoint{8.179865in}{0.550000in}}%
\pgfpathlineto{\pgfqpoint{8.179865in}{0.590750in}}%
\pgfpathlineto{\pgfqpoint{8.152344in}{0.590750in}}%
\pgfpathlineto{\pgfqpoint{8.152344in}{0.550000in}}%
\pgfpathclose%
\pgfusepath{fill}%
\end{pgfscope}%
\begin{pgfscope}%
\pgfpathrectangle{\pgfqpoint{1.250000in}{0.550000in}}{\pgfqpoint{7.750000in}{3.850000in}}%
\pgfusepath{clip}%
\pgfsetbuttcap%
\pgfsetmiterjoin%
\definecolor{currentfill}{rgb}{0.501961,0.501961,0.501961}%
\pgfsetfillcolor{currentfill}%
\pgfsetfillopacity{0.700000}%
\pgfsetlinewidth{0.000000pt}%
\definecolor{currentstroke}{rgb}{0.000000,0.000000,0.000000}%
\pgfsetstrokecolor{currentstroke}%
\pgfsetstrokeopacity{0.700000}%
\pgfsetdash{}{0pt}%
\pgfpathmoveto{\pgfqpoint{8.179865in}{0.550000in}}%
\pgfpathlineto{\pgfqpoint{8.207386in}{0.550000in}}%
\pgfpathlineto{\pgfqpoint{8.207386in}{0.582046in}}%
\pgfpathlineto{\pgfqpoint{8.179865in}{0.582046in}}%
\pgfpathlineto{\pgfqpoint{8.179865in}{0.550000in}}%
\pgfpathclose%
\pgfusepath{fill}%
\end{pgfscope}%
\begin{pgfscope}%
\pgfpathrectangle{\pgfqpoint{1.250000in}{0.550000in}}{\pgfqpoint{7.750000in}{3.850000in}}%
\pgfusepath{clip}%
\pgfsetbuttcap%
\pgfsetmiterjoin%
\definecolor{currentfill}{rgb}{0.501961,0.501961,0.501961}%
\pgfsetfillcolor{currentfill}%
\pgfsetfillopacity{0.700000}%
\pgfsetlinewidth{0.000000pt}%
\definecolor{currentstroke}{rgb}{0.000000,0.000000,0.000000}%
\pgfsetstrokecolor{currentstroke}%
\pgfsetstrokeopacity{0.700000}%
\pgfsetdash{}{0pt}%
\pgfpathmoveto{\pgfqpoint{8.207386in}{0.550000in}}%
\pgfpathlineto{\pgfqpoint{8.234908in}{0.550000in}}%
\pgfpathlineto{\pgfqpoint{8.234908in}{0.585606in}}%
\pgfpathlineto{\pgfqpoint{8.207386in}{0.585606in}}%
\pgfpathlineto{\pgfqpoint{8.207386in}{0.550000in}}%
\pgfpathclose%
\pgfusepath{fill}%
\end{pgfscope}%
\begin{pgfscope}%
\pgfpathrectangle{\pgfqpoint{1.250000in}{0.550000in}}{\pgfqpoint{7.750000in}{3.850000in}}%
\pgfusepath{clip}%
\pgfsetbuttcap%
\pgfsetmiterjoin%
\definecolor{currentfill}{rgb}{0.501961,0.501961,0.501961}%
\pgfsetfillcolor{currentfill}%
\pgfsetfillopacity{0.700000}%
\pgfsetlinewidth{0.000000pt}%
\definecolor{currentstroke}{rgb}{0.000000,0.000000,0.000000}%
\pgfsetstrokecolor{currentstroke}%
\pgfsetstrokeopacity{0.700000}%
\pgfsetdash{}{0pt}%
\pgfpathmoveto{\pgfqpoint{8.234908in}{0.550000in}}%
\pgfpathlineto{\pgfqpoint{8.262429in}{0.550000in}}%
\pgfpathlineto{\pgfqpoint{8.262429in}{0.605388in}}%
\pgfpathlineto{\pgfqpoint{8.234908in}{0.605388in}}%
\pgfpathlineto{\pgfqpoint{8.234908in}{0.550000in}}%
\pgfpathclose%
\pgfusepath{fill}%
\end{pgfscope}%
\begin{pgfscope}%
\pgfpathrectangle{\pgfqpoint{1.250000in}{0.550000in}}{\pgfqpoint{7.750000in}{3.850000in}}%
\pgfusepath{clip}%
\pgfsetbuttcap%
\pgfsetmiterjoin%
\definecolor{currentfill}{rgb}{0.501961,0.501961,0.501961}%
\pgfsetfillcolor{currentfill}%
\pgfsetfillopacity{0.700000}%
\pgfsetlinewidth{0.000000pt}%
\definecolor{currentstroke}{rgb}{0.000000,0.000000,0.000000}%
\pgfsetstrokecolor{currentstroke}%
\pgfsetstrokeopacity{0.700000}%
\pgfsetdash{}{0pt}%
\pgfpathmoveto{\pgfqpoint{8.262429in}{0.550000in}}%
\pgfpathlineto{\pgfqpoint{8.289950in}{0.550000in}}%
\pgfpathlineto{\pgfqpoint{8.289950in}{0.581650in}}%
\pgfpathlineto{\pgfqpoint{8.262429in}{0.581650in}}%
\pgfpathlineto{\pgfqpoint{8.262429in}{0.550000in}}%
\pgfpathclose%
\pgfusepath{fill}%
\end{pgfscope}%
\begin{pgfscope}%
\pgfpathrectangle{\pgfqpoint{1.250000in}{0.550000in}}{\pgfqpoint{7.750000in}{3.850000in}}%
\pgfusepath{clip}%
\pgfsetbuttcap%
\pgfsetmiterjoin%
\definecolor{currentfill}{rgb}{0.501961,0.501961,0.501961}%
\pgfsetfillcolor{currentfill}%
\pgfsetfillopacity{0.700000}%
\pgfsetlinewidth{0.000000pt}%
\definecolor{currentstroke}{rgb}{0.000000,0.000000,0.000000}%
\pgfsetstrokecolor{currentstroke}%
\pgfsetstrokeopacity{0.700000}%
\pgfsetdash{}{0pt}%
\pgfpathmoveto{\pgfqpoint{8.289950in}{0.550000in}}%
\pgfpathlineto{\pgfqpoint{8.317472in}{0.550000in}}%
\pgfpathlineto{\pgfqpoint{8.317472in}{0.579276in}}%
\pgfpathlineto{\pgfqpoint{8.289950in}{0.579276in}}%
\pgfpathlineto{\pgfqpoint{8.289950in}{0.550000in}}%
\pgfpathclose%
\pgfusepath{fill}%
\end{pgfscope}%
\begin{pgfscope}%
\pgfpathrectangle{\pgfqpoint{1.250000in}{0.550000in}}{\pgfqpoint{7.750000in}{3.850000in}}%
\pgfusepath{clip}%
\pgfsetbuttcap%
\pgfsetmiterjoin%
\definecolor{currentfill}{rgb}{0.501961,0.501961,0.501961}%
\pgfsetfillcolor{currentfill}%
\pgfsetfillopacity{0.700000}%
\pgfsetlinewidth{0.000000pt}%
\definecolor{currentstroke}{rgb}{0.000000,0.000000,0.000000}%
\pgfsetstrokecolor{currentstroke}%
\pgfsetstrokeopacity{0.700000}%
\pgfsetdash{}{0pt}%
\pgfpathmoveto{\pgfqpoint{8.317472in}{0.550000in}}%
\pgfpathlineto{\pgfqpoint{8.344993in}{0.550000in}}%
\pgfpathlineto{\pgfqpoint{8.344993in}{0.597871in}}%
\pgfpathlineto{\pgfqpoint{8.317472in}{0.597871in}}%
\pgfpathlineto{\pgfqpoint{8.317472in}{0.550000in}}%
\pgfpathclose%
\pgfusepath{fill}%
\end{pgfscope}%
\begin{pgfscope}%
\pgfpathrectangle{\pgfqpoint{1.250000in}{0.550000in}}{\pgfqpoint{7.750000in}{3.850000in}}%
\pgfusepath{clip}%
\pgfsetbuttcap%
\pgfsetmiterjoin%
\definecolor{currentfill}{rgb}{0.501961,0.501961,0.501961}%
\pgfsetfillcolor{currentfill}%
\pgfsetfillopacity{0.700000}%
\pgfsetlinewidth{0.000000pt}%
\definecolor{currentstroke}{rgb}{0.000000,0.000000,0.000000}%
\pgfsetstrokecolor{currentstroke}%
\pgfsetstrokeopacity{0.700000}%
\pgfsetdash{}{0pt}%
\pgfpathmoveto{\pgfqpoint{8.344993in}{0.550000in}}%
\pgfpathlineto{\pgfqpoint{8.372514in}{0.550000in}}%
\pgfpathlineto{\pgfqpoint{8.372514in}{0.590354in}}%
\pgfpathlineto{\pgfqpoint{8.344993in}{0.590354in}}%
\pgfpathlineto{\pgfqpoint{8.344993in}{0.550000in}}%
\pgfpathclose%
\pgfusepath{fill}%
\end{pgfscope}%
\begin{pgfscope}%
\pgfpathrectangle{\pgfqpoint{1.250000in}{0.550000in}}{\pgfqpoint{7.750000in}{3.850000in}}%
\pgfusepath{clip}%
\pgfsetbuttcap%
\pgfsetmiterjoin%
\definecolor{currentfill}{rgb}{0.501961,0.501961,0.501961}%
\pgfsetfillcolor{currentfill}%
\pgfsetfillopacity{0.700000}%
\pgfsetlinewidth{0.000000pt}%
\definecolor{currentstroke}{rgb}{0.000000,0.000000,0.000000}%
\pgfsetstrokecolor{currentstroke}%
\pgfsetstrokeopacity{0.700000}%
\pgfsetdash{}{0pt}%
\pgfpathmoveto{\pgfqpoint{8.372514in}{0.550000in}}%
\pgfpathlineto{\pgfqpoint{8.400036in}{0.550000in}}%
\pgfpathlineto{\pgfqpoint{8.400036in}{0.586398in}}%
\pgfpathlineto{\pgfqpoint{8.372514in}{0.586398in}}%
\pgfpathlineto{\pgfqpoint{8.372514in}{0.550000in}}%
\pgfpathclose%
\pgfusepath{fill}%
\end{pgfscope}%
\begin{pgfscope}%
\pgfpathrectangle{\pgfqpoint{1.250000in}{0.550000in}}{\pgfqpoint{7.750000in}{3.850000in}}%
\pgfusepath{clip}%
\pgfsetbuttcap%
\pgfsetmiterjoin%
\definecolor{currentfill}{rgb}{0.501961,0.501961,0.501961}%
\pgfsetfillcolor{currentfill}%
\pgfsetfillopacity{0.700000}%
\pgfsetlinewidth{0.000000pt}%
\definecolor{currentstroke}{rgb}{0.000000,0.000000,0.000000}%
\pgfsetstrokecolor{currentstroke}%
\pgfsetstrokeopacity{0.700000}%
\pgfsetdash{}{0pt}%
\pgfpathmoveto{\pgfqpoint{8.400036in}{0.550000in}}%
\pgfpathlineto{\pgfqpoint{8.427557in}{0.550000in}}%
\pgfpathlineto{\pgfqpoint{8.427557in}{0.595497in}}%
\pgfpathlineto{\pgfqpoint{8.400036in}{0.595497in}}%
\pgfpathlineto{\pgfqpoint{8.400036in}{0.550000in}}%
\pgfpathclose%
\pgfusepath{fill}%
\end{pgfscope}%
\begin{pgfscope}%
\pgfpathrectangle{\pgfqpoint{1.250000in}{0.550000in}}{\pgfqpoint{7.750000in}{3.850000in}}%
\pgfusepath{clip}%
\pgfsetbuttcap%
\pgfsetmiterjoin%
\definecolor{currentfill}{rgb}{0.501961,0.501961,0.501961}%
\pgfsetfillcolor{currentfill}%
\pgfsetfillopacity{0.700000}%
\pgfsetlinewidth{0.000000pt}%
\definecolor{currentstroke}{rgb}{0.000000,0.000000,0.000000}%
\pgfsetstrokecolor{currentstroke}%
\pgfsetstrokeopacity{0.700000}%
\pgfsetdash{}{0pt}%
\pgfpathmoveto{\pgfqpoint{8.427557in}{0.550000in}}%
\pgfpathlineto{\pgfqpoint{8.455078in}{0.550000in}}%
\pgfpathlineto{\pgfqpoint{8.455078in}{0.586398in}}%
\pgfpathlineto{\pgfqpoint{8.427557in}{0.586398in}}%
\pgfpathlineto{\pgfqpoint{8.427557in}{0.550000in}}%
\pgfpathclose%
\pgfusepath{fill}%
\end{pgfscope}%
\begin{pgfscope}%
\pgfpathrectangle{\pgfqpoint{1.250000in}{0.550000in}}{\pgfqpoint{7.750000in}{3.850000in}}%
\pgfusepath{clip}%
\pgfsetbuttcap%
\pgfsetmiterjoin%
\definecolor{currentfill}{rgb}{0.501961,0.501961,0.501961}%
\pgfsetfillcolor{currentfill}%
\pgfsetfillopacity{0.700000}%
\pgfsetlinewidth{0.000000pt}%
\definecolor{currentstroke}{rgb}{0.000000,0.000000,0.000000}%
\pgfsetstrokecolor{currentstroke}%
\pgfsetstrokeopacity{0.700000}%
\pgfsetdash{}{0pt}%
\pgfpathmoveto{\pgfqpoint{8.455078in}{0.550000in}}%
\pgfpathlineto{\pgfqpoint{8.482599in}{0.550000in}}%
\pgfpathlineto{\pgfqpoint{8.482599in}{0.593519in}}%
\pgfpathlineto{\pgfqpoint{8.455078in}{0.593519in}}%
\pgfpathlineto{\pgfqpoint{8.455078in}{0.550000in}}%
\pgfpathclose%
\pgfusepath{fill}%
\end{pgfscope}%
\begin{pgfscope}%
\pgfpathrectangle{\pgfqpoint{1.250000in}{0.550000in}}{\pgfqpoint{7.750000in}{3.850000in}}%
\pgfusepath{clip}%
\pgfsetbuttcap%
\pgfsetmiterjoin%
\definecolor{currentfill}{rgb}{0.501961,0.501961,0.501961}%
\pgfsetfillcolor{currentfill}%
\pgfsetfillopacity{0.700000}%
\pgfsetlinewidth{0.000000pt}%
\definecolor{currentstroke}{rgb}{0.000000,0.000000,0.000000}%
\pgfsetstrokecolor{currentstroke}%
\pgfsetstrokeopacity{0.700000}%
\pgfsetdash{}{0pt}%
\pgfpathmoveto{\pgfqpoint{8.482599in}{0.550000in}}%
\pgfpathlineto{\pgfqpoint{8.510121in}{0.550000in}}%
\pgfpathlineto{\pgfqpoint{8.510121in}{0.595497in}}%
\pgfpathlineto{\pgfqpoint{8.482599in}{0.595497in}}%
\pgfpathlineto{\pgfqpoint{8.482599in}{0.550000in}}%
\pgfpathclose%
\pgfusepath{fill}%
\end{pgfscope}%
\begin{pgfscope}%
\pgfpathrectangle{\pgfqpoint{1.250000in}{0.550000in}}{\pgfqpoint{7.750000in}{3.850000in}}%
\pgfusepath{clip}%
\pgfsetbuttcap%
\pgfsetmiterjoin%
\definecolor{currentfill}{rgb}{0.501961,0.501961,0.501961}%
\pgfsetfillcolor{currentfill}%
\pgfsetfillopacity{0.700000}%
\pgfsetlinewidth{0.000000pt}%
\definecolor{currentstroke}{rgb}{0.000000,0.000000,0.000000}%
\pgfsetstrokecolor{currentstroke}%
\pgfsetstrokeopacity{0.700000}%
\pgfsetdash{}{0pt}%
\pgfpathmoveto{\pgfqpoint{8.510121in}{0.550000in}}%
\pgfpathlineto{\pgfqpoint{8.537642in}{0.550000in}}%
\pgfpathlineto{\pgfqpoint{8.537642in}{0.584420in}}%
\pgfpathlineto{\pgfqpoint{8.510121in}{0.584420in}}%
\pgfpathlineto{\pgfqpoint{8.510121in}{0.550000in}}%
\pgfpathclose%
\pgfusepath{fill}%
\end{pgfscope}%
\begin{pgfscope}%
\pgfpathrectangle{\pgfqpoint{1.250000in}{0.550000in}}{\pgfqpoint{7.750000in}{3.850000in}}%
\pgfusepath{clip}%
\pgfsetbuttcap%
\pgfsetmiterjoin%
\definecolor{currentfill}{rgb}{0.501961,0.501961,0.501961}%
\pgfsetfillcolor{currentfill}%
\pgfsetfillopacity{0.700000}%
\pgfsetlinewidth{0.000000pt}%
\definecolor{currentstroke}{rgb}{0.000000,0.000000,0.000000}%
\pgfsetstrokecolor{currentstroke}%
\pgfsetstrokeopacity{0.700000}%
\pgfsetdash{}{0pt}%
\pgfpathmoveto{\pgfqpoint{8.537642in}{0.550000in}}%
\pgfpathlineto{\pgfqpoint{8.565163in}{0.550000in}}%
\pgfpathlineto{\pgfqpoint{8.565163in}{0.586398in}}%
\pgfpathlineto{\pgfqpoint{8.537642in}{0.586398in}}%
\pgfpathlineto{\pgfqpoint{8.537642in}{0.550000in}}%
\pgfpathclose%
\pgfusepath{fill}%
\end{pgfscope}%
\begin{pgfscope}%
\pgfpathrectangle{\pgfqpoint{1.250000in}{0.550000in}}{\pgfqpoint{7.750000in}{3.850000in}}%
\pgfusepath{clip}%
\pgfsetbuttcap%
\pgfsetmiterjoin%
\definecolor{currentfill}{rgb}{0.501961,0.501961,0.501961}%
\pgfsetfillcolor{currentfill}%
\pgfsetfillopacity{0.700000}%
\pgfsetlinewidth{0.000000pt}%
\definecolor{currentstroke}{rgb}{0.000000,0.000000,0.000000}%
\pgfsetstrokecolor{currentstroke}%
\pgfsetstrokeopacity{0.700000}%
\pgfsetdash{}{0pt}%
\pgfpathmoveto{\pgfqpoint{8.565163in}{0.550000in}}%
\pgfpathlineto{\pgfqpoint{8.592685in}{0.550000in}}%
\pgfpathlineto{\pgfqpoint{8.592685in}{0.589958in}}%
\pgfpathlineto{\pgfqpoint{8.565163in}{0.589958in}}%
\pgfpathlineto{\pgfqpoint{8.565163in}{0.550000in}}%
\pgfpathclose%
\pgfusepath{fill}%
\end{pgfscope}%
\begin{pgfscope}%
\pgfpathrectangle{\pgfqpoint{1.250000in}{0.550000in}}{\pgfqpoint{7.750000in}{3.850000in}}%
\pgfusepath{clip}%
\pgfsetbuttcap%
\pgfsetmiterjoin%
\definecolor{currentfill}{rgb}{0.501961,0.501961,0.501961}%
\pgfsetfillcolor{currentfill}%
\pgfsetfillopacity{0.700000}%
\pgfsetlinewidth{0.000000pt}%
\definecolor{currentstroke}{rgb}{0.000000,0.000000,0.000000}%
\pgfsetstrokecolor{currentstroke}%
\pgfsetstrokeopacity{0.700000}%
\pgfsetdash{}{0pt}%
\pgfpathmoveto{\pgfqpoint{8.592685in}{0.550000in}}%
\pgfpathlineto{\pgfqpoint{8.620206in}{0.550000in}}%
\pgfpathlineto{\pgfqpoint{8.620206in}{0.592332in}}%
\pgfpathlineto{\pgfqpoint{8.592685in}{0.592332in}}%
\pgfpathlineto{\pgfqpoint{8.592685in}{0.550000in}}%
\pgfpathclose%
\pgfusepath{fill}%
\end{pgfscope}%
\begin{pgfscope}%
\pgfpathrectangle{\pgfqpoint{1.250000in}{0.550000in}}{\pgfqpoint{7.750000in}{3.850000in}}%
\pgfusepath{clip}%
\pgfsetbuttcap%
\pgfsetmiterjoin%
\definecolor{currentfill}{rgb}{0.501961,0.501961,0.501961}%
\pgfsetfillcolor{currentfill}%
\pgfsetfillopacity{0.700000}%
\pgfsetlinewidth{0.000000pt}%
\definecolor{currentstroke}{rgb}{0.000000,0.000000,0.000000}%
\pgfsetstrokecolor{currentstroke}%
\pgfsetstrokeopacity{0.700000}%
\pgfsetdash{}{0pt}%
\pgfpathmoveto{\pgfqpoint{8.620206in}{0.550000in}}%
\pgfpathlineto{\pgfqpoint{8.647727in}{0.550000in}}%
\pgfpathlineto{\pgfqpoint{8.647727in}{0.573342in}}%
\pgfpathlineto{\pgfqpoint{8.620206in}{0.573342in}}%
\pgfpathlineto{\pgfqpoint{8.620206in}{0.550000in}}%
\pgfpathclose%
\pgfusepath{fill}%
\end{pgfscope}%
\begin{pgfscope}%
\pgfsetbuttcap%
\pgfsetroundjoin%
\definecolor{currentfill}{rgb}{0.000000,0.000000,0.000000}%
\pgfsetfillcolor{currentfill}%
\pgfsetlinewidth{0.803000pt}%
\definecolor{currentstroke}{rgb}{0.000000,0.000000,0.000000}%
\pgfsetstrokecolor{currentstroke}%
\pgfsetdash{}{0pt}%
\pgfsys@defobject{currentmarker}{\pgfqpoint{0.000000in}{-0.048611in}}{\pgfqpoint{0.000000in}{0.000000in}}{%
\pgfpathmoveto{\pgfqpoint{0.000000in}{0.000000in}}%
\pgfpathlineto{\pgfqpoint{0.000000in}{-0.048611in}}%
\pgfusepath{stroke,fill}%
}%
\begin{pgfscope}%
\pgfsys@transformshift{1.602273in}{0.550000in}%
\pgfsys@useobject{currentmarker}{}%
\end{pgfscope}%
\end{pgfscope}%
\begin{pgfscope}%
\definecolor{textcolor}{rgb}{0.000000,0.000000,0.000000}%
\pgfsetstrokecolor{textcolor}%
\pgfsetfillcolor{textcolor}%
\pgftext[x=1.602273in,y=0.452778in,,top]{\color{textcolor}{\sffamily\fontsize{10.000000}{12.000000}\selectfont\catcode`\^=\active\def^{\ifmmode\sp\else\^{}\fi}\catcode`\%=\active\def%{\%}0}}%
\end{pgfscope}%
\begin{pgfscope}%
\pgfsetbuttcap%
\pgfsetroundjoin%
\definecolor{currentfill}{rgb}{0.000000,0.000000,0.000000}%
\pgfsetfillcolor{currentfill}%
\pgfsetlinewidth{0.803000pt}%
\definecolor{currentstroke}{rgb}{0.000000,0.000000,0.000000}%
\pgfsetstrokecolor{currentstroke}%
\pgfsetdash{}{0pt}%
\pgfsys@defobject{currentmarker}{\pgfqpoint{0.000000in}{-0.048611in}}{\pgfqpoint{0.000000in}{0.000000in}}{%
\pgfpathmoveto{\pgfqpoint{0.000000in}{0.000000in}}%
\pgfpathlineto{\pgfqpoint{0.000000in}{-0.048611in}}%
\pgfusepath{stroke,fill}%
}%
\begin{pgfscope}%
\pgfsys@transformshift{2.978338in}{0.550000in}%
\pgfsys@useobject{currentmarker}{}%
\end{pgfscope}%
\end{pgfscope}%
\begin{pgfscope}%
\definecolor{textcolor}{rgb}{0.000000,0.000000,0.000000}%
\pgfsetstrokecolor{textcolor}%
\pgfsetfillcolor{textcolor}%
\pgftext[x=2.978338in,y=0.452778in,,top]{\color{textcolor}{\sffamily\fontsize{10.000000}{12.000000}\selectfont\catcode`\^=\active\def^{\ifmmode\sp\else\^{}\fi}\catcode`\%=\active\def%{\%}50}}%
\end{pgfscope}%
\begin{pgfscope}%
\pgfsetbuttcap%
\pgfsetroundjoin%
\definecolor{currentfill}{rgb}{0.000000,0.000000,0.000000}%
\pgfsetfillcolor{currentfill}%
\pgfsetlinewidth{0.803000pt}%
\definecolor{currentstroke}{rgb}{0.000000,0.000000,0.000000}%
\pgfsetstrokecolor{currentstroke}%
\pgfsetdash{}{0pt}%
\pgfsys@defobject{currentmarker}{\pgfqpoint{0.000000in}{-0.048611in}}{\pgfqpoint{0.000000in}{0.000000in}}{%
\pgfpathmoveto{\pgfqpoint{0.000000in}{0.000000in}}%
\pgfpathlineto{\pgfqpoint{0.000000in}{-0.048611in}}%
\pgfusepath{stroke,fill}%
}%
\begin{pgfscope}%
\pgfsys@transformshift{4.354403in}{0.550000in}%
\pgfsys@useobject{currentmarker}{}%
\end{pgfscope}%
\end{pgfscope}%
\begin{pgfscope}%
\definecolor{textcolor}{rgb}{0.000000,0.000000,0.000000}%
\pgfsetstrokecolor{textcolor}%
\pgfsetfillcolor{textcolor}%
\pgftext[x=4.354403in,y=0.452778in,,top]{\color{textcolor}{\sffamily\fontsize{10.000000}{12.000000}\selectfont\catcode`\^=\active\def^{\ifmmode\sp\else\^{}\fi}\catcode`\%=\active\def%{\%}100}}%
\end{pgfscope}%
\begin{pgfscope}%
\pgfsetbuttcap%
\pgfsetroundjoin%
\definecolor{currentfill}{rgb}{0.000000,0.000000,0.000000}%
\pgfsetfillcolor{currentfill}%
\pgfsetlinewidth{0.803000pt}%
\definecolor{currentstroke}{rgb}{0.000000,0.000000,0.000000}%
\pgfsetstrokecolor{currentstroke}%
\pgfsetdash{}{0pt}%
\pgfsys@defobject{currentmarker}{\pgfqpoint{0.000000in}{-0.048611in}}{\pgfqpoint{0.000000in}{0.000000in}}{%
\pgfpathmoveto{\pgfqpoint{0.000000in}{0.000000in}}%
\pgfpathlineto{\pgfqpoint{0.000000in}{-0.048611in}}%
\pgfusepath{stroke,fill}%
}%
\begin{pgfscope}%
\pgfsys@transformshift{5.730469in}{0.550000in}%
\pgfsys@useobject{currentmarker}{}%
\end{pgfscope}%
\end{pgfscope}%
\begin{pgfscope}%
\definecolor{textcolor}{rgb}{0.000000,0.000000,0.000000}%
\pgfsetstrokecolor{textcolor}%
\pgfsetfillcolor{textcolor}%
\pgftext[x=5.730469in,y=0.452778in,,top]{\color{textcolor}{\sffamily\fontsize{10.000000}{12.000000}\selectfont\catcode`\^=\active\def^{\ifmmode\sp\else\^{}\fi}\catcode`\%=\active\def%{\%}150}}%
\end{pgfscope}%
\begin{pgfscope}%
\pgfsetbuttcap%
\pgfsetroundjoin%
\definecolor{currentfill}{rgb}{0.000000,0.000000,0.000000}%
\pgfsetfillcolor{currentfill}%
\pgfsetlinewidth{0.803000pt}%
\definecolor{currentstroke}{rgb}{0.000000,0.000000,0.000000}%
\pgfsetstrokecolor{currentstroke}%
\pgfsetdash{}{0pt}%
\pgfsys@defobject{currentmarker}{\pgfqpoint{0.000000in}{-0.048611in}}{\pgfqpoint{0.000000in}{0.000000in}}{%
\pgfpathmoveto{\pgfqpoint{0.000000in}{0.000000in}}%
\pgfpathlineto{\pgfqpoint{0.000000in}{-0.048611in}}%
\pgfusepath{stroke,fill}%
}%
\begin{pgfscope}%
\pgfsys@transformshift{7.106534in}{0.550000in}%
\pgfsys@useobject{currentmarker}{}%
\end{pgfscope}%
\end{pgfscope}%
\begin{pgfscope}%
\definecolor{textcolor}{rgb}{0.000000,0.000000,0.000000}%
\pgfsetstrokecolor{textcolor}%
\pgfsetfillcolor{textcolor}%
\pgftext[x=7.106534in,y=0.452778in,,top]{\color{textcolor}{\sffamily\fontsize{10.000000}{12.000000}\selectfont\catcode`\^=\active\def^{\ifmmode\sp\else\^{}\fi}\catcode`\%=\active\def%{\%}200}}%
\end{pgfscope}%
\begin{pgfscope}%
\pgfsetbuttcap%
\pgfsetroundjoin%
\definecolor{currentfill}{rgb}{0.000000,0.000000,0.000000}%
\pgfsetfillcolor{currentfill}%
\pgfsetlinewidth{0.803000pt}%
\definecolor{currentstroke}{rgb}{0.000000,0.000000,0.000000}%
\pgfsetstrokecolor{currentstroke}%
\pgfsetdash{}{0pt}%
\pgfsys@defobject{currentmarker}{\pgfqpoint{0.000000in}{-0.048611in}}{\pgfqpoint{0.000000in}{0.000000in}}{%
\pgfpathmoveto{\pgfqpoint{0.000000in}{0.000000in}}%
\pgfpathlineto{\pgfqpoint{0.000000in}{-0.048611in}}%
\pgfusepath{stroke,fill}%
}%
\begin{pgfscope}%
\pgfsys@transformshift{8.482599in}{0.550000in}%
\pgfsys@useobject{currentmarker}{}%
\end{pgfscope}%
\end{pgfscope}%
\begin{pgfscope}%
\definecolor{textcolor}{rgb}{0.000000,0.000000,0.000000}%
\pgfsetstrokecolor{textcolor}%
\pgfsetfillcolor{textcolor}%
\pgftext[x=8.482599in,y=0.452778in,,top]{\color{textcolor}{\sffamily\fontsize{10.000000}{12.000000}\selectfont\catcode`\^=\active\def^{\ifmmode\sp\else\^{}\fi}\catcode`\%=\active\def%{\%}250}}%
\end{pgfscope}%
\begin{pgfscope}%
\definecolor{textcolor}{rgb}{0.000000,0.000000,0.000000}%
\pgfsetstrokecolor{textcolor}%
\pgfsetfillcolor{textcolor}%
\pgftext[x=5.125000in,y=0.262809in,,top]{\color{textcolor}{\sffamily\fontsize{10.000000}{12.000000}\selectfont\catcode`\^=\active\def^{\ifmmode\sp\else\^{}\fi}\catcode`\%=\active\def%{\%}Intensidad}}%
\end{pgfscope}%
\begin{pgfscope}%
\pgfsetbuttcap%
\pgfsetroundjoin%
\definecolor{currentfill}{rgb}{0.000000,0.000000,0.000000}%
\pgfsetfillcolor{currentfill}%
\pgfsetlinewidth{0.803000pt}%
\definecolor{currentstroke}{rgb}{0.000000,0.000000,0.000000}%
\pgfsetstrokecolor{currentstroke}%
\pgfsetdash{}{0pt}%
\pgfsys@defobject{currentmarker}{\pgfqpoint{-0.048611in}{0.000000in}}{\pgfqpoint{-0.000000in}{0.000000in}}{%
\pgfpathmoveto{\pgfqpoint{-0.000000in}{0.000000in}}%
\pgfpathlineto{\pgfqpoint{-0.048611in}{0.000000in}}%
\pgfusepath{stroke,fill}%
}%
\begin{pgfscope}%
\pgfsys@transformshift{1.250000in}{0.550000in}%
\pgfsys@useobject{currentmarker}{}%
\end{pgfscope}%
\end{pgfscope}%
\begin{pgfscope}%
\definecolor{textcolor}{rgb}{0.000000,0.000000,0.000000}%
\pgfsetstrokecolor{textcolor}%
\pgfsetfillcolor{textcolor}%
\pgftext[x=1.064412in, y=0.497238in, left, base]{\color{textcolor}{\sffamily\fontsize{10.000000}{12.000000}\selectfont\catcode`\^=\active\def^{\ifmmode\sp\else\^{}\fi}\catcode`\%=\active\def%{\%}0}}%
\end{pgfscope}%
\begin{pgfscope}%
\pgfsetbuttcap%
\pgfsetroundjoin%
\definecolor{currentfill}{rgb}{0.000000,0.000000,0.000000}%
\pgfsetfillcolor{currentfill}%
\pgfsetlinewidth{0.803000pt}%
\definecolor{currentstroke}{rgb}{0.000000,0.000000,0.000000}%
\pgfsetstrokecolor{currentstroke}%
\pgfsetdash{}{0pt}%
\pgfsys@defobject{currentmarker}{\pgfqpoint{-0.048611in}{0.000000in}}{\pgfqpoint{-0.000000in}{0.000000in}}{%
\pgfpathmoveto{\pgfqpoint{-0.000000in}{0.000000in}}%
\pgfpathlineto{\pgfqpoint{-0.048611in}{0.000000in}}%
\pgfusepath{stroke,fill}%
}%
\begin{pgfscope}%
\pgfsys@transformshift{1.250000in}{1.341253in}%
\pgfsys@useobject{currentmarker}{}%
\end{pgfscope}%
\end{pgfscope}%
\begin{pgfscope}%
\definecolor{textcolor}{rgb}{0.000000,0.000000,0.000000}%
\pgfsetstrokecolor{textcolor}%
\pgfsetfillcolor{textcolor}%
\pgftext[x=0.799316in, y=1.288492in, left, base]{\color{textcolor}{\sffamily\fontsize{10.000000}{12.000000}\selectfont\catcode`\^=\active\def^{\ifmmode\sp\else\^{}\fi}\catcode`\%=\active\def%{\%}2000}}%
\end{pgfscope}%
\begin{pgfscope}%
\pgfsetbuttcap%
\pgfsetroundjoin%
\definecolor{currentfill}{rgb}{0.000000,0.000000,0.000000}%
\pgfsetfillcolor{currentfill}%
\pgfsetlinewidth{0.803000pt}%
\definecolor{currentstroke}{rgb}{0.000000,0.000000,0.000000}%
\pgfsetstrokecolor{currentstroke}%
\pgfsetdash{}{0pt}%
\pgfsys@defobject{currentmarker}{\pgfqpoint{-0.048611in}{0.000000in}}{\pgfqpoint{-0.000000in}{0.000000in}}{%
\pgfpathmoveto{\pgfqpoint{-0.000000in}{0.000000in}}%
\pgfpathlineto{\pgfqpoint{-0.048611in}{0.000000in}}%
\pgfusepath{stroke,fill}%
}%
\begin{pgfscope}%
\pgfsys@transformshift{1.250000in}{2.132506in}%
\pgfsys@useobject{currentmarker}{}%
\end{pgfscope}%
\end{pgfscope}%
\begin{pgfscope}%
\definecolor{textcolor}{rgb}{0.000000,0.000000,0.000000}%
\pgfsetstrokecolor{textcolor}%
\pgfsetfillcolor{textcolor}%
\pgftext[x=0.799316in, y=2.079745in, left, base]{\color{textcolor}{\sffamily\fontsize{10.000000}{12.000000}\selectfont\catcode`\^=\active\def^{\ifmmode\sp\else\^{}\fi}\catcode`\%=\active\def%{\%}4000}}%
\end{pgfscope}%
\begin{pgfscope}%
\pgfsetbuttcap%
\pgfsetroundjoin%
\definecolor{currentfill}{rgb}{0.000000,0.000000,0.000000}%
\pgfsetfillcolor{currentfill}%
\pgfsetlinewidth{0.803000pt}%
\definecolor{currentstroke}{rgb}{0.000000,0.000000,0.000000}%
\pgfsetstrokecolor{currentstroke}%
\pgfsetdash{}{0pt}%
\pgfsys@defobject{currentmarker}{\pgfqpoint{-0.048611in}{0.000000in}}{\pgfqpoint{-0.000000in}{0.000000in}}{%
\pgfpathmoveto{\pgfqpoint{-0.000000in}{0.000000in}}%
\pgfpathlineto{\pgfqpoint{-0.048611in}{0.000000in}}%
\pgfusepath{stroke,fill}%
}%
\begin{pgfscope}%
\pgfsys@transformshift{1.250000in}{2.923759in}%
\pgfsys@useobject{currentmarker}{}%
\end{pgfscope}%
\end{pgfscope}%
\begin{pgfscope}%
\definecolor{textcolor}{rgb}{0.000000,0.000000,0.000000}%
\pgfsetstrokecolor{textcolor}%
\pgfsetfillcolor{textcolor}%
\pgftext[x=0.799316in, y=2.870998in, left, base]{\color{textcolor}{\sffamily\fontsize{10.000000}{12.000000}\selectfont\catcode`\^=\active\def^{\ifmmode\sp\else\^{}\fi}\catcode`\%=\active\def%{\%}6000}}%
\end{pgfscope}%
\begin{pgfscope}%
\pgfsetbuttcap%
\pgfsetroundjoin%
\definecolor{currentfill}{rgb}{0.000000,0.000000,0.000000}%
\pgfsetfillcolor{currentfill}%
\pgfsetlinewidth{0.803000pt}%
\definecolor{currentstroke}{rgb}{0.000000,0.000000,0.000000}%
\pgfsetstrokecolor{currentstroke}%
\pgfsetdash{}{0pt}%
\pgfsys@defobject{currentmarker}{\pgfqpoint{-0.048611in}{0.000000in}}{\pgfqpoint{-0.000000in}{0.000000in}}{%
\pgfpathmoveto{\pgfqpoint{-0.000000in}{0.000000in}}%
\pgfpathlineto{\pgfqpoint{-0.048611in}{0.000000in}}%
\pgfusepath{stroke,fill}%
}%
\begin{pgfscope}%
\pgfsys@transformshift{1.250000in}{3.715012in}%
\pgfsys@useobject{currentmarker}{}%
\end{pgfscope}%
\end{pgfscope}%
\begin{pgfscope}%
\definecolor{textcolor}{rgb}{0.000000,0.000000,0.000000}%
\pgfsetstrokecolor{textcolor}%
\pgfsetfillcolor{textcolor}%
\pgftext[x=0.799316in, y=3.662251in, left, base]{\color{textcolor}{\sffamily\fontsize{10.000000}{12.000000}\selectfont\catcode`\^=\active\def^{\ifmmode\sp\else\^{}\fi}\catcode`\%=\active\def%{\%}8000}}%
\end{pgfscope}%
\begin{pgfscope}%
\definecolor{textcolor}{rgb}{0.000000,0.000000,0.000000}%
\pgfsetstrokecolor{textcolor}%
\pgfsetfillcolor{textcolor}%
\pgftext[x=0.743761in,y=2.475000in,,bottom,rotate=90.000000]{\color{textcolor}{\sffamily\fontsize{10.000000}{12.000000}\selectfont\catcode`\^=\active\def^{\ifmmode\sp\else\^{}\fi}\catcode`\%=\active\def%{\%}Frecuencia}}%
\end{pgfscope}%
\begin{pgfscope}%
\pgfsetrectcap%
\pgfsetmiterjoin%
\pgfsetlinewidth{0.803000pt}%
\definecolor{currentstroke}{rgb}{0.000000,0.000000,0.000000}%
\pgfsetstrokecolor{currentstroke}%
\pgfsetdash{}{0pt}%
\pgfpathmoveto{\pgfqpoint{1.250000in}{0.550000in}}%
\pgfpathlineto{\pgfqpoint{1.250000in}{4.400000in}}%
\pgfusepath{stroke}%
\end{pgfscope}%
\begin{pgfscope}%
\pgfsetrectcap%
\pgfsetmiterjoin%
\pgfsetlinewidth{0.803000pt}%
\definecolor{currentstroke}{rgb}{0.000000,0.000000,0.000000}%
\pgfsetstrokecolor{currentstroke}%
\pgfsetdash{}{0pt}%
\pgfpathmoveto{\pgfqpoint{9.000000in}{0.550000in}}%
\pgfpathlineto{\pgfqpoint{9.000000in}{4.400000in}}%
\pgfusepath{stroke}%
\end{pgfscope}%
\begin{pgfscope}%
\pgfsetrectcap%
\pgfsetmiterjoin%
\pgfsetlinewidth{0.803000pt}%
\definecolor{currentstroke}{rgb}{0.000000,0.000000,0.000000}%
\pgfsetstrokecolor{currentstroke}%
\pgfsetdash{}{0pt}%
\pgfpathmoveto{\pgfqpoint{1.250000in}{0.550000in}}%
\pgfpathlineto{\pgfqpoint{9.000000in}{0.550000in}}%
\pgfusepath{stroke}%
\end{pgfscope}%
\begin{pgfscope}%
\pgfsetrectcap%
\pgfsetmiterjoin%
\pgfsetlinewidth{0.803000pt}%
\definecolor{currentstroke}{rgb}{0.000000,0.000000,0.000000}%
\pgfsetstrokecolor{currentstroke}%
\pgfsetdash{}{0pt}%
\pgfpathmoveto{\pgfqpoint{1.250000in}{4.400000in}}%
\pgfpathlineto{\pgfqpoint{9.000000in}{4.400000in}}%
\pgfusepath{stroke}%
\end{pgfscope}%
\begin{pgfscope}%
\definecolor{textcolor}{rgb}{0.000000,0.000000,0.000000}%
\pgfsetstrokecolor{textcolor}%
\pgfsetfillcolor{textcolor}%
\pgftext[x=5.125000in,y=4.483333in,,base]{\color{textcolor}{\sffamily\fontsize{12.000000}{14.400000}\selectfont\catcode`\^=\active\def^{\ifmmode\sp\else\^{}\fi}\catcode`\%=\active\def%{\%}Histograma de BHEPL-D}}%
\end{pgfscope}%
\end{pgfpicture}%
\makeatother%
\endgroup%
}
		\caption{BHEPL-D}
		\label{fig:bhepl_3}
	\end{subfigure}

	\caption{Comparación de histogramas para la imagen 2015\_00027}
	\label{fig:histogramas_3}
\end{figure}

\subsection{Resultados Cuantitativos}
\subsubsection{Análisis de AMBE}
- Comparación de los valores promedio de AMBE para cada algoritmo.

\subsubsection{Análisis de PSNR}
- Comparación de los valores promedio de PSNR para cada algoritmo.

\subsubsection{Análisis de Entropía}
- Comparación de los valores promedio de entropía para cada algoritmo.

\subsubsection{Análisis de Contraste}
- Comparación de los valores promedio de contraste para cada algoritmo.

\subsection{Discusión de los Resultados}
- Interpretación de los resultados obtenidos.
- Comparación de los algoritmos en términos de sus métricas de evaluación.
- Discusión sobre la efectividad de cada algoritmo.

\section{Conclusiones}
\subsection{Resumen de Hallazgos}
- Resumen de los principales resultados obtenidos de la comparación de los algoritmos.

\subsection{Recomendaciones}
- Recomendaciones sobre cuál algoritmo es el más efectivo en función de las métricas utilizadas.
- Posibles mejoras o ajustes que se podrían hacer en los algoritmos.

\subsection{Trabajo Futuro}
- Sugerencias para futuras investigaciones y mejoras en los algoritmos de mejora de contraste.

\section{Referencias}
- Lista de fuentes y trabajos relacionados con los algoritmos de mejora de contraste y procesamiento de imágenes.

\section{Apéndices (si es necesario)}
- Detalles adicionales que no se incluyeron en el cuerpo principal del informe, como códigos, tablas de resultados completos, gráficos adicionales, etc.

% Bibliografía
\nocite{*} % Incluye TODAS las referencias del .bib sin citarlas
\bibliographystyle{ACM-Reference-Format}
\bibliography{bibliography}

\end{document}
